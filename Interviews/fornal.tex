\section{Maria Fornal}
\begin{otherlanguage}{polish}
\textit{Maria Fornal (*1960) works as an expert on historic buildings and monuments in the Province Office in Zamość. \\
She studied ethnography at the Jagiellonian University in Kraków. She is passionate about Jewish culture and cares especially about protecting cultural heritage, also as a tour guide around Zamość and the Roztocze regions. From 2003 to 2011, she organised the Zamość Meetings of Cultures, presenting the history and heritage of the minorities living in Zamość from the 16th to the 20th century. She wrote many articles about Jews and Jewish culture, monuments and cultural heritage of the Zamość, Lublin, and Roztocze regions. She is also an author of numerous radio and TV programmes on preserving and spreading cultural heritage and on national minorities in the region of Zamość. In 2010, she was cooperating in collecting documents to a Polish and Jewish documentary ``Tam był kiedyś nasz dom'' (``There used to be our home there''), showing the lives of Jews in Zamość before the Second World War. For her activity, she was awarded by the Polish President (in 2000) and the Minister of Culture and National Heritage (in 1998 and 2006). \\
The interview took place on February 1st, 2018, in Zamość.}\par
\vspace*{2em}
\textbf{Wiemy, że interesuje się Pani i zajmuje historią i kulturą Żydów.}
	
\textbf{Maria Fornal:} Zajmuję się już trochę mniej. Studiowałam etnografię na UJ w Krakowie. Ponieważ każdy student w tamtych czasach szukał miejsc, gdzie można było dorobić, więc ja też trafiłam do takiego miejsca i jak się okazało, było to u osoby, która miała kontakty z Żydami. Nie wiedziałam wtedy, że te osoby, które do niej przychodzą, to Żydzi. Poznawaliśmy się, rozmawialiśmy, stopniowo poznawałam ich coraz bliżej. Krakowski Kazimierz wyglądał wtedy diametralnie różnie od tego, jak wygląda teraz. Codziennie przechodziłam obok tego żydowskiego Kazimierza i tak naprawdę poza tym, czego dowiadywałam się w trakcie zajęć – a miałam zajęcia z historii, z historii sztuki, z historii różnych nacji – słabo znałam historię Żydów krakowskich. Kazimierz wtedy to w zasadzie był jeden piękny budynek, czyli dawna synagoga na placu Wolnica, gdzie jest muzeum etnograficzne i tam miałam czasem zajęcia. Pozostałe budynki były w złym stanie i pamiętam do dzisiaj, było mnóstwo komórek, takich małych pomieszczeń, myśmy nazywali to "`kibelkami"', dlatego że było tam dosyć dużo osób, które załatwiały swoje potrzeby i strasznie był zaniedbany ten Kazimierz. Były to czasy, kiedy społeczność żydowska chyba nie bardzo odnajdywała się. Pod koniec lat osiemdziesiątych, w latach dziewięćdziesiątych, jak jeździłam już po studiach do Krakowa, ta dzielnica zaczęła zupełnie inaczej wyglądać. Teraz na Kazimierzu właśnie w tych dawnych "`kibelkach"', komórkach są małe kawiarenki i piękne sklepiki z pamiątkami. Kiedy jadę do Krakowa, staram się odwiedzać moich dawnych znajomych, ale też zawsze biegnę i patrzę, co nowego jest w tych miejscach, które zupełnie inaczej wtedy wyglądały. Wśród moich znajomych w Krakowie znalazł się pan Tadeusz Jakubowicz, on jest obecnie przewodniczącym gminy żydowskiej w Krakowie (wtedy jeszcze gmina żydowska nie funkcjonowała jako jednostka). Często spotykaliśmy się, zabierał mnie do synagogi Remuh, na cmentarz Remuh, na którym te najcenniejsze nagrobki jeszcze były pod warstwą ziemi. Od niego dowiedziałam się pierwszych rzeczy o Żydach, o kulturze żydowskiej. Po skończonych studiach musiałam wyjechać z Krakowa, przyjechałam do Zamościa i zaczęłam pracować w Biurze Badań Dokumentacji Zabytków, była to wówczas instytucja przy Wojewódzkim Konserwatorze Zabytków. Przyszłam do pracy jako etnograf, żeby wykonywać przede wszystkim dokumentację dla budownictwa drewnianego. Zaczęłam trochę czytać o historii Zamościa. Od razu zrobiłam kurs przewodnicki. I okazało się, że historia Żydów Zamościa i Zamojszczyzny jest tak bogata i nie do końca odkryta i tyle jest w niej rozbieżności, że pomyślałam, że skoro mam już jakieś podstawy, że mam możliwość dostępu do pewnych materiałów, pytania, wyjaśniania rzeczy, które dla mnie jeszcze były niezrozumiałe, to może tak po trochę zajmę się tym. No i tak się zaczęło. Z czasem troszkę więcej interesowały mnie te sprawy, dostałam propozycję pisania artykułów do Zamojskiego Kwartalnika Kulturalnego, miałam swoją rubrykę \textit{Judaika}. Pisałam i zachęcałam do pisania swoich znajomych, którzy też mieli wiedzę o Żydach. Kilka lat to funkcjonowało. Później musiałam zostawić tę rubrykę. Ale kiedy w Hrubieszowie odkopano macewy na cmentarzu żydowskim, powstał pomnik upamiętniający Żydów hrubieszowskich, była uroczystość, na którą przyjechali Żydzi i dawni mieszańcy Hrubieszowa i ich potomkowie, wydany Kwartalnik cały poświęcono właśnie Hrubieszowowi. I napisałam do niego artykuł o żydowskim Hrubieszowie. Wtedy zaczęłam głębiej wchodzić w historię \textit{sztetli}, czyli miasteczek żydowskich, gdzie czasami kilka różnych narodowości mieszkało koło siebie. Dla mnie jako dla etnografa, badacza kultury była to najciekawsza historia. Goście, którzy przyjechali z Izraela, zabrali ze sobą kilka egzemplarzy Kwartalnika. I po jakimś czasie zadzwoniła do mnie pani, powiedziała, że przeczytała mój artykuł i prosi o zgodę o przetłumaczenie i zamieszczenie w periodyku w Izraelu. Oczywiście, zgodziłam się. Tak się zaczęła moja znajomość z ziomkostwem zamojskim. Później okazało się, że ta pani, Ewa Bar-Zeev była szefową Związku Żydów Zamojskich, pochodziła z Zamościa, z rodziny Szperów. Wtedy zaczęła się między nami  wymiana informacji: ja ją pytałam o różne rzeczy w związku z tym, że ona miała kontakt z dawnymi Żydami zamojskimi, z tymi, którzy przeżyli wojnę, wyjechali do Izraela, z ich dziećmi, a ja jej pomagałam tutaj odnajdywać miejsca, ludzi, historie. Zżyłyśmy się dosyć mocno. Korespondowałyśmy przez kilkanaście lat, do 2008 roku. Była w Zamościu chyba trzy razy. Raz przyjechała z ekipą filmową, która kręciła film o Żydach zamojskich. Dzięki niej poznawałam kolejne osoby z ziomkostwa żydowskiego i w 2000 roku pojechałam do Izraela na zaproszenie tych osób. Oczywiście zwiedzałam wszystkie te miejsca, które dla nas jako katolików są istotne, ale dla mnie najważniejsze i najcenniejsze było to, że jeździłam od domu do domu, czasami wożona przez nich, czasami autobusem, stopem i poznawałam wewnętrzne życie rodzin żydowskich. Każda rodzina starała się pokazać mi jak najwięcej ciekawych rzeczy, ale też spędzaliśmy dużo czasu przy stole, siedząc, opowiadając sobie różne historie. Kiedyś w małej miejscowości kawałek za Galileą byłam gościem u pewnego młodego człowieka, którego zresztą przez przypadek poznałam w Zamościu i pomogłam mu znaleźć historię jego rodziny. Jego rodzina zaprosiła mnie do siebie. Jego babcia mieszkała w Zamościu na Nowym Mieście, na ulicy Ogrodowej. Przeżyła Holokaust jako jedyna z całej rodziny, tylko dlatego, że uciekła w trzydziestym dziewiątym roku na teren Związku Radzieckiego. Bardzo smutna historia, a dla mnie było ważne to, że mogłam pojechać tam i z nią porozmawiać, posłuchać jej wspomnień o tym wszystkim, co się działo w Zamościu, o tym, co ona zapamiętała. Przyjmowano mnie tam bardzo ciepło, nie czułam się gojką, nie czułam się osobą obcą, po prostu wchodziłam do domu i byłam członkiem każdej z tych rodzin, u której gościłam. Ta podróż jeszcze bardziej scementowała nasze kontakty, a nawet przyjaźń z kilkoma rodzinami. Myślę, że mam wyjątkowe szczęście, że poznałam tych ludzi.  Później oni przyjeżdżali do Zamościa, zapraszałam ich do siebie do domu i też starałam się robić wszystko, żeby im się przypomniały albo czasy dzieciństwa, albo też, jeśli byli to młodsi ludzie, to żeby poczuli, jakie smaki towarzyszyły ich rodzicom czy dziadkom. Tę atmosferę i moje kolacje potem wspominali w listach. Zawsze też każdy, z kim się spotykałam, na drogę dostawał de mnie woreczek grzybków suszonych, konfiturę z jeżyn leśnych. Te historie są w zasadzie może nic nieznaczące, ale jak patrzę z perspektywy na to wszystko, to dopiero teraz wiem, jak to było ważne dla tych osób, ważniejsze niż zwiedzanie miasta. Chociaż duże znaczenie miało również chodzenie do miejsc, które były związane z ich młodością, tam, gdzie była szkoła, kino, opowiadanie o historiach, o randkach między Żydami a katolikami, a Polakami. Ostatni człowiek, z którym utrzymywałam bliskie kontakty, często bywał w Zamościu. Nazywał się Yoram Goran, był szefem kinematografii Izraela, a w Zamościu mieszkał w domu Pereca. Był ostatnim strażnikiem upamiętniania Żydów i tego, co się działo w Zamościu. Zaczęło do mnie docierać, jak ci ludzie stali się ważni również dla mnie. Wciągnęłam się bardzo mocno w tę tematykę, zaczęłam badać kolejne miejscowości wokół Zamościa, poznawać więcej z historii Żydów zamojskich. Tutaj kiedyś była biblioteka w synagodze, a w 2004 roku wyprowadziła się do nowej siedziby. Synagoga pozostała pusta, nikt nie miał pomysłu, co z nią zrobić. Wtedy wspólnie z kolegą poprosiliśmy Fundację Ochrony Dziedzictwa Żydowskiego, która była właścicielem synagogi, o udostępnienie nam jej nieodpłatnie. Przez dwa lata organizowaliśmy tam spotkania, koncerty, spektakle i ta synagoga żyła niesamowicie. Mimo że warunki były straszne, udostępnialiśmy ją do zwiedzania. Urządziliśmy ją, część mebli zdobyliśmy i ustawiliśmy tak, żeby jak najbardziej przypominała wnętrze przedwojenne synagogi. Oczywiście była też muzyka. I wtedy okazało się, że mnóstwo Żydów tu przyjeżdżało, o których myśmy w ogóle nie wiedzieli. Poznawaliśmy ich historie. Zdarzało się, że ktoś w ławce siedział przez kilka godzin i pewnie coś sobie wspominał, w tle grała muzyka. To były dobre chwile. Trwało to do 2006 roku. Potem ludzie wspominali koncerty przy świecach, atmosferę, często mówili, że przywróciliśmy ducha w tym budynku i mimo że jest pusto, nie ma wyposażenia, które tam wcześniej było, że nie odprawiają się nabożeństwa, że nie ma Żydów, to kiedy tam wchodzą, jakby cofali się do czasów przedwojennych. Później znalazły się pieniądze na remont, ale ja już nie utrzymywałam kontaktu z synagogą. Teraz wygląda pięknie, tylko że nie ma tam ducha...\\
W tej chwili w zasadzie poza kontaktami, które mam, od czasu do czasu dzielę się swoimi materiałami. Dostałam od Żydów około tysiąca zdjęć ich, ich rodzin, większość z nich niestety zginęła w Holokauście. Trochę zdjęć udało mi się kupić. I zostały mi te zdjęcia. Ale cieszę się, że od czasu do czasu jeszcze w jakiś sposób mogę się podzielić z kimś czy swoimi doświadczeniami, czy wiedzą. Kiedyś wygrzebywaliśmy materiały z każdego zakątka, skąd się tylko dało. Dostęp do archiwów nie był taki łatwy, jak jest teraz. No, ale mam satysfakcję. Udało mi się też być swego rodzaju konsultantem przy powstawaniu niektórych książek, na przykład \textit{Szebreszin} Philipa Bibla, wspomnienia szczebrzeszyńskiego Żyda, dzięki niej można poznać ten prawdziwy świat żydowski, przedwojenny. Robiłam też dokumentację do filmu dokumentalnego \textit{Tam był kiedyś nasz dom}. Film powstał właśnie dzięki Yoramowi, ze strony polskiej w zdobywaniu pieniędzy w zasadzie na cały ten projekt uczestniczył Mirosław Chojecki, a Ewa Szprynger pisała scenariusz, reżyserowała. Zresztą też zaprzyjaźniłyśmy się przy tym filmie. Ja robiłam dokumentację, robiłam rozpoznanie oraz materiały ikonograficzne. Minęło trochę czasu, trochę inaczej to wszystko wygląda. Coraz więcej ludzi zajmuje się problematyką, tematami żydowskimi. Natomiast chcę wam powiedzieć, że jednej rzeczy się trzymam: nie uczestniczę w żadnych projektach, które dotyczą Holokaustu. Obiecałam sobie, że będę mówiła o tym, jak powstawała historia i kultura żydowska, o stosunkach międzyludzkich, bo to jest mi bliskie jako etnografowi. Wolę poznawać, spisywać, wysłuchiwać historii obyczajowych. I wolę też uczyć się o historii, która się tworzyła, a nie o tej, która była niszczona. 

\textbf{A czy są wśród tych historii takie, które dotyczą antysemityzmu? Czy ktoś z tych osób go doświadczył?}
 
\textbf{Maria Fornal:} A jak wy interpretujecie antysemityzm?

\textbf{Właśnie od tygodnia próbujemy znaleźć definicję.}

\textbf{Maria Fornal:} Opowiem wam historię. Kiedyś w czasie spaceru z moimi gośćmi natknęliśmy się na młodych ludzi, którzy rysowali szubienicę i gwiazdy Dawida. Podeszłam, zapytałam jednego z tych chłopaków: "`Co ty, synu, rysujesz?"' On odpowiedział: "`Śmierć Żydom"'. Ja mówię: "`Dlaczego? O co ci chodzi?"' "`Bo ich nie powinno być."' Ja mówię: "`A podnieś koszulkę. Masz na sobie Levi-Straussy. Przecież to są spodnie Żyda, on je wymyślił"', mówię. "`A jakiej muzyki słuchasz? Powiedz mi, jakich zespołów słuchasz, to ja ci powiem, ile w tych zespołach jest muzyków żydowskich."' I on zrobił się bordowy, powiedział brzydkie słowo i zwiał. Więc dlatego pytam was, jak wy interpretujecie antysemityzm. W filmie, o którym mówiłam, Yoram Golan, który jest w nim przewodnikiem, opowiada historie, które się działy przed wojną: o gettach ławkowych, o odsuwaniu młodzieży żydowskiej od wspólnych projektów typu spektakle teatralne, czy wspólnej zabawy. Ale też mówi tam bardzo ważne zdanie, że tak naprawdę wszyscy byli tutaj jedną rodziną, że różne rzeczy się działy, że Polacy podglądali Żydów, Żydzi podglądali Polaków i jedni od drugich czerpali pewne rzeczy i przenosili do swoich domów. Mimo że dochodziło do bójek w szkołach, bo czasami jacyś chłopcy uważali, że trzeba dać łupnia Żydom, to kilka dni później razem szli już do kina albo zastanawiali się razem i patrzyli, komu by tu można dosunąć. Już wtedy w grupie, bez podziału, że tu Żydzi, a tu Polacy. Ja nie chcę nazywać tego antysemityzmem, nie chcę tego w ogóle nazywać, bo by można to nazywać różnie. W filmie są też wspomnienia pani, która wróciła do Zamościa po wojnie, jej się ten Zamość cały czas śnił, wiedziała, że musi tutaj wrócić, bo w ogóle nie mogła spać, to było coś potwornego. W końcu wybrała się i kiedy zobaczyła, że mieszkanie było zajęte, a ludzie w nim mieszkający byli przestraszeni, bo obawiali się tego, że przyjechała zabrać im to mieszkanie, natychmiast wiedziała, że musi wrócić do Izraela, że to już nie jest jej miejsce. To na pewno wywoływało przykre sytuacje. Ja sama spotkałam się kiedyś z dwiema kobietami, których matka była ukrywana przez polską rodzinę niedaleko Tarnawatki. Pojechałyśmy to tego miejsca. Ci ludzie, którzy ukrywali tę matkę, już nie żyją, syn, czy córka tej osoby żyje do tej pory. W każdym bądź razie ta dziewczyna przeżyła. Potem była ukrywana w Zamościu, później po wojnie przez pewien czas tu mieszkała. Przeżyła wojnę, wróciła tam, do tej rodziny i powiedziała, że wszystko to, co było ich majątkiem, czyli dom jakiś stary i sporo gruntu, ona tej rodzinie zostawia za to, że uratowali ją w czasie wojny. (A kiedy córki tej pani przyjechały i szukały historii swojego wuja, przez przypadek trafiliśmy do pana, który był w oddziale Batalionów Chłopskich razem z tym wujem. Od niego dowiedziały się, że pod Adamowem w czasie jakiejś potyczki on niestety wpadł w ręce Niemców, przez tydzień był torturowany na Rotundzie, po czym rozstrzelany też na Rotundzie. Ponieważ w ogóle nie wiedziały, co się z nim stało, z jednej strony wiadomość o śmierci kolejnej osoby z ich rodziny była bolesna, ale z drugiej strony były szczęśliwe i tak wdzięczne, że poznały tę historię. I dla nich było ważne, że ten człowiek z taką dumą mówił, że był w oddziale właśnie z tym ich wujem, Żydem.) I udałam się z tymi kobietami do tej rodziny jako tłumacz, weszłyśmy, powiedziałam, kim one są i po co przyjechały. Trzeba było zobaczyć twarze tych ludzi, ich przerażenie, bo byli przekonani, że właśnie ktoś przyjechał zabrać im wszystko, co mieli, bo przecież oni sobie pobudowali już nowy dom na tych gruntach. Te kobiety powiedziały mi: "`Maria, powiedz im od razu, że my niczego nie chcemy. My tylko chciałyśmy przyjechać, zobaczyć to miejsce, gdzie nasza matka była chroniona."' Specjalnie dla niej była wykopana jama, na której postawiono komórkę ze składem drewna. Potem, gdy miejscowi dowiadywali się, podejrzewali tę rodzinę, że ukrywali Żydówkę, ta rodzina zabrała ją z tej jamy, a miejscowi spalili komórkę, przekonani, że była tam Żydówka. A ona była już ukryta na strychu. I tam kiedyś wpadli, by wyciągnąć tę ukrywaną Żydówę, a ona siedziała w wielkim pojemniku pełnym zboża. Na pewno miała problem z oddychaniem, nie wiem, ale zrobili tam jakąś rurkę, żeby odrobina powietrza docierała. I nie znaleźli jej, poszli gdzieś dalej, przeszukali całe gospodarstwo. I ta dziewczyna przeżyła, przeżyła wojnę, urodziła później dzieci. Widać więc, jakie było nastawienie.\\
Miałam kiedyś gości tutaj, przyjechali do rodziny niedaleko za Zamość. Ci ludzie znali adres, doskonale wiedzieli, że tam jest ktoś z ich rodziny, dziadkowie byli braćmi. Jeden przeżył, wyjechał, trafił do Palestyny, a drugi tutaj przeszedł na wiarę katolicką. Rodzina nie wiedziała o tym, że mają korzenie żydowskie. Aż tu nagle przyjeżdżają bliscy krewni, bo ze strony brata ich dziadka, więc oni szczęśliwi, zadowoleni, stół zastawiony, herbatka, przyjęcie, rozmowy o wszystkim. I nagle pada pytanie: "`A skąd wy przyjechaliście?"' Oni mówią: "`Z Izraela."' I cisza. Popatrzyli po sobie przestraszeni. "`Jak to z Izraela? No bo nasz dziadek tak się nazywał. A wasz dziadek jak się nazywał?"' "`Tak."' "`No to bracia byli przecież."' I przerażenie, żeby nie wyszło na jaw, że mają korzenie żydowskie. Nie wiem, co to jest, nie chcę definiować antysemityzmu, bo nie wiem, czy my możemy mówić o takim ogólnym, globalnym określeniu tego słowa. Dla jednego antysemityzmem będzie to, że ktoś napisze coś paskudnego na ścianie, tak jak mieliśmy niedawno przykład na synagodze, kiedy dzieciaki w ogóle niemające nawet pojęcia o tym, co to jest, napisały, już nie pamiętam, jakieś słowa o Żydach i Fundacja Ochrony Dziedzictwa Żydowskiego rozpętała wielką aferę, że antysemickie działania. Ja mówię, to napisało dziecko może ośmio- dziesięcioletnie, które gdzieś usłyszało, że ktoś coś powiedział i poszło sobie i tam napisało. Dla mnie to są działania, które wynikają z niewiedzy, z głupoty ludzkiej, ze złośliwości.\\
Ale nie słyszałam nigdy, żeby ci, którzy przeżyli wojnę i przyjeżdżali tutaj, nawet wiedząc, że nie mogą wrócić do swojego domu, widząc miejsca, gdzie stał kiedyś ich dom, oskarżali tych, którym się udało, że przyszli, zabrali. Nie wiem, może trafiałam na takich ludzi.

\textbf{Może dlatego, jak Pani mówiła, że było bardzo dobre współżycie tych dwóch społeczności, polskiej i żydowskiej, prawda? Że było podglądanie, o którym Pani wspomniała, czerpanie od siebie, a nie rywalizacja, zawiść. Może to coś zostawiło?}

\textbf{Maria Fornal:} No właśnie. W miejscu, gdzie jest parking koło hotelu "`Renesans"', kiedyś było kino "`Jutrzenka"', drewniany budynek. Przyjechali kiedyś tutaj dwaj bracia, którzy nie wiedzieli o tym, że o sobie nawzajem, że przeżyli, bo w trzydziestym dziewiątym roku obydwaj pojechali z transportem z Armią Czerwoną na teren Związku Radzieckiego. Potem ich drogi rozeszły się, nie wiedzieli o sobie. I po wojnie spotkali się zupełnie przypadkowo w Lublinie. Szukali tutaj śladów swojej rodziny. Opowiadali przeróżne historie, między innymi, że to kino "`Jutrzenka"' tak bardzo zapamiętali, dlatego że to było jedyne miejsce, gdzie mogli z polskimi dziewczynami chodzić na randki. Mówi: "`Przecież wie pani, po co się chodzi do kina..."' Babcia mojego przyjaciela opowiadała mi historię z lat trzydziestych, gdy był już nacisk na odsuwanie Żydów, niekupowanie w żydowskich sklepach, niekorzystanie z usług żydowskich rzemieślników. Estera była z dosyć ubogiej rodziny, jej ojciec był dekarzem, wszystkie dachy robił. Trudno mu było o pracę, bo po sąsiedzku były polskie warsztaty dekarzy i przede wszystkim oni byli brani do roboty. Opowiadała, że Janek, jej najbliższy przyjaciel (podejrzewam, że wielka miłość, chociaż nigdy tego słowa od niej nie usłyszałam, ale tak ciepło o nim mówiła i tak jej oczy błyszczały, jak wspominała tę historię), jak się dowiedział, że nie mają pieniędzy na jedzenie, że jest im ciężko, któregoś dnia wziął kamień, biegał i tłukł dachy po to, żeby zrobić dziurę i żeby jej ojciec miał robotę. I wiecie, można słuchać różnych historii, które zostały gdzieś opowiedziane, są zapisane. Ale ja zawsze obserwowałam tych ludzi, gdy o tym mówili. Widziałam, jak zamieniali się w szczeniaków, w młode dziewczyny z tamtego czasu, te oczy jak pięknie błyszczały i buzia uśmiechnięta… Miałam wrażanie, że gdy opowiadali, wracali na tę ulicę, na swoje podwórko, do tej gromadki, gdzie takie dobre rzeczy się zdarzały… 
\end{otherlanguage}