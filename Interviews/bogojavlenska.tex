\section{Dr Svetlana Bogojavlenska}
\begin{otherlanguage}{ngerman}
\textit{Dr. Svetlana Bogojavlenska studied history from 1995 to 2002 at the University of Latvia in Riga. Between 1997 and 2002 she worked in the Museum ``Jews in Latvia''. From 2003 to 2008 she worked as a doctoral student and lecturer at Johannes Gutenberg University Mainz. Her doctoral thesis is entitled ``The Formation and Position of Jewish Society in Riga and in the Government Courland 1795-1915''. From 2011 to 2016, she worked in the DFG project ``Russian in the Latvian Context: Russian Identity Formation in Latvia 1914-1940'' at the Department of History of the University of Mainz. Since 2018 she is Research Associate in the project ``Before the Cultural History: Functions and Dynamics Russian Historiography in the European Context (1750-1830)''. Her research interests include Jewish culture and history, the history and theology of the Orthodox Church, sacred music and art, comparative religion, minorities in Russia and the Baltic States, and cultural transfer.
For the interview she visited us in Nuremberg, the interview took place on September 5th, 2017 and was conducted in German.}\par
\vspace*{2em}
\textbf{Vielen Dank, dass Sie sich Zeit nehmen, uns ein Interview zu geben. Ich würde jetzt einmal mit der Frage beginnen, wann Sie, wenn Sie sich noch erinnern können, zum ersten Mal von dem Begriff "`Jude"' oder "`jüdisch"' gehört haben.}

\textbf{Dr. Svetlana Bogojavlenska:} In der Kindheit. Ich glaube, ich habe es zum ersten Mal von meiner Großmutter gehört, und der Begriff war schon mit Vorurteilen beladen. Es war so, dass ich das Gefühl hatte, dass das etwas Fremdartiges ist. Was vielleicht ein bisschen gefährlich ist, aber noch gefährlicher waren "`Zigeuner"'. In der Gegend, in der ich gewohnt habe, war das die Wahrnehmung der Bevölkerung. Oder die Wahrnehmung meiner Großmutter. Seitdem war der Begriff eigentlich immer wieder irgendwie da, aber ich habe mich dann nicht so viel damit beschäftigt. Bis ich dann selbst auf die Idee kam, dass ich eigentlich viele Klassenkameraden hatte, die plötzlich nach Israel ausgewandert waren.
Vorher wusste man gar nicht, dass sie etwas mit dem Judentum zu tun hatten. Man hätte es auch gar nicht vermuten können. Sie haben sich nicht speziell als Juden ausgegeben. Das waren ganz normale Klassenkameraden wie jede und das wurde gar nicht thematisiert. Dann waren viele plötzlich in Israel. Und dann habe ich angefangen, mich dafür zu interessieren, welche Geschichte dieses Volk hat - was haben sie in Lettland gemacht und wie kam es dazu, dass wir eigentlich so wenig über sie wissen? Über ihre Geschichte und warum sie so unscheinbar da sind. Und plötzlich wandern sie nach Israel aus, weil sie sagen: "`Ich möchte Jude sein"'. Warum kann man das nicht in Lettland sein, zum Beispiel? Auch das war meine Frage.

\textbf{Also das haben sie gesagt? Sie könnten in Lettland keine Juden sein?}

\textbf{Dr. Svetlana Bogojavlenska:} Zumindest ein Klassenkamerad von mir, mit dem ich Schriftverkehr hatte über Jahre. Seine Argumentation war: "`In Israel habe ich zu mir zurückgefunden"'. Weil es keine Hindernisse mehr gibt, weil man sich frei zur eigenen Identität äußern kann. Und das war schon eine Überraschung für mich, weil ich ihn vorher nicht als Juden wahrgenommen habe und er es auch nie in der Schule thematisiert hat.

\textbf{Sind ihre Klassenkameraden alleine ausgewandert oder mit ihren Familien?}

\textbf{Dr. Svetlana Bogojavlenska:} Mit den Familien zusammen. Es gab eine Welle an Auswanderung in Lettland, das hat wahrscheinlich '91/'92 angefangen. Und in meiner Erinnerung ist es so geblieben, dass Berge guter Bücher irgendwo in den Höfen lagen, zum Beispiel, oder einfach in die Schule gebracht worden sind von den Familien, die auswandern.  Das war dann um '94/'95.

\textbf{Abgesehen davon, wurde das Thema Judentum, jüdische Religion und jüdische Geschichte in der Schule behandelt?}

\textbf{Svetlana Bogojavlenska:} Überhaupt nicht, nein. Darüber haben wir nie geredet. Ich glaube mich erinnern zu können, dass im Geschichtsunterricht, als es um den Zweiten Weltkrieg ging, die Judenvernichtung erwähnt wurde, aber nur erwähnt. Und sonst nicht. Der Begriff Holocaust ist gar nicht erwähnt worden, er war völlig unbekannt. Dass Juden als ein Volk tatsächlich als Zielscheibe des Nationalsozialismus ausgewählt wurden und dann ausgelöscht wurden, das wurde schon thematisiert, aber wie gesagt sehr kurz. Mit ein paar Sätzen wurde das gesagt: Die Juden waren ein besonderes Ziel der Vernichtung, aber Hitler hatte vor ganz Osteuropa auszulöschen. - ungefähr in diesem Sinne.

\textbf{Vor den Holocaust zurück bzw. vor die Okkupation zurück. Wie war die Stellung der Juden in der Zwischenkriegszeit in Lettland?}

\textbf{Svetlana Bogojavlenska:} Juden waren gleichberechtigte Bürger, zum ersten Mal in ihrer Geschichte auf diesem Territorium. Sie wurden 1918 mit der Gründung Lettlands zu gleichberechtigten Bürgern, die sich auch vollständig in der Politik frei entfalten konnten, die sich auch tatsächlich in der Verwaltung betätigt haben, auch bei der Regierungsbildung. Bis 1934, bis zum Regierungsumsturz und bis zur Etablierung des Autoritarismus von Ulmanis in Lettland kann man tatsächlich von der vollständigen politischen Gleichberechtigung sprechen.
1934 hat sich die allgemeine nationale Politik des Staates geändert. Wenn bei der Regierungsdeklaration 1918 verkündet wurde, dass Lettland ein Staat für alle Ethnien ist, die auf diesem Territorium leben, dann wurde nach dem Regierungsumsturz 1934 propagiert, dass zwar alle bleiben und ihren Teil daran haben dürfen, aber Lettland doch für die Letten da ist. Das war sehr nationalistisch, aber das war nicht explizit gegen Juden gerichtet, sondern gegen alle ethnischen Minderheiten, die in Lettland gewohnt haben. Und da musste man schon zusehen, wie man sich mit dem Regime arrangiert.
Mit dem Bildungsgesetz von 1919 war es allen Minderheiten in Lettland erlaubt, eigene Schulen zu gründen und die Eltern konnten frei entscheiden, in welche Schule ihr Kind geschickt wird. Es kam deswegen sehr häufig vor, dass z.B. Kinder aus russischsprachigen jüdischen Familien in die russischen Schulen geschickt worden sind. Jüdisches Schulwesen existierte in 4 Sprachen: Es gab russische jüdische Schulen, deutsche jüdische Schulen, jiddische jüdische Schulen und einige hebräische jüdische Schulen. Und jede Familie durfte selbst entscheiden wohin das Kind geht. Nach '34 hat sich die Situation entschieden gewandelt, indem es dann vom Staat vorgeschrieben war, dass jede Ethnie ihre Kinder in die entsprechende Schule schicken muss. Wenn sie sich gegen diese Schule entscheiden, dann müssen sie ihre Kinder in die lettische Schule schicken, und die Sprache der Familie spielte dann keine Rolle mehr. Man konnte dann einfach nur noch sagen: "`Ich bin Jude."' - "`Wenn du Jude bist, dann gehst du halt in die jiddische oder hebräische Schule"'. Es gab keine russisch- oder deutschsprachigen jüdischen Schulen mehr. Und dann wurde es komplizierter sich frei zu entscheiden und sich frei einer Kultur anzuschließen. Vorher war das viel einfacher und auch z.B. von den Russen war das akzeptiert.

\textbf{Wie ungefähr sah die Zusammensetzung der jüdischen Bevölkerung aus? Sie haben ja schon erwähnt, dass es diese Schulen gab und dann dementsprechend auch die unterschiedlichen Sprachgruppen.}

\textbf{Svetlana Bogojavlenska:} Die überwiegende Mehrheit hat Jiddisch gesprochen. Bei den Volkszählungen, die damals durchgeführt worden sind, davon gab es drei, wurde Hebräisch fast gar nicht erwähnt. Vielleicht nur bei ein paar Leuten, weil nach der Muttersprache oder nach der Familiensprache gefragt wurde und Hebräisch damals noch keine Muttersprache der Juden in Lettland war.
Von 93 000 Juden, die in Lettland 1930 erfasst wurden, haben ca. 10\% Deutsch als Familiensprache angegeben. In Riga haben 15\% der Juden im Alltag Deutsch gesprochen, in den Städten Kurlands bis zu 21\%. Russisch haben 1930 etwa 6\% der Juden in ganz Lettland als Alltagssprache angegeben.\footnote{Skujenieks 1930: 456, 466}  Es kann davon ausgegangen werden, dass mehrere Sprachen in der Familie gesprochen wurden. Jiddisch zum Beispiel parallel zum Deutschen oder zum Russischen. 
Lettisch als Familiensprache haben nur wenige angegeben, 652 Personen.\footnote{Ibid: 456}  Lettisch als Muttersprache war unter den Juden nicht so verbreitet. Aber wenn man sie dann nach der Sprachenkenntnis gefragt hat, war unter den Juden Prozent derer, die Lettisch beherrschten ziemlich hoch: 62\% im Jahr 1930.\footnote{Salnītis 1938: 103}

\textbf{In der Zwischenkriegszeit erstarkte die antisemitische Bewegung in vielen europäischen Ländern, in Polen oder in Deutschland. Gab es solche Tendenzen auch in Lettland?}

\textbf{Svetlana Bogojavlenska:} Tendenzen gab es, es gab sogar eine offen antisemitische Organisation, die hieß Pērkonkrusts. Es gab auch antijüdische Übergriffe, aber das waren tatsächlich nur sehr vereinzelte Übergriffe. Das war keine Massenbewegung, und 1934 hat Ulmanis diese Organisation sogar verboten, als staatsfeindlich und als eine, die dem lettischen Geist gar nicht entspreche, weil sie fremdenfeindlich ist. Einige Mitglieder dieser Organisation wurden später, während des Krieges, zu Tätern. Sie haben sich an der Ermordung der Juden beteiligt.

\textbf{Wie stark war der Zuspruch in der lettischen Bevölkerung für diese Organisation?}

\textbf{Svetlana Bogojavlenska:} Nicht so stark. Man kann nicht sagen, dass die Bevölkerung das getragen oder unterstützt hätte. Das war schon eine sehr begrenzte Gruppe.
 
\textbf{Also nicht zu vergleichen mit Polen oder Deutschland?}

\textbf{Svetlana Bogojavlenska:} Nein, überhaupt nicht. Antisemitismus war in Lettland, das weiß ich aus meiner Forschung über das 19. Jahrhundert, keine alltägliche Sache. Ich habe beinahe alle Archivbestände durchgeforstet und die ganze lettische Presse durchgelesen. Ab den 1880er Jahren mit der Pogromwelle in Russland kommt es zu zwei oder drei Zwischenfällen in ganz Lettland, wo vielleicht Stände, Obststände der jüdischen Marktkaufleute entweder zerstört wurden oder einige Juden verprügelt wurden, aber es kam zu keinem Totschlag. Es kam zu keiner Massenbewegung gegen die Juden, und man kann nicht sagen, dass Letten besonders antisemitisch eingestellt wären. Ich habe auch die lettische Literatur, die als Quelle dienen kann, analysiert und da gab es einige Erzählungen, in denen Juden sogar positiv dargestellt werden. Und es gibt ein Theaterstück, „Schneidertage in Silmači“, das immer noch jedes Jahr vor dem lettischen Johannisfest aufgeführt wird,\footnote{\textit{Skroderdienas Silmačos} – Schneidertage in Silmači von Rudolfs Blaumanis, 1902 in Riga uraufgeführt} was sicherlich in Deutschland nicht durchgehen würde. In Lettland ist es aber kein Problem, es gehört zum lettischen Kulturerbe. Da ist eine jüdische Gestalt zu sehen, ein jüdischer Hausierer. Das ist eine ironische Gestalt, könnte man sagen, sein Lettisch ist gebrochen und er sorgt dann immer für einen Lacher im Saal, weil er alles andere als normal ist. Aber auch dieses Stück zeigt, denke ich, dass die Juden einfach dazugehörten, sie waren im Alltag der Letten immer da. Der jüdische Hausierer war nicht aus Lettland wegzudenken am Ende des 19., Anfang des 20. Jahrhunderts, und die Kontakte bestanden schon. Man beneidete die Juden für ihre wirtschaftlichen Erfolge, und es lässt sich bestätigen, dass die Letten die in Folge der lettischen nationalen Bewegung versuchten, sich in der Wirtschaft zu etablieren, auf die Juden trafen, die da schon ihre Nischen besetzt hatten. Dann kam es tatsächlich zu ökonomisch motiviertem Antisemitismus, der in den Zeitungen zu sehen ist. Aber man kann z.B. von keinem Antijudaismus sprechen, von keinen Ritualmordvorwürfen usw. Man hat es dann eben im Alltag gesehen - "`Das sind unsere Konkurrenten, und die gilt es zu vertreiben"', aber nicht, weil sie Juden sind, sondern weil sie Konkurrenten sind. Wäre da ein anderes Volk gewesen, gehe ich davon aus, dass die Reaktion ebenso ausfallen würde.

\textbf{Der Antisemitismus und Antijudaismus Luthers hat sich also nicht niedergeschlagen?}

\textbf{Svetlana Bogojavlenska:} Ich habe keine Beweise dafür gesehen. Eine der ersten Zeitungen in lettischer Sprache war Mājas Viesis, diese wurde von den lutherischen Pastoren für die Letten herausgegeben. Und erstaunlicherweise wurden darin ab den 1860er Jahren viele Geschichten aus dem Alten Testament wiedergegeben. Aber nicht nur das. Jüdische Feste wurden beschrieben, zum Beispiel Sukkot. Das hätte ich von einer lettischen Zeitung gar nicht erwartet. Da wurden Hintergründe beschrieben, "`Warum feiern die Juden es?"' usw. Es war sehr interessant zu lesen, dass die Letten darüber unterrichtet wurden, wie die Juden ihre Feste feiern ohne da irgendwie etwas Negatives draus zu ziehen. Sie waren eben Nachbarn, sie haben zusammen gelebt, vor allem in Ostlettland, denn östlicher Teil Lettlands, Lettgallen, gehörte zum Ansiedlungsrayon.\footnote{Als Ansiedlungsrayon der Juden (Russisch: \textit{čerta osedlosti}) bezeichnet man die Teile des russischen Reiches, überwiegend in den Gebieten der heutigen Ukraine, Weißrussland, Litauen und Lettland, in denen Juden ohne eine Sondergenehmigung leben und arbeiten durften. Der östliche Teil Lettlands, ethnographisches Gebiet Lettgallen, gehörte zum Gouvernement Vitebsk, das zum Ansiedlungsrayon gehörte. Der Ansiedlungsrayon wurde erst 1915 offiziell abgeschafft. Seit den Liberalisierungsreformen des Zaren Alexander II. durften verschiedene Gruppen von Juden, z.B. Handwerker mit ihren Familien auch außerhalb des Ansiedlungsrayons wohnen. Auch davor gab es schon jüdische Berufsgruppen, denen es erlaubt war mit Sondergenehmigung außerhalb des Rayons zu wohnen: so z.B. in Kurland, Riga, St. Petersburg usw. Vgl. z.B Kotowski et al. (2001): 180-195.} Da sich die Juden laut Gesetz in den Städten registrieren mussten, kam es tatsächlich dazu, dass bis zu 90\% der Stadtbevölkerung Juden waren, die mit den Letten immer in Kontakt waren, die in der ländlichen Umgebung gewohnt haben. Die Synagogen wurden errichtet, es ist kein einziger Fall bekannt, wo irgendein Anschlag gegen eine Synagoge verübt worden wäre.

\textbf{Sie sagen "`miteinander in Kontakt"'? Gab es auch familiäre Verbindungen, zum Beispiel? Oder nur Freundschaften und so etwas?}

\textbf{Svetlana Bogojavlenska:} Familiäre Verbindungen gab es eher nicht. Es gibt statistische Erhebungen, auch aus der Periode in der das Territorium von Lettland ans Russische Reich angegliedert war. Und da hat man auch kontrolliert, wer mit wem geheiratet hat. Im Allgemeinen gab es sehr wenige religiös gemischte Ehen. Es gab unter den Juden auch einige Übertreter zum Christentum, nicht aber umgekehrt, denn das war auch verboten. Es gab einen Versuch der deutschen, evangelischen Mission, beispielsweise Juden in Kurland zu christianisieren, sie evangelisch-lutherisch zu machen. Die Mission konnten allerdings keine großen Erfolge feiern. Die Übertritte beliefen sich auf 10 Personen pro Jahr. Insgesamt waren die konfessionellen Gruppen stark voneinander abgegrenzt.

\textbf{Wie ist heute in der lettischen Bevölkerung die mehrheitliche Einschätzung von Karlis Ulmanis? Wie bewertet man ihn heute, welchen Ruf hat er?}

\textbf{Svetlana Bogojavlenska:} Es gibt ein Denkmal für Ulmanis in Riga. Dieses Denkmal ist sehr umstritten. Es gibt einen Teil der Bevölkerung der glaubt, dass diese Zeit die Beste war. Man glaubt ja auch immer, dass alles besser war, als man jung war. Die Zeit des Autoritarismus haben sehr viele Letten, die heute noch leben, miterlebt, die Zeit des demokratischen Lettlands jedoch kaum. Diejenigen, die diese Unabhängigkeitszeit erlebt haben, die haben schon in dieser autoritären Zeit ihr Leben verbracht. Und was kam danach? Danach kam das sogenannte "`schreckliche Jahr"', das Jahr der sowjetischen Besatzung, das so schlimm war, dass in der Erinnerung der Letten die Herrschaftszeit von Ulmanis, die damals auch nicht unumstritten war, tatsächlich als eine paradiesische Periode die Zeit überdauert hat. Sie meinen, die Zeit von Ulmanis war die Blütezeit. Wobei das nicht so stimmt, da war ja auch die ökonomische Krise, man blendet das Ganze also aus. Denn im Vergleich zu dem, was danach kam, war das natürlich ein Paradies. Das würde ich auch vielleicht sagen, aber eine andere Sache ist, wie geht man damit um, dass man so einer Persönlichkeit, die die Demokratie abgeschafft hat, ein Denkmal im Zentrum von Riga aufstellt? Dieses Denkmal ist in seiner Darstellung kein heroisches Denkmal, deswegen gab es kaum Widerstand dagegen. Es steht da, in einem Park, da kommen ein paar alte Omas vielleicht ab und zu hin, stellen Blümchen hin. Sein Todestag, der noch nicht ganz geklärt ist, wird erinnert. Er wurde von den Sowjets verhaftet und ist in einem sowjetischen Gefängnis 1942 gestorben. Obwohl ihm versprochen wurde, dass er auf freien Fuß gesetzt wird, wenn er alles unterschreibt und nicht gegen die Sowjets kämpft, und die Erlaubnis erhält in die Schweiz auszureisen. Im Endeffekt endete er in einem Gefängnis in Turkmenistan. Er war einer der Gründer Lettlands und einer der aktivsten Politiker, auch in der demokratischen Periode. Deshalb kann man seine Persönlichkeit so oder so sehen. Natürlich ist es ein Verbrechen seinerseits, die Demokratie abzuschaffen. Unter Historikern herrscht die einhellige Meinung, dass er das nur gemacht hat, weil er um seine politische Karriere bemüht war und Angst hatte, dass bei der nächsten Wahl des Parlaments seine Partei nicht mehr ins Parlament kann und er dann politisch ausgeschaltet wird. Politik war sein ganzes Leben. Und deshalb ist er wirklich eine sehr widersprüchliche Persönlichkeit. Aber von der Mehrheit der Bevölkerung und ihrer Einstellung kann man gar nicht sprechen. Die Meinungen in der Bevölkerung sind auch sehr widersprüchlich.

\textbf{Wie hat sich dann die Situation der Juden mit der sowjetischen Besatzung verändert?}

\textbf{Svetlana Bogojavlenska:} Es hat sich alles geändert mit der sowjetischen Besatzung. Erstens gab es den lettischen Staat nicht mehr, zweitens wurde eine Deportation vorbereitet und durchgeführt. Bei dieser Deportation wurden auch ca. 5000 Juden deportiert. Prozentuell gesehen war die jüdische Bevölkerung die, die am meisten unter dieser Deportation gelitten hat. Bei der ersten Deportationswelle wurden Vertreter der Intelligenz, nicht nur der lettischen, sondern auch der jüdischen und der russischen aus Lettland nach Sibirien gebracht, und dieses eine Jahr der sowjetischen Besatzung bedeutete für die Juden, sowie für alle anderen, Enteignung. Und eine Verschlechterung ihrer Lage und ein Verschwinden der politisch Rechte. 
Aber es gibt auch die Legende, dass der Rabbiner Dubin\footnote{Mordehai Dubin, 1889-1956, Politiker und einflussreicher spiritueller Führer} vor der sowjetischen Besatzung gefragt wurde, "`Was sollen wir jetzt machen?"', und dass er zu der jüdischen Gemeinde gesagt habe- er war auch politischer Vertreter der Gemeinde: "`Lieber gebt die Schlüssel von euren Geschäften an Stalin, als eure Leben an Hitler."' Dadurch lässt sich auch erklären, dass viele Juden sich dazu fast gezwungen sahen, die sowjetische Macht zu begrüßen. Es gibt sehr viele Fotos von den Demonstrationen, die die Sowjetmacht in Lettland begrüßten. Die organisierten sich nicht spontan, sondern wurden von den Sowjets organisiert. Die Beteiligung an diesen Demonstrationen ist auch das, was von der lettischen Seite den Juden gegenüber vorgehalten wird. "`Ihr wart diejenigen, die die Sowjets damals mit Blumen empfangen haben!"' Es gibt tatsächlich auch Fotos auf denen man sieht, dass da überwiegend, oder sehr viele Juden sind in der Menge. Und das ist unter anderem damit zu erklären, dass sie zwischen Pest und Cholera auswählen mussten. 

\textbf{Waren die Juden die einzigen Demonstrationsteilnehmer?}

\textbf{Svetlana Bogojavlenska:} Nein, natürlich nicht. Nein, es waren von der sowjetischen Besatzungsmacht organisierte Demonstrationen, aber wenn man eine lettgallische Kleinstadt nimmt in der 90\% der Bevölkerung Juden sind, dann sind auf dem Bild natürlich nur Juden. Dann sieht man da kaum Letten oder andere. Andere sind in den Dörfern, lettische Bauern, russische Bauern, die werden dann auch natürlich zum Demonstrieren angehalten, fotografiert wird aber in der Stadt.

\textbf{Kann man sagen, warum es prozentual vor allem Juden waren, die von diesen Deportationen betroffen waren?}

\textbf{Svetlana Bogojavlenska:} Weil sie viel Eigentum hatten, wirtschaftlich erfolgreich waren und weil sie gefürchtet wurden. Man fürchtete, wenn man die reichen und gebildeten Juden dalässt, organisieren sie einen Widerstand. Es wurde von der jüdischen Gesellschaft kurz darauf befürchtet, dass auch eine zweite Deportationswelle bevorstand, in der dann noch mehr Juden deportiert worden wären, aber das ist dann nicht geschehen, weil Lettland von Deutschland überfallen wurde.

\textbf{Wie wurden diese Listen zusammengestellt? Woher wusste man, wer gefährlich sein konnte?} 

\textbf{Svetlana Bogojavlenska:} Ich kann das nicht ganz genau sagen, ich weiß nur, dass es schon vorher vorbereitet wurde. Die Besetzung Lettlands passierte ja nicht einfach so von einem Tag auf den anderen, sondern wurde von der sowjetischen Seite vorbereitet, es wurden Agenten nach Lettland geschickt. Die lettische Polizei verzeichnete auch den plötzlichen Anstieg der Exilletten, die im letzten Jahr der Unabhängigkeit aus der Sowjetunion nach Lettland gekommen waren. Die Listen der verdächtigen Personen wurden seit Herbst 1940 zusammengestellt.\\
An die Informationen zu kommen war nicht so kompliziert. Es waren auch wirklich bekannte Familien.

\textbf{Gab es unter den Sowjets und vorher Antisemitismus in Russland? Welche Art von Antisemitismus gab es?}

\textbf{Svetlana Bogojavlenska:} Ja, eindeutig. Es gab stark ausgeprägten Antisemitismus, der auch nach dem Krieg Durchschlag hatte.
Z.B. ging die Pogromwelle 1881 von Russland aus. 1880 hat es schon angefangen, '81 mit der Ermordung des Zaren wurden die Juden der Ermordung beschuldigt, obwohl es Anarchisten waren, die sich in der Wohnung eines Juden versammelt hatten, als sie diesen Anschlag auf den Zaren planten. Und der jüdische Historiker Simon Dubnow, der 1941 in Riga bei einer der Massenerschießungen ums Leben kam, ging davon aus, dass diese Pogromwelle sogar vom Staat organisiert wurde. Das wurde durch spätere Forschungen widerlegt, was aber nur das Zeugnis dafür abgibt, das die Bevölkerung eher antisemitisch eingestellt war und aus verschiedenen Gründen auch selbst Pogrome verübt hat. Was die Regierung sicher nicht gemacht hat, war die Pogrome rechtzeitig zu stoppen. Die Gendarmerie hat immer ein paar Tage gewartet, bis da nichts mehr zu retten war.
1905, während der Revolution, gab es auch Pogrome, und die gingen auch nicht von staatlicher Seite aus, aber sie beschränkten sich auf den Ansiedlungsrayon der Juden. Dort, wo es eine sehr hohe Konzentration jüdischer Bevölkerung gab, da kam es auch zu den Pogromen. 1905 gab es nur ein Pogrom in Lettgallen, der dokumentiert ist, in Ludsen, heute Ludza. Ansonsten nicht. 

\textbf{Kann man dann sagen, dass der Antisemitismus in der Zwischenkriegszeit in allen ethnischen Gruppen in Lettland gleich stark, bzw. gleich schwach vertreten war?}

\textbf{Svetlana Bogojavlenska:} Vom Antisemitismus in der Zwischenkriegszeit halte ich nicht viel. Ich habe mich mit der Zwischenkriegszeit und der russischen Bevölkerung beschäftigt. Die größte russische Zeitung wurde von einem Juden herausgegeben und hat eigentlich für beide Bevölkerungsgruppen gearbeitet und für verschiedene Parteien. Die Zeitung hat sowohl für die russischen als auch die jüdischen Parteien Wahlkampf gemacht und Probleme beider Minderheiten gemeinsam besprochen. In der russischen Bevölkerung war jeder Jude, der sich zur russischen Kultur bekannte, als Russe angenommen. Die haben verstanden, "`Wir sind hier alle fremd, aber wir haben alle eine gemeinsame Kultur."' Juden, die im russischen Imperium von Russen nicht als Russen akzeptiert wurden, wie sie sich auch bemühten, wurden in Lettland plötzlich anerkannt als russische Angehörige. Ohne dass sie sich von ihrer Religion oder ihrer ethnischen Abstammung lossagen mussten. Jeder Jude konnte sagen: "`Ich fühle mich der russischen Kultur angehörig."' Was die lettische Bevölkerung betrifft, ich habe das ja auch schon angesprochen, war Pērkonkrusts die wichtigste antisemitische Organisation. Man kann vielleicht noch die Universität Lettlands erwähnen, wo die studentischen Korporationen gegründet worden sind. Da gab es lettische Korporationen, in denen kein einziger Jude vertreten war. Aber die waren auch anderen gegenüber nicht aufgeschlossen, auch Russen wurden da nicht aufgenommen. In den russischen studentischen Organisationen, oder übrigens auch in den polnischen, waren Juden drin. Wenn sie gesagt haben, ich fühle mich hier angehörig, wurden sie auch aufgenommen. 

\textbf{Wie war das Verhältnis der Letten zu den Deutschen 1941?}

\textbf{Svetlana Bogojavlenska:} Ich würde nicht pauschalisieren und sagen, alle Letten hätten Deutsche mit Begeisterung empfangen. Aber ich stimme dem Argument zu, dass das eine sowjetische Jahr so ein tiefer Einschnitt in die Geschichte Lettlands war, dass die Feindschaft gegenüber den Deutschen durch die Erfahrung 1940/41 mit der Sowjetmacht weg war. Dass die Deutschen tatsächlich als Befreier von der sowjetischen Besatzung angesehen worden sind. Als eine Chance, Lettland vielleicht wieder unabhängig zu machen. Es gab vom lettischen Zentralrat (\textit{Latvijas centrālā padome}), das ist eine Widerstandsorganisation, die gegen beide, d.h. gegen die deutsche und gegen die sowjetische Seiten gekämpft hat, Versuche, Lettland wieder als unabhängigen Staat herzustellen. Die lettische Selbstverwaltung dagegen, die von den deutschen gebilligt wurde, wurde am Ende zur rechten Hand der deutschen Besatzungsmacht. Die nationalsozialistische Besatzung hat eine kluge Politik in diesem Sinne durchgeführt. So hat sie zum Beispiel die Gottesdienste in den Kirchen wieder erlaubt, die in dem Jahr der sowjetischen Besatzung verboten waren. Die haben die sog. Pskower Orthodoxe Mission gegründet und die orthodoxen Kirchen in den besetzten Territorien wieder aufgemacht, auch in Lettland. Das hat natürlich die Sympathien derjenigen gebracht, die nicht direkt vom nationalsozialistischem Terror und Mord betroffen waren.\\
Es gab auch eine Zeitung in russischer Sprache, die hieß \textit{Za Rodinu}, für das Vaterland, und man hat so versucht, die russische Bevölkerung zu instrumentalisieren. Es war sehr interessant, dass die russische Intelligenz und die russischen Bauern darauf angesprochen wurden, wie schlimm es der russischen Bevölkerung während der Zeit der sowjetischen Besatzung ging und dass der große Hitler natürlich alles besser machen würde. Es wurden sogar verschiedene Beiträge in dieser Zeitung veröffentlicht, die in der Zwischenkriegszeit in der russischen Zeitung \textit{Segodnya} publiziert wurden. Dabei wurde natürlich verschwiegen, dass diese Zeitung von einem jüdischen Herausgeber stammte. 
Die antisemitische Propaganda der Nazis wurde sehr sorgfältig vorbereitet und der wichtigste Punkt war, dass die Repressionen, die die lettische Bevölkerungen 1940/41 durch die Sowjets erlitten hatte, von Juden organisiert und durchgeführt worden seien. Als im Rigaer Zentralgefängnis die Leichen der dort getöteten politischen Häftlinge gefunden worden sind, hat man sie alle in einen Hof gelegt und als Opfer der jüdischen Kommunisten dargestellt. Und fast sofort danach kam der Aufruf in der propagandistischen deutschen Zeitung in lettischer Sprache Tēvija, “das Vaterland”, sich zu formieren, um das Vaterland zu beschützen. Selbstschutzeinheiten, Polizeieinheiten zu bilden. Es wurden zuerst tatsächlich nur Freiwillige aufgenommen. Auf diesen Aufruf hin hat sich auch das Arājs-Kommando gebildet.

\textbf{Kann man abschätzen, wie viele Letten kooperiert haben mit den Nazis, was die Judenvernichtung betrifft?}

\textbf{Svetlana Bogojavlenska:} Man kann schauen, wie viele danach verurteilt worden sind durch die Sowjets. Das waren einige Tausende. Aivars Stranga hat die Zahl geschätzt in seinem Beitrag\footnote{``The accurate number of Latvians who took direct or indirect part in the annihilation of the Jews is not easy to establish; an approximate figure could constitute a few thousands.'' (Stranga (2005) 167) } in dem Sammelband "`The Hidden and Forbidden History of Latvia"', das von der Historikerkommission Lettlands herausgegeben wurde. Man hat in der sowjetischen Zeit allerdings auch vermieden zu sagen, dass die Opfer Juden waren. Die Kollaborateure wurden wegen Verbrechen gegen sowjetische Bürger verurteilt. Wenn die Überlebenden des Holocausts nach dem Krieg in verschiedenen Städten, wie z.B. in Riga, einen Gedenkstein aufstellen wollten, durfte man nicht darauf hinweisen, dass das jüdische Opfer waren. Es stand dann "`Opfern des Krieges"', "`Opfern des Faschismus"' oder "`den sowjetischen Opfern"' auf den Gedenktafeln. Das einzige, was in Riga gelungen war, war der Bau einer Gedenkstätte im Wald von Rumbula in den 1960er Jahren. Die Inschrift, dass dort sowjetische Bürger ermordet worden sind, wurde in drei Sprachen geschrieben: Auf Russisch, Lettisch und Jiddisch. Allerdings, wenn man an den Gedenktagen der Rumbula-Aktionen im November und im Dezember dorthin fuhr, da wurde man vom KGB oder vom Sicherheitsdienst überwacht und einige wurden dann sogar verhaftet. In das sowjetische Bild passte es nicht, dass ein Volk mehr als die anderen Völker der Sowjetunion während des Krieges gelitten hat. Die Akten der Verurteilten wurden geheim gehalten, bis 1991. Erst dann konnte man sie einsehen. Wenn man die Verhörprotokolle liest, dann sieht man ganz deutlich, dass es da um die Juden ging. Dass das die "`Selbstschutzbataillone"' waren, die eigentlich in dieser Zeit nur dazu dienten, um die Juden in den Kleinstädten Lettlands zu ermorden. Bis Ende August 1941 waren alle Juden in Kleinstädten Lettlands tot, sie wurden alle ermordet.
 
\textbf{Das waren also kleine Einsatzgruppen?}

\textbf{Svetlana Bogojavlenska:} Genau, in der Provinz waren es in der Regel die Freiwilligen. Die Befehle wurden von der deutschen Seite erteilt, also waren es keine spontanen Aktionen von Seiten der Letten, sondern es kam nur durch die Befehle der Besatzungsmacht. Aber die Schützen wussten schon, wofür sie da sind, und sie hätten sich auch dagegen aussprechen können.
Bei den großen Massenaktionen in Riga in Liepāja, Libau, haben sie nicht geschossen. Sie haben "`nur"' eskortiert. Aber in den Kleinstädten hatten die Deutschen noch nicht so viele Militärangehörige, und deshalb hat man die Einheimischen genommen. Das ist ja das Tragische: Es waren die Nachbarn, die die Leute umgebracht haben. Die Kranken, Kinder, Frauen, Älteren, das war egal, alle. In jeder Stadt wurden diese Einheiten gebildet, es waren nur Freiwillige. In dieser Zeit wurde noch keiner zwangseinberufen, wie das später der Fall war. In den ersten Monaten der Besatzung hat man nur auf Freiwillige gesetzt.

\textbf{Wurden bewusst die entsprechenden Nachbarn als Freiwillige eingesetzt?}

\textbf{Svetlana Bogojavlenska:} Ja, es ist in der Geschichtswissenschaft in Lettland inzwischen auch ein Konsens, dass es von der deutschen Seite so gewollt war, um das als spontane Aktionen zu verkaufen. Dass die Einheimischen die eigenen Nachbarn umbringen, dass der Antisemitismus, der da herrschte, einfach selbst ausgebrochen sei, die "`Vergeltung"' für die vermeintlichen Verbrechen der Juden während des einen Jahres der sowjetischen Besatzung, so wurde das dargestellt. In den Aufrufen, sich zu den freiwilligen "`Selbstschutzmannschaften"' zu melden, stand dann schon, dass diese Einheiten nicht nur für den Schutz da seien, sondern auch dazu, das Vaterland von "`fremden Elementen"' zu bereinigen. Von Kommunisten, von Juden und so weiter. Und das war auch allgemein in der lettischen Bevölkerung bekannt. Noch Jahrzehnte danach war bekannt, wer beteiligt war, auch wenn die Städte sich verwandelt hatten, denn die jüdische Bevölkerung fehlt ja völlig. In die Häuser, die der jüdischen Bevölkerung gehörten, zogen Letten ein. In kleineren Städten war auch bekannt, wessen Familien an der Ermordung beteiligt waren. Man wusste dann auch, wer von den Sowjets verurteilt wurde. Wer dann die 10 Jahre in Sibirien oder in einem Gefängnis verbracht hat und warum er da war. Und diese Familien, so wie ich das in einer Stadt in Lettgallen in Gesprächen mit den Einheimischen Anfang der 2000er Jahre verstanden habe, waren nicht so beliebt. Also die galten nicht als Opfer der sowjetischen Verfolgung. Diejenigen, die bei der zweiten Deportationswelle 1947 deportiert worden sind, die schon.

\textbf{Gab es innerhalb der Freiwilligenverbände im Nachhinein irgendwelche, die sich reuig geäußert haben?}

\textbf{Svetlana Bogojavlenska:} Es gibt eigentlich nur eine Forschung darüber, das ist eine Forschung von Rudīte Vīksne\footnote{Vīksne (2005)}.  Sie hat die Verhörprotokolle von den Freiwilligen aus dem Arājs-Kommando analysiert. Da wurden die Beschuldigten auch nach den Gründen gefragt, warum sie dem Kommando beigetreten waren. Nur sehr wenige meinten explizit, dass sie Juden hassten. Und viele meinten, man konnte das nicht mitansehen, wie die Juden ermordet worden sind, sie hätten es sich völlig anders vorgestellt. Aber wirkliche Reue zeigte keiner. Zumindest geben es diese Verhörprotokolle nicht wieder. Man muss aber damit auch sehr vorsichtig umgehen, die Verhörmethoden vom KGB sind ja weltbekannt. Das, was ich selbst gelesen habe, sind Unterhaltungen von Richtern und Verhörenden aus Preiļi: "`Was haben Sie denn gefühlt, wenn Sie diese Menschen umgebracht haben? Das waren doch Ihre Nachbarn, die sie gekannt haben! Das waren doch Kinder und Frauen, das sind doch auch Menschen!"'. Und dann kam die Gegenfrage, "`Sind die Juden etwa Menschen?"'. Also die Gehirnwäsche, oder tatsächlich die persönliche Überzeugung hat da wahrscheinlich gewirkt.

\textbf{Das sowjetische Regime wollte sich ja nach dem Krieg möglichst stark von der Naziideologie abgrenzen. Ist es dann nicht verwunderlich, dass der Antisemitismus nicht verschwunden ist? Wie passt das zusammen?}

\textbf{Svetlana Bogojavlenska:} Es gibt eine sehr bekannte Rede von Stalin, in der sich Stalin bei allen Völkern der Sowjetunion für den Sieg im Großen Vaterländischen Krieg bedankt, aber besonders bei dem russischen Volk, dass das größte Volk unter allen Völkern der Sowjetunion sei. Und jeder, der das bezweifelte, galt als Feind, obwohl Stalin selbst kein Russe war, sondern Georgier. Dieser ganze Internationalismus beinhaltete immer die Unterstreichung der Rolle eines Volkes. Alle mussten zu sowjetischen Bürgern werden, nicht zu lettischen Bürgern der Sowjetunion. Es gab eine Tendenz zur Russifizierung und Sowjetisierung. Diejenigen, die versucht haben, ihre nationale Kultur in der Zeit des Stalinismus zu pflegen oder wiederzubeleben, wie es Juden versucht haben, waren zum Scheitern verurteilt. Da es für die jüdische Bevölkerung sehr wichtig war, nach dem Krieg die Reste zusammenzuhalten, haben sie versucht das jüdische Theater in Riga wiederzueröffnen. Das wurde ihnen verboten. 
Das war noch Anfang der 50er vor Stalins Tod, nach der Zerschlagung des Jüdischen antifaschistischen Komitees 1948.

\textbf{Gab es da schon Auswanderungswellen nach Israel?}

\textbf{Svetlana Bogojavlenska:} Ja, es gab zumindest Versuche. Auch aus Litauen emigrierten viele Überlebenden Juden. Die erste größere Auswanderungswelle nach Israel gab es in den 1970ern, als es vielen gelungen war über Ungarn die Sowjetunion zu verlassen. Und danach noch einmal in den 80ern. 

\textbf{Gab es auch vorher schon Auswanderungswellen?}

\textbf{Svetlana Bogojavlenska:}Ja, nach den jüdischen Pogromen im 19. Jahrhundert, auch aus Lettland. Und nach der Revolution 1905, die dann auch mit jüdischen Pogromen einherging. 1905 ging es aber überwiegend in die USA und nur ein sehr kleiner Teil nach Palästina. Der Zionismus war ziemlich weit verbreitet, es gab verschiedene zionistische Organisationen. Es gab auch Sommerfreizeiten für die Jugendlichen, die speziell geschult worden sind, wie sie in Palästina die Wüstengebiete in Ackerland verwandeln konnten, "`\textit{Hachschara}"'. Das war sehr populär. Zu Beginn waren es nicht viele, die ausgewandert sind, denn die Zwischenkriegszeit war gerade die Blütezeit der jüdischen Gemeinde in Lettland. Diejenigen, die hingefahren sind und dort sich niedergelassen haben, wurden als Helden angesehen. Aber man kann da nicht von einer großen Auswanderungswelle nach Palästina sprechen. Den Staat gab es ja noch nicht und die Idealisten meinten, "`Wir bereiten jetzt unsere Jugend hier bei uns vor."' Und dann wurde auch Hebräisch in jeder jüdischen Schule gelehrt.

\textbf{Wie sahen die politischen Parteien in der Zwischenkriegszeit aus?}

\textbf{Svetlana Bogojavlenska:}  Die waren sehr zersplittert. Aus dieser Zeit stammt der jüdische Witz: Wo drei Juden sind, sind fünf Parteien. Die jüdische Minderheit vertrat sehr verschiedene Auffassungen, auch was das Leben im lettischen Staat anging. Und man kann auch von keiner Homogenität der jüdischen Bevölkerung in Lettland sprechen. Es gab die sozialistischen Bundisten, die Erzkonservativen Zionisten, die sich einen Gottesstaat in Palästina gewünscht hatten, und auch demokratische Bewegungen.

\textbf{Gab es von diesen Bewegungen Anti-Antisemitische Arbeit?}

\textbf{Svetlana Bogojavlenska:}  Nein. Anti-Antisemitische nicht; die Juden, wie die Russen, die Deutschen und die Weißrussen und die Polen, wie alle Minderheiten, mussten auch in der ganzen demokratischen Periode für ihre Rechte kämpfen. Für ihre Rechte als ethnische Minderheit, dafür, ihre eigenen Schulen zu haben und sie selbst zu verwalten. Weil es immer wieder von Seiten der Regierung Bestrebungen gab, das Schulsystem zu unifizieren und die Schulen zu lettisieren um eine homogene Bevölkerung zu bilden. Es gab Regierungen, die sehr aufgeschlossen den Minderheiten gegenüber waren, unter denen man weniger darum kämpfen musste. Und es gab auch Regierungen, die alles daran gesetzt haben, diese Gesetze, wenn nicht ganz abzuschaffen, so doch ihre Wirkungsbereiche zu begrenzen. Indem sie zum Beispiel die Inspekteure für ethnische Schulen abgeschafft haben und gesagt haben: Jetzt kontrolliert das Bildungsministerium die Schulen der Minderheiten. Das galt zwei Jahre lang, dann hat die neue Regierung die Stellen wieder eingeführt auf Nachdruck der Minderheiten, die dann dafür zusammengearbeitet haben.

\textbf{Wie wurde die Auswanderung nach dem Zusammenbruch der Sowjetunion in Lettland wahrgenommen?}

\textbf{Svetlana Bogojavlenska:}  Dazu habe ich schon Eingangs einiges erzählt. Das Bewusstsein war da, dass Juden auswandern. Aber auch Verwunderung. Was die Migration der jüdischen Bevölkerung in der Sowjetunion betrifft, so muss dazu gesagt werden, dass die Kontakte zwischen den Juden und der russischsprachigen Bevölkerung in Sowjetrussland besser ausgeprägt waren, weil fast alle Juden, die in Sowjetlettland gelebt haben, nach dem Krieg Russisch gesprochen haben. Viele von ihnen sind nach dem Krieg nach Lettland eingewandert, aus der Ukraine oder aus Russland. Und die hatten auch eigentlich keinen Bezug zum vorherigen Lettland. Den Holocaust haben knapp 1500 lettische Juden überlebt, und nicht alle von ihnen sind nach Lettland zurückgekehrt. Diese 1500 sind mit denjenigen zusammengerechnet, die es geschafft haben, vor der deutschen Besatzung Lettland zu verlassen. Von denjenigen, die dort geblieben waren, sind vielleicht knapp 1000 am Leben geblieben.

\textbf{Wie haben sie den Holocaust überlebt? Wurden sie versteckt?}

\textbf{Svetlana Bogojavlenska:} Ja, einige haben sich versteckt. Es sind sogar bis zu 600 Fälle der Judenrettung in Lettland heute bekannt. Das heißt aber nicht, dass sie alle überlebt haben. Anfang der 1990er waren um die 200 Fälle bekannt, in denen tatsächlich Juden überlebt haben. Einige Fälle sind in den Akten der Polizei aufgezeichnet worden, d.h. dass jemand gefunden wurde, der einen Juden versteckt hat. Das bedeute für beide den Tod. Wie überlebt man den Holocaust? Schwierig zu sagen. Viele sind 1944 nach Auschwitz gekommen und dort dann die Befreiung erlebt, einigen ist die Flucht 1944 gelungen, als die sowjetische Armee sich näherte.

\textbf{Haben Sie ein Beispiel von einer Person, die überlebt hat?}

\textbf{Svetlana Bogojavlenska:}  Herr Marģers Vestermanis ist die prominenteste überlebende Persönlichkeit in der jüdischen Gemeinde. Als ich dort noch gearbeitet habe, waren häufig seine Freunde da, also Leidensgenossen, mit denen er zusammen im Ghetto und im KZ war, Konzentrationslager Kaiserwald. Die haben sich auch gegenseitig Bestätigungen darüber ausgestellt, dass sie bezeugen können sich im Ghetto oder im KZ begegnet zu haben. Weil es keine Papiere gibt, die das belegen können. Als die Wehrmacht sich aus Lettland zurückgezogen hat, wurden die Dokumente verbrannt. Deshalb war es schwierig nachzuvollziehen, wie viele da inhaftiert wurden. Die Familie von Herrn Vestermanis ist bei der Rumbula-Aktion in Riga 1941 umgekommen. Er hatte zwei Geschwister, einen älteren Bruder und eine Schwester. Sein Vater war Textilfabrikant in Riga. Er war ein sogenannter kurländischer Jude aus dem Westen Lettlands. Seine Muttersprache war Deutsch, und seine Mutter war russische Jüdin. Die erste Muttersprache von Herrn Verstermanis war Lettisch, weil er eine lettische Nanny hatte. Dann hat er Deutsch gelernt, er war im deutschen Kindergarten. Er meinte, wäre die Familie bei der sowjetischen Deportation von Juni 1941 deportiert gewesen, hätten sie eine Chance gehabt zu überleben. Er selbst hat sich als Elektriker im Ghetto ausgegeben. Er meinte auch, dass er wie durch ein Wunder überlebt hat, weil ihm das abgekauft wurde, obwohl er einmal das ganze Haus, an dem er arbeiten musste, lahmgelegt hat. Und er hat sich danach im Gespräch mit mir auch gewundert, dass er damals deswegen nicht umgebracht worden war.

\textbf{Kommen wir zu der Veranstaltung am 16. März in Lettland, die jedes Jahr einen internationalen Aufschrei erregt. Vielleicht können Sie darauf eingehen, wie der Charakter dieser Veranstaltung ist und welche Interessen hinter dieser Veranstaltung stehen.}

\textbf{Svetlana Bogojavlenska:} Das ist eine Veranstaltung der Veteranen der lettischen Waffen-SS Legion, die 1943 gegründet wurde. Die lettische Waffen-SS hieß auch wie überall in Europa Freiwilligen Waffen-SS Division, was aber nicht stimmte. Freiwillig waren nur die Schutzmannschaftsbataillone, zu denen sich Letten 1941 tatsächlich freiwillig gemeldet haben. Diese wurden dann 1943 in die 15. Waffen-SS Division eingegliedert. Das waren auch diejenigen, die an der Judenvernichtung 1941 teilgenommen haben, auch an den "Sonderexpeditionen" und an der Vernichtung der Zivilbevölkerung in Weißrussland beteiligt waren. All die anderen wurden einberufen, sind aber auch zur Musterung erschienen. Das sage ich, weil uns aus dem Baltikum ein anderer Fall bekannt ist. Die Waffen-SS Legion wurde auch in Estland gegründet und in Litauen. In Estland ist es gelungen, in Lettland ist es gelungen, in Litauen nicht. Nur jeder Fünfte erschien dort zur Musterung. Und diejenigen die erschienen, nur dürftig ausgestattet, haben sich dann geweigert den Eid auf Hitler zu leisten. Und 1944 musste man einsehen, dass sie nutzlos sind. Und deswegen gibt es keine litauische Waffen-SS Legion. 

\textbf{Das heißt aber, Widerstand dagegen war offenbar möglich und wurde auch nicht so sehr geahndet?}

\textbf{Svetlana Bogojavlenska:} Zumindest nicht in Litauen, in Lettland schon mehr. Die ganzen Familien konnten inhaftiert werden, man hat nach ihnen gesucht. Aber wenn Sie den berüchtigten Film "`Lettische Legion"' anschauen, von Uldis Neiburgs,\footnote{Latvian Legion, Filmstudio "`Deviņi"', 2000.} da haben einige Einberufenen erzählt: "`Es kam dieses Zettelchen, ich muss dort erscheinen an dem und dem Datum. Und dann fragt der Vater: "`Und was machst du jetzt? Gehst du hin oder gehst du in den Wald?"'. Was konnte man machen, man wusste ja, die Familie wird verfolgt, dann ging ich hin."' Herr Vestermanis erzählte, dass die Partisaneneinheit, die er im Wald getroffen hatte, als er während des Todesmarsches geflohen war, zum großen Teil aus den Deserteuren der lettischen Waffen-SS bestand. Es gab Leute, die sich gewehrt haben. In Litauen sah es natürlich so aus, dass man sich massenhaft dagegen gewehrt hatte. In Lettland nicht. Sie haben sich dann tatsächlich zusammen mit der deutschen Wehrmacht an den Kämpfen an der Front beteiligt. Sie wurden aber nicht mehr zur Ermordung der Zivilbevölkerung eingesetzt, deshalb kann man tatsächlich sagen, dass die 1943 Einberufenen keine Verbrecher im Sinne des Nürnberger Tribunals sind. Aber es sind auch keine Helden. Sie haben den Eid auf Hitler geleistet, auch wenn viele danach erzählt hatten, "`Ich hatte so ein ungutes Gefühl dabei"'. Es ist auch so, dass dieser 16. März tatsächlich als ein offizieller Feiertag in den lettischen Kalender aufgenommen wurde. Ein paar Jahre später, als es einen großen internationalen Aufschrei gab, wurde es auf Vorschlag der damaligen Präsidentin Vaira Vīķe-Freiberga wieder aus dem Kalender gestrichen. Und jetzt ist es in Riga so, dass jede öffentliche Veranstaltung bei der Stadtverwaltung angemeldet sein muss. Jedes Jahr reichen die Veranstalter die Anmeldung ein, bekommen eine Absage und gehen gerichtlich dagegen vor. Und das Gericht erlaubt es jedes Mal wieder. Das Schlimme ist, dass diese Veranstaltung auch die Neonazis mobilisiert, die antisemitisch eingestellt sind, aus dem In- und Ausland. Auch Rechtsradikale aus der Ukraine, aus Ungarn, sind jetzt jedes Jahr wieder dabei.
Es gibt einen Legionären Friedhof in Vidzeme, in Lestene. Da wurde auch ein Denkmal errichtet, aber es ist anders als in Deutschland, wo es mit Scham gesehen wird. In Lettland gelten die Waffen SS Veteranen tatsächlich als Helden, nicht als Opfer. Zum Beispiel, sieht man in besagtem Film am Ende einen Mann, der in der sowjetischen Armee gekämpft hat, gegen seinen eigenen Bruder, im kurländischen Kessel. Er war jünger, er wurde also später einberufen, als sowjetische Armee nach Lettland einrückte und die Wehrmacht und die Legion sich nach Westen, Kurland zurückzogen. Sein Bruder ist im kurländischen Kessel gefallen. Und er weiß auch, dass er vielleicht derjenige ist, der ihn erschossen hat. Er weiß es ja nicht. Er sagt da: „Zu Sowjetzeit sind die die Verbrecher gewesen, jetzt sind wir die Verbrecher." Das ist eine lettische Familie, die auf beiden Seiten kämpfen musste.
Das Interessante ist, als die ersten Märsche zum 16. März organisiert worden sind, wurde Herr Vestermanis, der auch Historiker ist, gefragt, was er dazu meint. Er war damals der Meinung, die sind keine Verbrecher, die wurden einberufen. Die Verbrecher wurden danach tatsächlich auch von der Sowjetmacht verurteilt und zur Rechnung gezogen. Das sind einfach Veteranen. Es ist unklug, dass sie jetzt da rausgehen. Sie durften das die ganze Sowjetzeit nicht, sie galten ja als Verbrecher, wurden aber nicht bestraft dafür, dass sie in der Legion waren. Die Mehrheit nicht. Herr Vestermanis hat sie damals verteidigt.

\textbf{Aber er wird nicht die Sichtweise der Waffen-SS Veteranen verteidigen, oder?}

\textbf{Svetlana Bogojavlenska:} Nein, natürlich nicht. Er hat sie auch als Opfer des Krieges dargestellt. Man kann sie aber nicht mit den anderen Opfern vergleichen. Einmal war sogar der Verteidigungsminister bei deren Gedenkveranstaltung in Riga dabei. Seine Handlung wurde natürlich danach verurteilt von der Regierung. Die evangelische lutherische Kirche macht allerdings jedes Mal mit. Es gibt einen großen Gottesdienst im Dom. Die marschieren dann von der Domkirche durch die ganze Altstadt zum Freiheitsdenkmal. Früher war das nur auf dem Friedhof in Lestene. Aber jetzt hat es offizielle Züge durch die Beteiligung der lutherischen Kirche, die im Dom auch zum Beispiel am Unabhängigkeitstag Lettlands einen Gottesdienst feiert. In derselben Kirche, mit demselben Pfarrer. Jedes Jahr sind am 16. März auch mehrere Parlamentsabgeordnete dabei.

\textbf{Gibt es lettische Politiker oder andere bedeutende lettische Figuren, die sich antisemitisch äußern?}

\textbf{Svetlana Bogojavlenska:}  Ja, es gibt die Partei Tēvzemei un Brīvībai, “für das Vaterland und Freiheit”. In dieser Partei gibt es ultranationale Kräfte, die den extremen Nationalismus salonfähig machen wollen. Sie kommt immer ins Parlament hinein, seit Anbeginn. Von denen kommt ab und zu schon was, aber nicht von den Abgeordneten in der Saeima. Die sind natürlich gegen alle, nicht nur gegen Juden.

\textbf{Gibt es zivilgesellschaftliche Initiativen gegen Antisemitismus und Rassismus in Lettland?}

\textbf{Svetlana Bogojavlenska:}  Staatliche Maßnahmen gibt es jede Menge. Es gibt immer wieder europäische Projekte, es gibt immer wieder soziologische Untersuchungen, es gibt immer wieder Empfehlungen der Wissenschaft an die Politik, was man noch machen könnte, um die Gesellschaft zu konsolidieren und demokratischer zu gestalten. Aber ob das alles tatsächlich da ankommt, wo es ankommen sollte, das ist schwer zu beurteilen. Es sieht nicht danach aus. Offen antisemitische Politiker werden auch nicht so stark angeprangert. Die Kritik kommt dann immer nur von der Opposition und nicht von den etablierten lettischen Parteien.

\textbf{In welcher Weise werden jüdische Letten heute im Alltag mit Antisemitismus konfrontiert?}

\textbf{Svetlana Bogojavlenska:} Offen überhaupt nicht, aber sie zeigen sich sehr selten als Juden in der Öffentlichkeit. Ich habe meine jüdischen Freunde in Lettland gerade gefragt und die meinen, latenter Antisemitismus sei da. Man spürt, dass man anders behandelt wird. Eine gewisse Aggressivität im Verhalten, vor allem wenn es um den Beruf geht. Man wird da nicht so gerne gesehen und wird ausgegrenzt, wenn man als Jude erkannt wird. 

\textbf{Und wie groß ist das Problembewusstsein? Wie sehr ist Antisemitismus im Bewusstsein der Leute in Lettland?}

\textbf{Svetlana Bogojavlenska:} Ich glaube, es ist nicht im Bewusstsein. Es gibt lettische Wissenschaftler, die sehr wohl wissen und verstehen, dass es dieses Problem gibt. Aber ich denke nicht, dass in der Mehrheit der Bevölkerung ein Bewusstsein dafür da ist. Wenn es darauf ankommt, dann sagt man sowas wie hier in Deutschland: "`Man darf das wohl noch sagen..."', "`Ich bin kein Antisemit, aber..."' 

\textbf{Wissen Sie, ob jüdische Institutionen dem Antisemitismus ausgesetzt sind?}

\textbf{Svetlana Bogojavlenska:} Ich weiß, dass die Synagoge in Riga von der Polizei rund um die Uhr bewacht wird. Einmal wurde da eine Flasche mit Brennstoff reingeworfen. Und die Wände der Synagoge wurden beschmiert. Aber das liegt schon mehr als 10 Jahre zurück. Man hat dann auch auf den Bildern der Überwachungskamera gesehen, dass das Jugendliche waren. Und der damalige Rabbiner, Natans Barkans, meinte, die sollten einfach zuhause von ihren Eltern dafür bestraft werden. Er glaubte auch nicht, dass das irgendwie fundiert antisemitisch war. Aber dass die Synagoge als Zielscheibe diente, das ist schon markant. Man könnte dann sagen, wenn das Jugendliche sind, woher haben sie das? Und dann bietet sich die Antwort, die haben das aus dem Umfeld, aus der Familie. Höchstwahrscheinlich haben sie etwas gehört und dann ist es in die Taten übergegangen. Diese Jugendlichen sind jetzt inzwischen erwachsen. Was sie jetzt darüber denken, weiß man nicht. Die Synagoge wird seitdem rund um die Uhr von der Polizei bewacht. Da steht immer ein Bus voller Polizisten. Die jüdische Gemeinde hatte mal ein sehr offenes Durchgangssystem. Man konnte da einfach so rein. Und viele Ausländer haben das bewundert und uns im Museum gefragt: "`Haben Sie hier keine Angst, einem Angriff der Antisemiten zum Opfer zu fallen?"' Man hatte tatsächlich keine Angst. Inzwischen kommt man nur durch die eine Tür rein. Die zweite Tür ist geschlossen. Auch um ins Museum zu kommen muss man sagen wohin und zu wem man möchte.
Ab und zu sind es Letten, die dort arbeiten. Nur leider sind das Ausnahmen. Die Letten werden dann immer gefragt "`Was machst du da? Das ist doch ein jüdisches Museum."' Ich wurde auch immer gefragt "`Bist du Jüdin?"', und musste sagen "`Nein"', "`Und was machst du dann da?"', "`Arbeiten?"'. Als die renovierte Synagoge wiedereröffnet wurde, wurde darüber in der lettischen Presse berichtet. Ich habe die Kommentare der Leser gelesen und Angst bekommen. Einer der harmlosesten Kommentare war "`Müssen wir Letten tatsächlich davon in Kenntnis gesetzt werden?"'. Das ist diese Nicht-Bereitschaft, das als Teil der Kultur und Geschichte Lettlands anzuerkennen.
Was allerdings die Aufarbeitung des Holocaust betrifft, da hat Lettland, zumindest die lettische Wissenschaft, große Fortschritte gemacht. Es ist fast alles aufgearbeitet worden, was aufgearbeitet werden konnte. Fast alle Archivbestände sind gesichtet und systematisiert worden. Und das Okkupations-Museum hat sogar zusammen mit dem jüdischen Museum vor einigen Jahren, 2011, eine Ausstellung zusammen gemacht namens "`Rumbula. 1941. Anatomie des Verbrechens"',\footnote{Die Ausstellung ist auf der Internetseite des Museums digital zugänglich: http://okupacijasmuzejs.lv/rumbula/en} explizit zur Judenvernichtung.

\textbf{Ich hätte noch eine Frage zur Schulbildung in Lettland. Wie sehr sind in den Lehrplänen die Themen Judentum und Diskriminierung gegen Juden verankert?}

\textbf{Svetlana Bogojavlenska:} Das Thema ist auf jeden Fall präsent. Es war sogar eine Lettin, Ieva Gundare, sie arbeitete bis vor kurzem im Okkupationsmuseum, eine der ersten, die sich überhaupt überlegt hat, wie man das den Letten erklären kann, ohne den Nationalstolz zu verletzen. Zu sagen, das lettische Volk hat bei der Judenvernichtung mitgemacht, nicht das ganze Volk, sondern einige Letten. Dem jüdischen Museum war es sehr wichtig zu sagen, dass nicht das ganze Volk am Judenmord beteiligt war, sondern bestimmte Individuen, die namentlich bekannt sind. Man kann nicht das ganze Volk für den Judenmord verurteilen. Die anderen waren ja auch Opfer. Und das war sehr wichtig auch für Ieva Gundare und das Okkupationsmuseum. Es gab bestimmte Menschen die am Judenmord beteiligt waren, es gab aber auch Menschen, die den Juden geholfen haben. Und es gab auch diejenigen, die sich das gleichgültig angeschaut haben. Oder mit Schaudern. Oder, mit Staunen, "`was passiert jetzt?"', und sie wussten nicht, was sie tun könnten. Frau Gundare hat Pädagogik studiert und war diejenige, die 2001-2002 die Arbeitsmaterialien, die ersten Arbeitsmaterialien für lettische Schulen in engen Beratungen mit dem Museum "`Juden in Lettland"' ausgearbeitet hat, um das Thema den Schulkindern nahe zu bringen. Sie hat dann danach im Okkupationsmuseum das gleiche für die sowjetischen Deportationen gemacht. In dem sie auch aufgezeichnet hat, dass nicht nur Letten darunter gelitten haben, sondern auch andere ethnische Gruppen, die in Lettland zu dieser Zeit gewohnt haben.
Es gab schon die Tendenz zu sagen, dass es ein Genozid gegen das lettische Volk war. Was aber nicht stimmt, es war kein Genozid gegen das lettische Volk. Das waren Repressionen gegen alle Völker Lettlands, gegen die lettische Bevölkerung oder vielmehr gegen Bevölkerung Lettlands. Weil wenn man lettisch sagt, denkt man in Lettland nur an die Letten, nicht an die Bevölkerung Lettlands, die ja multiethnisch aufgestellt ist. Und das war ihr dann auch gut gelungen.
Es gibt in der Schule in jedem Fall ein Programm. Nur muss man bedenken, dass der Schulplan natürlich anders ist als hier. Während hier dem Nationalsozialismus viel Zeit gewidmet wird, ist es dort anders, man hat dort andere wichtige Themen. Und es hängt sehr vom Lehrer ab. Ich habe auch schon Berichte gehört, dass die Lehrer das Thema einfach weglassen, sodass der Holocaust gar nicht erwähnt wird. An der Universität gehört es dazu. In der Schule muss es wie gesagt mindestens eine Stunde sein, aber die wird nicht von jedem Lehrer durchgeführt. Aber ich weiß, dass man an den Schulen Projektwochen hat, und erstaunlicherweise entscheiden sich immer noch ziemlich viele Schulen, auch aus der Provinz, für einen Gang ins jüdische Museum.
\end{otherlanguage}