\section{Dr. Svetlana Bogojavlenska}
\begin{otherlanguage}{ngerman}
\textit{Dr. Svetlana Bogojavlenska studied history from 1995 to 2002 at the University of Latvia in Riga. Between 1997 and 2002, she worked at the museum ``Jews in Latvia''. From 2003 to 2008, she worked as a doctoral student and lecturer at Johannes Gutenberg University Mainz. Her doctoral thesis is entitled ``The Formation and Position of Jewish Society in Riga and in Courland Governorate 1795-1915''. From 2011 to 2016, she worked in the DFG project ``Russian in the Latvian Context: Russian Identity Formation in Latvia 1914-1940'' at the Department of History of the University of Mainz. Since 2018, she is a Research Associate in the project ``Before the Cultural History: Functions and Dynamics in Russian Historiography in the European Context (1750-1830)''. Her research interests include Jewish culture and history, the history and theology of the Orthodox Church, sacred music and art, comparative religion, minorities in Russia and the Baltic States, and cultural transfer.\\
For the interview she visited us in Nuremberg, where the interview took place on September 5th, 2017. \\
The interview was in large parts centred around the Interwar Period (1918-1940). The respective passages have been crossed out for the print version of our book, however, and can be found in the online version.}\par
\vspace*{2em}
\textbf{Wann haben Sie, wenn Sie sich noch erinnern können, zum ersten Mal von dem Begriff "`Jude"' oder "`jüdisch"' gehört?}

\textbf{Svetlana Bogojavlenska:} In der Kindheit. Ich glaube, ich habe es zum ersten Mal von meiner Großmutter gehört und der Begriff war schon mit Vorurteilen beladen. Ich hatte das Gefühl, dass das etwas Fremdartiges ist, was vielleicht ein bisschen gefährlich ist. Noch gefährlicher waren "`Zigeuner"'. In der Gegend, in der ich gewohnt habe, war das die Wahrnehmung der Bevölkerung oder zumindest die Wahrnehmung meiner Großmutter. Seitdem war der Begriff immer wieder irgendwie da, aber ich habe mich nicht viel damit beschäftigt, bis ich auf die Idee kam, dass ich eigentlich viele Klassenkameraden hatte, die plötzlich nach Israel ausgewandert waren.\\
Vorher wusste man gar nicht, dass sie etwas mit dem Judentum zu tun hatten. Man hätte es auch gar nicht vermuten können. Sie haben sich nicht speziell als Juden ausgegeben. Das waren ganz normale Klassenkameraden wie jede und das wurde gar nicht thematisiert. Dann waren viele plötzlich in Israel. Dann habe ich angefangen, mich dafür zu interessieren, welche Geschichte dieses Volk hat - was haben sie in Lettland gemacht und wie kam es dazu, dass wir eigentlich so wenig über sie wissen? Über ihre Geschichte und warum sie so unscheinbar da sind. Und plötzlich wandern sie nach Israel aus, weil sie sagen: "`Ich möchte Jude sein"'. Warum kann man das nicht in Lettland sein, zum Beispiel? Auch das war meine Frage.

\textbf{Haben sie gesagt, sie könnten in Lettland keine Juden sein?}

\textbf{Svetlana Bogojavlenska:} Zumindest ein Klassenkamerad von mir, mit dem ich Schriftverkehr hatte über Jahre. Seine Argumentation war: "`In Israel habe ich zu mir zurückgefunden"'. Weil es keine Hindernisse mehr gibt, weil man sich frei zur eigenen Identität äußern kann. Das war schon eine Überraschung für mich, weil ich ihn vorher nicht als Juden wahrgenommen habe und er es auch nie in der Schule thematisiert hat.

\textbf{Sind ihre Klassenkameraden alleine ausgewandert oder mit ihren Familien?}

\textbf{Svetlana Bogojavlenska:} Mit den Familien zusammen. Es gab eine Welle an Auswanderung in Lettland, das hat wahrscheinlich 91/92 angefangen. In meiner Erinnerung ist geblieben, dass Berge guter Bücher irgendwo in den Höfen lage oder einfach in die Schule gebracht worden sind von den Familien, die auswandern. Das war dann um 94/95.

\textbf{Wurde die Themen Judentum, jüdische Religion und jüdische Geschichte abgesehen davon in der Schule behandelt?}

\textbf{Svetlana Bogojavlenska:} Überhaupt nicht, nein. Darüber haben wir nie geredet. Ich glaube mich erinnern zu können, dass im Geschichtsunterricht, als es um den Zweiten Weltkrieg ging, die Judenvernichtung erwähnt wurde, aber nur erwähnt. Der Begriff Holocaust ist gar nicht erwähnt worden, er war völlig unbekannt. Dass Juden als ein Volk tatsächlich als Zielscheibe des Nationalsozialismus ausgewählt wurden und dann ausgelöscht wurden, wurde schon thematisiert, aber wie gesagt sehr kurz. Mit ein paar Sätzen wurde gesagt: Die Juden waren ein besonderes Ziel der Vernichtung, aber Hitler hatte vor, ganz Osteuropa auszulöschen - ungefähr in diesem Sinne.

\textbf{Gab es unter den Sowjets und vorher Antisemitismus in Russland? Welche Art von Antisemitismus gab es?}

\textbf{Svetlana Bogojavlenska:} Ja, eindeutig. Es gab stark ausgeprägten Antisemitismus, der auch nach dem Krieg Durchschlag hatte. Z.B. ging die Pogromwelle 1881 von Russland aus. 1880 hat es schon angefangen, aber 81 wurden die Juden der Ermordung des Zaren beschuldigt, obwohl es Anarchisten waren, die sich in der Wohnung eines Juden versammelt hatten, als sie diesen Anschlag auf den Zaren planten. Der jüdische Historiker Simon Dubnow, der 1941 in Riga bei einer der Massenerschießungen ums Leben kam, ging davon aus, dass diese Pogromwelle sogar vom Staat organisiert wurde. Das wurde durch spätere Forschungen widerlegt, was aber nur das Zeugnis dafür abgibt, das die Bevölkerung eher antisemitisch eingestellt war und aus verschiedenen Gründen auch selbst Pogrome verübt hat. Was die Regierung sicher nicht gemacht hat, war die Pogrome rechtzeitig zu stoppen. Die Gendarmerie hat immer ein paar Tage gewartet, bis da nichts mehr zu retten war.\\
Während der Revolution 1905 gab es auch Pogrome. Die gingen auch nicht von staatlicher Seite aus, aber sie beschränkten sich auf den Ansiedlungsrayon der Juden. Wo es eine sehr hohe Konzentration jüdischer Bevölkerung gab, kam es auch zu Pogromen. 1905 gab es nur ein Pogrom in Lettgallen, der dokumentiert ist, in Ludsen, heute Ludza. 

\textbf{Das sowjetische Regime wollte sich ja nach dem Krieg möglichst stark von der Naziideologie abgrenzen. Ist es dann nicht verwunderlich, dass der Antisemitismus nicht verschwunden ist? Wie passt das zusammen?}

\textbf{Svetlana Bogojavlenska:} Es gibt eine sehr bekannte Rede von Stalin, in der er sich bei allen Völkern der Sowjetunion für den Sieg im Großen Vaterländischen Krieg bedankt, aber besonders bei dem russischen Volk, das das größte Volk unter allen Völkern der Sowjetunion sei. Jeder, der das bezweifelte, galt als Feind, obwohl Stalin selbst kein Russe war, sondern Georgier. Dieser ganze Internationalismus beinhaltete immer die Unterstreichung der Rolle eines Volkes. Alle mussten zu sowjetischen Bürgern werden, nicht zu lettischen Bürgern der Sowjetunion. Es gab eine Tendenz zur Russifizierung und Sowjetisierung. Diejenigen, die versucht haben, ihre nationale Kultur in der Zeit des Stalinismus zu pflegen oder wiederzubeleben, wie es Juden versucht haben, waren zum Scheitern verurteilt. Da es für die jüdische Bevölkerung sehr wichtig war, nach dem Krieg die Reste zusammenzuhalten, haben sie versucht, das jüdische Theater in Riga wiederzueröffnen. Das wurde ihnen verboten. 
Das war noch Anfang der 50er vor Stalins Tod, nach der Zerschlagung des Jüdischen Antifaschistischen Komitees 1948.

\textbf{Gab es da schon Auswanderungswellen nach Israel?}

\textbf{Svetlana Bogojavlenska:} Ja, es gab zumindest Versuche. Auch aus Litauen emigrierten viele überlebende Juden. Die erste größere Auswanderungswelle nach Israel gab es in den 1970ern, als es vielen gelungen war, über Ungarn die Sowjetunion zu verlassen. Danach noch einmal in den 80ern. 

\textbf{Wie wurde die Auswanderung nach dem Zusammenbruch der Sowjetunion in Lettland wahrgenommen?}

\textbf{Svetlana Bogojavlenska:} Das Bewusstsein war da, dass Juden auswandern, aber auch Verwunderung. Was die Migration der jüdischen Bevölkerung in der Sowjetunion betrifft, muss dazu gesagt werden, dass die Kontakte zwischen den Juden und der russischsprachigen Bevölkerung in Sowjetrussland besser ausgeprägt waren, weil fast alle Juden, die in Sowjetlettland gelebt haben, nach dem Krieg Russisch gesprochen haben. Viele von ihnen sind nach dem Krieg nach Lettland eingewandert, aus der Ukraine oder aus Russland. Die hatten auch eigentlich keinen Bezug zum vorherigen Lettland. Den Holocaust haben knapp 1500 lettische Juden überlebt und nicht alle von ihnen sind nach Lettland zurückgekehrt. Diese 1500 sind mit denjenigen zusammengerechnet, die es geschafft haben, vor der deutschen Besatzung Lettland zu verlassen. Von denjenigen, die dort geblieben waren, sind vielleicht knapp 1000 am Leben geblieben.

\textbf{Wie haben sie den Holocaust überlebt? Wurden sie versteckt?}

\textbf{Svetlana Bogojavlenska:} Ja, einige haben sich versteckt. Es sind sogar bis zu 600 Fälle der Judenrettung in Lettland heute bekannt. Das heißt aber nicht, dass sie alle überlebt haben. Anfang der 1990er waren um die 200 Fälle bekannt, in denen tatsächlich Juden überlebt haben. Einige Fälle sind in den Akten der Polizei aufgezeichnet worden, d.h. dass jemand gefunden wurde, der einen Juden versteckt hat. Das bedeute für beide den Tod. Wie überlebt man den Holocaust? Schwierig zu sagen. Viele sind 1944 nach Auschwitz gekommen und haben dort die Befreiung erlebt, einigen ist die Flucht 1944 gelungen, als die sowjetische Armee sich näherte.

\textbf{Haben Sie ein Beispiel von einer Person, die überlebt hat?}

\textbf{Svetlana Bogojavlenska:} Herr Marģers Vestermanis ist die prominenteste überlebende Persönlichkeit in der jüdischen Gemeinde. Als ich dort noch gearbeitet habe, waren häufig seine Freunde da, also Leidensgenossen, mit denen er zusammen im Ghetto und im Konzentrationslager Kaiserwald war. Die haben sich auch gegenseitig Bestätigungen darüber ausgestellt, dass sie bezeugen können, sich im Ghetto oder im KZ begegnet zu sein. Es gibt keine Papiere, die das belegen können. Als die Wehrmacht sich aus Lettland zurückgezogen hat, wurden die Dokumente verbrannt. Deshalb war es schwierig nachzuvollziehen, wie viele da inhaftiert wurden. Die Familie von Herrn Vestermanis ist bei der Rumbula-Aktion in Riga 1941 umgekommen. Er hatte zwei Geschwister, einen älteren Bruder und eine Schwester. Sein Vater war Textilfabrikant in Riga. Er war ein sogenannter kurländischer Jude aus dem Westen Lettlands. Seine Muttersprache war Deutsch, und seine Mutter war russische Jüdin. Die erste Muttersprache von Herrn Verstermanis war Lettisch, weil er eine lettische Nanny hatte. Dann hat er Deutsch gelernt, er war im deutschen Kindergarten. Er meinte, wäre die Familie bei der sowjetischen Deportation von Juni 1941 deportiert worden, hätten sie eine Chance gehabt zu überleben. Er selbst hat sich als Elektriker im Ghetto ausgegeben. Er meinte auch, dass er wie durch ein Wunder überlebt hat, weil ihm das abgekauft wurde, obwohl er einmal das ganze Haus, an dem er arbeiten musste, lahmgelegt hat. Er hat sich im Gespräch mit mir auch gewundert, dass er damals deswegen nicht umgebracht worden war.

\textbf{Kommen wir zu der Veranstaltung am 16. März in Lettland, die jedes Jahr einen internationalen Aufschrei erregt. Was ist der Charakter dieser Veranstaltung und welche Interessen stehen dahinter?}

\textbf{Svetlana Bogojavlenska:} Das ist eine Veranstaltung der Veteranen der Lettischen Waffen-SS Legion, die 1943 gegründet wurde. Die lettische Waffen-SS hieß wie überall in Europa Freiwilligen Waffen-SS Division, was aber nicht stimmte. Freiwillig waren nur die Schutzmannschaftsbataillone, zu denen sich Letten 1941 tatsächlich freiwillig gemeldet haben. Diese wurden 1943 in die 15. Waffen-SS Division eingegliedert. Das waren auch diejenigen, die an der Judenvernichtung 1941 teilgenommen haben und an den "Sonderexpeditionen" und an der Vernichtung der Zivilbevölkerung in Weißrussland beteiligt waren. All die anderen wurden einberufen, sind aber auch zur Musterung erschienen. Das sage ich, weil uns aus dem Baltikum ein anderer Fall bekannt ist. Die Waffen-SS Legion wurde auch in Estland gegründet und in Litauen. In Estland ist es gelungen, in Lettland ist es gelungen, in Litauen nicht. Nur jeder Fünfte erschien dort zur Musterung. Diejenigen die erschienen, nur dürftig ausgestattet, haben sich dann geweigert, den Eid auf Hitler zu leisten. 1944 musste man einsehen, dass sie nutzlos sind. Deswegen gab es keine litauische Waffen-SS Legion. 

\textbf{Das heißt, Widerstand dagegen war möglich und wurde nicht so sehr geahndet?}

\textbf{Svetlana Bogojavlenska:} Zumindest nicht in Litauen, in Lettland schon mehr. Die ganze Familie konnte inhaftiert werden, man hat nach ihnen gesucht. In dem berüchtigten Film "`Lettische Legion"' von Uldis Neiburgs,\footnote{Latvian Legion, Filmstudio "`Deviņi"', 2000.} haben einige Einberufene erzählt: "`Es kam dieses Zettelchen, ich muss dort erscheinen an dem und dem Datum. Und dann fragt der Vater: "`Und was machst du jetzt? Gehst du hin oder gehst du in den Wald?"'. Was konnte man machen, man wusste ja, die Familie wird verfolgt, dann ging ich hin."' Herr Vestermanis erzählte, dass die Partisaneneinheit, die er im Wald getroffen hatte, als er während des Todesmarsches geflohen war, zum großen Teil aus den Deserteuren der lettischen Waffen-SS bestand. Es gab Leute, die sich gewehrt haben.\\ 
In Litauen sah es so aus, dass man sich massenhaft dagegen gewehrt hatte, in Lettland nicht. Sie haben sich dann tatsächlich zusammen mit der deutschen Wehrmacht an den Kämpfen an der Front beteiligt. Sie wurden aber nicht mehr zur Ermordung der Zivilbevölkerung eingesetzt, deshalb kann man tatsächlich sagen, dass die 1943 Einberufenen keine Verbrecher im Sinne des Nürnberger Tribunals sind. Es sind aber auch keine Helden. Sie haben den Eid auf Hitler geleistet, auch wenn viele danach erzählt haben: "`Ich hatte so ein ungutes Gefühl dabei"'.\\ 
Dieser 16. März wurde tatsächlich als ein offizieller Feiertag in den lettischen Kalender aufgenommen. Ein paar Jahre später, als es einen großen internationalen Aufschrei gab, wurde er auf Vorschlag der damaligen Präsidentin Vaira Vīķe-Freiberga wieder aus dem Kalender gestrichen. In Riga muss jede öffentliche Veranstaltung bei der Stadtverwaltung angemeldet sein. Jedes Jahr reichen die Veranstalter die Anmeldung ein, bekommen eine Absage und gehen gerichtlich dagegen vor. Das Gericht erlaubt es jedes Mal wieder. Das Schlimme ist, dass diese Veranstaltung auch Neonazis aus dem In- und Ausland mobilisiert, die antisemitisch eingestellt sind, . Auch Rechtsradikale aus der Ukraine, aus Ungarn, sind jetzt jedes Jahr wieder dabei.\\
Es gibt einen Legionärsfriedhof in Vidzeme, in Lestene. Da wurde auch ein Denkmal errichtet, aber es ist anders als in Deutschland, wo es mit Scham gesehen wird. In Lettland gelten die Waffen-SS-Veteranen tatsächlich als Helden, nicht als Opfer. Zum Beispiel sieht man in besagtem Film am Ende einen Mann, der in der sowjetischen Armee gekämpft gegen seinen eigenen Bruder hat, im kurländischen Kessel. Er war jünger, er wurde also später einberufen, als die sowjetische Armee nach Lettland einrückte und die Wehrmacht und die Legion sich nach Westen, Kurland zurückzogen. Sein Bruder ist im kurländischen Kessel gefallen. Er weiß, dass er vielleicht derjenige ist, der ihn erschossen hat. Er sagt da: "`Zu Sowjetzeit sind die die Verbrecher gewesen, jetzt sind wir die Verbrecher."' Das ist eine lettische Familie, die auf beiden Seiten kämpfen musste.\\
Das Interessante ist, als die ersten Märsche zum 16. März organisiert worden sind, wurde Herr Vestermanis, der auch Historiker ist, gefragt, was er dazu meint. Er war damals der Meinung, dass die keine Verbrecher sind, da sie einberufen wurden. Die Verbrecher wurden danach von der Sowjetmacht verurteilt und zur Rechenschaft gezogen. Das sind einfach Veteranen. Es ist unklug, dass sie jetzt da rausgehen. Sie durften das die ganze Sowjetzeit nicht, sie galten ja als Verbrecher, wurden aber mehrheitlich nicht dafür bestraft, dass sie in der Legion waren. Herr Vestermanis hat sie damals verteidigt.

\textbf{Er wird aber nicht die Sichtweise der Waffen-SS-Veteranen verteidigen?}

\textbf{Svetlana Bogojavlenska:} Nein, natürlich nicht. Er hat sie auch als Opfer des Krieges dargestellt. Man kann sie aber nicht mit den anderen Opfern vergleichen. Einmal war sogar der Verteidigungsminister bei deren Gedenkveranstaltung in Riga dabei. Seine Handlung wurde natürlich danach verurteilt von der Regierung. Die evangelisch-lutherische Kirche macht allerdings jedes Mal mit. Es gibt einen großen Gottesdienst im Dom. Die marschieren dann von der Domkirche durch die ganze Altstadt zum Freiheitsdenkmal. Früher war das nur auf dem Friedhof in Lestene. Jetzt hat es aber offizielle Züge durch die Beteiligung der lutherischen Kirche, die im Dom auch zum Beispiel am Unabhängigkeitstag Lettlands einen Gottesdienst feiert, also in derselben Kirche, mit demselben Pfarrer. Jedes Jahr sind am 16. März auch mehrere Parlamentsabgeordnete dabei.

\textbf{Gibt es lettische Politiker oder andere bedeutende lettische Figuren, die sich antisemitisch äußern?}

\textbf{Svetlana Bogojavlenska:} Ja, es gibt die Partei \textit{Tēvzemei un Brīvībai}, "`Für das Vaterland und Freiheit"' \footnote{2011 löste sich die Partei nach einem Wahlbündnis mit der als rechtsextrem eingestuften Partei \textit{"`Visu Latvijai"'} (deutsch: "`Alles für Lettland"') im Jahr 2010 in die Nationale Vereinigung "`Alles für Lettland"' – "`Für Vaterland und Freiheit/Lettische Nationale Unabhängigkeitsbewegung"' (lettisch \textit{Nacionālā apvienība "`Visu Latvijai!"' – "`Tēvzemei un Brīvībai/LNNK"'}) auf.}. In dieser Partei gibt es ultranationale Kräfte, die den extremen Nationalismus salonfähig machen wollen. Sie kommt immer ins Parlament hinein. Von denen kommt ab und zu was, aber nicht von den Abgeordneten in der Saeima. Die sind natürlich gegen alle, nicht nur gegen Juden.

\textbf{Gibt es zivilgesellschaftliche Initiativen gegen Antisemitismus und Rassismus in Lettland?}

\textbf{Svetlana Bogojavlenska:} Staatliche Maßnahmen gibt es jede Menge. Es gibt immer wieder europäische Projekte, es gibt immer wieder soziologische Untersuchungen, es gibt immer wieder Empfehlungen der Wissenschaft an die Politik, was man noch machen könnte, um die Gesellschaft zu konsolidieren und demokratischer zu gestalten. Ob das alles tatsächlich da ankommt, wo es ankommen sollte, ist schwer zu beurteilen. Es sieht nicht danach aus. Offen antisemitische Politiker werden auch nicht stark angeprangert. Die Kritik kommt dann immer nur von der Opposition und nicht von den etablierten lettischen Parteien.

\textbf{In welcher Weise werden jüdische Letten heute im Alltag mit Antisemitismus konfrontiert?}

\textbf{Svetlana Bogojavlenska:} Offen überhaupt nicht, aber sie zeigen sich sehr selten als Juden in der Öffentlichkeit. Ich habe meine jüdischen Freunde in Lettland gerade gefragt und die meinen, latenter Antisemitismus sei da. Man spürt, dass man anders behandelt wird. Es gibt eine gewisse Aggressivität im Verhalten, vor allem wenn es um den Beruf geht. Man wird da nicht so gerne gesehen und wird ausgegrenzt, wenn man als Jude erkannt wird. 

\textbf{Wie groß ist das Problembewusstsein? Wie sehr ist Antisemitismus im Bewusstsein der Leute in Lettland?}

\textbf{Svetlana Bogojavlenska:} Ich glaube, es ist nicht im Bewusstsein. Es gibt lettische Wissenschaftler, die sehr wohl wissen und verstehen, dass es dieses Problem gibt. Ich denke aber nicht, dass in der Mehrheit der Bevölkerung ein Bewusstsein dafür da ist. Wenn es darauf ankommt, sagt man so etwas wie hier in Deutschland: "`Man darf das wohl noch sagen..."', "`Ich bin kein Antisemit, aber..."' 

\textbf{Wissen Sie, ob jüdische Institutionen dem Antisemitismus ausgesetzt sind?}

\textbf{Svetlana Bogojavlenska:} Ich weiß, dass die Synagoge in Riga von der Polizei rund um die Uhr bewacht wird. Einmal wurde da eine Flasche mit Brennstoff reingeworfen. Die Wände der Synagoge wurden beschmiert. Das liegt aber schon mehr als 10 Jahre zurück. Man hat auf den Bildern der Überwachungskamera gesehen, dass das Jugendliche waren. Der damalige Rabbiner Natans Barkans meinte, die sollten einfach zu Hause von ihren Eltern dafür bestraft werden. Er glaubte nicht, dass das fundiert antisemitisch war. Dass die Synagoge als Zielscheibe diente, ist aber schon markant. Man könnte fragen, woher Jugendliche das haben. Dann bietet sich die Antwort an, dass die das aus dem Umfeld, aus der Familie, haben. Höchstwahrscheinlich haben sie etwas gehört und dann ist es in Taten übergegangen. Diese Jugendlichen sind inzwischen erwachsen. Was sie jetzt darüber denken, weiß man nicht.\\ 
Die Synagoge wird seitdem rund um die Uhr von der Polizei bewacht. Da steht immer ein Bus voller Polizisten. Die jüdische Gemeinde hatte früher ein sehr offenes Durchgangssystem. Man konnte da einfach so rein. Viele Ausländer haben das bewundert und uns im Museum gefragt: "`Haben Sie hier keine Angst, einem Angriff der Antisemiten zum Opfer zu fallen?"' Man hatte tatsächlich keine Angst. Inzwischen kommt man nur durch die eine Tür rein. Die zweite Tür ist geschlossen. Auch um ins Museum zu kommen, muss man sagen, wohin und zu wem man möchte.\\
Ab und zu sind es Letten, die dort arbeiten, nur leider sind das Ausnahmen. Die Letten werden dann immer gefragt: "`Was machst du da? Das ist doch ein jüdisches Museum."' Ich wurde auch immer gefragt: "`Bist du Jüdin?"', und musste sagen "`nein"', "`Und was machst du dann da?"' - "`Arbeiten?"'. Als die renovierte Synagoge wiedereröffnet wurde, wurde darüber in der lettischen Presse berichtet. Ich habe die Kommentare der Leser gelesen und Angst bekommen. Einer der harmlosesten Kommentare war: "`Müssen wir Letten tatsächlich davon in Kenntnis gesetzt werden?"'. Das ist diese Nicht-Bereitschaft, das als Teil der Kultur und Geschichte Lettlands anzuerkennen.\\
Was allerdings die Aufarbeitung des Holocaust betrifft, hat Lettland, zumindest die lettische Wissenschaft, große Fortschritte gemacht. Es ist fast alles aufgearbeitet worden, was aufgearbeitet werden konnte. Fast alle Archivbestände sind gesichtet und systematisiert worden. Das Okkupations-Museum hat sogar zusammen mit dem jüdischen Museum vor einigen Jahren, 2011, eine Ausstellung explizit zur Judenvernichtung gemacht namens "`Rumbula. 1941. Anatomie des Verbrechens"'\footnote{Die Ausstellung ist auf der Internetseite des Museums digital zugänglich: http://okupacijasmuzejs.lv/rumbula/en} .

\textbf{Wie sehr sind in den lettischen Lehrplänen die Themen Judentum und Diskriminierung gegen Juden verankert?}

\textbf{Svetlana Bogojavlenska:} Das Thema ist auf jeden Fall präsent. Es war sogar eine Lettin, Ieva Gundare, eine der ersten, die sich überlegt hat, wie man das den Letten überhaupt erklären kann, ohne den Nationalstolz zu verletzen, zu sagen, dass nicht das ganze lettische Volk bei der Judenvernichtung mitgemacht hat, sondern einige Letten. Sie arbeitete bis vor kurzem im Okkupationsmuseum. Dem jüdischen Museum war es sehr wichtig zu sagen, dass nicht das ganze Volk am Judenmord beteiligt war, sondern bestimmte Individuen, die namentlich bekannt sind. Man kann nicht das ganze Volk für den Judenmord verurteilen. Die anderen waren ja auch Opfer. Das war auch für Ieva Gundare und das Okkupationsmuseum sehr wichtig. Es gab bestimmte Menschen, die am Judenmord beteiligt waren, es gab aber auch Menschen, die den Juden geholfen haben. Es gab auch diejenigen, die sich das gleichgültig angeschaut haben. Oder mit Schaudern. Oder mit Staunen, "`was passiert jetzt?"', und sie wussten nicht, was sie tun könnten.\\ 
Frau Gundare hat Pädagogik studiert und war diejenige, die 2001, 2002 die ersten Arbeitsmaterialien für lettische Schulen in engen Beratungen mit dem Museum "`Juden in Lettland"' ausgearbeitet hat, um das Thema den Schulkindern nahe zu bringen. Sie hat danach im Okkupationsmuseum das gleiche für die sowjetischen Deportationen gemacht, indem sie auch aufgezeichnet hat, dass nicht nur Letten darunter gelitten haben, sondern auch andere ethnische Gruppen, die in Lettland zu dieser Zeit gewohnt haben.\\
Es gab schon die Tendenz zu sagen, dass es ein Genozid gegen das lettische Volk war. Das stimmt aber nicht, es war kein Genozid gegen das lettische Volk. Das waren Repressionen gegen alle Völker Lettlands, gegen die lettische Bevölkerung oder vielmehr gegen die Bevölkerung Lettlands. Wenn man lettisch sagt, denkt man in Lettland nämich nur an die Letten, nicht an die Bevölkerung Lettlands, die ja multiethnisch aufgestellt ist. Das war ihr dann auch gut gelungen.\\
Es gibt in der Schule in jedem Fall ein Programm, nur muss man bedenken, dass der Schulplan natürlich anders ist als in Deutschland. Während hier dem Nationalsozialismus viel Zeit gewidmet wird, ist es dort anders. Man hat dort andere wichtige Themen und es hängt sehr vom Lehrer ab. Ich habe auch schon Berichte gehört, dass die Lehrer das Thema einfach weglassen, sodass der Holocaust gar nicht erwähnt wird. An der Universität gehört es dazu. In der Schule muss es wie gesagt mindestens eine Stunde sein, aber die wird nicht von jedem Lehrer durchgeführt. Ich weiß aber, dass man an den Schulen Projektwochen hat, und erstaunlicherweise entscheiden sich immer noch ziemlich viele Schulen, auch aus der Provinz, für einen Gang ins jüdische Museum.
\end{otherlanguage}
