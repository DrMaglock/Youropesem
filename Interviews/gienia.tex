\section{Gienia}
\begin{otherlanguage}{polish}
\textit{Gienia was born in 1924 in Kołomyi, Ukraine. She will tell about her biography over the course of the interview. We met her in a retirement home in Ramat Gan in the outskirts of Tel Aviv on September 21st, 2016.}\par
\vspace*{2em}
\textbf{Proszę opowiedzieć o swoim dzieciństwie.}

\textbf{Gienia:} Chodziłem do szkoły w Kołomyi na Ukrainie. W 1939 roku zajechało do nas wojsko rosyjskie. To było święto Rosz Haszana - Nowy Rok. Wszystko rozrabowali... Przez trzy tygodnie wojsko rosyjskie wysłało całą młodzież do kopalni węgla w Rumunii. Tam było już pełno młodzieży, więc powiedzieli, że będziemy jeszcze jechać. Może tydzień, może dwa... Jechaliśmy sześć tygodni do Gruzji. Posłali nas na plantację herbaty, bardzo trudno było tam pracować, było bardzo gorąco. Dostaliśmy białe kapelusze i białe fartuchy. Wyjechałam stamtąd do miasta z mamusią i młodszą o dziewięć lat siostrą. Była tam wielka fabryka, w której wyrabiano cienką, złotą niteczkę - i tam pracowałam. Dostawałam cienki placek do jedzenia, później był obiad, więc było mi tam dobrze. Któregoś dnia biegł za mną młody Uzbek i krzyczał do mnie po rosyjsku: "`Dziewczynko! Jestem Żydem, nie bój się mnie!"'. Kiedy przyszłam do domu, opowiedziałam o tym mamusi. Wieczorem wychodziłyśmy z dziewczynkami spacerować. Któregoś dnia jednej z nas brakowało, a była to córka rabina. Wróciłam z płaczem do mamusi, wszystkie dziewczynki razem ze mną. Na trzeci dzień przyprowadzono dziewczynkę w to samo miejsce. Mamusia bez żadnych pytań wzięła ją na ręce i przekonała ją, że wszystko w życiu może nas spotkać i po prostu niech o tym zapomni. Postanowiłyśmy, że trzeba stamtąd wyjechać. Popłynęliśmy - ja, siostra, mamusia, dziewczynki i chłopcy - na Ukrainę okrętem. Każdy miał przy sobie parę groszy. Na noc zostaliśmy w porcie, tam nas obrabowano, nie mieliśmy nic, żeby móc pojechać dalej. Wtedy podeszła do nas kobieta, która spytała: "`Rozumiecie po rusku?"' Odpowiedzieliśmy, że tak. Powiedziałam jej, że ukradziono nam pieniądze. Kobieta wzięła nas do siebie i dała coś do jedzenia, chłopcom pościeliła na podłodze, a dziewczynkom, gdzie tylko się dało. Następnego dnia zjedliśmy śniadanie i dostaliśmy od niej pieniądze na bilet, żeby dojechać pociągiem do Perwomajska. Podziękowaliśmy jej, ucałowaliśmy ją za to, co dla nas zrobiła i odjechaliśmy. Stamtąd poszliśmy zameldować się jako uchodźcy. Dostaliśmy pracę, a po pracy siedzieliśmy na stacji kolejowej i czekaliśmy, aż przyjedzie moja mama i siostra. W tym samym miesiącu wybuchła wojna. Kupiliśmy konia, mieliśmy też furkę. Była z nami kobieta w ciąży z mężem, moja mamusia, ja i siostra. Zaczęliśmy iść po górach, żeby gdziekolwiek dojść, a silne deszcze jeszcze bardziej nam to utrudniały. Koń już ledwo ciągnął, gdy doszliśmy do jakiejś wsi. Przy szosie zauważyła nas starsza kobieta, przyniosła nam cały bochenek chleba. Powiedziała, że na tej wsi już nikogo nie ma. Z nieba zaczęły spadać bomby, a my szliśmy dalej. Ktoś krzyknął: "`Stój! Będę strzelać!"' Stanęliśmy, a on wziął nas do lasu. Tam był sztab. Dali nam chleb. Nad ranem przywieźli bańki z mlekiem. Jechaliśmy koleją, którą wieziono czołgi i armaty. Dojechaliśmy do jakiegoś miejsca, wtedy nagle podpalili całe pole pszenicy. Ktoś wyskoczył, komuś nogi odcięto, a kto miał szczęście, ten został przy życiu... Znowu się zatrzymaliśmy, maszynista powiedział, że musimy zejść, ale niedaleko jest wioska i możemy się tam zatrzymać. Ciężko pracowałam w polu przez cały dzień, a w nocy spałam na ziemi ze wszystkimi. Była tam babuszka, która dawała nam po kawałeczku chleba - zawsze się modlę o to, żeby była w niebie za to, że niejeden raz uratowała nas od głodu. Przyszedł jej zięć i spytał: "`Mamo, gdzie twoi Żydzi?"' Ona odpowiedziała, że u nich nie ma żadnych Żydów. Zięć z nożem w ręku zagroził, że nam wszystkim odetnie głowy. Babuszka wepchnęła go do pokoju, dała mocną kawę i powiedziała, że jak się obudzi, przyniesie mu wódkę. Gdy tylko wyszła z jego pokoju, dała nam po kawałku chleba, żebyśmy mieli siłę uciekać. Furmanką jechała kobieta z mężem, zatrzymali się i pojechaliśmy z nimi. Tam też już spadały bomby... Zdążyliśmy stamtąd uciec. Nie mieliśmy ze sobą żadnego bagażu, tylko dzbanek na wodę, który niosła moja siostra. Znowu jechaliśmy pociągiem. Mamusia bała się, że mnie stamtąd wyrzucą, kiedy pociąg się zatrzymał. Wtedy klęknęła z jedną nogą pod kołem. Chciałam do niej skoczyć, ale mnie złapali i zaczęli mnie przeklinać. Pociąg znowu się zatrzymał, chciałam biec, ale mi nie pozwolili. Słyszałam tylko z daleka: "`Gieniusiu! Gieniusiu!"' Mamusia była cała w krwi, przynieśli ją i kazali mi ją nieść. Zawieźliśmy ją do szpitala i tam już została...  Potem zachorowałam na tyfus. Przeleżałam miesiąc, nawet nie wiedząc, że jestem chora na tyfus. Ten na górze chciał, żebym została i mogła żyć. Po miesiącu byłam już tak wychudzona, że prawie nie miałam piersi. Trzeba było iść do szpitala, cały dzień pieszo. Doszłyśmy z siostrą pod wieczór, z jedną pałką w dłoni i dzbanuszkiem, który był jedynym naszym bogactwem. W szpitalu napisałam karteczkę: "`Mamusiu, jeśli ty żyjesz, my, Gienia i Basia jesteśmy na dole"'. Nie pozwalali nam wejść do szpitala, nawet nie mogłyśmy przejść przez bramę. Mamusia skakała na jednej nodze o kulach, nawet przyniosła nam za cały miesiąc trzydzieści kawałeczków chleba, których sama nie jadła. Kobieta, która leżała obok, pytała mamusię, czemu ona to wszystko odkłada.  Mamusia odpowiedziała: "`Ja nie mogę jeść. Mam dwoje dzieci, może przyjdą"'.
Ciężko pracowałam w kołchozie na wsi. Nawadniałam owoce. Do dzisiaj cierpię przez to na nogi, ledwo chodzę. Mamusia po dwóch miesiącach wróciła o kulach i od razu zachorowała na tyfus. Nocami dyżurowałam tam, gdzie rosną winogrona. Były takie malutkie i bardzo kwaśne, ale byłam bardzo głodna... Zachorowałam na dyzenterię. Nigdzie nie było żadnych leków, nawet ranni w szpitalach ich nie dostawali! Moja mamusia bardzo ciężko zachorowała. Dostałam wóz, żeby zawiózł nas czterdzieści kilometrów do szpitala. Przeleżała tam pięć dni. Codziennie z rana, jeszcze przed pójściem do pracy, chodziłam do niej. "`Mamusiu, dzień dobry!"' Piątego dnia nikt mi nie odpowiedział...
Co piątek zapalam znicze i proszę Boga, żeby ta babuszka, która uchroniła nas od ścięcia głów, była w niebie. Była jeszcze jedna kobieta, która uratowała mnie, gdy zachorowałam na dżumę. Dostałam pięć koni, żebym je czyściła i karmiła. Jeden z nich był tak spragniony, że chciał mi zabrać wodę i uderzył mnie, a ona wyskoczyła i mnie uratowała. Pracowałam z końmi i tak przeżyłam tam pięć lat. Gdy już można było wyjechać w 1945 roku, to ja nie mogłam wyjechać, nie miałam nic. Kim ja jestem? Gdzie ja jestem? Skąd ja jestem?

\textbf{A czy ma Pani jeszcze jakieś wspomnienia, jeśli chodzi o Polaków? Dobre, złe? Jacy byli wobec Pani?}

\textbf{Gienia:} W klasie były różne dzieci - Żydówki i Polki, i ja byłam...

\textbf{A czy to dobre wspomnienia?}

\textbf{Gienia:} Tak, to dobre wspomnienia. Byli Polacy bardzo dobrzy. Byli też Polacy, którzy mówili: "`Precz z Żydami, Żydówki z nami"'... Chodziłam z koleżanką, Polką do kościoła. Powiedziałam jej tak: "`Powiedz mi, kiedy trzeba klęczeć, powiedz, kiedy trzeba się przeżegnać"'. I lubiłam z nią tam chodzić. Są Żydzi dobrzy, są Żydzi nachalni, są Żydzi-chamy i są Żydzi nadzwyczajni… Tak samo jest z każdym - nieważne, czy z Polski, Rumunii, Maroka, Rosji, Ameryki... Skąd by się nie było, jest się człowiekiem. Bolało mnie, że na sklepie było napisane: "`Precz z Żydami, Żydówki z nami"'. Bolało mnie, gdy byłam w parku, a tam grupa chłopaków machała scyzorykami w moją stronę i mówili, że zrobią mi krzywdę.

\textbf{A była Pani w Polsce niedawno?}

\textbf{Gienia:} Nie, nie byłam w Polsce. Nie byłam w Polsce dlatego, że Eichmann pracował w Polsce. Trzydzieści tysięcy ludzi kopało jamę i przypadkowo jeden z tych trzydziestu tysięcy został zakopany żywcem. Jego córka była moją koleżanką, siedziałyśmy obok siebie w szkole. Udało mu się odkopać i uciec do lasu. Był głównym świadkiem w procesie Eichmanna. Znalazł mnie, kiedy byłam tutaj, w Izraelu . Zauważył, że jestem podobna do mamusi. Wtedy powiedział: "`Ty jesteś Tratner! Ja twoją mamę znam"'. Opowiedział mi, że widział ciało mojego wujka, wpadającego do grobu. Była tam cała moja rodzina...
\end{otherlanguage}