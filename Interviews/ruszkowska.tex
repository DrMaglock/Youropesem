\section{Dr Sonia Ruszkowska}

\textit{Sonia Ruszkowska is an educator at the POLIN Museum of the History of Polish Jews since 2013. She has participated in the educational program of the main exhibition from the beginning. Mrs. Ruszkowska is responsible for the educational programme of the museum directed at schools. She is specialised in drama theatre workshops and anti-discriminatory education with a special focus on civil rights and the fight against anti-Semitism. Her educational programmes are mainly aimed at children, youth, and teachers. She studied philosophy and wrote her doctoral thesis about the Holocaust, the death in the gas chambers, and the possibility of bringing back subjectivity for victims of the mass deaths. Before she started her work at the museum, she worked as a schoolteacher for philosophy and ethics and participated in the organization ``Forum for Dialogue'', which conducts projects with school children in different towns and cities in Poland, raising awareness about their Jewish history. We met her at POLIN Museum on January 29th, 2018.}\par  
\vspace*{2em}
\textbf{You told us about ``Forum for Dialogue''. Is it only devoted to small towns and villages in Poland or also to bigger cities? As an educator, do you see a greater need of educating in smaller towns?} 

\textbf{Sonia Ruszkowska:} There are programmes of ``Forum for Dialogue'' in bigger cities like Białystok or Warsaw, but generally the project is run in smaller towns. I think the need is the same in smaller and bigger cities, but in bigger cities there are more educational programmes available for students.\\
These kinds of programme are very necessary because kids do not know the Jewish history of their towns at all, they have no idea that Jews were living there. If they know, they don't know what it means because they know nothing about Jewish culture. But they are very interested. If we go there and start talking about the topic, they are really.\\
In their homes, many times they encounter very anti-Semitic perspectives, so it’s harder for them to really go beyond that, if they never meet somebody for whom Jews are positively connoted. 

\textbf{Where do these anti-Semitic attitudes come? Are the parents anti-Semitic because their parents were anti-Semitic?} 

\textbf{Sonia Ruszkowska:} That’s a big topic. We know from history that anti-Semitism comes for example from the Christian-Jewish relationship, but also from economic rivalry. We show that in our exhibition. From an individual perspective, I meet mostly pupils and they say, for example: ``My grandma heard that there’s a Jewish workshop in the museum and she told me not to bring my wallet.'' This kid was laughing, like, ``oh, what is grandma saying'', so this is already okay, because he trusted us, assuming that we are Jewish if we make the Jewish workshops. This is not the case for me, for example, but there is this assumption. They hear it at home, from their grandparents and parents. Very small children, who do not really think in a conscious way, have very bad assumptions about the word ``Jew''. At the beginning of the workshop in the Museum, I normally ask them: ``What does it mean that somebody is a Jew, what does this word mean?''. Many times they said: ``Oh, a Jew is a person who does not want to share with others''.\\ 
In Polish, it’s ``skąpiec'', ``greedy''. I always ask myself: ``How does this six-year-old child know these stereotypes already?''. This is in the language because the Polish verb ``żydzić'', which comes from ``Żyd'', ``Jew'', and means ``being stingy''. So it’s basically the language that gives them these stereotypes, but I also have the impression that generally, in Polish society, there is a very bad association with Jewish culture, with Jews. Of course, there are always people who are fascinated by Jewish culture. 

\textbf{How we can fight these stereotypes in your opinion?}

\textbf{Sonia Ruszkowska:} In my workshops, I use this method called Nonviolent Communication (NVC). I try to give empathy to a person and not to judge this person, like: ``You’re stupid, you’re anti-Semitic, how can you say these things?'' I try to understand why this person is saying this. Mostly, people want to feel safe in the end. If they don’t know something, they have this association of danger, for example, that their values are in a way endangered. So I try to understand this person, what this person is saying, and then convert it into a language of needs ― what does this person need? Then, when this person really feels that I’m listening to him or her and is calm, I can say how it is for me. I can say that I know many Jews, many of them are my friends and I would like people to respect them. Of course, I can also say that some Jews can do negative things as everybody else, but generally, for me the respect is important. I can say that I wrote a book about the Holocaust and know many testimonies, so it’s really important for me not to make jokes about the Holocaust. So, I try to make it personal.\\ 
Another thing is that people need knowledge because they don’t know many things and have only some images. For example, this topic that is always coming back that Poles were so great for Jews during the war and how Jews could say that Poles were murdering Jews. If people do not know that before the war there was a lot of discrimination against Jews in Poland and a lot of tension, they will never understand why Poles reacted the way they did during the war, why these tensions were even greater because of the occupation. If people learn things and understand things, they are more open.\\  
Another thing: I believe in meetings. If people have only images and they have never seen any Jewish person, they have really only an image. So, Israeli-Polish meetings for example, even if they last only for one hour, can really change something. What we do here is that we want to give a positive association with Jews and the Jewish culture. Not like ``Jews means only Holocaust and Antisemitism – everything which is negative or difficult'', but also to show life, to show joy in Jewish culture, tradition and all these positive things. We want to show both. 

\textbf{Many interview partners have said that with the internet, there is much more anti-Semitism and hate speech and it’s spreading. What do you think normal people could do against that? Writing back comments is usually not that helpful.} 

\textbf{Sonia Ruszkowska:} I invented an anti-discriminatory workshop for youth, which is about hate speech, for example on Facebook. It is a drama workshop, so there is a hero that is the target of hate speech and others who use hate speech. People have to play both parts and then we discuss and try to invent a solution: what they could do in this situation. I believe in transferring the online situation to a real situation between people because this way I feel that we can do more. In both cases the conclusion was that it’s always good to search the people who can support you, who can help you. The person who is using hate speech doesn't have the whole power, it’s mostly like one or two people and there’s a lot of people in the class or in the group that are bystanders. If you can bring them to your side or ask them for help, in many cases they will join you.  

\textbf{How common are anti-Semitic stereotypes in Poland?} 

\textbf{Sonia Ruszkowska:} I think everybody knows them. Every little child knows this anti-Semitic perspective, somehow. I’m not saying that children consciously think like this. Not everybody says that they are true of course, but the knowledge is common. I think in smaller towns, many people really think this way, also because of the trauma after the war. In many little towns, people live in ``post-Jewish houses''. It is too hard for them to really confront themselves with this story, to really work with it. It’s easier to forget about this, but this conflict stays. It always makes you frustrated, so anti-Semitism is maybe the way to express this frustration against Jews.\\
I think anti-Semitic stereotypes exist, but it’s not necessarily the case that when Jews are in the public space in Poland, there would be some act of violence against them. It is more about saying to other people that Jews are for example greedy. In big cities, there are some small communities that are fascinated by Jewish culture. There are people coming to our museum for every festival, every concert, or event. Many of them are well-educated or interested in culture in a general way. There are also people who just don’t care about Jews or Jewish culture. 

\textbf{You said that pupils do not know much about Jews. Is that part of the compulsory education?} 

\textbf{Sonia Ruszkowska:} You can teach something in school, but children do not know everything that’s in their books. It’s only a small part of the curriculum and it depends on the teacher. Of course, there are teachers who are interested in the Jewish history. It is possible to talk a lot about this history during lessons, but if you don’t want to, you can do only one lesson about this. Then there is the Holocaust. The Holocaust is really present, but you can also teach it in different ways: is it about this Polish heroism or is it more about the Holocaust? Mostly, I think, Jewish history in the school programme is connected to war. It creates such an image: there are no Jewish people, then we see, out of the blue, there are many of them, and then they are murdered. That is basically the image that children have after school education. Now there is also a lot of Islamophobia in Poland. These two things are combined If I ask children about Jews, they often say that the book of the Jews is the Quran and that Jews are Muslims. Really, it’s very common now. The Jew is the alien, now the Muslim is an alien, so Jews are Muslims.  

\textbf{During the Holocaust, there were Polish people who were helping Jews to survive. You can read a lot about that and there are whole exhibitions in the museums about that. But there were also collaborators, Polish collaborators. Is this also well-known? Is it easy to do research on that topic?} 

\textbf{Sonia Ruszkowska:} It’s a very interesting story because there are many scientific books about it. If you want to know about it, it’s no problem. But no, people don’t know about this. Students don’t know about it at all. We have the workshop about Jewish-Polish relationships during the war, I made a film with material from the USC Shoah Foundation Archives, and there are people talking about their deportation to ghettos. They told about the reaction of Poles during these deportations, when they were entering the ghetto. They said that people were laughing, people were doing bad things to them, they were cruel.\\ 
We show this short film to students and they are always shocked. They had no idea. Of course, it’s not that every Pole during the war was doing such things. There were many Poles that were sad and full of compassion during the deportations of Jews to the ghettos. But there were many people reacting with cruelty.\\ 
In fact, I think I realized that only during my studies. Nobody ever told me about this in my whole school education. Personally, I remember this moment of realization very well and I felt that they lied to me all my life. I finished school in Warsaw, humanistic classes, and nobody ever told me about this. Especially in primary school, it was always ``Poles ― the heroes, great''. Then I discovered it’s not true, it’s a lot more complicated. This positive image of Polish people is everywhere and the truth is not so beautiful. It’s kind of hidden in the consciousness of society, but the knowledge is really there, if you want to see it. 

\textbf{Do you think the government wants everybody to think that the Poles have helped the Jews, but no Poles have helped the Nazis kill the Jews? Is it difficult to do research in that direction? Does the government control what is shown in the museums?} 

\textbf{Sonia Ruszkowska:} Yes, I think so. I think there is such a danger because people who do research need grants and the government has influence on who will receive these grants. There is always this question how much the researchers will be dependent, which university will get more money, which less, and so on. It’s never really objective, there’s always some interest in there. 

\textbf{Why do you think normal people don’t want to admit that Poles didn’t help enough?} 

\textbf{Sonia Ruszkowska:} It’s not nice to see yourself like this. 

\textbf{But why do you see yourself immediately as one of them? It’s just a fact, nothing personal.} 

\textbf{Sonia Ruszkowska:} It is personal because they identify with a nation. You want to have the good image of your nation. So many people say: ``It's not true that Poles were not helping enough, only Jews say this.''\\
There is a good book about this by Grzegorz Niziołek, ``Polski Teatr Zagłady''\footnote{``The Polish Theatre of the Holocaust''}. It’s about the theatre spectacles about the war that were made just after the war. He writes from a psychoanalytical point of view. He claims that the bystander position is very difficult because it’s somebody who sees what is happening, who feels guilty that he or she didn’t do anything, and who is afraid because he thinks, ``I can be the next one''. He's also happy that it’s not him who was killed and then he feels guilty because of this joy. There are many emotional layers. It’s a very difficult position. There is a big, big guilt. The situation is not black and white. With Jews and Germans, it’s somehow black and white. Germans started the war, they were murderers and Jews were victims. Poles are kind of in between. They are victims, but they are also murderers and they are bystanders.\\ 
The second thing is that many Jews have also no true idea about the Polish history of the war. Many of them never heard about the Warsaw uprising, for example, and about the fact that Poles were actually fighting against Germans, that so many Poles – non-Jewish Poles – were killed. These two narrations about the war (Polish and Jewish) are so different and neither of them is entirely true. I think the truth is very complex, it’s not at all black and white. Poles want to see themselves as total heroes, totally purified people who only helped Jews. Jews often think that Poles didn’t help them, that they were perpetrators. I think the truth is somewhere in between, but somehow people prefer to have a clear image. Why do people simplify the truth? That’s a philosophical question, it’s very interesting. Psychologists say that ambivalence is the most difficult feeling or state of mind. We prefer clarity. 

\textbf{I can say from the perspective of a teacher that in Polish education, there is still not enough time to teach about everything in detail. If you've only got several hours to devote to the Holocaust, you would prefer to talk about ``good Poles'', not about ``bad ones''. Probably, that’s why students are not well educated about the bad sides of Poles. Do you agree that it might get uncomfortable for teachers because of the lack of time and the comfort?} 

\textbf{Sonia Ruszkowska:} I don’t really agree, it’s an excuse. I think it’s just a difficult topic. Teachers maybe do not feel really prepared to teach about it because they would have to really think about it personally, make peace with it in themselves. It took me many years to really dig in and to really feel acceptance with this topic. It’s adifficult topic and teachers also need education about it. In the museum we make many workshops for teachers about the Holocaust. If you have only one hour to teach about the Holocaust, maybe you could devote half an hour to tell about Polish heroism and half an hour to tell about Poles murdering Jews. 

\textbf{As you said, it’s a good excuse. I wouldn’t like to do it. I would avoid being accused by parents of my students that I said something about bad Polish people during the war.} 

\textbf{Sonia Ruszkowska:} Yes, of course. We in the museum are in a privileged position in this respect because we don’t have contact with children’s parents. Sometimes teachers also say that we show too much of Poles’ ``bad side''. So, we always try to say that it’s only one perspective and that there is also different one. But I think that children will hear about this positive side of Poles anyway, about the Righteous among the Nations, it’s a hundred percent certain. I would prefer to say other things which they might not know about. 

\textbf{What would you say about the development of anti-Semitism? Does is get less because people become more educated or does it get stronger because the Holocaust happened seventy years ago?} 

\textbf{Sonia Ruszkowska:} I think there is a possibility that anti-Semitism will grow because if there is more Islamophobia and homophobia and all this kind of attitudes, it is always connected with anti-Semitism. I don’t know why, but it happens. I think it will grow. Besides, anti-Semitism is now mixed with anti-Israeli reactions. This is also very complex topic. I think the situation in Israel will not be very calm in the near future, I fear, so it will also not be the element that can help. I don’t see anything in the big scale that would really help. 
 
\textbf{I had a discussion with a roommate and he told me that he doesn’t like it that little kids learn about the Holocaust. He thinks that it will do some harm to them. At what age, do you think, can we tell a child about the Holocaust and how can you start telling these things?} 

\textbf{Sonia Ruszkowska:} In the museum, we start these workshops in fourth class, so the kids are around ten years old, and we do it with literature, not with images. I think it’s more about the question of how to teach it, not if we teach it. I also had an interesting discussion about this with friends from Israel and they said, ``we teach about the Holocaust very early, very little children.'' In Israel, there are a many institutionally organised days about the Holocaust during the school year, so these children will hear about it anyway. It’s only the question if they do it in an appropriate way and prepare them for it.\\ 
I think in Poland it’s a bit the same. If you walk around in Warsaw, for example, every few metres there’s a plaque saying something like, ``here, fifteen people were shot by Germans''. So you cannot really avoid the memory of war, it’s really everywhere. Children know it very early, probably earlier than in other countries, so I think it’s okay to teach them, but in a wise way, not with images of naked corpses or other horror things. Even if they say they want to see it. In fact, whenever we go to the interwar period gallery with small children, they say: ``Oh, we want to go to the war, we want to go to the war!'' We say, ``no, you will do it when you’re older'', and they react: ``No, we want now, we want to see it!'' It’s like a computer war game for them. Of course, they don’t understand what the Second World War really was.\\ 
I think it should be a process. First you talk about values and tell children about individual stories. Then there are more and more details. I think, for example, that you should only take people to Auschwitz at the end of high school. For me, you shouldn’t talk too early about gas chambers and the mass death. It’s so horrible that people should be prepared for it to really understand it.  