\section{Ilya Lensky}

\textit{The interview took place}.\par
\vspace*{2em}
\textbf{ As we couldn’t find anything about your biography online, could you give us a summary of you ended up being the leader of this museum first?}

\textbf{Ilya Lensky:} At rather young age, with six or seven, my parents brought me to the Jewish community, to all kinds of institutions. I was in summer camps, I was in a Jewish youth theatre which we used to have here, I’ve graduated at a Jewish school, I was involved in Jewish institutions my entire life. I’ve graduated at the history department in the University of Latvia, and it was quite logical then, when I was seeking a job, and after several unsuccessful experiments, that I ended up working in the museum as a historian, as also some very involved in Jewish community life. And then, in 2008, Mr Vestermanis offered me to take the position, because he himself was not very sure about his capacities, because of his health, to run the museum and to keep a hand on all everyday matters, and now he’s the curator of the museum, and he’s still a leading expert on the Holocaust, while I manage more the everyday life of the museum. That’s my personal story, and of course, it’s not available online because I don’t see anything special about it. I don’t have any scientific results that may get public. 

\textbf{What do you think, is anti-Semitism serious problem for the world?}

\textbf{Ilya Lensky:} No. Let’s put it like that, the Jewish community does not see anti-Semitism as a important challenge to its existence, which does not mean that there is no anti-Semitism. Of course, with anti-Semitism, we can’t say if it’s strong or not because usually, it won’t manifest openly, so even if quite big part of Latvian society is infested with anti-Semitism, it does not result in violent acts or anything like that. The main manifestation of anti-Semitism we have is in comments on the internet. It’s part of the broader problem of hate speech on the internet, which has to do not only with the Jews, but generally with all kinds of minority groups. Basically, in Latvia, the main minority that most of the hate speech is directed against is the Russian minority, most of hate speech on Latvian internet would be anti-Russian rather than anti-Semitic, but still, these comments on the internet are the main problem for the Jewish community in the field of anti-Semitism. Obviously, there are some prominent intellectuals who are anti-Semitic, some politicians, but we don’t see it as a main challenge, as I said, to our existence. 

\textbf{And do people treat Jews differently and in a specific way nowadays in Latvia?} 

\textbf{Ilya Lensky:} First of all, it's important that the majority of Latvian Jews are not different from surrounding population, visually or in any other way. When you encounter a person, usually in Latvia you do not know their ethnic background. And of course we have issues with, for example, non-white people being treated differently - this is a very strong problem, we have anti-black racism, for example - but not so much against the Jews, because the Jews look more or less like anybody else. Then, we have not heard of cases of people, for example, being dismissed from work because of their Jewish origin or something like that, which does not mean that it does not happen, but at least we haven't received reports about anything like. Actually, during the Soviet years, when there were rather strongly, although unofficially anti-Semitic policies of the, of the state, Latvia was one of the regions where the overall situation was better. It was easier for the Jews to enrol in university in Latvia than, let’s say, Russia or in Ukraine. This was one of the reasons why many people would come to Riga to study, because they could just not enrol in university in their native region. I can hardly say that the Jews will be treated differently on the everyday level. 

\textbf{Today, we talked to someone who said that those people who make anti-Russian statements also make anti-Semitic statements and vice versa. Is that correct?}

\textbf{Ilya Lensky:} More or less, although of course you can have ``pure anti-Semites'' and ``pure Russophobes''. We also have a quite significant group of people who are strongly anti-Russian, but constantly make philo-Semitic statements, which I do not believe, because to me, it more seems like they’re trying to whitewash their image by showing, ``you see, we are no hardcore xenophobes, we’re distinguished'', and they would also emphasise that there are good Russians, whom they do not hate, and so on. Generally, I don’t believe such statements. I think if a person is racist or xenophobic, then this resentment covers all possible groups, including LGBT, women, and so on. I would say that there are different groups, but also of course it’s true that if they are real xenophobes, they would hate both Russians and Jews. But we have to keep in mind that there is a rather strong anti-Semitic sentiment in Russian minority as well. 

\textbf{Those people you were referring to, what kind of philo-semitic statements do they make?} 

\textbf{Ilya Lensky:} They would always emphasise that the Jews are a traditional minority, a good minority, which never had any problems with the surrounding population, and that they are so sorry that they were exterminated during the Holocaust, something like that. Obviously, they are always great fans of Israel. To me, it has kind of an overtone of ``Jews, go to Israel. Leave Latvia and go to Israel.'' - that every nation should stay in their own country. Russians should pack their suitcases and go to Russia, Jews should pack their suitcases and go to Israel, and as Latvians, we want to live here in a racially pure state. It’s not stated explicitly. Maybe I’m wrong. But it is very much the feeling that I have when someone suddenly is very philo-Semitic and very pro-Israeli - to me it is always suspicious. 

\textbf{Is Jewish culture popular in Latvia? Are there a lot of places like this museum?}

\textbf{Ilya Lensky:} I would rather say no. It’s hard to say what popular means, of course, but if we compare it, for example, to what happens in Germany or what happens in Poland with all the festivals of Jewish culture being organised in different localities and involving mostly local, non-Jewish population, we have almost nothing like that in Latvia. Of course, Latvia can’t be compared in the sense of size and importance to Germany and Poland, I mean historically, because in Poland there was the dominant Jewish community and in Germany, the Jewish community was culturally dominant. Here, many municipalities would like to have something Jewish such as organising Jewish events, but usually they would invite us to organise them, they would say, ``we want to hold a Jewish culture day, could you bring us a choir or a dance company or something'', rather than organising it on site. I can’t say that we do much to change the situation. To me, of course, it is very important that the Jewish culture would be perceived just as one of many cultures, and that you don’t have to be sorry for your interest in Jewish culture - if you can have, say, Salsa contests and Salsa courses, which are part of Latin American culture, and if you can be fan of Indian cuisine, there is nothing to feel sorry for in being interested in Jewish culture. It just normal, just as liking French movies. We want Jewish culture to be treated like that, like something normal. Currently, I know only one place in Latvia where you have local, non-Jewish population in Jewish cultural activities: In Rēzekne, they have a Klezmer band. In other places, nothing like that happens. On the other hand, in several local history museums we have sections on the Jewish history of the respective locality, so the interest is there, but not in a way comparable to, as I said, Germany or Poland or Hungary. So I think we were only at the beginning of this process, but obviously, the situation has changed significantly compared to with what it was like 15, 20 years ago. 

\textbf{What about anti-Semitism under the Soviet dominance? Was it different?}

\textbf{Ilya Lensky:} As I said, there were several levels: On one hand, there was this state anti-Semitism, and on the other hand, there was the local context. Generally, in Latvia, as I said, anti-Semitism was weaker than in certain other regions of the Soviet Union, if we speak of such things as the possibility to enrol in the university - if you take, for example, Moscow, major universities there were just closed for the Jews in the 70s and 80s. It was clear that if you’re Jewish, you cannot enrol in these universities - you can try, but you won’t pass the exam. Especially in departments like maths, physics, and so on,it was considered that the Jews are dominating Soviet maths, Soviet physics, so we will not allow them to the universities. For many Soviet Jews, this was the first stimulus to think about their Jewish identity because Soviet authorities, as it turns out, were pressuring them to become Jews. They did not perceive themselves as Jewish. They wanted to be regular Soviet physicists, for example, but Soviet authorities said ``no, you’re a Jew and a Jew cannot be a Soviet physicist'', so they somehow had to formu thir attitude or Jewish identity. In Latvia, this was much weaker, you could enrol in the university, you could find a better job, but of course, for example, during late Stalinist period, in '48, '49, '50, when there was a big anti-Jewish campaign in the Soviet Union, Latvia was also part of this campaign, many prominent intellectuals were arrested and imprisoned, they spent several years in prisons or camps. Basically, all the possibilities to revive  Jewish culture after the Holocaust in Latvia were destroyed in 1949 and 1950. This was also a time when situation changed, for example, with regard to language: After that, most of the parents would be afraid to speak Yiddish to their children, to transmit Jewish language to them, they would rather switch either to Russian or to Latvian. This was during late Stalinist years; in the 70s and 80s, it was still better than in many other regions of the Soviet Union, but of course it’s an enormously big topic, and yet under-researched very much: Riga was one of the centres of Jewish national movement in the Soviet Union, so there were number of underground groups in Riga. We cannot call them resistance groups or dissident groups, it was different, and this was also very important: They emphasised that they don’t want to be called dissidents, because dissident is someone who tries to change the Soviet Union; for them, the main thing is either preserving Jewish culture or getting the possibility to emigrate from the Soviet Union, and the Jews were lucky to be one of the few groups that were allowed to emigrate, between 1970 and 1979, there was a window where it was allowed. About one third of Latvian Jewish community emigrated in these years and one additional third after 1989, when it became possible to emigrate again. 

\textbf{Do they still emigrate from Latvia?} 

\textbf{Ilya Lensky:} Hardly. Today, it’s again part of a broader problem of emigration from Latvia - to Ireland, to England, to Germany, Norway or wherever. I mean, since we became part of the European Union as there are open borders in the EU, Latvia lost about 15\% of its population in general due to emigration. Of course, the Jews are part of this movement, but it’s not particularly a Jewish movement, as it was in the 90s when the Jews emigrated mostly to Israel, but some of them also to Germany or United States or Canada or wherever. From 2001, I would say, there is very small specifically Jewish emigration, and those people who emigrate to Israel do it because of certain individual reasons: People go to study, or they move with their relatives, or they want to develop business, whatever, but these are personal stories rather than mass emigration, as it was in the 70s and in the 90s. Several years ago, the local representative of the Jewish agency, the main institution that works with emigrants, said, ``I know each and every Jew who emigrates from Latvia, both of them are great guys''. Maybe it was a little bit of an exaggeration, and it's not two people, but, let’s say, ten, but anyhow we’re speaking of very limited movement. 

\textbf{Is the Holocaust an important topic in the history lessons? I have the feeling that in Poland, we pay more attention to that. What is it like in Latvia?} 

\textbf{Ilya Lensky}: In Latvia, as far as I remember now, they have two lessons about Holocaust in grade 9 and two lessons in grade 12, so four lessons altogether as a part of the history course. But this is not the biggest problem because actually, we don’t have a reference point of how many lessons there should be. It's the same when I’m constantly asked why Latvia does not have ``normal'' policy of commemoration - what is normal? ``Like in Germany'', they say, what does that mean, like in Germany?  You can’t say what is proper and what is improper. It's the same with Holocaust lessons, we have a group of teachers, not very big, but I would say that 50, maybe 100 teachers in Latvia, who are really interested in the topic, most of them have been once, at least once, to Yad Vashem, to the special teacher training seminars they have. They come to our museum, they try to bring their pupils to our museum, so we have a group of teachers who teach Holocaust well, who are interested and who also maybe try to do some research on the topic. On another hand, of course, there are a lot of teachers who are not interested in the topic, and the biggest challenge for them is how to balance, in these four lessons, the general context and the local context because today, it seems to me that the main emphasis is on the general aspects of Holocaust, so that pupils will know about the \textit{Kristallnacht} and about Auschwitz, but they would not know what happened to the Jews in their native town. But this is a general problem with the school curriculum: Constantly there are discussions whether Latvian history should be integrated in the general history course or if it should be a separate subject.\\
Generally, Latvia is a very centralised country with a certain obsession with centralisation and with very sceptical and suspicious attitude towards local identities and local stories. And I think it’s problematic because of course, for Latvia, the history of the Holocaust is not about Auschwitz, it’s something very different. An absolute majority of Latvian Jews were killed here on spot, there are approximately 130 Holocaust killing sites in Latvia. Usually, only small group of people in the town would know something about it. Although again, the situation changes now, but still, I feel it’s a bit insufficient. 

\textbf{Is the subject present in arts, in literature and in films?} 

PROCEED HERE

\textbf{Ilya Lensky:} Yes, but of course not to the same extent as in Polish culture. Although, as far as I understand, now they have one movie ready [..., “The Mover”?, “Tēvs Nakts”] presents next year, about Holocaust, but it’s not only about Holocaust, it’s about story of Žanis Lipke, who rescued more than 55 Jews. In the literature, it actually was present all through during the Soviet times, starting with the first post-war years, 48, 49, and up till now, but I think the first work of literature that would be dedicated to the topic of the Holocaust appeared two years, three years ago, I think. The novel of Māris Bērziņš, “Svina garša”, the taste of lead, which is specifical dedicated to the Holocaust. Then, important thing was the memoir of Valentīna Freimane, who is very popular art historian and theatre historian, and most of Latvian theatre actors studied under her, and so she being a, one of the prominent figures in Latvian intellectual world, wrote a memoir about her childhood, well, she is Jewish, so she wrote a memoir about her growing up as a Jewish child and Jewish teenager in 20s and 30s in Latvia and then about her Holocaust experience, so she did, decided not to go to the ghetto when all Riga Jews were transferred, were ordered to resettle to the ghetto, so she went into hiding and she was hiding all through the occupation, so actually more than three years she spent in hidings in different places in Riga, and so she, she wrote a memoir about it. It was a bestseller and it, to me it seems that really for many people it was kind of mind-changing thing that wow, this prominent Latvian intellectual turns out, she is Jewish, and turns out you can be Latvian intellectual, you can be normal person and Jewish. Yeah, and not only Jewish, but Jewish with a kind of Holocaust experience. I think it for, for many weeks it was number one bestseller list of Latvian bookshops. Still, it’s rather popular. So, by the way now it’s being published in German as well or maybe already it is published, I don’t know. \\
So, in, in theatre, there were several theatre performances touching the topic of the Holocaust in one way or another. The same novel of Māris Bērziņš that I mentioned, “The Taste of Lead”, was also presented as a, as a theatre play in Latvian National Theatre, I think. Then, in so-called Riga New Theatre, which is main [...] modernist theatre, several years ago there was a play called “The Grandfather” about a man who searches for his grandfather and he encounters three men with the same name, who possibly can be his grandfather, and each of them had different fate. One of them served during the war, immediately after the war in Soviet army, and he in a very detailed manner tell [...] about the Holocaust in his native town. Yeah, and National, when theatre worked on the play, they quite closely cooperated with us, so they came to us, they researched on the topic of Holocaust in Viļaka, north-eastern Latvia, small town, with significant Jewish population. And so, so this was a, the story about Holocaust was historically very correct in the play. Yeah so, so that’s more or less of course, it’s not as prominent topic as it was in Polish culture, but again, I mean in Poland not, not immediately it became prominent. And also all the discussions now happening in Poland concerning the movie “Ida” and so on. That’s different story. 

Peter Zinke: How is the Latvian state dealing with the Latvian SS forces? 

Ilya Lensky: It’s complicated story. No, as you understand, there were all kind of different groups in which Latvians fought in the Nazi-occupied territory and [...], I mean, one story is about Waffen-SS, to which people were conscripted in 1943, so basically, we [...] that these people, who were drafted against their will to German army, after the Holocaust, basically when most of the Jewish were already killed, obviously in this Waffen-SS also were people who were involved in the Holocaust, but that’s different story. Much more problematic of course is, are the police battalions, so-called, and of course the auxiliary police units like Arājs Kommando, which was the main collaborators’ unit, which basically travelled from Riga to local small, small, small localities to murder the Jews there. On the official level, of course, they are not glorified in any way and, actually, when Latvia became independent, most of the legal cases of people who were accused during Soviet times and sentenced during the Soviet times on political basis, they were once again checked, reviewed and quite often people would be rehabilitated. Yeah, for example, many people who were drafted to these Waffen-SS, they would be during the Soviet times sentenced to five years, ten years, twenty years of imprisonment for betrayal, although of course, as Latvia today does not recognize Soviet Union and considers it as occupant force, obviously, you cannot betray the country of which you are not citizen and to which you do not serve, so obviously these accusations could be dropped. But if person was accused not only of these political, purely political, also for example, what we would call crimes against humanity, then they would not be rehabilitated, yeah, and as far as I understand, then he does not receive, for example, the status of “victim of political persecutions” and actually, what’s interesting, in Latvia today, the status of ”victim of political persecution” have not only people who were imprisoned by the Soviets, but also the Holocaust survivors, so they are also seen as victims of political persecution. So, in this sense I would say that Latvia deals quite, quite well, yeah, although we still have some problems with that, but at least on the official level, the government, every public figure speaking, at least at the commemorative events, considers it’s important to mention and to, to mention the participation of Latvians in the Holocaust and to say something bad about it, yeah, so, so, it’s, it’s I think important thing because I mean, 25 years ago, it would be a bit different. 

Peter Zinke: But for example, the National Historical Commission has no clear statement the 17th of, to the 16th of March. 

Ilya Lensky: No, 16th of March is absolute, it’s different story. So it was for two years, it was an official day, but in 2000 it was, it was dropped from the calendar, so today basically 16th of March is a, well, do you understand what happens on 16th of March? Today 16th of March is the private event, we can say, organized by the nationalist groups, usually with very limited participation of real World War II soldiers, well, because of different reasons, because of age: As you understand, these people are not 18 years anymore. But also I think, several years ago the main organization issued a statement that it advises its members to abstain from the participation in the event, yeah, so, so that was once to commemorate, it said, they should on 16th of March go to Lestene, where the big Brethren cemetery, and commemorate there, which is absolutely no problem, I think, in my opinion. So, Prime Ministers for several years now, they demand from the members of government not to attend the events, I think one of the ministers had to be dismissed because he wanted to come to the event several years ago. Some of the members of parliament you can still see, but it’s mostly a private event, and it’s as, one of my colleagues wrote an article about it where he described it as a political circus. So it’s, it’s very much an event to advertise certain political forces, but of course also so-called anti-fascists come, which are basically Russian-affiliated organisations and so on and for them, it’s also a possibility to advertise themselves. So, so the Jewish community [...], every year journalists want our comment about 16th of March, and every year we’ve said that we have no comments on that event because it has nothing to do with Jewish history, it has something to do with current political issues in Latvia and we’re abstaining from comments on current political situation. 

Johannes Probst: So, are there, are there any anti-Semitic statements on this, on this demonstration? 

Ilya Lensky: Sometimes, but usually not on the official level, not, you know, carrying it on posters, but in the crowd, but we ignore that. 
It’s, for us it’s more or less on the same level as anti-Semitic talks someone, you know, starts cursing the Jews in the, in the tram, yeah. 

Johannes Probst: And today we talked to a politician from the Harmony party, his name was Jānis Urbanovičs, and he said that some of the members of parliament participate in these events. 

Ilya Lensky: Yeah, so, it was governmental event in 1998 and 1999, but not now anymore, yeah, so, so this is, the government tries to distance itself. Again, how sincere are people, that’s a different story, but at least on the official level, which is important, they, they distance themselves from the event. In recent years also the level who participate in the demonstrations would be carrying posters denouncing both Nazi and Soviet occupation and so on and trying to present Legionnaires as freedom fighters or, again, that’s, that’s purely political showing off, but, but I mean it’s not an official event anymore, yeah, and, and they, they try to find new modes of organising this event because obviously, it’s not commemorative event anymore, as well, yeah, it’s obviously glorifying event now and obviously, it’s affiliated with certain political parties. 

Peter Zinke: But of course the government could find clearer words. 

Ilya Lensky: Then it would be different government. Different governments, but for example, I mean, Riga city council several times tried to ban this event, but then the court overruled this decision. And again, it makes sense because it’s non-violent event. They, officially, they do not glorify Nazi or Soviet regimes, so basically, there is no basis to ban the event, yeah, as we understand in a democratic society, even very unpleasant people have the right to demonstrate, yeah, and so, so I think that’s, that’s obviously the right thing and I prefer it happening like that, so on one hand, the Riga city council shows their attitude, another hand, the court does what the court has to do. Yeah, so to me, I do not see this event as problematic, for me in person.

Rafael Schütz: We read that there is always some kind of religious service in the Lutheran church after that, and my question would be: How are the relations between the Lutheran church and the Jewish community, and also between the other religious communities and the Jewish community.

Ilya Lensky: As, as I understand, this service in the colorbox{yellow}{Saint Peter Cathedral/Church} is mostly organised by one of the priests who is affiliated with this event, who often comes to all kind of similar events and so on. The, the relation between the Jewish community and most of other religious denominations are quite good, yeah, so, well, the Jews of course are minor religious group, so we have like three, four let’s say dominant religious groups that would be represented in all of the events, Lutherans, Catholics, Russian Orthodoxes and Baptists, so other smaller groups would usually not be represented officially, but of course their representatives would be, would be invited and so on. So we have quite, quite good relations. We don’t have any sort of commissions on cooperation and so on. There is organisation called Jewish-Christian Council in Latvia [look up] that’s not very active, I think, because of all kind of reasons, yeah, as [...] small country, you can’t devote 100 percent of your time to NGOs, you have to, to do something in NGOs or your, your hobby, usually. But generally, I would say that relations are quite good, yeah. 

Peter Zinke: There is no traditional Lutheran anti-Semitism? You know Luther was an anti-Semite? 

Ilya Lensky: Yeah, yeah, I understand, we, we don’t have that. So, there were attempts to make Lutheran church very strongly anti-Semitic during the 30s and then of course, during the Nazi occupation, it seems that the Lutheran church was quite, let’s say quite conform, towards the, towards the regime, different from Catholics, for example, because we have a number of Catholics, Catholic priests who rescued the Jews, and actually there were even, at least one priest, I mean, who was imprisoned because of a sermon he gave on occasion of murder of Jews in his native town, Aglona, it was Agluona, (latv. Aglona, latg. Aglyuna) where the main Catholic basilica of Latvia is located, yeah, so Aloizijs Broks (in Latgallian Alozis Broks), gave a sermon condemning murder of the Jews, so he was imprisoned and he perished, I think, in Neuengamme, if I’m correct. So, then, well of course, we have a lot, and I think Mr. Vestermanis will tell much more about it tomorrow when we meet him, Baptists and Seventh-Day Adventists and other religious minorities were quite actively involved in rescuing the Jews because of different reasons, yeah but so today, we don’t have Lutheran anti-Semitism, yeah, I think last time Luther’s book “The Jews and their lies” was published in 1943, I think, since, or 42, since then it was not reprinted, basically today, I think, Lutheran church in Latvia, although it’s very conservative, it’s one of the most conservative Lutheran churches in Europe, it has more serious issues than, than the Jews.

Peter Zinke: So they did not allow women to be priests? 

Ilya Lensky: Yeah, well, not just they do not allow, they banned women, because women could be priests until very recently, but very recently, they changed it. So that’s not just conservative, it becomes more conservative, but anti-Semitism is not part of their agenda. And generally, our relations are normal, and of course to all of our important events, we would invite the heads of all the religious denominations, not always the heads themselves can attend, but of course, for example the Catholic archbishop Stankevičs, he quite often comes to our events and so on. 

Rafael Schütz: Another question about the, the general population, if they had to estimate how many Jews live in Latvia, what would they estimate and how wrong would it be? 

Ilya Lensky: That’s, well, that’s an interesting question, actually we never, we never checked that, I know that in some other countries  there were, not even polls but yeah, we can call them polls like that, not in Latvia, I don’t know. I don’t know what would, what would people think, well what we know of course, many people think that Jews have too much influence on Latvian life, on Latvian economy, on Latvian politics and so on, but I don’t think that they have any estimate about the number of the Jews. 

Peter Zinke: So, in the United States, people are estimating that 30 until 40 percent of the citizens are Jews, and in fact there is just, just two percent [...] 

Ilya Lensky: Yeah, but it’s, it’s very much the same as what people estimate persons that are Muslims in Central European countries, so also they would estimate 25 percent, 30 percent, and it would turn out five, six, seven percent, yeah, but concerning, concerning Jews in Latvia, I just, I don’t know. It can be anything. 

Peter Zinke: In Germany, open anti-Semitism is of course taboo, officially at least, and so we don’t have too many anti-Semitic acts, but especially when Palestine is fighting against the Israelis or the other way around, there is some extreme criticism against Israel and according to, to my thinking it’s the expression of hidden or shame-faced anti-Semitism. Do you have it?

Ilya Lensky: It could be, it could be, so here we, let’s say we’re not so interested in, in Israeli-Palestinian conflict, I think when there was last war in Gaza, three years ago, four years ago probably, then the polls, people were asked during a poll, who do you support in this conflict, and about equal shares like seven and eight percent said that they support Palestine or Israel and 85 percent said that they don’t care or don’t support anybody. On another hand of course, so when during this war, there were two anti-Israeli demonstrations near Israeli embassy, which gathered combined only 16 people 
Yeah, and well, well, pro-Israeli demonstration which was organized by a group of, group of activists and also by some Israeli expats living in Latvia, they had about 150 people, yeah, but generally for example, as Latvia is very pro-American, it de facto is also very pro-Israeli, yeah, we have very good relations with Israel. In recent years, we have very good and very active Israeli ambassadors here, who try to outreach and who try to explain their position and who do not run around, you know, advertising Israel, as sometimes it happens to Israeli diplomats, no, they, they try to explain, and also the complexity of the conflict, yeah, so although obviously they are Israeli diplomats, they are on the Israeli side of the conflict, but they do not avoid problematic questions and so on, so I think it’s, it’s very important, so we, we almost don’t have this, this so-called new anti-Semitism masking as anti-Israeli sentiments. We hear if people are anti-Semitic, they can be openly anti-Semitic and for example, we also, what is important, yeah so as we understand quite often in this anti-Israeli demonstration,  people from emigrant communities, yeah so the same Palestinians and so on. In Latvia, we don’t have that problem with, for example, the Muslim communities being anti-Semitic or whatever, yeah so even people with emigrant background and Muslim communities, they, I don’t know if they have or don’t have this sentiment, but we don’t have that issue, for example, and so basically, the only hardcore anti-Israeli and also strongly anti-Semitic group in the Muslim community, for example, are recent converts to Islam, Latvians, Russians, actually one of the most prominent Latvian anti-Semites converted to Islam, and to me it seems that for him the motivation to convert to Islam was his anti-Semitism because he thought that Islam is the main anti-Semitic force of today, but then Muslim community just tried to get rid of him, yeah because we, as a, as a normal community, they want to deal with religious, cultural issues and so on, they don’t want to get involved in anti-Israeli demonstrations, anti-Semitic hate speech and so on, Holocaust denial and all those things, I mean, so, so, so for us it’s, it’s very good, and also of course speaking about immigrant issues, of course Jewish community is, well, I can’t say scared, but we watch with certain suspicion the anti-immigrant activity because most of the people who organise anti-immigrant demonstrations and so on, of course these are people with strong record of anti-Semitism, yeah and, so of course we see it as a danger, yeah so this, this anti-immigrant activism can also switch to anti-Semitic and broadly xenophobic activism. 

Rafael Schütz: This man you were talking about, what’s his name? 

Ilya Lensky: Roberts Klimovičs [Latvian name].\footnote{\colorbox{yellow}{possibly insert above}} Ahmed Roberts Klimovičs. And so I mean he’s, he’s a prominent figure, he’s rather active on, on the media, he is former journalist and so on. 

Rafael Schütz: And today the, the man we were talking to, he mentioned a book that was, well, about Semion Shustin, the main communist organiser of the deportations of Latvians, who was also like half-Jewish or something like that. It was a book that like, was, was again to do with this anti-Semitic statement that it’s like Jewish Bolshevism. 

Ilya Lensky: Generally of course, we have this issue, this myth of Jewish Bolshevism, and the same like in Poland with the story of Żydokomuna is [...] in Latvia of course, we, we have that. It’s, it’s a strong belief of part of Latvian society. We published a book, so actually we sponsored publishing a book about Jews and Soviet authorities, how they communicated and what was the attitude and so on. This was book written by group of Latvian historians, university professors and so on, so we did not pay them to write the book, yeah so they, they were employed for the project by the Institute of Sociology and Philosophy at Latvian University. The Jewish community just provided funds for the book to be published when it was ready, yeah, and so there I think was counted each and every Jew who was somehow involved in the communist activities, well maybe if someone was, I don’t know, a janitor in the Ministry of Interior, he was not mentioned in the book [laughter]. If, if he was at any position, then it would be mentioned. Needless to say that anti-Semites don’t read books like that.\\
Yeah, so there we see that actually involvement of the Jews in 1940, 1941 in the oppressive institutions or any kind of institutions that would making decisions was actually even lower than would be estimated, so we see that in only one organisation, the Jews were really active, that’s the communist youth organisation, where they was, yeah, quite high, but still not high percentage of the Jews, yeah, but we did not see Jews, for example, on leading position in party organisations and so on. And also one thing that we forget quite often when we speak about the role of the Jews in Latvian Communist Party, that, yeah, the Jews were significant part of Communist party of Latvia, but in 1940 when Soviets invaded, Communist Party of Latvia was about, I think, 400 people, of whom about 200 were in prisons. So this basically was not existent political force. That’s one thing and second thing that most of the people were involved in communist activities during the independence years, they were not appointed to any position during this first year of Soviet occupation. It is also something, something very important and this is one of the reasons why, for example, we, in our museum, you see, we have the story about 1940, 1941, the first year of Soviet occupation, and we also mention the involvement of Jews in the communist organisations, but we do not mention Semion Shustin, yeah, who’s the kind of main image in anti-Semitic, in Nazi anti-Semitic propaganda. The reason is very simple: He’s not Latvian Jew, he was just Soviet official appointed to this position. Before that, he was on a similar position somewhere in northern Russia in the Ministry of Interiors, yeah so, the same, they could send anybody else, yeah for example Estonia had I think, Russian or Soviet Estonian in the same position, yeah so basically, it was just accidental thing that Latvia got the Jewish Minister State Security, I’m sorry about that, oh yeah so, what was the name of the book? 

Rafael Schütz: Yeah so, I’m, I was wrong, it was not about him, but about Herberts Cukurs. 

Ilya Lensky: Ah, that’s a different story, that’s a different story. Yeah, then I know, the journalist is Baiba Šāberte, story of Herberts Cukurs is very, extremely complex, extremely complex, and [..] Herberts Cukurs, you know who Herberts Cukurs was? He was a famous Latvian pilot who in 1941 became the member of Arājs Kommando, and he was the head of transport section there and also the main arms officer, so he was responsible for distributing arms to the units going to the shootings and so on. Then in 1944, 45, he retreated with the German army to the west, and through France, I think, he emigrated to Brazil. He lived in Brazil under fake identity and so in, several times he was accused as a, as Nazi collaborator and war criminal. Brazilian governments, I think at least twice, refused to grant him Brazilian citizenship. There was big pressure and there were investigations of different international organisations, but never any formal trial against him. And so in 1965, he was murdered in Uruguay, allegedly by Mossad agents. So, basically, his story is not part of the story of the Holocaust. It’s very much a story of how we perceive the Holocaust, and what was the driving force for people to involve in the Holocaust, and there, there are a lot, a lot of legends, not legends, myths about him, misconceptions about him. What we know, what we see from the documents, that he was a member of Arājs Kommando, he was in quite high position there, he personally was involved at least several times in shootings. What he did not do is what is stated in many books that he was not leader of Arājs Kommando, he did not murder 30000 people and so on, yeah. So we, we tried to, to show that he was kind of, I don’t know how to, we, we, we formulated like a murderer or, on offer. So as it happened in many people who collaborated, yeah so they, basically, they did not have anti-Semitic background. They were offered by the Nazis to participate in killing of the Jews and they agreed, why not? If they would not be offered, they would not participate. Yeah so, Cukurs for example, one of the reasons, main reason he came to participate in Arājs Kommando was that during Soviet year, he was favored by new authorities as a pilot, as an aviation engineer. He went to Moscow to offer his help to construction companies there and so on and so forth. So when he returned, actually as, as far as I remember the story, someone even tried to arrest him for being a Soviet collaborator. So for him joining the Nazi collaborators unit was basically the way whitewash him, himself. And he knew Viktors Arājs from before, so he went to him for protection, basically. So, another important issue with Cukurs is that he, technically, he saved at least one Jewish girl, who was his lover, and he made her his lover already during the Holocaust. So, to me it’s problematic to call it rescue, but still, technically she survived because she was affiliated with him, and he really took her when he escaped, and he took her with him to Brazil. And we have her witness account that she gave in Brazil, where she mentions him as rescuer and so on and so forth. So it’s a very complicated story, and for, for many years now, the right tends to whitewash him and to make him kind of national hero, whom treacherous, murderous Jews have killed despite him being such semite. And there is not only book by [00007.MTS, 28:27, Latvian journalist’s name]. I think there is another book by Armands Puče, which is published about year ago. Armands Puče is sports journalist, so his knowledge and understanding of Second World War issues is not very professional, let’s say. Then there was a musical staged about Herberts Cukurs. Let’s say very bad musical [laughter], besides everything, it’s just bad as a musical, and so I’m, I’m very happy that the company that staged it, the producer that staged, went bankrupt [laughter]. So, and but, but what was problematic with this musical, because of this musical I started researching a little bit on Cukurs, is that in musical, many stories were told in a very distorted way, yeah, I mean, one story, if he personally murdered someone or not, this can be questioned, yeah. We can say yes, the witness accounts who mentioned him with a gun, okay, these are witness accounts, they can be critical. But definitely he was member of Arājs Kommando, we know that, but about if he personally murdered somebody, okay, we can be extra sceptical, if you want. But what we know, for example, there is [...] that he rescues a seven year old Jewish boy, whose father asks Cukurs to protect this boy, and first this father, when the members of Arājs Kommando come to arrest him, the father tries to give them money for, for them to leave him alone and so on and when it turns out that anyhow they are arresting him, so then he asks Cukurs to protect this boy. First of all, we, we know the name of the boy, the boy died three years ago or five years ago and 12 I think. He was not a boy. Yeah, in musical, seven year old boy, but real it was 17 year or 18 year. Second thing, of course, his father did not offer money, but we know from [...] of the boy that the father showed his ID card of independence war veteran. He was independence war veteran, he presumed that as someone who has fought for Latvia's independence, he would not be arrested, yeah, so again you see there was this very strong anti-Semitic myth of the Jewish money and Jews being in a, trying to, to sneak out with financial means and so on. Then, of course, Cukurs did not rescue the 17 year old boy, actually what happened, he did not kill him. We, and, and, and, that’s reflected actually both in the memoirs of this boy but also another witness accounts, so Cukurs killed some, according, if I remember correctly this memoir [...], he killed someone else, but he decided not to kill this boy, who worked in the, in Arājs Kommando headquarters. Hardly this can be called a rescue. So, so, so, the musical is also very problematic from this point of view, so this pressured us to write an article about it and so on. But let's say, I was quite surprised, there were about, I think, 300 comments on the internet, under this article. I would say that about quarter of them were positive, so yes, 75\% were anti-Semitic and so on and so forth, but about one-quarter said, ok, I'm surprised, because I mean the, our, in our article, we gave a huge amount of witness accounts and we analysed them and we said that this account, which is quite often mentioned, cannot be trusted, and there, there, there is an important problem there. For example, Wiesenthal centre when discussing Cukurs uses a lot of witness accounts that they found in Yad Vashem, which are absolutely not believable, they're not usable as historic source, there just nonsense, yeah, but still Wiesenthal centre refers to them because they are very colourful, yeah, to us it seems that Holocaust is dreadful itself, yeah, we don't have to, to, to decorate it more than, so, so we mentioned a lot of accounts and we analysed what we can trust and actually people were surprised. The Jewish community tries to analyse it cold-mindedly, normally, and so on, yeah, because that we're not trying in all possible ways to say that all Jews are good and all Latvians are bad and that's right that he was killed and so on, yeah, so, actually, currently the Commission of Historians did not give their  statement on bad boy, but several members of Commission of Historians gave their opinions, so one of them being, so, so,  parallel to that there was also discussion of possibility of reinter Cukurs from the cemetery in Brazil, where he is buried to Brethren cemetery in Riga, which is kind of pantheon, yeah, why he can't be buried there because he was an independence war veteran, well, he was conscripted and he fought in independence war and Brethren cemetery, independence war veterans can be buried, so they wanted to reinter him there. Commission of Historians strongly opposed that. They said that this would be politically very problematic issue and also historically, someone who was member of Arajs Kommando cannot be Latvia's hero. So this was stated explicitly in an article by Karlis Kangeris and basically that's it,  yeah, so, so it's really very complicated story, yeah, and it, it shows that, I mean, you, you can not [...] history with just black and white [...], yeah, but, yeah, with, this was one of the most important of my encounters with Holocaust history in recent years and we really had to explain lots, to speak a lot of this issue, we’re glad that I think in last, at least last year, there were no more discussions concerning him. 

Johannes Probst: I, I have another question. 
You said that anti-Semitism in Latvia is not very strong, but it, but it exists.Co uld you give us some examples of anti-Semitic discrimination that has happened during the last twenty years. 

Ilya Lensky: Well, as I, as I said, as I said, we don’t have discrimination. What we have, we have cases of vandalism in Jewish cemeteries and Holocaust memorials. We have sporadic anti-Semitic statements by certain politicians. I think the last statement was about two years ago, probably, by member of parliament Kārlis Seržants. He apologized, he apologized. Then, of course, we think that main reason for Latvia not solving the restitution issue is anti-Semitism, yeah, so, what else? We don’t have violent attacks, I think at least in last fifteen years, probably. We almost don’t have Holocaust denial, although we, sometimes it appears mostly on the Internet, yeah, so not in the public sphere.

Johannes Probst: Does it happen that Jews get insulted in public? 

Ilya Lensky: Sometimes happens, sometimes happens, but it’s [...] compared to the, to the extent how it happens in Hungary, for example, yeah, so in, in Riga you can easily walk dressed as a traditional Jew and nobody cares about that, yeah, so I, I heard of maybe two, three cases in last years of someone being insulted in the street. So generally, anti-Semitism seems to becomes more and more a taboo, yeah, so, well, the day before yesterday, I was taking the taxi and the driver started talking, not knowing that I’m Jewish, that I come from the Jewish seminar, started talking that he likes Jewish, Jewish proverb, and so on. Well, of course, I’ve heard all the possible stereotypes about how Jews are good with money and so on and so forth, and that’s why we should learn from them. That’s an important topic, but when, when speaking, he tried to use, well, in Latvian, there are two words for the Jews, current, currently is used the word ebreji, in 1920s, 1930s, there was more used the word žīdi, which today is derogatory, but occasionally still used in vernacular speech, but so this driver said, well, I know now politically correct you should say ebreji, so he continued using only this word, not knowing that I’m Jewish. Yeah, so, so for him, it was clear that with someone you don’t know, you should use politically correct terms. It’s a, I think it’s a big step forward, already, yeah, and you have to understand that in Latvia, political correctness and tolerance, these are perceived as, as improper words, yeah, so Latvia generally is very intolerant country, yeah, where we take pride in our intolerance, yeah, so, so I think it’s, it’s an important change. 

Johannes Probst: And do you have hope that anti-Semitism will completely stop somewhere in future? Is it possible? 

Ilya Lensky: No, I don’t think it’s possible. I, I think as, as with most xenophobias, they never can disappear completely because they have not only, not only social roots, we, social roots we can work, but also sometimes individual roots, I don’t know, psychiatric roots, they are very deeply connected with conspiracy theories and so on, yes, and and these are such individual things that you never can fully, fully destroy them, make them fully disappear, so no, definitely not disappear. 

Peter Zinke: Even in Eastern Germany, we have many counties with a high percentage of anti-Semitism, but without any Jew, so you don’t need to have a Jew or to have Jews for anti-Semites, okay. 

Ilya Lensky: Yeah, also, of course. Yeah. So that’s why I don’t believe that anti-Semitism will disappear, but I think it will be at a certain point decreased to the level when it is just a, not an issue, so it will be the same as you know that there are people who believe the earth is flat.\\
Probably, we can’t do anything with these people, but that’s a small marginal group, so probably at a certain moment, anti-Semites, anti-Semites will be decreased to also such group of marginal freaks. 

Rafael Schütz: You were mentioning this 25\% of comments under your article that were not anti-Semitic, the rest was anti-Semitic? But you were surprised that? 

Ilya Lensky: Yeah, that was a surprise that you had a, you know, a significant number of not anti-Semitic comments. 

Rafael Schütz: So, is there a lot of hate speech directed against the Jewish community? 

Ilya Lensky: On the internet, yeah, yeah. But again, [...] problem [...] on the internet, is that you don’t know either maybe that’s one person writing under 25 nicknames, or whatever. And really, we don’t have capacity to, to research it more, yeah, so there are certain organisations that do the monitoring of hate speech on the internet like Latvian Centre for Human Rights [look up] and so on. So for us, we, well, we can’t do everything, yeah, so what we do when we get a certain [...] that there are comments, not just anti-Semitic comments, but incitement to violence, then we’d record an we have very good cooperation with security police, but, but it’s only when there are incitement to violence, yeah, so if there is ranting against the Jews, you ignore it, I mean the moderators of all the web sites, they would do their job, and actually the most harsh comments will be deleted and some of the web sites, they basically [...], the articles related to the Jewish community, sometimes they would not allow comments there, switch off the comments, [...] and it usually happens under articles dedicated to the Holocaust somehow, yeah, because it's very troll-provocative topic [...], yeah, so, and, and we, we have quite good cooperation with the let’s say authorities [...] 

Rafael Schütz: Are there any anti-Semitic phone calls or letters? 

Ilya Lensky: No. No, well, very rarely, definitely not, not phone calls, sometimes there are all kind of weird letters, but again we, we don’t take them seriously, yeah, so generally I can say that Latvia is rather safe country for Jews, I mean I’m for probably an hour now telling you about anti-Semitism in Latvia, so you have to understand that Latvia still rather, rather safe country, and, well, as you’ve seen, we have no guards here in the building, we have, yeah, we have permanent police monitoring, presence in the synagogue, but they’re not checking anybody, yeah so, so they are more to scare people away, bad people, away from the synagogue than really to check everybody who enters and so on, which is impossible in many other European countries as you know.
The Jewish community institutions have really strong security presence and definitely with, with reasons because they’re constantly under threat of attacks from different extreme groups, so here we, thanks G~d, we not have this issue. 

Johannes Probst: One last question. Now that you’re here. when people come to this museum, does it happen that people maybe, for example, pupils together with their school class, that some people have anti-Semitic opinions and, and learn something in this museum that, sort of, well, mind-changing. 

Ilya Lensky: Well, I think, I think we have maybe one or two cases like that, that schoolchildren already came with certain anti-Semitic mindset, but that’s very rare. I don’t think what they learn here would very much change their attitude, well, for, for grown-ups it sometimes changes, so, I think, but, you see, we’re lucky that I guess, if a person is anti-Semitic, he does not come to this museum. So, we have certain distorted prism because we do not communicate with anti-Semites. 
But, but I don’t think that, that you can change this mindset, because I mean if, if you’re really a hard-core anti-Semite, it means that you’ve invested certain time in, you know, educating yourself, in reading all kind of texts. And the texts you read led you to the strengthening of your anti-Semitic belief. Yeah, so hardly you can encounter it, change, change it by encounter with the Jewish museum or something. Yeah, so usually I think our museum contributes to the understanding of many people who are not anti-Semitic, yeah, who didn’t know nothing about Jews and they turn out that the Jews were kind of a bit important part of Latvian landscape, not too important, but important one. And, also, also among politicians, for example, yeah, so many, many of them have no [...] of the museum [...] coming [...] they’re surprised what they see here. Yeah, so, but, but not, definitely not hardcore anti-Semites. 

Rafael Schütz: Are people in general interested in visiting the museum, visiting the synagogue? 

Ilya Lensky: Well, we see quite big interest during the night of the museums, you know the event? Yeah, so we usually have several hundred visitors, the biggest we had 1200, more even, visitors, but you seen how limited is our space, so we’re not very fit for such huge number of people. Most of the people who come, these are locals, these are not tourists, who come during the night of the museums. On regular days, of course, the majority of our visitors are tourists. So to me it seems that many people are afraid to come to our museum because of different reasons. I know personally, some of my friends didn’t want to come to our museum because they thought that it’s mostly concentrated on Holocaust and it will be too dark, too grim and so on. This can be one of the reasons. Another reason probably can be that people think that, same as being interested in Jewish culture, something kind of guilty pleasure, that you should be ashamed of it. The same going for the Jewish museum is something that you should be ashamed of, yeah so, that’s something like not normal. 

Rafael Schütz: Okay, I don’t get it. Why, where does it come from? 

Ilya Lensky: I have feeling like that. That people still are afraid to go to the Jewish museum because they think that then, I don’t know, neighbours or classmates will [...]: Are you Jewish? Why do you go to the Jewish museum? Why are you interested in Jewish culture? Why do you read books about Jews? And so on and so forth. Yeah, yeah, to me it seems like that, obviously, I can be wrong. 

One question, I think you mentioned before the universities that discriminated against Jews during Soviet times, was it something that universities did on their own or was it from [...] 

Ilya Lenksy: No, no, no, it was state policy. 
State policy, yeah, so, you have to understand, it was not formulated, yeah, so there was no law on discriminating against the Jews, well, different from other European countries in the 20s and 30s, for example, yeah, where, basically, there were official anti-Semitic laws and so on and so forth, including Poland where  and all kinds of things like that, yeah, so special seats for the Jewish students in the universities, but by the way, in Latvia in 20s and 30s, there were also officially no anti-Jewish legislation. Inofficially, there was a lot of legislation, but inofficially, a lot of regulations. And the same happened during the Soviet times. Yeah, so, but it was not, as it was not officially regulated, there always was kind of space for manoeuvre for the, for the administration of the university. Yeah, and we know that, for example, in some universities many professors also on their own would impose restrictions for the Jews. Yeah, another hand, I mean in, in Latvia, they did not have that much discrimination and it was okay. Yeah, so, again we don’t have to think that the Soviet system, the same as Nazi system, was kind of monolith, there was a lot of space for all kind of things. 

Rafael Schütz: And then, we read an interview with the leader of the Jewish community in Latvia saying that there is no anti-Semitism in Latvia. I mean, as I gathered from what you were saying, that’s not entirely correct, but 

Ilya Lensky: No, I mean, you see, for, for the interview, I would say more or less the same thing. Because, because the interview is something where from two hours of talk, you get ten minutes of interview, and so on. So if I would be asked, open, answer me correct, short and concise, I would also say, there is no anti-Semitism. Yeah, where, as a formula for we don’t anti-Semitism as important challenge for the Jewish community, yeah, I mean as, as your project is dedicated to researching anti-Semitism, that’s why we’re talking deeply and nuancedly about it in details, and so on. But in general, as I said, we, at least compared to many other European countries, I would say that we don’t have anti-Semitism, we don’t have it as, as a problem, for example.