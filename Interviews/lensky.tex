\section{Iļja Ļenskis}

\textit{Iļja Ļenskis is the director of the museum ``Jews in Latvia'' in Riga. He is involved in Jewish communty life in Latvia since his childhood. He graduated from history at the University of Latvia in Riga and worked at the Jewish museums as a historian until he was offered his current position in 2008. \\
The interview with him took place in Riga on September 25th, 2017}.\par
\vspace*{2em}

\textbf{Do you think that anti-Semitism is a serious problem for the world?}

\textbf{Iļja Ļenskis:} No. Let’s put it like that, the Jewish community does not see anti-Semitism as a important challenge to its existence, which does not mean that there is no anti-Semitism. Of course, with anti-Semitism, we can’t say if it’s strong or not because usually, it won’t manifest openly, so even if quite a big part of Latvian society is infested with anti-Semitism, it does not result in violent acts or anything like that. The main manifestation of anti-Semitism we have is in comments on the internet. It’s part of the broader problem of hate speech on the internet, which has to do not only with the Jews, but generally with all kinds of minority groups. Basically, in Latvia, the main minority that most of the hate speech is directed against is the Russian minority, most of hate speech on Latvian internet would be anti-Russian rather than anti-Semitic, but still, these comments on the internet are the main problem for the Jewish community in the field of anti-Semitism. Obviously, there are some prominent intellectuals who are anti-Semitic, some politicians, but we don’t see it as a main challenge, as I said, to our existence. 

\textbf{Do people treat Jews differently and in a specific way nowadays in Latvia?} 

\textbf{Iļja Ļenskis:} First of all, it's important that the majority of Latvian Jews are not different from the surrounding population, neither visually nor in any other way. When you encounter a person, usually in Latvia you do not know their ethnic background. And of course we have issues with, for example, non-white people being treated differently - this is a very strong problem, we have anti-black racism, for example - but not so much against the Jews, because the Jews look more or less like anybody else. Then, we have not heard of cases of people, for example, being dismissed from work because of their Jewish origin or something like that, which does not mean that it does not happen, but at least we haven't received reports about anything like that. Actually, during the Soviet years, when there were rather strongly, although unofficially anti-Semitic policies of the state, Latvia was one of the regions where the overall situation was better. It was easier for the Jews to enrol in university in Latvia than in Russia or in Ukraine. This was one of the reasons why many people would come to Riga to study, because they could just not enrol in university in their native region. I can hardly say that the Jews will be treated differently on the everyday level. 

\textbf{This discrimination against Jews, was it something that universities decided to do on their own?\footnote{This question was originally asked at a later stage of the interview and inserted here for this publication to avoid thematic breaks.}} 

\textbf{Ilja Lenskis:} No, it was state policy. 
It was not enunciated; there was no law on discrimination against the Jews, different from other European countries in the 20s and 30s, where there were official anti-Semitic laws, including Poland, where they had all kinds of things like that, for example special seats for the Jewish students in the universities, but in Latvia in the 20s and 30s, there was no official anti-Jewish legislation. Unofficially, there was a lot of legislation, a lot of regulations. And the same happened during the Soviet times. But as it was not officially regulated, there always was some space for manoeuvring for the administration of the university. And we know that in some universities, many professors also would impose restrictions for the Jews on their own. We don’t have to think of the Soviet system and of the Nazi system as a monolith, there was a lot of space for all kinds of things. 

\textbf{Today, we talked to someone who said that those people who make anti-Russian statements also make anti-Semitic statements and vice versa. Is that correct?}

\textbf{Iļja Ļenskis:} More or less, although of course you can have ``pure anti-Semites'' and ``pure Russophobes''. We also have a quite significant group of people who are strongly anti-Russian, but constantly make philo-Semitic statements, which I do not believe, because to me, it more seems like they’re trying to whitewash their image by showing, ``you see, we are no hardcore xenophobes, we’re distinguished'', and they would also emphasise that there are good Russians, whom they do not hate, and so on. Generally, I don’t believe such statements. I think if a person is racist or xenophobic, then this resentment covers all possible groups, including LGBT, women, and so on. I would say that there are different groups, but also, of course, it’s true that if they are real xenophobes, they would hate both Russians and Jews. But we have to keep in mind that there is a rather strong anti-Semitic sentiment in the Russian minority as well. 

\textbf{Those people you were referring to, what kind of philo-semitic statements do they make?} 

\textbf{Iļja Ļenskis:} They would always emphasise that the Jews are a traditional minority, a good minority, which never had any problems with the surrounding population, and that they are so sorry that Jews were exterminated during the Holocaust, something like that. Obviously, they are always great fans of Israel. To me, it has kind of an overtone of  ``Jews, go to Israel. Leave Latvia and go to Israel.'', that every nation should stay in their own country. Russians should pack their suitcases and go to Russia, Jews should pack their suitcases and go to Israel, and as Latvians, we want to live here in a racially pure state. It’s not stated explicitly. Maybe I’m wrong. But it is very much the feeling that I have when someone suddenly is very philo-Semitic and very pro-Israeli - to me, it is always suspicious. 

\textbf{Is Jewish culture popular in Latvia? Are there a lot of places like this museum?}

\textbf{Iļja Ļenskis:} I would rather say no. It’s hard to say what popular means, of course, but if we compare it, for example, to what happens in Germany or what happens in Poland, with all the festivals of Jewish culture being organised in different localities and involving mostly local, non-Jewish population, we have almost nothing like that in Latvia. Of course, Latvia can’t be compared in size or importance to Germany and Poland, I mean historically, because in Poland, there was the dominant Jewish community and in Germany, the Jewish community was culturally dominant. Here, many municipalities would like to have something Jewish such as organising Jewish events, but usually, they will invite us to organise them, they will say, ``we want to hold a Jewish culture day, could you bring us a choir or a dance company or something'', rather than organising it on site. I can’t say that we do much to change the situation. To me, of course, it is very important that the Jewish culture is perceived just as one of many cultures, and that you don’t have to be sorry for your interest in Jewish culture - if you can have, say, Salsa contests and Salsa courses, which are part of Latin American culture, and if you can be a fan of Indian cuisine, there is nothing to feel sorry about in being interested in Jewish culture. It's normal, just as liking French films. We want Jewish culture to be treated like that, like something normal. Currently, I know only one place in Latvia where you have local, non-Jewish population in Jewish cultural activities: In Rēzekne, they have a Klezmer band. In other places, nothing like that happens. On the other hand, in several local history museums we have sections on the Jewish history of the respective locality, so the interest is there, but not in a way comparable to, as I said, Germany or Poland or Hungary. I think we are only at the beginning of this process, but obviously, the situation has changed significantly compared to with what it was like 15, 20 years ago. 

\textbf{What about anti-Semitism under the Soviet dominance? Was it different?}

\textbf{Iļja Ļenskis:} As I said, there were several levels: On one hand, there was this state anti-Semitism, and on the other hand, there was the local context. Generally, in Latvia, as I said, anti-Semitism was weaker than in certain other regions of the Soviet Union, if we speak of such things as the possibility to enrol in the university - if you take, for example, Moscow, major universities there were just closed for the Jews in the 70s and 80s. It was clear that if you’re Jewish, you cannot enrol in these universities - you can try, but you won’t pass the exam. Especially in departments like maths, physics, and so on, it was considered that the Jews are dominating Soviet maths, Soviet physics, so they won't be allowed to the universities. For many Soviet Jews, this was the first stimulus to think about their Jewish identity because Soviet authorities, as it turns out, were pressuring them to become Jews. They did not perceive themselves as Jewish. They wanted to be regular Soviet physicists, for example, but Soviet authorities said ``no, you’re a Jew, and a Jew cannot be a Soviet physicist'', so they somehow had to form their attitude and Jewish identity. In Latvia, this was much weaker, you could enrol in the university, you could find a better job, but of course, for example, during the late Stalinist period, in '48, '49, '50, when there was a big anti-Jewish campaign in the Soviet Union, Latvia was also part of this campaign, many prominent intellectuals were arrested and imprisoned, they spent several years in prisons or camps. Basically, all the possibilities to revive  Jewish culture after the Holocaust in Latvia were destroyed in 1949 and 1950. This was also a time when the situation changed, for example with regards to language: After that, most of the parents would be afraid to speak Yiddish to their children, to transmit Jewish language to them, they would rather switch either to Russian or to Latvian. This was during late Stalinist years; in the 70s and 80s, it was still better than in many other regions of the Soviet Union, but of course it’s an enormously big topic, and yet under-researched very much: Riga was one of the centres of the Jewish national movement in the Soviet Union, so there were a number of underground groups in Riga. We cannot call them resistance groups or dissident groups, it was different, and this was also very important: They emphasised that they don’t want to be called dissidents, because a dissident is someone who tries to change the Soviet Union; for them, the main thing is either preserving Jewish culture or getting the possibility to emigrate from the Soviet Union, and the Jews were lucky to be one of the few groups that were allowed to emigrate: Between 1970 and 1979, there was a window where it was allowed. About one third of the Latvian Jewish community emigrated in these years and one additional third after 1989, when it became possible to emigrate again. 

\textbf{Do they still emigrate from Latvia?} 

\textbf{Iļja Ļenskis:} Hardly. Today, it’s again part of a broader problem of emigration from Latvia - to Ireland, to England, to Germany, Norway or wherever. I mean, since we became part of the European Union, as there are open borders in the EU, Latvia lost about 15\% of its general population due to emigration. Of course, the Jews are part of this movement, but it’s not particularly a Jewish movement, as it was in the 90s when the Jews emigrated mostly to Israel, but some of them also to Germany or United States or Canada or wherever. From 2001, I would say, there is very small specifically Jewish emigration, and those people who emigrate to Israel do it because of certain individual reasons: People go to study, or they move with their relatives, or they want to develop business, whatever, but these are personal stories rather than mass emigration, as it was in the 70s and in the 90s. Several years ago, the local representative of the Jewish agency, the main institution that works with emigrants, said, ``I know each and every Jew who emigrates from Latvia, both of them are great guys''. Maybe it was a little bit of an exaggeration, and it were not just two people, but, let’s say, ten, but anyhow we’re speaking of very limited movement. 

\textbf{Is the Holocaust an important topic in the history lessons? I have the feeling that in Poland, we pay more attention to that. What is it like in Latvia?} 

\textbf{Iļja Ļenskis}: In Latvia, as far as I remember now, they have two lessons about Holocaust in grade 9 and two lessons in grade 12, so four lessons altogether as a part of the history course. But this is not the biggest problem because actually, we don’t have a reference point of how many lessons there should be. It's the same when I’m constantly asked why Latvia does not have a ``normal'' policy of commemoration - what is normal? ``Like in Germany'', they say, what does that mean, like in Germany?  You can’t say what is proper and what is improper. It's the same with Holocaust lessons, we have a group of teachers, not very big, but I would say 50, maybe 100 teachers in Latvia, who are really interested in the topic, most of them have been at least once to Yad Vashem, to the special teacher training seminars they have. They come to our museum, they try to bring their pupils to our museum, so we have a group of teachers who teach about the Holocaust well, who are interested and who maybe also try to do some research on the topic. On the other hand, of course, there are a lot of teachers who are not interested in the topic, and the biggest challenge for them is how to balance, in these four lessons, the general context and the local context because today, it seems to me that the main emphasis is on the general aspects of Holocaust, so that pupils will know about the \textit{Kristallnacht} and about Auschwitz, but they would not know what happened to the Jews in their native town. But this is a general problem with the school curriculum: Constantly there are discussions whether Latvian history should be integrated in the general history course or if it should be a separate subject.\\
Generally, Latvia is a very centralised country with a certain obsession with centralisation and with a very sceptical and suspicious attitude towards local identities and local stories. And I think it’s problematic because of course, for Latvia, the history of the Holocaust is not about Auschwitz, it’s something very different. An absolute majority of Latvian Jews were killed here on the spot, there are approximately 130 Holocaust killing sites in Latvia. Usually, only a small group of people in the town would know something about it. Although again, the situation changes now, but still, I feel it’s a bit insufficient. 

\textbf{Is the subject present in arts, in literature and in films?} 

\textbf{Iļja Ļenskis:} Yes, but not to the same extent as in Polish culture. Although, as far as I understand, now they have one film ready about the Holocaust, ``\textit{Tēvs Nakts}'', which will be presented next year, but it’s not only about Holocaust, it’s about story of Žanis Lipke, who rescued more than 55 Jews. In the literature, the topic was present throughout the Soviet times, starting with the first post-war years, '48, '49, and up until now, but I think the first work of literature that would be dedicated to the topic of the Holocaust appeared two years or three years ago - a novel by Māris Bērziņš, “\textit{Svina garša}”, ``The Taste of Lead'', which is dedicated specifically to the Holocaust. Another important thing was the memoir of Valentīna Freimane, who is a very popular art and theatre historian, most of Latvian theatre actors studied under her, and as one of the prominent figures in Latvian intellectual world, she wrote a memoir about her childhood, she is Jewish, so she wrote a memoir about her growing up as a Jewish child and teenager in the 20s and 30s in Latvia, and then about her Holocaust experience. She decided not to go to the ghetto when all Riga Jews were ordered to resettle to the ghetto, she went into hiding instead, and she was hiding all through the occupation, more than three years she spent in hidings in different places in Riga, and she wrote her memoir about it. It was a bestseller and to me it seems that for many people, it was kind of a mind-changing thing, that this prominent Latvian intellectual turns out to be Jewish, and it turns out you can be a Latvian intellectual, a normal person, and be Jewish, and not only Jewish, but Jewish with a Holocaust experience. I think it was number one on the bestseller list of Latvian bookshops for many weeks, and it’s still rather popular. \\
In theatre, there were several performances touching on the topic of the Holocaust in one way or another. The same novel of Māris Bērziņš that I mentioned, “The Taste of Lead”, was also presented as a theatre play in the Latvian National Theatre, I think. Then, in Riga New Theatre, which is the main modernist theatre, there was a play several years ago called “The Grandfather” about a man who searches for his grandfather and encounters three men with the same name, each of whom could be his grandfather and each of whom had a different fate. One of them served in the Soviet army during the War and immediately after the War, and he tells about the Holocaust in his native town in a very detailed manner. When the theatre worked on the play, they cooperated quite closely with us; they came to us, they researched on the topic of the Holocaust in Viļaka, a small town in north-eastern Latvia with a significant Jewish population. The history of the Holocaust was depicted very correctly in the play. That’s it, more or less, of course, it’s not as prominent a topic as it was in Polish culture, but even in Poland it did not become prominent immediately.

\textbf{How is the Latvian state dealing with the Latvian SS forces?}

\textbf{Iļja Ļenskis:} It’s a complicated story. As you understand, there were all kind of different groups in which Latvians fought in the Nazi-occupied territory and one of them is the \textit{Waffen-SS}, to which people were conscripted in 1943. Among these people, who were drafted against their will to the German army after the Holocaust, basically, when most of the Jews were already killed, obviously in this \textit{Waffen-SS} also were people who were involved in the Holocaust, but that’s a different story. Much more problematic are the so-called police battalions and the auxiliary police units, like the Arājs Kommando, which was the main collaborators’ unit, which travelled from Riga to small localities to murder the Jews there. On the official level, of course, they are not glorified in any way. Actually, when Latvia became independent, most of the legal cases of people who were accused and sentenced during the Soviet times on a political basis, they were once again checked, and quite often, people would be rehabilitated. For example, many people who were drafted to the \textit{Waffen-SS}, they would be during the Soviet times sentenced to five years, ten years, twenty years of imprisonment for betrayal, although of course, as Latvia today does not recognize the Soviet Union and considers it as occupant force, obviously, you cannot betray a country of which you are not a citizen and to which you do not serve, so these accusations could be dropped. But if a person was accused not only of these political reason, but also for what we would call crimes against humanity, then they would not be rehabilitated, and as far as I understand, they would not receive, for example, the status of ``victim of political persecutions''. What’s interesting is that in Latvia today, the status of ``victim of political persecution'' is not only given to people who were imprisoned by the Soviets, but also to Holocaust Survivors, they are also seen as victims of political persecution. In this sense, I would say that Latvia deals quite well with its past, although we still have some problems with that, but at least on the official level, the government, every public figure speaking, at least at the commemorative events, considers it to be important to mention the participation of Latvians in the Holocaust and to say something bad about it, and I think it's an important thing because 25 years ago, it was different. 

\textbf{But, for example, the National Historical Commission has no clear stance towards the 16th of March.}

\textbf{Iļja Ļenskis:} No, 16th of March is a completely different story. For two years, it was an official day, but in 2000 it was dropped from the calendar, and today, 16th of March is a private event, we can say, organized by the nationalist groups, usually with very limited participation of real World War II soldiers, for different reasons, because of age, for example. But also, I think, several years ago the main veteran organisation issued a statement that it advises its members to abstain from the participation in the event, it said they should go to Lestene instead  on 16th of March, where the big Brethren cemetery is, and commemorate there, which is absolutely not a problem in my opinion. So, the Prime Ministers for several years now demand from the members of government not to attend the events, I think one of the ministers had to be dismissed because he wanted to attend the event several years ago. You can still see some of the members of parliament, but it’s mostly a private event, and one of my colleagues wrote an article about it where he described it as a political circus. It’s very much an event to advertise certain political forces, but of course also so-called anti-fascists come, which are basically Russian-affiliated organisations, and for them, it’s also a possibility to advertise themselves. Every year, the journalists want comments by the Jewish community about 16th of March, and every year we say that we have no comments on that event because it has nothing to do with Jewish history, it has something to do with current political issues in Latvia and we’re abstaining from comments on the current political situation. 

\textbf{Are there anti-Semitic statements being made on this demonstration?} 

\textbf{Iļja Ļenskis:} Sometimes, but usually not on the official level, not carried on posters, but in the crowd, and we ignore that. 
For us, it’s more or less on the same level as anti-Semitic talks, as someone who starts cursing the Jews in the tram. 

\textbf{Today we talked to a politician from the Harmony party, his name was Jānis Urbanovičs, and he said that some of the members of parliament participate in these events.} 

\textbf{Iļja Ļenskis:} It was a governmental event in 1998 and 1999, but not any more, the government tries to distance itself. Again, how sincere people are, that’s a different story, but at least on the official level, which is important, they distance themselves from the event. In recent years, the level of those who participate in the demonstrations and carry posters denouncing both the Nazi and the Soviet occupation  is higher, and those who try to present Legionnaires as freedom fighters - it’s merely political showing-off, but it’s not an official event any more, and they try to find new modes of organising this event because obviously, it’s not a commemorative event any more either, it’s a glorifying event now and obviously, it’s affiliated with certain political parties. 

\textbf{But of course, the government could find clearer words.} 

\textbf{Iļja Ļenskis:} If it did, it would be a different government. The Riga city council  has tried several times to ban this event, but then the court overruled this decision. And again, it makes sense because it’s a non-violent event. Officially, they do not glorify Nazi or Soviet regimes, so there is no basis to ban the event - as we understand it in a democratic society, even very unpleasant people have the right to demonstrate, so I think that’s the right thing and I prefer it happening like that: On the one hand, the Riga city council shows their attitude, on the other hand, the court does what the court has to do. I do not see this event as problematic, for me in person.

\textbf{We read that there is always some kind of religious service in the Lutheran church after that, and my question would be: How are the relations between the Lutheran church and the Jewish community, and also between the other religious communities and the Jewish community?}

\textbf{Iļja Ļenskis:} As I understand, this service in Riga Cathedral is mostly organised by one of the priests who is affiliated with this event and who often comes to all kinds of similar events. The relation between the Jewish community and most of other religious denominations are quite good. Of course, the Jews are a minor religious group, we have four dominant religious groups that would be represented in all of the events - Lutherans, Catholics, Russian Orthodoxes and Baptists, and other smaller groups would usually not be represented officially, but their representatives would be invited. We don’t have any sort of commissions on cooperation. There is an organisation called Jewish-Christian Council in Latvia which is not very active, I think, for all kinds of reasons - as a small country, you can’t devote 100 percent of your time to NGOs, NGOS are more of a hobby. But generally, I would say that relations are quite good. 

\textbf{I have another question about the general population: If they had to estimate how many Jews live in Latvia, what would they estimate and how wrong would it be?}

\textbf{Iļja Ļenskis:} That’s an interesting question, actually, we never checked that, I know that in some other countries, there were polls, but not in Latvia. I don’t know what people would think, what we know, of course, is that many people think that Jews have too much influence on Latvian life, on Latvian economy, on Latvian politics and so on, but I don’t think that they have any estimate about the number of the Jews. 

\textbf{In Germany, open anti-Semitism is a taboo, at least officially, and we don’t have too many anti-Semitic acts, but especially when Palestine is fighting against the Israelis or the other way around, there is some extreme criticism on Israel and according to my thinking, it’s an expression of hidden or shamefaced anti-Semitism. Do you find it here as well?}

\textbf{Iļja Ļenskis:} It could be. We’re not so interested in the Israeli-Palestinian conflict here; I think when there was the last war in Gaza three years ago, people were asked during a poll whom they support in this conflict, and about equal shares, seven or eight percent, said that they support either Palestine or Israel, and 85 percent said that they don’t care or don’t support anybody. On the other hand, during this war there were two anti-Israeli demonstrations near the Israeli embassy, which gathered 16 people altogether, 
and a pro-Israeli demonstration which was organised by a group of activists and also by some Israeli expats living in Latvia, there were about 150 people there. Generally, as Latvia is very pro-American, it de facto is also very pro-Israeli, we have very good relations with Israel. In recent years, we have very good and very active Israeli ambassadors here, who try to outreach and who try to explain their position and who do not run around advertising Israel, as it sometimes happens with Israeli diplomats. They try to explain the complexity of the conflict, although, obviously, they are Israeli diplomats and therefore are on the Israeli side of the conflict, they do not avoid problematic questions, and I think it's very important. We almost don’t have this so-called new anti-Semitism masking as anti-Israeli sentiment. We hear if people are anti-Semitic, and for example, what is important: Quite often in these anti-Israeli demonstratiosn, we find people from immigrant communities, Palestinians and so on. But in Latvia, we don’t have that problem with, for example, the Muslim communities being anti-Semitic or whatever, even people with a migration background and the Muslim communities, - I don’t know if they have or don’t have this sentiment, but we don’t have that issue in public life, and the only hardcore anti-Israeli and also strongly anti-Semitic group in the Muslim community are recent converts to Islam, Latvians, Russians. Actually, one of the most prominent Latvian anti-Semites converted to Islam - Ahmed Roberts Klimovičs, he’s a prominent figure, he’s rather active on the media and a former journalist -, and to me it seems that for him the motivation to convert to Islam was his anti-Semitism, because he thought that Islam is the main anti-Semitic force of today. But then, the Muslim community just tried to get rid of him, because they, as a normal community, want to deal with religious and cultural issues, they don’t want to get involved in anti-Israeli demonstrations, anti-Semitic hate speech, Holocaust denial and all those things. \\
For us it’s, it’s very good, and speaking about immigrant issues, the Jewish community watches with certain suspicion the anti-immigrant activity because most of the people who organise anti-immigrant demonstrations are people with a strong record of anti-Semitism, and of course we see it as a danger, that this anti-immigrant activism can also switch to anti-Semitic and broad xenophobic activism. 

\textbf{Today, someone we were talking to mentioned a book about Herberts Cukurs. Is the way that Herberts Cukurs is remembered anti-Semitic?} 

\textbf{Iļja Ļenskis:} The story of Herberts Cukurs is very complex. Herberts Cukurs was a famous Latvian pilot who in 1941 became member of the Arājs Kommando, and he was head of the transport section there and also the main arms officer, he was responsible for distributing arms to the units going to the shootings and so on. Then in 1944, '45, he retreated with the German army to the west, and through France, I think, he emigrated to Brazil. He lived in Brazil with a fake identity and was accused several times as a Nazi collaborator and war criminal. The Brazilian government refused at least twice to grant him Brazilian citizenship. There was big pressure and there were investigations of different international organisations, but never any formal trial against him. In 1965, he was murdered in Uruguay, allegedly by Mossad agents. Basically, his story is not part of the story of the Holocaust, it’s very much a story of how we perceive the Holocaust, and what was the driving force for people to get involved in the Holocaust, and there are a lot of myths and misconceptions about him. What we know, what we see from the documents is that he was a member of the Arājs Kommando, he was in quite a high position there, and he personally was involved several times in shootings. What he did not do is what is stated in many books: He was not the leader of the Arājs Kommando, he did not murder 30,000 people. We tried to show that he was, as we put it, a murderer on offer, as was the case with many people who collaborated: They did not have an anti-Semitic background. They were offered by the Nazis to participate in killing of the Jews and they agreed, ``why not''? If they would not be offered, they would not have participated. For Cukurs, for example, the main reason why he came to participate in the Arājs Kommando was that during the Soviet year, he was favoured by new authorities as a pilot, as an aviation engineer. He went to Moscow to offer his help to construction companies there. When he returned, as far as I remember the story, someone even tried to arrest him for being a Soviet collaborator. So for him, joining the Nazi collaborators unit was a way to whitewash himself. And he knew Viktors Arājs before, so he went to him to ask for protection. Another important issue with Cukurs is that he technically saved at least one Jewish girl, who was his lover, and he made her his lover already during the Holocaust. To me, it’s problematic to call it a rescue, but still, technically she Survived because she was affiliated with him, and he really took her with him when he escaped to Brazil, we have the witness account that she gave in Brazil where she mentions him as rescuer. \\
It’s a very complicated story, and for many years now, the right tends to whitewash him and to make him kind of a national hero, whom treacherous, murderous Jews have killed despite him being such a philo-Semite. There is a book by Armands Puče, which was published about year ago. Armands Puče is a sports journalist, his knowledge and understanding of Second World War issues is not very professional. Then there was a musical staged about Herberts Cukurs, and what was problematic with this musical is that in the musical, many stories were told in a very distorted way. We can say, the witness accounts who mentioned him with a gun, okay, these are witness accounts, they can be critical, but he definitely was member of the Arājs Kommando, we know that, and about the question if he personally murdered somebody, we can be extra sceptical, if you want. But what we know, for example, is this: In the musical, he rescues a seven year old Jewish boy, the father asks Cukurs to protect this boy, and first this father, when the members of Arājs Kommando come to arrest him, tries to give them money to leave him alone, and when it turns out that anyhow they are arresting him, he asks Cukurs to protect this boy. First of all, we know the name of the boy, he died a few years ago, and he was not a boy back then. In the musical, it's a seven year-old boy, but in reality, he was 17 or 18 years old. Second thing, his father did not offer money, but we know from the testimony of the boy that the father showed his ID card that proofed he was an Independence War veteran; he presumed that as someone who has fought for Latvia's independence, he would not be arrested. Again, you see that there is this very strong anti-Semitic myth of the Jewish money and Jews trying sneak out with financial means and so on. Then, of course, Cukurs did not rescue the 17 year-old boy - what actually happened was that he did not kill him, and that’s reflected actually both in the memoirs of this boy, but also in another witness account: Cukurs killed someone else, but he decided not to kill this boy, who worked in the Arājs Kommando headquarters. This can hardly be called a rescue. The musical is very problematic from this point of view, and this pressured us to write an article about it. I was quite surprised, there were about 300 comments on the internet under this article. I would say that about quarter of them were positive, while 75\% were anti-Semitic, but about one-quarter said, ``okay, I'm surprised''. In our article, we gave a huge amount of witness accounts and we analysed them and we said that this account, which is quite often mentioned, cannot be trusted. For example, the Wiesenthal Centre, when discussing Cukurs, uses a lot of witness accounts that they found in Yad Vashem, which are absolutely not believable, they're not usable as historic sources, there just nonsense, but still Wiesenthal Centre refers to them because they are very colourful. To us, it seems that the Holocaust is dreadful itself, we don't have to decorate it even more. We mentioned a lot of accounts and we analysed what we can trust, and actually, people were surprised. The Jewish community tries to analyse it cool-headedly, we're not trying in all possible ways to say that all Jews are good and all Latvians are bad. The Commission of Historians did not give their  statement on this ``bad boy'', but several members of the commission gave their opinions. Parallel to that, there was also a discussion of the possibility of retrieving the corpse of Cukurs from the cemetery in Brazil, where he is buried, to Brethren cemetery in Riga, which is kind of pantheon - why he can't be buried there, he was conscripted, and he fought in Independence War,  and on Brethren cemetery, Independence War veterans can be buried, so they wanted to transfer him there. The Commission of Historians strongly opposed that, they said that this would be a very problematic issue politically and also historically, someone who was member of Arajs Kommando cannot be Latvia's hero. This was stated explicitly in an article by Karlis Kangeris. It's really a very complicated story, and it shows that history is not just black and white. For me, this was one of the most important encounters with Holocaust history in recent years and we really had speak a lot about this issue, we’re glad that at least last year, there were no more discussions concerning him. 

\textbf{You said that anti-Semitism in Latvia is not very strong, but it exists. Could you give us some examples of anti-Semitic discrimination that has happened during the last twenty years?} 

\textbf{Iļja Ļenskis:} As I said, we don’t have discrimination. What we have are cases of vandalism in Jewish cemeteries and Holocaust memorials. We have sporadic anti-Semitic statements by certain politicians. I think the last statement was about two years ago, by member of parliament, Kārlis Seržants. He apologised. Then, of course, we think that the main reason for Latvia not solving the restitution issue is anti-Semitism. We didn't have violent attacks during the last fifteen years. We almost don’t have Holocaust denial, sometimes it appears on the Internet, but not in the public sphere.

\textbf{Does it happen that Jews get insulted in public?} 

\textbf{Iļja Ļenskis:} It sometimes happens, but not to the same extent as in Hungary. In Riga, you can easily walk dressed as a traditional Jew and nobody cares about that. I heard of two or three cases in the last years of someone being insulted on the street. Generally, anti-Semitism seems to becomes more and more of a taboo. The day before yesterday, I was taking the taxi and the driver, not knowing that I’m Jewish and that I come from the Jewish seminar, started talking that he likes Jewish, Jewish proverbs, and so on. And he tried to use the politically correct word - in Latvian, there are two words for the Jews, currently, the word ebreji is used, in 1920s, 1930s, the word žīdi was used more often, which today is derogatory, but occasionally still used in vernacular speech - and this driver said, ``well, I know now, politically correct you should say ebreji'', so he continued using only this word, not knowing that I’m Jewish. For him, it was clear that speaking with someone you don’t know, you should use politically correct terms. I think it’s a big step forward, already, and you have to understand that in Latvia, political correctness and tolerance are perceived as improper, Latvia generally is very a intolerant country where we take pride in our intolerance, so I think it’s an important change. 

\textbf{Do you have hope that anti-Semitism will completely cease sometime in the future? Is it possible?} 

\textbf{Iļja Ļenskis:} No, I don’t think it’s possible. I think that most forms of xenophobia can never disappear completely because they have not only social roots, but also individual roots, psychological roots, they are very deeply connected with conspiracy theories and so on, and these are such individual things that you never can fully destroy them. That’s why I don’t believe that anti-Semitism will disappear, but I think it will be at a certain point decreased to the level where it is just not an issue, it will be the same as some people who believing that the earth is flat. Probably, we can’t do anything with these people, but that’s a small marginal group, so probably at a certain moment, anti-Semites will be decreased to such a group of marginal freaks as well. 

\textbf{You were mentioning that 25\% of the comments under your article were not anti-Semitic and the rest was anti-Semitic, and that you were surprised  by that? Is there a lot of hate speech directed against the Jewish community on the internet?} 

\textbf{Iļja Ļenskis:} On the internet, yes, there is. The problem is that on the internet, you never know if it's maybe only one person writing under 25 nicknames. And we don’t have the capacity to research more on it. There are certain organisations that do the monitoring of hate speech on the internet, like the Latvian Centre for Human Rights. For us, as we can’t do everything, what we do when we get certain comments, not just anti-Semitic comments, but incitements to violence, then we record them, and we have a very good cooperation with the security police, but this is only for cases of incitement of violence, so if there is ranting against the Jews, you just ignore it. The moderators of all the web sites, they do their job, and actually the harshest comments will be deleted and on some web sites, under articles related to the Jewish community comments won't be allowed, and it usually happens under articles dedicated to the Holocaust, because it's a very troll-provocative topic.

\textbf{Are there any anti-Semitic phone calls or letters?} 

\textbf{Iļja Ļenskis:} Very rarely, there are definitely no phone calls, sometimes there are all kind of weird letters, but we don’t take them seriously. I can say that generally, Latvia is a rather safe country for Jews, and as you’ve seen, we have no guards here in the building, we do have a permanent police monitoring and presence in the synagogue, but they’re not checking anybody, they are rather there to scare bad people away from the synagogue than to really check everybody who enters, and this is impossible in many other European countries, as you know. The Jewish community institutions there have really strong security presence, and definitely for a reason, because they’re constantly under threat of attacks from different extreme groups, so here, thank G$\sim$d, we do not have this issue.

\textbf{When people come to this museum, does it happen that people, say, pupils together with their school class, have anti-Semitic opinions and learn something in this museum that, is sort of mind-changing for them?} 

\textbf{Iļja Ļenskis:} I think we have had maybe one or two cases like that: That schoolchildren already came with certain anti-Semitic mindset, but that’s very rare. I don’t think what they learn here would very much change their attitude. For grown-ups it sometimes changes something, but, you see, we’re lucky that if a person is anti-Semitic, they do not come to this museum. We certainly have a distorted prism because we do not communicate with anti-Semites. 
I don’t think that you can change this mindset, because if you’re really are a hard-core anti-Semite, it means that you’ve invested certain time in educating yourself, in reading all kind of texts. And the texts you read lead to the strengthening of your anti-Semitic belief. You will hardly find a change through an encounter with the Jewish museum. I think our museum contributes to the understanding of many people who are not anti-Semitic, who didn't know nothing about Jews and they find out that the Jews were an important part of Latvian landscape, not too important, but still an important one. But definitely not hardcore anti-Semites. 

\textbf{Are people in general interested in visiting the museum or the synagogue?} 

\textbf{Iļja Ļenskis:} We see quite big interest during the Night of the Museums. We usually have several hundred visitors, the biggest number we had were more than 1200 visitors, but you've seen how limited is our space is, so we’re not very fit for such huge number of people. Most of the people are locals, not tourists, who come during the Night of the Museums. On regular days, the majority of our visitors are tourists. To me it seems that many people are afraid to come to our museum, for different reasons. I know that some of my friends didn't want to come to our museum because they thought that it’s mostly concentrated on the Holocaust and it will be too dark and too grim. This can be one of the reasons. Another reason could be that people think that, just as they feel that being interested in Jewish culture is something like a guilty pleasure, going to the Jewish museum is something that you should be ashamed of, something that's not normal. I have feeling like that. People still are afraid to go to the Jewish museum because they think that neighbours or classmates will ask: ``Are you Jewish? Why do you go to the Jewish museum? Why are you interested in Jewish culture? Why do you read books about Jews?'' And so on and so forth.

\textbf{We read an interview with the leader of the Jewish community in Latvia saying that there is no anti-Semitism in Latvia. As I gathered from what you were saying, that’s not entirely correct...} 

\textbf{Iļja Ļenskis:} In an interview like that, I would say more or less the same thing. An interview is something where from two hours of talking, you will eventually extract a ten-minute interview. So, if I would be asked for a correct, short and concise answer, I would also say that there is no anti-Semitism, as a formula for we don’t see anti-Semitism as an important challenge for the Jewish community. As your project is dedicated to researching anti-Semitism, we’re talking about it in a detailed and nuanced way. But in general, as I said, we, at least compared to many other European countries, don’t have anti-Semitism, we don’t have it as a problem.
