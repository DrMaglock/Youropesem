\section{Dieter Hegwein and Robert Sandmann}
\begin{otherlanguage}{ngerman}
\textit{Dieter Hegwein was until March 2019 head of Sachgebiet E35 (Staatsschutz) at the Polizeipräsidium Mittelfranken, so responsible for investigating politically motivated crime in the Middle Franconia administrative region. Since then, Mr. Hegwein is head of the Kriminalpolizeiinspektion Ansbach (criminal investigation department of Ansbach). Robert Sandmann is press officer of the Polizeipräsidium Mittelfranken (police department of Middle Franconia administrative region). 
The interview took place on January 4th, 2018, at the Polizeipräsidium Mittelfranken in Nuremberg.
}\par 
\vspace*{2em}
\textbf{Mit welchen Fällen von Antisemitismus hatten Sie in den letzten Jahren zu tun?}

\textbf{Dieter Hegwein:} Wir haben auf Ihre Anfrage hin vom Bayerischen Landeskriminalamt die sogenannten PMK-Zahlen, ``Politisch motivierte Kriminalität'', zum Thema Antisemitismus erheben lassen, bezogen auf den Bereich der Stadt Nürnberg in den letzten fünf Jahren. Das Ergebnis war für mich nicht überraschend: Wir hatten jedes Jahr eine einstellige Zahl von Fällen in Nürnberg. Da geht es in aller Regel um sogenannte Propagandadelikte. Das können Schmierereien an Wänden oder Posts im Internet sein. Wenn diese in Verbindung mit Antisemitismus stehen, dann wird das Themenfeld Antisemitismus angenommen, ich betone, angenommen. Wissen wird man es, wenn man den Täter ermittelt und der Täter aussagt, aus Judenhass so gehandelt zu haben. Häufig gibt es leider keinen Ermittlungsansatz und die Täter bleiben unbekannt. Nicht nur in Nürnberg, sondern in ganz Mittelfranken gibt es sehr wenige Delikte im Themenfeld Antisemitismus. Unser momentaner Arbeitsschwerpunkt im Staatsschutzbereich ist der religiös motivierte Islamismus. Der Rechtsextremismus spielt natürlich auch eine Rolle, aber das Themenfeld Antisemitismus ist aktuell eher untergeordnet. 

\textbf{Welche Rolle spielt der Antisemitismus im Islamismus?}

\textbf{Dieter Hegwein:} Bei uns momentan keine. Es ist klar, dass islamistische Gruppierungen, Organisationen und auch die Islamisten, die diesen Gruppierungen anhängen oder sie unterstützen, von Grund auf antisemitisch eingestellt sind. Islamismus bedeutet in aller Regel auch Antisemitismus. Bezogen auf die aktuelle Gefahrenlage oder die aktuelle PMK-Lage spielt diese Verbindung keine Rolle. Wir sehen die Gefahr für Deutschland beziehungsweise für Bayern oder Mittelfranken, aber nicht in erster Linie für jüdische Einrichtungen oder für Juden. Die Gefahr würde darin bestehen, auf öffentlichen Plätzen möglicherweise Anschläge zu planen, aber nicht gezielt gegen jüdische Einrichtungen, die es als solche natürlich auch in Nürnberg gibt. Daher haben wir diese „gefährdeten Objekte“ von polizeilicher Seite zwar im Blick, aber wir sehen keine explizite Gefahr des Islamismus bezüglich jüdischer Einrichtungen oder Personen.

\textbf{Also anders als zum Beispiel in Frankreich, wo jüdische Einrichtungen von islamistischen Terroristen bedroht oder Juden getötet wurden. Fälle dieser Art gab es in Deutschland nicht?}

\textbf{Dieter Hegwein:} Wir hatten leider im Juli 2016 einen islamistisch motivierten Terroranschlag in Ansbach. Die Art des letztendlich Gott sei Dank missglückten Anschlags hat sich in die Gefahrenlage eingefügt. Das heißt, diesem Täter wird unterstellt, in dieses Open-Air-Konzert gehen und dort eine Bombe zünden zu wollen. Das hat nichts mit dem Judentum per se zu tun, sondern fügt sich in diese Gefahrenlage unter dem Aspekt: „Ich will möglichst viele Menschen erreichen, Öffentlichkeit erzielen und Angst auslösen“ ein. Es fällt das Stichwort Terrorismus, also das Ziel, Angst an allen Orten und Stellen zu verbreiten. Dabei sehen wir öffentliche Veranstaltungen, beispielsweise den Christkindlesmarkt, als Ereignisse, die besonders im Fokus der Polizei stehen müssen. Das soll natürlich nicht heißen, dass wir als Polizeipräsidium Mittelfranken jüdische Einrichtungen vernachlässigen würden.

\textbf{Beim Gazakrieg 2014 gab es eine Demonstration in Nürnberg von Palästinensern und möglicherweise Islamisten, aber auch von durchaus Friedensbewegten und linksorientierten Demonstranten mit Parolen wie „Juden raus!“. Die Demonstranten haben einen McDonald‘s belagert, weil sie glaubten, McDonald‘s sei eine jüdische Einrichtung, und haben dort randaliert. Ist seitdem so etwas nochmals passiert?}

\textbf{Robert Sandmann:} 2014 gab es eine Demonstration im März und eine im April, die als Demonstrationsgeschehen mit dieser Thematik erfasst wurden.
\textbf{Dieter Hegwein:} Seitdem sind keine weiteren Fälle unter dieser Thematik bekannt.

\textbf{Gab es Tötungsdelikte seit der Ermordung des damaligen IKG-Vorsitzenden Shlomo Lewin und seiner Freundin Frida Poeschke 1980?}

\textbf{Dieter Hegwein:} Wir hatten leider den Fall eines versuchten Tötungsdelikts am 01. Januar 2016, als ein letztendlich auch deswegen verurteilter deutscher Staatsangehöriger einen anderen Menschen auf ein U-Bahngleis geschubst hat. Er gab als Motiv an, der andere sei Jude. Der Geschädigte blieb unverletzt bzw. leicht verletzt. Glücklicherweise fuhr gerade keine U-Bahn ein. Der Täter ist wegen versuchten Totschlags verurteilt und letztendlich aufgrund seiner Äußerung: „Das habe ich getan, weil er Jude ist“ als antisemitischer Gewalttäter eingestuft worden. Eine weitere Gewalttat in dem Sinne hat es in den letzten Jahren nicht gegeben. 

\textbf{Der Zentralrat der Juden hat kurz vor den Wahlen 56 antisemitische Zitate der AfD aufgelistet. Ist bezüglich der AfD und dem Themenfeld Antisemitismus in Nürnberg etwas bekannt?}

\textbf{Dieter Hegwein:} Bei der geringen Anzahl an Fällen mit antisemitischem Bezug war kein auffälliger Zusammenhang mit der AfD erkennbar. 

\textbf{Ein sehr großer Teil der Rechtsextremisten ist judenfeindlich orientiert. Auf welchem Stand ist der Rechtsextremismus in der Region?}

\textbf{Dieter Hegwein:} Wir haben die abgeschlossenen Zahlen für 2017 noch nicht, aber tendenziell lässt sich sagen, dass bei den rechtsgerichteten Straftaten ein deutlicher Rückgang im Jahr 2017 erkennbar ist.

\textbf{Nach einem deutlichen Anstieg im Jahr 2016?}

\textbf{Dieter Hegwein:} 2016 hat es überregional einen Anstieg gegeben, aber eher weniger in Nürnberg. Die PMK-Straftaten in Nürnberg waren letztendlich einer Person geschuldet, die in dem Bereich mit vielen Straftaten aufgefallen ist und in der PMK als rechts verortet worden ist. Davon abgesehen waren der Anstieg 2016 insbesondere in der Fläche verortet. Das hat sich aber 2017, ohne konkrete Zahlen zu präsentieren, wieder nivelliert.

\textbf{Schwerpunktregionen der rechten Straftaten waren, zumindest einige Jahre lang, Neustadt Aisch und Bad Windsheim, mit dem versuchten Mordanschlag auf ein Flüchtlingsheim und mit geschändeten Friedhöfen. Sind dahingehend neuere Straftaten bekannt oder hat sich die Lage wieder beruhigt?}

\textbf{Dieter Hegwein:} Diese Straftaten sind völlig zurückgegangen im Vergleich zum Jahr 2006, gerade in den Landkreisen Neustadt Aisch und Bad Windsheim.

\textbf{Was führt zu solchen Häufungen von Fällen in bestimmten Regionen? Was trägt dazu bei, dass zeitlich manchmal mehr, manchmal weniger passiert?}

\textbf{Dieter Hegwein:} Wenn man die Situation in Nürnberg betrachtet, dann ist es genau einer Person geschuldet, die für eine zweistellige Zahl von Straftaten verantwortlich ist. 

\textbf{War das Matthias Fischer? }

\textbf{Dieter Hegwein:} Ich will natürlich keine Namen nennen. Wäre dieser Mensch nicht in Nürnberg, hätte die Stadt Nürnberg zwischen zwanzig und vierzig Straftaten weniger. Die Flüchtlingszeit, gerade das Jahresende 2015 und das Jahr 2016, hat auch in Mittelfranken, so wie in der gesamten Bundesrepublik, dazu beigetragen, dass rechtsextremistische Propagandadelikte zugenommen haben. In der gleichen linearen Weise haben diese Propagandadelikte mit dem Rückgang dieser massiven Zuwanderung auch wieder abgenommen. Wir haben immer noch Zuwanderung, aber wir sehen eben nicht mehr diese Zuwanderungen, wie wir es im Sommer oder im Herbst 2015 gehabt haben. Entsprechend gehen auch diese rechtsgerichteten Propagandadelikte zurück.

\textbf{Kommt es bei Eskalation des Nahostkonfliktes zu mehr Straftaten?}

\textbf{Dieter Hegwein:} Nein, dem kann ich so nicht zustimmen. Wir hatten jüngst eine Aussage in Bezug auf Israel oder Jerusalem, die keinerlei Auswirkungen hier in Nürnberg oder Mittelfranken hatte. Es gab keine Straftaten deswegen. 

\textbf{Gilt das auch nicht in Fällen wie 2012 oder 2014, wenn es einen Raketenbeschuss aus dem Gazastreifen und Einsätze der israelischen Armee im Gazastreifen gibt?}

\textbf{Dieter Hegwein:} Diesbezüglich haben wir Daten erhoben, in diesem Zusammenhang gab es demonstrative Ereignisse, überschaubar von der Anzahl, aber dennoch als eine erkennbare Auswirkung auch hier in der Region.

\textbf{Ging es dabei nur um Demonstrationen?} 

\textbf{Dieter Hegwein:} Ja. Um einen Vergleich zu ziehen: Wir haben momentan eine bestimmte Situation im Iran und wir haben aktuell entsprechende demonstrative Ereignisse in Nürnberg zum Thema Situation im Iran. Diese weltpolitischen Ereignisse zeichnen sich bezüglich des Versammlungsgeschehens in Nürnberg ab. Das ist allerdings nicht gleichbedeutend mit Straftaten.

\textbf{Gibt es Überschneidungen zwischen der Reichsbürgerbewegung uns dem Antisemitismus?}

\textbf{Dieter Hegwein:} Es gibt in Einzelfällen tatsächlich Überschneidungen, also Fälle, in denen Reichsbürger auch als Rechtsextremisten verortet werden. Diese Personen sind gleichzeitig auch Antisemiten. Dabei handelt es sich um Einzelfälle, das ist nicht der Regelfall bei den Reichsbürgern.

\textbf{Wenn eine solche Straftat passiert und diese als politisch motiviert kategorisiert wird, wie genau läuft das Verfahren ab? Wird direkt aufgeschrieben, in welchen Bereich man die Straftat verorten würde oder ist das schon vorgegeben und man wählt irgendetwas aus? Wie kann man sich das vorstellen?}

\textbf{Dieter Hegwein:}  Zunächst einmal muss man die politische Motivation erkennen, die möglicherweise hinter einer Straftat steckt. Wenn man bei diesen Propagandadelikten bleibt, die gerade im Rechtsextremismus einen sehr großen Part einnehmen, sind das Zeichen oder Aussprüche verbotener Organisationen. Dann sind diese Straftaten sehr schnell identifizierbar. Das ist aber nicht in jedem Fall so. Manchmal muss man auch etwas dahinter blicken, um die politische Motivation zu erkennen. Es gibt klare Definitionen, wann eine politische Motivation vorliegt und in welche Themenfelder die Straftat einzuordnen ist: Antisemitismus ist ein Themenfeld, ebenso wie Hasskriminalität oder Antiimperialismus im linksextremistischen Bereich. Nach diesen Themenfeldern werden die Straftaten dann kategorisiert, um zu sehen, welche Lage im Linksbereich, im Rechtsbereich oder im ausländerextremistischen Bereich erkennbar sind.

\textbf{Kann eine Tat in mehrere Kategorien eingeordnet werden oder muss man einen einzelnen Bereich auswählen?} 

\textbf{Dieter Hegwein:} Von den Themenfeldern können natürlich mehrere Kategorien betroffen sein. Die Tat ist aber entweder links-, rechts- oder ausländerextremistisch oder eben islamistisch geprägt. Sie kann unmöglich links- und rechtsextremistisch sein, da würde sich etwas beißen.

\textbf{Wenn Schmierereien wie „Juden raus!“ entdeckt werden, in welche Kategorie wird das einsortiert? Werden diese Parolen automatisch als rechtsextremistisch einsortiert?}

\textbf{Dieter Hegwein:} Ja, Antisemitismus ist ein Themenfeld aus dem Bereich Rechtsextremismus. Alles, was als antisemitisch eingeordnet wird, ist gleichzeitig auch rechtsextremistisch.

\textbf{Obwohl es auch linke Antisemiten gibt.}

\textbf{Dieter Hegwein:} Das mag es geben. Dann müssten aber noch deutliche Zeichen dazukommen, bei denen gesagt werden kann, dass das eher im linksextremistischen Bereich anzusiedeln ist. Wenn ein solcher Spruch gesprüht wird, wie Sie ihn gerade geschildert haben, wird man diesen rechtsgerichtet zuordnen wollen.

\textbf{Wenn noch etwas dazu käme, sodass man diese Straftat in einen linksextremistischen Bereich einsortieren würde, es in diesem Bereich aber kein Themenfeld „Antisemitismus“ gibt, was passiert dann?} 

\textbf{Dieter Hegwein:} In solchen Zweifelsfällen entscheidet letztendlich das Bayerische Landesamt für Verfassungsschutz, in welche Kategorie die Straftat fällt. Man spricht dann von Prüffällen, bei denen es tatsächlich nicht eindeutig feststellbar ist. Der Verfassungsschutz ist die Stelle, die Organisationen als extremistisch bewertet. Solche Entscheidungen kann nicht die Polizei treffen, das kann allein der Verfassungsschutz.

\textbf{Wenn das Delikt in den linksextremistischen oder islamistischen Bereich einsortiert würde, könnte es gleichzeitig als antisemitisch erfasst werden?}

\textbf{Dieter Hegwein:} Hierbei geht es um Plausibilitätsgeschichten bei der Erfassung. In diesem Bereich bin ich mir nicht sicher. Ich glaube, wenn Antisemitismus verortet wird, befindet sich die Straftat gleichzeitig im Rechtsextremismus. Mir ist noch kein Fall untergekommen, bei dem wir den Linksextremismus vorgangsmäßig bedient und gleichzeitig Antisemitismus angekreuzt haben.

\textbf{Wer entscheidet, dass eine Straftat zu einem Prüffall wird?}

\textbf{Dieter Hegwein:} So wie wir diskutieren, wenn für uns oder für die sachbearbeitende Dienststelle nicht eindeutig klar ist, welches Themenfeld belegt ist, würden wir Prüffall ankreuzen und das Landesamt für Verfassungsschutz würde sich dann festlegen. 

\textbf{Kennen Sie in der Polizei jüdische Kollegen?}

\textbf{Robert Sandmann:} Ich persönlich kenne niemanden.

\textbf{Dieter Hegwein:} Ich weiß es persönlich nicht, weil ich meine Kollegen nicht nach ihrer Konfessionszugehörigkeit frage. Möglicherweise kenne ich jemanden, ohne dass ich es weiß. Ins Auge sticht mir niemand dabei.

\end{otherlanguage}
