\section{Miriam Linyal}

\textit{Miriam Linyal (*1922) was born in Poland, in a village near Poznań. In 1942 she was transported to the ghetto in Łódź, from there she was taken to a work camp where she was carrying rocks. She returned to the Łódź ghetto and in 1944 she was sent to Auschwitz. All her family died there. After liberating the camp, she worked in a factory in Germany, later she came back to Poland. After a short time, she emigrated illegally to Palestine. Today, she lives in an Elderly’s home in Ramat Gan in the outskirts of Tel Aviv, many of whose residents are German-speaking Survivors. The interview took place in there on September 21st, 2016.}\par
\vspace*{2em}
\textbf{Gdzie się Pani urodziła i gdzie mieszkała?} 

\textbf{Miriam:} Urodziłam się i mieszkałam w Koźminku, niedaleko Kalisza. Miałam ładne życie, bardzo spokojne, przyjemne.

\textbf{Miała Pani rodzeństwo?} 

\textbf{Miriam:} Tak, siedmioro, ale ja byłam najstarsza w domu. Czworo z nich zginęło, a troje żeśmy przyjechali tutaj do Izraela. Mieszkaliśmy razem z rodzicami. Chodziło się do szkoły, było tam dużo kolegów. Byli Polacy. W ogóle nie czuło się, że to małe miasteczko. Byliśmy bardzo zadowoleni, dopóty nie wybuchła wojna.

\textbf{ Co wtedy się stało? Jakie są Pani wspomnienia?} 

\textbf{Miriam:} Jak Niemcy weszli, była katastrofa. Wojna wybuchła pierwszego września, trzeciego już byli u nas. Niemcy weszli, byli po drodze do Warszawy. Tej samej nocy już zabili pierwszego Żyda. Szukali, co robić z nami. Dzień w dzień było gorzej. Jedzenia nie było. Każdy się bał, bał wizyt w domu. Ja rozumiem dlaczego. Bo jest wojna, to się boję. Wzięło się nas zaraz do pracy i tak dzień w dzień coś innego.

\textbf{Czyli nie zabrano Was do jednego miejsca, by pracować, tylko codziennie jakieś inne zajęcia?} 

\textbf{Miriam:} Nie, na początku codziennie do innej pracy. Szukali tylko, co z nami robić złego. Zabrali nas do pracy. Jak była praca w polu, to było w porządku, ale zabrali nas do jakiejś pracy, której w ogóle się nie potrzebowało. Żyło się w strachu. Żyło się w bardzo niedobrej sytuacji, bo jeden dzień nie był podobny do drugiego dnia. 

\textbf{Czy w tej miejscowości, z której Pani pochodzi, mieszkało dużo żydowskiej ludności?} 

\textbf{Miriam:} Myśmy byli w domu aż do czterdziestego roku. W czterdziestym roku posłali nas do Łodzi. Łódź była pełna. Już było getto w Łodzi, nie było miejsca dla nas. Zaprowadzili nas z powrotem z Łodzi do Koźminka. W czterdziestym roku otworzyli getto. Myśmy nawet nie wiedzieli, co to znaczy getto. I byliśmy w Koźminku dwa lata, od czterdziestego do czterdziestego drugiego roku. Dzień w dzień coś innego, dzień w dzień wysyłali nas do innej pracy. Na przykład jednego dnia prosili o listę dzieci od jedenastu lat do czternastu. Wśród tych dzieci był mój brat. Zabrali ich do Niemiec do pracy i wiem o tym, że on potem zginął w Oświęcimiu. Możecie sobie wyobrazić, jak jedenastoletnie dziecko bierze się do pracy? Do jakiej pracy to dziecko było przyzwyczajone? Zabrali też dzieci za małe, nie mogły pracować w getcie, nie było tam dla nich pracy.  

\textbf{A jak wyglądały warunki w getcie?} 

\textbf{Miriam:}  Życie było ze strachem. Przypadkowo dostałam dobrą pracę, pracowałam na żandarmerii, czyściłam i to była najlepsza robota, bo czasami, jak oddałam wiadro i skończyłam pracę, to kobiety dawały nam kawałek chleba. I tak się pracowało codziennie. Dzień w dzień napady, dzień w dzień szukali Żydów, wyjść nie wolno 

\textbf{Ile miała Pani wtedy lat? } 

\textbf{Miriam:}  Jak wojna wybuchła, miałam szesnaście i pół roku. Dwa lata później była zgroza. Były płacze, były krzyki dzieci. Co chcieli, robili z nami. Nie wolno było wyjść z getta. Było strasznie. I siedzieliśmy w getcie przez dwa lata do czterdziestego drugiego roku. I później wysłali nas z powrotem do getta łódzkiego. Wówczas już było miejsce, bo w łódzkim getcie bardzo dużo ludzi umarło. Nie było nic do jedzenia. Do pracy się chodziło. Praca to nie była najgorsza rzecz, ale że mordowano dzień w dzień kogoś innego. Z rana mam rodziców, a wieczorem już nie mam albo z rana matka miała jeszcze dziecko, a wieczorem już dziecka nie było. Nie wiadomo, dokąd ich wysłali, nie wiadomo, co z nimi zrobili, ale jak się zna historię, to się wie, że zginęli w jakimś obozie. W łódzkim getcie na początku pracowałam w kuchni. Później szyliśmy buty dla żołnierzy niemieckich.

\textbf{A czy do getta w Łodzi trafiła cała Pani rodzina?} 

\textbf{Miriam:} Przyjechałam do getta łódzkiego z rodziną. Moja cała rodzina zginęła. My troje żeśmy zostali przy życiu.  

\textbf{Czy kiedy byli Państwo w getcie, mieli Państwo kontakty z polską ludnością?} 

\textbf{Miriam:} W ogóle nie było kontaktu. Nie wolno było wyjść z getta ani wejść do getta. Życie się nasze skończyło z tym.

\textbf{A czy Państwo w getcie stawiali jakiś opór, buntowali się, czy przyjmowali los, jaki zgotowali okupanci?} 

\textbf{Miriam:} A co mieliśmy robić? Nie wolno nam wyjść, my nic nie mamy do mówienia. Nie ma gazety, nie ma niczego. Raz na miesiąc dostajesz jedzenie, chleb raz na tydzień. Życie było okropne, ale starało się być zadowolonym. Chodziło się do pracy, siedziało się z rodzicami, płakało się razem, śmiało się razem. Życie się toczyło, nie wiadomo było, jak długo to jeszcze potrwa, ale jednak było życie i to się jednego dnia także skończyło... 

\textbf{Kiedy zostali Państwo przewiezieni do Auschwitz? Czy bezpośrednio z łódzkiego getta? }

\textbf{Miriam:} W czterdziestym czwartym roku. Myśmy w ogóle nie wiedzieli, co to znaczy Oświęcim. Słyszało się, że Rosjanie nadchodzą, ale nie wiedziało się dokładnie. Były różne wersje, może Rosjanie jeszcze wejdą, może zostaniemy przy życiu. To nie była droga, taka normalna, jak wejść do pociągu i przyjechać. Myśmy weszli do pociągu, ale nie wiedziało się, dokąd nas wiozą. To była zgroza. Nie ma człowieka, nie ma malarza, żeby mógł wyobrazić, co znaczy chwila w Oświęcimiu. W ogóle nie wiedzieliśmy, że jest Oświęcim. W ogóle nie wiedzieliśmy, że jest taka zagłada. Nikt nie wiedział, co się z nimi dzieje… Nie wierzę, że można przejść Oświęcim i jeszcze żyć i być człowiekiem. Kiedy przyjechaliśmy, mężczyźni zostali oddzieleni od kobiet. Widziałam ojca z daleka, "`Szalom"' nawet nie powiedziałam, bo nie było można dojść. Matka trzymała jedno dziecko przy ręce. Siostra poleciała do niej, żeby być z nią razem, żeby ktoś mógł pomóc matce… Nie udało się, ona została z dzieckiem, z siostrą, która miała pięć lat. Więcej nie widzieliśmy mamy… Zabrali nas, obcięli nam włosy we wszystkich miejscach. Jeden nie poznał drugiego. Chodziło się, patrzyło na twarze. Jak patrzyłam na swoją siostrę, nie mogłam w żaden sposób poznać, czy to ona. Wyglądaliśmy jak małpy. Tak nas zostawili na kilka godzin. Później zabrali nas do sauny. Nikt z nas nie wiedział, co to jest sauna. Myśleliśmy, że to ostatnie chwile naszego życia. Krzyki "`Szma Israel"'. Strasznie płakaliśmy. Godziny mijały za godzinami. Dali nam duże buty, nikt nie mógł ich nosić, bo były za duże. Trzymało się je w rękach. Dostałam spódniczkę, która była dla mnie kilka numerów za duża. Godziny trwało, zanim nas wysłali do baraków, a na podłodze nie mieliśmy pryczy. Pięć nas było, trzy siostry i dwie kuzynki. Cały czas nas liczyli. 

\textbf{A czy w obozie Państwo kontaktowali się z Polakami, czy były możliwości spotkania?}

\textbf{Miriam:} Nie, skąd! Nawet z siostrą nie mogłam mówić, to było okropne. Nie można było komu powiedzieć, że nam ciężko. 

\textbf{Kiedy Pani wydostała się z obozu?} 

\textbf{Miriam:} W tym samym dniu, kiedy się skończyła wojna. Później nas posłali znowu do pracy.  

\textbf{Dokąd?} 

\textbf{Miriam:} W Sudetach, w Niemczech. Ja byłam pięć i pół roku. Pracowało się bardzo ciężko, jako młode dziewczynki, może od trzeciej nad ranem, czasami do dwunastej w nocy. 

\textbf{A z Niemiec wróciła Pani do Polski?} 

\textbf{Miriam:} Polacy zabrali nas do Polski, do Łodzi. Wyjechało się z Łodzi, to i wróciło się. A tam dokąd masz pójść? Przecież nie jestem łodzianką, a jeśli nawet, to ktoś już mieszka w moim domu, nie wiesz kto. Na ziemi się leżało, na stacji kolejowej. Ale byliśmy szczęśliwi, że dostawaliśmy kawałek chleba do jedzenia. Nie można sobie wyobrazić, co to jest kawałek chleba. Czekało się, dokąd pójść, gdzie nocować, co dalej będzie. W nasze grupie była jakaś dziewczynka, co wyszła na ulicę i spotkała siostrę, która była w innym obozie. Uciecha! Siostra miała kolegę, który słyszał, że w jakimś miejscu na Północnej jest mieszkanie puste. Dach tam trzeba było zreperować. A kto to robi? Nie ma forsy, nie ma niczego. I tak się zaczęło to życie. Słyszałam, że ciotka moja żyje, moja mama była z rodziny z trzynaściorgiem dzieci, tylko ciotka jedna jedyna została przy życiu. Miała dwoje dzieci. Całe rodziny tam zginęły.

\textbf{A czy po powrocie do Łodzi dostaliście jakąś pomoc od Polaków? } 

\textbf{Miriam:}  Nie, nikt nam nie pomógł. 

\textbf{Dlaczego?} 

\textbf{Miriam:} Nie wiedzieliśmy, do kogo pójść, nikt się nami nie interesował. Rosjanie jeszcze tam byli. Szłam do miasta, szukałam pracy. Spotkałam kolegę, z którym pracowałam w kuchni. On mnie nie poznał, byłyśmy bez włosów, suknie jeszcze były z numerem. Powiedział, że przed wojną mieli fabrykę pończoch i rodzice dali to Polakom, ale jak ktoś przeżyje, to mieli mu oddać. I Polacy mu oddali tę fabrykę i dał mi pracę. Jak już zaczęłam pracować, mogłam pojechać do ciotki do Kalisza. Ja i moje dwie siostry żeśmy pojechałyśmy do ciotki. Uciecha! Uciecha i płacz. Tuliło się do ciotki jak do matki. 

\textbf{Czy dużo Żydów wróciło do Łodzi? }

\textbf{Miriam:} Bardzo mało, bardzo. Było nas tysiące, ale ile wróciło? Trzynaścioro rodzeństwa w rodzinie mojej matki i prawie nikt nie został… W każdej rodzinie było tak samo, nie było jednej rodziny całej.  

\textbf{Jak długo zostaliście w Łodzi?} 

\textbf{Miriam:} Niedługo. Wyjechaliśmy zaraz.

\textbf{Dokąd? Do Izraela?}

\textbf{Miriam:}  Ja przyjechałam do Izraela. Polacy nas po drodze złapali, siedziałyśmy we więzieniu. Nie tak szybko było. Później żeśmy postanowiły, że nie ma dla nas miejsca i wyjeżdżamy do Izraela. Przyjechałyśmy nielegalnie, ale to jest nieważne. Każdy z nas chciał tylko, żebyśmy mieli swój własny kraj. 

\textbf{Czy ktoś wam pomagał?} 

\textbf{Miriam:}  Była pomoc, z wyjściem to już była pomoc. Była organizacja, wszystko przez organizacje. I zaczęło się życie tu w kraju, bardzo ciężkie życie. Tu wojny się zaczęły, przecież nie było państwa. Dopiero teraz powstało państwo, ale byliśmy szczęśliwi, że będzie kraj. 

\textbf{Czy utrzymywaliście jakieś kontakty z Polską?} 

\textbf{Miriam:} Ja już byłam trzy razy. Pojechałam do Oświęcimia z dziećmi, chciałam, żeby widzieli Oświęcim. Byłam w Warszawie, w Krakowie, byłam w różnych miejscach. Ale o Polsce nic złego nie mówię. Ja na początku bardzo tęskniłam za Polską. Polacy w Kielcach zabili Żydów, co wrócili z Rosji. Jest tablica, że tu leżą ci, którzy przeszli wojnę i ich zabili, bo byli Żydami. Ale jednak ja tęskniłam strasznie za Polską. 

\textbf{Czyli te dobre wspomnienia zostały w Pani bardziej?} 

\textbf{Miriam:}  Nie odczuwałam nic złego, było mi dobrze. Pojechałam do naszego miasteczka. Przeszłam kilka razy przed naszym domem, był zamknięty. Nasz dom… Nie mogłam wejść i zobaczyć co się tam dzieje. 

\textbf{Dlaczego?} 

\textbf{Miriam:} Nie wiem. Nie weszłam, nie widziałam nic. Może jak bym poszła do gminy, to by mi otworzyli, ale nie mogłam… 

\textbf{A tam nikt nie mieszkał? } 

\textbf{Miriam:} Teraz mieszkają. A jedna rzecz mi się nie podobała w Polsce. Powiedziałam to burmistrzowi w Koźminku, że nie mogę zrozumieć tego, że wsadzili Polaków do synagogi. Teraz tam Polacy mieszkają! W synagodze. Powiedziałam "`Nie wstydzicie się?"' Tego do dzisiejszego dnia nie mogę zapomnieć. Dobrze znam burmistrza, przyjeżdżają czasami do mnie i wtedy ich przyjmuję i wszystko dobrze, ale uraz mam tylko o to. Ale co można zrobić? Jak jest wojna, to wszystko jest odwrotnie. 

\textbf{Czy więcej dobra czy więcej zła doświadczyła Pani od Polaków?} 

\textbf{Miriam:} Nie wiem. Pamiętam wszystko dobre, nie pamiętam złego. Myśmy żyli bardzo dobrze z Polakami. Nasi sąsiedzi to byli Polacy. Żyliśmy bardzo dobrze, dopóty wojna nie wybuchła, później to zupełnie coś innego. Po wojnie także byłam kilka razy, było wszystko w porządku. 

\textbf{A w czasie wojny Polacy nie pomagali Żydom?}

\textbf{Miriam:}  Nie mogli, nie można tego wymagać. Niemcy nie pozwolili. 

\textbf{A czy zdarzały się sytuacje, że ktoś pomimo tego ryzykował życie i pomagał Państwu?} 

\textbf{Miriam:}  Nie. Ale ja nie mam pretensji. Mam pretensje tylko do Niemców, a do nikogo więcej nie mam. Antysemityzm zawsze był i zostanie jeszcze, tak mi się zdaje. 

\textbf{A czy ma Pani pomysł, skąd on się wziął w Polsce? Bo często się mówi, że Polacy są antysemitami.} 

\textbf{Miriam:} Nie mogę tego powiedzieć. Myśmy bardzo ładnie żyli z sąsiadami Polakami. Chodziłam do polskiej szkoły, uczyłam się z Polkami. Jeden jest taki, drugi jest taki i podczas wojny… Przeszło i żeby więcej nie było takich wojen. 