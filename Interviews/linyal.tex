\section{Miriam Linyal}

\textit{Miriam Linyal (*1922) was born in Poland, in a village near Poznań. In 1942 she was transported to the ghetto in Łódź, from there she was taken to a work camp where she was carrying rocks. She returned to the Łódź ghetto and in 1944 she was sent to Auschwitz. All her family died there. After liberating the camp, she worked in a factory in Germany, later she came back to Poland. After a short time, she emigrated illegally to Palestine. Today, she lives in an Elderly’s home in Ramat Gan in the outskirts of Tel Aviv, many of whose residents are German-speaking Survivors. The interview took place in thereon September 21st, 2016.}\par
\vspace*{2em}
\textbf{Jak się pani nazywa, gdzie pani mieszkała, jak się pani urodziła?} 

\textbf{Miriam:} Ja mieszkałam w Koźminku. To jest obok Kalisza, niedaleko Poznania. Ja już wszystko wiem, mapa jest na stole.  

Zaraz po wojnie wyjechałam z Kalisza, także po wojnie byłam w Kaliszu. Oprócz tego już byłam dwa razy w Polsce. Byłam w Oświęcimiu, myślałam: a nuż jeszcze kogoś spotkam. 

\textbf{Czy Pani się urodziła w Kaliszu?} 

\textbf{Miriam:} Nie. Ja się urodziłam w Koźminku, to jest osiemnaście kilometrów od Kalisza. 

\textbf{Proszę nam powiedzieć coś o swojej rodzinie, o początkach, o pierwszych wspomnieniach.} 

\textbf{Miriam:} Ładne życie, nie można powiedzieć. Było bardzo ładne, spokojne życie. Nikt nie patrzył na innych. Byliśmy zadowoleni, chodziło się do szkoły. Życie było bardzo przyjemne. Dopóki nie wybuchła wojna, było bardzo przyjemne życie. 

\textbf{Miała pani rodzeństwo?} 

\textbf{Miriam:} Tak, siedmioro, ja byłam najstarsza w domu. Czworo zginęło, a my we troje przyjechaliśmy tutaj do Izraela. 

\textbf{Powiedziała Pani, że ma Pani przyjemne wspomnienia. Proszę powiedzieć troszkę o swojej młodości, o dzieciństwie.} 

\textbf{Miriam:} O młodości, ja nie wiem. Pytają się, ale ja nie wiem, co mam powiedzieć. 

\textbf{O szkole?} 

\textbf{Miriam:} Mieszkaliśmy razem z rodzicami. Chodziło się do szkoły, było tam dużo kolegów. Byli Polacy, byli… W ogóle nie czuło się, że to małe miasteczko. Byliśmy bardzo zadowoleni, dopóki nie wybuchła wojna. 

\textbf{Co wtedy się stało? Jakie jest pani pierwsze wspomnienie?} 

\textbf{Miriam:} Co się wtedy stało, ojojoj... Nie wiem, co wy wiecie, ale… 

\textbf{To znaczy wiemy dużo z filmów, z historii, ale chcielibyśmy...} 

\textbf{Miriam:} Ale co chcecie wiedzieć, jak Niemcy weszli, co zaczęli robić? Była katastrofa. Wybucha wojna pierwszego sierpnia, tak? 

\textbf{Pierwszego września.} 

\textbf{Miriam:} Trzeci był już u nas Niemcy. To było drogie. Niemcy weszli, byli w drodze do Warszawy. Tej samej nocy zabili już pierwszego Żyda. I zaczęło się od tego dnia, od momentu wkroczenia Niemców. Szukali, czego szukać i co z nami zrobić. Dzień po dniu było coraz gorzej. Nie było jedzenia. Wszyscy się bali. Rozumiem dlaczego. Obawiam się, że jest wojna. Ale zabrano nas do pracy i codziennie coś innego.

\textbf{Czyli nie zabrano Was do jednego miejsca, by pracować, tylko codziennie jakieś inne działania?} 

\textbf{Miriam:} Nie, na początku codziennie do innej pracy. Jak była praca, z której można było mieć zadowolenie, to byliśmy zadowoleni, ale nie było takiej pracy. Szukali tylko, co z nami robić złego. Ja nie wiem co wy wiecie od początku. Co wy wiecie, co było? Cała historia zaczyna się od tego pierwszego dnia, co Niemcy weszli do… do… do Polski. Zaczęło się na początku „Jude” żółte. Słyszałyście pewnie. 

\textbf{Żółte oznaczenia?}

\textbf{Miriam:} To było „\textit{Jude}” na żółtym tle i zabrali nas do pracy. Jak była jakaś praca w polu, to było w porządku, ale zabierali nas do jakiejś pracy, której w ogóle się nie potrzebowało. Mimo tego zabrali, pracowało się, wychodziło się z domu, bało się wyjść, ale jednak chodziło się do domu, do szkoły się nie chodziło. Życie się zaczęło i tak skończyło się nasze życie. I tak myśmy byli w domu, żyło się w strachu. Żyło się w bardzo niedobrej sytuacji, bo jeden dzień nie był podobny do drugiego dnia. Jedna godzina nie była podobna do drugiej godziny. 

\textbf{A proszę powiedzieć czy w tej miejscowości, z której Pani pochodzi, mieszkało dużo żydowskiej ludności?}

\textbf{Miriam:} To znaczy, myśmy byli w domu aż do czterdziestego roku. W czterdziestym roku posłali nas do.. do Warsz…, do Łodzi. Łódź była pełna. Już było getto w Łodzi, nie było miejsca dla nas. Zaprowadzili nas z powrotem z Łodzi do Koźminka. W czterdziestym roku otworzyli getto. Myśmy nawet nie wiedzieli, co to znaczy getto. I byliśmy w Koźminku dwa lata, od czterdziestego do czterdziestego drugiego roku. Dzień w dzień coś innego, dzień w dzień wysyłali nas do innej pracy. Na przykład jednego dnia prosili listę dzieci od jedenastu lat do czternastu. Wśród tych dzieci także był mój brat. Zabrali ich do Niemiec do pracy i wiem o tym, że on zginął w Oświęcimiu. Możecie sobie wyobrazić dziecko, jedenastoletnie dziecko bierze się do pracy? Do jakiej pracy to dziecko było przyzwyczajone? Chodziło się, jakbyśmy byli na innym świecie. Zabrali też za małe dzieci. Nie mogą pracować w getcie, nie ma pracy dla nich w getcie. Zamordowali ich gazem, to też wiemy dokładnie, gdzie to było i wiemy, w którym kierunku. 

\textbf{A jak wyglądały warunki w getcie? Bo wiemy, że one różniły się od siebie.} 

\textbf{Miriam:} Słuchaj. Już trzy matki nie miały dzieci, już dzieci nie miały rodziców. Życie było ze strachem. Przypadkowo ja jeszcze dostałam dobrą pracę jako… no… jako... No chwileczkę, przypomnę… Są słowa, czy ja już zapomniałam? Już lata temu wyjechałam z Polski. Pracowałam na żandarmerii, czyściłam. Przypadkowo i to była najlepsza robota, dlaczego?, bo czasami jak oddałam wiadro i skończyłam pracę, to kobiety dawały nam kawałek chleba jeść. To była bardzo dobra praca. I tak się pracowało dzień w dzień. Dzień w dzień… napady, dzień w dzień szukali Żydów, dzień w dzień wyjść nie wolno. Jedzenie tylko co oni dają, nie można więcej jeść dzień w dzień. Jak się wstało jakiegoś dnia, byliśmy pewni, że aż do wieczora będzie coś, ale co nie było wiadomo. Zabrali dzieci, które miały dwa czy trzy latka, nie wiedziało się dokąd. Do dzisiejszego dnia nie wiemy, gdzie zginęli. Dzień w dzień przez dwa lata żaden dzień nie był podobny do drugiego dnia. 

\textbf{Ile miała wtedy Pani lat?} 

\textbf{Miriam:} Jak wojna wybuchła, miałam szesnaście i pół. Dwa lata później była zgroza. Były płacze, były krzyki dzieci. Nic nie pomogło, to, co chcieli, robili z nami. Nie wolno było wyjść z getta. Było strasznie, strasznie. Ja całą moją historie w ciągu tylu lat w pół godziny mogę skończyć i mogę skończyć na przyszły rok... I siedziałyśmy w getcie, w tym getcie przez dwa lata do czterdziestego drugiego roku. Dzień dzisiejszy podobny do drugiego dnia. I później wysłali nas z powrotem do getta łódzkiego. Wówczas już było miejsce, bo w łódzkim getcie  bardzo dużo ludzi umarło. Nie było nic do jedzenia. Do pracy się chodziło. Praca to nie była najgorsza rzecz, ale zamordować dzień w dzień kogoś innego. Z rana mam rodziców, a wieczorem już nie mam albo z rana matka miała jeszcze to dziecko, a wieczorem już tego dziecka też nie było. Nie wiadomo, dokąd ich wysłali, nie wiadomo, co z nimi zrobili, ale… jak się zna historię, wie się co tam było, to wiadomo, że zginęli w jakimś... obozie. Wszyscy zginęli w Oświęcimiu. Najwięcej ludzi zginęło w Oświęcimiu. Myśmy nawet nie wiedzieli, że jest Oświęcim. Bo jak ja przyjechałam do Oświęcimia, jeszcze mamy czas do Oświęcimia, do czterdziestego czwartego roku, to nie jest tak powiedziane. Ja widziałam, jak jest napisane, że tu jest Oświęcim. Ja wiedziałam, że jest Oświęcim, wiedziałam, że jest takie miasto na mapie. Ale nie wiedziałam, że jest taka zagłada. Coś okropnego. Nie ma człowieka, co może opowiedzieć, czym była chwila jedna w Oświęcimiu. Nie ma… Historia jest naprawdę znana, ale opowiedzieć jest bardzo, bardzo ciężko. W łódzkim getcie każdy pracował przy innej pracy. Ja na początku pracowałam w kuchni. Później zabrali mnie do... pracowałam… myśmy szyli buty dla żołnierzy niemieckich. Ja nie wiem, jak wam powiedzieć. Możemy w jednym słowie wszystko skończyć. 

\textbf{A czy do getta w Łodzi dostała się cała pani rodzina?} 

\textbf{Miriam:} Ja przyjechałam do getta łódzkiego z rodziną. Moja cała rodzina zginęła, cała rodzina zginęła. My troje żeśmy, z siedmioro, my troje żeśmy zostali przy życiu. A moja mamusia była z rodziną, co było trzynaście osób, trzynaście sióstr i bracia. Jedna siostra została przy życiu, a jej dzieci też zginęły. 

\textbf{A proszę powiedzieć, czy kiedy byli państwo w getcie, czy doświadczali państwo…, mieli państwo kontakty z polską ludnością?} 

\textbf{Miriam:} W ogóle się nie było w kontakcie, w ogóle nie było kontaktu. Myśmy w getcie nie mieli żadnego kontaktu. Nie wolno wyjść z getta, nie wolno wejść do getta. Życie się nasze skończyło z tym. Nikt z nas nie wiedział jak długo, co będzie, ale jednak to było. Było strasznie. To jest historia, co jeszcze lata, lata będziecie się uczyć i nie wiadomo, co jeszcze można z tego wynieść. 

\textbf{Każda… Czasy były te same, ale każda historia jest inna, prawda? Każdy miał swoje własne przeżycia.} 

\textbf{Miriam:} Każdy przeżył okropnie, to jest nie normalne przejście. Mam rodziców i tu zabierają w jednej chwili i nie wiadomo dokąd? Wiedziało się, że jeżeli zabierają... Moja mama miała czterdzieści lat, mój tatuś miał czterdzieści lat. Co to jest czterdzieści lat w życiu? Nic, ale wszystko zginęło, niczego nie ma… Do kogo mieć pretensje? 

\textbf{A czy Państwo w getcie stawiali jakiś opór, buntowali się, czy pokornie przyjmowali los, jaki zgotowali okupanci?}

\textbf{Miriam:} A co mieliśmy robić? Nie wolno nam wyjść, my nic nie mamy do mówienia. Nie ma gazety, nie ma niczego. Z rana jak wstaniesz... raz na miesiąc dostajesz jedzenie, chleb raz na tydzień. Są ludzie, co nie mogą czekać z tym chlebem cały tydzień, jedli pierwszego dnia albo drugiego. Okropnie było. Życie było okropne, ale starało się być zadowolonym. Chodziło się do pracy, wracało się do pracy, siedziało się z rodzicami, płakało się razem, śmiało się razem. Życie szło, nie wiadomo było, jak długo to jeszcze potrwa, ale jednak było życie i to się jednego dnia także skończyło… 

\textbf{Kiedy zostali Państwo przewiezieni do Auschwitz? Czy to bezpośrednio z łódzkiego getta przewieźli Państwa do Oświęcimia?} 

\textbf{Miriam:} Oświęcimia... 

\textbf{Kiedy to było? W którym roku?}

\textbf{Miriam:} W czterdziestym czwartym roku. Myśmy w ogóle nie wiedzieli, co to znaczy Oświęcim. Nie wiedziało się, że istnieje Oświęcim, ale… to było w czterdziestym czwartym roku. Rosjanie się już… Słyszało się, że Rosjanie nadchodzą, ale nie wiedziało się dokładnie. Były różne wersje, może Rosjanie jeszcze wejdą, może zostaniemy przy życiu. Każdy myślał o tym, co chciał. 

\textbf{Miał nadzieję…} 

\textbf{Miriam:} Tak. W ogóle się nie wiedziało, że jest Auschwitz. W ogóle się nie wiedziało, że jest taka rzecz. Gdzie my mamy jechać? To nie była droga, taka normalna, jak żeście weszli do pociągu i żeście przyjechali. Myśmy także weszli do pociągu, ale to jest...  ale nie wiedziało się, dokąd oni nas wiozą. To jest zgroza. Auschwitz nie ma człowieka, nie ma malarza, nie ma, nie wiem kogo, żeby mógł wyobrazić wam, co to znaczy chwila w Oświęcimie. Nie ma tego. W ogóle się nie wie, że jest Oświęcim. W ogóle się nie wie, że jest taka zagłada, w ogóle się nie widziało, dokąd wzięli nasze dzieci. W pierwszej chwili małe dzieci zostały bez rodziców, co to znaczy? Tu mam mamę, tu mam ojca. Mężczyźni oddzielnie, kobiety oddzielnie. Nikt nie wiedział, co się z nimi dzieje… Zagłada. Nawet zagłada nie była normalna. To jest trochę… bardzo ciężko, ale jednak ja wam powiem, że to jest.. Nie wiem, jak to wytłumaczyć. Nie ma człowieka, co by mógł wyobrazić chwile w Oświęcimie, nie ma. Nie w książkach, w niczym nie można opowiedzieć, co było w Oświęcimie. Nawet w jednej chwili, co tam było. 

\textbf{Czy od razu w Oświęcimiu zostali państwo rozdzieleni jako rodzina?} 

\textbf{Miriam:} Tak. Ojca, mężczyźni byli oddzielnie i kobiety oddzielnie. Tatuś trzymał... Jedenaście lat miał brat… Jedenaście lat miał brat. Nawet „Shalom” się nie mówiło, nie wiedziało się, co się tu dzieje. Myśmy nie wiedzieli, że istnieje taka rzecz. 

\textbf{Jak Państwo w tych okropnych warunkach organizowali sobie życie, żeby przetrwać?} 

\textbf{Miriam:} W jaki sposób? W jaki sposób? Przecież nie można wyjść, nie można samemu pójść. My nie wiemy tutaj, jaka zagłada tam jest. Ja mówię… jak wybrałam te cztery lata i mówiłam jakby było tylko dziesięć minut… U nas to nie było „klali” albo oddzielnie. Zamordowali wszystkich. Rodziny pękły, nie ma już rodzin. Nie ma matki, nie ma ojca, żeby ci położył rękę na... i powiedzieć „jutro będzie lepiej”. Nikogo nie ma już. Gdzie są, nikt nie wie. Ja nie wiedziałam, że istnieje Auschwitz, ja nie wiedziałam, co to słowo Auschwitz. Ja wiedziałam, że jest miasto na mapie i nazywa się Oświęcim. Ale więcej nie wiedziałam. 

\textbf{Jak długo Pani tam była, w obozie?} 

\textbf{Miriam:} Kilka tygodni, kilka miesięcy. Później nas posłali znowu do pracy. 

\textbf{Dokąd?} 

\textbf{Miriam:} W Sudetach. Ja byłam pięć i pół roku, pięć i pół roku. Od pierwszej chwili do ostatniej chwili, jak się skończyła wojna. 

\textbf{W jakiej miejscowości?} 

\textbf{Miriam:} Proszę? Nie, nie wszystko w Niemczech. Myśmy skończyli w Sudetach, w Niemczech. Pracowało się bardzo ciężko. Jak młode dziecinki, dziewczynki, może byłam waszym wieku. Nie wiem dokładnie, ale może od trzeciej nad ranem, czasami do dwunastej w nocy. 

\textbf{Chciałabym żebyśmy kontynuowały Pani opowiadanie. Opowiadała Pani o Auschwitz.}

\textbf{Miriam:} Nie „Pani”, ja nazywam się Miriam. Przyjechałyśmy do Oświęcimia. Słuchaj, w pięć minut ja opowiedziałam cztery lata, cztery lata. Ty wiesz, co to znaczy? Ja w łódzkim getcie oparzyłam się jak pracowałam w kuchni. Upadłam i kocioł, sześćdziesiąt litrów zupy, wszystko było na mnie. Byłam spalona, i zabrali mnie do szpitala. Leżałam kilka tygodni w szpitalu. Wyszłam i żyję, i żyję! Wszędzie było, przeszło i żyje się, ale była zgroza, także w łódzkim getcie. Ale później to już była zagłada. O tym człowiek w ogóle nie może sobie wyobrazić, co było w Oświęcimie. Ja, co ja tu przeszłam. Nie wierzę, że można to przejść i jeszcze żyć i być człowiekiem. W tej samej chwili, co się przyjechało, mężczyźni oddzielnie, kobiety oddzielnie. Widziałam ojca z daleka, „Szalom” nawet nie powiedziałam, bo nie było można dojść. Matka trzymała jedno dziecko przy ręce. Siostra poleciała do niej, żeby być z nią razem, żeby ktoś mógł pomóc matce… Nie udało się, ona została z dzieckiem, z cór…, no, siostrą, co miała pięć lat. Więcej myśmy nie widzieli mamy… Zabrali nas, obcięli nam włosy we wszystkich miejscach. Jeden nie poznał drugiego. Chodziło się, patrzało się na twarze. Jak patrzałam na swoją siostrę i nie mogłam, nie mogłam w żaden sposób, poznać, czy to jest moja siostra. Wyglądało się jak małpy. Wiesz, w owym czasie kobiety, dziewczynki się wstydzili chodzić bez… bez ubrania i bez włosów ze wszystkich stron. Nie wiedziało się, gdzie kłaść ręce, nie wiedziało się, z kim, z kim mówić, kto ty jesteś i ty nie wiedziałaś, kto ja jestem. Tak nas zostawili kilka godzin. Później zabrali nas. Ty wiesz, co to znaczy ,,sauna”? Zabrali nas do sauny. Myśmy, nikt z nas nie wiedział, co to jest sauna. Myśmy myśleli, że to są ostatnie chwile naszego życia. Krzyki „Szma Israel”. To jest modlitwa nasza. Nie wiemy, gdzie jesteśmy. Ja nie wiem, kto mnie trzyma przy rękach, to była moja siostra. Wyglądało się naprawdę jak małpy. Jeszcze nigdy nikt nie opowiadał, co było w Oświęcimiu. Nie ma tego człowieka, że może opisać pięć minut w Oświęcimie. Myśmy strasznie płakali, nie wiedziało się, co to jest zagłada. Myślało się, że matkę wysłano do innej pracy, tośmy płakali, co mama zrobi. Mama była chora i już nie mogła chodzić na nogach. Kto jej pomoże? Co to będzie? Ty nie wiesz, gdzie jesteś, ty nie wiesz, co się z tobą dzieje. Ty już nie jesteś ty! Godziny łaziło się. Zabrali nam te buty, wszystko zabrało, niczego nie ma. I w saunie jak to szli do sauny... Myśmy nie wiedzieli, co to jest sauna. Widocznie weszło się na te schody, które były bardzo gorące, to wszystkie dziewczyny za jednym razem zaczęli „Szma Israel”, to jest nasza modlitwa na... Myślało się, że to są ostatnie chwile naszego życia, za każdym razem myślało się, że to jest ostatnia. My nawet nie wiemy, że jest…, że palą tam… ciała. Myśmy nic nie wiedzieli o Oświęcimiu, dlatego się nie wiedziało, co się z nami dzieje. Godziny szły za godzinami. Dali nam tylko duże… buty. Nikt nie mógł nosić je, bo były za duże. Trzymało się w rękach, ja dostałam spódniczkę, która była kilka numerów za duża dla minie. Nie wiedziałam, co mam trzymać, może mi spaść, a jak spadnie, to zaraz wszystko. To ręka leży już na ciale i dostajesz… i biją. Godzinę trwało, dopóty nas wysłali do baraków, ale na podłodze myśmy nie mieli pryczy, tylko na podłodze taki kawałek. Pięcioro, były trzy siostry i dwie kuzynki, żeśmy leżeli na podłodze kilka dni. Nie wiedziało się nawet, co to jest, gdzie my jesteśmy. Cały czas nas liczyli. Dwa razy dziennie wzięli nas do ubikacji, a ubikacje byli, przecież widziałaś, jak tam jest. Tak się chodziło, wszystko za zegarkiem. Zegarka nie było, ale wszystko było w tym czasie to, w tym czasie, to, w tym czasie to. Była zgroza, trudno to sobie wyobrazić… Byłyśmy kilka… kilka tygodni. 

\textbf{Pani była w grupie, w której byli tyko Żydzi, prawda? Powiedziano nam, że były trzy grupy. Byli Niemcy, którzy byli za jakieś przestępstwa skazani, Polacy i Żydzi. W Oświęcimiu, trzy jakby części obozu.} 

\textbf{Miriam:} A ja nie wiem, ja nie pamiętam, w którym roku. 

\textbf{W każdym razie, tam gdzie Pani była, byli tylko Żydzi, czy byli też Polacy razem z wami?} 

\textbf{Miriam:} Nie nie,… polskie.. z łódzkiego getta żeśmy pojechali do Oświęcimia. 

\textbf{Ale tylko ludność Żydowska, czy Polacy też?} 

\textbf{Miriam:} Nie, nie, nie nie było Polaków, z getta żeśmy pojechali. 

\textbf{A czy w obozie Państwo kontaktowaliście się, nie było możliwości spotkania?}

\textbf{Miriam:} Nie, skąd! Nawet z siostrą nie mogłam mówić, nawet z siostrą nie mogłam mówić, to było okropne, okropne. Nie można było komu powiedzieć, że ciężko nam. 

\textbf{Jak długo była…?}  

\textbf{Miriam:} Kilka tygodni, ja dokładnie nie wiem. Oświęcim to jest bez daty. 

\textbf{Kalendarza nie było, prawda?} 

\textbf{Miriam:} Oj, nic nie było. Nie było rąk, nie było nóg, nie było niczego. Wiesz jak małpy, jakby były drzewa, to bym powiedziała, że to są prawdziwe mały. Nie ma określenia do tego, nie ma żadnego określenia. 

\textbf{Kiedy Pani wydostała się z obozu? Kiedy go Pani opuściła?} 

\textbf{Miriam:} W tym samym dniu. U nas się skończyła wojna, w tym samym dniu, kiedy się skończyła wojna. Myśmy byli w getcie, potem żeśmy byli w obozie, bardzo ciężko myśmy pracowali nad ranem.  

\textbf{Gdzie to było?} 

\textbf{Miriam:} W Sudetach, tam po stronie niemieckiej. Pracowało się we „Flax” fabryce, była tam bardzo ładna dziewczyna niemiecka. Nie mamy do kogo mówić słowa, nie wolno wymówić słowa, a jak chciało się pójść do ubikacji, nie mogłaś sama pójść.\\  
\textit{(ucięte nagranie)}\\
A w Oświęcimie to nawet nie było nic do jedzenia. W Oświęcimie żeśmy dostali… że się nic nie jadło.( ucięte nagranie). Wychowanie miało, ale talerzy nie dali nam, tylko garnek bez niczego. Dostała jeden taki garnek i tak musiało się pić zupę. Wyobraź sobie, w Niemczech i taka rzecz. Ja idę przed siostrą i się boję więcej jeść, bo jeśli to należy do twojej siostry albo do kogoś innego.  

\textbf{…żeby nie zabrakło?}

\textbf{Miriam:} Słuchaj, taka zgroza z jedzeniem, nie masz pojęcia. Bała się zjeść, może kartofel będzie wewnątrz, a ten kartofel trzeba podzielić na pięć. Więc dzień w Oświęcimie nie można opowiedzieć, ile by nie było cierpliwości, nie można opowiedzieć, co to jest Oświęcim. 

\textbf{A proszę powiedzieć z Sudetów z Niemiec wróciła Pani do Polski czy…?}

\textbf{Miriam:} Nie, wróciło się do Polski, Rosjanie weszli, to zaczęli znowu z kobietami... Po kilku tygodniach to żeśmy… Polacy zabrali nas do Polski i wróciło się do Polski. Wróciło się do Polski, do Łodzi. Wyjechało się z Łodzi, to i wróciło się.  Dokąd masz pójść? Przecież nie jestem łodzianką, a nawet łodzianki, to ktoś już mieszka w moim domu, przecież nie wiesz kto. Nasze życie było… tylko na ziemi się leżało, na stacji kolejowej. Ty przecież wiesz, że w Polsce pada jeszcze deszcz, ale kogo to obchodzi. Ale byliśmy szczęśliwi, że dostawaliśmy kawałek chleba do jedzenia i to było... Nie można sobie wyobrazić, co to jest kawałek chleba. Dostawało się, ile chcesz chleba i to było to wszystkie nasze życie. Czekało się, dokąd pójść, gdzie nocować, co to będzie? W nasze grupie była jakaś dziewczynka, co wyszła na ulicę i spotkała siostrę, która była w innym obozie. Uciecha! Siostra miała kolegę, który się zaczął z nami zobowiązywać. I słyszał, że w jakimś miejscu na Północnej jest mieszkanie puste. Deszcz padał i spadł do nas. Ten dach trzeba było zreperować. A kto to robi? Nie ma kogo, kto to zrobi. Nie ma forsy, nie ma niczego. I tak się zaczęło to życie. Słyszałam, że ciotka moja żyje, tak. To była z rodziny… moja mama była z rodziny z trzynastu dzieci. Wyobraź sobie, że tylko ciotka jedna, jedyna została przy życiu. Miała dwoje dzieci. Jedna córkę spotkałam w Oświęcimie, ale ona zginęła tam. To wszystko, całe rodziny, całe rodziny zginęły tam. 

\textbf{A czy w Łodzi, po powrocie do Łodzi dostaliście jakąś pomoc od Polaków? Czy w jakiś, jakikolwiek sposób Polacy pomagali wam w Łodzi?} 

\textbf{Miriam:} Nie, nikt nam nie pomógł. 

\textbf{Dlaczego?} 

\textbf{Miriam:} Ja nie wiem, do kogo mamy pójść, nikt się nami nie interesował. 

\textbf{Czy wtedy groziła jeszcze jakaś… kara za pomoc?} 

\textbf{Miriam:} Rosjanie jeszcze byli tam, nikt się nami nie interesował. Szłam do miasta, szukałam, co można zrobić, szuka się pracy. Trzeba jeść kawałek chleba, nie? Spotkałam kolegę, co pracowałam z nim w kuchni. On mnie nie poznał, my byłyśmy bez włosów i bez... Suknie jeszcze były z numerem. Wyglądało się jak małpy. On powiedział, że przed wojna mieli fabrykę pończoch i rodzice oddali to Polakom, znaczy się, jak ktoś przeżyje, to mu oddadzą. I Polacy mu oddali tą fabrykę i dał mi do pracowania, że mogę przyjść i pracować u niego. No i jak już zaczęłam pracować, to już mogłam pojechać, bo ciotka mieszkała w Kaliszu, to już mogłam pojechać do ciotki. Ja i moje dwie siostry żeśmy pojechałyśmy do ciotki. Uciecha! Uciecha i płacz. Myślała, że jej córka żyje, miała piętnaście lat. Piękna dziewczyna, jak marzenie. Ja ją spotkałam także w Oświęcimie…, nie syn, nie córka, nie mąż. Tuliło się do ciotki, myślałyśmy, że to jest matka. Nie można opowiadać takich rzeczy i takich rzeczy się nie uczy, ale to było okropne. 

\textbf{No właśnie, my nie możemy takich rzeczy przeczytać w książkach, jedyne co to...} 

\textbf{Miriam:} Ja jakbym tego nie przeszła i jak bym czytała książkę z jakiegoś obozu, to ja bym nie uwierzyła, że to może być. 

\textbf{Dlatego tak ważne, że my możemy się z Państwem spotkać, że możemy wysłuchać opowieści.} 

\textbf{Miriam:} No, ja nie wiem, nie może być, nie ma słów. Brakuje mi słowa, brakuje mi łzy. Łzy szły cały czas, cały czas się płakało. Pracowało się w płaczu, wszystko było w płaczu. 

\textbf{Czy dużo… dużo Żydów wróciło do Łodzi?} 

\textbf{Miriam:} Bardzo mało, bardzo. Znaczy, ja nie wiem, co znaczy dużo. Myśmy byli tysiące, ale ile wróciło? Jak ci powiedziałam, trzynaście sióstr miała moja matka i no, jak się to nazywa..., braci. Nikt ci nie został, ty wiesz, to jest cała rodzina, ile potomstwa to jest. To było w każdej rodzinie to samo. Nie było jednej rodziny całej. 

\textbf{Jak długo zostaliście w Łodzi?} 

\textbf{Miriam:} W Łodzi niedługo, niedługo, wyjechało się z Łodzi zaraz. 

\textbf{Dokąd? Do Izraela?} 

\textbf{Miriam:} Ja przyjechałam do Izraela... Ja z siostrą żeśmy przyjecha… To tak, że Polacy nas po drodze złapali, siedziało się we więzieniu. Nie tak szybko było. No i później żeśmy postanowili, że nie ma dla nas miejsca i wyjeżdżamy do Izraelu. Do Izraelu już można było przyjechać. Nielegalnie. Myśmy przyjechały nielegalnie, ale to jest nie ważne. Każdy z nas chciał tylko, żebyśmy mieli swój własny kraj. 

\textbf{Ktoś wam pomagał?} 

\textbf{Miriam:} Były, były pomoce z tym… z wyjściem to już były pomoce. No, była organizacja, wszystko przez organizacje. Każdy postanowił, że tu musi być państwo dla nas. Jesteśmy jak, no nawet nie wiem, jak to powiedzieć. Małpy to nie wyglądały lepiej od nas, wyglądały lepiej od nas. \\
No i zaczęło się to życie tu w kraju. Było bardzo ciężkie życie. Tu wojny się zaczęły, przecież nie…, nie…, nie miało się państwa. Dopiero teraz powstało państwo, ale byliśmy szczęśliwi, że będzie kraj. 

\textbf{Czy utrzymywaliście jakieś kontakty z Polską? Czy ktoś...} 

\textbf{Miriam:} Ja już byłam. Ja już byłam dwa, trzy razy, wyjechałam do Oświęcimia. 

\textbf{Ale tuż po przyjeździe do Izraela czy jakieś kontakty..?}

\textbf{Miriam:} Nie! Oni interesowali się nami. Zupełnie! Każdy tylko marzył, żeby był nasz kraj. Dosyć, dosyć! Dosyć się wydarzyło na tym całym świecie. Bez grosza, bez pracy. Pracowało się bardzo ciężko, ale każdy z nas był zadowolony, będziemy mieli państwo. 

\textbf{Czyli nie traktowała Pani wtedy Polski jako kraju którego.. nie miała Pani pozytywnego nastawienia do Polski?} 

\textbf{Miriam:} Dwa razy pojechałem do Polski. Kiedyś z dziećmi chciałem, żeby zobaczyli Oświęcim. Byłem w Warszawie, byłem w Krakowie, byłem w różnych miejscach. Rozmawiałem z naszym burmistrzem, ale nie był zbyt zainteresowany, ale nie wiem, może tak powinno być. Nie jestem zły ... Ale nie mówię nic złego o Polsce. Na początku bardzo tęskniłem za Polską. Urodziłem się tutaj w Polsce, miałem rodziców, miałem siostry. Na wesele nie było nic.

\textbf{To niesamowite, że Pani to mówi z uśmiechem, kiedy tyle złego w Polsce też Panią spotkało, bo na przykład te obozy, prawda?} 

\textbf{Miriam:} Nie, nic złego, ja nie mówię na Polskę. 

\textbf{To znaczy to było w Polsce, nie z rąk Polaków, ale w Polsce.} 

\textbf{Miriam:} Niee, słuchaj… Wrócili z Rosji.. Jeszcze było… Rosjanie... Jak wrócili z Rosji, Polacy jeszcze  w Kielsku (w Kielcach?) zabili Żydów, co wrócili z Rosji. Jedź do Kielska (do Kielc) to zobaczysz, że jest tablica. Tak, ja byłam, widziałam, że tu leżą ci, którzy przeszli wojnę i ich zabili... bo są Żydami. Jak pojedziesz do Kielska (do Kielc) to zobaczysz, że jest tablica na tym domu. Ale jednak ja tęskniłam strasznie za Polską, nie wiem, na początku ja tęskniłam. 

\textbf{Czyli jednak te dobre wspomnienia zostały Pani bardziej.} 

\textbf{Miriam:} Słuchaj, ja nie odczuwałam nic złego. Było mi dobrze. Pojechałam do naszego miasteczka. Przeszłam kilka razy przed naszym domem. Dom był zamknięty. Nasz dom… Nie mogłam wejść i zobaczyć co się tam dzieje. 

\textbf{Dlaczego?} 

\textbf{Miriam:} Co? Ja nie wiem. Nie weszłam, nie widziałam nic. 

\textbf{Ale nie mogła Pani, czy Pani nie chciała?} 

\textbf{Miriam:} Ja nie wiem. Może jak bym poszła do gminy, to by mi otworzyli, ale nie mogłam… 

\textbf{A tam nikt nie mieszkał?}

\textbf{Miriam:} Aaa, niee. Teraz mieszkają, teraz ja wiem. A jedna rzecz, co mi się nie podobało w Polsce, to ja powiedziałam do burmistrza Koźminku, że jedną rzecz nie mogę zrozumieć.. Że wsadzili Polaków do Synagogi. 

\textbf{A co się w tej Synagodze teraz znajduje?} 

\textbf{Miriam:} Synagoga to jest najświętsza rzecz, nie? I teraz tam Polacy mieszkają! W Synagodze. 

\textbf{Aha, dom mają po prostu.} 

\textbf{Miriam:} Nie, nie, nie. Ja się burmistrza powiedziałam „W jak sposób? Nie wstydzicie się, wsadzić tu tyle...” Przecież każdy coś zostawił, jakiś majątek w domu. Tego do dzisiejszego dnia nie mogę zapomnieć. Ja burmistrza dobrze znam i przyjeżdżają czasami, ja ich wtedy przyjmuję i wszystko dobrze, ale uraz mam tylko do tego. 

\textbf{Czyli jednak ważną świętość…}

\textbf{Miriam:} Ale słuchaj, co możesz zrobić? Jeśli toczy się wojna, wszystko jest odwrócone, wręcz przeciwnie.

\textbf{Cieszę się, że mimo wszystko, mimo tak trudnej przeszłości ma Pani uśmiech na twarzy, kiedy Pani mówi o Polsce. Bo my pracujemy nad projektem, który...}

\textbf{Miriam:} To jest życie. To jest historia życia, nie? 

\textbf{A proszę powiedzieć czy więcej dobra czy więcej zła doświadczyła Pani od Polaków?} 

\textbf{Miriam:} Ja nie wiem. Ja pamiętam wszystko dobre, ja nie pamiętam złe. Myśmy żyli bardzo dobrze z Polakami. Nasi sąsiedzi to byli Polacy. Myśmy żyli bardzo bobrze, dopóty wojna nie wybuchła, później to zupełnie coś innego. Po wojnie także byłam kilka razy, było wszystko w porządku. 

\textbf{A w czasie wojny Polacy nie pomagali Żydom?} 

\textbf{Miriam:} Nie mogli, nie mogli, nie można tego wymagać. Niemcy nie pozwolili. Nie. 

\textbf{A czy zdarzały się takie sytuacje, że ktoś pomimo to ryzykował życie i pomagał Państwu? Czy ktoś mimo tego, że było zagrożenie życia, pomagał wam?} 

\textbf{Miriam:} Nie, nie, nie. Ale ja nie mam pretensji. Wszystko mi się podoba, dobrze.  

\textbf{Cieszę się, że ma…}

\textbf{Miriam:} Ja nie mam pretensji. Mam pretensje tylko do Niemców, a do nikogo więcej nie mam. Antysemityzm zawsze był i zostanie jeszcze. Tak mi się zdaje. 

\textbf{A czy Pani ma pomysł skąd on się wziął w Polsce? Bo często się mówi, że Polacy są antysemitami.}

\textbf{Miriam:} Nie wiem, nie mogę tego powiedzieć. Nasi sąsiedzi byli Polakami, żyliśmy bardzo dobrze. Chodziłem do polskiej szkoły, uczyłem się z polskimi kobietami i mieszkałem z nimi bardzo dobrze. Nie mogę tego powiedzieć. Jeden taki jest, drugi taki jest i podczas wojny wszystko się odwraca i nikogo nie winię.

\textbf{A getto to też nie był Polski pomysł, prawda? Więc to nie Polacy zgotowali getta.} 

\textbf{Miriam:} Nie, nie. To Niemcy, Niemcy. Niee, oni się też bali. Było zupełnie inne życie, nic nie można było zrobić. Do getta nie mogli wejść, z getta nie można było wyjść. Getto jest getto, zamknięte. Nie ma słów, ale to jest wojna. Przeszło i żeby więcej nie było takich wojen. 

\textbf{Może chce Pani odpocząć?}

\textbf{Miriam:} Ja wiem? Niee, ja wczoraj byłam w szpitalu. Dopiero wyszłam ze szpitala wczoraj po południu. To jestem trochę jeszcze osłabiona jeszcze, kie. Ale widzisz, jednak postanowiłam, że będę mówiła i mówię. 

\textbf{Bardzo się cieszę.} 

\textbf{Miriam:} Jakby nie one [\textit{o osobie, która przerwała wywiad}], to ja bym mogła mówić cały czas. Ja nie chcę, ja mówię, że wszystko jest w porządku cały czas.  