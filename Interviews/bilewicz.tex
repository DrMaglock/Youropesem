\section{Dr hab. Michał Bilewicz, Prof. UW}

\textit{Michał Bilewicz (*1980) is a professor at the Faculty of Psychology of the University of Warsaw, where he works since 2008. \\
He studied sociology and philosophy in Warsaw between 1999 and 2003. His Ph.D. thesis, completed in 2007, carries the title `''Between self-verification and social identity processes: Social psychology of threatened ingroup status'''. 
He has worked in New York and Delaware, U.S., and in Jena, Germany. \\
His research interests, among others, are conspiracy theories, prejudices, intergroup conflict, the threat of positive social identity and dehumanization, mainly on the example of xenophobia, anti-Semitism and ethnic conflicts, reconciliation mechanisms after genocides, and the influence of cognitive mechanisms and public language on the exclusion of minorities.
He is vice president of Forum for Dialogue, an organization dedicated to inspiring new connections between contemporary Poland and the Jewish people. \\
The interview took place in the Faculty of Psychology of the University of Warsaw on January 30th, 2018.}\par
\vspace*{2em}
\textbf{Michał Bilewicz:} You're probably fully aware that you're coming to Poland in a very special time in terms of politics. This topic that you want to explore is actually a very big issue and a very sensitive one in the Polish-Israeli-Jewish relations.\\
I'm doing many collaborative projects with Germany and we're doing a lot of research exactly on the topic that you are working on, for example with Roland Imhoff, who is professor at the University of Mainz, and similar projects with another grant that we're doing with the University in Leipzig, with Immo Fritsche, so there are several researchers who are doing research very similar to mine. We're collaborating, looking at how Germans, Israelis and Poles are viewing the history of the Holocaust, how they view the responsibility and the reasons, for example, we're looking at explanations of how it was possible that the Holocaust happened. Through representative surveys with big samples in the three countries, we found that in Poland and in Israel, people explain the Holocaust in a very similar way: They would say that the reason for the Holocaust was German anti-Semitism and the German culture of obedience. In Germany, most people would say that it was rather because of the economic crisis, because of social circumstances that Hitler seized power, and this is why the Holocaust took place. They're looking more at the social circumstances rather than on national character. This is a typical pattern, actually, that everybody, every nation wants to look on its own misdeeds as caused by circumstances and by social factors, while own glories and own victories are seen as rather caused by our personality. It's just the opposite if you're talking about another nation. If you're talking about another nation, then you'd say that the failures of this nation are caused by their personality, but about the successes, you'd rather say that they are caused by the circumstances.\\
Those of you who are from Poland, which city are you from? 

\textbf{We're from Zamość.}

\textbf{Michał Bilewicz:} That's a very interesting area from the perspective of Polish-Jewish relations. The whole conception of Polish-Jewish relations is a tricky one because it implies a certain understanding that Jews are not Poles, which is something located at the core of the discussion that we have now, a debate about the relations between Poland and Israel and the history during the Second World War. There is this implied reasoning that the Polish state has to defend the good name of Poland and to show the world that all Polish people were rescuers of Jews. Actually, most of the Jews that were killed here were Polish citizens. So the question is why the Polish nation or the Polish state today has to represent the history of some of its citizens, but not others? In my opinion, the Polish minister should also represent the interests of those people who were murdered.\\
Zamość is also interesting because the best testimony from the times of the Holocaust written in Polish language comes from there. No, one of the two best. As the first, I would consider Calek Perechodnik, ``\textit{Spowiedź}'', ``The Testimony''. Calek Perechodnik was a Jewish person who was part of the Jewish police in Otwock ghetto. He was a collaborator, a Jewish policeman, and he sent his own family to the death camp. He survived until 1944. He was hidden on the Aryan side in Warsaw and then somebody betrayed his hiding place and he was blackmailed and then finally killed by Gestapo. He writes his testimony in 1943/44 and as he writes about his memories, he says: ``The time of the Holocaust is the time when the morality collapsed. Nobody was moral.'' He said that neither Poles nor Germans, of course Germans were the evil at that time, but he says that Poles were immoral at that time and they betrayed Jews and Jews were immoral at that time because Jews trusted that everything will be fine and actually collaborated in a way. This is something which is very hard to read today because if you're Polish, it makes you feel uncomfortable, if you're Jewish, it makse you feel uncomfortable, and if you're German, of course it makes you uncomfortable. I think this is the truth about the Holocaust, though: That nobody can feel really comfortable. There is no way to feel comfortable. The second testimony which is really worth reading is that of Zygmunt Klukowski, who was a doctor working very close to Zamość, in Szczebrzeszyn. Most of the book is actually about Zamość because he spent half of the time in Zamość and half of the time in Szczebrzeszyn. Szczebrzeszyn was a small town and he was an educated doctor, he wanted to party and to be part of this nice cultural place of Zamość, where there were good cafés and so on. This is where he spent most of his time and he writes about the reactions to the Holocaust of people in Zamość, in Szczebrzeszyn, in Izbica, and in the whole area. About the problem of people who are sometimes trying to earn some money, to earn some property, to get houses or apartments, taking advantage of the fact that their neighbours are being sent to Izbica and then to Bełżec. He's Polish, he's not Jewish, but he writes, how is it possible that this can happen, that now neighbours treat neighbours as an opportunity to loot, an opportunity to get some property? This is very difficult to read. It's a very interesting perspective because it's the perspective of a person who is just there, who's part of the Polish Underground, who was part of the \textit{żołnierze wyklęci} after the Holocaust, and still asks these questions. It shows that the history is very complex, it’s not black and white. For me as a psychologist, it’s very important to know that we never can use our moral standards to make judgements about what happened during the Holocaust because it’s so different and nobody in this room knows how they woulld act if they were placed in such situations.

\textbf{You mentioned that this picture of history and of how the Holocaust happened is different in these three countries. Is this a general thing, shared by all people, or would you say that a certain group of people in each of the countries has a different view?}

\textbf{Michał Bilewicz:} I was talking about a kind of average, the average German, average Pole and average Israeli. Of course we know that average people do not exist. It's a composition of many different groups. I am sure you are aware of the diversity in Germany, for example. If you look at the distribution of attitudes in Germany, you find that if you go from the east, from the Saxonian Ore Mountain Range, where you have high popularity of AfD and very anti-immigrant views to, say, the Rhineland, where generally, people are rather open towards immigration, then you realise, that we have very different histories here because those people in Eastern Germany lived in a very homogeneous country for a long time, while the people in the Rhineland were always very open, they always had a lot of contacts with France. There are many different histories and the same is true for education levels: People who are more educated usually feel more of a sense of responsibility or guilt for what happened in the past and people who are less educated usually are more defensive and would like to have this very idealised image: ``We were always good and never did anything bad''. This is about Germany, but talking about Poland it is the same: The more educated people in Poland are, the more willing they are to talk about the dark aspects of history and not only the glorious ones. Besides, we have the same regional pattern as in Germany, actually. If you go more to the east, people are more defensive, if you go more to the west, people are more open-minded, so paradoxically, there is also this east-west dimension operating in Poland when you talk about history. If you look at research on Lubelskie region, where Zamość is loacted, and Podlaskie region, these are the two Voivodeships where the willingness to talk about the negative aspects of history is the lowest. People would rather like to talk about the glorious history and be silent about the negative aspects of history. If you go to Israel, it's more or less the same, it's not about an east-west dimension there, but also about education, and people who are less educated would also like to perceive Jews as only glorious, not doing any harm, being absolutely good. People who are more educated sometimes acknowledge that there were some crimes against Palestinians, the Arab population, and are more willing to talk about \textit{Nakba}, for example, the moment when Arabs were pushed from Jerusalem, many Arabs were killed and forced to migrate from what was to be Israel. Not everybody in Israel commemorates that.\\
I know very little about Latvia. I know that Latvia has a very traumatic history in this respect, that there was also a lot of collaboration going on in Latvia. Even in Poland, we have memories of Latvians actually serving in the German troops during the war. But I don't know about the debates in Latvia today, whether there are debates about the collaborationist parts of SS. 

\textbf{There are no broad debates about this topic because people don't like to speak about it, people just want to stick to the image of either heroes or victims. ``We, the Latvian people, were either victims or heroes,  we didn’t collaborate.'' There is official recognition that Latvians participated in the Holocaust, but it is officially said that these people were only a small group. Actually, it is not so small, it was shown that the collaboration was very broad. Many Latvians collaborated.} 

\textbf{Michał Bilewicz:} And the combatants who were part of \textit{Waffen-SS} have veteran rights, which from a Polish perspective sounds quite strange. 

\textbf{As you said, there is no black and white in history. There is no historical evidence that the Latvian Legion, the \textit{Waffen-SS}, participated in the crimes against Jews. There is evidence about the \textit{Sonderkommando Arājs}, but there is no evidence about \textit{Waffen-SS}, so you cannot condemn them.}

\textbf{Michał Bilewicz:} This is an interesting thing because it's also part of the debate in Poland: We had, for example, a part of the Polish underground in the Naliboki region, which belongs to Belorussian Lithuania, close to Vilnius, Nowogródek, Daliboki. These were areas where the Polish Home Army, which was the main Polish underground force for some time, collaborated with the SS. They were fighting against the Soviet partisans under Nazi occupation. This was a part of the history of Poland where some Polish troops were actually fighting on the side of the Axis against the Allies, which is something you cannot understand if you went to school in Poland. You always thought that Poland was on the side of the Allies, not of the Axis, but the history gets complicated when you dig deeper into it and suddenly you can find the reasons why they did it. 

\textbf{There are some people that believe that this \textit{Waffen-SS} was fighting for the independence of Latvia. I think that it’s not true because restoration of the independence of Latvia wasn't part of the plans of the Nazis. Actually, we had one participant which decided to leave our group because she didn't agree on this formulation.} 

\textbf{Michał Bilewicz:} This is a problem. In Poland right now, we try to use the term ``Nazi Germany'' or ``German camps'' and so on, but I think that using the term ``Nazi''' is very essential here because the Nazi system of understanding was of course imposed by Germany in Europe, but it was very clever - depending on the meaning of ``clever''. They were very instrumental in using the ambitions of other nations, so even if we know now that the Germans did not have the plan to establish independent Latvia and we know now that the Germans didn't have a plan to establish independent Ukraine, I am really confident that at least for some time, they were very good at giving the impression that if Ukrainians collaborate, they will get their own statehood. This is why there were many Ukrainians collaborating with the Nazis. In Poland, we have of course this memory of crimes committed by the Banderist Ukrainian Insurgent Army, \textit{UPA}. In the end, the Nazis imprisoned all the leaders of \textit{UPA} and they killed many of them, but in the beginning they created some forms of collaboration. There was \textit{SS-Galizien}, another part of the \textit{Waffen-SS}, that was composed of Ukrainians. These Ukrainians really believed that this could be a way to establish Ukrainian statehood after being a part of Poland. They really wanted to have their statehood for a long time and their leaders were sent to prison in pre-War Poland. We had Bereza Kartuska and Brześć, we had two notorious camps for the leaders of the Ukrainian nationalist movement. The more you look at this history, the more you see the reasoning behind all actions of people. Even if you take this Nowogródek Home Army, the Polish underground who was collaborating with Germany for some time and fighting against the Soviet partisans, you realise that they do this after they were betrayed by the Soviet partisans. The Soviet partisans didn't want to support them in their fight and this was like a response. It's good to analyse that, to analyse the motivations, and not to be very simplistic in the way you analyse history. 

\textbf{Would you agree with Jan Gross that during the Second World War, Poles killed more Jews than Germans?}

\textbf{Michał Bilewicz:} This is a very provocative statement when you say it like that, but then if you go into details, you understand that it's not that provocative. How easy was it to kill a German under German occupation in Poland? What would happen if you would kill a German soldier in Zamość in 1941?

\textbf{Nothing good.}

\textbf{Michał Bilewicz:} Probably a whole lot would happen. Probably not only your family would be killed, probably all your neighbours would be killed, and half of the people from Zamość will be sent to Germany. And Zamojsczyzna knows quite well what it means to lose all your children being sent to Germany, to become Germanised. All these kinds of things happened, repressions for killing German people were extremely harsh. It was a very hard thing to do. Now, even if most people would really love to do that, they would not do it because of fear. If you have a minority of the population which really would like to benefit from betraying Jews or even killing them for money, how easy was it to do that? It was very easy. First of all, you could even get money from Gestapo for doing that and you knew this was not a problem. Jews were not considered humans by the Nazi system. If you think about it that way, it’s not surprising and it doesn’t tell a lot about Polish attitudes. Before the Warsaw Uprising, the whole Polish resistance was waiting for the good chance to start the general uprising in the whole country. Knowing that their forces were so weak compared to the Nazis, they were just waiting. Then, the Warsaw Uprising started. This was the moment when you had a lot of struggle between the Polish underground and the Nazis. Before that, they rarely did that because they knew that the repression will not target partisans, the repression will target civilians. This shows the complexity of the situation. I expect the numbers that Jan Gross provided were correct. They were based on estimations by Margins Ayemba, who is probably one of the best Polish historians today. It’s good to understand the source of these numbers.  

\textbf{We’re also very much interested in what you can do against anti-Semitism. You mentioned that the more highly educated people are and the more exposure they have towards different kinds of people, the more willing they will be to accept bad aspects of history. Are they also less anti-Semitic, for example?} 

\textbf{Michał Bilewicz:} Yes. We always find a tight relation between education and anti-Semitism. The more educated people are, the less anti-Semitic they are and the more tolerant they are in general. It’s normal because the more you are exposed to any knowledge, like being exposed to other ethnic groups, you'll learn that by being more educated you get more information. Some of you travelled to Israel and having travelled to Israel you realize that Jews are not people with hats, beards, and prayers, and all this outlook which probably your great-grandfather told you about, but they are just normal people like we are. This is the process of changing your knowledge structure from being very simplistic to being more complicated. Then you realize, some of them look like that, some of them wear those beards and so on, which you'll probably see in Jerusalem, but if you go to Tel Aviv, there are these fit young people who are like everybody everywhere else in Europe. 

\textbf{But from my experience, I can also say that in Latvia there is the highest percentage of people with higher education and still, these stereotypes and prejudices about Jews live in the society and you cannot reduce them.} 

\textbf{Michał Bilewicz:} The point with countries like Latvia and Poland and eastern parts of Germany is that these areas are ethnically very homogeneous. This is the problem: You can be well-educated, but if you are educated in a very homogeneous environment...

\textbf{Actually, Latvia was never homogeneous.}

\textbf{Michał Bilewicz:} Ok, there were Russians and Latvians.

\textbf{Not only, there were also many Jews, Poles and Germans.}

\textbf{Michał Bilewicz:} I see. I know that in the past, the best way in Poland to a get good education was to go to Tartu, to Dorpat. The Tartu University was the best university in this whole part of Europe I know that there were lot of Poles going there to get their education. But those people who were raised in the Soviet times, I don’t think that they had so much exposure to any otherness. This is certainly the case for Poland and Germany: If you look at this east-west dimension in Germany, it’s also quite true that people are mostly prejudiced and mostly Islamophobic and mostly anti-immigrant in those areas. I mentioned to you those mountains of Erzgebirge because I went cross-country skiing two years ago there. I was really impressed because this was a beautiful area and there are nice mountaineers, who have their own tradition and culture. It’s a fascinating place, but on the other hand, they are very homogeneous, there are only white, German protestants, maybe secular in a way, but in the mountains not so much. You cannot see any people looking differently or having a different culture there. When you see the results of far right-wing parties there, this is probably the reason for that. It's the same if you look at Poland, in the more heterogeneous  places like bigger cities, you will have much less popularity of far right-wing parties. I remember being in Tomaszów Lubelski, maybe ten years ago. On the streets in Tomaszów Lubelski, I saw posters everywhere of \textit{Młodzież Wszechpolska}\footnote{English: ``All-Polish Youth'', an ultra-nationalist organisation forming part of the National Movement party}. I was surprised because in Warsaw, I didn’t see any posters like that, and I realised that this is the difference: In the Poland I know, which is the Poland of, Warsaw, Wrocław, Gdańsk, maybe Kraków, I don’t see much of that because there is a lot of diversity. You just hear English language spoken here at this university and if those people see a poster of \textit{Młodzież Wszechpolska}, which spreads hatred, they will just take this poster down. It would not be attractive for them. Now, if you go to Tomaszów Lubelski, I don’t think there would be any people who would say, ``this is bullshit, take the poster down'', because they don’t know anything about what it means to be African, to be Muslim. They will be afraid of meeting Muslims and such parties will get popular.

\textbf{The Germans during the Nazi time weren’t uneducated, right? They also knew about maths, they knew literature, some of them were very well-educated. And still they were able to commit something as horrible as the Holocaust. So, my question is, what makes a \textit{good} education? Because, obviously, there were educated people that still were very evil.}

\textbf{Michał Bilewicz:} That’s a big question because basically we see that education makes people more open-minded and you can say the same about this heterogeneity. Some of these Nazi criminals, like Adolf Eichmann, for example - he had many Jewish friends, he was friends with the president of the Jewish community in Budapest, he even spoke some Hebrew and some Yiddish. And he was the person that designed the whole destruction of the Jews, he designed the whole logistics of the Holocaust. Of course, you have many well-educated people holding doctoral degrees, which was very rare at that time. The biggest criminal in the history of my city was Oscar Dirlewanger. He created the Dirlewanger Brigade, which was part of the SS, a brigade of former poachers and hunters. They were skilled because they were hunters, they were poachers before, they knew how to fight in the forest very well and they were used to fight partisans. When the Warsaw Uprising happened, they were sent to Warsaw to kill the insurgents, the partisans who were fighting in the city during the Warsaw Uprising. In fact, they were just killing civilians. They made this huge slaughter of Wola, a region of Warsa. They came from house to house, they were raping all women and they were killing all men whom they would see in that whole part of Warsaw. We are talking not even about Jewish people, they were killing Polish people. And Oscar Dirlewanger held a doctoral degree, he was a political scientist before. So, you're right that not all education makes you moral and open-minded. I think that even Goebbels was holding a doctoral degree. There were more of them holding doctoral degrees among the top Nazis, Mengele, too, who was a medical doctor. Now, there is this big question for me and I try to pass it on to my students as far as I can: There is always this question of how you understand your responsibilities and your duties if you are a scientist. The case of Mengele is a good example because he was thinking that he’s actually doing medical research. He was thinking: ``I have these Jewish inmates of the Auschwitz Camp and I can do experiments on them that are normally considered unethical. Now, nobody will check whether I do ethical or non-ethical experiments. So, if I can do these experiments on twins, for example, I can discover some very important patterns about human physiology that maybe later on could cure thousands of people. Just because I killed these 10 or 15 Jews or Gypsies.'' This is a kind of reasoning which is a big risk of studying science and of doing research. At some point, you could say: ``I can make the world a better place. But to make the world a better place, I have to waste the life of some people.'' This is something that we have learned from the Holocaust as well. This is a trick that really works for some very well-educated people. This is something that we have to be really cautious about. Many well-educated people in Germany were attracted by Nazi ideology because Nazi ideology claimed to be based on science. ``First, we are  looking at correlations, we are doing some statistics and are looking at how different diseases are distributed in a population. We say that among Slavic people, - Poles, Czechs, Slovaks, Ukrainians, Russians - you find that there is a significantly higher level of some diseases than among Germans, Austrians, Swiss people, and Dutch people.'' This is their discovery and then, there’s somebody who tells you: ``Now, what can we do to make sure that our population will not have these diseases? Maybe we should do something that they don’t marry these Slavic people, because it will be bad for the health of our children, and we really want to have healthy children.'' Today, it sounds awkward, but if the Holocaust had never happened... This was the way people would think in a scientific, reasonable way, this was also the way Americans were thinking: When people were trying to get from Europe to America, in the United States they had quotas of how many people they would let in from different countries. So, if they knew there were more diseases in Ireland or if there was a higher criminality rate in Ireland, they would say: ``We want to lower the quota of immigration from Ireland, we don’t want Irish people to come''. There was Ellis island, where they were selecting people, and they’d say, ``Irish people, go back. Jewish people – maybe we can take them, because statistically, they are quite smart. They are quite criminal, but - okay, let’s take them. Polish people, they’re quite healthy, hard-working, we can take them - German people, we can take them.'' It was all based on a statistical analysis of criminality and this way of thinking was very popular in the whole pre-war era, even in democratic societies. Nazi Germany took this whole idea and said: ``Let’s design some more things even, let’s design whom we are going to marry and whom we are not going to marry, then later on design who is going to be alive and whom we will kill.\\
I really recommend you to visit Bełzec. At Bełzec, there is very beautiful monument, very powerful. To me, it’s the best way of commemorating the Holocaust that I have ever seen in Poland. Auschwitz is very brutal, naturalistic, because everything is there, but if you go to Bełzec, you see more of a symbolic meaning. Anyhow, go down to the small museum which is there, and try to learn as much as possible about the biographies of SS troopers in Bełzec. Who were the people that were operating gas chambers, who were the people that were operating the crematoria in Bełzec? They were all from Berlin, they were doctors, people from the programme \textit{Tiergartenstraße 4}, and they were all part of the medical staff of mental hospitals in Germany, like Hadamar and other mental hospitals in Germany. How is it possible that doctors – some of them were doctors, some of them were medical assistants – from a mental hospital somewhere in Germany would end up in Bełzec, killing Jews in a massive way? To understand that, you have to understand the problem with psychology and psychiatry at that time. There was this whole beautiful idea of reform psychiatry. If you take the whole history of psychiatry and psychology, it looked like that: There were people in mental hospitals. Those people were usually just kept there and nobody wanted to cure them. There was no such idea like psychotherapy. It was times before Freud and Freud was this new, fashionable way of thinking. But at that time, no attempts were made to cure these people. They were just retarded, you had to keep them there – and it costs a lot to keep them. The society had to pay money to keep these people in the hospitals. If you read Michel Foucault, this is how mental hospitals operated in the past. Then there was this very bright idea of very progressive people in Germany, who said: ``Let’s make psychotherapy, let’s cure people. The best way to cure people is to ask them to work, that they go out of the hospital and start working in the normal society and then go back to the hospital.'' There is also a combination of therapy with making them work. As you realise that some of these patients can work, it costs even more to make them work and to provide a therapy for them and you are very happy that after a few years of therapy, they return into society. You had all these patients with a depression, with anxiety disorders and so on, and you cure them. But it costs a lot. Then finally, there are these people to whom you do the same, but you cannot cure them - people with schizophrenia, sometimes people with neurological problems. You realise that you really need money to provide a therapy for those people whom you can cure and you need to save money somehow, and somebody tells you: ``Now, if you cannot cure them, their lives are kind of worthless. Maybe we can create eugenic ways of killing these people whom we cannot help. By killing them, we save a lot of money and we can help much more people who actually can be cured.'' The most progressive people at that time believed that this is what you should do in times where you don’t have sufficient money for the medical system. Once you design this idea, you need to train some medical assistants that can take people and put them into gas chambers and kill them. This is how it all started. Once you have people who know how to do that, you just tell them to think about different populations: ``Now we occupy whole Europe. We know from statistics that some nations have a high level of certain illnesses. Jews are spreading typhus, they have parasites, and they have problems with their mental health as well – it will be better and safer for Europe as a whole if we just kill them.'' It was easy for those people who were medical assistants and who were told: ``You are curing the nation now, not only patients, but you’re curing the whole nation. You can do that.'' This is a very, very dangerous way of thinking, and I consider elements of it to be part of scientific thinking even today. And I’m very afraid of that.

\textbf{Can you give an example?} 

\textbf{Michał Bilewicz:} Today, there is this question in medicine, for example, whether people should be cured from all illnesses and what we should do with patients who cannot be cured. How much should we invest in them? There is this question about genetic improvements of different kinds. The wealthier you are – it’s not the case right now, but it will be in the future -  the more you will be able to plan whether your children will have certain traits or not. There is this whole issue of bioethics today, which touches upon important moral questions: We are kind of okay with the situation in China, where there are constraints on fertility. People can have only two children in China. Nobody would suggest anything like that for Germans, Americans, Canadians, or Swiss people. What’s the difference? Why are we agreeing that this is completely fine in China, while we would probably be really scared if somebody said that Chinese people can have as much children as they want. I think this kind of thinking is present in the works of today as well, that not everybody should have the same amount of children. When we ask these questions in surveys, there is this thing, for example, with immigrants, that we should invest in some form of birth control among immigrants – and many people would say, ``yes, we should do that''.\\ If you turn on Polish TV, you also can hear things like: ``We cannot take more refugees to Poland because if we do, Poland will not be Poland anymore, it will no longer be a Christian country''. This is a way of thinking about people who are in the worst situation you can imagine: They are, say, from Syria, escaping from a war zone, and you don’t rescue them. Even top politicians in Poland were saying things like that. I see elements of this thinking that not every life has the same value, that the survival of some races, some nations or some ethnicities has a lower value and our life has higher value.

\textbf{I have a very different question: What about the freedom of historical science and research in Poland at this moment?} 

\textbf{Michał Bilewicz:} I would say that at the moment there is a full freedom of historical research and any science in Poland. What’s more, we have a very good funding system, I cannot complain. I’m having a co-founded research grant from \textit{Deutsche Forschungsgemeinschaft} and Polish National Science Centre. Looking at this grant, we can see that we have very similar system like in Germany. It's not working on an ideological basis. You can get money for research which is completely against the government’s ideology. Now, we have this new law which was passed by the lower chamber of the Polish parliament on Friday. There was an immediate reaction to that and I don’t think that it will be passed in that form. It’s a very long story. This law was first invented in 2006. They have been trying for ten years to have this law and there always was a clash. It has been voted down by the constitutional tribunal. I really hope it will not succeed because if it does, it will be kind of impossible to do scientific research in some forms and to talk about the things that we are talking about here, for example, and to talk about all these complexities.
I am friends with many people doing Holocaust research in Poland, discovering very negative facts about Polish history. I never heard about any of them having any criminal case in Poland, but of course, there are some pressures. For example, in our research centre, we’re doing research about the levels of anti-Semitism in Poland. We try to measure anti-Semitism in different parts of the country, we're trying to look at the sources of anti-Semitism and at ways to decrease it. It was three times already that we were invited to speak at the Polish parliament’s ethnic minorities’ commission and present our studies. Once, after we presented these results, there was a letter from a group of members of the Law and Justice party, PiS, to the rector of the University of Warsaw, and they asked the rector about the centre – where are they getting money from? How many are they? How much do they get from the Polish budget? and so on and so on. They were putting a little bit of pressure on the chancellor of the university, trying to check what’s going on, signaling that they don’t want these guys to do the research that they are doing. But, to show you how independent our academic institution is, I can tell you that the rector responded, ``any academic at my university can express any views properly, and it is not for me to criticise that''. He stood in favour of academic freedoms at the university. I consider this to be a very strong standard at universities right now, but you never know about the future.

\textbf{Can you tell us more about the research you do on anti-Semitism today, about the results of the research?} 

\textbf{Michał Bilewicz:} There are many studies. Basically, we are distinguishing three forms of anti-Semitism: The first form of anti-Semitism is conspiracy anti-Semitism. This is basically the belief that Jews are organising secret plans to take control over the world, control over the media, and so on. This kind of thinking is very widespread in Poland, more people agree with such statements than disagree with them. The second form of anti-Semitism is something which we always consider to be very historical: Believing that Jews are responsible for killing Jesus, that Jews are using Christian blood for ritual purposes, that they are kidnapping children and so on. We always thought that it’s gone, but it’s still approximately 20\% percent of people in Poland who think that some Jews did that. Different studies find that 15-20\% of the population believe that. We are surprised by that; it’s almost entirely in villages and small towns, you don’t find this in cities, and it’s very much correlated with education, it occurs mostly among people with lower levels of education. The third form of anti-Semitism is what we called secondary anti-Semitism – from German research, \textit{sekundärer Antisemitismus} - which is exactly what we are talking about today: Unwillingness to talk about history, unwillingness to acknowledge any responsibility, and the perception that Jews are exploiting history in order to have some gains. It’s very much related to the idea of a Jewish conspiracy, but in Poland, it's also driven by a sense of comparative victimhood - who was the bigger victim in history? - and also a sense of a lack of acknowledgement, which actually in Poland is a big problem: That, globally, the history of the Second World War and the Polish victimhood was not very well acknowledged. When you go to America and talk about Warsaw Uprising there, for example, nobody knows about Warsaw Uprising. Even in Germany, if you talk about the fate of the Polish people during the war, they don’t really know about it sufficiently. Polish people are aware of the fact that the world does not really know about Polish suffering and therefore there is this fight: ``Why do they know so much about Jewish suffering, while they don’t know about our suffering?'' We did some research on this as well. Once you give Polish people the information that Polish people are recognised, for example if you read statements by Barack Obama or by other big figures to them, saying that Polish people were brave and fighting very bravely with Nazi Germany, that Poland was the only country which had an underground state, and so on, then finally, Polish people are more willing to recognise the Jewish suffering. I think that it's not only that Poles are responsible because they are anti-Semites, but there wasn't sufficient recognition of Polish history for quite a long time in Poland.

\textbf{What about the concept that Poland is a Christ of Nations?}

\textbf{Michał Bilewicz:} This concept comes from an old romantic tradition in Poland. If you read Mickiewicz and romantic poets in Poland, you have this image of Poland as the Messiah of Nations or Poland as the Christ of Nation and the martyrdom of the Polish nation is a way to liberate the world. This was how Mickiewicz was thinking about it: The more you suffer, the more the other people will be liberated by our suffering. I don’t think people will think like that anymore. These are not the good old times when Adam Mickiewicz himself went to Turkey to organise insurgents. Polish romantic poets were going to Bulgaria, Albania, Mickiewicz was also in Italy as well as in France. He went everywhere, he tried to organise revolutions, local fighters for independence. I don’t think we have this messianic idea that we have to help all other occupied nations in the world anymore now, unfortunately, because this was a very nice aspect of Polish culture, that somehow we don’t care what happens with us now, but we want to help others.

\textbf{Kaczyński and Smolensk has nothing to do with it?}

\textbf{Michał Bilewicz:} In the life of Kaczyński, one element has something to do with it because I think late Kaczyński really believed that way. When there was a war between Russia and Georgia, Kaczyński flew to Georgia, he went to the border and he said, ``Me, president of Poland, I’m there, you cannot cross the border''. This was his way of showing that he felt he has to go to Georgia and to step up with Georgia against Russia to show solidarity, even if he loses his life. I remember that at that time, Adam Michnik, the Polish liberal, the chief editor of \textit{Gazeta Wyborcza}, who was very, very much against Kaczyński, said, ``wow, this was so brave and so glorious''. Kaczyński also organised other politicians from the region, the president of Lithuania was also there, and some other political leaders whom he took with him to the Russian-Georgian border. There were some incidents like that. I think that Lech Kaczyński was part of that story. I don’t think that Jarosław Kaczyński would do things like that anymore. Lech Kaczyński did something like that once when he went to Ukraine and took part in the Orange Revolution very actively, supporting freedom of Ukraine at that time. All of this was taken from this idea in Poland that we, because of having the history of suffering, because of being victims, have to help other victims. These are elements of thinking which have been present in Poland for a very long time, but for some reasons, I now think that it’s more or less gone. I think it’s a mission of you who are teachers and you who want to be teachers, to also give your pupils this view on history: If your nation had a history of suffering, or, like in Germany, your nation had a history of crime, it really should make us have a sense of responsibility of what we do to others today. This is a very big challenge for educators.
