\section{Dr Michał Bilewicz}

\textit{Michał Bilewicz (*1980) is an Assistant Professor at the Faculty of Psychology of the University of Warsaw since 2008. \\
He studied sociology and philosophy in Warsaw between 1999 and 2003. His Ph.D. thesis, completed in 2007, carries the title `'Between self-verification and social identity processes: Social psychology of threatened ingroup status'. 
He has worked in New York and Delaware in the USA and in Jena, Germany. \\
His research interests, among others, are conspiracy theories, prejudices, intergroup conflict, the threat of positive social identity and dehumanization, mainly on the example of xenophobia, anti-Semitism and ethnic conflicts, reconciliation mechanisms after genocides, the influence of cognitive mechanisms and public language on the exclusion of minorities.
He is the vice president of the Forum Dialogue Foundation, that is dedicated to fostering the relation between contemporary Poland and the Jewish people. \\
The interview took place in the Faculty of Psychology of the University of Warsaw on January 30th, 2018.}\par
\vspace*{2em}
\textbf{Michał Bilewicz:} You're probably fully aware that you're coming to Poland in a very special time in terms of politics now. And this topic that you want to explore is actually a very big issue and a very sensitive one in the Polish-Israeli-Jewish relations.\\
I'm having a lot of collaborative projects with Germany and we're doing a lot of research actually exactly on the topic that you are working on, for example with Roland Imhoff, who is professor at the University of Mainz,and similar projects with another grant that we're doing with the University in Leipzig, with Immo Fritsche, so there are several researchers who are doing research very similar to mine, and we're collaborating, looking on how Germans, Israelis and Poles are viewing the history of the Holocaust, how they view the responsibility and the reasons, for example, we're looking on explanations of how it was possible that the Holocaust happened. And thorugh representative surveys, big samples in the three countries, we found that in Poland and in Israel, people explain the Holocaust in a very similar way: They would say, the reason for the Holocaust was German anti-semitism and the German culture of obedience, and in Germany, most people would say that it was rather because of the economic crisis, because of social circumstances that Hitler seized power, and this is why the Holocaust took place - they're looking more on the social circumstances rather than on national character. And this is a typical pattern, actually, that everybody, every nation wants to look on its own misdeeds as caused by circumstances and by social factors, while own glories and own victories are seen as rather caused by our personality. It's just the opposite if you're talking about another nation. If you're talking about another nation, then you'd say that the failures of this nation are caused by their personality, but about the successes, you'd rather say that they are caused by the circumstances.\\
Those of you who are from Poland, which city are you from? 

\textbf{We're from Zamość.}

\textbf{Michał Bilewicz:} That's a very interesting area from the perspective of, Polish-Jewish relations. The whole conception of Polish-Jewish relations is a tricky one because it implies certain understanding that Jews are not Poles, which is, which is something located at the core of the discussion that we have now, a debate about the relations between Poland and Israel and the history during Second World War. There is this implied reasoning that the Polish state has to defend the good name of Poland and to show the world that all Polish people were rescuers of Jews. The question is: Actually, Jews were killed here, most of them were Polish citizens. So, why the Polish nation or the Polish state today has to represent the history of some of its citizens, but not others of its citizens? In my opinion, the Polish minister should also represent the interests of those people who were murdered.\\
Zamość is also interesting because of probably the best testimony from the times of the Holocaust written in Polish language, no, it's one of the two best. As the first of them I would consider Calek Perechodnik, ``\textit{Spowiedź}'', ``The Testimony''. Calek Perechodnik was a Jewish person who was part of the Jewish police in Otwock ghetto, he was a collaborator, a Jewish policeman, and he sent his own family to the death camp. He survived until 1944, he was hidden on the Aryan side in Warsaw and then somebody betrayed his hiding place and he was blackmailed and then finally killed by Gestapo. He writes his testimony in 1943/44, and as he writes about his memory, he says: ``The times of the Holocaust is the times when the morality collapsed. Nobody was moral.'' And he said that neither Poles nor Germans - of course Germans were the evil at that time, but he says that Poles were immoral at that time and they betrayed Jews, and Jews were immoral at that time because Jews trusted that everything will be fine and actually collaborated in a way. This is something which is very hard to read today because if you're Polish, it makes you feel uncomfortable, if you're Jewish, it makse you fee uncomfortable, and if you're German, of course it makes you uncomfortable, but I think this is what is the truth about the Holocaust: That nobody can feel really comfortable. There is no way to feel comfortable. The second testimony which is really worth reading is that of Zygmunt Klukowski, who was a doctor working very close to Zamość, in Szczebrzeszyn. Most of the book is actually about Zamość because he was travelling to Zamość, he spent half of the time in Zamość and half of the time in Szczebrzeszyn. Szczebrzeszyn was a small town and he was educated doctor, he wanted to party and to be part of this nice cultural place of Zamość, you know, where there were good cafes and so on, this is where he spent most of his time, and he writes about the reactions of people in Zamość and in Szczebrzeszyn and in Izbica and in the whole area to the Holocaust. And about the problem of people who are sometimes trying to, you know, to earn some money, to earn some property, to get houses or apartments, taking advantage of the fact that their neighbours are being sent to Izbica and then to Bełżec. He's Polish, he's not Jewish, but he writes, how is it possible that this can happen, that now neighbours treat neighbours as an opportunity to loot, an opportunity to get some property? This is very difficult to read, it's a very interesting perspective because it's the perspective of a person who is just there, who's part of the Polish Underground, of \colorbox{red}{name} after the Holocaust, and still asks these questions. It shows that the history is very complex, it’s not black and white. For me as a psychologist, it’s very important to know that we never can use our moral standards to make judgements about what happened during the Holocaust, because it’s so different, and nobody in this room knows how they woulld act if they were placed in such situations.

\textbf{You mentioned that this picture of history and of how the Holocaust happened is different in these three countries. Is this a general thing, shared by all people, or would you say that a certain group of people in each of the countries has a different view?}

\textbf{Michał Bilewicz:} I was talking about a kind of average, the average German, average Pole and average Israeli, and of course we know that average people do not exist. It's a composition of many different groups. In Germany I am sure you are aware of the diversity of the country. If you look at the distribution of attitudes in Germany you find that if you go from the east, from the Saxonian Ore Mountain Range, where you have high popularity of AfD and very anti-immigrant views to, say, the Rhineland, where generally, people are rather open towards immigration, then you realise, that we have very different histories here because those people in Eastern Germany lived in a very homogeneous country for a long time, while the people in the Rhineland were always very open, they always had a lot of contacts with France. There is a lot of different histories, and the same is true for education levels: People who are more educated usually feel more of a sense of responsibility or guilt for what happened in the past and people who are less educated usually are more defensive and would like to have this very idealised image ``We were always good and never did anything bad''. This is about Germany, but talking about Poland it is the same: The more educated people in Poland are, the more willing they are to talk about the dark aspects of the history and not only the glorious ones. Besides, we have the same regional pattern as in Germany, actually, if you go more to the east, people are more defensive, if you go more to the west, people are more open-minded, so paradoxically, there is also this east-west dimension operating in Poland when you talk about history. If you look on research on Lubelskie, where Zamość is loacted, and Podlaskie, these are the two Voivodeships where the willingness to talk about the negative aspects of the history is the lowest. People would rather like to talk about the glorious history and be silent about the negative aspects of history. And if you go to Israel, it's more or less the same, it's not about an east-west dimension there, but also about education, and people who are less educated would also like to perceive Jews as only glorious, not doing any harm, being absolutely good. While üeople who are more educated, they are sometimes acknowledging that there were some crimes against Palestinians, the Arab population, and the willingness to talk about \textit{Nakba}, for example, the moment where Arabs were pushed from Jerusalem, many Arabs were killed and forced to migrate from what was to be Israel. Not everybody in Israel commemorates that.\\
I know very little about Latvia. I know that Latvia has a very traumatic history in this respect, that there was also a lot of collaboration going on in Latvia and even in Poland we have memories of Latvians actually serving in the German troops during the War. But I don't know about the debates in Latvia today, whether there are debates about the collaborationist parts of SS. 

\textbf{There are no broad debates about this topic because people don't like to speak about it, people just want to stick to the image of either heroes or victims. We, the Latvian people, were either victims or heroes,  we didn’t collaborate. There is an official recognition that Latvians participated in the Holocaust, but it is officially said that this people were only a small group, but actually it is not so small - it was shown that the collaboration was very broad. Many Latvians collaborated.} 

\textbf{Michał Bilewicz:} And there are these veterans, the combatants who were part of \textit{Waffen-SS}, who have veteran rights, which from a Polish perspective sounds quite strange. 

\textbf{As you said, there is no black and white in history. There is no historical evidence that this Legion, the \textit{Waffen-SS}, participated in the crimes against Jews. There are evidences about the \textit{Sonderkommando}, but there is not evidences about \textit{Waffen-SS}, so you cannot condemn them.}

\textbf{Michał Bilewicz:} This is an interesting thing because it's also a part of the debate in Poland:That we had, for example, a part of the Polish underground in the \colorbox{red}{unintelligible name} region, today this belongs to Belorussian Lithuania, close to Vilnius \colorbox{red}{unintelligible name}. These were areas where the Polish Home Army, which was the main Polish underground force for some time, collaborated with the SS. They were fighting against the Soviet partisans  under Nazi occupation, and this was a part of the history of Poland where some Polish troops were actually fighting on the side of Axis against the Allies, which is something you cannot understand if you always thought that Poland was on the side of Allies, not of Axis, but the history gets complicated when you dig deeper into it, and suddenly, you can find the reasons why they did it. 

\textbf{There are some people that believe that this \textit{Waffen-SS}, was fighting for the independence of Latvia. I think that it’s not true because restoration of the independence of Latvia wasn't part of the plans of the Nazis. Actually, we had the, one participant which decided to leave our group because she didn't agree on this formulation.} 

PROCEED HERE

Michał Bilewicz: Yeah, but this is a problem because maybe, I mean the point is that the Nazi system, so this is why I mean in Poland now we try to use the term Nazi Germany or German camps and so on, but I think that using term Nazi is very essential here because Nazi, system of, system of, of, of understanding, I mean this was not only, I mean of course it was imposed by Germany in Europe, but it was a very clever, depending on the meaning of clever, but, but they were very how to say, instrumental in using ambitions of other nations, 

Peter Zinke: Yes. 

Michał Bilewicz: so even if we know now that Germans did not have plan to establish independent Latvia and we know now that Germans didn't have a plan to establish independent Ukraine, I am really confident that at least for some time they were very good in making an impression that if Ukrainians collaborate, they will get their own statehood. And this is the reason why Ukrainians, there were really many Ukrainians collaborating with the Nazis, and in Poland, we have of course this memory of crimes committed by Banderist [after Stepan Bandera] Ukrainian Insurgent Army UPA

David Gurevich: Yeah. 

Michał Bilewicz: And, you know, this, this, these were forces, these were troops and UPA was, in the end, I mean, Nazis imprisoned all the leaders 

David Gurevich: Yes. 

Michał Bilewicz: and they killed many of them, the Germans, yes, but, but in the beginning they created some forms of, of collaboration. There was SS-Galizien, yes, another part of SS, Waffen-SS that was composed of Ukrainians and these Ukrainians really believed that this is a way to establish Ukrainian statehood after being part, you know, after being part of Poland and they really wanted to have their statehood for long time and their leaders were also sent to prison by, in the, by prewar Poland, yes, we had \colorbox{red}{unintelligible name}, we had two notorious camps actually for the leaders of Ukrainian nationalist movement. So, the more you look at this history, the more you see reasoning behind all actions of people and also even if you take this \colorbox{red}{unintelligible names}  Home Army, so the Polish, Polish underground who was collaborating with, with Germany for some time and fighting against the Soviet partisans and you realise that they do this after they were betrayed by the Soviet partisans. So the Soviet partisans didn't want to support them in their fight and, and this was like a response to, so yes, it's all getting, you know, so it's good to analyse that and to analyse the motivations and not to be very simplistic in the way you, you analyse the history. 

Peter Zinke: So would you agree with Jan Gross that during the Second World War the Polish people were killing more Jews than Germans? 

Michał Bilewicz: This is a very provocative statement when you say it like that, but then if you go into details you understand that it's not so provocative, ok? How easy was it to kill a German under German occupation in Poland? What would happen if you would kill a German soldier in Zamość in 1941?

\textbf{Nothing good.}

Michał Bilewicz: Probably a whole lot. Probably not only your family will be killed, probably all your neighbours will be killed and people, and, and half people from Zamość will be sent to, to Germany. And Zamojsczyzny [?, ZOOM0008.WAV, ] knows it quite well what does it mean to lose all your children being sent to, to, to Germany, to become Germanised, yes? And I mean, all this kind of things, they, it happened, I mean, repressions for killing German people were extremely harsh, so it was very hard to do. Now, so even if most people would really love to do that, they would not do that because of fear, ok? No if you have a minority of the population which really would like to benefit on betraying Jews or even killing them for money, how easy was it to do that? It was very easy. First of all, you could get even money from Gestapo for doing that and you know this was not a problem, Jews were not considered humans by the Nazi system. So actually if you think about it that way, it’s not surprising and it doesn’t tell a lot about Polish attitudes, if you just say about the number, that there were anti-Wars uprising so we had this Burza action, so the Warsaw uprising was a part, so when the whole Polish residence was waiting for the good chance to start the general upraising in the whole country. But knowing that their forces were so weak comparing to the Nazis they were just waiting. And then Warsaw uprising started, and this was the moment when you have a lot of struggle between Polish underground and the Nazis. But before that they rarely did that because they knew that the repression will not target partisans, the repressions will target civilians. And they were representative civilians. So, this shows the complexity of the situation. And behind of that I probably expected the numbers that Jan Gross said were correct and they were based on estimations by Margins Ayemba who is probably one of the best Polish historians today who make this estimation so I think they are correct but it’s good to understand, what’s the source of these numbers.  

I: We’re also very much interested in what you can do against anti-Semitism for example. So, you mentioned that the more highly educated people are and the more exposure they have towards different kinds of people, the less, well, the more they’re willing to accept bad aspects of history. Are they also less antisemitic for example? 

MB: Yeah. We always find them quietly near relation between education and anti-Semitism. So, the more people are educated, the less they are antisemitic, the more they are tolerant in general. And it’s normal, because the more you know, the more you are exposed to any knowledge, it’s like being exposed also to other ethnic groups, so you know by being more educated you get more information. It’s the same about, you know, some of you travelled to Israel and having travelled to Israel you realize that you know Jews are not people with hats, beards and payers and you know all this outlook which probably your great grandfather told you about, no, they are just normal people like we are, yes? And this is the process of changing your knowledge structure from being very simplistic to being more complicated. And then you realize, some of them look like that, some of them [...] beards and so on which you see probably in Jerusalem, but if you go to Tel Aviv this is you know fit young people who are you know like everybody everywhere else in Europe and this is the complexity, so this is something that you also get from good education, that you learn, that the world is not simple, it’s complex. 

I: But just, from my perspective, from my experience, I can say this [...] Latvia, I think there is the highest percentage of people with higher education, but those stereotypes and prejudices about Jews still live in the society, still live among people and you cannot deal with it, you cannot reducate it, just leave there. 

MB: Yeah, the point with countries like Latvia and Poland and Eastern parts of Germany is that these areas are very ethno homogeneous. And this is this problem. That even you can be well educated, but if you are educated in a very homogeneous environment, so- 

I: Actually, Latvia was never a homogeneous… 

MB: Ok, it was Russian and Latvian… 

I: Not only, many, many, also Jews, Poles and Germans… 

MB: Ah ok. I mean in the past, I know, that in the past this was the all, you know, best educated, the best way in Poland to get good education was to go to Tartu, yes, the Dorpat. The Tartu University was the best university in the whole part of, this whole part of Europe, where, so I know that there were lot of Poles going there to get their education. But I think that those people who were raised in the Soviet times, I don’t think that they had so much exposure to any otherness, you know, and, maybe not, maybe that’s not true, but certainly about Poland and certainly about Germany if you look on this East-West dimension in Germany, that’s also quite true that, you know, people are mostly prejudiced and mostly Islamophobic and mostly anti-immigrant in those areas – I mentioned to you those mountains of Erzgebirge because I went cross-country skiing two years ago there, and I was really impressed because this was, you know, a beautiful area, and nice mountaineers, who have their own tradition and culture, it’s a fascinating place, but on the other hand, they are very homogeneous, there are, you know, like only white, German protestants – maybe secular, in a way, but in the mountains not so much – they are really like you cannot see any people looking different there, looking in a different way or having a different culture. And then, when you see the results of far right-wing parties there, this is probably the reason for that. And also if you look on Poland, the more heterogeneous place, which are like bigger cities, you would have much less popularity of far right-wing parties. And you, if you go – I could name (?) some, I remember going, maybe ten years ago, I remember being in Tomaszow Lubelski. And on the street in in Tomaszow Lubelski, everywhere I saw posters of Młodzież Wszechpolska. And I was surprised because in Warsaw, I didn’t see any posters like that, and I realised that this is this difference that, you know, in Poland, in Poland that I know, which is the Poland of, you know, Warsaw, Wrocław, I don’t know, Gdansk, maybe, Krakow, I don’t see much of that because there is a lot of different, you know, you just hear English language here spoken at this department, at this university, and you know, if those (?) people see a poster of Młodzież Wszechpolska, which will spread hatred, these people will just, you know, take this poster down. That would not be attractive. Now, if you go to Tomaszow Lubelski, I don’t think there would be any people who would say, yeah, this is bullshit, take the poster down, because, you know, they don’t know anything about, you know, what does it mean to be African, what does it mean to be Muslim, so, they will be afraid of meeting Muslims, and such parties will get popularit

\textbf{The Germans during the Nazi time, the Germans weren’t uneducated, right? They also knew about maths, they knew literature, some of them were very well-educated. And still they were able to commit something as horrible as the Holocaust. So, my question is, what makes a \textit{good} education? Because, obviously, there were educated people that still were very evil.}

MB: Yes, that’s a, that’s a big question, because basically we see that education makes people more open-minded, and also you can say(?) the same about this heterogeneity: Some of these Nazi criminals, like Adolf Eichmann, for example, he had many Jewish friends, he was friends with the president of the Jewish community in Budapest, he even spoke some Hebrew and some Yiddish. And the, this person was, you know, he designed the whole construction destruction of Jews, I meand he designed the whole logistics of the Holocaust. And of course, you have many well-educated people holding doctoral degrees – which, at that time, was very rare, today, everybody has doctoral degree, well, not everybody, but it’s quite common, now, if you go to the 1930s, it was a big thing. And the biggest criminal in the history of my city was, his name was Dirlewanger – so, the second name, the first name was – sorry, I am not. But he was the head, he headed, he created this brigade, it was in SS, the brigade, Dirlewanger Brigade, I think, it was a brigade of former poachers and hunters, who were skilled, you know, because they were hunters, they were poachers before, they knew how to fight in the forest very well. And they were used to fight partisans. And then, when there was Warsaw Uprising, they were sent to Warsaw to kill the insurgents, to kill the partisans who were fighting in the city during the Warsaw Uprising. But in fact, it turned out that they were just killing civilians, that they made this huge slaughter of Wola, region of Warsaw, so, they came from house to house, they were raping all women and they were killing all men whom they would see in the whole part of Warsaw. And we are talking not even about Jewish people, they were killing Polish people. So, this is the history of Warsaw, and this guy – Otto Dirlewanger? I think it was Otto Direlwanger sure [Oscar Dirlewanger, I looked it up on the internet, MF]. – he had a doctoral degree, okay? So, he was really well-educated, a lawyer or political scientist, I think, before. So, you are right that not all education makes you moral and makes you open-minded, and I think that there were more, I think even Goebbels was holding a doctoral degree, there were more of them who were holding doctoral degrees, in the top Nazi - [Peter Zinke: Mengele, for example] Mengele, also, medical doctor, yes. Now, there is a question, big question for me, and I try to pass it to my students as far as I can, that it’s always a question of how you understand your responsibilities and your duties if you are a scientist. And the example of Mengele is a good example because he was, he was thinking that he’s actually doing medical research, okay. So, he was thinking, if I have these Jewish inmates of Auschwitz camp, and if I can do experiments on them that are normally considered unethical, but now, nobody will check whether I do ethical or non-ethical experiments. But I can do that. [End of video pt one] So, if I can do these experiments on twins, for example, I can discover some very important patterns about human physiology, you know, that maybe later on could cure thousands of people. Just because I killed these ten or 15 Jews or Gypsies. And this is a kind of reasoning which, if you think in a scientific way, this is a big risk of studying science and of doing research. Because at some point, you could say, “Okay, I can make the world a better place. But to make the world a better place, I have to waste life of some people. And this is something that we have learned from the Holocaust as well, that this is a trick that really works for some very well-educated people. And this is something that we have to be really cautious about. And this is why many well-educated people in Germany were attracted by Nazi ideology. Because Nazi ideology said, “We are based on science”. Okay, first we are doing […] correlations, we do some statistics. We are doing statistics and are looking at how different diseases are distributed in a population. And we say, okay, among Slavic people, like, you know, Poles, Czechs, Slovaks, Ukrainians, Russians, you find that there is significantly more of some diseases than among Germans, Austrians, Swiss people, and Dutch people. So, this is a discovery, you find it. And then there’s somebody who tells you, “Okay, now, what can we do to make sure that our population will not have these diseases. Maybe we should do something that they don’t marry these Slavic people, because then it’s bad for the health of our children, and what we care is health for children, yes, and we want really to have healthy children. So, this is kind of – I mean, today, it sounds awkward, but you know, if Holocaust never happened, this was the way scientific people would think in a scientific, reasonable way, this was the way Americans were thinking, so when people were trying to get from Europe to America, America had quotas, in the United States, of how many people they would from different countries. So, if they knew there were more diseases in Ireland, or if they there was a higher criminality rate in Ireland, they would say, “We want lower quota of immigration from Ireland, we don’t want Irish people to come”. And there was this island, Ellis island, yes, where they were selecting people, and they’d say, “Okay, Irish people, go back. Jewish people – yeah, maybe, we can take them, because statistically, they are quite smart. They are quite criminal, but […], okay, let’s take them. Polish people, hm, yeah, they’re quite healthy, hard-working, we can take them, German people, we can take them. So, this was all based on statistical analysis of criminality and this whole way of thinks was very popular in the whole pre-War era, in even democratic societies. And this was – Nazi Germany took this whole idea and said, “Yeah, okay, let’s make, let’s design something more, let’s design whom we are going to marry, whom we are not going to marry. Then, later on, design who is going to be alive and whom we will to kill. Also, we are now in a very special building, which is now the Department of Psychology, and I’m teaching students of psychology, because there should also be a worry about how we understand psychological illness and mental health. Because this was also very important in thinking about the Holocaust. You are, some of you are from Zamość – have all of you been to Zamość? […] You’re going today. If you are in Zamość, I really recommend you to visit also Bełzec. […] Ah, you will do that, okay. Once you go to Bełzec, there is a monument which is a very beautiful monument, very powerful, I would say it’s probably – to me, it’s the best way of commemorating the Holocaust that I have ever seen in Poland. It’s more powerful – I mean, Auschwitz is very brutal, naturalistic, because everything is there, but if you got to Bełzec, you see it’s more of a symbolic meaning, which is – but, you go down to the small museum which is there, and try to learn as much as possible about the biographies of SS troopers in Bełzec. Who were people that were operating gas chambers, who were people that were operating crematoria in Bełzec. […] They were all from [Berlin], they were [doctors]. They were people from the programme “Tiergartenstraße 4”, the programme of medical – they were part, they were basically part of the medical staff of mental hospitals in Germany, like Hadamar and other medical, other mental hospitals in Germany. How is it possible that doctors – some of them were doctors, some of them were more like nurses, I don’t know if you - male nurses, yes, medical assistants – from a mental hospital somewhere in Germany, how could they end up in Bełzec, killing Jews in a massive way. To understand that, you have to understand the problem with psychology and psychiatry at that times. Because there was this whole beautiful idea of the reform psychiatry. There idea of reform psychiatry said – so before, if you take the history, the whole history of psychiatry and psychology, so, it looked like that: There were people in mental hospitals. Those people were usually just kept there, and nobody wanted to cure them. So, there was no idea like psychotherapy – didn’t exist. It was times before Freud, and Freud was this new, you know, new fashionable way of thinking. And at the time people made no – you did not cure them. They were just retarded you have to keep them there – and it costs. It costs a lot to keep them. So, the society had to pay money to keep this people in the hospitals. And this is like, if you read Michel Foucault, this is how in the past, mental hospitals operated. And then there was this very bright idea in Germany, of very progressive people, who said, “okay, let’s make psychotherapy, let’s cure people”. The best way to cure people is to ask them to work, that they work, they go out of the hospital, and they start working in the normal society. And then, they go back to the hospital. And there is a combination of therapy with making them work. And then, realise that some of these patients can work, it costs even more to make them work and to make therapy for them, and then you’re very happy because after a few years of therapy and, you know, after this thing that you do, they go back to the society. You had all these patients with a depression, with anxiety disorder and so on, and then, you cure them. But it costs a lot. And then finally, there are this people, you do the same to them, but they are not. You cannot cure them. People with schizophrenia, sometimes, with some forms of schizophrenia, sometimes people with neurological problems – you cannot cure them. And you realise that you really need money to make therapy for those people whom you can’t cure, and you need to save money somehow, and somebody tells you, now, if you cannot cure them, their lives are kind of worthless, yes, it’s like these worthless lives. So, what should we do? And the, somebody say, yes, maybe we can create these very eugenic ways of killing these people whom we cannot help. And then, by saving money, by killing them, we save a lot of money, we can help much more people who can be cured. And the most progressive people at that time believed this is the way you should do in the times you don’t have money. If you don’t have sufficient money for the medical system, this is the thing you should do. So now, once you design this idea, you need to train some medical assistants that they can, you know, just take people and put them into gas chambers and kill them. And this is how it all started, and this is the whole training, that once you have people who know how to do that, you just tell them, okay, now think about different populations, now we occupy whole Europe. And we know statistically that some nations have high level of different illnesses. Jews are spreading typhus, they have parasites, and they have problems with their mental health as well – it will be better and safer for Europe as a whole if we just kill them. I mean, this was easy for those people who were medical assistants, who were assigned different tasks, and who were told, “you are curing the nation now, not only patients, but you’re curing the whole nation. You can do that.” This is a very, very dangerous was of thinking, and I consider it to be a part of scientific thinking even today. There are ways when we think, okay, we should improve – I mean, there are elements of this thinking even today. And I’m very afraid of that, but this is a way…this is a way of thinking which is not completely gone from us, unfortunately.

\textbf{Can you give an example?} 

MB: Of, of…The programme of euthanasia, because we’re talking about the euthanasia programme…Today, this is a question of, for example, of medicine and whether people should be cured from all illnesses. And what should we do with patience who cannot be cured, how much should we invest in them. There is a question now about the genetic improvements of different kind. So, for example, should you be able to control your, you know…The more wealthy you are today – I mean, it’s not a situation right now, but in the future, the more wealthy you are, the more you will be able to plan whether your children will have certain traits or not. And there is a whole problem of bioethics today, which, I mean, there are questions, important moral questions of whether, okay, we are kind of okay with the situation that in China, there are constraints with the fertility, so people can have only one – no, I think two – children in China, and nobody would suggest anything like that for Germany, Americans, Canadians, Swiss people, “Okay, have two children and not more”. What’s the difference? Why are we agreeing that this completely fine in China, and probably we would be really scared if somebody would say, “No, that’s fine, Chinese people can have as much children as they want”. And then, we have billions of Chinese people everywhere. And this is the whole…I think, this kind of thinking is present in the works of today as well, that you know, not everybody should have the same amount of children. When we ask these questions in surveys, there is this thing, for example, with immigrants, that we should control the fertility of immigrants. That we should, you know, invest in some form of birth control among immigrants – many people would say, yes, we should do that. Now – why? I mean, this is, so, this kind of thinking in a demographic way, that, you know, we don’t want Germany to be…to be Syria. This kind of thinking, limiting…this is something that you see everywhere. If you turn on Polish TV, you also can hear about, you know, we cannot take more refugees to Poland, because if we take more refugees, Poland will not be Poland anymore, it will not be a Christian country. We will have all these Muslims, they will rape Polish women and…This is things that you can hear in Polish television. So, this is also a way of thinking about some people who are in the worst situation you can imagine, because, you know, they are from Syria, escaping from a war zone…and, you know, you don’t rescue them. You say, “Okay, we can rescue them there, rescue them in Syria”. You know, if there’s a war going on there. It’s impossible! But then, this is a kind of – even top politicians in Poland were saying things like that. I see elements of this thinking that not every life has the same value, that some races, some nations of some ethnicities, that their survival has lower value, and our life has higher value. [bilewicz06, 00:13:33] 

I: I have a total different question. What about the freedom of historical science and research in Poland at this moment? 

MB: I would say that at the moment, there is a full freedom of historical research and any science in Poland. What’s more, we have a very good funding system, so, I cannot complain. I’m having a co-founded research grant from Deutsche Forschungsgemeinschaft and Polish National Science Centre, and looking at this grant, we can see that we have very similar system like in Germany. In Germany, you have research funding system, in Poland you have research funding system, working in a very similar way, not ideological, you can get money for research which is completely against the government’s ideology. But, […] exactly, there are tendencies, which are always confronted – which I think it’s very good that they are confronted. Now, we have this new law which was passed by the law chamber of the Polish parliament on Friday, there was immediately a reaction to that, and I don’t think that it, in that form, will be passed. It’s a very long story, this law was first invented in 2006, so, they are trying from ten years to have this law, there’s always, you know, a clash. It has been voted down by the constitutional tribunal. So, this is how it is…And I really hope it will not succeed, because if it does, it will be kind of impossible to do scientific research in some forms and to talk about things that we are talking here about, for example, to talk about all these complexities. [bilewicz06, 00:15:15] 

I: I’m asking because Jan Grosz has been interrogated for five hours by the state, they told me that there is a criminal case against him. 

MB: Really? I didn’t know that, but this is…But I think that was not after his research, but after this interview that he did. The interview that he did where he said – what did he say? It was not about history… [bilewicz06, 00:15:45] 

I: He said that the Polish people in the Second World War killed more Jews than… 

MB: Ah, it was about this. I don’t think…But these numbers were quite correct, so…I don’t…It’s a signal, probably it’s a signal…It’s a criminal case? […] Oh, really. But that’s, yeah, that’s not something…I mean, I am friends with many people doing Holocaust research in Poland, and discovering facts which are very, let’s say, negative about Polish history, and I never heard about any of them having any criminal case in Poland. But of course, there are some pressures. For example, I’m…our research centre, we’re doing research about the levels of anti-Semitism in Poland, for example, we are to measure anti-Semitism in different parts of the country, we are trying to look what are the sources of anti-Semitism, what are the ways to decrease anti-Semitism. And, a few times – it was three times already that we were invited to the parliament, to speak at the Polish parliament’s commission, ethnic minorities’ commission, and present our studies. And once, after we presented these results, there was a letter from the members of…from a group of members of the Law and Justice party, of PiS, to the rector of the University of Warsaw, and they asked the rector, what does the Centre – where are they getting money from? How many of them are they? How much do they get from Polish budget? and so on and so on. So – a little bit of making pressure, on the rector, on the chancellor of the university, trying to check what’s going on, we don’t want these guys to do the research that they are doing. But, to show you how independent is our academic institution, I can tell you that the rector responded that any academic at my university can express any views properly, and this is not of my thing to criticise that, so he stood in favour of academic freedoms at the university, and I consider this to be a very strong standard at universities right now, but – you never know about the future. You never know what happens if Frauke Petry becomes the chancellor of Germany, or if our government becomes more and more radical. And these are things which you cannot really predict, and this is hopefully not going to happen, but the risks are always there. [bilewicz06, 00:18:04] 

I: Can you tell us more about the research you do on anti-Semitism today, about the results of the research? 

MB: Oh, there’s a lot of studies, and basically, we are distinguishing three forms of anti-Semitism, so, first form of anti-Semitism is conspiracy anti-Semitism, believing in Jewish conspiracy. So, this is basically the belief that Jews are organising secret plans to take control over the world, control over the media and so on. And this kind of thinking is very widespread in Poland, so more people agree with such statements than disagree with them. The second form of anti-Semitism is something which we always consider to be very historical, believing that the Jews responsible for killing Jesus, and that Jews are using Christian blood for ritual purposes, kidnapping children and so on. And we always thought that’s it’s gone, but it’s still approximately 20% percent of people in Poland who think that some Jews did that […]. Up to 20%, different studies who find 15-20% of the population that believe that. We are surprised by that, it’s almost entirely in villages and small towns, you don’t find this in cities. And it’s very much correlated with education, with the people with lower education levels. But it still exists in Poland, but it’s smaller, of course. The third form of anti-Semitism is what we called secondary anti-Semitism – from German research, sekundärer Anti-Semitismus. Which is about exactly what we are talking about today. Unwillingness to talk about the history, unwillingness to acknowledge any responsibility. And perception that Jews are exploiting history in order to have some gains, and it’s not – that it’s all not about history, that it’s just, you know, that it’s just Jewish [?, 20:01]. And it’s very much related to an idea of Jewish conspiracy, but it’s very much driven in Poland by a sense of comparative victimhood- who was the bigger victim in history? And, also a sense of acknowledgement, which actually in Poland is a big problem that, globally, the history of Second World War and the Polish victimhood was not very well acknowledged. When you got to America, and you talk about Warsaw uprising, for example, nobody knows about Warsaw uprising. If you – even in Germany, if you talk about the fate of the Polish people during the War, they don’t really know sufficiently. And Polish people are aware of the fact that the world does not really know about Polish suffering and therefore, there is this fight over, like, you know, “Why do they know so much about Jewish suffering, and they don’t know about our suffering?” And we did some research on this as well, that once you give Polish people information about the fact that Polish people are recognised, so if they read, for example, statements by Barak Obama, by other big figures, that, you know, Polish people were brave and fighting very bravely with Nazi Germany, that Poland was the only country which had an underground state, and so on, the finally Polish people are more willing to recognise the Jewish suffering. I think there is some issue here which is – Poles are not only responsible because they are anti-Semites, but there was not sufficient recognition of Polish history for quite long in Poland, and I think that then, in Poland, there is this sense of we didn’t get recognised sufficiently. [bilewicz06, 00:21:42] 

I: What about the concept that Poland is a Christ of Nations? […] 

MB: Yeah, there is this concept that, you know, by suffering…I mean, it’s an old romantic tradition in Poland, like from, if you read Mickiewicz and romantic poets in Poland, you have this image of Poland as the messiah of nation, or Poland as the Christ of Nation, so the martyrdom of Polish nation is a way to liberate the world. Okay, so this was how Mickiewicz was thinking about it, the more you suffer, the more, actually… You know, it’s like a Christ, the other people are…will be liberated by our suffering. So, [incomprehensible, 22:40] I don’t think that is so much anymore now, and Poles don’t care so much about other nations liberation today. So, if it was true that much people say, okay, because we are suffering so much, then, Arab people are liberated. I don’t think people will think like that anymore, these are not the good old times where Adam Mickiewicz himself went to Turkey to organise, you know, insurgents – you know, Polish romantic poets were going to Bulgaria, Albania, Mickiewicz was also – I think he was in Italy as well, in France – everywhere he went, he tried to organise revolutions, local independent fighters to fight for independence, and I don’t think it’s so much now anymore, we don’t have this messianic idea that we have to help all other occupied nations in the world - unfortunately, because this was a very nice aspect of Polish culture, you know, that somehow, we don’t care what happens with us now, but we want to help others, you know. And- [bilewicz06, 00:23:44] 

I: Kaczyński and Smolensk has nothing to do with it? 

MB: Kaczyński – in the life of Kaczyński, one element has something to do with it, because I think late Kaczyński really believed that way. He’s [?, 23:55] in Georgia. When there was a war between Russia and Georgia, Kaczyński flight to Georgia, he went to the border, and he [?, 24:01] “Me, president of Poland, I’m there, you cannot cross the border. And this was like his way of, exactly, like showing, now I have to go to Georgia, and to state up with Georgia against Russia to show our solidarity, even if I lose my life. And this was really, at that time, I remember, Adam Michnik, the Polish liberal, you know, the chief editor of Gazeta Wyborcza, was very, very much against Kaczyński,  and a that time he said, “Wow, this was so brave and so glorious”, and Kaczyński also organised other politicians from the region, there was also the president of Lithuania and some other political leaders whom he took with him to the Russian-Georgian border. So, there were some incidents like that, so I think that Lech Kaczyński, he was part of that story, I don’t think that Jarosław Kaczyński would do things like that anymore. Alexander Kaczyński did something like that once, when he went to Ukraine, and he took very active part in the Orange Revolution, and you know, supporting freedom of Ukraine at that time. So, all of this was taken from this idea in Poland that we, because of having the history of suffering, because of being victims, we have to help other victims. These are elements of thinking which have been present in Poland for very long time, but for some reasons, I now think that it’s more or less gone. And I think it’s a mission of you who are teachers and you who want to be teachers to also give your pupils this view on history that if your nation had a history of suffering, or, like in Germany, you nation had a history of crime, were perpetrators, it really should make us have a sense of responsibility of what we do to others today, and also maybe to help others who are in a [?, 25:58] situation today. So, this is a very big challenge for educators. [bilewicz06, 26:03] I think that I need to finish because there will be some exam in this room, and I heard that you need to leave as well at quarter past four, but I think it’s great that you came, and I hope that the rest of your trip will be great, and remember to read in Bełzec the biographies of the SS who were operating the camp. 