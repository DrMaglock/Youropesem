\section{Vitali Liberov}

\textit{Mr Liberov will introduce himself in the course of the interview, which took place in Nuremberg on March 8th, 2017. The interview was conducted in Russian and translated into English for this publication.}\par
\vspace*{2em}
\begin{otherlanguage}{russian}
	\textbf{Не могли бы Вы рассказать о себе?} 
	
	\textbf{Виталий Либеров:} Меня зовут Виталий Либеров. Мне сорок лет. Я – частный предприниматель. Я родился в России, в Брянске. Я рос как самый обыкновенный советский ребенок, я был пионером, затем вступил в комсомол, поэтому я не был особо знаком с религией. Но я помню: пока я рос, несколько раз в год мужчины в нашей семье куда-то вместе собирались и уходили, шептались о чем-то, возвращались домой с большими кусками мацы, завернутыми в газету, а также другие вещи. Однако это все было настолько странно, что я, как ребенок, не обращал на это внимания. Первый раз, когда я осознал, что я был евреем, был, когда я в шестнадцать лет пришел получать паспорт и паспортистка спросила: «Мальчик, ты что, хочешь записаться евреем?». Я наивно ответил: «А как же иначе?». Это как раз было время Перестройки, 1989-й или 1990-й. Тогда как раз в Советском Союзе стали появляться первые отделения Еврейского агентства «Сохнут», и это была одна из причин, почему я решил открыть для себя свое происхождение. Я узнал, что мои бабушки и дедушки в молодости говорили на идише, что имя моей бабушки было не Галина, как я был уверен, а Голда. Однако страх перед советской системой был настолько велик, что все в семье боялись говорить о наших еврейских корнях. Вот почему я знал всего лишь несколько слов на идише, которыми иногда по случаю пользовались пожилые люди в нашей семье. И вот я стою с паспортом, в который вписано слово «еврей», и не понимаю, что это значит. Поэтому я стал активным членом «Сохнута» в Брянске. Я начал работать как мадрих, и мне это нравилось. Это было действительно смешно поначалу, потому что нам нужно организовывать лагеря для детей, не имея никаких знаний об иудаизме и традициях. Поэтому перед мероприятиями мы вечерами и ночами сидели и учились сами, чтобы на следующий день это рассказывать детям. Это было достаточно интересное время. Затем мои родители решили эмигрировать в Германию. Для меня лично это было сложное решение, мне оно не нравилось, потому что я хотел уехать в Израиль. Однако не мне было решать, поэтому в 1996-м году я переехал в Германию, и уже сейчас я в Германии живу дольше, чем жил в Брянске. Когда я приехал в Германию, я сразу же записался в Центральном совете еврейских общин Германии. Так я стал членом еврейской общины. 
	
	\textbf{Правильно, что Вы осознали себя евреем в возрасте шестнадцати лет?}  
	
	\textbf{Виталий Либеров:} Понимаете, я всегда знал, что я еврей. Как ребенку мне часто приходилось драться с другими детьми, которые издевались над моим еврейским происхождением. Но до шестнадцати лет я никогда не воспринимал себя евреем в религиозном смысле этого слова. Трудно объяснить, что есть разница между понятием «еврей» как национальностью и «иудей» как религией. В возрасте шестнадцати лет я решил не разделять эти два понятия – я принял как свою национальность, так и религиозную принадлежность. 
	
	\textbf{А какие были Ваши отношения с другими детьми до шестнадцати лет?}
	
	\textbf{Виталий Либеров:} Ну, я не хочу сказать, что был несчастным ребенком. В советской школе официально нельзя было разделять людей по национальностям, потому что все были октябрятами, затем пионерами и так далее. И если кто-то пытался сказать что-то плохое в адрес евреев, учителя всегда за это наказывали. Государственная пропаганда показывала Советский Союз как многонациональное государство дружбы между народами, однако на самом деле был колоссальный контроль над количеством евреев в университетах, так называемые квоты. На улице, конечно же, антисемитизм среди детей был. 
	
	\textbf{Как Ваша семья сохраняла еврейские традиции?} 
	
	\textbf{Виталий Либеров:} Это все было в очень рудиментарной форме. Например, бабушки и дедушки не ели свинину, не ели хлеб во время Песаха или ограничивались в еде во время Йом-Кипура. Вопрос скорее был не в соблюдении традиций, а в очень маленьких нюансах. И, конечно же, никто не говорил на идише. Но забавно, когда я приехал в Германию, я к своему удивлению обнаружил, что немцы, особенно пожилые в Баварии, используют идишские слова, даже не осознавая этого. Например, «мазл», что означает «счастье», или «тахлис», что означает «быть честным, говорить правду». 
	
	\textbf{Как Вы стали членом еврейской общины Нюрнберга?} 
	
	\textbf{Виталий Либеров:} Я просто пришел не ежедневную молитву в синагогу и познакомился с другими членами. Я просто хотел продолжить и сохранить то, что я начал делать в Брянске. К счастью, началась эмиграция из постсоветских стран, и у меня появилось много русскоязычных друзей. Конечно же, особенность иудаизма заключается в том, что когда бы ты ни молился, когда бы ты ни отмечал Шабат или другой праздник, в любой стране мира есть люди, которые делают то же самое. Вот почему ты никогда не можешь быть одинок. Я никогда не был одинок с тех пор, как стал членом общины в Нюрнберге. 
	
	\textbf{Еврейская община Нюрнберга сегодня отличается от той, которая была, когда Вы приехали сюда?}
	
	\textbf{Виталий Либеров:} Разница огромная. Я приехал перед тем, как началась большая волна эмиграции. Перед этим в синагоге не мог даже собраться миньян для молитвы. Но в конце 1990-х община сильно возросла. У нас есть члены общины, которые регулярно посещают церемонии. У нас есть дом престарелых, большой зал для мероприятий, классы для молодежи, мы даже думаем открыть детский сад. Иногда сотни людей приходят на праздники. Но, конечно же, у нас есть и другого рода мероприятия, например, лекции, занятия, конференции. Мы приглашаем людей из других организаций, мы открыты для всех. Я думаю, Нюрнберг приобретает новое «лицо» благодаря растущей еврейской общине, поэтому немцы тоже заинтересованы к нам приходить.  
	
	\textbf{Сколько евреев проживает в Нюрнберге?} 
	
	\textbf{Виталий Либеров:} Я знаю, что у общины есть примерно две тысячи членов, и, я думаю, это примерно сорок процентов от общего числа евреев в Нюрнберге. На самом деле, наша община – не единственная. Так сложилось исторически, что у нас также есть община Хабада и несколько альтернативных групп. У этих групп есть даже свои раввины, которые менее консервативны. Наша община – классическая ортодоксальная община. Наша литургия была специально написана для нашей общины в середине XIX века. 
	
	\textbf{Ваша община поддерживает связи с Израилем?} 
	
	\textbf{Виталий Либеров:} Безусловно. У нас есть занятия по ивриту. Городом-побратимом Нюрнберга является Хадера, поэтому у нас есть обменные программы со школами в Израиле. Много израильских студентов приезжают в Нюрнберг и, особенно, в нашу общину.  
	
	\textbf{Есть ли в наши дни антисемитизм в обществе Германии?} 
	
	\textbf{Виталий Либеров:} Мой ответ будет очень субъективным. Я просто хочу вспомнить лето 2014-го года, когда была очередная военная операция в Газе. В это время мне было действительно страшно, потому что все – крайне правые, крайне левые, мусульманские организации – объединились в ненависти к евреям. Люди шли по улицам, выкрикивая антисемитские лозунги вроде ``\textit{Hamas, Hamas, Juden ins Gas}'', и полиция не делала ничего, чтобы их остановить. Огромная толпа молодчиков ворвалась в здание вокзала, потому что они думали, что владельцы ``Burger King''  и ``McDonald’s'' – евреи, хотя на самом деле в Нюрнберге они принадлежат мусульманам. Они громили здание, потому что ими двигала ненависть к евреям. Поэтому я определенно могу сказать, что в Германии антисемитизм есть. Я думаю, это огромный провал в образовании. После Войны в немецких школах тема Холокоста активно обсуждалась, более того, были живы люди, которые могли рассказать об ужасах Войны. На сегодняшний день темы Холокоста и Войны не настолько близки детям, насколько они близки людям моего поколения. К сожалению, для многих мусульманских семей тема Холокоста не существенна, или они даже воспринимают ее как неправду. По моему мнению, проблема в системе образования. Нет единой системы даже между федеральными землями Германии. Нет единого педагогического плана. Конечно, есть стандарты Министерства образования, но, в конце концов, все зависит от учителей, которые боятся дискуссии. Поэтому многие учителя просто задают детям написать сочинение про «Список Шиндлера» Спилберга. Но это же не показывает всей картины Холокоста. Говоря о причинах антисемитизма, мне было хотелось обратиться к директору «Сохнута» Натану Щаранскому. У него есть так называемая теория «Трех Д»: делигитимизация, демонизация и двойные стандарты. Каждая из этих трех «Д» является основой антисемитизма. Некоторые говорят: «Да, Израиль имеет право защищать себя, но у израильтян хотя бы есть автоматы, а у палестинцев – нет», – или, – «Да, евреев убивали во время Войны, но ведь другие народы убивали тоже». На сегодняшний день антисемитизм не настолько примитивен, как раньше, потому что люди прячут его под «левой» борьбой с капитализмом и глобализмом. Некоторые используют эвфемизм «антисионизм» или «критика Израиля», что является чистой формой антисемитизма. Иногда это те же самые люди, которые атакуют беженцев, как на самом деле, так и в социальных сетях. Люди не осознают, что антисемитизм – проблема не только евреев, а общества в целом. Даже когда я разговариваю с моими немецкими друзьями, они признают, что не считают антисемитизм их проблемой. Я помню, когда был мадрихом в одном лагере во Франкфурте в 1998-м году, какие-то немецкие ребятишки пробрались к нам, чтобы посмотреть, как евреи пьют кровь. Этот пример показывает, насколько антисемитизм глубоко в обществе и насколько разные формы он может принимать.  
	
	\textbf{Что может быть сделано для борьбы с антисемитизмом?} 
	
	\textbf{Виталий Либеров:} Во-первых, нужно работать с детьми. В школах необходимо показывать последствия ненависти и нетерпимости. Во-вторых, нужно заставить политиков слушать. Конечно же, в конце лета 2014-го года была огромная манифестация в Берлине возле Бранденбургских ворот, организованная Центральным советом еврейских общин Германии. Даже Ангела Меркель посетила. Некоторые политики говорили правильные вещи, но большинству из них все равно. В Германии до сих пор есть фашистские партии, которые не запрещены. Они просто меняют свои названия с «Немецкого национального союза» на «Собрание немецких националистов», а затем на «Национальную партию Германии». Они притворяются демократами, но всем ясна их сущность, несмотря на то, что Бюро по защите Конституции следит за ними. Были даже подпольные группы, которые убивали иностранцев, громили магазины. К сожалению, полиция мало что может с ними делать. 
	
	\textbf{Не могли бы Вы рассказать о Форуме еврейской культуры в Германии?} 
	
	\textbf{Виталий Либеров:} Это организация в Нюрнберге. Я состою членом правления. Это общество, которое состоит из еврейских и нееврейских членов. Нашей целью является рассказывать о еврейской истории и традициях в Германии и в Нюрнберге в частности. Мы организовываем различные мероприятия, лекции, мы приглашаем представителей различных конфессий для дискуссий на различные темы, например, свадьба, воспитание детей и так далее. Мы также организовываем экскурсии и семинары. Людям это нравится, и количество участников все время возрастает. 
	
	\textbf{Рассказывает ли Форум о Реформации и отношении Мартина Лютера к евреям, о его антисемитизме?} 
	
	\textbf{Виталий Либеров:} Безусловно. И эта тема было особенно интересна католикам. Но Нюрнберг, главным образом, протестантский город, поэтому нелегко об этом говорить. К сожалению, некоторые церкви в Германии до сих пор содержат такую антисемитскую вещь как ``\textit{Judensau}'', или «Еврейская свиноматка». Несколько лет назад в ступеньках южной башни Нюрнбергского Собора святого Лоренца были найдены могильные плиты с еврейского кладбища. Бывший глава нашей общины, покойный Арно Гамбургер, пытался достать плиты из Собора , но со стороны лютеранской церкви было много препятствий. В конце концов, нам это удалось, и сейчас плиты снова находятся на кладбище. История евреев в Нюрнберге очень богатая. У нас одно из самых старых кладбищ. Кстати, Нюрнберг трагически связан и с Ригой, потому что многие нюрнбергские евреи во время Войны были депортированы в Ригу и расстреляны в Бикерниекском лесу. У нас есть так называемый Рижский комитет, который трудится над увековечиванием памяти этих людей.    
\end{otherlanguage} 
\newpage
\section{Vitali Liberov (English translation)}

\textbf{Could you please tell us something about yourself first?} 

\textbf{Vitali Liberov:} My name is Vitali Liberov. I am forty years old. I am an entrepreneur. I was born in Bryansk, Russia. I was raised as a usual Soviet kid, I was a \textit{Young Pioneer}\footnote{Communist scouting movement in the Soviet Union}, then I joined \textit{Komsomol}\footnote{Communist youth organisation in the Soviet Union}, so I was not familiar with religion. But as I was growing up, I remember men in our family going together somewhere several times a year, whispering about something, coming home with big slices of Matzah wrapped up in paper, and things like that. But it all was so strange to me that I did not pay attention. The first moment I realised I was a Jew was when I came to the government office to receive my passport at the age of sixteen. The official asked me: “Kid, do you really want to be registered as a Jew?”. I naively answered: “How could that be otherwise?”. It was the time of Perestroika, 1989 or 1990. The first branches of the Jewish Agency Sokhnut\footnote{The Jewish Agency for Israel (Hebrew: \textit{HaSokhnut HaYehudit L'Eretz Yisra'el}) is a Jewish non-profit organisation founded in the early 20th century that seeks to connect Jews worldwide with their people and heritage, primarily by fostering immigration to Israel (\textit{aliyah}). With the collapse of the Soviet Union, many thousands of Jews emigrated from its territory into Israel. The Sokhnut supported this process.} started to appear in the Soviet Union, and that was the reason why I decided to discover my origin. I discovered that my grandparents spoke Yiddish in their youth, my grandmother’s original name was not Galina as I thought, but Golda. The fear of the Soviet system was so great that everybody in our family was afraid of talking about their Jewish origin. That is why I knew only a few words in Yiddish, as elderly people in our family used them occasionally. And there I was standing with my passport, in which the word “Jew” was written down, and I knew nothing about that. That is how I became an extremely active member of the Sokhnut Bryansk branch. I started working as a \textit{Madrich}\footnote{Hebrew word for caretaker.} and I liked it.  It was really amusing because we had to organise camps for children without any knowledge of Judaism and traditions. So, there were evenings and nights before the events when we had to learn ourselves. That was quite an interesting time. Then my parents decided to emigrate to Germany. It was quite a difficult decision for me personally. I was not fond of my parents’ decision because I wanted to go to Israel. But it was not for me to decide, that is why in 1996, I came to Germany, and now I live in Germany for a longer time than I lived in Bryansk. When I came to Germany, I directly applied to the Central Council of Jews in Germany. That is how I became a member of a Jewish community.  

\textbf{Is it correct that you realised that you are a Jew at the age of sixteen?} 

\textbf{Vitali Liberov:} I always knew I was a Jew. Even as a kid, I had to fight with other kids who were mocking me because I was a Jew. But until the age of sixteen, I never considered myself as a Jew religiously. It is hard to explain that there is a difference between being Jewish in ethnicity and being Hebrew as a religion. But at the age of sixteen, I decided not to distinguish these two meanings – I accepted both ethnicity and religion.  

\textbf{Were your relationships with other children good before the age of sixteen?} 

\textbf{Vitali Liberov:} I do not want to say that I was an unhappy child. In Soviet school, it was officially not acceptable to discriminate people by ethnicity because all were \textit{Little Octobrists}\footnote{Soviet communist organisation for children of age between 7 and 9.}, then Young Pioneers, and so forth. If someone was trying to say something bad about Jews in school, then the teacher had to punish this kid. State propaganda showed the Soviet Union as a multi-national state of fraternal friendship among the peoples, but in fact, there was a huge control over the number of Jews accepted to universities, so-called quotas. In the streets, of course, there was anti-Semitism among the children.  

\textbf{How did your family preserve Jewish traditions?} 

\textbf{Vitali Liberov:} In a very rudimental form. For example, my grandparents used not to eat pork, not to eat bread during Passover or to restrain themselves from food during Yom Kippur. It was not the question of preserving traditions, but rather some little nuances. And, of course, no Yiddish could be spoken. But, by the way, when I came to Germany, I surprisingly discovered that German people, especially elderly in Bavaria, use some Yiddish words without realising it. For example, ``Massel'', which means ``happiness'', or ``Tacheles'', which means being honest or speaking the truth.  

\textbf{How did you become a member of the Jewish Community of Nuremberg?} 

\textbf{Vitali Liberov:} I just came to the synagogue for daily prayer and I got acquainted with other members. I wanted to keep doing what I was doing in Bryansk. Luckily, the Jewish emigration from post-Soviet countries started and I got a lot of friends for whom Russian was their native language. And of course, there is a peculiarity of Judaism, that whenever you pray, whenever you celebrate Shabbat or any other holiday, there are people in every country of the world that are doing the same. That is why you cannot be alone. I was never alone when I became a member of Nuremberg community.  

\textbf{Is the Jewish community of Nuremberg different today from what it was like when you came?} 

\textbf{Vitali Liberov:} Tremendously. I came before the large wave of emigration started. Before that, the people at the synagogue could not gather for the \textit{Minyan}\footnote{Gathering of ten adult men required for a prayer in Judaism.} to read the prayer. But in the end of the 1990s, the community got a lot larger. There are many members of the community who regularly attend ceremonies. We have a nursing home, a big hall for events, classes for youth, we even think about opening a kindergarten. Sometimes hundreds of people attend celebrations.  But of course, we have also other kinds of events, for example lectures, classes, conferences. We invite people from other organisations, everybody is welcome. I think that Nuremberg is getting a new ``face'' thanks to the growing Jewish community, because German people are interested in coming to us.    

\textbf{How many Jews live in Nuremberg at the moment?} 

\textbf{Vitali Liberov:} I know that there are two thousand members of our community and I think it is forty percent of all Jews in Nuremberg. Actually, our community is not the only one. It just happened historically that there is also Chabad community and several alternative groups. These groups even have their own rabbis who are less conservative. Our community is a classic orthodox community. 
Our liturgy was written in the middle of the 19th century specifically for the needs of community.   

\textbf{Does your community have contacts with Israel?} 

\textbf{Vitali Liberov:} We have Hebrew classes. Nuremberg’s sister city is Hadera, that is why we have exchange programmes with Israeli schools. Many Israeli students come to Nuremberg and especially to our community. 

\textbf{Do you think that there is anti-Semitism in German society?} 

\textbf{Vitali Liberov:} My answer would be very subjective. I just want to remember the summer of 2014 when there was a military operation in Gaza. At this time, I was really afraid because everyone – far-right, far-left, Muslim organisations – united in hatred against the Jews. People were out on the streets shouting anti-Semitic things like ``\textit{Hamas, Hamas, Juden ins Gas}'' and the Police did not do a thing to stop them. A huge crowd of young men broke into the main station building because they thought the owners of Burger King and McDonald’s are Jews, whereas in Nuremberg, these stores belong to Muslims. They vandalised the building because of hatred for the Jews. That is why I can definitely say that there is anti-Semitism in Germany. I think there is a huge gap in the educational system. After the war, the topic of the Holocaust was broadly discussed at schools in Germany. Moreover, there were people who survived during the war and they could tell a lot. Nowadays, there are many children at schools to whom the topic of war and the Holocaust is not as close as to people of my generation. Unfortunately, to children from some Muslim families, the topic of the Holocaust is irrelevant, or they consider it to be fake. In my opinion, this is a problem or the educational system. There is no unified system of teaching history among the \textit{Länder} in Germany. There is no unified curriculum. Of course, there are standards set by the Ministry of Education, but in the end, everything depends on the teachers, who often are afraid to have a discussion. That is why many teachers just tell pupils to write an essay on Spielberg’s Schindler’s list, which does not show the Holocaust. Speaking about reasons of anti-Semitism, I would like to refer to the director of Sokhnut, Nathan Sharansky. He developed the so-called ``Three D'' test of anti-Semitism: deligitimisation, demonisation and double standards. Each of these ``three Ds'' is a basic of anti-Semitism. Some people say, ``Yes, Israelis are entitled to protect themselves, but at least, they have automatic rifles, Palestinians do not'', or ``Yes, Jews were killed during the war, but other peoples were killed, too''. Nowadays, anti-Semitism is not as primitive as before because some people tend to hide anti-Semitism under the fight against capitalism and globalism. Some tend to use euphemisms like ``antizionism'' or ``critique of Israel'', which is a pure form of anti-Semitism. Sometimes, these are the same people that attack refugees, both in reality and on social media. People do not realise that anti-Semitism is not only a problem of Jews – it is a problem of society as a whole. Even when I speak with my German friends, they admit that they do not consider anti-Semitism as their problem. I remember that when I was a \textit{Madrich} in a Jewish camp in Frankfurt in 1998, some German kids sneaked to the venue to see, as they said, how the Jews drink blood. This example shows how deeply anti-Semitism is rooted in society, and how different the forms are that it can take.  

\textbf{What can be done to fight anti-Semitism?} 

\textbf{Vitali Liberov:} First of all, we have to work with children. It is necessary to show at school the consequences of hatred and intolerance. Secondly, we must make our politicians listen. Of course, in the end of summer of 2014 there was a huge manifestation in Berlin near the Brandenburg gate organised by the Central Council of Jews in Germany. Even Angela Merkel attended. Some politicians came and said right things, but most of them do not care. There are still fascist parties in Germany which are not prohibited. It just changes its name from ``German National Union'' to ``Gathering of German Nationalists'' and then to ``National Party of Germany''. They pretend to be democratic, but in reality, everyone knows their nature. Despite being under surveillance by the \textit{Verfassungsschutz}, there were underground groups murdering people, vandalising shops and businesses. Unfortunately, the police are not efficient in catching them. 

\textbf{We would like to ask you about the \textit{Forum für jüdische Geschichte und Kultur}. Could you please tell us about this organisation?}  

\textbf{Vitali Liberov:} This organisation is based in Nuremberg. I am a member of the managing board. It is a society which consists of Jewish and non-Jewish members. Our purpose is to tell about the history and traditions of Jews in Germany and in Nuremberg, particularly. We organise some events, lectures, we invite representatives of different confessions to discuss some topics, for example marriage, raising children, and so on. We also organise excursions and seminars. People like it and the number of participants increases.  

\textbf{Does the Forum speak about the Reformation and Martin Luther’s attitude toward the Jews, about his anti-Semitism?} 

\textbf{Vitali Liberov:} Yes and this topic was especially interesting for Catholics. But Nuremberg is mainly a Protestant city, that is why it is not so easy to speak about this. Unfortunately, in some churches in Germany there is still such an anti-Semitic thing as the \textit{``Judensau''}, ``Jewish pig''. Just a few years ago, stones from a Jewish cemetery were found in the floor of the southern tower of the \textit{Lorenzkirche} in Nuremberg. The former chairman of the Community, Arno Hamburger, was working on taking these stones out of the church. There were many obstacles imposed by the Lutheran church. But we managed to do it and now these stones are back on the cemetery. The history of Jews in Nuremberg is very rich. One of the oldest Jewish cemeteries is in Nuremberg. Actually, Nuremberg is tragically connected to Riga, since many Nuremberg Jews were deported to Riga and shot in Biķernieki forest during the War. We also have a so-called Riga Committee, which is working on commemorating those people.
