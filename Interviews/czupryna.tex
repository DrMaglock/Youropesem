\section{Serhii Czupryna}

\textit{Serhii Czupryna (*1996) is a member of the Jewish community in Kraków. He is involved in educational activities fostering dialogue between Jewish and non-Jewish people. He was born in Ukraine and is of Polish-Jewish descent. He has lived in Kraków since 2013. While studying as a kid and later as a university student, he lived in Israel for several years. He studies at the Institute of the Middle and Far East of the Jagiellonian University. \\
We met him in the Jewish Community Centre (JCC) in Kraków on January 27th, 2018.}\par
\vspace*{2em}
\textbf{Have you ever experienced anti-Semitism in your life?}\par
\textbf{Serhii Czupryna:} Yes. Living in Israel, I’ve been to the West Bank many times and I wanted to go to Gaza very much, but that was difficult for an Israeli. In Poland, I’ve never experienced anti-Semitism as harsh as it might be perceived by people from outside of Poland. The first question many Jews, especially in Israel, ask me when I say I live in Poland and that’s where I choose to live, is: ``Why? Six million. Why?'' But I see Poland not only as a place of historical depreciation of Jewish culture, but as a great place for its revival. I feel very comfortable with being Jewish here, I feel safe to walk around the city with a kippah on my head. I’ve never seen anybody in Kraków being mean to me just because of the fact that I’m Jewish. \par
\textbf{Do you think anti-Semitism is a serious problem in Poland?}\par 
\textbf{Serhii Czupryna:} I don’t think of it in terms of ``the Poles against the Jews and the Jews against the Poles''. I see it as a lack of education about Jews, who they are, what they do. I’m very happy to be with a group like you and to answer questions, because dialogue is the first step to understand another culture. \par
\textbf{Is anti-Semitism in Kraków more passive than in other cities?}\par
\textbf{Serhii Czupryna:} Maybe it's more passive. I see Kraków as a big, globalised city. In Poland, it’s one of the biggest cities. There are many universities here, there are many young people who come here to work for the outsourcing companies. From year to year, I see that this city is becoming more and more international and multicultural. I don’t see that hate for any other culture is increasing. That’s a good thing, and it's kind of a change that I’ve seen since I came to Poland. Now, it's not only tourists who come here and see the Old City, go to Auschwitz, and go back home, but there are people who don’t even speak Polish, yet they live here, work here, carry out their everyday business here.\par 
\textbf{Do you think people in Kraków want to know more about Jewish culture?}\par
\textbf{Serhii Czupryna:} Yes. There has been a Jewish Culture Festival annually for twenty-something years. Visiting this event year by year, I see that more and more people are coming. Simply because this event exists, I would say that people are definitely interested in getting to know Jewish culture more.\par
%\textbf{Do you as a student work with secondary school students? Do you do anything to educate people about Jewish culture and heritage?} \par
%\textbf{Serhii Czupryna:} I would love to, but as I do my master's right now and work full-time, I don’t have much time to do so. But as soon as I’m done with my master's, I’m going to educate people about the Jewish culture and its influence of on Polish culture.\par
\textbf{Are students taught about Jewish culture at school?}\par
\textbf{Serhii Czupryna:} Yes, there is a whole department of Jewish studies at Jagiellonian University, it’s both historical and cultural. Academic studies on Jewish culture thrive in the city.\\ 
I also know some NGOs who go to schools and educate teachers. It’s difficult to teach about the Holocaust, though. It’s taught in an inappropriate way, maybe some teachers just go through it or don’t even mention this subject. Some will say, ``six million people were murdered here during the Holocaust'', and then full stop. Many people don’t mention that after that there was communism and that in the 60s, many Jews had survived the Holocaust and either stayed or left the country. It’s very important to educate about Jews correctly and about the historical context that the Jews are in here. I’m very supportive of those NGOs. I’ve participated in a few of their events.\\
Also, here in the Centre, there are groups coming from the US or from Israel annually, so this kind of impact on teachers is not only on the Polish side. When Israeli groups come to Kraków, for example, they only see the history, they see the Holocaust, and many groups just go through this building and say: ``Ok, this is the Jewish Centre, it’s cool that the Jews were here'', but they don’t see the community in here. So, for two years there has been an annual course for the guides of those groups from the US and Israel and they teach them how to raise more awareness for the existence of this community here.\par  
\textbf{Would you agree that Polish anti-Semitism operates on a verbal level, mostly? Do you think there’s a chance to decrease anti-Semitism just by educating young people, especially in small villages and towns, where they have few possibilities to meet Jews and Jewish culture?} \par
\textbf{Serhii Czupryna:} Exactly. I would say that it’s a very important matter because the bigger the city is, the more chances you get to interact with people different from you. But for example, if you’re from a small village, you have one school in this village, you have the homogenous society of Poles, you may have certain stereotypes about other people and other nationalities. It’s like this: ``Everybody hates Jews, I also hate Jews, I don’t know why''. I definitely see that there’s a very high importance of coming to smaller cities or villages to educate people on those matters because very often, nobody even knows that before the war, the whole village was 80\% Jewish, while now it’s 100\% percent Polish. A historian would come and say that there was the house of this and that person, there was a cemetery here and there, but locals don’t even know about this.\\
Education would help to change the verbal level of anti-Semitism. Many people use the word ``Jew'' in this somewhat pejorative form of ``\textit{Żydek}''. There are two different words: the person who believes in Judaism and the person who is born Jewish by nationality, and those are both different from ``Jew'', but the word itself, ``Żyd'' comes from the three Hebrew letters that are the base of the word ``Jehud'', which stands for the person who believes in Judaism. When people use the form of the word ``Jew'' in Polish in a pejorative manner, I explain them ``Guys, this is how it works, and stop because it really has nothing to do with this matter.'' But the level of the verbal anti-Semitism itself is, from what I’ve noticed, not that harsh in Poland. It definitely exists, but comparing to Ukraine, there is much less verbal anti-Semitism. I’ve come across this a few times in my life, but I also don’t see it as anti-Semitism because people maybe just didn’t know what they were saying. Education is the key to change all the negative meanings in our lives and our societies. As soon as we educate ourselves and other people on certain matters, the negative aspect of them definitely go away.\par 
\textbf{Many history teachers say that students are usually not interested in the lessons about the Holocaust and anti-Semitism. Do you think that it would be a better idea to introduce some practical classes about it, for example going to Auschwitz or visiting a Jewish community?}\par
\textbf{Serhii Czupryna:} I would totally agree with this, but I would not necessarily agree with the fact that high school students are the best audience to go to Auschwitz. I went to Auschwitz for the first time when I was twenty. I realised that I was still not ready for it yet. So, I would not say that visiting Auschwitz is the best way of practical studying about the Holocaust and anti-Semitism, but such things may on the other hand increase the interest of the students.\par  
\textbf{Do you as a Jewish Community Center organise workshops for students?} \par  
\textbf{Serhii Czupryna:} There is an organisation based in Kraków doing a project called ``\textit{Alev Bet} of Jewish culture'', ``the alphabet of Jewish culture''. It takes place annually in the autumn or in winter. They educate people on the whole basis of the Jewish culture, explaining the basics of certain traditions and celebrations.\par 
\textbf{Are you a religious person, do you take part in Jewish celebrations?} \par
\textbf{Serhii Czupryna:} I’m not that religious, but yes, I would definitely say that what I believe in is Judaism. Not the orthodox version of it, but I also live in a kosher home, I don’t put lights on on Shabbat, and there are mezuzahs at the entrances to my rooms in the flat. I live with two more religious friends. From time to time, I go to the synagogue. \\
There is the possibility of being a religious Jew in the city. There are three synagogues functioning more or less full-time. One is the Izaac synagogue, probably it’s the biggest in a matter of size, and it’s run right now by the movement Chabad-Lubawicz. Their aim is to teach the Jews how to be a Jew and also to teach other people what the Jews are, and educate about Judaism. Their main matter here is education, both to Jews and to non-Jews. They also run religious Jewish Sunday school. Here, there is a Jewish preschool, where children also get to know more about Judaism, about the tradition, religion, and certain religious celebrations. Three times a day there is a \textit{minyan}, just like in the more orthodox Remah synagogue. The Chabad synagogue is more open for everybody. There is a full-time rabbi from an organisation which is called Szewa Israel, who works for the Chief Rabbi of Poland, Michael Schudrich, and he is a full-time employee of the JCC. He also provides the Thora studies on Shabbat and helps people to convert. There are many different options of being a religious Jew in this city, as well as in other cities in Poland.\par  
\textbf{Do you speak Yiddish and, in general, how many people do?} \par
\textbf{Serhii Czupryna:} I don’t speak Yiddish because of my family’s historical background. There is a certain amount of people who live here who speak Yiddish. The more religious, orthodox Americans who chose to live here probably speak Yiddish or at least some variety. Until this summer, we had one of the oldest members of the JCC, Mr Mundek, for whom Yiddish was the first language, so he learnt Polish as a second, not native, language. He spoke perfect Yiddish; unfortunately, he passed away this summer. There are certainly other members of the community who speak Yiddish. In this building, there is an opportunity of having a private course of Yiddish. Besides, they teach Yiddish in the Jewish studies department of the Jagiellonian Univeristy.\par
\textbf{What about Hebrew?} \par
\textbf{Serhii Czupryna:} I do speak Hebrew. Right now, I work as a kind of Hebrew native speaker in one company in the city. There is also the possibility of joining a Hebrew course here, it is provided for ten different levels of Hebrew language skills, so there are people who started learning it when the JCC opened in 2008, and they are still in their groups that continued from 2008, almost for ten years. Hebrew is nowadays very popular in Kraków, I’d say.  \par 
\textbf{Are there any orthodox Jews in Kraków?} \par 
\textbf{Serhii Czupryna:} Yes, there are many orthodox members of the community in general, and a few orthodox members of the JCC community. First of all, not every orthodox will have a beard, not every orthodox will have a \textit{sztrajmł}, this huge hat. Many young people find themselves feeling much more comfortable in an orthodox movement, they look just like us, you can spot only certain things which show that a person may be orthodox. When it comes to interacting with them in everyday life, you would notice that if you say ``Let’s eat a sandwich.'', this person says ``Oh, but I need to wash my hands.'' And there is the whole ritual, it’s complicated, Judaism is complicated. \par
\textbf{Last November there was demonstration in Warsaw with sixty thousand people, and hundreds of them were shouting ``\textit{Sieg Heil!}'' and ``Jews out''.}\par 
\textbf{Serhii Czupryna:} It’s the question that I often get from the people who ask me, ``why do you chose to live in Poland? You’re Jewish, go to Israel''. You’re talking about the Independence Day demonstration. There were the far-right neo-Nazi movements, and it may be not very safe to interfere with them in Warsaw on the Independence Day. I was there once, I just had to help my friends. I had two friends from Israel visiting me in Kraków, and we came to Warsaw together on  November 11th. One of them is of half-Indian, half-Iraqi descent, he was born in Israel, and the second is from Ukraine. The problem was that none of them speaks Polish and one of them looks definitely not Polish. So, I was just afraid of them going to Warsaw by themselves, especially seeing this in the media. But as I came there with them, I realised that as soon as we just don’t go any close to the demonstration, which is controlled by police to a certain extent, we’re fine, there’s no problem with speaking Hebrew in the centre of Warsaw in the evening of the 11th. I would say that it’s very important that media show this, but I also see it as just a different opinion as long as it doesn’t interfere. They may destroy certain parts of Warsaw, but they don’t hurt people. As long as they don’t hurt people, I’m fine with it. That’s not the majority of Poland. Their ideology is definitely a bad thing, but I see the lack of knowledge about the Jews. One of the main claims was ``Jews, go home.'' Why? Lots of Jews were born here, Poland was pretty much Jewish before the war, so where should we go home to? I can go home to Ukraine, but my ancestor is from Poland. I see both of these countries as my home. So, to which home should I go? And then the conversation goes on, ``oh, okay, you were born here, so you’re fine, you speak Polish, it’s a different thing''. No. It’s totally the same thing, start educating those people about what is Jewish, who is a Jewish person. Sometimes, I can open their eyes and they say ``Okay, I was wrong'' or ``That’s probably the ideology, maybe it’s not that perfect, so maybe I should switch to something less far-right.'' I definitely see that there’s lack of education, and what I see as a problem is that the Polish government is not intervening in that situation. \par
\textbf{The whole demonstration consisted of 60,000 people. And some hundreds were shouting.}\par 
\textbf{Serhii Czupryna:} Yes, it’s a small percentage. \par
\textbf{But nobody from the sixty thousand cared about it. Normally, you would throw these people out of your demonstration.} \par 
\textbf{Serhii Czupryna:} Yes, that’s also an organisational problem. What I dislike more about those marches on 11th of November is that each year, there was a symbolical burning of the statue of a rainbow on one of the squares, which stood for diversity and tolerance. If it was removed, so that it’s not going to be burnt annually, that would also be a good step, first of all not to symbolise the intolerance while celebrating the pride of being Polish. I cannot call it patriotism; I call it nationalism. But certain people go there to celebrate patriotism, certain people go there to celebrate nationalism. That’s the problem that maybe the organisers of those marches face.\par
\textbf{Do you think that teenagers are more anti-Semitic than older people because they frequently use the internet and see a lot of hate directed against Jews? }\par
\textbf{Serhii Czupryna:} I think that the middle-aged group is the most anti-Semitic of all age groups. What I like about the globalisation is that if you’ve seen a lot of hate on the internet, you can just google certain things that may disprove this hate, and I see the availability of the information for teenagers as a very good thing. I would definitely not say that teenagers may be more anti-Semitic just because they see some propaganda on the internet or in the media. Now, in my generation, I’m twenty-one, it’s always like ``Okay, I would look for this, but also I want to find the arguments from the other side''. I would say that a lot of people would choose the peaceful version of perceiving some other minority, not the hateful one, and they would probably want to learn more about the conflict.\par
\textbf{Do you think that the problem of anti-Semitism will grow in the future?}\par
\textbf{Serhii Czupryna:} It’s an interesting thing. It would definitely depend on the region. I would say that in Poland not because from what I see right now, non-Jewish Poles are getting really interested in the culture, and they start understanding the influence of Jewish culture on Polish culture. Maybe in countries like France and Great Britain, which are more globalised, anti-Semitism may grow because of a negative change in the society, connected to other negative factors that are happening with the society. As the country struggles with certain things, there is a possibility for other negative factors to increase. But the world learnt through the Second World War that anti-Semitism was the depreciation of a certain national minority, and they learnt from the Holocaust that it happened once, so let’s not allow it to happen again to any group, not only the Jews.\par
\textbf{There is a new law in Poland called the law for protection of the good name of Poland. What do you think about this law?} \par
\textbf{Serhii Czupryna:} I disagree with many laws that the current government brings on, not only this one. From the patriotic side, this matter actually stands in line of the tension between patriotism and nationalism of the current government. It’s a really hard topic: Auschwitz, the relations between Israel and Poland. I think that as soon as the government changes, it’s going to be fine. I don’t think that it will drastically change the relations between Israel and Poland because diplomatic relations have been built up for a long time, with a lot of effort, so I don’t think that one law like this will change a lot.\par
\textbf{According to Jan Gross and his book ``Fear'', in Poland solidarity with the Jews during the Second World War was not a mainstream thing. In France, people were celebrating themselves because they helped Jews, and in Poland, many people are ashamed and they don’t want to be mentioned publicly. Is that right?}\par
\textbf{Serhii Czupryna:} The whole matter of helping Jews during the Second World War is a very touchy subject because many people say ``Oh, Poland is this death camp where everything happened.'' No, it’s not. It’s just because of the fact that Jews were living here, and it was the easiest way from the Nazi point of view to make it all happen here. Helping the Jews during the war, you could have got a death penalty, you and your whole family. And probably this fear still remains in certain people. When Israelis ask me ``Why do you live in Poland? Most of it happened in Poland.'', the other side is that also the biggest number of Righteous Among the Nations live in Poland, they are Polish, and they are proud of it, and they celebrate helping the Jews in Poland during the war.\par
\textbf{Have you experienced any anti-Semitic acts from Muslims when you were in Israel?}\par
\textbf{Serhii Czupryna:} Actually, no. I felt very safe living in Israel. It’s just a fact that the country exists in the region where it has to be militarized to a certain extent, just to exist. In Poland, you would feel weird if you went on a bus and there was a soldier with a gun. In Israel, it’s a totally different thing. I feel safer when I see a soldier with a gun, just a normal servant who is of my age and who just has to do the service for his country. I have many Palestinian friends, certainly some of them are Muslims, and I’ve never experienced any negative acts from their side. Also in Poland, I have many Muslim friends, and I get along with them very well just because of some common things between our cultures and religious beliefs. \par 
\textbf{How do Polish people behave towards you? Are they curious about your culture and religion? Or are they intolerant and reserved?}\par
\textbf{Serhii Czupryna:} They’re definitely interested. It’s interesting to see how people who are not even friends are interested in the fact that I’m a Jew and my friend is a Muslim. How come that we sit at the same table? We explain and say that it’s just a religion. Why should we sit at separate tables or not even go to the same café? There is a certain amount of curiosity. People may see me coming from the synagogue with a kipah, and my Muslim friend in hijab, and we’re just coming across the same street, stopping to say hello to each other. Seeing the face of these people with certain stereotypes that they may have about our cultures, and seeing us destroying those stereotypes – it’s really pleasant and funny.\par 
\textbf{You said that people in Israel are always surprised when you are saying that you chose to live in Poland? Why do you think they are?}\par
\textbf{Serhii Czupryna:} I would say because of history and because of stereotypes. The biggest after-war influx of Polish Jews to Israel was in the 60s and in ‘68. That generation already has kids and grand-children, and probably a lot of them have never been to Poland except for Auschwitz. They think ``My family survived the Holocaust, then they were kicked out of Poland.'' This is the vision that they have had about Poland since their childhood, from their parents and grandparents. I love bringing my Israeli friends to Poland to show them that this country is not black and white, it’s a country which has many beautiful sites they definitely should visit. I like to see those people being shocked when we talk Hebrew in public and everybody is fine with it. It’s also a matter of education, destroying the stereotypes.\par
\sloppy
\textbf{Often tourists from Israel are isolated by bodyguards. They only go to Auschwitz,  they don’t integrate with other people, so they can’t even get to know Polish culture.}\par
\textbf{Serhii Czupryna:} It’s true. Most tours from Israel are organised with the help of certain ministries, the Ministry of Education or the Ministry of Foreign Affairs. They should have bodyguards, it’s fine. The problem comes when these groups go to synagogues. Can you imagine? You have to go on a school trip to Italy, and even if you’re not religious, you go with bodyguards to the Sunday mass, and you’re supposed to spend the whole time in there. Still, when those groups come to the synagogues, it’s one of the small windows for them to interact with the local community because technically, we also go to the same synagogues. Often when I try to enter the synagogue, I’m stopped and asked by Israeli bodyguards ``Wait! Stay here. Where are you going?'' Obviously, I have a kippah on my head. ``Come on, why should a Jewish person go to synagogue on Friday?'' I’ve noticed that the same bodyguards come with different groups. After a few situations like these, they start to remember your face and then say ``\textit{Shabat shalom, shalom}''. Certain Israeli teachers, if you get to interact with them, they start saying ``Okay, oh, so you live here? You’re Jewish? Fine. Maybe we can make some meeting of our group with you?'' It just needs some time to develop and also to explain to the Israeli teens that Poland is not black and white, as on a picture from the war, people here are smiling, it’s not bad to talk to them, and you don’t have to have bodyguards when coming here.\par
\textbf{In most of the classes, there are sixteen or seventeen-year-olds, and when they come to Poland, they visit two extermination camps a day. Of course, it’s too much.}\par  
\textbf{Serhii Czupryna:} It’s tricky because in the present situation in Israel, both political and military, I see that this propaganda is needed, the propaganda of ``Let’s not let other people to destroy Jewish nation.'' But it’s done inappropriately in the case of Poland because instead of having this feeling that ``Yes, we need to stay together as a nation, we should not fight between each other and keep everything as peaceful as possible.'', those children probably learn about the fact that all the killings took place in Poland, and that’s the only thing they know about Poland. That is a problem, it’s inappropriate propaganda. But still, I see the need. I hope that Israel will come up with a change one day. \par 
