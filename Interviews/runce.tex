\section{Inese Runce}

\textit{Inese Runce, Institute of Philosophy and Sociology of the University of Latvia in Riga.\\ 
The interview took place in Riga in September 2017.
\colorbox{yellow}{In the introduction,\\ some information has to be inserted about whom she's talking about in the beginning.}}\par
\vspace*{2em}
\textbf{Inese Runce:} You can imagine what kind of reaction there was in general towards, this particular person and when the War started and the Soviets were preparing for leaving, the Soviet army left and so did the Soviet institutions, and they were about to organise kind of an evacuation, and those communists were the first to leave.  He wanted to take this family, he wanted to leave, but then the wife's family came and said, ``you're responsible for everything they've done, for all of those atrocities, you're the one, if you want, you can leave, we haven't done anything wrong, we are staying''. When the Nazi army came and this cleansing started, this partly Latvian family and the Jewish family ended up in the nearby forest, for the same reason, because the Latvian family were communists, and the Jewish family because they were Jewish people.\\
The interregnum period was not about Jewish people, but this very kind of revenge was taking place, and in just a few moments, mainly in the Rūjiena area, if I'm correct, this was about the ones who were associated with the Soviet regime during the first year of the Soviet occupation. Why did those Nazi slogans sometimes appear at the beginning of the Nazi occupation? Of course we can put it in this Nazi ideology in general, according to which the Nazi were liberating Europe from the bolsheviks. And then there was another person, \colorbox{yellow}{Semion Shustin}, a Soviet Jew who was sent here to organise the  deportations, and for the Nazis - not only from Germany, but also local Nazis, the sympathizers with the Nazi regime -, Semion Shustin was a perfect example to be placed as the icon of the Soviet regime: He's a Jew, he's a communist, and he's the one who's responsible for the Soviet deportations, and for the murder of the local population - a mystical Jew,in a way. You will be able to find something like this in Lithuania or in Estonia as well, one particular person, they are in Estonian and Lithuanian propaganda of the Nazi time. So, there you can find something very mystical, yeah, kind of Jew, the bolshevik. Here in Latvia, it had a particular face. There were very few Jewish people in Latvia in this interregnum period, and the ones who were associated with this first year of the Soviet regime, it was not like particular against Jewish community.
	
\textbf{Does anti-Semitism play a role in the public discussion at the moment?}

\textbf{Inese Runce:} I think it was a very important instrument when the discussion started in the 90 and in the beginning of the 2000, when the first memorials, the Holocaust memorials started to be erected all across the Latvia, and it took years to finish the research, to deal with all of this not very pleasant topics in the history of that, of the, of the modern day Latvia, the issue of the collaboration with the both, with the Nazis and with the Soviets. I think we have done with the Soviet regime like about who were the collaborating with the Soviet, the Nazi regime, but we haven’t yet finished our discussions about the collaborating with the Soviet regime. It's like, how to say, the recovery of the cultural memory, this very painful process for all of different groups. For example, it was a very typical, very strange and very typical kind of left over from the Nazi ideology that the Jewish people, which was very popular and even the historians were kind of arguing, discussing and so on and so on because they were no data to operate, like the Jewish people, there were so many Jewish people who were in the Communist Party and they are the ones who were kind of organising the deportations, so that, the slogans which were used by the Nazi regime,  by the Nazi occupation somehow appeared in kind of the way in the public, in the 90s common the beginning of the 2000s, and then my colleague, Dribins [Leo Dribins, look up], right now he is the,  he is leading researcher here at the Institute of philosophy and Sociology, but he is also the, he's also the one who heads the Holocaust survivors community here in Latvia and he did big research and organised the people and so, in particular about the Jewish people and the Communist Party in Latvia, kind of trying to find, not actually what came out: that the Jews were this minor group in the Latvian Communist Party in the late 30s and after the Second World War. The biggest group, the first biggest group where the kind of the communists percentage, in terms of the percentage, those were Latvians, and the second Russians, and then the third the Jewish population. 
 
\textbf{Inese Runce:} So, for example the same professor, Aivars Stranga, he researched the Archives of the Latvian Communist Party in the late 30s like to find out the percentage, the ethnic division in this context. So, the Latvians were overwhelming majority, not the Jewish people, but this kind of somehow left from the Nazi regime, from the Nazi. kind of the Nazi ideology come from, from this Nazi propaganda somehow left alive in the, because after the Second World War there was no, no possibility in general to deal with normal history in the Soviet era. There was no possibility to discuss what happened during the Second World War about the issue of the, and then there were those propaganda elements, yeah, appearing in the, in the understanding of the history and explaining of the certain historical events and so on and so on, and it's, that,  that's interesting in the 90s, in the beginning of the 2000, when those topics were researched, the results came out absolutely totally different, yeah?

\textbf{How folkloristic is the Museum of Occupation? Is it rather scientific or rather folkloristic?}

\textbf{Inese Runce:} I must say that I like what the Museum of Occupation does. For example, I think that they did few years ago very good exhibition on Rumbula, together with the Jewish museum come up because I was working at the Jewish museum for 5 years, and I must say that the exhibition that was made by the museum of occupation on Rumbula was very good from the different, from the historical perspective, from the artistic perspective. I was maybe a little bit upset that that, that the Jewish museum didn’t use the possibility. But they participated. Of course they, they had very little economic resources, but. The groups that the Occupation Museum is, I think, powerful enough, strong enough,  good enough, as much as they can, but the groups were sometimes kind of volunteering in the Museum of Occupation, so they tried to push their own history forward, kind of emphasize the certain topics, yeah? For example, that they had been, usually they have a very good seminars and the meetings between different groups on occasion of, on May 8th, yeah, like bringing the, the people who are like for example the Latvian soldiers who were in the Legion, of course they are nowadays they are not very many of them alive, very just few, but they, they bring people together to discuss the different things, the different experiences

\textbf{But don't they equate the Nazi occupation with the Russian occupation? I think it's a problem to equate the Nazi occupation and the Russian, the Russian occupation. If you make a museum of occupation in general, I see the danger to equate it.} 
 
\textbf{Inese Runce:} Yeah because the thing is, the different, the different groups have a different memory. For example, like normal Latvians have nothing to do with the Nazi institutions or the Soviet institutions. They have a simple memory what they experienced in the daily life.  For example, the Soviet chaos, poor people, whatever, rudeness of the soldiers, Soviet soldiers, the different things that they might remember. And then they remember the Nazi time. But they had nothing to do with the Nazi ideology occupation, so they have a, let's sa,y they remember, they remember mainly kind of the German soldiers, whom they contacted, if there was such a  chance, who were nicely  dressed up, [...],  they were organised. This is what they remember: they're totally different, yeah, so they, they remember like they had nothing to do with this kind of the Nazi terror, they had nothing to do with official Nazi institutions and so, but they remember, let's say, good, nice people whom they, because they knew German, they were able to communicate, this is kind of the same German culture and milieu until the times of the First World War, so for them this was kind of something familiar. And they didn't think of them as Nazis, but they thought of them as Germans,  yeah, who are like very typical for children, that the soldiers were giving the chocolate  candies, yeah, around. And then, on opposite side, they have a, let's say, some, the, the Russian army who is robbering who are, let's say, not well behaved, whom you are, kind of escape, kind of, well, first of all you don't speak Russian, you don't know the Russian language and so, and then this is, this is the, like the normal average Latvian memory about the Nazi time, and this is of course not the whole nation, because the, the national history is composed of different memories, yeah, it’s normal to have a different point of view, yeah, so and, and therefore sometimes, you can see also such a presentation about the Nazi, the times of the Nazi occupation. Yeah, as a better, as a better era than the Russian era, yeah, quotation.
 
\textbf{What about the other groups, for example the Russian people or the Polish people living in Latvia at that time, how did they perceive the Nazi occupation?}

\textbf{Inese Runce:} It also depends that they are not one unified Russian community, neither one unified Polish community. Because the Polish community is composed of two groups, for example:  the Polish people who are local and the Polish people who came after the Second World War from different parts of the Soviet Union. The same about Russians. The Russians are a even more diverse group in itself. There is no homogeneous Russian community, because there are Russian Old Believers, for example, who were drafted into the Latvian Legion. There were different Russian groups. For example the, the Putin's refugees, as we call them, yeah, the ones who are moving here, keep moving from, from Russia in modern day has absolutely nothing to do with anything, who lives sometimes very kind of. And then you have Russians or Russian speaking population who is composed of the different groups, Ukrainians, Belarusians, Russians, the Russificated, I don't know, the people from the different parts of the Soviet Union, and their cultural memory might be totally different, yeah? So, then for example if you speak about Russian Old Believers, who are kind of indigenous group for a long period of time, so their remember, kind of memory is very close to the Latvian. They, they experienced the same time, they experienced the same events except this kind of, the race theory, they made the Slavic groups very attentive, yeah, because the,  the, the for example the Latvians, Estonians, they were placed higher in this race pyramid. The Slavic groups were kind of lower, on a lower scale, yeah, so for example, even for the Polish people, so this was kind of issue, big issue, so the Slavic background in general, that they made them very attentive. The Latvians, they were like normal Latvians, yeah, they were not as much attentive to this.

\textbf{Do you maybe know what the people of Latvia expected from the Nazis before the occupation?}


\textbf{Inese Runce:} They didn't expect nothing. There was at the end of the 30s, there was kind of far, like how to survive. On one side, you have the Soviet regime, on opposite side, you have the Nazi regime. On one side you have this crazy man called Stalin, on opposite side you have this crazy man called Hitler, yeah, so how to survive, how to keep this independence.  This was the biggest reality for the political elite, for the intellectual elite. 

\textbf{And I think there were many Latvians who were fighting either against Stalin and against Hitler.} 

\textbf{Inese Runce} And they weren't fighting voluntarily. For example like again my family: The grandfather from the paternal side, he was taken to serve in the Nazi army, and the grandfather from [...] went to serve in the Soviet army. I knew very many families in Latvia who oldest son was taken to serve in the Nazi army and the youngest son was taken to serve in the Soviet army, and they were fighting around Volkova [look up] to each other, shooting to each other around Volkova. Because the Soviet army hierarchy sent the Latvian troops to Volkova and the Nazi army sent Latvian Legion to Volkova.

\textbf{And did the Jewish community in Latvia know about the crimes that were committed from 39 to 41 by the German army,about the beginning of the Holocaust?}

\textbf{Inese Runce:} There is another thing: authoritarian regime of Kārlis Ulmanis, which was established in the May of 1934, they very strongly controlled newspapers, all of the media, radio and, not to publish bad news as much as possible.

\textbf{It was like in Poland, they could have known it, but they didn’t believe it.}

\textbf{Inese Runce:} That’s also, that’s also true, for example when the, Latvia was actually the last country in Europe which accepted the Jewish refugees from Austria and from Germany. And when the people came and they started to share, the local Jewish people said: ``I don’t believe you.'', ``you are simply exaggerating''.
They were although quite a few articles published about what’s going on, but censorship measures usually controlled that nothing wrong was said about Hitler, in order not to provoke Nazi Germany, not to give a reason for occupation, and on the other hand, that nobody spoke of what happened under the Soviet regime, because it could also give Stalin possibility to occupy us.\\
Another thing is that for authoritarian regime of Kārlis Ulmanis, it was a very important to create the image of paradise, so everything is good, everything is quiet, everything is perfect, it’s not only, we will, we will make it, yeah, everything will be okay, the crisis, the political crisis in the Europe will be over and the bright future and, and so. This image started to disappear in 1939, which was also the, the beginning of the significant economical crisis, when the, the, the German population, the Baltic Germans started to be resettled. And you know, here in locally, there is kind of saying: If the Germans are leaving, it means Russians are coming.
And therefore you don't pay attention to also another things, You're so concentrated how to, how to survive, what to do, what's going to be, and what's happening in Poland, it just loses it sense, But there were not very many kind of news published in that case.



