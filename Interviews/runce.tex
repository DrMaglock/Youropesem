\section{Inese Runce}

\textit{Inese Runce, Institute of Philosophy and Sociology of the University of Latvia in Riga.\\ 
The interview took place in Riga in September 2017.}\par
\vspace*{2em}
\textbf{Inese Runce:} You can imagine what kind of reaction there was in general towards this particular person\footnote{\colorbox{yellow}{Explain whom she is talking about}} and when the War started and the Soviets were preparing for leaving, when the Soviet army and the Soviet institutions left, they were about to organise an evacuation, and the communists were the first to leave.  He wanted to take this family, he wanted to leave, but then the wife's family came and said, ``you're responsible for everything they've done, for all of those atrocities, you're the one, if you want, you can leave, we haven't done anything wrong, we are staying''. When the Nazi army came and this cleansing started, this partly Latvian family and the Jewish family ended up in the nearby forest, for the same reason, the Latvian family because they were communists, and the Jewish family because they were Jewish.\\
The interregnum period actually was not about Jewish people, but there was revenge taking place, and in just a few moments, mainly in the Rūjiena area, if I'm correct, it was about the ones who were associated with the Soviet regime during the first year of the Soviet occupation. Why did those anti-Semitic Nazi slogans sometimes appear at the beginning of the Nazi occupation? Of course, we can relate this to Nazi ideology in general, according to which the Nazi were liberating Europe from the bolsheviks. But there also was another person, Semion Shustin, a Soviet Jew who was sent here to organise the  deportations, and for the Nazis - not only from Germany, but also local Nazis, the sympathizers with the Nazi regime -, Semion Shustin was a perfect example to be placed as the icon of the Soviet regime: He's a Jew, he's a communist, he's the one who's responsible for the Soviet deportations, and for the murder of the local population - a mystical Jew,in a way. You will be able to find something like this in Estonian and Lithuanian propaganda of the Nazi time as well, one particular person, a mystical figure, the Jew, the bolshevik. Here in Latvia, it had a particular face. There were very few Jewish people in Latvia in this interregnum period, and also only a few ones who were associated with this first year of the Soviet regime, it was not the Jewish community in general.
	
\textbf{Does anti-Semitism play a role in the public discussion at the moment?}

\textbf{Inese Runce:} I think it was a very important instrument when the discussion started in the 90s and in the beginning of the 2000, when the first Holocaust memorials started to be erected all across the Latvia, and it took years to finish the research, to deal with all of this not very pleasant topics in the history of modern Latvia, the issue of the collaboration with both the Nazis and the Soviets. I think we have worked on the issue of who was collaborating with the the Nazi regime, but we haven’t finished our discussions about the collaboration with the Soviet regime yet.
The recovery of the cultural memory is a very painful process for all the different groups. For example, it was a very typical and very strange leftover from the Nazi ideology - this was very popular and even the historians were arguing because there were no data to operate with - that there were so many Jewish people who were in the Communist Party, and that they were the ones who were organising the deportations. The slogans which were used by the Nazi regime,  during th occupation somehow appeared in the public in the 90s and the beginning of the 2000s, and then my colleague, Leo Dribins, who is right now the leading researcher here at the Institute of Philosophy and Sociology, and head of the Holocaust Survivors' community here in Latvia, did a lot of research, in particular about the Jewish people and the Communist Party in Latvia. And he found out that Jews were a minor group in the Latvian Communist Party in the late 30s and after the Second World War. The biggest group among the communists, both in relative and absolute numbers, were Latvians, the second group were Russians, and then the third the Jewish population.\\
Besides, professor Aivars Stranga researched the Archives of the Latvian Communist Party in the late 30s to find out about the percentage and the ethnic division in this context. The Latvians were the overwhelming majority, not the Jewish people, but this thinking is somehow left from the Nazi regime. The propaganda and ideology somehow stayed alive because after the Second World War, there was no possibility in general to deal with history in the Soviet era. There was no possibility to discuss what happened during the Second World War, and then there were those propaganda elements, appearing in the understanding of history and in the explaining of certain historical events, and it's interesting that in the 90s, in the beginning of the 2000, when those topics were researched, the results showed something totally different.

\textbf{The Museum of Occupation, is it rather scientific or rather folkloristic?}

\textbf{Inese Runce:} I must say that I like what the Museum of Occupation does. For example, I think that they did a very good exhibition on Rumbula a few years ago, together with the Jewish museum, where I was working for 5 years, and I must say that the exhibition that was made by the Museum of Occupation on Rumbula was very good, both from the historical perspective and from the artistic perspective. The groups that run the Occupation Museum are, I think, strong and good enough, they do as much as they can, but sometimes, groups are volunteering in the Museum of Occupation, and they tried to push their own  history forward, to emphasise certain topics. For example, they have a very good seminars and meetings between different groups on the occasion of May 8th, bringing people together, for example the Latvian soldiers who were in the Legion. Of course, nowadays they are not very many of them alive, its' just a few, but they bring people together to discuss the different experiences....

\textbf{But don't they equate the Nazi occupation with the Russian occupation? I think it's a problem to equate the Nazi occupation and the Russian occupation, and if you make a Museum of Occupation in general, I see the danger to equate it.} 
 
\textbf{Inese Runce:} The point is that the different groups have a different memories. For example, normal Latvians have nothing to do with the Nazi institutions or the Soviet institutions. They have a simple memory of what they experienced in daily life.  For example, the Soviet chaos, poor people, rudeness of Soviet soldiers - different things that they might remember. And then they remember the Nazi time. But they had nothing to do with the Nazi ideology, and they remember mainly the German soldiers, whom they had gotten in contact with if there was such a chance, and who were nicely  dressed up and organised. This is what they remember: they're totally different, as though they had nothing to do with this Nazi terror, they had nothing to do with official Nazi institutions, they remember, let's say, good, nice people whom they, because they knew German, were able to communicate with, and it was more or less the same German culture and milieu until the times of the First World War, so this was something familiar. And they didn't think of them as Nazis, but they thought of them as Germans. For example,  it's very typical to remember that the soldiers were giving chocolate  candies to the children around. And then, on the opposite side, there is the Russian army who, let's say, is robbing, who are not well-behaving, whom you try to escape, and first of all, you don't know the Russian language. \\
So, this might be the average Latvian's memory about the Nazi time. This is of course not valid for the whole nation, because the, the national history is composed of different memories, it’s normal to have different points of view, and therefore sometimes, you can also see such a presentation about the times of the Nazi occupation: As a better era than the Russian era.
 
\textbf{What about the other groups, for example the Russian people or the Polish people living in Latvia at that time, how did they perceive the Nazi occupation?}

\textbf{Inese Runce:} It depends, there is not one unified Russian community, neither one unified Polish community. The Polish community is composed of two groups, for example the Polish people who are local and the Polish people who came after the Second World War from different parts of the Soviet Union. The same applies for Russians, the Russians are a even more diverse group. There is no homogeneous Russian community, and there are Russian Old Believers, for example, who were drafted into the Latvian Legion, or the Putin's refugees, as we call them, the ones who are moving here from Russia in modern days. And then you have a Russian-speaking population that is composed of different groups, Ukrainians, Belarusians, Russians, the ``Russificated'', people from the different parts of the Soviet Union, and their cultural memory might be totally different. Then, for example, if you speak about Russian Old Believers, who are indigenous group for a long period of time, their memory is very close to the Latvian. They experienced the same events, except for the race theory which made the Slavic groups very attentive because the Latvians and Estonians, for example, were placed higher in this race pyramid. The Slavic groups were on a lower scale, yeah, so even for the Polish people, so this was a big issue, the Slavic background in general, that made them very attentive. The Latvians were not as much attentive to this.

\textbf{Do you maybe know what the people of Latvia expected from the Nazis before the occupation?}

\textbf{Inese Runce:} They didn't expect anything. At the end of the 30s, there major issue was how to survive. On one side, you have the Soviet regime, on the opposite side, you have the Nazi regime. On one side you have this crazy man called Stalin, on the opposite side you have this crazy man called Hitler - so, how to survive, how to keep independence?  This was the reality for the political elite, for the intellectual elite. 

\textbf{And I think there were many Latvians who were fighting either against Stalin and against Hitler.} 

\textbf{Inese Runce} And they weren't fighting voluntarily, like my family, for example: The grandfather from the paternal side, he was taken to serve in the Nazi army, and the grandfather from the maternal side went to serve in the Soviet army. I know many families in Latvia whose oldest son was taken to serve in the Nazi army and the youngest son was taken to serve in the Soviet army, and they were fighting around Pulkowa to each other, shooting at each other around Pulkowa. The Soviet army hierarchy sent the Latvian troops to Volkova, and the Nazi army sent Latvian Legion to Pulkowa.

\textbf{And did the Jewish community in Latvia know about the crimes that were committed from '39 to '41 by the German army,about the beginning of the Holocaust?}

\textbf{Inese Runce:} The authoritarian regime of Kārlis Ulmanis, which was established in May of 1934, very strongly controlled the newspapers and radio, all of the media radio, in order not to publish bad news whenever possible.

\textbf{It was like in Poland: People could have known it, but they didn’t believe it.}

\textbf{Inese Runce:} That’s also true. Latvia was actually the last country in Europe which accepted Jewish refugees from Austria and Germany. And when the people came and they started to share, the local Jewish people said: ``I don’t believe you'', ``you are simply exaggerating''.
There were, however, quite a few articles published about what’s going on, but censorship measures usually controlled that nothing wrong was said about Hitler, in order not to provoke Nazi Germany, not to give a reason for occupation, and on the other hand, that nobody spoke of what happened under the Soviet regime, because it could also give Stalin possibility to occupy us.\\
Another thing is that for the authoritarian regime of Kārlis Ulmanis, it was a very important to create an image of paradise - everything is good, everything is quiet, everything is perfect, we will make it, everything will be okay, the political crisis in the Europe will be over and there will be a bright future. This image started to disappear in 1939, which was also the beginning of the significant economical crisis, when the German population, the Baltic Germans started to be resettled. There is this local saying here: If the Germans are leaving, it means Russians are coming.
And therefore you also don't pay so much attention to other things, You're so concentrated on how to survive, on what to do, on what's going to be, and what's happening in Poland, it just loses its sense, and there were not very many news published in this regard.