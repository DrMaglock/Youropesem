\section{Dr Inese Runce}

\textit{Inese Runce (*1976) is a lecturer at the Institute of Philosophy and Sociology of the University of Latvia in Riga since 2006 and head of a national research project of the Latvian  Council of Science titled ``The construction  of political nation in Latvia after regaining independence: 1991-2011'' since 2009.\\
She studied cultural studies, religion, and religious education and history in Riga and at Fordham University, New York City, between 1994 and 1998, and received a doctoral degree in modern history of Latvia in 2000. Form 2002 to 2007, she worked as a researcher at the museum ``Jews in Latvia'' in Riga. Her research interests in the fields of sociology and history include Baltic and Latvian cultural history, church and state relations and the history of the church in Latvia, religious identities, regional identities, and their historical and contemporary development.\\
The interview took place in Riga on September 25th, 2017.}\par
\vspace*{2em}

\textbf{Does anti-Semitism play a role in the public discussion at the moment?}

\textbf{Inese Runce:} I think it was a very important instrument when the discussion started in the 90s and in the beginning of the 2000, when the first Holocaust memorials started to be erected all across Latvia. It took years to finish the research, to deal with all these not very pleasant topics in the history of modern Latvia, the issue of collaboration with both the Nazis and the Soviets. I think we have worked on the issue of who was collaborating with the Nazi regime, but we haven’t finished our discussions about the collaboration with the Soviet regime yet.\\
The recovery of the cultural memory is a very painful process for all the different groups. For example, it was a very typical and very strange leftover from the Nazi ideology - this was very popular and even the historians were arguing, because there were no data to operate with - that there were so many Jewish people in the Communist Party, and that they were the ones who were organising the deportations. The slogans which were used by the Nazi regime  during the occupation somehow appeared in the public in the 90s and in the beginning of the 2000s, and then my colleague, Leo Dribins, who is right now the leading researcher here at the Institute of Philosophy and Sociology and head of the Holocaust Survivors' community here in Latvia, did a lot of research, in particular about the Jewish people and the Communist Party in Latvia. He found out that Jews were a minor group in the Latvian Communist Party in the late 30s and after the Second World War. The biggest group among the communists, both in relative and absolute numbers, were Latvians, the second group were Russians, and then the third the Jewish population.\\
Besides, professor Aivars Stranga researched the Archives of the Latvian Communist Party in the late 30s to find out about the percentages and the ethnic division in this context. The Latvians were the overwhelming majority, not the Jewish people, but this thinking is somehow left from the Nazi regime. The propaganda and ideology somehow stayed alive because after the Second World War, there generally was no possibility to deal with history in the Soviet era, there was no possibility to discuss what happened during the Second World War, and then there were those propaganda elements appearing in the understanding of history and in the explaining of certain historical events. It's interesting that in the 90s, in the beginning of the 2000, when those topics were researched, the results showed something totally different.

\textbf{Is the Museum of Occupation rather scientific or rather folkloristic?}

\textbf{Inese Runce:} I must say that I like what the Museum of Occupation does. For example, I think that they did a very good exhibition on Rumbula a few years ago, together with the Jewish museum, where I was working for 5 years, and I must say that the exhibition that was made by the Museum of Occupation on Rumbula was very good, both from the historical perspective and from the artistic perspective. The groups that run the Occupation Museum are, I think, strong and good enough, they do as much as they can, and sometimes, groups are volunteering in the Museum of Occupation, and they try to push their own  history forward, to emphasise certain topics. For example, they have very good seminars and meetings between different groups on the occasion of May 8th, bringing people together, for example the Latvian soldiers who were in the Legion. Of course, nowadays they are not very many of them alive, its' just a few, but they bring people together to discuss the different experiences...

\textbf{But don't they equate the Nazi occupation with the Soviet occupation? If you make a museum of occupation in general, I see the danger to equate it.} 
 
\textbf{Inese Runce:} The point is that the different groups have different memories. For example, normal Latvians had nothing to do with the Nazi institutions or the Soviet institutions. They have a simple memory of what they experienced in daily life.  For example, the Soviet chaos, poor people, rudeness of Soviet soldiers - different things that they might remember. And then they remember the Nazi time. But they had nothing to do with the Nazi ideology, and they remember mainly the German soldiers, whom they had gotten in contact with if there was such a chance, and who were nicely  dressed up and organised. This is what they remember: they're totally different, as though they had nothing to do with this Nazi terror, they had nothing to do with official Nazi institutions, they remember, let's say, good, nice people whom they, because they knew German, were able to communicate with, and there was more or less the same German culture and milieu in Latvia until the times of the First World War, so this was something familiar. They didn't think of them as Nazis, but they thought of them as Germans. For example,  it's very typical to remember that the soldiers were giving chocolate  candies to the children around. And then, on the opposite side, there are the Russian soldiers who, let's say, are robbing, who are not well-behaving, whom you try to escape from, and first of all, you don't know the Russian language. \\
So, this might be the average Latvian's memory about the Nazi time. This is of course not valid for the whole nation, because the national history is composed of different memories, it’s normal to have different points of view, and therefore sometimes, you can also see such a presentation about the times of the Nazi occupation: As a better era than the Russian era.