\section{Teresa Klimowicz}

\textit{Teresa Klimowicz graduated from Paideia - The European Institute for Jewish Studies in Stockholm and from the Hochschule für Jüdische Studien in Heidelberg. She obtained a PhD in philosophy in Lublin. For more than ten years, she has been professionally involved in Holocaust education and multicultural education with her NGO ``The Well of Memory'', which is based in Lublin. She also works at the ``Grodzka Gate - NN Theatre'' Centre, a cultural institution in Lublin, where she researches the Jewish history of Lublin. \\ 
The interview took place in Lublin on January 31st, 2018.}\par
\vspace*{2em}
\textbf{Teresa Klimowicz:} I think it's really intriguing how anti-Semitism serves as a function in the society throughout the ages in general and I guess it could be interesting to see whether it has the same function in all those different contexts that you are researching. I need to give you a little bit of background about who I am. I'm not an expert on anti-Semitism, I haven't been doing academic research on it, but I've been involved in Holocaust education and multicultural education in general for over 10 years in my small NGO, which is called the Well of Memory, from here in Lublin. For a couple of years now, I've been involved here at the Grodzka Gate centre where similar activities are taking place. I'm researching here the Jewish history of Lublin. I should also say that I'm not Jewish and that I am. This is my perspective.\\
Poland has experienced the whole spectrum of anti-Semitism, starting with religious anti-Semitism. Since Poland is a Catholic country, that would be not necessarily Polish anti-Semitism, but rather Catholic anti-Semitism. This is based on blaming the Jews for the death of Christ.\\
Religious anti-Judaism later developed to become a more national concept, maybe in the 19th century when the concept of the nation was invented, so to speak. This birth of national identity also included here in Poland the aspect of anti-Semitism. I think the interwar period of Poland of the 20s and 30s would be the proper example for what was going on. Another aspect would be what was going on during the Second World War and the Holocaust and another aspect is something that was part of the post-war history of Poland, so anti-Semitism based on the association of Jews with communism, what we call in Poland żydokomuna. And then another aspect is what is going on today. Until two weeks ago, I would probably have said that the function that anti-Semitism used to play for the Poles is now exchanged for general xenophobia and in particular, in my opinion, Islamophobia, so I think that is becoming a bigger problem than anti-Semitism in Poland. Through the very recent events, we see however that anti-Semitism is being awakened again in Poland.  

\textbf{What about the period between the two world wars? What about the plan to resettle Polish Jews to Madagascar?}

\textbf{Teresa Klimowicz:} These were the ideas of the right-wing parties in Poland. They were in the parliament, but not in the government. There was discrimination on different levels. A famous example is the so-called ghetto benches, so Jews were supposed to sit in a segregated place in class at university. There was numerus clausus at university, so only a certain number of Jews would be accepted. In the 30s, there were pogroms, and the independence of Poland was also followed by pogroms. In the 30s, there is a general wave of anti-Semitism all over Europe. I think it's important to remember that in many of those cases, it's not something particular for Poland. Both this Catholic anti-Semitism and this wave of anti-Semitism in the 30s are part of a larger problem with local variations. \\
I think the figure of ''the Jew'' in Poland in general serves as the figure of ''the other'' that helps to define a national identity. That's why it can be easily exchanged for something else, some other ''other''. Right now, it's refugees or Muslims or Ukrainians, especially here in Lublin because this is an important minority in this region. And whenever there is an issue that evokes this need for referring to ''the other'', the figure of ''the Jew'' also. Dealing with Jewish heritage and Jewish history myself, I cannot say that I've experienced any violent anti-Semitism. It’s rather a sort of myth based on some nuances of the language, like when a bird shits on the window of your car, you can say that you have a Jew on your car. Problems of this kind in the language were also researched by Professor Tokarska-Bakir. What has been going on in Poland for the last couple of days is the confrontation between the Polish version of the Second World War and the Israeli or Jewish version of the events. When you look at today's newspapers, all those myths about the Jews appear except for the one in which the Jews made matzah out of Christian blood: ''Why do the Jews blame the Poles for the Holocaust? They are ungrateful, the Poles were rescuing the Jews, we have the biggest number of Righteous among the Nations. The government gives a lot of money for the restoration of the Jewish cemetery in Warsaw. Why don't the rich Jews pay for their cemeteries themselves? Also, let’s not forget that the Jews were part of the oppressive system of communism, especially in the 40s.'' So we have the rich Jews, we have \textit{żydokomuna}, and we have Jews as ''the other''.  

\textbf{You're talking about the new law to protect the good name of Poland.} 

\textbf{Teresa Klimowicz:} We had a law protecting the good name of Poland before. The new thing is to penalize accusations against Poles. If somebody claims that they took part in the crimes of the Third Reich, for example. 

\textbf{Like Jan Gross?} 

\textbf{Teresa Klimowicz:} For example, but there is a debate whether it would fall under those articles. Anyway, I think this kind of ideas is derived from Jan Gross’ book ''Neighbours''. That was the book about Jedwabne that initially sparked this whole debate, together with several other books. Interestingly enough, the crimes at Jedwabne were also researched by the Institute of National Remembrance. They admitted that Poles participated in those crimes. The new law is connected to the same institution that actually admitted this. They try to make it about facts, but it's not about facts, it's about myths and the ways we want to remember things. There is a conviction of Jewish conspiracy in Poland where you don't need to be Jewish to be called or seen as a Jew. When you’re successful and powerful, that could be enough. Also, there have been competing versions of the past between the Jews and the Poles because the Poles see themselves as victims above anything else. A couple of years ago, I did an international project with people from Germany, Ukraine, Israel, and Poland. When we had a workshop about the position of each nation during the Second World War and the Holocaust, everybody wrote that they were victims. That was a very interesting experience. The Germans hesitated for a moment, but still, and everybody else had absolutely no doubts about the fact that they were victims of the war.  

\textbf{Was anti-Semitism used in Poland historically as part of the creation of a Polish identity that says that Poles are Catholics and Jews are something separate?} 

\textbf{Teresa Klimowicz:} The process of the creation of nations started late and then need you some reference point. I would say in Poland it was mainly the religious denomination creating the difference, the Ukrainians were Orthodox, the Poles were Catholics and the Jews were Jews.  

\textbf{How did anti-Semitism develop after the Holocaust?}  

\textbf{Teresa Klimowicz:} Maybe Lublin is a good starting point because Lublin was the first city to be liberated by the Red Army already in July 1944. Lublin became a centre of the re-establishment of Jewish life in Poland. People that had survived in the Soviet Union and in the surrounding \textit{shtetls} or camps would come to Lublin even before Warsaw was liberated. During the Warsaw Uprising, Jewish institutions were already being created here in Lublin in August 1944. 

\textbf{For example, Abba Kovner with his Nakam group was also in Lublin.} 

\textbf{Teresa Klimowicz:} Right. There were Jewish committees that moved to Łodz and Warsaw after the whole of Poland was taken over by the victorious army. They became like a base for self-government within communist Poland. In the beginning, the communists had an idea of creating a more open society. Their politics towards minorities in Poland had evolved. Their statements in the beginning were to provide full rights to the Jews in this newly established political system. But this very soon changed into a persecution, which was represented by the centralisation of different institutions. It was not applied only to the Jewish community, though. Some historians even claim that the Jewish community was privileged in communist Poland over the Ukrainian community, Łemkowie or other minorities. There were certain Jewish institutions functioning, for example the central Jewish committee, which later became a secular institution. That was cancelled in 1950 and another Jewish institution was created, the \textit{Towarzystwo Społeczno-Kulturalne Żydów w Polsce}, the so-called TSKŻ. Parallelly, there was a religious organisation called Kongregacje Wyznaniowe. These two institutions were functioning in Poland throughout the whole communist time. On the other hand, in the 40s in Lublin region, there were still pogroms, especially in Parczew, but there were also attempts at pogroms here in Lublin, so many people would flee from this area, first of all because a bigger part of the Jewish community were refugees from the Soviet Union. They were just in transit to other parts of Poland or other parts of the world. Still, it's important to stress that Lublin region was, apart from Kielce region, the area where most pogroms of the 40s occurred. Another important point is the anti-Semitic campaign of 1968, which is also a case of competing memories because for the Poles, March 1968 is a glorious event of student strikes and the formation of the opposition, whereas in the Jewish experience, it's mainly a factor contributing to another wave of forced emigration. On the one hand, this anti-Semitic campaign was a political play between different factions of the leading party and on the other hand, it was a propaganda wave in newspapers all over Poland, also in local newspapers like here in Lublin.  

\textbf{Was the background of the discourse of emigration the Six-Day-War?} 

\textbf{Teresa Klimowicz:} It started with the student strikes. In Warsaw, they banned the performance of a play by Adam Mickiewicz titled Dziady, which is important for the Polish identity. It was considered anti-Russian, so it was banned. Then, the students of the University of Warsaw went on strike and protested the censorship. That's how the riots began. Within the party, there was a propaganda that was probably intentional, it's quite a complex issue. They were saying that those that started the strikes are Jewish or have Jewish parents, that they are banana youth, and that they should get back to their studies instead of protesting. Banana youth was a term in opposition to the workers because bananas were an exclusive product. Only those people from the intelligentsia, who were well-off supposedly, could afford them. This was used as an argument against the students, that they are pretty well-off, that they protest against a system that is created for the promotion of the working class, and that they are Jewish. Actually, they were not called Jewish, it was disguised as anti-Zionism. There were protests with people carrying flags saying, ''Zionists to Zion''. Supposedly, these demonstrations were organised by the party. This is the main narrative of historical writing today: On the one hand, there were the students that represent true Polishness, on the other hand, there are these people inspired by the governing party holding anti-Zionist flags. All the newspapers would also refer to the military conflict in the Middle East because it all started in the international background of the Six-Day-War. Several layers of issues were in picture in March 1968 because of the political situation in the world and the involvement of the United States and the Soviet Union in the conflict in the Middle East. Poland had to take the same position in that situation as the Soviet Union and it became connected with those student protests against censorship. Now from the Jewish perspective, hundreds of people were forced to leave the country. There was a famous speech by Gomułka in which he was saying that we are happy to give emigration passports to those people who consider Israel their fatherland. This was a credible statement that they should leave. They were given one-way passports, so once they had left, their Polish passport would be taken away and their citizenship was cancelled. Again, the Poles saw it in a different way because many people wanted to leave Poland and they couldn't, so in the testimonies that we gather, I also hear that people thought that the Jews were lucky to leave, privileged even. Even I, when I’m writing about this March 1968, have a problem to put it together and to create one narrative because there seem to be two different experiences, the glorious one for the Poles and this fear of expulsion, of being rejected for the Jews. This was truly a nail in the coffin for the functioning of Jewish communities. Usually, we think it was the Holocaust that ended the Jewish presence in Poland, but I think it happened in March 1968. Today we don't even know how many Jews live in Poland because the statistics differ a lot. According to official statistics where you declare your nationality, there are about 1000 people declaring that they are Jewish. This is not reliable, though, because many people have Jewish roots and don't consider themselves Jewish or they just don't want to admit it. The Jewish community of Warsaw, for example, counts 700 people. In Lublin, we don’t know either because we don't have a separate Jewish community, it's a branch of Warsaw Jewish community.  

\textbf{Is there such a thing in Poland as not considering Jews as a part of Polish society?} 

\textbf{Teresa Klimowicz:} Yes and no at the same time, I think. There is a huge debate about that on the example of the POLIN Museum of the History of Polish Jews in Warsaw, which is supposedly showing the history of Polish Jews. Those who criticize it say that it's rather presenting the history of Poland with some Jewish aspect. The problem is again that there are competing memories between the Jewish experience and the Polish experience concerning events both hundreds of years ago and more contemporary ones. On the other hand, there has been, I believe, deliberate politics from 1989 onwards to include Jewish heritage into the narrative of many places in Poland, little towns, \textit{shtetls}, to confront them with their Jewish past. 

\textbf{That’s very interesting because, for example, in Latvia there is an opinion that Jews are separate. It’s very difficult to explain to people that the Jews that died during the Holocaust were actually neighbours, they were citizens of Latvia. People just don't accept this fact. How would you deal with it, how would you explain people that those people were not strangers, but neighbours?} 

\textbf{Teresa Klimowicz:} In fact, this is what I'm doing with the NGO ''Well of Memory''. We have a project supported by the Ministry of Culture and National Heritage, which is currently right-wing. We have a sizeable amount of money now to work with Jewish cemeteries. I have a feeling that I have found the key to talking to people in different places in the province and to convince them that the Jews were part of the history of their own town and had connections to their families. Going to the Jewish cemetery, engaging the school, the local government, the local media and so on, bringing attention to those places, it seems to work. Pointing out to people the abandoned Jewish cemetery close to them and telling them that it's their responsibility to take care of it, I think this is something that can work and unblock things. This is my ideal. I think it's also important from the Israeli perspective. When they come here, I can tell them about our common heritage. In Israel, they also have a different narrative, they are focused on the new state in Palestine, so they are not being educated that much about the heritage they had in the diaspora. Sometimes, I must explain to them important Talmudists of Lublin. They don't have a clue about that because they are not necessarily religious, either. 

\textbf{Is there such a thing in Poland as justification of the Holocaust?}  

\textbf{Teresa Klimowicz:} I don't think so.  

\textbf{Is there such an idea that Jews got what they deserved?}

\textbf{Teresa Klimowicz:} I know that's what some people in Israel think, I've been talking about that with my Israeli friends recently, but I don't think this is the general feeling of the Poles. I think with the Holocaust it’s rather about this competition of victims, so they were the victims, we were also the victims. People want to hear an acknowledgement of general suffering of the Poles, I believe. I don't think missing knowledge about the Holocaust is a problem in Poland, either. Remembering my own education, living in Lublin and in the shadow of Majdanek, so to speak, I think the awareness of the fact that the Holocaust took place and that it was important is clearly there. The problem is rather the connection to the fact that the Jews that died during the Holocaust were the neighbours that lived in the same street of Lublin. The narrative of the Holocaust was created in a way centralised around Auschwitz mainly. Even the camps of \textit{Aktion Reinhardt}, which are in Lublin region, are still forgotten in many ways. That's why it became an abstract idea, not something connected to the place you live in. Another Lublin example: There was a murder in Lublin of a guy that survived Bełżec, Chaim Hirschmann. He was killed on the streets of Lublin in the 40s and now the debate is whether he was killed by the Polish Home Army. Lublin was already liberated then, so the question is whether he was killed because he was Jewish or because he was thought to be a member of the secret police.  

\textbf{What is the main narrative?}  

\textbf{Teresa Klimowicz:} I cannot say that there is a main narrative because there are two competing narratives. When I guide and talk about that, I always say both versions because I don't know what was in the head of those that were shooting. 