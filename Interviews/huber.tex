 \begin{flushleft}
 	\section{Dr hab. Steffen Huber, Paweł Karpiński, and Krzysztof Turek}
 \end{flushleft}

\textit{Steffen Huber has been a senior lecturer at the Department of Polish Philosophy at the Institute for Philosophy of Jagiellonian University in Kraków since 2005. He is also a member of the Policy Council of the Józef Tischner Institute, an institute that was founded by pupils and friends of the philosopher and priest Józef Tischner (1931-2000) for the purpose of preserving and spreading knowledge about his works and of continuing research and development about the most important aspects of his philosophy.\\ 
Steffen Huber’s research interests are Polish philosophy of the Renaissance, social philosophy, and translations of philosophical texts. 
Paweł Karpiński and Krysztof Turek study philosophy at Jagiellonian University in Kraków and are active in the students' union.\\ 
The interview took place on January 26th, 2017, at Jagiellonian University in Kraków. The first part of the interview was conducted in German and has been translated to English for this publication.}\par
\vspace*{2em}
\textbf{As you've lived in both countries, we'd be particularly interested in a comparison of Germany and Poland regarding anti-Semitism. Which anti-Semitic incidents have you witnessed yourself or heard about second-hand? How is anti-Semitism rooted in the history of ideas in the two countries?} 

\textbf{Steffen Huber:} That's a big task. I have not dealt scientifically with the subject of anti-Semitism. For me it is a marginal phenomenon which one has to deal with every once in a while. It is also a very contested area, especially the question of what anti-Semitism is. We’ve also had this here at the university. Even today, you often encounter anti-Semitism and then have to hear people say that there was none. That happens even if the classical motives of conspiracy, cultural destruction and a low level of development of the Jews are mentioned. \\
If I compare Germany and Poland, I would say the major difference is that anti-Semitism in Germany has no basis of experience for most people in the 20th century. People had virtually no experience with Jews who were culturally recognisable as such, and certainly not as groups. There was practically no Jewish cultural group in Germany at the beginning of the 20th century. These were basically fantasies of the extreme right.\\
These two parameters are different in Poland at the moment. In other words, Jews in Poland have a much more strongly recognisable Jewish identity, even as a group identity, which goes well beyond the religious and becomes tangible as a cultural, linguistic, ethnic, and economic identity. At first, there were considerably more areas of conflict. One should not only rate conflicts negatively. There are also productive differences in a society that stimulate development. There is a much larger base for this kind of conflict in Poland, also because Jews were not assimilated over the centuries for a much longer period than in Germany.  \\
The second point is that, of course, where there are Jews, there is also anti-Semitism. Anti-Semitism has always existed in Poland, but I would not say that Poland is particularly marked by anti-Semitism. Of course, there have always been tensions. There were pogroms, but these took place more in the east, especially in the Russian area of Poland. Nobody ever thought up something similar to the Holocaust in Poland. Those who defined themselves as anti-Semites in the first half of the 20th century have, for the most part, never gone as far as they did in Germany. On the contrary, there were very important people, including in this area, who started from racial nationalism and thought the Jews were a threat to Poland. These people then started to save Jews under the impression of the Holocaust. In that sense, I believe that the two countries cannot be compared directly at all.\\
There is a sore spot in Poland: One has the feeling that the Germans have committed the Holocaust and now they are moving around, accusing people of anti-Semitism. This is a situation many people can't deal with. 

\textbf{Which experiences with anti-Semitism have you made here? } 

\textbf{Steffen Huber:} First I have to clarify how I understand anti-Semitism. I think that there is a quite useful definition by Hannah Arendt according to which anti-Semitism is based on images of the dangerous Jew, of the Jew who rules, of the Jew who is not rooted in the society, the culture, or even in the ethnical history of a nation. Of course, you can find this in Poland as you can find this in any other country. I'm travelling around in Eastern Europe and I would not say that Poland is a hotspot of anti-Semitism. We have to remember that there was a very intense development of Jewish culture in Poland which later on moved to Israel. We know that in the first year of the Knesset, the informal second language of the parliament was Polish because everyone from this part of Europe more or less used the \textit{lingua franca} of Polish, for example also those from the Ukrainian and Belarusian parts. I encountered anti-Semitism in Poland in two ways: First, there is a classical form of anti-Semitism that we find in the writings of Feliks Koneczny, for example. He was a historian working here in the University of Kraków and he wrote some books on history from a very specific perspective. He presented a theory of civilizations saying that the highest form of civilization is the Latin one, then we have the Eastern one, which means Russia, and then we have the Jewish civilization. Of course, the Jewish civilization is the worst and the least and the most dangerous. These texts are used by the right-wing political movement in Poland, they're quite popular. They are even used in some part of the academic discourse and of course, we have very hard conflicts over that.\\
Another example of anti-Semitism from Kraków: I talked to a man whose family owns some flats in the former Jewish part of Kraków, Kazimierz, and when I asked him about the Jews who lived in that house, he stopped talking to me. This does not mean that this man is an anti-Semite. Perhaps this means that his family had some bad experiences with the Germans or with anyone else in this context. This also means that what seems to be anti-Semitism sometimes is just the inability to speak about something what has happened to your own family, even if it was 70 years ago.

\textbf{So, the idea would be that you cannot talk about this experience and then you switch back to some common stereotypes that you can easily state and then use?} 

\textbf{Steffen Huber:} Yes, we should learn to differ between this and the classical anti-Semitism I saw in Poland. This intrinsic stupidity of anti-Semitism can be exported to other topics and to other ethnic or religious communities. This is what happens in Hungary now. We observed this in Poland as well. I think the technology of exporting the logic of anti-Semitism to other topics, such as refugees from Syria, was invented by the Hungarian government and now it's being used by the Polish government. It’s a way of copying anti-Semitic patterns and using them in a new political discourse. 

\textbf{Is this limited to the present government? Do politicians from other parties also speak in this manner?} 

\textbf{Steffen Huber:} I know some people who are really attached to the government of \textit{Prawo i Sprawiedliwość}, \textit{de facto} of Kaczyński. I'm absolutely convinced that they are no anti-Semites. They are deeply rooted in that kind of even pro-Jewish romantic tradition in Poland and they would never accept any kind of anti-Semitism. In the last weeks, we also observed a process which I personally welcome, with the government and the official media trying to fight anti-Semitism and right-wing ideology - this has to be stated as well. On the other hand, I met people who are rather left-wing, liberal, pro-European, and so on, and they are anti-Semitic. This shows that there is no clear correlation. If there is a clear correlation of right-wing thought of the conservative types with the nationalist or even racist type, this is the extreme right. But this is not true for the main part of the conservative people in Poland. \par
\textbf{Pawel Karpiński:} The government and the media are trying to fight anti-Semitism and xenophobic ideology, but I doubt that it is a clear intention. For example, our new Minister of Interior once said that he clearly does not tolerate any kind of racism or xenophobia, but when legal procedures are run against nationalist parties, nothing happens. These cases are dismissed and this is a clear sign that it’s being tolerated.\par
\textbf{Steffen Huber:} Take as an example Robert Winnicki, who is the leader of \textit{Ruch Narodowy}. He is talking about racial separation and he came to Dresden and shouted the Nazi slogan “\textit{Deutschland erwache}”. If you shout “\textit{Deutschland erwache}” at a meeting of Pegida in Dresden, this is Nazi ideology. After that, he declared he did not know what it means.  

\textbf{Did you ever encounter the stereotype of Judeo-Marxism?} 

\textbf{Steffen Huber:} Yes, it is used sometimes, but it's based on the romantic heritage of the 19th century in the fight against the Russians, the Prussians, and the Austrians. This was very closely culturally connected to the Jewish experience, so this is a very difficult situation.\par
\textbf{Krzysztof Turek:} There is another thing connected to Jews. It has a little bit of a different character, because the Communist state has openly dismissed Jews in the 60s. Even the Communists in Poland, 20 years after World War II, excluded thousands of Jews and people with Jewish roots. They lost their posts in public institutions, they were all removed from the universities, and they were removed from the party, even those who were loyal party members. 

\textbf{You have mentioned that these people see liberalism and atheism as a big problem. What is the position of the Catholic Church in Poland regarding anti-Semitism?} 

\textbf{Steffen Huber:} I think there is not just one Catholic church in Poland and I'm wondering why they don't split - which, I think, would have happened if Poland had a different history. The Church has been the strongest and most endurable institution in Poland for 1000 years and it is very well trained not to split in a situation of conflict. The Church was much stronger than the state for centuries. So, you have a part of the Catholic Church which is clearly pro-European; some texts written by Pope John Paul II some 30 years ago would be unbearably liberal for a big part of the Polish society. If you don't tell them that it was written by the Pope, they will say this is liberal ideology from the West. On the other hand, you have a very long tradition of Catholic nationalism and of Catholic anti-Semitism in Poland. 

\textbf{Have the last years of economic development had an influence on anti-Semitism? Is there any connection between the socioeconomic status of a society and the level of anti-Semitism?} 

\textbf{Paweł Karpiński:} I suppose there might only be a connection between low status and identifying a threat, some sentiment arising from seeing those who are well situated, who got money, and feeling it's somehow unfair. \par  
\textbf{Steffen Huber:} That is true, but it's also a stereotype of anti-Semites and racists. The people with conservative, nationalist, or racist convictions in parts of the society were absolutely not bad situated or in a bad economic situation. The rural part of Poland has developed very strongly over the last 10 or 15 years. It is true that it has not developed as fast in the 1990s and before Poland became a member of the European Union, but over the last 15 years, you could see a very, very strong development. I rather feel attached to those philosophical theories that say that anti-Semitism is not a political conviction or a political instrument. Of course, it happens to be one, but its most substantial element is the need to feel better than someone else.\\ 
\textbf{Steffen Huber:} In my opinion, in rural Poland, they don't like liberals and they don't like the European Union. They make use of the European Union, but they don't like it. They don't like the liberal elites in Poland, but they are no anti-Semites. The anti-Semites I met where rather well-situated, but that depends on your personal experience.

\textbf{What about the Jewish Communities? Do you personally know any of their members? How do they view the situation of anti-Semitism in Poland? Do you have heard anything about that?}  

\textbf{Krzysztof Turek:} I don’t know any person who is Jewish. After World War II and after the next cleansing made by the Communist Party, there were barely any Jews in Poland. \par
\textbf{Steffen Huber:} I met some people. They are living in a quite normal manner and of course, they will tell you about anti-Semitism. It’s not an everyday experience. It's not systematic physical aggression, but they will tell you that of course it happens from time to time. 

\textbf{You were also talking about the role of romanticism. Do you think it only is conducive to anti-Semitism because of its negative stance towards rationality?} 

\textbf{Steffen Huber:} No, this has to be treated very carefully. You’ll find some roots of aggressive racist nationalism in romanticism. Some authors are clearly pro-Jewish, though, and they take many basics from the Jewish tradition. Those who fought romanticism in the 19th century in Poland belonged to the positivist movement, which in the beginning was very liberal, but after 30 or 40 years, at the beginning of the 20th century, turned into the strongest and most serious anti-Semitic force in Poland. The National Democratic Party is rooted in the positivist movement. This is quite strange and you cannot say that anti-Semitism is romantic, anti-rational, and so on, while positivism is pro-Western, liberal, and rational. It is just not true. I think a substantial difference between Poland and Germany is that the Jewish culture in Poland was much more conservative, much more religious, and much more community and family-based. This was the experience of the Poles and this is how they tried in the 19th century to make some kind of Polish-Jewish dialogue, which really worked out in a very great manner. You have great pieces of literature and theatre which deal with these common metaphysical feelings. This is really a great part of the Polish literature and culture and this is also in the writings of Pope John Paul II, which of course are not read by the average Polish reader right now. ``\textit{Lingua Tertii Imperii}'' by Klemperer shows it in the clearest way. He said that the Nazi ideology is based on a romantic pattern, but this is a thing you cannot say about Polish culture; it wouldn't work out here.