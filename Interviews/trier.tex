\section{Emilia, Esther, Gennadiy, Margaretha, Sophia}
\begin{otherlanguage}{ngerman}
\textit{Emilia, Esther, Gennadiy, Margaretha, and Sophia are involved in the Jewish youth centre in Trier. Gennadiy and Sophia, both 23 years old at the time of the interview, manage the youth centre. Emilia (then 18), Esther (then 19), and Margaretha (then 19) introduce Jewish children into Judaism as} Madrichim \textit{(meaning ``guide'' or ``leader''). Emilia’s parents are from Ukraine and emigrated to Beta Tikva, Israel during the 90s. She was born in Israel and came to Trier as a small child. Esther was born and has grown up in Trier. Gennadiy grew up in Trier. He studies in Kaiserslautern and lives in Trier. Margaretha was born in Kiew and came to Trier as a small child. She has lived there ever since. Sophia was born in Chernivtsi, Ukraine, and came to Kaiserslautern as a small child. She grew up there and moved to Trier in 2014 for her studies.\\
The interview took place in Trier on September 23rd, 2018.}\par   
\vspace*{2em}
\textbf{Habt ihr jemals persönlich Erfahrungen mit Antisemitismus gemacht?} 

\textbf{Emilia:} Ich wurde nie körperlich angegriffen, aber ich habe zwei Vorfälle erlebt. Der eine Vorfall war in der Grundschule. Meine Eltern wollten mich beurlauben lassen an einem jüdischen Feiertag, um nach Israel zu fahren. Als wir schon in Israel waren, kam ein Brief zu uns nach Hause von der Direktorin, in dem stand, dass sie den Feiertag nicht anerkennt. Das ist ein Feiertag im Judentum und wir Juden können uns da auch beurlauben lassen, aber die Direktorin hat das nicht akzeptiert. Ich weiß nicht, ob wir eine Strafe zahlen mussten, aber es war krass, dass sie den Feiertag nicht anerkannt hat. Der andere Vorfall war vor ein paar Monaten. Mich hat eine wildfremde Person auf Instagram angeschrieben. Ich habe einen Screenshot davon gemacht: "`Du hässliches verfluchtes Wesen, du wurdest also auch von deiner Mutter in den Arsch gebumst. Du Dreckige ekelst mich richtig an, ich zähle schon die Tage, wo ihr Juden in die Hölle kommt."' Also, das fand ich halt schon ziemlich krass, was die gesagt hat.  Vor allem, da ich diese Person überhaupt nicht kenne und nicht weiß, warum sie mich auf einmal angeschrieben hat. 

\textbf{Hast du auf die Nachricht geantwortet oder sie ignoriert?}  

\textbf{Emilia:} Ich habe sie ignoriert, einen Screenshot davon gemacht und in einer Gruppe von Likrat davon geschrieben. Die haben gesagt, ich muss damit auf jeden Fall zur Polizei. Ich bin dann zur Polizei gegangen und habe eine Anzeige erstattet. 

\textbf{Wie hat die Polizei darauf reagiert?}  

\textbf{Emilia:} Sie fanden das auch richtig krass. Die Reaktion, während ich da war, war angebracht, die waren selbst total schockiert. Aber als dann ein Brief kam, dass sie diese Person nicht finden konnten auf Instagram, da denkst du, eigentlich findet man auf Instagram leicht diese Leute, auch wo die wohnen, einfach durch die IP-Adresse. Das hat mich schon gewundert, ich wollte aber nicht weiter darauf herumhacken. 

\textbf{Weißt du, ob deine Anzeige in die Bereiche politische Kriminalität und Antisemitismus einsortiert wurde?} 

\textbf{Emilia:} Das wurde unter Volksverhetzung eingetragen. 

\textbf{Du hast gerade eben den Daumen hochgehalten, als sie gesagt hat, dass sie nicht auf die Nachricht reagiert hat. Warum findest du das gut?} 

\textbf{Sophia:} Weil das einfach nur Internet-Hate ist und solche Menschen darauf aus sind, ganz viele Leute anzuschreiben, copy paste, und nur provozieren wollen. Ich glaube schon, dass diese Person spezifisch Leute ausgesucht hat. Manchmal kann man Interessen an einem Account erkennen, wenn Emilia z. B. verschiedene Sachen verlinkt hat. Da kann man eins und eins zusammenzählen. Die machen das fast aus Marketinggründen, um irgendwie Aufmerksamkeit zu schaffen. Dann gucken sich Leute dein Profil an und finden vielleicht irgendetwas, was sie als Konsument interessant finden. In diesem Fall vielleicht nicht, aber es passiert aus einem bestimmten Grund. Das ist so richtiger Hate-Trash, darauf sollte man am besten nicht reagieren, wenn es anonym ist und man die Person nicht kennt. Wenn es in deinem direkten Umfeld passiert, wenn du die Person kennst, wenn diese Person eine wichtige Bedeutung in deinen Augen hat, z. B. beruflich oder in der Schule, dann ist es etwas ganz Anderes. 

\textbf{Margaretha:} Bei mir gab es einen Vorfall in der fünften oder sechsten Klasse. Da war ich noch auf einer Realschule. Wir hatte gerade in Ethik das Thema Religion und dann auch das Thema Judentum. In der Klasse war ein muslimischer Junge, der irgendwann gefragt hat, ob wir eigentlich Juden auf der Schule haben. Ich hatte niemandem erzählt, dass ich jüdisch bin. Die Lehrerin hat gemeint, sie kennt niemanden, aber bestimmt gibt es Juden auf der Schule. Dann hat er gemeint, wenn er einen sieht, dann tötet er seine gesamte Familie, dann bringt er alle um. Das war noch ein Kind, da habe ich mir schon gedacht, was soll das eigentlich, und dann hatte ich ein bisschen Angst und blieb bei meinem Vorhaben, niemanden etwas zu erzählen. Dann habe ich die Schule gewechselt und ein paar Jahre lang keinem in meiner Klasse etwas gesagt, eigentlich aus diesem Grund. Es hat halt niemanden zu interessieren. Nach einiger Zeit habe ich aber gemerkt, dass die Leute da ein bisschen anders sind als auf meiner alten Schule. Meinen Freunden gegenüber habe ich mich dann geöffnet, also nach ein paar Jahren erzählt, so, ich gehe jetzt in die Synagoge. Die haben ganz anders reagiert, die fanden das alle cool und waren sehr offen. Wenn auf Instagram etwas über das Judentum gepostet wird, findet man in den Kommentaren immer richtig ekelhafte Sachen.\\ 
Im Netz findet man generell unter öffentlichen Posts solche Sachen. Ich melde die dann immer direkt und hoffe, dass sie blockiert werden. Manchmal kriege ich eine Nachricht, dass etwas erfolgreich gemeldet wurde, aber ziemlich oft auch nicht. Ich habe mir nie richtig angeguckt, was die Leute schreiben, aber irgendwann hat es mich so genervt, dass ich angefangen habe, die Leute alle nacheinander zu melden. Wenn ich so etwas sehe, gehe ich ganz schnell auf das Profil und klicke melden. Viel Aufwand ist das nicht.  

\textbf{Emilia:} Ich würde etwas zu meinem Instagram-Vorfall ergänzen. Ich habe das auch gemeldet, aber dann kam die Antwort, dass sie nichts Schlimmes gefunden haben. Das war schon komisch. Es standen ja eindeutig echt nicht gute Sachen drin.  

\textbf{Sophia:} In der Grundschule hatte ich einen Mitschüler türkisch-muslimischer Herkunft in meiner Klasse. Ich weiß nicht, woher ein siebenjähriges Kind so viel Hass auf mich hatte, nur weil er wusste, dass ich jüdisch bin. Er hat mich jahrelang unglaublich fertig gemacht, nicht nur deswegen, einfach, weil er mich nicht gemocht hat, aber unter anderem hat er mich sehr oft damit aufgezogen, dass ich jüdisch bin. Manchmal haben wir uns dann in die Haare bekommen, auch körperlich, das waren so Kindersachen. Ich fand es erschreckend, weil der Junge aus einer akademischen Familie kommt, sein Vater ist ein krasser Geschäftsmann, der ihm beigebracht hat, ganz viele Bücher zu lesen. Witzig ist, dass wir dann noch auf dasselbe Gymnasium und wieder in dieselbe Klasse kamen. Das war für mich der Horror. In der siebten, achten Klasse habe ich dann mitbekommen, dass er Krieg und Frieden liest und solche Sachen. Je älter er wurde, desto aufgeklärter wurde er wahrscheinlich. Dann kam nie mehr etwas in dieser Richtung und er war relativ respektvoll.  

\textbf{Hat er sich jemals entschuldigt, habt ihr euch ausgesprochen?}

\textbf{Sophia:} Nein, wir waren relativ kühl zueinander in der Schulzeit. Heute sehe ich, dass er sich immer meine Insta-Story anguckt. Ich mache sehr viel Kunst, ich zeichne sehr viel. Dafür interessiert er sich. Das ist das Paradoxe. Mehr Kontakt haben wir nicht. Ich würde mich gerne einmal mit ihm treffen. Ich würde einfach gerne von ihm wissen, woher er diesen Hass hatte. Einige Leute haben mir gesagt, dass der Antisemitismus in den arabischen Ländern unter anderem deshalb existiert, weil die Nationalsozialisten das sehr stark verbreitet haben. Das hat sich etabliert, steht in Schulbüchern und den Kindern wird von klein auf dieses Klischee beigebracht.   

\textbf{Haben in der Grundschule deine Eltern mit den Lehrern oder mit seinen Eltern gesprochen?}   

\textbf{Sophia:} Ja, das wurde alles besprochen, aber das hat zu nichts geführt. Ich habe auch nie großartig versteckt, dass ich jüdisch bin. Mir wurde aber ständig gesagt, ich sollte das verstecken. Mein Vater hatte ein bisschen Angst. Er ist selbst nicht jüdisch, er hat immer gesagt, sei vorsichtig. Meine Mutter hat mir immer Stolz beigebracht, du darfst dich nicht unterkriegen lassen. Als Jugendliche mit etwa 13 Jahren kannte ich ein Mädel, die 6, 7 Jahre älter als ich war. Zu ihr habe ich hochgeguckt. Wir waren mit einer Gruppe von Jugendlichen und jungen Erwachsenen auf einer Gedenkfahrt zu einem KZ in Frankreich. Wir waren die einzigen zwei Juden. Sie hat gesagt, sag auf keinen Fall jemandem, dass du jüdisch bist. Ich kam damit nicht klar und habe das nicht verstanden und habe es trotzdem den Mädels erzählt, mit denen ich mich dann angefreundet habe. Die fanden es dann cool, aber solche Tendenzen gibt es auch, dass man ständig dieses Bedürfnis hat, sich zu verstecken.  

\textbf{Esther:} Meine Eltern haben mir von klein auf gesagt, ich soll ein bisschen vorsichtig sein, z. B. an der Grundschule nie angeben, dass ich dem Judentum angehöre. Ich war dann im Ethik-Unterricht. Ich habe dementsprechend keine Probleme gehabt, weder in der Grundschule noch im Gymnasium. Ich hatte auch sehr viele Freunde mit muslimischem Hintergrund. 

\textbf{Emilia:} Ich habe immer angegeben, dass ich jüdisch bin. Meine Eltern sagen auch immer, ich soll aufpassen. Ich soll nicht verschweigen, dass ich jüdisch bin, aber es kommt natürlich auf die Situation an. Ich soll halt nicht die Nase hochziehen und sozusagen überall prahlen, dass ich jüdisch bin. Das habe ich auch nie gemacht, nur früher sehr oft nebenbei erwähnt, dass ich jüdisch bin. Heute ist es nicht das erste, was ich sagen würde, wenn ich mich vorstelle, sagen wir es mal so. Meine Eltern sagen auch oft, dass ich nicht meinen Davidstern anziehen soll. In die Synagoge ziehe ich relativ oft einen Davidstern an, aber in der Öffentlichkeit sagen sie meistens, ich soll es verstecken oder gar nicht tragen.  

\textbf{Gab es bei dir nur diesen einen Vorfall?} 

\textbf{Sophia:} Das war das Heftigste, glaube ich. Alles andere war nicht direkt krass antisemitisch. Das waren Äußerungen von Leuten, die nicht aufgeklärt genug sind oder sich nicht wirklich dafür interessiert haben. Sehr oft hat das damit zu tun, dass Leute alles über einen Kamm scheren und mich gleich mit Israel assoziieren. Es kommt auch sehr oft vor, dass Leute sagen, dass ich der einzige Jude bin, den sie je kennengelernt haben. 

\textbf{Finden sie das dann gut, schlecht, interessant oder ist es ihnen egal?} 

\textbf{Sophia:} Den meisten ist es ganz und gar gleichgültig. Manche interessieren sich dafür oder haben schon Hintergrundwissen. Dann ist es eine ganz andere Kommunikation, sehr locker. Heutzutage ist es ja so, dass viele sagen, dass es ihnen egal ist, welche Sexualität, welche Hautfarbe, welche Herkunft du hast, wichtig ist, wie du als Mensch bist. Sehr oft wird dadurch dein Hintergrund ignoriert, aber ich nehme es nicht so radikal. Es gibt viel krassere Sachen in Deutschland, harten Rassismus z. B. Das ist viel brutaler, finde ich.  

\textbf{Wurdest du in deiner Kindheit und Jugend mit Antisemitismus konfrontiert?} 

\textbf{Esther:} Es gab Momente, in denen die Aufgeklärtheit gefehlt hat, aber das war kein Antisemitismus. Man hat manchmal bemerkt, ok, die einen sind mehr auf der palästinensischen Seite, die anderen mehr auf der israelischen. Da spiegelt sich dann immer dieser Konflikt wider.  

\textbf{Hast du persönliche Erfahrungen mit Antisemitismus gemacht?} 

\textbf{Gennadiy:} Gott sei Dank nicht. Ich hoffe, das bleibt auch so.  

\textbf{Wo und wie hat ihr etwas über das Judentum, über jüdische Geschichte gelernt? Von euren Eltern, über die Gemeinde, über die Schule?} 

\textbf{Esther:} Ich habe das von Anfang an eher im Jugendzentrum gelernt. Dann war ich noch für ein paar Jahre beim Religionsunterricht. Im Jugendzentrum habe ich deutlich mehr gelernt und mich auch mehr dafür interessiert und selbstständig etwas dazu nachgeschaut. In der Schule wurde das Thema Judentum kaum behandelt. Ich habe immer gehofft, dass das Thema Judentum drankommt, aber das wurde immer nur kurz erwähnt, genauso wie der Buddhismus. Jetzt in der 13. Klasse geht es um den Holocaust, aber das Judentum als Religion habe ich in keinem einzigen Fach durchgemacht, zur jüdischen Geschichte auch nur Holocaust. 

\textbf{Gennadiy:} Das meiste, was ich zum Judentum gelernt habe, kam sogar erst nach dem Abitur. Das Thema Judentum hatten wir wenn dann nur im Vergleich, niemals an sich, weder in Ethik noch in Geschichte noch sonst wo. Das Thema Nationalsozialismus kam natürlich dran in der 10. Klasse und 13. Klasse, die Judenverfolgung wurde aber nie ausführlich besprochen. Es kann sein, dass es an der Lehrerin lag. Es gab eine Unterrichtsstunde, in der sie angefangen hat zu weinen, als sie angesprochen hat, wie die deutschen Gefangenen aus Russland wieder zurückkamen zu ihren Kindern. Bei diesem Aspekt hat sie angefangen zu weinen, beim Thema Holocaust hat sie mit keiner Wimper gezuckt. Ich vermute mal, sie hat eine Vorgeschichte mit dem Nationalsozialismus, anders kann ich mir das nicht vorstellen. 

\textbf{Emilia:} Ich habe in der Schule nichts Neues über das Judentum gelernt, vielleicht ein paar Einzelheiten zum Thema Holocaust. In Ethik musste ich sogar manchmal den Unterricht machen sozusagen, weil der Lehrer mich immer etwas gefragt hat. Bei meinem Bruder hat der Lehrer letztens sogar etwas Falsches gesagt, was mein Bruder dann korrigieren musste. Ich bin jetzt auch in jüdischer Religion. Da lerne ich etwas über das Judentum, im Jugendzentrum auch, vom ganzen jüdischen Umfeld. Mein Papa ist etwas religiöser, er hält z. B. die Speisegesetze ein. 

\textbf{Margaretha:} Wir hatten in der Schule in Ethik immer ganz kurz das Thema Judentum. Einer hat ein Referat über das Judentum gehalten, einer über den Buddhismus, dann war das Thema wieder abgehakt. In der zehnten Klasse gibt es bei uns an der Schule Projekte zum Holocaust-Gedenktag, z.B. kann man nach Hinzert in die Gedenkstätte fahren. An einem Abend trifft sich die gesamte Stufe in der Aula und man stellt vor, was man gemacht hat. Das gibt es aber nur bei uns. Das finde ich auch ganz gut, aber zum Judentum an sich habe ich eigentlich alles durch das Jugendzentrum gelernt. Dafür machen wir das auch, damit jüdische Kinder etwas über ihre Wurzeln lernen. In der Schule lernt man das nicht und zu Hause ist auch nicht jeder besonders religiös. 

\textbf{Bei dir?} 

\textbf{Sophia:} Bei mir auch durch die Eltern, am meisten durch meine Mutter. In Kaiserslautern bin ich in unsere jüdische Gemeinde gegangen, wir haben keine Synagoge. Dort hatte ich Religionsunterricht. Ich war ein-, zweimal auf Machane. Wenn ich über die Schule reflektiere, regt es mich auf, dass es, falls überhaupt etwas vom Judentum erzählt wird, immer in der Opferrolle oder als eine Art Fremdkörper dargestellt wird, der vollkommen vernichtet wurde. Immer, immer wieder diese Opferrolle. Dass das Judentum ein sehr wichtiger Teil von Deutschland war und noch immer ist, fällt unter den Tisch. Es gibt noch immer Floskeln im Deutschen, die aus dem Jiddischen stammen, oder sogar Essensgewohnheiten. Die Menschen, die damals vernichtet wurden, waren ganz normale Deutsche. Es kommt dann wirklich auf den Lehrer an. Wir hatten im Deutschunterricht eine super Lehrerin, die immer erwähnt hat, wenn der Autor Jude war. Die war eine sehr aufgeklärte Frau, vielseitig interessiert, auch an Israel, Kibbuz und so weiter. 

\textbf{Haben eure Eltern euch etwas von antisemitischen Erlebnissen erzählt?}\par                                                 \textbf{Sophia:} In der Sowjetunion war das ständig, allein schon, wenn du in der Sowjetunion deinen Pass geöffnet hast und da schon stand, dass du Jude bist. Meine Mutter hat mir erzählt, dass sie deswegen an der Uni manchmal nicht durchgelassen wurde, obwohl sie Klassenbeste war. Wenn sich jüdische Gruppen formiert haben, nicht politisch, einfach jüdische junge Erwachsene, Studenten, die sich getroffen haben und zusammen getrunken, erzählt, dann wurden sie vom KGB abgehört, weil das als jüdische Verschwörung galt. Meine Eltern haben mir auch erzählt, dass die Leute sich alle partiell als gleich angesehen haben, weil alle gleich arm waren und dieselben Defizite ertragen mussten. Es war nicht dieser harte Antisemitismus, nicht vergleichbar mit dem, was ich jetzt aus der Ukraine gehört habe, wo wirklich schlimme Sachen passieren und die halb wieder zur Fascho-Nation wird. Alte Bekannte meiner Eltern haben mir erzählt, dass an der Synagoge in der Stadt, wo ich geboren wurde, ein Hakenkreuz ist. Straßen werden nach alten Faschisten wie Stepan Bandera benannt. Meine Eltern sind sehr erschreckt und wirklich froh, dass sie hier in Deutschland sind. Ich auch. Ich muss schon sagen, dass die Deutschen sich Mühe geben, die Erinnerungskultur ist ihnen wichtig. Was hier als Selbstverständlichkeit gilt, ist es in anderen Ländern nicht. 

\textbf{Emilia:} Mein Papa hat mir erzählt, dass seine Oma immer heimlich in die Synagoge gegangen ist, um für Pessach Matzot abzuholen. Er ist auf Schulen gewesen, auf denen fast nur jüdische Schüler waren, deswegen war das kein Problem.  

\textbf{Esther:} Meine Eltern haben auch erzählt, dass man immer Klassenbester sein musste, ansonsten hat man viel weniger Chancen gehabt, auf einer guten Schule angenommen zu werden. Von meiner Mutter habe ich auch gehört, dass der Freundeskreis automatisch jüdisch war, nicht weil man sich mit den anderen nicht verstanden hat, sondern weil man dadurch eine Art Zweitfamilie hatte mit derselben Vorgeschichte, der man nichts erklären musste. 

\textbf{Gennadiy:} Meine Eltern haben mir nur erzählt, dass die Religion in der Sowjetunion kein großes Thema war. Sie wurde nicht unterdrückt, aber sie wurde sozusagen nicht laut ausgesprochen, nicht gelebt. Deshalb konnten mir meine Eltern nicht viel weitergeben. Ich vermute, dass es in Russland zumindest in den Großstädten jetzt anders ist, es gibt in Moskau z.B. eine jüdische Universität. Die Ukraine ist ein anderer Fall.  

\textbf{Wie kann man eurer Meinung nach am besten gegen Antisemitismus vorgehen?} 

\textbf{Emilia:} Das Projekt Likrat habe ich schon erwähnt. Likrat ist eine Begegnung zwischen gleichaltrigen jüdischen und nichtjüdischen Menschen, um Vorurteile abzubauen. Dieses Projekt läuft seit zwei Jahren. Da nehmen jüdische Jugendliche mehrmals im Jahr an Seminaren teil und werden dann an Schulen geschickt. Die Schüler können Fragen zu jüdischen Themen stellen, aber auch alltägliche Fragen, damit man ins Gespräch kommt und merkt, dass die Juden gar nicht anders sind als die anderen. Das klappt bis jetzt ganz gut, ich habe von keinen negativen Erfahrungen gehört. Ich glaube, das ist eine sehr gute Möglichkeit, Vorurteile abzubauen und ein Schritt in die richtige Richtung.                                                                                              

\textbf{Glaubst du, dass es sinnvoller ist, sich an junge Leute zu wenden, weil die noch nicht so festgefahrene Vorurteile haben?} 

\textbf{Emilia:} Junge Leute sind einfacher zu beeinflussen, ältere sind oft sturer. Ich denke aber, dass man auch bei den Älteren noch die Vorurteile abbauen kann.  

\textbf{Margaretha:} Ich denke, es ist ein ziemlich guter Weg, Leute zu melden, die im Internet Hassreden verbreiten. Die werden dann hoffentlich gelöscht. Dann beeinflusst das niemanden mehr, vor allem Kinder, die nicht verstehen, ob das stimmt, was Leute im Internet verbreiten. Ich denke, jeder Mensch sollte das melden, auch Nichtjuden. 

\textbf{Esther:} Ich habe vor einigen Tagen bei uns an der Schule den Religionslehrern Bescheid gesagt auf einer Fachkonferenz, dass ich in den Unterricht genommen werden kann, falls sie das Thema Judentum durchnehmen. Ich habe ihnen gesagt, dass ich Jüdin bin und dass die Kinder einfach normale Fragen stellen können, angelehnt an das Projekt Likrat. 

\textbf{Gennadiy:} Die Medien haben auf alle einen verdammt großen Einfluss, ob es die Tagesschau ist um 20.15 Uhr oder Spiegel Online bei Facebook um 8 Uhr morgens. Die heutige Kritik am Staat Israel ist irrational. Die Überschriften sind wahr, aber die Abfolge der Taten stimmt nicht. Erst wenn man den Artikel öffnet, erkennt man im zweiten Absatz, was tatsächlich passiert ist. Aber wenn wir ehrlich sind, öffnen wir nicht viele Artikel, wenn wir mit dem Smartphone bei Facebook rumscrollen. Wir lesen nur die Überschrift und dann tun wir so, als ob wir die größten Kritiker dieser Welt sind und geben das dann weiter.  

\textbf{Margaretha:} Ich denke, der moderne Antisemitismus, der vor allem aus der muslimischen Richtung kommt, konzentriert sich auf Israel. Früher ging es um Geldgeschäfte und die jüdische Weltverschwörung und andere komische Vorurteile. Heute ist es meist politisch. 

\textbf{Hast du noch eine Idee, wie man Antisemitismus bekämpfen kann, nicht nur persönlich?} 

\textbf{Sophia:} Durch Aufklärung, indem man miteinander ins Gespräch kommt und gemeinsame Aktivitäten macht. Das ist sehr idealistisch gesprochen, daran muss man einfach arbeiten. Juden sollten auch stolz sein und vollkommen selbstbewusst. Das war bei uns allen ein Problem, wir hatten Angst, das so zu benennen. Dass man ins Jugendzentrum geht und da einfach mit anderen jüdischen Jugendlichen zusammen ist und selbst so sein jüdisches Bewusstsein aufbaut, das ist eine super Sache. Wenn ich aber an Leute denke, die die AfD wählen, rassistische oder antisemitische Vorurteile haben, dann sind das sehr oft Leute, die nie in ihrem Leben einen Juden gesehen haben und immer noch ein komisches Konstrukt in ihrem Kopf haben. Vieles hat meiner Meinung nach auch mit Angst zu tun und Angst verursacht Hass. Deshalb muss man einfach nur aktiv werden, egal in welchem Bereich.     
                                    
\textbf{Sophia:} Es gibt sogar spezielle AfD-Seiten, die Leute anspornen, ganz viele Hasskommentare zu veröffentlichen, egal unter welchem Thema. Dagegen muss man wirken und das dann blockieren oder melden. 

\textbf{Welche Art von Aufklärung glaubt ihr ist wirkungsvoller, dass man mehr über das Judentum, z. B. die Realität des Judentums hier in Trier lernt oder dass man mehr lernt über die Verbrechen, die an Juden begangen  wurden?}\par                                                 \textbf{Gennadiy:} Was früher passiert ist, ist ein eigenes Kapitel. Das muss man individuell abarbeiten. Man kann das nicht als Werkzeug benutzen. Was heute passiert, muss man anders schaffen. 
\end{otherlanguage}