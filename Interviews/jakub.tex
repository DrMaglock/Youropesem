\section{Jakub (Yaakov Missingfamilyname)}

Yaakov (Jakub) (*1926) was born in a Polish town Frampol. Together with his family he spent three years (1939-1941) in the ghetto Biłgoraj in Poland. Frequently he used to get out of the ghetto to get some food and unfortunately got caught. He thought his days were numbered, but the German who caught him agreed to let him live in his house. Yaakov escaped to Russia in 1941 and later to Palestine. During his life, he learned and did many different jobs, and he speaks six languages. Today, he lives in an Elderly’s home in in Ramat Gan in the outskirts of Tel Aviv, Israel, many of whose residents are German-speaking Survivors The interview took place in Tel Aviv on September 21st, 2016. 

Jakub:  Urodziłem się we Frampolu w tysiąc dziewięćset dwudziestym ósmym roku. Rozpocząć opowiadać o tym co przeszedłem? 

Odpowiedź: Tak, bardzo prosimy. 

Jakub: Później w trzydziestym dziewiątym roku przed nowym rokiem zaczęła się wojna Niemiec          z Polską i przyleciały do nas samoloty i ja je słyszałem, miałem dziesięć lat i my zdążyliśmy uciec             z domu, ja i ojciec, i matka, i siostra i brat,  a jeszcze jeden brat widział jak zrzucają  bomby i on został zabity, mój starszy brat. My uciekliśmy i we Frampolu, zaraz za miastem jest taka góra, i tam myśmy się schowali, może ze dwie godziny żeśmy tam siedzieli i ustał szum bomb. Poszliśmy zobaczyć co się tam dzieje w mieście. We Frampolu nic nie zostało, rozbili całe miasto, zostały tylko szkoła, synagoga. Szkoła, gdzie ja się uczyłem i kościół. Niemcy zostawili to dlatego, że jak oni wyjdą, żeby mieli gdzie mieszkać,  myśmy przyszli i zobaczyli, że nic nie ma, nie wiemy co mamy robić, wszyscy co zostali,    nie zdążyli uciec, wszystkich zabili.  

Pytanie: Mógłby Pan opowiedzieć więcej o swoim dzieciństwie? Szkole? 

Jakub: Dobrze. Ja długo się nie uczyłem, skończyłem dwie klasy we Frampolu, wybuchła wojna i już się więcej nie uczyłem. Jak rozbili nam miasto, nie było gdzie szukać jedzenia i miejsca do spania. Mój ojciec szedł, on znał tych co mieszkali za miastem. Chodziliśmy tu i tam, by trochę zjeść . Tak się kręciliśmy. Raz poszedłem do sadu, poszukać jabłek, co spadają i jeść. Poszedłem do jednej chałupy,  jedni mleko noszą, inni mięso. Nie mieliśmy gdzie iść, liczyliśmy, że może w mieście coś zorganizują. Poszliśmy do miasta, do szkoły, trochę tu, trochę tam, gdzie ostały się jeszcze domy. Niemcy byli tam we Frampolu i wzięli nas do Biłgoraja. Biłgoraj to większe miasto niż u nas.  Tam była granica z Rosją, tam jeździły pociągi. Myśmy tak w mieście żyli trochę. Rozdzielili nas w mieszkaniach, stworzyli obóz. Jeść dawali - trochę zupy i chleba. Ludzie umierali z głodu. Mnie widział Niemiec jeden,                       był naczelnikiem obozu, taki Hans, i powiedział do mnie „idź pracować”. No co zrobić, poszedłem do niego i ja tam był w jednym pokoju. Ciężko. 

Pytanie: Jak zniszczyli Frampol, jak Pan tam z tatą, po tych wioskach chodziliście, to Polacy pomagali? 

Jakub: Dali jeść. 

Pytanie: Czyli nie byli wrogo nastawieni, pomagali? 

Jakub: Dali nam jeść, ale nie pomagali więcej. Bali się Niemców, za ukrywanie Żyda mogli ich zabić. My tam żyliśmy w tym obozie w Biłgoraju, żyłem tam trzy lata. Pracowaliśmy, ludzie dorośli byli brani do pracy. Zbierali kamienie i wywozili do Niemiec. Ja u tego Niemca. 

Pytanie: Czy kiedy był pan w Biłgoraju, to Polacy utrzymywali kontakt z Żydami, pomagali? 

Jakub: Pomagali. 

Pytanie: Potem co się z państwem stało? 

Jakub: Byłem trzy lata w getcie - my tam żyliśmy, więcej już złota nie było. Trzeba było kraść. Ja          w nocy poszedłem, wiedziałem kiedy Niemiec śpi, przekradłem się przez druty i poszedłem do wioski. Raz poszedłem, było bardzo ciemno, wszedłem w siano i do jedzenia nic nie znalazłem. To było chyba w tysiąc dziewięćset czterdziestym pierwszym roku, zima była ciężka. Wyszła Ukrainka, bo usłyszała szum. Wlazłem w to siano. Ona wyszła, patrzy i poznała, że ktoś rozrzucił siano. Wzięła widły i zaczęła widłami mnie kuć. Mam jeszcze dziury na nogach. Ona mnie dźgała, krzyknęła Antek złapałam Żyda. Ciężko.  

 Ja tam pracowałem u tego Niemca i pomagałem mu.  Jak on spał to ja już znałem drogi,        by przechodzić. Było ciężko to robić. Jak wyszedłem z tego siana, ledwo uciekłem. Ona zawołała tego Antka, żeby pojechał do miasta po Niemca. W tym mieście nie było Niemców. Oni w niewielu się bali wejść. Partyzanci, bandyci. On wziął wóz i pojechał po Niemca, żeby mnie zabrać. To trwało jakieś pół godziny. Przyjechał Niemiec. No i on mówił do Polaka, zobaczył na mnie i mówi do mnie „uciekaj”.    Ja uciekłem. Matka z ojcem stali w oknie całą noc, nie spali. Ciężko. Ja przybiegłem do domu, matka płakała. Nie chcieliśmy tak dalej żyć. Uciekliśmy do Rosji. Był nieopodal pociąg. Wykradliśmy się           i pojechaliśmy pociągiem. Rozpoczęła się wojna Niemiec z Rosją. Wszedłem do pociągu. Jechaliśmy dwie niedziele. Przyszedł jakiś ruski oficer i spytał: „czy wiecie dokąd jedziecie? Wy jedziecie do Rosji”. Dał takie dokumenty i podpisał, zapytał o obywatelstwo. Powiedział, że pojedziemy na Sybir. Nie było widać słońca, same lasy. Dał siekierę. Jedni rąbali. Rzeka tam była. My tam byliśmy, jedna rodzina z Warszawy i inne. Pracowaliśmy tam i było ciężko. Rodzina z Warszawy miała dzieci, które powinny się uczyć. Ja chodziłem za ojcem i przyglądałem się. Przyszedł oficer rano i spytał, dlaczego nie idziemy do pracy. Nie dostaniecie jedzenia. To mówi ten z Warszawy, mądry człowiek.                 Jak człowiek chory to wszyscy siedzą i pracują. Widzieli, że nie damy rady to wysłali nas do miasta. Ojca nie dali do wojska, bo on był zmiennik rodziny. Nie chciał przyjąć obywatelstwa. Pracowaliśmy w tym mieście, w fabryce. Miałem tam polskiego kolegę. Szedł z roboty, ktoś miał pół chleba. Powiedział, że musi to zabrać. Wzięli go, mnie puścili do domu. Musiałem umieć kraść, ktoś mnie zawołał:  „pójdź do matki, żeby dala coś jeść, bo zabierają mnie daleko”. Ona nie miała co dać. Ojciec był szewcem. Dawali jakąś kapustę,  żeby można było przeżyć. Jakoś przeżyliśmy. Miałem też nieprzyjemne sytuacje. Opowiadałem jak wracaliśmy z Uralu, oni nas podpuszczali, że bliżej do Polski. Posłali nas na Ukrainę. Kradliśmy tam. Przyszedł nowy rok, mówił, trzeba jakieś mięso, kury. Kradliśmy kury. On kradł ja trzymałem furtkę. Kiedy on tam wszedł to kury narobiły szumu. Obudził właściciela. W Polsce organizowano nowy rząd. To nie wszystko, my tam pracowaliśmy w tej Rosji, to było ciężkie życie. Jeszcze gorsze niż w obozie. Tam jeszcze mogłem kraść jedzenie. Raz mnie pojmał. Takie to było życie. Kiedy zaczynam wspominać, to budzi strach. Do dzisiaj jeszcze jestem chory.  

Jakub: Myśmy przyjechali w czterdziestym szóstym roku na Dolny Śląsk, i ja tam zacząłem pracować, poszedłem do kopalni, szkoły nie było, uczyć się nie było czasu to poszedłem, żeby pomóc ojcu pracować do kopalni, no i tam pracowałem, zrobili mnie ze stachanowca, wybrali mnie i robiłem sto dwadzieścia  procent i poszedłem do kibucy. Kibuc, też nie wiedzie, co to kibuc? 

Odpowiedź: Nie 

Jakub: Nie, Kibuc to taki, że ludzie żyją razem i jedzą wszystko razem. I byłem tam w kibucu                    i przyjechali z Izraela uczyć nas, przygotowywać, bo to miał powstać Izrael w czterdziestym ósmym roku, mieli przygotować nas, my byliśmy młodzi jeszcze. Kiedy mieliśmy jechać do Izraela, przeszliśmy granicę do Czecho-Słowacji i nie przepuścili nas. Wróciliśmy. No, a że kibuc się rozleciał i pracowałem w kopalni i dziesięć lat tam jeszcze byłem, to jest na Dolnym Śląsku. W końcu przyjechaliśmy do Izraela, znaczy nie dali nam wyjechać, aż się zamienił rząd, Gomułka doszedł do władzy i pozwolił, ja jeszcze jeździłem do Warszawy i ja tam byłem delegatem, bo pracowałem, byłem jedynym Żydem, który pracował w kopalni. Tak to było i przyjechaliśmy tutaj, tutaj szukałem i nie było pracy, jeśli nie spodobało im się to odsyłali nas do domu, ciężko było tak, jedzenia też nie było, bo to dawali na kartki, ciężko było, aż nas zaprosili, tam miasto było tu niedaleko z Jordanii i przychodzili ci Beduini      i Arabowie i kupowali, co mogli od nas i to poszło do wojska do Biszmaragu, jak Biszmarag był to było pogranicze i tam byłem z nimi i pracowaliśmy, a potem pracowałem z nimi na budowie trzydzieści lat, potem wyszedłem na pensję, ja miałem już pięćdziesiąt dwa lata, byłem chory, to poszedłem jeszcze pracować, do szpitala, pracowałem tam na Bitahom, pilnowałem tam żeby nie rozrabiali nam pracownicy, to było strasznie, tam i dzisiaj jest niewesoło, co zrobić, no ja myślę, że powinni to zakończyć, „to twoje to moje i do widzenia” . Tu byli Anglicy, Turcy byli, ale nie było gospodarstwa Palestyny. Ci Arabowie stąd, stamtąd, z Jordanii, z Libanu, to byli ci co teraz żądają  gospodarstwa, dajcie im gospodarstwo, niech będzie pokój, kto przeszedł wojnę, ten wie co to jest wojna. 

Pytanie: A jak wrócił Pan do Polski wtedy z Rosji, to Polacy zachowywali się w porządku czy byli wrogo nastawieni? 

 Jakub: No jak myśmy do Polski przyjechali, to moja żona i matka jej, i dwie siostry się uratowały        w Warszawie w getcie. W bunkrach siedzieli całą wojnę i ona nawet miała papiery polskie, bo ona była taka inna z wyglądu, jak na Żydówkę, a Żydów od razu mówili Niemcy żeby zabijać. 

Pani: Słuchaj można mówić jeszcze lata o tym co myśmy przeżyli, nawet lata, ja nie mówię jeden rok nie dwa lata, w jeden dzień nie można opowiedzieć, co to znaczy. 

Jakub: Tę historię, co ja opowiadam to nawet dzieci ze szkoły przychodzą wysłuchać, co raz to przychodzą delegacje, różne np. z wojska, ja przeszedłem ciężką drogę, ja nawet miałem kolegów w Polsce, po wojnie jeszcze żyłem dziesięć lat w Wałbrzychu, i co ja chciałem kończyć z wami te… Żona miała matkę i dwie siostry co uratowały się w bunkrze tam w Warszawie, było powstanie warszawskie i w czterdziestym szóstym roku myśmy wyszli po wojnie, minął jakiś rok, już przeszło, to napadli na nich studenci, jacyś tam mordercy, pogromy organizowali w Kielcach. I zabili wszystkich,    a moja żona poszła wtedy do kina i uratowała się, wtedy poginęło wielu Żydów, to było już po wojnie, pogrom w Kielcach. No i jak ja się ożeniłem, jeszcze żyliśmy razem pięćdziesiąt lat.  

Pytanie: Utrzymuje Pan jakieś kontakty z Polską, ma Pan jakichś znajomych, przyjaciół?  

Jakub: Oni z każdego miasta prawie są, no i ja bym pojechał, ale ja jeszcze ani razu nie leciałem samolotem. 

Pytanie: Może warto jeszcze spróbować, a utrzymuje Pan jakieś kontakty z kimś z Polski? 

 Jakub: Z Polski tutaj byli nieraz, Polacy z Warszawy robili film, wszystko nagrywali, chodziliśmy nad morze razem, u mnie byliśmy w mieszkaniu. Żyliśmy jak dobrzy koledzy, kiedy wyjeżdżali to myśmy się całowali, płakałem. 

Pytanie: A chciałby Pan pojechać do Polski? 

 Jakub: A polskiego miałem kolegę - Staszek, miałem dużo kolegów, ale to już minęło tyle lat, to ja tutaj w czterdziestym piątym/szóstym roku przyjechałem, to już sześćdziesiąt, ponad sześćdziesiąt lat. Ale jak była moja żona, to myśmy rozmawiali w domu po polsku, a potem były pogromy, a Polska jest mi jeszcze winna, ponieważ ja dziesięć lat pracowałem w kopalni po wojnie i ja byłem stachanowcem pracy i posłali mnie do Warszawy, z Bierutem siedziałem, mam tam fotografie, gdzieś w domu leżą i tam kiedy odkryli pomnik warszawskiego getta na Warszawie, to ja miałem zrobione zdjęcie koło pomnika, jak tylko odkryli. Były delegacje z Ameryki, chcieli mnie zabrać do Ameryki, ale ja nie lubiłem tak jeździć, mówią, że Izraelczycy lubią jeździć po świecie, ja tutaj siedzę i nie ruszam się stąd. 

 Możecie się o wszystko zapytać, ja wam wszystko odpowiem. Ja mam odpowiedzi na różne pytania, jak się mnie pyta to ja zaczynam wspominać. 

Pytanie: Jakie jest Pana wspomnienie ogółem o Polakach? 

Jakub: Jakie wspomnienie ja wam nie powiem, są różni ludzie na świecie, dobrzy i niedobrzy. Czasem się zdarzy pośród tysiąca ludzi, jeden co będzie świnia, tak to nie mam złych, oni dawali nam jedzenie, ja wam opowiadałem, że myśmy szli z jednej wioski do drugiej i dawali nam jedzenie. 

Pytanie: A w ilu językach Pan mówi? 

Jakub: Polski, żydowski, ruski, jidisz, niemiecki i arabski trochę znam. Ja żyłem tutaj i kiedy zacząłem pracować na budowie, sporo ciężkiej pracy było, nie było tych maszyn, nosiliśmy różne bloki, no          i potem ja już poszedłem na kurs i uczyłem się trochę, ze trzy lata i wyszedłem jako majster. Ja teraz nawet jeżdżę pomagać ludziom, którzy nie wiedzą co robić na budowach, jeździłem uczyć ich. Ja żyłem, pracowałem i mam dom swój, i od kiedy żona umarła mieszkam sam już dwanaście lat. Jak żona umarła, ciężko było samemu w domu, więc przyszedłem tutaj i tutaj nie jest źle.  

Pytanie: Będąc tutaj mówi Pan często po polsku? 

Jakub: Tutaj też rozmawiam po polsku, dlaczego nie? Mam się kogo bać? 

Odpowiedź: Nie, bo na przykład wczoraj byłyśmy u takiego Pana, który na co dzień nie mówi po polsku, bo po prostu nie ma okazji. 

Jakub: Aha nie ma z kim rozmawiać, ja też nie miałem z kim tu rozmawiać, z żoną ja rozmawiałem w domu po polsku, ona była Żydówką, ale wychowaną  w Warszawie, więc tylko po polsku rozmawiała i po żydowsku, trochę. Ja w domu zawsze i syn mój też rozmawiał po polsku. Mam syna, żyje w Natanie, mam 4 wnuków i 8 prawnuków.  