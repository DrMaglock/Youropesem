\section{Jakub (Yaakov)}
\begin{otherlanguage}{polish}
\textit{Jakub (Yaakov) was born in Frampol, Poland, in 1928. Together with his family, he spent three years (1939-1941) in Biłgoraj Ghetto in Poland. He used to get out of the ghetto to get some food and got caught. He thought his days were numbered, but the German who caught him agreed to let him live in his house. Jakub escaped to Russia in 1941 and later to Palestine. During his life, he learned and worked in many different jobs. He speaks six languages. Today, he lives in a retirement home in Ramat Gan in the outskirts of Tel Aviv, Israel. The interview took place there on September 21st, 2016.}\par
\vspace*{2em}
\textbf{Jakub:} Urodziłem się we Frampolu w 1928 roku. Kiedy w trzydziestym dziewiątym roku zaczęła się wojna Niemiec z Polską, przyleciały samoloty, słyszałem je, miałem dziesięć lat. Zdążyliśmy uciec z domu, ja, ojciec, matka, siostra i brat. We Frampolu, zaraz za miastem jest taka góra i tam się schowaliśmy, może ze dwie godziny żeśmy tam siedzieli, aż ustał szum bomb. Poszliśmy zobaczyć, co się dzieje w mieście. Niewiele zostało, rozbili całe miasto, zostały tylko synagoga, kościół i szkoła, w której się uczyłem. Wszyscy, którzy nie zdążyli uciec, zostali zabici. Nie wiedzieliśmy, co robić.  

\sloppy
\textbf{Proszę opowiedzieć więcej o swoim dzieciństwie, o szkole.} 

\textbf{Jakub:} Długo się nie uczyłem, skończyłem dwie klasy we Frampolu, wybuchła wojna i już więcej nie chodziłem do szkoły. Jak rozbili nam miasto, nie było gdzie szukać jedzenia i miejsca do spania. Mój ojciec znał ludzi, którzy mieszkali za miastem. Chodziliśmy tu i tam, by trochę zjeść. W sadzie szukaliśmy jabłek, w jednej chałupie dali mleko, w innej mięso. Nie mieliśmy gdzie iść, liczyliśmy na to, że może w mieście coś zorganizują. Poszliśmy do miasta, do szkoły, tam, gdzie ostały się jeszcze domy. Niemcy byli we Frampolu i wzięli nas do Biłgoraja. Biłgoraj to większe miasto. Tam była granica z Rosją, tam jeździły pociągi. Rozdzielili nas w mieszkaniach, stworzyli getto. Jeść dawali - trochę zupy i chleba. Ale ludzie umierali z głodu. Jeden Niemiec, naczelnik obozu, Hans powiedział do mnie "`idź pracować"'. No co zrobić, poszedłem do niego pracować. Ciężko było.

\textbf{Czy oprócz jedzenia Polacy udzielali Wam pomocy, gdy chodziliście po wioskach? Nie byli wrogo nastawieni?} 

\textbf{Jakub:} Dali nam jeść, ale nie pomagali więcej. Bali się Niemców, za ukrywanie Żyda mogli zostać zabici. Żyliśmy w tym obozie w Biłgoraju trzy lata. Ludzie dorośli byli brani do pracy, zbierali kamienie i wywozili do Niemiec. Ja pracowałem u tego Niemca.

\textbf{Co się później z Państwem stało?} 

\textbf{Jakub:} Byłem trzy lata w getcie. A kiedy już złota nie było, trzeba było kraść. Poszedłem w nocy, wiedziałem kiedy Niemiec śpi, przekradłem się przez druty i udałem się do wioski. Było bardzo ciemno, wszedłem w siano, ale do jedzenia nic nie znalazłem. To było chyba w tysiąc dziewięćset czterdziestym pierwszym roku, zima była ciężka. Wyszła Ukrainka, bo usłyszała szum. Poznała, że ktoś rozrzucił siano. Wzięła widły i zaczęła mnie nimi kłuć. Mam jeszcze dziury na nogach. Dźgała mnie i krzyknęła "`Antek, złapałam Żyda"'... Ledwo uciekłem z tego siana. Ona zawołała tego Antka, żeby pojechał po Niemca do miasta, bo w tej miejscowości nie było Niemców, bali się partyzantów, bandytów. Ten wziął wóz i pojechał po Niemca, żeby mnie zabrał. Trwało to jakieś pół godziny. Przyjechał ten Niemiec, spojrzał na mnie i mówi do mnie "`uciekaj"'. I uciekłem. Matka z ojcem stali w oknie całą noc, nie spali. Przybiegłem do domu, matka płakała. Nie chcieliśmy tak dalej żyć. Rozpoczęła się wojna Niemiec z Rosją. Uciekliśmy do Rosji. Wykradliśmy się i pojechaliśmy pociągiem. Jechaliśmy dwie niedziele. Przyszedł jakiś rosyjski oficer i spytał: "`Czy wiecie, dokąd jedziecie? Wy jedziecie do Rosji"'. Dał jakieś dokumenty i podpisał, zapytał o obywatelstwo. Powiedział, że pojedziemy na Sybir. Nie było widać słońca, same lasy. Razem z nami były inne rodziny. Dali nam siekiery do rąbania drzew. Pracowaliśmy tam i było ciężko, jeszcze gorzej niż w getcie. Brakowało jedzenia, musieliśmy kraść, ale jakoś przeżyliśmy. W czterdziestym szóstym roku przyjechaliśmy na Dolny Śląsk i tam zacząłem pracować w kopalni. Szkoły nie było, uczyć się nie było czasu, więc poszedłem, żeby pomóc ojcu. Zrobili ze mnie stachanowca, wybrali mnie, robiłem sto dwadzieścia procent i poszedłem do kibucu. Przyjechali z Izraela uczyć nas, przygotowywać do życia w kibucu, bo miał powstać Izrael w czterdziestym ósmym roku. Byliśmy jeszcze młodzi. Kiedy wyjeżdżaliśmy do Izraela, chcieliśmy przejść granicę z Czechosłowacją, ale nie przepuścili nas. Wróciliśmy. Jeszcze dziesięć lat zostałem na Dolnym Śląsku i nadal pracowałem w kopalni. Jeździłem do Warszawy, byłem delegatem jako jedyny Żyd, który pracował w kopalni. W końcu przyjechaliśmy do Izraela, to znaczy nie dali nam wyjechać, aż się zamienił rząd. Gomułka doszedł do władzy i pozwolił. Tak to było. I przyjechaliśmy tutaj, szukałem pracy, ale nie było łatwo, jedzenia też nie było, bo dawali na kartki.

\textbf{Czy utrzymuje Pan kontakty z Polską, ma Pan tam znajomych, przyjaciół?} 

\textbf{Jakub:} Mam przyjaciół w wielu miastach w Polsce. Byli tutaj nieraz, Polacy z Warszawy robili film, nagrywali, chodziliśmy nad morze razem, byliśmy u mnie w mieszkaniu. Żyliśmy jak dobrzy koledzy, a kiedy wyjeżdżali, płakałem. Miałem wielu kolegów, ale już minęło tyle lat. Ale jak żyła moja żona, to myśmy rozmawiali w domu po polsku. Polska jest mi jeszcze winna, ponieważ ja dziesięć lat pracowałem w kopalni po wojnie. Były delegacje z Ameryki, chcieli mnie zabrać do Ameryki, ale ja nie lubiłem tak jeździć. 

\textbf{Jakie jest Pana ogólne wspomnienie Polakach?}

\textbf{Jakub:} Są różni ludzie na świecie, dobrzy i niedobrzy. Czasem się zdarzy pośród tysiąca ludzi jeden, co będzie zły. W zasadzie to nie mam złych wspomnień, dawali nam jedzenie. 
\end{otherlanguage}