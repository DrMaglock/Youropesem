\section{Dr. Axel Töllner}
\begin{otherlanguage}{ngerman}
\textit{Dr. Axel Töllner (*1968) is an Evangelical Lutheran pastor. Since 2014, he is} Beauftragter für christlich-jüdischen Dialog in der Evang.-Luth. Kirche in Bayern \textit{(responsible for Christian-Jewish dialogue at the Evangelical Lutheran Church in Bavaria). After graduating from high school in Munich in 1987, he completed his civil service and studied Protestant theology in Erlangen, Kiel, and Jerusalem. During his studies, he completed additional training in Christian journalism and journalism. From 2003 to 2011, he worked as a pastor in various communities. He received his doctorate in 2007 at the University of Koblenz-Landau. Then, he worked until 2010 as a research associate at the Modern Church History at the Friedrich-Alexander-University Erlangen-Nuremberg, participating in the project} Synagogen-Gedenkband Bayern, \textit{a book series about Bavarian synagogues. From 2010 to 2011, he worked as a research associate at the working group ``Catholicism and Protestantism - Research for the NS Documentation Centre Munich'', then until 2014 again for the Synagogen-Gedenkband Bayern. \\
The interview took place on April 21st, 2017, in Nuremberg.}\par  
\vspace*{2em}
\textbf{Axel Töllner:} Bei unseren Recherchen zu einem Projekt mit etwa 200 jüdischen Gemeinden in Unterfranken sind wir auf das Problem gestoßen, dass die Quellenlage zu diesen vielen kleinen Orten total disparat ist. In einem Ort gibt es unendlich viel und im anderen Ort gar nichts. Ein anderes Phänomen, auf das wir gestoßen sind, ist dass die Leute im einen Ort schon relativ früh angefangen haben, sich mit ihrer jüdischen Geschichte zu beschäftigen, und in anderen Orten scheint bis heute irgendwie ein Deckel drauf zu sein. Das folgt aber auch keiner Gesetzmäßigkeit, das lässt sich keiner bestimmten Region zuordnen, im Nachbarort kann das schon wieder ganz anders sein. Das ist auch sehr abhängig von einzelnen Personen – oft sind das Leute, die da nicht groß geworden, sondern zugezogen sind.\\
Es ist auch ganz interessant, wenn man sich Ortschroniken anschaut, Vereinschroniken oder Ähnliches, da sind meistens nur kurze Notizen: 1933 wurde "`Der-und-der"' Vorsitzender und der nächste Eintrag ist dann erst 1945 – die ganzen Heimkehrer oder Flüchtlinge, die dann dazugekommen sind und dem Vereinsleben eine ganz neue Richtung gegeben haben. Die ganzen Ausschlüsse durch irgendwelche Arier-Paragraphen, und solche Geschichten wurden nicht dokumentiert.\\ 
Es ist auch oft schwierig, an das Archivgut der Gemeinden heranzukommen, weil die Gemeindearchive zum Teil nicht geordnet sind. Ich habe von einem Ort gehört, wo der frühere Bürgermeister sich mit der jüdischen Geschichte des Ortes beschäftigt hat und eine Publikation dazu geschrieben hat. Von seinen Quellen war dann aber überhaupt nichts mehr auffindbar und jetzt haben sie in den letzten Jahren erst hinten in der Ecke des Gemeindearchivs den ganzen Stapel gefunden. Den hat der einfach beiseite geschafft, das Material also nicht entsorgt, sondern im Archiv versteckt. Das ist eigentlich ein perfektes Versteck. 

\textbf{Man stößt bei solchen Recherchen immer noch auf sehr starke antisemitische Ressentiments. Kannst du deine Erfahrungen an ein paar Beispielen anführen?} 

\textbf{Axel Töllner:} Mir fällt immer wieder auf, dass selbst die Leute, die sich mit der jüdischen Ortgeschichte beschäftigen und sagen, dass das wichtig ist, immer von den Deutschen reden und von den Juden. Das ist also eine Sondergruppe, die anders ist als die Anderen. Dass die seit Jahrhunderten in dem Dorf gelebt haben und das Dorf mitgeprägt haben, ist im Bewusstsein nicht so eingedrungen. Das finde ich sehr interessant: Selbst bei den Leuten, die sich bemühen, die jüdische Geschichte ins Gedächtnis zu rufen, ist noch die Vorstellung da, dass die Juden irgendwie die Anderen sind.\\  
Ich kann mich an ein Forschungsprojekt mit den Schulen und den Kirchengemeinden in einem unterfränkischen Dorf erinnern: "`Jüdisches Leben in unserem Dorf"'. Das mündete dann in eine Broschüre und es wurden Stolpersteine verlegt. Ein Künstler im Ort hat versucht, von der Synagoge, die 1938 zerstört wurde, eine schöne Zeichnung zu machen, basierend auf historischen Fotos. Interessant ist, wie er die Leute dargestellt hat, die in die Synagoge reingehen. Das sind nämlich Leute, die gekleidet sind wie Ultraorthodoxe mit Schläfenlocken, schwarzen Hüten und Kaftanen. Das ist überhaupt nicht durch historische Fotos gedeckt. Die Fotos zeigen Leute, die gekleidet sind wie alle anderen auch. Ich finde es interessant, dass sich dieses Bild so festgesetzt hat.\\ 
Gelegentlich wurde ich an verschiedenen Orten mit der Haltung konfrontiert: "`Das alte Judenzeugs, das ist doch lange vorbei, das interessiert doch keinen mehr."' Da gibt es also Leute, die scheinbar nicht wollen, dass man etwas aufrührt. Ich habe auch mit verschiedenen Heimatforschern gesprochen, die sich zum Teil seit vielen Jahren auf diesem Gebiet engagiert beschäftigen. Die erzählen dann auch von Schwierigkeiten – besonders, wenn es darum geht, wer ein Haus gekauft hat, in dem früher eine jüdische Familie gelebt hat, unter welchen Umständen man an dieses Haus oder an irgendwelche Möbelstücke gelangt ist.  
Ich habe den Eindruck, die heutigen Leute identifizieren sich sehr mit ihrer Großelterngeneration. Die wissen schon, dass da irgendetwas nicht mit rechten Dingen zuging, aber "`das sind ja unsere Verwandten. Die werden es schon richtiggemacht haben."' 

\textbf{Wie gehen die Gemeinden mit Synagogen um, die noch erhalten sind?} 

\textbf{Axel Töllner:} Das ist ganz unterschiedlich. Es gibt Synagogen, die restauriert oder wieder zugänglich gemacht wurden. Ein ganz schönes Beispiel ist die ehemalige Synagoge in Arnstein, ungefähr 30 Kilometer nördlich von Würzburg – eine wunderbare Synagoge, frühes 19. Jahrhundert, klassizistisch – die im frühen 20. Jahrhundert noch eine ganz tolle Jugendstilbemalung bekommen hat, neben dem früheren Toraschrein, der zerstört worden ist. Oben rechts und links waren Greife aufgemalt und jeder dieser Greife hatte ein Wappen in der Hand - das eine ist weiß-blau und das andere ist schwarz-weiß-rot. Ein wunderbares Beispiel für das Selbstverständnis der Juden.\\ 
Die Stadt hat diese Synagoge vor einiger Zeit gekauft. Vorher war sie in Privatbesitz zu Wohnraum umgebaut worden und durch die nachträglich eingebauten Zwischengeschosse ein paar Sachen kaputtgegangen, aber es ist noch relativ viel von der Bemalung erhalten. 2010 wurde die Synagoge saniert und ist seitdem als Museum und Veranstaltungsort zugänglich.\\ 
In anderen Orten findet sich niemand, der bereit ist, etwas an der Synagoge zu machen oder Geld zu investieren. Viele der Kommunen, vor allem in Unterfranken, leiden darunter, dass die Leute wegziehen. Da gibt es viele leerstehende Häuser im Ort, die Infrastruktur bricht langsam zusammen und dann gibt es vielleicht noch eine Synagoge. Das wertzuschätzen, Energie hineinzustecken, um das Ortsleben neu beleben zu können, das findet man eigentlich selten. Die Synagoge ist da, aber andere Sachen sind uns jetzt einfach wichtiger. Ob da irgendwie antisemitische Ressentiments dahinterstecken, weiß ich nicht – das ist schwierig.\\ 
Interessant ist, wenn verschiedene Heimatforscher erzählen, wie sie in ihren Orten angefeindet werden. Für uns ist es ja in gewisser Weise komfortabel, wir gehen dahin, wir unterhalten uns mit den Leuten und sind dann wieder weg. Die, die da leben, sind damit konfrontiert, dass andere sie schief anschauen und immer sagen: "`die mit ihrem Judenzeug da"' Die pflegen zum Teil auch Kontakte zu Nachkommen von Überlebenden oder Migranten und laden sie dorthin ein und führen sie durch den Ort. Dann gucken die Deutschen: "`Was ist denn jetzt mit dene? Wollen die womöglich irgendwelche Rückerstattungsansprüche geltend machen?"'\\ 
Man hört auch Sachen wie: "`Meine Großeltern, die 1936 oder so einem Juden das Haus abgenommen haben, weil der halt wegmusste, haben natürlich einen fairen Preis bezahlt! Und nach dem Krieg wollten die nochmal Geld haben! Da fühlt man sich dann irgendwie schon beschissen."' Dieses Stereotyp, dass die Juden nach Geld gieren und immer auf ihren eigenen Vorteil schauen, kommt da schon durch. Das hört man immer wieder.\\ 
Man hört aber auch das Andere: "`Die Juden sind ja besonders geschäftstüchtig gewesen, haben dann eben viele erfolgreiche Geschäfte gehabt hier im Ort"', also auch hier diese Sonderstellung, sie hätten besondere Qualitäten und deswegen waren sie so erfolgreich und deswegen haben sie sich den Neid der anderen zugezogen. Ich habe den Eindruck, das es für viele Leute oft sehr schwierig ist, das so einzuordnen, dass das ganz normale Kaufleute waren, von denen einige gute Ideen hatten, wie sie bestimmte Vertriebsmodelle einfach ausbauen und dadurch auch sehr innovativ auf dem Land wirken konnten.\\ 
Man hört auch manchmal: "`Bis 1933 war das Zusammenleben in unserem Ort eigentlich wunderbar; die Christen und die Juden haben sich super verstanden."' Die Frage, warum das dann so plötzlich anders geworden ist, können die Leute dann nicht beantworten. "`Ja, das war halt Propaganda und die Leute waren abgeschnitten von Informationen! Das einzige, was es im Ort gegeben hat, war der Stürmerkasten. Wenn man da jahrelang geprägt ist, dann glaubt man dem irgendwann."'\\ 
Da ist sicherlich was dran, aber die Vorstellung "`du bist mein jüdischer Nachbar, dich kenne ich, du bist natürlich anders, das weiß ich schon, aber die Juden sonst, die sind problematisch und gegen die muss man was machen"' ist schon seltsam. 
 
\textbf{Zumindest in Mittelfranken war die evangelische Kirche schon ziemlich frühzeitig nationalistisch und antisemitisch geprägt. Hast du dafür irgendwelche Erklärungen?} 

\textbf{Axel Töllner:} In Akten aus dem 19. Jahrhundert finden sich überhaupt keine konfessionellen Unterschiede, was judenfeindliche Ressentiments angeht. Man könnte vielleicht sagen, dass im katholischen Bereich Legenden – Geschichte, dass die Juden Ritualmorde verüben würden und so etwas – verbreiteter waren.\\ 
Der Katholizismus hat durch die Priester, die tendenziell in der Bayerischen Volkspartei organisiert gewesen sind, aus parteipolitischen Gründen zunächst gegen die Nazis opponiert. Auch die Erfahrungen aus dem Kaiserreich haben bedingt, dass da eine gewisse Distanz da war. Es zeigt sich jetzt in Unterfranken, wobei das konfessionell total zersplittert ist, dass der Klerus, als der kirchliche Druck weg war, die Meinung vertreten hat, die Nationalsozialisten seien nicht wählbar für Katholiken, weil sie bestimmte christliche oder katholische Dogmen in Zweifel ziehen oder bestreiten. Sobald das wegbrach, setzte doch relativ schnell ein Nazifizierungsprozess ein.\\ 
Was zum Beispiel antisemitische Gewalt in den einzelnen Orten angeht, kann man heute nicht sagen, dass das in den evangelisch geprägten Orten exzessiver gewesen wäre als in den katholisch geprägten Orten. Ich denke, Mittelfranken hat natürlich mit Julius Streicher eine Galionsfigur gehabt, die das Ganze unglaublich hochgeschaukelt und die Situation für die Juden schon extrem früh zur Hölle gemacht hat. Otto Hellmuth als Gauleiter in Unterfranken hat längst nicht dieselben Möglichkeiten gehabt wie Streicher, obwohl auch der natürlich sehr rigide gewesen ist.\\  
Erklärungsmuster zu finden, ist wirklich sehr, sehr kompliziert. Viele Protestanten haben ihre Hoffnung in die NSDAP als "`evangelische Partei"' gesetzt.

\textbf{Hat das etwas mit Luther zu tun?} 

\textbf{Axel Töllner:} Die Zwanzigerjahre sind dafür ganz interessant. Da ist die Auffassung von Luther als Antisemit gar nicht so präsent, zumindest im klassisch-kirchlichen Protestantismus. Die völkische Bewegung – also Theodor Fritsch in seinem Antisemiten-Katechismus – hat verschiedene Lutherschriften zusammengestellt, um zu zeigen, dass Luther schon sehr früh erkannt hat, wie die Juden "`wirklich sind"'. Das hat sicherlich bei manchen dazu geführt, dass der Gedanke aufkam, das sei anschlussfähig. \\
Es gibt aber auch Stimmen, die sagen: "`Die Nazis instrumentalisieren Luther.'''  Die versuchen ihn noch irgendwie zu retten: "`Luther hat schon sehr heftige Sachen gegen die Juden gesagt, aber im Grunde ging es ihm ja um eine religiöse Auseinandersetzung und nicht darum, die Juden in jeder Hinsicht zu diffamieren."' Das sehe ich skeptischer als manche damals. Das muss man, denke ich, auch als Apologetik einordnen. "`Wir wollen die Deutungshoheit über Luther behalten."' Das hat sicherlich etwas damit zu tun, aber es gibt im Katholizismus auch solche vehementen Judenhasser. Das unterscheidet sich nicht sehr. 

\textbf{Nürnberg war gewissermaßen die protestantische Hochburg des Antisemitismus, Wien dagegen zu gleicher Zeit die katholische Hochburg des Antisemitismus. 
Jetzt ist gerade das Lutherjahr und es finde viele Veranstaltungen zu Luther statt. Es scheint aber relativ wenig zu geben, was über seine antisemitischen Veröffentlichungen und Haltungen aufklärt.} 

\textbf{Axel Töllner:} In der Forschung ist das inzwischen unübersehbar. Ich glaube, bei diesem Jubiläum wird das erstmals ein bisschen fundamentaler angegangen. Allerdings gibt es natürlich unterschiedliche Strömungen. Manchen geht es viel zu weit. Die sagen: "`Wenn wir uns so kritisch mit dem Luther auseinandersetzen, dann betreiben wir Geschichtsklitterung."' Andere sagen: "`Es geht nicht weit genug und man hört nicht genug darüber."'\\ 
Ich habe verschiedene Vorträge zu dem Thema gehalten und finde es total interessant, dass die immer gut besucht sind und ich wirklich auf ein großes Interesse stoße. Viele der Zuhörer sagen, dass sie das gar nicht gewusst hätten und dass das sehr erschütternd wäre. Dabei gibt es viele Gemeinden, sowohl Kirchengemeinden als auch Kommunen, die Veranstaltungen dazu anbieten. Es gibt auch verschiedene Wanderausstellungen zu dem Thema, die zum Beispiel in ehemaligen Synagogen ausgestellt werden. Manche Gemeinden machen auch kleine Veranstaltungsreihen zu dem ganzen Thema und beschäftigen sich damit. Ich würde sagen, das hat in den letzten Jahren zugenommen und die Fragen sind auch ernsthafter und selbstkritischer als früher.\\  
Die EKD-Synode hat im Herbst 2015 eine – wie ich finde – recht beachtenswerte Erklärung verabschiedet, wo sie sich aber nur – und das ist meine Kritik – auf ein paar wenige Stellen innerhalb von Luthers Werk beschränkt, also diese klassischen Forderungen, die Synagogen anzuzünden, die Juden zu vertreiben und zu vernichten, Zwangsarbeit und solche Dinge. Dieser Forderungskatalog ist furchtbar genug, aber das ist eben nur die Spitze des Eisbergs. Man kann sich nicht damit begnügen, zu sagen: "`Das ist furchtbar gewesen an Luther und war eine Entgleisung des späten Luther, aber ansonsten sind wir froh, dass er auch noch anderes gesagt hat."' Das geht sehr, sehr viel weiter und ich versuche auch immer den Fokus darauf zu legen, dass die Art und Weise, wie Luther die Bibel interpretiert hat, unser eigentliches Problem ist.\\ 
Gegen so exzessiven Judenhass werden sich schnell viele Leute stellen. Da sind wir uns schnell einig, da braucht man keine Vorträge zu halten, da braucht man auch keine Ausstellungen zu machen, sondern jeder wird das ekelhaft finden. Wenn es aber um die Frage geht, welches religiöse Recht die Juden in Luthers Werk haben und wie er sie insgesamt betrachtet – das müsste mehr in den Fokus treten. Bei Luther ist es nicht ganz einheitlich. Es gibt auch ein paar Stellen, die einen optimistischer stimmen, aber insgesamt finde ich es ziemlich ernüchternd. 
 
\textbf{Man liest heute noch manchmal die Positionierung, dass Christen Juden missionieren sollten. Wie verbreitet ist diese Haltung?} 

\textbf{Axel Töllner:} Die gibt es immer noch, aber auch die EKD-Synode hat im letzten November eine Erklärung veröffentlicht, wo sie das, was in den letzten Jahren als kirchlicher Konsens unter den Gremien und Kirchenleitungen gewachsen ist, zusammengefasst hat: "`Wir sind nicht dazu berufen, als Christen den Juden den Weg zum Heil zu weisen."' Das hat heftigen Widerspruch hervorgerufen. Die Frage: "`Darf ich dann als Christ einem Juden nicht mehr sagen, dass ich an Jesus glaube und dass Jesus für alle Menschen gut ist?"' ist damit nicht gemeint. Wenn ich mich mit Juden unterhalte und die etwas von mir wissen wollen, dann erzähle ich denen natürlich auch, was ich glaube, weil ich von ihnen auch erwarte, dass sie mir das erzählen, so wie sie das erleben. Ich verfolge aber nicht das Ziel, dass sie das genauso glauben müssen wie ich, und das ist der Unterschied. Jeder soll für seinen Glauben einstehen und seinen Standpunkt zeigen, aber mein Gegenüber muss auf derselben Augenhöhe stehen und dasselbe Recht haben. Der muss auch seinen Standpunkt sagen können und mich damit konfrontieren. \\ 
Es kommt natürlich darauf an, wie ich das mache. Ich würde mich nicht in Tel Aviv oder Jerusalem mit einem Stand auf die Straße stellen, aber wenn ich in Vorträgen gefragt werde: "`Sag mal was als Christ dazu"' oder wenn ich in Diskussionen, in denen wir als Juden und Christen zusammen über die Bibel reden, gefragt werde: "`Wie siehst du das als Christ?"', dann sage ich natürlich, wie ich das sehe. Dann sagen die, wie sie das sehen und wir stellen fest, da es vielleicht Gemeinsamkeiten gibt und dass es Unterschiede gibt. Damit habe ich kein Problem. Ich denke mir, Gott wird sich schon etwas dabei gedacht haben, dass er das so gemacht hat. Er hätte es ja auch einfacher haben können.  

\textbf{Hast du bei der Recherche zu den Synagogen Unterschiede zwischen den Generationen festgestellt bezüglich der Haltung zum Judentum oder dem Interesse bzw. Desinteresse?} 
 
\textbf{Axel Töllner:} Ich habe in allen Generationen engagierte und desinteressierte Leute getroffen – wobei mein Eindruck ist, dass in den Schulen das Thema jüdische Geschichte mittlerweile zum Heimat- und Sachkundeunterricht dazugehört. Ich verspreche mir davon, dass es selbstverständlicher ist für die jungen Leute. Das hilft, das als Realität anzuerkennen: "`Unseren Ort haben über Jahrhunderte hinweg Juden mitgeprägt."' Dass das Realität ist, dass zur Heimatgeschichte auch die jüdische Geschichte dazugehört, ist ein Unterschied im Denken zu früheren Generationen. Ich habe den Eindruck, dass es unter den älteren Leuten einige gibt, die mittlerweile auch schon gestorben sind, die sich mit der jüdischen Geschichte beschäftigt haben und irgendeine Art Wiedergutmachung erzielen. Die haben als Kinder oder Jugendliche vielleicht miterlebt, wie ihre jüdischen Klassenkameraden gehänselt und verfolgt wurden und haben vielleicht selbst mitgemacht. Die hatten dann hinterher das Gefühl: "`Ich muss da irgendwas machen."' Mit den Veranstaltungen 1988, 50 Jahre Novemberpogrome, haben einige Leute angefangen, sich mit der jüdischen Geschichte im Ort zu beschäftigen und haben dann sehr engagierte Sachen geschrieben. Die können zwar nicht unbedingt wissenschaftlichen Kriterien standhalten, aber es wird deutlich, dass die Leute von irgendeiner Schuld getrieben sind und versuchen, das auf irgendeine Weise wieder gutzumachen. 

\textbf{Vielleicht ist da Gunzenhausen ein ganz gutes Beispiel. Da gibt es eine Schule, die da ziemlich engagiert ist. Die haben Kontakt aufgenommen mit Nachfahren der Juden, die jetzt in den USA wohnen, haben einen Austausch organisiert und dadurch einen persönlichen Zugang geschaffen.}
 
\textbf{Axel Töllner:} Ja, das ist natürlich ein Beispiel für wirklich nachhaltige Arbeit. Es gibt ja auch Orte mit nur einer engagierten Lehrkraft. Wenn die wieder weg ist, dann schläft das wieder ein. Dass das in Gunzenhausen schon über viele Jahre hinweg gelingt, finde ich eine tolle Sache. Ich glaube, da haben die Kleinstädte auch gewisse Vorteile.\\  
Es gibt eine ganze Reihe von Beispielen, wo Schulklassen recherchiert haben und dann über die Heimatforscher an Überlebende oder deren Nachkommen gekommen sind und dadurch Informationen über die Menschen bekommen haben und zum Teil auch Leute Interesse bekommen haben, sich einmal anzuschauen, wo ihre Familie eigentlich herkommt. Da sind dann auch Kontakte entstanden, aber es hängt in der Tat von den einzelnen Leuten vor Ort ab, die da engagiert sind. 

\textbf{Inwiefern gibt es eigentlich noch die ganz archaischen antisemitischen Vorurteile wie zum Beispiel in dem Buch zum Thema 100 Jahre Wassertrüdingen aus den 80er Jahren? Da steht drin, dass die Juden deshalb Steine auf die Gräber legen, damit sie, wenn der Messias wiederkommt, ihn nochmal steinigen können. Gibt es so etwas immer noch?} 

\textbf{Axel Töllner:} So etwas habe ich schon immer wieder gehört. Volksglaube ist unglaublich nachhaltig. Das kommt, denke ich, auch von bestimmten Vorstellungen, die Juden als Sondergruppe sehen - "`die wollen unter sich bleiben"' und "`die haben so komische Bräuche und am Schabbat mussten dann immer die Christen zu ihnen kommen und den Ofen anschüren"'. Das hat es ja gegeben, aber die Vorstellung, dass Christen unter Juden dienen müssen, gibt das Gefühl, die Juden hätten sich für etwas Besseres gehalten. Das ist schon interessant, wie zäh solche Vorstellungen sind. 

\textbf{Zum Glück ist ein direkter, offener Antisemitismus in Deutschland nicht gesellschaftsfähig, sondern eher tabuisiert. Antisemitismus ist trotzdem weiterhin vorhanden, was wissenschaftlich untersucht ist – die Frage ist, wie dieser versteckte Antisemitismus in Erscheinung tritt.} 

\textbf{Axel Töllner:} Es gibt zum Beispiel immer wieder Friedhofsschändungen. Deswegen sind die Friedhöfe verschlossen und es ist manchmal schwierig, herauszufinden, woher man den Schlüssel dafür kriegen kann.  

\textbf{Wir haben ein Interview mit dem Rabbiner aus Fürth geführt, der sagt, dass antisemitische Briefe oder Mails häufig mit: "`Kritik unter Freunden muss doch wohl erlaubt sein"' eingeleitet werden. Und dann legen sie los.} 

\textbf{Axel Töllner:} Was ich auch immer wieder finde, ist die Aussage: "`Das war ja so schlimm damals, was die Nazis mit den Juden gemacht haben. Was die Israelis jetzt aber mit den Palästinensern machen, das ist doch eigentlich nicht sehr viel anders. Das ist doch eigentlich schlimm, die sollten es doch besser wissen."' Das ist sehr weit verbreitet und das höre ich auch oft nach Vorträgen, wenn Leute sich melden und fragen: "`Und, was sagen Sie denn jetzt dazu?"' So etwas wie "`Kritik unter Freunden"' ist tendenziell auch eine Generationenfrage. Das scheint mir eher die ältere Generation zu sein, die das sagt. Viele von denen haben angefangen, weil sie etwas gegen den Antisemitismus machen wollten und mit Juden freundlich reden wollten, aber letztlich das Gefühl hatten: "`Ganz das Richtige ist das Judentum nicht"'. 

\textbf{Da ist auch bei den deutschen Juden so eine Unsicherheit. Eine Art Selbsthass, weil ein "`anständiger Jude"' – sagen die Israelis zum Beispiel – "`ein Zionist ist und nicht auf den Gräbern seiner Vorfahren wohnt"'. Ich habe die Erfahrung gemacht: Israelis, die hier zu Besuch sind, wollen eigentlich mit den Nürnberger Juden nichts zu tun haben. Ein anständiger Zionist wohnt nun mal nicht in Nürnberg, sondern in Tel Aviv. Ich denke, auch bei den Nürnberger Juden, die ich so kenne, ist eine Unsicherheit bzw. Selbsthass: "`Eigentlich müsste ich ja in Israel wohnen"'. Die meisten machen dann einen Kompromiss und haben eine kleine Eigentumswohnung in Tel Aviv, damit sie sagen können: "`Ich wohne auch in Israel"'.}  

\textbf{Axel Töllner:} Das kommt noch dazu. Das jüdische Leben in Deutschland nach 1945 ist schon immer ein angefochtenes Leben gewesen. Die wurden von den Nicht-Juden genauso wie von Juden in Israel oder europäischen Ländern kritisiert, die für sich in Anspruch genommen haben: "`Wir haben wenigstens nicht die Shoah erfunden"'. Auf der anderen Seite nehme ich doch wahr, dass auch bei den Leuten, die aus der ehemaligen Sowjetunion zugewandert sind, die Älteren, die es ohnehin schwierig haben, die mit Deutschland erstmal nichts zu tun haben, die hier auch keine Wurzeln haben oder sich hier nicht in den Nachkriegswirren niedergelassen zu haben, diejenigen sind, die sich erst einmal zu diesem Land und zu diesem Staat verhalten müssen. Es ist interessant, dass viele von denen, die in den 90er Jahren als jüngere Leute gekommen sind, sehr dankbar sind dafür, dass sie hier einen Ort gefunden haben, wo sie leben können und auch sehr sensibel sind, was antisemitische Tendenzen angeht, weil sie das aus ihrer Kindheit und Jugend aus der Sowjetunion noch kennen. Die sagen: "`Deutschland ist für mich das Land, wo es diesen staatlichen Antisemitismus nicht gibt, weswegen ich hier auch gut sein kann. Klar habe ich viele Verwandte in Israel, aber da ist es mir zu heiß und da ist alles komisch, das ist so eine fremde Welt. Hier ist es mir doch noch lieber."' \\ 
Dann gibt es mittlerweile die Erfahrung, von der ich letztes Jahr in Israel gehört habe, dass viele junge Israelis ein ganz positives Deutschland-Bild haben und die Shoah jetzt von Deutschland abspalten und auf Polen projizieren wegen der Klassenfahrten in der 9./10. Klasse nach Polen, nach Auschwitz und in andere Vernichtungslager . Nach dem Motto: "`Das hat alles da stattgefunden. Polen ist das böse Land und die Polen sind die Bösen! Und die Deutschen sind aber unsere Freunde."' Das gibt dann zum Teil eine ganz komische Schlagseite. 

\textbf{Darum reagieren die polnischen Politiker dann gereizt, wenn von "`polnischen KZs"' die Rede ist.} 

\textbf{Axel Töllner:} Genau, das ist aber auch ein wirkliches Problem! Ich glaube, dass es auch Deutsche gibt, denen das gar nicht unlieb ist und die sagen: "`Das können wir jetzt outsourcen!"', um das mal ganz flapsig zu sagen. Umso wichtiger ist es dann, hier die Begegnungen zu haben.\\
Was ich auch interessant finde ist, dass in den letzten ca. 10 Jahren auch die Vereine in den Dörfern langsam ihre jüdischen Mitglieder wiederentdecken oder auch die Leute, die im späten 19. Jahrhundert diese Vereine mitgegründet haben - Turnverein, Freiwillige Feuerwehr, etc. Da gab es ja wirklich an vielen Orten Juden, die da ganz vorne mitgearbeitet haben. Auch wenn man sich neuere Festschriften anschaut, hat man Fotos, in denen dann Aussagen stehen wie: "`Unser jüdisches Mitglied so-und-so"'. Da verändert sich schon einiges. Es ist aber nach wie vor sehr unterschiedlich, je nachdem, wo man sich befindet.\\
Dabei fällt mir noch die Meiser-Debatte ein, bei der doch deutlich geworden ist, dass hier noch einiges schlummert, wobei Leute versuchen, das auf einer definitorischen Ebene zu lösen und zu sagen: "`Das ist Antisemitismus, das ist kein Antisemitismus"'. Das, finde ich, kann es eigentlich nicht sein. Das Niveau der Diskussionen fand ich zum Teil echt erschütternd. 

\textbf{Ja, am besten haben das die Münchner gelöst. Die haben die Meiserstraße in Katharina-von-Bora-Straße umbenannt, haben also eine Antisemitin für einen Antisemiten...} 

\textbf{Axel Töllner:} Wir wissen ja von ihr eigentlich zu wenig. Ich meine, es gibt eigentlich nur diese Geschichte, dass Luther in einem Brief an seine Frau 1546 schreibt, er sei an einem Dorf vorbeigefahren, wo Juden wohnen und der giftige Judenwind wäre ihm kalt ins Kreuz gefahren oder so ähnlich und er schrieb dann: "`Wenn du jetzt da gewesen wärst, würdest du sagen, es wäre das Werk der Juden gewesen."' So sinngemäß,  dass er sterbenskrank sei und die Juden ihn vergiftet hätten.\\
Die Namensumbenennungen haben auch ihre Probleme. Ich denke, man kann auch auf die Art und Weise Geschichte entsorgen. Wenn da diese Meiserstraße Meiserstraße heißt, dann ist sie für mich jedenfalls ein permanenter Pfahl im Fleisch, dann muss ich mich damit auseinandersetzen. Jetzt habe ich die Katharina von Bora, eine Frau, super. Ich habe jemanden, der nichts mit München zu tun hat, jemanden von vor 500 Jahren. Das ist auch eine Form der Geschichtsbewältigung, die ich problematisch finde. 

\textbf{Wäre es nicht besser, die Straße so zu belassen und eine Erläuterung hinzuzufügen?} 

\textbf{Axel Töllner:} Das hatte ja der Eckart Dietzfelbinger\footnote{Wissenschaftlicher Mitarbeiter im Dokumentationszentrum Reichsparteitagsgelände} hier in Nürnberg vorgeschlagen. Das hätte ich persönlich auch besser gefunden, aber der hat von einem antisemitischen Nationalprotestanten und einem nationalprotestantischen Antisemiten gesprochen. Da fühlte sich wiederum die Familie Meiser diffamiert. Da geht es eben wirklich um: "`Was ist jetzt antisemitisch?"' oder: "`Darf man jemanden Antisemiten nennen?"' Das hat in der Form nicht geklappt. Bis heute gibt es die Partei, die sagt, das sei alles ein Fehler gewesen. In München gab es eine Gemeinde, die einen Saal in ihrem Gemeindehaus danach in "`Hans-Meiser-Saal"' umbenannt hat, aus Trotz. Ich finde, dass die inhaltliche Auseinandersetzung mit dem Meiser da ein bisschen auf der Strecke geblieben ist. Die Suche nach einem Etikett, um das auf eine bündige Formel zu bringen, hat manches damals schwierig gemacht. \\
Für mich persönlich war der Meiser ein Antisemit. Das Problem ist nur: Wenn ich diesen Satz sage, habe ich damit noch nicht sehr viel gesagt, denn die einen Leute verstehen da eine Sache darunter und die anderen verstehen etwas ganz Anderes. Das ist, finde ich, nach wie vor schwierig auf den Punkt zu bringen, zu erklären, warum der Meiser Antisemit gewesen ist. Und dass Antisemit nicht gleich Antisemit bedeutet. Das ist etwas, dass man auch ganz schwer vermitteln kann, dass der Antisemitismus sehr viele Spielarten hat und auch die Frage, dass jemand antisemitische Denkstrukturen haben kann, ohne dass er jetzt in jedem zweiten Satz sagt: "`Die Juden sind unser Unglück!"' Ich sage das jetzt mal bisschen plakativ. Es gibt da subtile Formen, die bestimmte, jahrhundertealte Denkmuster weiter tradieren und in diese Schiene hineingehören. Das finde ich schwer zu vermitteln. Zumindest habe ich damit gerade in gutkirchlichen Kreisen meine Probleme, mich da verständlich zu machen. Auch jemand, der ein Antisemit ist, kann etwas getan haben, das aus unserer heutigen Sicht als gut bezeichnet werden kann. Fällst du mit dem Urteil, jemand war ein Antisemit, ein Totalurteil über das ganze Leben von ihm oder kann man sagen: "`Der war Antisemit, aber wir haben ihm auch in mancher Hinsicht etwas zu verdanken"'? 
 
\textbf{Das war in Polen sehr extrem. Da gab es zum Beispiel die Żegota, eine antisemitische polnische Organisation, derer Mitglieder Juden gerettet haben und dadurch ihr Leben aufs Spiel gesetzt haben. Sie haben gesagt: "`Ja, Juden raus aus Polen, Juden nach Madagaskar – aber umbringen? Nein."'} 

\textbf{Axel Töllner:} Je intensiver man sich damit beschäftigt, desto vielschichtiger werden die Graustufen, die man wahrnimmt. Dass ein bisschen bündiger zu vermitteln mit einer Erklärungstafel an einem Straßenschild, ist schwierig. Ich finde aber, das wäre eigentlich das Ziel. 

\textbf{Ein positives Beispiel: In der vorderen Ledergasse in Nürnberg gibt es ein Kriegerdenkmal für alle möglichen gefallenen Nürnberger im Boxeraufstand in China und im Herero-Aufstand. Zu den Herero und zum Boxeraufstand gibt es dann darunter noch eine Tafel. Das ist gut gelöst.} 

\textbf{Axel Töllner:} Das finde ich auch. Es gehört auch zur Geschichte Nürnbergs dazu, dass der Stadtrat damals beschlossen hat, die Spitalgasse nach dem Nürnberger Hans Meiser zu benennen. Ich finde auch, wir können uns nicht nur die Sahnestückchen heraussuchen und sagen: "`Das ist prima und den Rest, den schieben wir ganz nach hinten, wo ihn niemand finden kann, ins Archiv beispielsweise."' Ich denke, heutzutage könnte man mit elektronischen Medien, mit Apps sicherlich auch einiges machen. Zum Beispiel irgendeinen QR-Code, damit die Leute ein paar weitergehende Informationen bekommen, wenn sie sich dafür interessieren und mehr darüber erfahren möchten oder es noch nicht verstanden haben. Das wäre vielleicht noch eine Möglichkeit. \\
Es gibt auch eine ganze Reihe von Fotos und Zeichnungen, wo man zumindest noch die Häuser erkennen kann, die zum jüdischen Viertel in Nürnberg gehört haben. Ich finde ganz interessant, dass es in den 1980ern oder 90ern einmal Grabungen in der Frauenkirche gab, bei denen auch ein Sockel von einer Säule der Synagoge gefunden wurde. Dann haben die Archäologen von dem Säulenabdruck sagen können, wie hoch die Säule wahrscheinlich war und wie groß die Synagoge dann wohl gewesen ist. Ein bisschen vergleichbar mit der in Regensburg, die man ein bisschen von diesem Altdorfer Stich kennt. So etwas noch sichtbarer zu machen, wäre eine tolle Sache. Es gibt viele schöne Sachen, die man machen könnte, bis dahin, alles was da so unter dem Hauptmarkt ist, einmal richtig archäologisch zu untersuchen. Das wäre eigentlich eine Pflicht, die der Stadt des Friedens und der Menschenrechte gut anstehen würde. 
\end{otherlanguage}