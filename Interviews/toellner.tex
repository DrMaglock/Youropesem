\section{Dr Axel Töllner}

Dr. Axel Töllner (* 1968) is a Protestant pastor and, since 2014, Federal Church Commissioner for Christian-Jewish Dialogue in Bavaria, Germany. After graduating from high school in Munich in 1987, he completed his civil service and studied Protestant theology in Erlangen, Kiel and Jerusalem. During his studies he completed additional training in Christian journalism and journalism. From 2003 to 2011 he worked as a pastor in various communities. He received his doctorate in 2007 at the University of Koblenz-Landau and then worked until 2010 as a research associate at the project Synagogues-Memorial Band Bavaria at the Chair of Modern Church History at the Friedrich-Alexander-University Erlangen-Nuremberg. From 2010 to 2011 he worked as a research associate at the Working Group Catholicism and Protestantism Research for the NS Documentation Center Munich, then until 2014 again at the Synagogue Memorial Band Bavaria.  \\
The interview with him took place on April 21st, 2017 in Nuremberg.  

 

Dr. Axel Töllner: Bei unseren Recherchen zu einem Projekt mit etwa 200 jüdischen Gemeinden in Unterfranken sind wir auf das Problem gestoßen, dass die Quellenlage zu diesen vielen kleinen Orten total disparat ist. In einem Ort gibt es unendlich viel und im anderen Ort gar nichts. Ein anderes Phänomen, auf das wir gestoßen sind, ist dass die Leute im einen Ort Leute schon relativ früh angefangen haben, sich mit ihrer jüdischen Geschichte zu beschäftigen, und in anderen Orten scheint bis heute irgendwie ein Deckel drauf zu sein. Aber das folgt auch keiner Gesetzmäßigkeit, das lässt sich keiner bestimmten Region zuordnen, im Nachbarort kann das schon wieder ganz anders sein. Das ist auch sehr abhängig von einzelnen Personen – oft sind das Leute, die da nicht groß geworden, sondern zugezogen sind. 

Das ist auch ganz interessant, wenn man sich Ortschroniken anschaut, oder Vereinschroniken oder Ähnliches, da sind meistens nur kurze Notizen: 1933 wurde „Der-und-der“ Vorsitzender und der nächste Eintrag ist dann erst 1945 – die ganzen Heimkehrer oder Flüchtlinge, die dann dazugekommen sind und dem Vereinsleben eine ganz neue Richtung gegeben haben… Die ganzen Ausschlüsse durch irgendwelche Arier-Paragraphen, solche Geschichten, das wurde nicht dokumentiert. 
Es ist auch oft schwierig an das Archivgut der Gemeinden ran zu kommen, weil die Gemeindearchive zum Teil nicht geordnet sind. Ich habe von einem Ort gehört, wo der frühere Bürgermeister sich mit der jüdischen Geschichte des Ortes beschäftigt hat und eine Publikation dazu geschrieben hat. Von seinen Quellen war dann aber überhaupt nichts mehr auffindbar, und jetzt haben sie in den letzten Jahren erst hinten in der Ecke des Gemeindearchivs den ganzen Stapel gefunden, den hat der einfach beiseitegeschafft, er hat das Material also nicht entsorgt, sondern im Archiv versteckt, und das ist eigentlich ein perfektes Versteck. 

 
Man stößt ja bei solchen Recherchen immer noch auf sehr starke antisemitische Ressentiments.  
Es gab mal die Initiative eines Juden, der in den USA lebt, der einen ziemlich hohen Geldbetrag spenden wollte, für eine Gedenktafel, die an die Synagoge und an die Juden erinnert und da war der ganze Gemeinderat dagegen. Vielleicht kannst du mal deine Erfahrungen an ein paar Beispielen anführen? 

Dr. Axel Töllner: Also, was mir immer wieder auffällt ist, dass selbst die Leute, die sich mit der jüdischen Ortgeschichte beschäftigen und sagen, dass das wichtig ist, immer von den Deutschen reden und von den Juden. Das ist also eine Sondergruppe, die anders ist als die Anderen. Dass die seit Jahrhunderten in dem Dorf gelebt haben, das Dorf mitgeprägt haben, das ist im Bewusstsein nicht so eingedrungen. Das finde ich sehr interessant, dass selbst bei den Leuten, die sich bemühen, die jüdische Geschichte ins Gedächtnis zu rufen, noch die Vorstellung da ist, dass die Juden irgendwie die Anderen sind.  
Und ich kann mich an ein Forschungsprojekt erinnern, auch aus einem unterfränkischen Dorf, mit den Schulen und den Kirchengemeinden vor Ort: jüdisches Leben in unserem Dorf. Das mündete dann in einer Broschüre und es wurden Stolpersteine verlegt. Und ein Künstler im Ort hat versucht, von der Synagoge, die 1938 zerstört wurde, eine schöne Zeichnung zu machen, basierend auf historischen Fotos. Und interessant ist, wie er die Leute dargestellt hat, die in die Synagoge reingehen. Das sind nämlich Leute, die gekleidet sind wie Ultraorthodoxe mit Schläfenlocken, mit schwarzen Hüten und Kaftanen und so etwas. Das ist überhaupt nicht durch historische Fotos gedeckt, sondern die Fotos zeigen Leute, die gekleidet sind wie alle anderen auch. Ich finde es interessant, dass sich dieses Bild so festgesetzt hat. 
Gelegentlich wurde ich an verschiedenen Orten mit der Haltung konfrontiert „das alte Judenzeugs, das ist doch lange vorbei, das interessiert doch keinen mehr“ usw. - da gibt es also Leute, die scheinbar nicht wollen, dass man da was aufrührt. Ich habe auch mit verschiedenen Heimatforschern gesprochen, die sich zum Teil seit vielen Jahren auf diesem Gebiet engagiert beschäftigen. Die erzählen dann auch – besonder wenn es darum geht, wer ein Haus gekauft hat, in dem früher eine jüdische Familie gelebt hat – unter welchen Umständen man an dieses Haus gelangt ist, oder an irgendwelche Möbelstücke. Das ist immer noch schwierig.  
Ich habe den Eindruck, die heutigen Leute identifizieren sich so sehr mit ihrer Großelterngeneration – die wissen schon, dass da irgendwas nicht mit rechten Dingen zuging, aber „das sind ja unsere Verwandten, die werden es schon richtiggemacht haben.“ 

Wie gehen eigentlich die Gemeinden mit Synagogen um, die noch erhalten sind? 

Dr. Axel Töllner: Das ist ganz unterschiedlich. Es gibt Synagogen, die restauriert oder wieder zugänglich gemacht wurden. Ein ganz schönes Beispiel ist die ehemalige Synagoge in Arnstein – ungefähr 30 Kilometer nördlich von Würzburg – eine wunderbare Synagoge: frühes 19. Jahrhundert, klassizistisch – die hat im frühen 20. Jahrhundert noch eine ganz tolle Jugendstilbemalung bekommen, neben dem früheren Toraschrein, der zerstört worden ist. Oben rechts und links waren Greife aufgemalt, und jeder dieser Greife hatte ein Wappen in der Hand - das eine ist weiß-blau und das andere ist schwarz-weiß-rot. Ein wunderbares Beispiel für das Selbstverständnis der Juden. 
Die Stadt hat diese Synagoge vor einiger Zeit gekauft, weil sie vorher in Privatbesitz war, sie war zu Wohnraum umgebaut worden und durch die nachträglich eingebauten Zwischengeschosse sind ein paar Sachen kaputtgegangen, aber es ist im Grunde noch relativ viel von der Bemalung erhalten – 2010 wurde die Synagoge saniert und ist seitdem als Museum und Veranstaltungsort zugänglich.  
In anderen Orten findet sich niemand, der bereit ist, etwas an der Synagoge zu machen oder Geld zu investieren. Viele der Kommunen, vor allem in Unterfranken, leiden darunter, dass die Leute wegziehen – da gibt es viele leerstehende Häuser im Ort, die Infrastruktur bricht langsam zusammen und dann gibt es vielleicht noch eine Synagoge - das wertzuschätzen und da Energie rein zu stecken und das Ortsleben neu beleben zu können, das findet man eigentlich selten. Die Synagoge ist da, aber andere Sachen sind uns jetzt einfach wichtiger. Ob da irgendwie antisemitische Ressentiments dahinterstecken, das weiß ich nicht – das ist schwierig. 
Interessant ist, was verschiedene Heimatforscher erzählen, wie sie in ihren Orten angefeindet werden – für uns ist es ja in gewisser Weise komfortabel, wir gehen dahin, wir unterhalten uns da mit den Leuten und sind dann wieder weg. Und die, die da leben, sind damit konfrontiert, dass andere sie schief anschauen und immer sagen „die mit ihrem Judenzeug da“ und die zum Teil auch Kontakte zu Nachkommen von Überlebenden oder Migranten pflegen und die dorthin einladen und durch den Ort führen, wo dann die Deutschen gucken, „was ist denn jetzt mit denen und wollen die womöglich irgendwelche Rückerstattungsansprüche geltend machen?“ 
Und man hört auch so Sachen wie „meine Großeltern, die haben damals das Haus 1936 oder so einem Juden abgenommen, weil der halt wegmusste, die haben natürlich einen fairen Preis bezahlt! Und nach dem Krieg wollten die nochmal Geld haben, da fühlt man sich dann irgendwie schon beschissen!“, Dieses Stereotyp, dass die Juden irgendwie nach Geld gieren, dass die immer auf ihren eigenen Vorteil schauen und sowas, das kommt da schon durch, das hört man immer wieder mal.  
Aber man hört auch das Andere, „die Juden, die sind ja besonders geschäftstüchtig gewesen, haben dann eben viele erfolgreiche Geschäfte gehabt hier im Ort“, also auch hier diese Sonderstellung, sie hätten besondere Qualitäten und deswegen waren sie so erfolgreich und deswegen haben sie sich den Neid der anderen zugezogen. Ich habe den Eindruck, das ist für viele Leute oft sehr schwierig, das so einzuordnen, dass das ganz normale Kaufleute waren, von denen einige gute Ideen hatten, wie sie bestimmte Vertriebsmodelle einfach ausbauen und dadurch auch sehr innovativ auf dem Land wirken konnten. 
Man hört auch mal „bis 1933 war das Zusammenleben in unserem Ort eigentlich wunderbar; die Christen und die Juden haben sich super verstanden.“ Die Frage, warum das dann so plötzlich anders geworden ist, können die Leute dann nicht beantworten. „Ja, das war halt Propaganda und die Leute waren abgeschnitten von Informationen! Das einzige, was es im Ort gegeben hat war der Stürmerkasten und wenn man da jahrelang geprägt ist, dann glaubt man dem irgendwann.“ 
Da ist ja sicherlich was dran, aber die Vorstellung „du bist mein jüdischer Nachbar, dich kenne ich, du bist natürlich anders, das weiß ich schon, aber die Juden sonst, die sind problematisch und gegen die muss man was machen.“, das ist schon seltsam. 

 
Zumindest in Mittelfranken war ja die evangelische Kirche schon ziemlich frühzeitig nationalistisch und antisemitisch geprägt, hast du da irgendwelche Erklärungen dafür? 

Dr. Axel Töllner: In Akten aus dem 19. Jahrhundert finden sich überhaupt keine konfessionellen Unterschiede, was judenfeindliche Ressentiments angeht. Man könnte vielleicht sagen, dass im katholischen Bereich Legenden – Geschichte, dass die Juden Ritualmorde verüben würden und sowas – verbreiteter waren. 
Der Katholizismus hat durch die Priester, die tendenziell in der bayrischen Volkspartei organisiert gewesen sind, aus parteipolitischen Gründen zunächst gegen die Nazis opponiert. Und auch die Erfahrungen aus dem Kaiserreich haben bedingt, dass da eine gewisse Distanz da war, aber es zeigt sich jetzt in Unterfranken, wobei das konfessionell total zersplittert ist, dass der Klerus, als der kirchliche Druck weg war, die Meinung vertreten hat, die Nationalsozialisten seien nicht wählbar für Katholiken, weil sie bestimmte christliche oder katholische Dogmen in Zweifel ziehen oder bestreiten. Sobald das wegbricht, setzt doch relativ schnell ein Nazifizierungsprozess ein. Was zum Beispiel antisemitische Gewalt in den einzelnen Orten angeht, kann man heute nicht sagen, dass das in den evangelisch geprägten Orten exzessiver wäre als in den katholisch geprägten Orten. Ich denke, Mittelfranken hat natürlich mit Julius Streicher auch noch eine Gallionsfigur gehabt, der hat insgesamt das Ganze unglaublich hochgeschaukelt und die Situation für die Juden wirklich schon extrem früh zur Hölle gemacht. Otto Hellmuth als Gauleiter in Unterfranken hat längst nicht diese Möglichkeiten gehabt wie Streicher, obwohl auch der natürlich auch schon sehr rigide gewesen ist.  
Also - Erklärungsmuster zu finden, das ist wirklich sehr, sehr kompliziert und viele Protestanten haben ihre Hoffnung in die NSDAP gesetzt, als „evangelische Partei“ und so... 

Hat das vielleicht auch was mit Luther zu tun? 

Dr. Axel Töllner: Die Zwanzigerjahre sind dafür ganz interessant, da ist die Auffassung von Luther als Antisemit gar nicht mal so präsent, zumindest im klassisch-kirchlichen Protestantismus.  
Die völkische Bewegung – also Theodor Fritsch in seinem Antisemiten-Katechismus – hat verschiedene Lutherschriften zusammengestellt, um zu zeigen, dass Luther schon sehr früh erkannt hat, wie die Juden „wirklich sind“ und so… Das hat sicherlich bei manchen dazu geführt, dass der Gedanke aufkam, das sei anschlussfähig.  

Es gibt aber auch Stimmen, die sagen, „die Nazis instrumentalisieren Luther und versuchen ihn dann noch irgendwie zu retten – Luther hat schon sehr heftige Sachen gegen die Juden gesagt, aber im Grunde ging es ihm ja um eine religiöse Auseinandersetzung und nicht darum, die Juden da in jeder Hinsicht zu diffamieren“, da würde ich heute sagen, das sehe ich skeptischer als manche damals, aber das muss man, denke ich, auch als Apologetik einordnen. Wir wollen die Deutungshoheit über Luther behalten. Das hat sicherlich auch etwas damit zu tun, aber es gibt im Katholizismus auch genau solche vehementen Judenhasser – das unterscheidet sich nicht sehr. 

Das ist sehr interessant – Nürnberg war ja gewissermaßen die protestantische Hochburg des Antisemitismus, Wien dagegen zu gleicher Zeit die katholische Hochburg des Antisemitismus. Das scheint regional sehr unterschiedlich zu sein und von verschiedensten Faktoren abzuhängen. 
Jetzt ist ja gerade das Lutherjahr und es finde viele Veranstaltungen zu Luther statt. Luther wird zwar nicht gefeiert wegen seinem Rassismus, aber es relativ wenig zu geben, was über seine antisemitischen Veröffentlichungen und Haltungen aufklärt, oder?  

Dr. Axel Töllner: In der Forschung ist das inzwischen unübersehbar. Ich glaube, bei diesem Jubiläum wird das erstmals ein bisschen fundamentaler angegangen. Allerdings gibt es natürlich unterschiedliche Strömungen – manchen geht es viel zu weit, die sagen „Wenn wir uns so kritisch mit dem Luther auseinandersetzen, dann betreiben wir Geschichtsklitterung“ und andere sagen „es geht nicht weit genug und man hört nicht genug darüber.“ Ich habe verschiedene Vorträge zu dem Thema gehalten und finde es total interessant, dass die immer gut besucht sind und ich wirklich auf ein großes Interesse stoße. Viele der Zuhörer sagen, das hätten sie gar nicht gewusst, das wäre sehr erschütternd. Dabei gibt es viele Gemeinden, sowohl Kirchengemeinden als auch Kommunen, die Veranstaltungen dazu anbieten, es gibt auch verschiedene Wanderausstellungen zu dem Thema, die zum Beispiel in ehemaligen Synagogen ausgestellt werden. Manche Gemeinden machen auch kleine Veranstaltungsreihen zu dem ganzen Thema und beschäftigen sich damit – ich würde sagen, das hat in den letzten Jahren zugenommen und die Fragen sind auch ernsthafter und selbstkritischer als früher.  
Die EKD-Synode hat im Herbst 2015 eine – wie ich finde – recht beachtenswerte Erklärung verabschiedet, wo sie sich nicht nur – und das ist meine Kritik – auf ein paar wenige Stellen innerhalb von Luthers Werk beschränkt, also diese klassischen Forderungen, die Synagogen anzuzünden, die Juden zu vertreiben und zu vernichten Zwangsarbeit und solche Dinge. Dieser Forderungskatalog, der ist furchtbar genug, aber das ist eben nur die Spitze des Eisbergs und man kann sich nicht damit begnügen, zu sagen, „das ist furchtbar gewesen an Luther und war eine Entgleisung des späten Luthers, aber ansonsten sind wir froh, dass er auch noch anderes gesagt hat“. Das geht sehr, sehr viel weiter und da versuche ich auch immer den Fokus darauf zu legen, dass die Art und Weise, wie Luther die Bibel interpretiert hat, unser eigentliches Problem ist. 
Gegen so exzessiven Judenhass werden sich schnell viele Leute stellen. Da sind wir uns schnell einig, da braucht man keine Vorträge zu halten, da braucht man auch keine Ausstellungen zu machen, sondern jeder wird das ekelhaft finden. Aber wenn es um die Frage geht, welches religiöse Recht die Juden in Luthers Werk haben und wie er sie insgesamt betrachtet – das müssten mehr in den Fokus treten. Also bei Luther ist es nicht ganz einheitlich, es gibt auch ein paar Stellen, die einen optimistischer stimmen, aber insgesamt, finde ich es ziemlich ernüchternd. 

 
Man liest ja manchmal heute noch die Positionierung, dass Christen Juden missionieren sollten. Wie verbreitet ist diese Haltung? 

 
Dr. Axel Töllner: Die gibt es immer noch, aber auch die EKD-Synode hat im letzten November eine Erklärung veröffentlicht, wo sie das, was in den letzten Jahren als kirchlicher Konsens unter den Gremien und Kirchenleitungen gewachsen ist, zusammengefasst hat: „wir sind nicht dazu berufen, als Christen den Juden den Weg zum Heil zu weisen.“ Das hat heftigen Widerspruch hervorgerufen. Die Frage, „darf ich dann als Christ einem Juden nicht mehr sagen, dass ich an Jesus glaube, und dass Jesus für alle Menschen gut ist?“ – das ist damit nicht gemeint. Wenn ich mich mit Juden unterhalte, und die etwas von mir wissen wollen, dann erzähle ich denen natürlich auch, was ich glaube – weil ich von ihnen auch erwarte, dass sie mir das erzählen, so wie sie das erleben. Aber ich verfolge nicht das Ziel, dass sie das genauso glauben müssen, wie ich, und das ist der Unterschied. Jeder soll für seinen Glauben einstehen seinen Standpunkt zeigen, aber mein Gegenüber muss auf derselben Augenhöhe stehen und das selbe Recht haben. Der muss auch seinen Standpunkt sagen können und mich damit konfrontieren, so verstehe ich das.  
Aber es kommt natürlich drauf an, wie ich das mache. Ich würde ich mich jetzt nicht in Tel Aviv oder Jerusalem mit einem Stand auf die Straße stellen – aber, wenn ich in Vorträgen gefragt werde „Sag mal was als Christ dazu“ oder, wenn ich in Diskussionen, in denen wir als Juden und Christen zusammen über die Bibel reden, gefragt werde „Wie siehst du das als Christ?“, dann sage ich natürlich, wie ich das sehe. Und dann sagen die, wie sie das sehen und wir stellen fest, da gibt es vielleicht Gemeinsamkeiten und da gibt es auch Unterschiede und damit habe ich kein Problem. Und ich denke mir, Gott wird sich schon was dabei gedacht haben, dass er das so gemacht hat, er hätte es ja auch einfacher haben können.  

 
Nochmal zu Ihren Erfahrungen bei der Recherche zu den Synagogen – haben Sie Unterschiede zwischen den Generationen festgestellt bezüglich der Haltung zum Judentum oder dem Interesse und Desinteresse? 

 
Dr. Axel Töllner: Ich habe in allen Generationen engagierte und desinteressierte Leute getroffen – wobei mein Eindruck ist, dass in den Schulen das Thema jüdische Geschichte mittlerweile zum Heimat- und Sachkundeunterricht dazugehört. Ich verspreche mir davon, dass es selbstverständlicher ist für die jungen Leute. Ob daraus ein Fazit erwächst, da würde ich jetzt mal nicht zu optimistisch sein – aber zumindest, dass es dabei hilft, das mal als Realität anzuerkennen! Unseren Ort haben über Jahrhunderte hinweg Juden mitgeprägt. Dass das Realität ist, dass zur Heimatgeschichte auch die jüdische Geschichte dazu gehört, das ist nochmal ein Unterschied im Denken zu früheren Generationen. Ich habe den Eindruck, es gibt unter den älteren Leuten einige, die jetzt mittlerweile auch schon gestorben sind, die sich mit der jüdischen Geschichte beschäftigt haben, auch einige, die erzielen irgendeine Art Wiedergutmachung. Die haben als Kinder oder Jugendliche vielleicht miterlebt, wie ihre jüdischen Klassenkameraden gehänselt und verfolgt wurden, haben vielleicht selber mitgemacht. Die hatten dann hinterher das Gefühl „Ich muss da irgendwas machen“ – mit den Veranstaltungen 1988, 50 Jahre Novemberpogrome, haben einige Leute angefangen, sich mit der jüdischen Geschichte im Ort zu beschäftigen und haben dann sehr engagierte Sachen geschrieben. Die können zwar nicht unbedingt wissenschaftlichen Kriterien standhalten, aber es wird deutlich, dass die Leute von irgendeiner Schuld getrieben sind und versuchen, das auf irgendeine Weise wieder gut zu machen. 

 
Vielleicht ist da Gunzenhausen ein ganz gutes Beispiel? Da gibt es eine Schule, die sind da ziemlich engagiert - die haben Kontakt aufgenommen mit Nachfahren der Juden, die jetzt in den USA wohnen, haben einen Austausch organisiert und dadurch einen persönlichen Zugang geschaffen. 

 
Dr. Axel Töllner: Ja, das ist natürlich ein Beispiel für wirklich nachhaltige Arbeit. Es gibt ja auch Orte, da ist eine engagierte Lehrkraft, und wenn die wieder weg ist, dann schläft das wieder ein. Aber dass das dort schon über viele Jahre hinweg gelingt, das finde ich eine tolle Sache. Ich glaube, da haben die Kleinstädte auch gewisse Vorteile.  
Es gibt eine ganze Reihe von Beispielen, wo Schulklassen recherchiert haben und dann über die Heimatforscher an Überlebende oder deren Nachkommen gekommen sind und dadurch Informationen über die Menschen bekommen haben und zum Teil auch Leute Interesse bekommen haben, sich mal anzuschauen, wo ihre Familie eigentlich herkommt, wo dann auch Kontakte entstanden sind. Aber es hängt in der Tat von den einzelnen Leuten vor Ort ab, die da engagiert sind. 

 
Inwiefern gibt es eigentlich noch die ganz archaischen, antisemitischen Vorurteile, zum Beispiel in dem Buch zum Thema 100 Jahre Wasserthüringen aus den 80er Jahren? Da steht drin, die Juden legen deshalb die Steine auf die Gräber, damit, wenn der Messias wiederkommt, sie ihn nochmal steinigen können. Gibt es sowas immer noch? 

 
Dr. Axel Töllner: Sowas habe ich schon immer wieder mal gehört. Volksglaube ist unglaublich nachhaltig, das kommt, denke ich, auch von bestimmten Vorstellungen, die Juden als Sondergruppe sehen - „die wollen unter sich bleiben“ und „die haben so komische Bräuche“ und am Shabbat mussten dann immer die Christen zu ihnen kommen und den Ofen anschüren und sowas. Natürlich, das hat es ja gegeben – aber die Vorstellung, dass Christen unter Juden dienen müssen, das gibt das Gefühl, die Juden hätten sich für etwas Besseres gehalten und das merkt man schon auch noch immer wieder. Das ist schon interessant, wie zäh solche Vorstellungen sind. 

 
Jetzt ist ja zum Glück ein direkter, offener Antisemitismus in Deutschland – im Gegensatz zu Polen, Ungarn, Österreich und vielen anderen Ländern – nicht gesellschaftsfähig, sondern eher tabuisiert. Antisemitismus ist trotzdem weiterhin vorhanden, was wissenschaftlich untersucht ist – die Frage ist, wie dieser versteckte Antisemitismus in Erscheinung tritt. 

 
Dr. Axel Töllner: Es gibt zum Beispiel immer wieder Friedhofsschändungen, deswegen sind die Friedhöfe verschlossen, und es ist manchmal schwierig, herauszufinden, woher man den Schlüssel dafür kriegen kann.  

 Wir haben auch ein Interview mit dem Rabbiner aus Fürth geführt, und der sagt, typisch für ihn seien, auch während der Woche der Brüderlichkeit, antisemitische Tendenzen unter Freunden. die Christen leiten das dann immer ein, er hat da Stapelweise Briefe, mit „Kritik unter Freunden muss doch wohl erlaubt sein“, und dann legen sie los. 

 
Dr. Axel Töllner: Was ich auch immer wieder finde, ist die Aussage „Das war ja so schlimm damals, was die Nazis mit den Juden gemacht haben, aber jetzt, was die Israelis mit den Palästinensern machen, das ist doch eigentlich nicht sehr viel anders. Das ist doch eigentlich schlimm, die sollten es doch besser wissen.“ und so. Das ist sehr weit verbreitet und das höre ich auch oft nach Vorträgen, wenn Leute sich melden und fragen „Und, was sagen Sie denn jetzt dazu?“ – und natürlich, sowas wie „Kritik unter Freunden“, da würde ich tendenziell sagen, das ist auch eine Generationenfrage – das scheint mir doch eher die ältere Generation zu sein, die das sagt, weil viele von denen haben angefangen in den Gesellschaften, weil sie natürlich was gegen den Antisemitismus machen wollten und mit Juden freundlich reden wollten, aber letztlich das Gefühl hatten „ganz das richtige ist das Judentum nicht“. 

 
Da ist doch auch bei den deutschen Juden so eine Unsicherheit. Eine Art Selbsthass, weil ein „anständiger Jude“ – sagen die Israelis zum Beispiel – ein Zionist ist und nicht auf den Gräbern seiner Vorfahren wohnt. Ich habe die Erfahrung gemacht: Israelis, die hier zu Besuch sind, wollen eigentlich mit den Nürnberger Juden nichts zu tun haben. Ein anständiger Zionist wohnt nun mal nicht in Nürnberg, sondern in Tel Aviv. Und ich denke, da ist auch bei den Nürnberger Juden, die ich so kenne, eine Unsicherheit bzw. Selbsthass – „eigentlich müsste ich ja in Israel wohnen“. Die meisten machen dann einen Kompromiss und haben eine kleine Eigentumswohnung in Tel Aviv, damit sie sagen können „ich wohne auch in Israel“.  

 
Dr. Axel Töllner: Das kommt noch dazu. Das jüdische Leben in Deutschland nach 1945 ist schon immer ein angefochtenes Leben gewesen. Die wurden von den Nicht-Juden genauso wie von Juden in Israel oder europäischen Ländern kritisiert, die für sich in Anspruch genommen haben „Wir haben wenigstens nicht die Shoah erfunden“. Auf der anderen Seite nehme ich doch wahr, dass es auch bei den Leuten, die aus der ehemaligen Sowjetunion zugewandert sind, die älteren, die es ohnehin schwierig haben, die mit Deutschland erstmal nichts zu tun haben – und hier auch keine Wurzeln haben oder sich hier nicht in den Nachkriegswirren niedergelassen zu haben – die sich erstmal zu diesem Land und zu diesem Staat verhalten müssen. Wobei es interessant ist, dass diejenigen, die in den 90er Jahren als jüngere Leute gekommen sind, dass viele von denen sehr dankbar sind dafür, dass sie hier einen Ort gefunden haben, wo sie leben können – auch sehr sensibel sind, was so antisemitische Tendenzen angeht – wenn man weiß, dass sie das aus ihrer Kindheit und Jugend aus der Sowjetunion noch kennen und eigentlich sagen „Deutschland ist für mich das Land, wo es diesen staatlichen Antisemitismus nicht gibt, und weswegen ich hier auch gut sein kann – und klar, ich hab viele Verwandte in Israel, aber da ist es mir zu heiß und da ist alles komisch, das ist so eine fremde Welt – hier ist es mir doch noch lieber.“ 

 
Dr. Axel Töllner: Und dann gibt es mittlerweile die Erfahrung, das hab ich in Israel gehört, letztes Jahr, es gibt ja die Klassenfahrten in der 9./10. Klasse nach Polen, Auschwitz und andere Vernichtungslager, und das Interessante ist, dass viele junge Israelis ein ganz positives Deutschland-Bild haben, und die Shoah jetzt von Deutschland abspalten und auf Polen projizieren. So nach dem Motto „Das hat alles da stattgefunden – Polen ist das böse Land und die Polen sind die Bösen! Und die Deutschen sind aber unsere Freunde“ und so – und das gibt dann zum Teil eine ganz komische Schlagseite... 

 
Darum reagieren die polnischen Politiker dann gereizt, wenn von „polnischen KZs“ die Rede ist. 

 
Dr. Axel Töllner: Genau, das ist aber auch ein wirkliches Problem! Ich glaube, es gibt natürlich auch Deutsche, denen das gar nicht unlieb ist, die sagen „Das können wir jetzt outsourcen!“, um das mal ganz flapsig zu sagen. Aber umso wichtiger ist es dann, hier die Begegnungen zu haben. 

Was ich auch interessant finde ist, dass in den letzten ca. 10 Jahren auch die Vereine in den Dörfern langsam ihre jüdischen Mitglieder wiederentdecken, oder auch die Leute, die im späten 19. Jhd. diese Vereine mitgegründet haben - Turnverein, Freiwillige Feuerwehr, etc. Da gab es ja wirklich an vielen Orten Juden, die da ganz vorne mitgearbeitet haben. Auch wenn man sich neuere Festschriften anschaut, hat man Fotos, in welchen dann Aussagen stehen wie: „Unser jüdisches Mitglied so-und-so“. Da verändert sich schon einiges. Aber es ist nach wie vor sehr unterschiedlich, je nachdem, wo man sich befindet. 
Dabei fällt mir noch die Meiser-Debatte ein, bei der doch deutlich geworden ist, dass hierbei noch einiges schlummert, wobei es Leute versuchen, auf einer definitorischen Ebene zu lösen und zu sagen: „Das ist Antisemitismus, das ist kein Antisemitismus“. Das, finde ich, kann es eigentlich nicht sein. Und das Niveau der Diskussionen fand ich zum Teil schon echt erschütternd, muss ich sagen. 

 
Ja, am besten haben das die Münchner gelöst. Die haben die Meiserstraße in Katharina-von-Bora-Straße umbenannt, haben also eine Antisemitin für einen Antisemiten... 

 
Dr. Axel Töllner: Wir wissen ja von ihr eigentlich zu wenig. Ich meine, es gibt eigentlich nur diese Geschichte, dass Luther sagt: „Wenn du jetzt hier wärst, dann würdest du sagen: ‚Die Juden haben‘s gemacht‘“, aber er schreibt auch in diesem Brief an seine Frau 1546, er sei an einem Dorf vorbeigefahren, wo Juden wohnen und der giftige Judenwind wäre ihm kalt ins Kreuz gefahren oder so ähnlich und er schrieb dann: ‚Wenn du jetzt da gewesen wärst, würdest du sagen, es wäre das Werk der Juden gewesen.‘ So sinngemäß. Dass er sterbenskrank sei und die Juden ihn vergiftet hätten. 

Die Namensumbenennungen haben auch ihre Probleme. Also, ich denke, man kann auch auf die Art und Weise Geschichte entsorgen. Wenn da diese Meiserstraße Meiserstraße heißt, dann ist sie für mich jedenfalls ein permanenter Pfahl im Fleisch, dann muss ich mich damit auseinandersetzen. Und jetzt habe ich die Katharina von Bora, eine Frau, super. Ich habe jemanden, der nichts mit München zu tun hat, jemand von vor 500 Jahren. Das ist auch eine Form der Geschichtsbewältigung, die ich problematisch finde. 

 
Wäre es nicht besser, die Straße so zu belassen und dann wenigstens irgendwie eine Erläuterung hinzuzufügen? 

 
Dr. Axel Töllner: Das hatte ja dem Eckart Dietzfelbinger (Anm. d. Red: Wissenschaftlicher Mitarbeiter im Dokumentationszentrum Reichsparteitagsgelände) hier in Nürnberg vorgeschlagen. Das hätte ich persönlich auch besser gefunden, aber der hat von einem antisemitischen Nationalprotestanten und einem nationalprotestantischen Antisemiten gesprochen. Da fühlte sich wiederum die Familie Meiser diffamiert. Da geht es eben wirklich um: „Was ist jetzt antisemitisch?“ Oder: „Darf man jemanden Antisemiten nennen?“ Das hat in der Form nicht geklappt. Aber bis heute gibt es dann die Partei, die sagt, das sei alles ein Fehler gewesen. In München gab es eine Gemeinde, die einen Saal in ihrem Gemeindehaus danach dann in „Hans-Meiser-Saal“ umbenannt haben, aus Trotz. Und ich finde, dass die inhaltliche Auseinandersetzung mit dem Meiser da ein bisschen auf der Strecke geblieben ist. Die Suche nach einem Etikett, um das auf eine bündige Formel zu bringen, die hat manches damals schwierig gemacht. Für mich persönlich war der Meiser ein Antisemit. 
Das Problem ist nur: Wenn ich diesen Satz sage, habe ich damit noch nicht sehr viel gesagt. Denn die einen Leute verstehen da eine Sache darunter und die anderen verstehen etwas ganz Anderes. Das ist, finde ich, nach wie vor schwierig auf den Punkt zu bringen, zu erklären, warum der Meiser Antisemit gewesen ist. Und dass Antisemit nicht gleich Antisemit bedeutet. Das ist etwas, dass man auch ganz schwer vermitteln kann, dass der Antisemitismus sehr viele Spielarten hat und auch die Frage, dass jemand antisemitische Denkstrukturen haben kann, ohne dass er jetzt in jedem zweiten Satz sagt: „Die Juden sind unser Unglück!“, ich sage das jetzt mal bisschen plakativ. Sondern dass es da subtile Formen gibt, die bestimmte, jahrhundertealte Denkmuster weiter tradieren und in diese Schiene reingehören. Das finde ich schwer zu vermitteln. Zumindest habe ich damit gerade in so gut kirchlichen Kreisen meine Probleme, mich da verständlich zu machen. Und auch, dass auch jemand, der ein Antisemit ist, etwas getan haben kann, das aus unserer heutigen Sicht als gut bezeichnet werden kann. Fällst du mit dem Urteil, jemand war ein Antisemit, ein Totalurteil über das ganze Leben von ihm, oder kann man sagen: „Der war Antisemit, aber wir haben ihm auch in mancher Hinsicht was zu verdanken.“? 
 

Das war in Polen sehr extrem. Da gab es ja die Pelota, das war eine antisemitische polnische Organisation, eine erklärt antisemitische polnische Organisation, die Juden gerettet haben und dadurch ihr Leben aufs Spiel gesetzt haben. Weil sie gesagt haben: „Ja, Juden nach Madagaskar, Juden raus aus Polen – aber umbringen? Nein.“ Also auch extrem, die Position. 

 
Dr. Axel Töllner: Ja, und das Problem ist, je intensiver man sich damit beschäftigt, desto vielschichtiger werden die Graustufen, die man dann wahrnimmt. Und das irgendwie so ein bisschen bündiger zu vermitteln, ich denke jetzt noch mal zurück an diese Erklärungstafel an so einem Straßenschild. Das ist schwierig. Aber ich finde, das wäre eigentlich das Ziel. 

 
Ein positives Beispiel: Ich weiß nicht, ob ihr das Kriegerdenkmal kennt in der vorderen Ledergasse. Das ist so ein Denkmal für alle möglichen gefallenen Nürnberger im Boxeraufstand in China, Herero-Aufstand - und zu den Herero gibt es dann unten drunter noch eine Tafel, und auch zum Boxeraufstand. Das ist gut gelöst. 

 
Dr. Axel Töllner: Das finde ich auch. Weil es im Grunde ja auch zur Geschichte Nürnbergs dazugehört, dass der Stadtrat damals beschlossen hat, die Spitalgasse nach dem Nürnberger Hans Meiser zu benennen und ich finde auch, wir können uns nicht nur die Sahnestückchen raussuchen und sagen: „Das ist prima und den Rest, den schieben wir ganz nach hinten, wo ihn niemand finden kann. Im Archiv beispielsweise.“ Ich denke, heutzutage könnte man mit elektronischen Medien, mit Apps sicherlich auch einiges machen. Zum Beispiel irgendeinen QR-Code, damit die Leute ein paar weitergehende Informationen bekommen, wenn sie sich dafür interessieren und mehr darüber erfahren möchten oder es noch nicht verstanden haben. Das wäre vielleicht noch eine Möglichkeit.  

Es gibt auch eine ganze Reihe von Fotos und Zeichnungen, wo man zumindest noch die Häuser erkennen kann, die zum jüdischen Viertel in Nürnberg gehört haben. Ich finde ganz interessant, dass es in den 1980ern oder 1990ern, einmal Grabungen in der Frauenkirche gab und da wurde auch so ein Sockel von einer Säule der Synagoge gefunden. Und dann haben die Archäologen von dem Säulenabdruck sagen können, wie hoch die Säule wahrscheinlich war und wie groß die Synagoge dann wohl gewesen ist. Ein bisschen vergleichbar mit Regensburg, die von diesem Altdorfer Stich, die man so ein bisschen kennt. So was noch sichtbarer zu machen, das wäre eine tolle Sache. Aber es gibt so viele schöne Sachen, die man machen könnte. Bis dahin, dass ich mir denke, alles was da so unter dem Hauptmarkt drunter ist mal richtig archäologisch zu untersuchen, das wäre eigentlich eine Pflicht, die der Stadt des Friedens und der Menschenrechte gut anstehen würde. 

 