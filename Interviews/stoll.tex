\section{Dr Katrin Stoll} 

\textit{Katrin Stoll is a Holocaust scholar from Germany who has lived and worked in Warsaw for many years. From 2015 to 2018 she worked at the German Historical Institute in Warsaw (in the research group "`Functionality of History in Late Modernity"'). She is a member of the German-French research group "`Early Modes of Writing the Shoah: Practices of Knowledge and Textual Practices of Jewish Survivors in Europe"' (1942–1965). As a member of this group she retrieved and safeguarded the Nachman Blumental Collection at the University of British Columbia in Vancouver in 2018 and ensured that 32 boxes containing Holocaust-related material were shipped to YIVO in New York. \\
Her research interests include anti-Semitism; Holocaust historiography and testimonies; Täterforschung; criminal prosecution of Nazi crimes in the Federal Republic of Germany; representations of the Holocaust in Germany and Poland. \\
We met Katrin at the German Historical Institute in Warsaw on the 30th of January 2018.}\par 
\vspace*{2em}
\textbf{Katrin Stoll:} Can I maybe start with the following: On Friday, a new law was passed by the Polish Sejm, by the parliament. It states: “Whoever accuses, publicly and against the facts, the Polish nation, or the Polish state, of being responsible or complicit in the Nazi crimes committed by the Third German Reich… shall be subject to a fine or a penalty of imprisonment of up to three years.” What we are witnessing at the moment is an attempt to regulate history from above, through legal measures.  What is at stake for the Polish government is the so-called good name of Poland and the Poles. This law was passed on Friday, one day before International Holocaust Memorial Day, the day of the liberation of Auschwitz on 27 January. It’s an interesting development: the fact that the crimes were committed in the first place are not the cause of outrage. I mean, I think that everybody knows that it was the SS who established the concentration and extermination camps, and that the Holocaust was a German Nazi state crime, that’s pretty obvious, but the very fact that these crimes were committed does not produce outrage. What evokes outrage and fear among high-ranking bureaucrats of the Polish state is that they might be falsely ascribed any responsibility for the crime. This is their fear: That they, before the world, appear as the ones who committed this crime. It’s an attempt to regulate history from above. The experience of the Jews is of no importance whatsoever in this kind of history politics. It is very disturbing. I mean, the first priority should be to reflect of what happened at Auschwitz instead of thinking of one’s own narcissistic feelings. 

\textbf{Do you feel that this fear is substantiated in any way, for example when articles regularly refer to Polish concentration camps, will it lead people to believe that these were connected to the Polish nation?}

\textbf{Katrin Stoll:} As I said, only people who are completely ignorant could come to the conclusion that Poles established the Nazi concentration camps. The Nazis established Auschwitz for Polish political prisoners. Auschwitz is the symbol of the murder of the European Jews. The real issue here is to regulate public discourse and to suppress a free discussion about the whole dimension of the Holocaust. The Holocaust was not only a German state crime, not only a confrontation between Germans and Jews. It was also a confrontation between the non-Jewish majority communities and the Jews, in the case of Poland under German occupation: between the Christian Poles and the Jewish population living in this country. One third of the population in Warsaw was Jewish, the largest ghetto was in Warsaw, 350,000 Jews were deported to Treblinka by the Germans in the summer of 1942. However, the majority of Polish Jews lived in small cities and shtetls. The Germans carried out the deportations of the ghettos in a very cruel manner and shot many Jews on the spot in front of the eyes of non-Jewish Poles who largely accepted the genocidal project of the German occupiers. It was a visible, public crime, visible to the Polish population, to Polish society, so the question is, how did they respond to that?  And the real issue is the involvement in the crimes and the role of the “participating observers” (Elżbieta Janicka). This is what the advocates of Polish nationalist history politics are trying to regulate: the discourse. They don’t want to speak about certain things. We have certain avoidance discourses like the Righteous discourse, it’s like a fig leaf. Jan Grabowski speaks of ‘Righteous defence’. The nationalists always talk about the Polish Righteous in order not to talk about difficult subjects, like anti-Semitism, for example, and the participation of sections of Polish society in the Holocaust. no sane person would ascribe responsibility to the Polish nation as a whole because it’s always individuals who act. We’re talking about individuals, individual participation or individuals acting as part of an organization. Sometimes several individuals got together to act, like for example in Jedwabne where Polish neighbours murdered the entire Jewish population – men, women, children of the town – in 1941, burning them alive in a barn. Since 2015, when PiS came to power, the highest leaders of the Polish state have openly negated the fact that Polish citizens murdered the Jews of Jedwabne. I think this is the direction of this Holocaust-speech law, it is directed against Holocaust recognition, i.e. the active and passive participation of parts of Polish society in the Holocaust. The ‘Polish camp’ issue is not the real issue, because, as I said everybody knows that the SS established the Nazi death camps. 

\textbf{Don’t you think that there are a lot of journalists and influential people all over the world – as you said, ignorant people – the ones who spread the idea of Polish, so-called Polish concentration camps, for example in Italy. So probably, it’s maybe not the main, but an [important] idea of this law, that we should stop it, because this mistake is spread over the world. And there are many people [who] have no clue about Poland, about what happened during the War, and if they listen to or read journalists who say ‘Polish concentration camps’, they create a kind of idea of Polish nation at the time. So, probably, it’s still necessary to say ‘not Polish, but Nazi concentration camps’, I don’t think it’s just a minor issue.} 

\textbf{Katrin Stoll:} I didn’t say that it was a minor issue, I said that the real issue is evaded. The real issue is to ask ourselves: Why were these crimes committed? How was Auschwitz possible? Why was there no solidarity with the Jews? Apparently, it’s no big deal that we have ‘a little Auschwitz’, ‘a little Holocaust’?! Why are we not talking about what happened at Auschwitz, why are we not talking about what happened in the Warsaw Ghetto, in Treblinka? For people, the ‘Polish camp’ subject is an issue because they want to avoid a confrontation with the reality of what happened during the Holocaust, this is what I’m saying. And I think it’s an obsession: [For example], the Polish foreign ministry has established the term ‘false code of memory’. “Polish concentration camps”, according to them, fall into this category. Of course, it’s false, but what matters to them is that their good name is kept clean. I think the real issue is avoided.

\textbf{But do you think that in this law, the Polish government wants to negate what happened in Auschwitz?} 

\textbf{Katrin Stoll:} No, of course, nobody negates it, nobody would negate Auschwitz. But it’s rather: ‘I don’t want to be connected with these crimes in any way’, and there is the issue of Jedwabne, and of the attitude of Polish majority society to the persecution and murder of Jews. The Poles were forced to become eyewitnesses and co-presents of the Holocaust carried out by the German occupiers and they benefitted from the crimes.  “The lie of Jedwabne” has become official state doctrine, the head of the Institute of National Remembrance (IPN) has publicly propagated the idea that the Germans murdered the Jews of Jedwabne, not the Poles. Again, we are talking here about avoidance and defence mechanisms. 

\textbf{Would you say that it’s completely avoided? In history politics, are there some political measures that deal with the victims and take responsibility for them?} 

\textbf{Katrin Stoll:} Yes, I mentioned the Centre for Holocaust Research in Poland, they have done a lot, published a lot, also testimonies that actually focus on the experience of the victims, Jewish victims, and the difficulties of surviving on the so-called Aryan side. We have to get the facts straight. Now, what are the facts? The Germans murdered 90\% of the Polish Jews – 3.5 million. Out of the 3.5 million Jews, only 30,000 to 50,000 Jews managed to Survive in the German-occupied territory. It is a very small number. And after the Germans had liquidated all the ghettos and murdered most of the Jews in the Nazi extermination camps – in Bełżec, Treblinka, and Sobibór – there were approximately 200,000 Jews still alive – out of them, only 30,000 to 50,000 Survived. So, the question is – what happened to them? And, the Polish Holocaust researchers – Barbara Engelking, Jan Grabowski and others – have studied what happened during this so-called third phase of the Holocaust. Of course, nobody negates that the Nazis were the main perpetrators. But in order to make the Holocaust so terribly effective, the Germans depended on parts of the Polish population. Take, for example, the Blue Police under German command: the police participated in the deportations of Jews, Jan Grabowski has written about the so-called Blue Police. This is what the Polish Holocaust researchers have studied. In my view, the Polish government attempts to block this self-critical and analytical Holocaust discourse and to marginalize the Polish school of Holocaust research.  

\textbf{Is there a connection between such a, well, let’s call it one-sided view on the history of the Holocaust, and the topic of anti-Semitism? How is it connected?} 

\textbf{Katrin Stoll:}This is an interesting question. Yesterday, the Polish president, Andrzej Duda said that Jews have the right to fight anti-Semitism, and that Poles have the right to fight the slandering of the Polish nation and the good name of Poland. You can see that a false symmetry is being created here, as if anti-Semitism and anti-Polishness were the same thing. Anti-Polishness is a trope that appeared in Polish discourse after the anti-Semitic violence against Jews after World War I. People who condemned the pogroms was reproached with being “anti-Polish”. Maria Janion and others have pointed out that in the history of anti-Semitism, the ‘Jew’ has been portrayed as the intruder, as a state within the state, as the one who is not part of the German nation, the Polish nation any nation in Europe. This is an integral notion of European anti-Semitism: the Jew as the intruder, and, in the Polish case, very often, the Jew as the non-Pole: There is a concept of the Pole that says that a Polish person is only a Catholic person (polak-katolik). But anti-Semitism, as I understand it – there are different definitions – is based on false projections, it’s something that happens in your mind, it’s a phantasmatic concept. Perpetrators of anti-Semitic violence respond to a certain image that has been created and has circulated in their cultural tradition. [And there are different variations:] For example, Nazi-anti-Semitism was a redemptive anti-Semitism, the Nazis wanted to redeem the world of the Jews. In Polish anti-Semitism, we have a strong tradition of Christian anti-Semitism. German anti-Semitism led to Auschwitz, Polish anti-Semitism to pogroms. But we have a certain mental structure that we always find in anti-Semitism. And this is, I think, the conspiracy aspect. In the anti-Semitic worldview an endless power is ascribed to Jews, the power to rule the world, to undermine things, in Christian anti-Semitism, the power to kill God, in modern anti-Semitism the power to rule the world by means of capitalism, socialism, communism, liberalism. We have to understand that it is a mental concept that functions relatively independent of what Jews do or don’t do. Anti-Semitic images can be activated in certain situations. You have probably heard about the nationalist march, last year, on the 11th of November in Warsaw. This manifestation was organised by the so-called Radical Nationalist Camp, ONR, an organization that refers to pre-war fascism. 60,000 people marched behind nationalist, racist banners like “White Europe” and so on, and one person – he was on TV – was asked “Why are you participating in this manifestation?”, and he replied, “In order to remove Jews from power”. This is a classic example of an anti-Semitic worldview, the idea that it is the Jews who rule the world – it doesn’t matter if there are real Jews in the government or not, the anti-Semite reacts to an imagined Jew.\\
It’s interesting to see then how this was discussed afterwards: A lot of people argued that we cannot say that all these people who participated in this manifestation were fascists. But the very fact is that by marching behind these banners they agreed to that, and they knew that it was the ONR who organised this manifestation. I have noticed – I’ve been living in Poland for nine years now – that there has always been an attempt to differentiate between patriotism and nationalism [a case in point is Adam Michnik in his book “Kościół, lewica, dialog” , 1977]. It’s the idea that aggressive nationalism is bad, and that patriotism is good. But sometimes you can see that these things intermingle. And there is an article – I think, 256 of the Polish penal code – which makes hate speech a criminal offence, but the problem is that this article is hardly ever applied, even though the Polish president officially condemns these racist statements. We have an official distancing, but if we look a bit closer, we can see that, for example when it comes to criminal persecution, cases have been closed. For example, there was one case against Justyna Helcyk, she’s an ONR leading figure in Silesia, and this case was closed, she intervened and the Minister of Justice, who is the chief public persecutor at the same time, made sure that the case was closed. And we have several of those cases, maybe you have heard of the case of Piotr Rybak, who publicly burnt an effigy symbolizing a religious Jew. He was sentenced, but later on, the sentence was reduced. So, we have a double-bind situation: Officially, the Polish government distances itself from right-wing extremism, but when it comes to the legislation and the prosecution of these crimes, they are lenient. They don’t see a connection between their policies and the rise of right-wing extremism. A survey demonstrated recently that 30\% of Polish men support this fascist organisation ONR. For me the puzzling question is – all of this is happening after the Holocaust – that the ONR, an openly fascist organization, is not outlawed in Poland, that they have freedom of speech.  

\textbf{But on the other hand, there are criminal cases against Jan Gross, the Polish Holocaust historian. He was interrogated five hours, and there is still a criminal case.} 

\textbf{Katrin Stoll:} Yes, the criminal code contains an article whereby those who slander the good name of Poland can be prosecuted. What was his ‘crime’? Gross wrote an article in which he said that Polish people had killed more Jews than they had killed Germans during the German occupation of Poland. He argued that the present-day stance of Polish society towards the refugees has its roots in the fact that Polish society has not dealt with its own role during the Holocaust and the murder of Jews. Jan Gross seeks to take Poland, the self-declared Christ of nations, from the cross. This is the real reason why he is being prosecuted and portrayed as a vampire.  

\textbf{You mentioned this article by Maria Janion, and she writes about these intellectuals who seem to spend their time fantasising about Jews, demonising them, reading anti-Semitic pamphlets and writing them anew. I was wondering if that kind of activity was only done by intellectuals, or if writings also had an impact on general society? Or was there like a two-layered transmission of anti-Semitism – you have a kind of intellectual anti-Semitism and you have this general population anti-Semitism?}\par
\textbf{And how are these writings perceived today?}\\ 

\textbf{Katrin Stoll:} I think we have to look at the Polish Catholic Church here because anti-Semitism was an integral part of Catholic Church teaching. So, when we ask, “what ideas did the Polish peasants have?”, we can say that their idea of the Jews was not shaped by what the intellectual elite wrote. I don’t know how many of them were able to read. Their idea of “the Jews” was shaped by the Polish Catholic Church. For example, that the Jews were the Christ-killers, this anti-Semitic idea. I think that you can say that the population in the 18th and 19th century was more influenced by religious anti-Semitic teachings, not so much by what intellectuals wrote, like Krasiński and others. And, regarding your second question, I think there has not been enough distancing from these people and their writings. Take Staszic, for example, the Polish Academy of Science is named after him. I think, when it comes to that, a lot has still to be done. The deconstruction of, let’s say, the negative side of Polish culture. This is what Maria Janion1 analyses, she wants to understand what the characteristic features of Polish culture and anti-Semitism are. According to her, anti-Semitism is an integral part of Polish culture She wants to understand why people need the anti-Semitic figure of the ‘Jew’ for the construction of their own identity. 

\textbf{And were the clergy of the Polish Catholic church in turn influenced by these intellectuals or did they just always transmit their idea of Jews as the Christ-killers and that’s it? Or did they also fantasise?}

\textbf{Katrin Stoll:} I haven’t studied this period, it’s not my field of expertise, the end of the 19th century, so I cannot say anything about that, but I can give you an example of a more recent case, the anti-Semitic campaign in Poland in 1968 for example. It was an official state campaign; the highest officials of the party unleashed this campaign. And the consequence was that approximately 14,000 Jews or those considered as Jews by the authorities left Poland. And the Catholic Church in Poland did not condemn this campaign thereby approving of it. And there are cases where anti-Semitism functions in political discourse, for example in the presidential elections in 1990, when there were the candidates Wałęsa and Mazowiecki, and Wałęsa suggested that Mazowiecki had Jewish roots. And what did the Mazowiecki campaign do? Instead of naming this as anti-Semitism they propagated that way back to previous centuries Mazowiecki’s family was of ethnically Polish origin, that there was no Jew in their family. It is very disturbing that they did that. I think that the subject of the Catholic Church as well as anti-Semitism within the Catholic Church is a great taboo in Polish discourse. I am convinced that if the Catholic clergy who in the countryside were the main authority during World War II and the German occupation, in 1941, when the pogroms happened, I think that if the Catholic clergy had intervened and said “We are not going to kill the Jews” that people would have listened to them. I think that this subject – the stance of the Catholic Church towards anti-Semitism and the Holocaust – has not been dealt with enough. 

\textbf{But let us not forget about hundreds of Poles who helped Jews, and children, babies, who stayed alive, and the whole villages saving Jews. You cannot forget that, can you?}

\textbf{Katrin Stoll:} Nobody is forgetting about that. To the contrary.  

\textbf{I mean, the role of the Church is not only negative, there is also some positive aspect of their acting and they also helped thousands of people – Jewish people.} 

\textbf{Katrin Stoll:} Nobody denies that. I’m just saying that anti-Semitism has been an integral part of Christianity in general and the Catholic Church in particular.  

\textbf{One Polish historian told me that Polish people were very extremist: They were very extremist in collaborating with the Nazis, and they were very extremist in helping the Jews.}\par 
\textbf{But what is discussed about widely is the bad extremism. What I think is that the bad extremes are mostly shown. And those who helped, we don’t talk about them...}

\textbf{Katrin Stoll:} Can I maybe say something about this issue of help. The official Polish state discourse – for example during the anti-Semitic campaign of 1968 – portrays the entire Polish nation as rescuers of Jews. But we have to understand that those few people who rescued Jews were an absolute minority, they were exceptions, complete exceptions, and afterwards people blow this out of proportion and say these people were representative of the Polish nation. They were not! They were not representative, they represented only themselves, they acted against the norm, the societal norm. Under the German occupation, rescuing and helping Jews was not considered an act of resistance against the German occupiers. The helpers also had to hide Jews from their own neighbours, not only from the German occupiers. We have to differentiate between real people who helped and between the discursive figure of the Polish Righteous in public discourse. This discursive figure, which is detached from real people, always appears in public discourse in Poland when there is anti-Semitic violence. The Polish Righteous were brought up in 1946, after the Kielce pogrom, they were brought up in 1968, they were brought after the Jedwabne debate in order to cover up the anti-Semitic acts. Irena Sendler for example said that after the Jedwabne debate a hero was needed. We have to ask ourselves: What is the function of the discursive figure of the so-called Polish Righteous which portrays Poland as a nation of rescuers of Jews? 

\textbf{Another problem is that there are Polish Catholics who saved and hid some Jews, but they were not able to state it or to publish it. They were shamed by the mainstream after 1945.} 

\textbf{Katrin Stoll:} Yes, they were considered outlaws and they were persecuted in the immediate aftermath of the events. After liberation from German occupation they were in mortal danger. These people did not want to reveal [what they did], and this testifies to the fact that the majority disapproved of their behaviour. I have to say that I was a bit shocked when Morawiecki, the Polish Prime Minister, mentioned the Righteous in the context of Auschwitz. What do the Righteous have to do with of Auschwitz? Nothing, nothing at all, we are talking about a German Nazi concentration and extermination camp where so many Jews from all countries in Europe were murdered – the Germans deported Jews from Corfu in Greece to Auschwitz, and the only thing that this man can think of is the Polish Righteous? This is so absurd, so grotesque because they were no Righteous at Auschwitz. It was the site of mass murder of the European Jews. The Jewish victims don’t really matter in Polish discourse. Hanna Krall once said, when the Poles speak about the Holocaust, they always speak about themselves. 

\textbf{Why do you think that people always take this so personal? I mean, in Germany and Poland, when we talk about crimes of Poles, Germans, Latvians people take it personally even if it was so many years ago.}

\textbf{Katrin Stoll:} I think it’s an inability to face up to reality. I think that you can only bear reality if you can face it, and you cannot uphold a Polish identity of Christ among the nations, of Poles as the eternal victims, if you acknowledge that there were people in Polish society who murdered Jews, right? So, it’s not possible to create an unspoiled, clean identity. I think, this is why there is this strong reaction.  

\textbf{But this idea of the Christ among the nations doesn’t exist, does it?} 

\textbf{Katrin Stoll:} I would say that is what differentiates, let’s say, Polish nationalism from Hungarian nationalism, or Latvian nationalism, or Lithuanian nationalism etc. The characteristic feature of Polish nationalism is that Poland, in the Polish literary canon, has been portrayed as the Christ among the nations. It’s undeniable, and it has shaped people’s perception of reality and their behaviour, it has shaped the way people look at the world. I mean why is it so hard to get rid of nationalism? According to Pierre Bourdieu, the sociologist, it’s so hard to get rid of nationalism because we are talking here about dispositions, which are internalised, like bodily dispositions. You grow up with your education at school, you read specific texts, you build a certain understanding of the world based on ethnically homogeneous groups2. You can see how this works in Poland: During the so-called refugee crisis for example the head of the party PiS, Kaczyński, made openly racist statements. He said that refugees were carriers of diseases and so on, thereby winning the election. And if we look at this, we can see again that we are talking about phantasmatic concepts. There is no single refugee in Poland. But what he tried to do was to stimulate this fear of imagined refugees who undermine the religiously homogeneous nation. This was the enemy portrayal by Kaczyński. Again, if we want to understand how concepts like nationalism and anti-Semitism function, we have to look at the phantasmatic conceptions. 

\textbf{I didn’t want to negate that there is the anti-Semitic picture in Polish literature. But they asked one million Polish students whether they liked the poems and the drama, and they hate it. They do not identify with this idea of being a Christ of nations, not at all, not today.} 

\textbf{Katrin Stoll:}  I would say that the concept of the nation is still the main reference: that people don’t necessarily define themselves as human beings or as Europeans, but as Poles. And of course, you are right, there are different variations, but I think it’s still the main frame of reference in talking about so many issues in Poland, all kinds of problems are discussed under the heading of the nation. 

\textbf{Maybe, it’s again this thing that intellectuals read this kind of literature and they maybe identify with this idea of Christ among the nations, but regular people do not do this, they never ever read 19th century literature unless they are forced to at school, but still they are nationalist. But it’s transmitted in another way.} 

\textbf{Katrin Stoll:} Okay, I’ll tell you one thing. Poland has been part of the European Union since 2004. There has been a free Poland since 1989, general elections and so on and afterwards a liberal democracy in Poland. So, explain this to me: Why has there never been another narrative? A narrative based on an understanding of a nation based on citizenship, not ethnicity. It has not happened! And this is something that people in the so-called West do not understand, that when we talk about, let’s say, PO and PiS, the main political parties: they are both right-wing conservative political parties. For both parties the nation is the main frame of reference. There has been no other narrative of what it means to be Polish, or I haven’t come across that. The question is: Why? Why is there no narrative of Poland in the European context, or of what it means to be a European Pole, a Polish European? 

\textbf{So, do you think that Poland is a dangerous place for Jews? Cause you’re saying it like that.}  
 
\textbf{Katrin Stoll:} I’m not saying that. I think I would say it like this: It’s a dangerous place for anybody who does not fit a certain image of what it means to be Polish. It could also [be dangerous] if you’re a homosexual and do not fit into the idea of what it means to belong to a good Polish family. It’s a dangerous place for people who do not correspond to a certain idea of Polishness. I was extremely frightened, I have to say, [on] the 11th of November 2017 when, in Warsaw, which was completely destroyed by the Germans, where the Germans killed so many Poles and Jews, 60,000 people were marching under the heading of “white power”, marching through the city. During this fascist manifestation I had to hide in my flat because I felt that I could not go outside. You could say, “Okay, these are extreme people” but what is dangerous is the fact that the right-wing discourse has become hegemonic. The right-wing nationalist discourse has become the dominant discourse. I would always say that in Germany it is worse, you have Nazis in the Bundestag, we have attacks on asylum homes and so on, we have the National Socialist underground, we have the legacy of National Socialism, but the important, the dangerous thing is that the right-wing discourse is the dominant discourse, not only in Poland, in other countries as well.  

\textbf{In Poland, there was a very strong anti-Semitic movement, for example, in the 1930s, they wanted to expel all the Jews to Madagascar, in 1946, there was the Kielce pogrom. Is there a political historical discussion about this anti-Semitic past?} 

\textbf{Katrin Stoll:} Yes, among outsiders. It’s not mainstream Polish historiography which focuses on these issues. Cultural anthropologists have studies these questions, Joanna Tokarska-Bakir3 for example has written on the Kielce pogrom. I think the problem is not that studies do not exist or that the subject is not being studied. The problem is that these subjects have not entered official mainstream historical discourse and historiography. It’s not that knowledge doesn’t exist. The problem is that it does not circulate.  

\textbf{So – how could it trickle down?}

\textbf{Katrin Stoll:} It can only be achieved by means of education, as I said, you have to understand how anti-Semitism functions: it has nothing to do with reality, it is something that is made up and used for specific purposes You have to make anti-Semitism morally, politically, socially unacceptable. This is the disturbing thing: That history is repeating itself. Are we again at the end of the 1920s, when democracies collapsed in Europe and there was the rise of fascism and National Socialism, are we witnessing this again?  Why is it okay to be an anti-Semite after Auschwitz; what it means to be an anti-Semite after Auschwitz is to approve of Auschwitz, it’s to approve of the murder, the mass murder of Jews, an anti-Semite wants Jews to be dead.  

\textbf{How can we fight such an expression of anti-Semitism, justifying anti-Semitism and especially the Holocaust? Because for example in Latvia, there is [this myth, [it’s in the] media, that Jews were those who betrayed the state of Latvia, who supported the Soviet occupation How can I deal with this problem? Because when I try to tell people “It is a disgusting lie”, people think that I don’t know history, that I’m a Jew sympathiser...}

\textbf{Katrin Stoll:} Very important question it is also a problem in Poland, this anti-Semitic stereotype and a myth of żydokomuna, the merging of Jews and communists.  in Nazi anti-Semitism this was also very important element, this notion of a Jewish-Bolshevik world conspiracy. So, how to dismantle that? Again, you have to ask: what function does this myth fulfil in your country’s public discourse, who disseminates it, propagates it, which purposes does it serve? So, what would you? 

\textbf{Well, mainly, this myth is not officially supported, of course, you can not officially support it, but it exists between people, just ordinary people. The majority of Latvians think that Jews were traitors.} 

\textbf{Katrin Stoll:} So, we have another element here which is very important in the anti-Semitic world view, namely the notion of Jewish aggression: The idea that Jews do something against your nation, the Latvian nation in this case. The only thing you can do is to say that this is a myth, and to say that it is a phantasmatic construct. For the construction of this phantasma it doesn’t matter if there were real Jews who were communists and part of the Soviet authorities. The problem is that people imagine Jews in a certain way and imagine Jews doing things in a certain way. So, then we have to ask: Why? So, why is this particular notion so important for the Latvian identity? 

\textbf{ Maybe people just don’t want to see themselves as murderers. Because when it comes to the Holocaust, we can just say “Okay, Jews got what they deserved. We had the right to revenge”.} 

\textbf{Katrin Stoll:} This is a very important point: the anti-Semitic construct is used to justify one’s own crimes against the Jews. In the case of Lithuania and Latvia, we have certain groups and organisations actively helping the German Nazis murder the Jews shooting, murdering the Jews in the forest and elsewhere. In order not to admit to the crime itself, people simply say “We took revenge. Because the Jews were the first who did something bad to us, and we just took revenge and penalised them because they were supporters of communism.” We see here the false notion of double genocide theory, the notion of a ‘red Holocaust’ and a ‘brown Holocaust’, amounting to an equalization of Soviet policies or with the unprecedented Nazi persecution and extermination policy.  

\textbf{In the case of Latvia and Lithuania, Germans helped Latvians and Lithuanians to kill Jews, because mainly, neighbours killed Jews in the forest.}

\textbf{Katrin Stoll:} Yes. In September, I was travelling through Belarus and Latvia. I visited a lot of Holocaust burial sites and sites of mass execution in the forests, and it occurred to me that without local people, the Germans would have never found these spots and places. They needed collaborators.  But this is an important point: The notion of “Jewish Communism” served as a justification for anti-Semitic crimes. The same happened in Jedwabne, there it was also used as a justification for the murder of Jews. The perpetrators make use of an anti-Semitic idea in order to retrospectively justify their own acts of murder. We can name several crimes and study the mechanisms. And afterwards we have to deconstruct these mechanisms. 

\textbf{Would that work? I mean, if you deconstruct it, you show to people the false things they believe, you present them the historical reality. Will they listen or will the just continue to ignore it? Would it be an alternative to speak about the positive role that Jews did always have for the country, how they worked.} 

\textbf{Katrin Stoll:} No, that’s completely wrong, because again:  We are not talking about real people, we are talking about a certain notion of what a Jew is, what a Jew does, what a Jew thinks, what a Jew looks like, of a certain idea of being.  You have to destroy the concept, the narrative. And you can only do that if you are aware of the fact that it is a mental construct with the element of a conspiracy at its core. I mean, there are a lot of anti-Semites who have never seen a real Jewish person. I’m completely against this idea that you can fight any form of racism, anti-Semitism, homophobia, xenophobia, by having a real meeting with the people who are stigmatized because this is not how it works. 

\textbf{So, we were talking to the leader of the education programme at the Polin museum yesterday. And she said – especially when she goes to rural villages, she often meets children who have no idea what Jews actually are, but they just get an impression because of how the adults talk about Jews – for example, a greedy person who doesn’t want to share. And she said that it is really helpful for them to meet with Jews so that they can learn that these are all just stereotypes.} 

\textbf{Katrin Stoll:} This just shows that we have a completely different understanding of how anti-Semitism functions. Because, I believe, in agreement with Slavoj Žižek, that anti-Semites react to a certain image of ‘the Jew’ that has been circulated in their tradition4. The idea of the greedy Jew for example has circulated for centuries, and people have internalised this idea. And you have to destroy the idea. I mean maybe you are lucky, and, in this case, the person meets somebody and thinks “Oh, it’s a nice person” – but what happens if she meets another person and thinks – “This is a representation of the greedy type.” This is not how you are able to get rid of anti-Semitic concepts. 

\textbf{So, in your opinion, is there anything we can do about these anti-Semitic projections, in order to fight them?} 

\textbf{Katrin Stoll:} We can fight them by demonstrating that they are false projections. The Holocaust perpetrators projected all kinds of thing onto the Jews calling them traitors, aggressors, dishonest, greedy people. The perpetrators projected these notions onto a whole group, and this is also something that we have to understand about anti-Semitism and other forms of racism: The Nazis persecuted Jews as a group, they were persecuted independent of their social status. It did not matter if they had green eyes, blue eyes, if they were religious, non-religious Jews, if they were involved in a political party or not. For the Nazis ‘the Jews’ constituted a homogeneous group. Every individual was put in a certain group constructed by a Nazi mindset. If we want to fight these things, we have to start with language. Everybody has only one life and is an individual. And we have to teach critical thinking.   

\textbf{According to my personal experience, education can sometimes help against anti-Semitism. For example, I was socialised anti-Semitic in the kindergarten, and until I was 14 or 15, I was an anti-Semite. And afterwards, I read some books about the Holocaust, and I changed my mind.}

\textbf{Katrin Stoll:} Yes, in Germany, people discussion the question of whether it makes a difference if people visit memorial sites and learn about the Holocaust, if a visit to a memorial site is a way of protection against all kinds of bad things like racism, anti-Semitism and so on. I have my doubts because either you know that you don’t murder a human being or you don’t, and if you don’t know that, you won’t learn it by visiting a memorial site. It’s a decision that you make: in my view you need to become aware of the indoctrinations you have been exposed to so that you have the chance to distance yourself from them. I think, it’s a matter of awareness, of critical reflection, critical thinking… 

\textbf{For me, it was a matter of education, of information.} 

\textbf{Katrin Stoll:} Okay. But generally speaking, I don’t think that information helps. Let’s take the example of the Holocaust deniers.  These people are obsessed with details. Real Holocaust deniers know an awful lot about the facts themselves. For example, they argue that it’s not possible to kill so many people in the gas chambers and so on. You cannot fight Holocaust denial on the level of facts, this is impossible. You have to ask: Why do they make these statements in the first place? This is how you confront Holocaust deniers. What is their agenda, what are they up to? 

\textbf{In Latvia, every time when there is some news about a Holocaust memorial site or some commemoration ceremonies, all those people write in the commentary sections things like “Jews were not the only people killed, Roma were killed, or other groups...“ Do you think it’s an expression of anti-Semitism?} 

\textbf{Katrin Stoll:} No, I think it’s more an expression of this competition for victimhood, maybe, and the inability to comprehend the nature of the German Nazi Holocaust. Why was the murder of the European Jews different from the Nazi persecution of other groups? It was different because the Nazis attempted to murder every Jew on this planet, this was unprecedented, because they tried to erase what they called “the Jewish spirit” – the Jewish spirit they imagined to be in all kinds of things – in language, in literature – in everything, so everything had to be erased completely – complete extermination. It’s an inability, I think, to face the real nature of the events, an inability to confront reality. 

\textbf{Now, this also comes from a lack of education.} 

\textbf{Katrin Stoll:} No, I think it’s not a lack of education because in Latvia for example the Jews were not murdered in extermination camps but on the spot: with non-Jewish people watching, observing, benefiting from the murder. So, I think, it’s maybe also the fear of being in any way connected with this event and held responsible for one’s actions. 

\textbf{I think, at least in the Silesian history, the comparisons and relativizations of the victims are an expression of anti-Semitism. For example, my parents said “Okay, we killed six million Jews. But we also were victims. We were expelled.”}  

\textbf{Katrin Stoll:}This is a very typical German way of talking about the past, this Aufrechnungsdiskurs and Schuldabwehrdiskurs, which is aimed at negating the perpetration of the crime itself, minimizing its magnitude and negating that the Germans were the main perpetrators.\\
My hero is Beate Klarsfeld, an anti-fascist who gave Nazi \textit{Bundeskanzler}\footnote{chancellor} Kiesinger a slap in the face saying, “\textit{Nazi Kiesinger, abtreten}!” – “Step down!”, in November 1968 at a Bundesparteitag of the CDU. Why? Because Kiesinger was chancellor of the new Federal Republic which defined itself as the anti-Nazi state. How was it possible that the Nazi Kiesinger, member of the NSDAP from 1933 onwards, a high figure in the Nazi state, a propagandist, was elected by the Germans as a chancellor?  I think, she achieved a lot by this symbolic gesture demonstrating that certain things are unacceptable. She was charged with one year in prison for this supposed act of violence and she said, “It’s an act of violence to have a Nazi chancellor as \textit{Bundeskanzler}”. So, we have to analyse reality and don’t be fooled by propagandists.  