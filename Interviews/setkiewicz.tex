\section{Dr Piotr Setkiewicz}

\textit{Piotr Setkiewicz (born in 1963) is the director of Centre for Research at the Auschwitz-Birkenau State Museum and a graduate of the Faculty of History at the Jagiellonian University in Kraków. Mr. Setkiewicz received his Ph.D. degree in 1999 at the University of Silesia in Katowice for the work entitled ``IG Farben - Werk Auschwitz 1941-1945''. He is the editor-in-chief of scientific publication The Auschwitz Journals (Zeszyty Oświęcimskie) as the head historian at the Auschwitz Museum. We met him in the Auschwitz Museum on January 27th, 2018.}\par 
\vspace*{2em}
\textbf{Thank you very much for your time for us today. We are a group of students from Latvia, Poland and Germany doing an Erasmus project about anti-Semitism and how it developed after the War. First, we'd like to ask you if you have had any first-hand or second-hand experience with anti-Semitism over the course of your life?}

\textbf{Piotr Setkiewicz:} Over the course of my life? Always, almost every day. Just yesterday, I was called by a man who is anti-Semitic, who is denying the truth, who said that I can be blamed for all mistakes, all the false information that appeared in the media about the Holocaust. That I'm supporting the fake version of history which is based on Jewish, Polish and communist lies about the Holocaust, that we are promoting the Survivors who are lying, and that we falsified the documents in my archive. And that we are taking money from Jews in Israeli shekels. Many such things. Many people in different countries, not only in Germany but also in United States, practically all over Europe, including Poland, surprisingly, tried to persuade everyone that the Holocaust never happened, that Auschwitz was built, or at least the most important objects here, like the crematoria and the gas chambers, only after the war, and that the gas chambers were actually used for delousing purposes, for disinfection. My reaction usually is that I don't want to be involved in any sort of discussions with Nazis because as far as I know from my own experience, it's just a waste of my time and the arguments always stay the same. For instance, that conditions in Auschwitz were relatively good, that it was necessary to isolate the enemies of the Reich because it was a war and that of course, sometimes it might happen that somebody died here. That, nevertheless, the number of victims is highly exaggerated, the number of one million people who were murdered in Auschwitz. That obviously, it's not true because the capacity of the crematoria was many times lower than we read in books, read in documents, etc. Even some strange and surprising arguments are in the discussions: For instance, there is a picture taken by somebody from the staff of \textit{IG Farben} works. \textit{IG Farben} was a huge factory of synthetic rubber; it was situated on the other side of the City. Nevertheless, during the War many thousand prisoners worked at the construction site there. Apart from prisoners, of course there was German staff, men and women, about 7,000 people, and, they resided in the Siedlung, the settlement, they lived somehow apart from the concentration camp, and they got, for instance, a sports club. And there is a picture of two people who are fencing, and there is a group of people in the background of the picture, and over them there is an inscription, the name of the sport club in German, the sports club \textit{IG Auschwitz}. Now, the Nazis believe that this is a typical representation of the living conditions in Auschwitz, that the people in the picture are prisoners - which is stupid, actually, because that was a German club and they got even the training courses for pilots of incomprehensible, 05:21 and many such things, the soccer section, the athletic section, etc. Other than that, one of the most important arguments is the presence of the swimming pool in Auschwitz, which is visible just behind us. We know that it was the one of many pools or rather tanks for water for the fire brigade. In 1944, the Allies, American or British planes began to appear over Auschwitz and the SS suspected the threat of mass bombings and a fire, particularly in Auschwitz-Birkenau So, they built a number of such pools, reserves of water for the camp fire brigade. And surprisingly enough, in Auschwitz, the SS, or rather prisoners themselves, the functionaries, the capos or block overseers, mostly the German habitual criminal prisoners, they organised something like a competitional swimming in this pool. There was something like that, as we know from the testimonies of Survivors and from somebody who was taking pictures, who was a fan. Nevertheless, there was a difference between the situation of the functionary prisoners and the real fate of regular prisoners at the concentration camp. 
So, as I said, there a many such arguments, many stupid discussions. The most difficult and rather new problem is the huge amount of such articles, comments, entries on the internet. On the internet you can write anything, and there’s plenty of web pages explaining that Auschwitz never happened. The problem is that recently, we are faced with more and more people denying the Holocaust in Poland which, for me , was something I could not believe because many Poles lost their lives in Auschwitz, as well as the Polish Jews, and the knowledge about the German crimes in Auschwitz was obvious for everyone in Poland, for many years after the War. Now, I am afraid that there is a third generation of people who have no personal experience, they probably heard ``My grand-grandfather was in Auschwitz'', but the distance to history is too long. Just a few days ago, the Polish television showed a ceremony of the anniversary Adolf Hitler's birthday on the 20th of April, a group of Neo-Nazis meeting in the forest somewhere in Silesia. They were celebrating the anniversary, they had the Wehrmacht uniforms, they baked a cake for the Führer with the swastika. Perhaps that is still marginal in Poland. Nevertheless, these views are shared by the people who recently protested against immigrants, against attempts from the European Union to persuade the Polish government to accept a certain amount of people from Syria and other countries. It's still not the same as with the Nazis, nevertheless, within the relatively large group of Poles who don't want to even hear about immigrants from Arab countries, there is a narrow margin of people, particularly young people – I’d say 20-25 years old, football fans, something like that - who try to make a link between their anti-immigrant views and the Nazi ideology. In the Museum there are very rare cases where we can meet such people here. It's not a problem for us, but of course, we believe that one day, we will face the situation that a group of people with the Nazi insignia will wish to visit the Museum - so, the problem is, what to do then? Our security guards should do something and prevent it, but something like this might happen. That's a good reason for us to do our work, to publish books. I don’t want to be involved any sort of direct discussions with the Nazis, nevertheless, we are providing arguments for a lot of teachers, arming them with the documents. German documents, I'd say, a more useful in such discussions between students and teachers because if we gather the testimony of a Survivor, it’s easier for a denier to say ``Well, it’s just nonsense, what this man is talking about''. For example, a few weeks ago I a critical review of a book written by a certain Survivor on the internet, and the Survivor wrote about 300 pages book about his experience in Auschwitz. And this is basically a good witness - I mean, there are many Survivors that are simply trying to tell us their stories. Nevertheless, there is unfortunately, and it's quite normal and natural, some people who believe that they should add something to their stories. For instance, if they were in Auschwitz, they testify about certain happenings that took place in Birkenau, and it’s not true because they could have heard about the Sonderkommando rebellion, but they had no chance to bear witness of this uprising. Nevertheless, I understand that something like this might happen, that particularly the Survivors who used to take part in certain ceremonies used to meet with schoolchildren, being asked about the War experience, wish to say ``I was in Auschwitz, I saw everything''. Going back to this book: Although on one side he was generally credible, at one point in his book he said that he remembers that there was a small room in one of the barracks where the Germans, the capos, collected the bodies of those prisoners who died during the nights. That it was only after three of four days when they collected the bodies and they were put in the truck and the truck was taking the bodies to the crematoria to be burned. And, the author of this article on the internet, he made use of the fact that arguments like that are obviously not true, because I found a German document in which the commander of the Birkenau ordered to remove the bodies every day. For him, this is an obvious argument that this man was lying. And if he was lying in this particular point, in this particular page of the 300-pages book, probably the remaining 300 pages, are also not credible. 
Another category of sources, of course, are the testimonies by the guards themselves They testified before the military tribunals of the Allied powers in West Germany, immediately after the War before the Polish courts, and then in Germany in the Frankfurt trials, and in many other locations. And at least some of them they accepted their guilt and admitted ``Yes, I saw the people being selected on the ramp. I saw the people being driven to the gas chambers, I saw the thousands of bodies and the flames over the chimneys of crematoria.'' So, what is an argument of deniers against it? That they were forced to testify in this way, beaten by the guards, by British, Polish, whatever, communist, Jewish guards. That was why they had to give these false descriptions. 
And the last category - the German documents: The problem is that most of the German files from the chancelleries, from the office administration of Auschwitz, they have been burned immediately before the final evacuation of the camp. If you take any testimony of a Survivor who was here in January ‘45, typically you can find therein information like ``I saw the piles of papers, of documents on the crossroads of the camp being burned by the guards''. And that's true. We've got a very small amount of original documents from German administration, a relatively large amount of documents from the German construction office, the Bauleitung, probably because that was a separate unit of the SS-administration, which was situated in several barracks at a distance about 300 metres from here, not inside the camp. So, by chance, for some reason, when the SS escaped from Auschwitz, the just forgot to burn these files. If you take into account all these surviving documents, they are good material for historical research, alone these plans and construction files of the central construction office of the SS. The amount is about 150,000 pages of documents, a large number of documents, and they are a very valuable source, very useful for our discussions – indirect discussions with these deniers. There is, for instance, a well-known author who wrote a number of books, and in one of them, he tried to summarise the most important arguments of the deniers. And one of them was, that if we talk about the first temporary gas chambers in Birkenau, called ``the little red house'' and ``the little white house'', the ruins of little white house are still in Birkenau, nevertheless he said that it’s impossible that in so many surviving documents there is no mentioning, not any hint about these temporary gas chambers. So, the prisoners were lying, Survivors were lying, because they were Communists, Jews, Poles, whatever – that’s obvious, and the main proof that these provisional gas chambers - they were abused mostly in ‘42/’43 – did not exist at all is that there is nothing written in the documents. His argument is as follows: That, if something like that had existed in Birkenau, then it must have been in the records of the trips of the trucks, especially for delivering bricks, and other construction materials to the site of these gas chambers – there must be anything. Information about the bills, about the amount of money spent for the construction of the gas chambers – but there is nothing. Thus, he tried to persuade his readers. But unfortunately, we have found seven such documents that clearly indicate that these gas chambers existed. That one of them costed 14,000 Reichsmark, for instance. That the SS ordered three wooden barracks for each of these gas chambers, which were used as a temporary store of clothes of the victims. And each of these barracks costed about 17,000 Reichsmark, and they were removed from the area in the moment when the Germans completed more modern and huge crematories in Birkenau, so the old gas chambers were dismantled and destroyed. We have found requests from the Bauleitung asking for the remaining wooden barracks, because they might be used for some other purposes in the camp. And we’ve got the information that there is a need to send 500 prisoners on a day to dig trenches around the bunkers. That was because one day in July 1942, Heinrich Himmler, during his visit in Auschwitz, ordered to empty the old mass graves and burn the bodies on piles of woods. 
Then, what else have we got to prove? Apart from testimonies of Survivors, apart from reports of Polish resistance members, some of them Survived. There is a document in which the SS ordered a cable, a long electric cable, to provide electricity to nineteen burning places. It was necessary to install lamps there because the burning took place not only in the day, but also during the nights. There are many direct or indirect proofs that these burning pits existed. I don’t think that you are familiar with the arguments by deniers, but we’ve got a relatively large number of counterarguments. The problem is that it’s difficult to discuss with such people because when they say that all – let’s say, 10,000 testimonies, the witnesses were lying, that these thousands of documents were fabricated by Poles, Jews, communists after the War - in such circumstances, it’s rather difficult to raise any arguments.

\textbf{What do you think, what makes people want to deny the fact of Auschwitz or the Shoah?}

\textbf{Piotr Setkiewicz:} Why? For many reasons. Perhaps, some of them believe that this form of government in Germany in the 30ies, that it was ideal, that everyone knew his place in the society - but now, what do we see around us? The politicians, thinking only about how to gain more money, who are no true patriots. And the Führer, that was a man who solved the problem of the economic crisis in Germany. He built the highways. He restored the pride of the German nation. These might be the arguments of the German deniers. In other countries, I think, the reason is mainly anti-Semitism. Because the Holocaust, as a part of the Jewish history, by most Jewish historians is identified as a marking point in the history of the modern nation. In some books written by Jewish historians, you may see that they divided Jewish history in two periods: Before and after the Holocaust. It’s a very important point on the map of the Jewish history. So, for those people who are anti-Semitic, who don’t like the modern state of Israel, the Holocaust is something which is a guilt and a giant point which should be attacked. And, in the history of the Holocaust, Auschwitz is as central point. There is the famous quotation from the interview with David Irving, the leading British denier, who said ``If we sunk the battleship of Auschwitz, it would solve all the problems of the 20th century’s history''. Auschwitz is a focus, the most important part of the story. If we would be able to proof that Auschwitz never happened it would be easier to destroy the myth of 6 million of Jewish victims.

\textbf{Would you say that these Holocaust deniers are becoming more or rather less over the last twenty-five years? Is there any development, so that it becomes more usual?}

\textbf{Piotr Setkiewicz:} Immediately after the War, it was the group of mostly the German veterans, particularly from the unions of the SS, who were deniers. It was a number of even the guards from Auschwitz who took part in certain ceremonies, where they solemnly declared that they were in Auschwitz, but they never heard about any gas chambers or crematoria. That was the first wave of denial. Then, it was in the 60s, there was a rise of interest in the Holocaust. On the one hand, we got the trial of Eichmann in Jerusalem, and on the other hand, a few years later, a series of so-called Auschwitz trials in Frankfurt. And it was the time of student’s rebellion in Europe, the leftists began to be interested the history of their own country, in the leftist press appeared the pictures of leading Nazis who were still living in Germany, the titles of the articles were as follows, ``The assassins, the killers are still among us''. So, there was this popular belief of the leftist that something must be done in order to cope with our past. And to punish the people who, in many cases, continued their pre-war career after the War. All the judges prosecuted those who had a Nazi past – the technicians, engineers from the companies like \textit{IG Farben}, they were renowned chemists and board directors in different companies in Germany. So, these people, the young people at this time, thought that we should do something with this problem. The problem of the past, of the history of our grandparents. On the one hand, if you observe the rise of the interest in the Holocaust, it was a, so to say, natural reaction from the side of the right-wing radicals to deny the Holocaust. Then, in the 70ies, they got some financial problems, some newspapers and journals published by the Nazi organisations disappeared. And now, I believe, we can observe again the wave is going up – perhaps because of the internet. If, in the 70ies, there was a problem with publishing the journals, then now, in the era of internet, its costs almost nothing. 
And there is perhaps also reaction on an idea of a united Europe. As for Brussels, we should rather think of ourselves as Europeans, to a certain extent, not only as Poles, Czechs, whatever, but we are also Europeans, the members of a European family, which share the same values, the same budget. So, there is a reaction of the people who don’t like this concept of a united Europe, who think ``Primarily, we are Poles. And we don’t want to be governed from Brussels, we got our own government, our own Polish Złoty, our own history'' And , of course, the problem of immigrants, the terrorist attacks in other countries, among Poles there is a tendency to talk about it in this way. ``Because our friends - England, Germany, accepted many thousand Arab refugees, and among them, there are of course terrorists, they got a problem. We don’t want to have any refugees on the Polish soil, so, in the consequence, we haven’t got the terrorist attacks and bombs in Polish cities. Therefore, we are right.''

\textbf{And how is this connected to Anti-Semitism?} 

\textbf{Piotr Setkiewicz:} Because, if you believe that we are the better ones, that our values are better because we are Christians, or Europeans, it’s just one step to not liking the Jews because the Jews, they got too much money, they are directors of different banks, they control the world bank system, and the Jews are in primary leftists, they promote the leftist values, so, because we are nationalist to a certain extent, we don’t like to accept these outer values. There is Mr Soros, who is a Jewish billionaire, and who is supporting the leftist organisations in Poland – on the one hand we don’t like him because he is leftist, and because he is Jewish. It is not far from this concept of national unity, of ``Because we are united, we represent the united nation, all Poles are brothers, and when we see these poor waves of Ukrainians, coming to Poland to work in different jobs on the construction sites, in the shops...'' Unfortunately, there is – quite human, I’d say – a tendency to look at them as people who are not us, of course, and who represent some lower cultural standards. Surprisingly, when Poles are going to the Western countries to work, they hear these anti-Polish jokes. There are very bad anti-Polish jokes, aren’t there? It is not nice to listen such things, but now, if we are looking at these Ukrainians, who come to Poland, at least some Poles unfortunately believe ``Now, we are in the positions of the British, we can look at these Ukrainians from a certain higher level'' It’s difficult – there is the question of policy, the question of a lack of education, and such tendencies to look at other countries as an enemy, as the potential threat, are a result of the modern nationalist eruptive at the end of the 19th century, and it continued. There is a model of patriotic education in the schools, in many countries, of course, to cultivate remembrance of national heroes. In many countries, there is a tendency to talk about the history of these countries in such a way: That we were always good guys, that we are always attacked by the neighbours. We’ve got of course our heroes, nevertheless, we are still fighting for our independence, and it is necessary to remember that independence is the most important value. We, as nation, have to remember that ourselves in such a way. And there is also this looking at the strangers in a specific way, that we can accept a limited number of foreigners, but if they ever exceed a certain amount, that is a potential threat for our identity. We are Christians, we are not Muslims.
Nevertheless, Auschwitz is still present somewhere in the background of such discussions as a good illustration of what might happen if we would forget about universal values. When we see the strangers not only as the other people, but also as the representatives of the black forces of international conspiracies. And then, it’s very easy to cross the line, to be on the black side of the force. There is the question of reactions to the Nazism of German society in the 30ies, here, in the camp, the reaction on certain people who were not Germans, but who accepted the posts of Blockältester, the Capos – many Poles and even Jews here in Auschwitz who received a stick and tried to exercise force over prisoners, to beat them, because the prisoners could not work with efficiently, the stick is the best argument. These were symbols and pieces of man history that might have been used in the constant discussions about the condition of the modern world and the societies here. What should we do when we hear about certain atrocities, like a genocide in other countries? Is it possible to do such a thing? We heard about the massacres in Syria, or in Iraq, or in Sudan and such countries. What should we do? There’s not a single good answer for it. Nevertheless, if we talk about it in terms of constant discussions about human rights, for instance about the need to combat racism in many countries, Auschwitz is always somewhere, is being used as a symbol, as a turning point, as a milestone in such discussions. I was surprised when I saw once a textbook for history in Singapore schools. And there is a quite comprehensive chapter about Auschwitz, on the other side of the globe. If you talk about the history of Europe, Second World War, with, say, Koreans, or people from Argentine, most people say that they know very little about it. But in 99\% of all cases, they have heard the name of Auschwitz.
 
\textbf{Do you think that Holocaust denial is the most dangerous form of anti-Semitism nowadays?}

\textbf{Piotr Setkiewicz:} This is a part of the ideology - important, but not the only one. Of course, anti-Semitism is something much wider, it’s a popular belief, that the Jews are controlling the world economy, that, in some cases, leading politicians in certain countries have Jewish roots. That the governments are under the constant pressure from the national Jewish organisations. Holocaust denial is only a part of the ideology. But nevertheless, it is a very important one because anti-Semites believe that Jews are using the Holocaust as an argument to defend themselves, to blame other nations, that they wish to control the market in countries like in Poland, to go back to their property that was confiscated during the War by the Nazis, and after the War, it was taken over by the communist government in these countries. That’s a good illustration of the situation in Poland. Taken altogether, the Jews are members of the world conspiracy, and they are trying to use the argument of the Holocaust in their propaganda. ``We are poor guys who were persecuted in the course of centuries, and it’s virtually impossible that we can represent a threat for the others. We were the victims, not the oppressors.'' So, if we, the deniers, would destroy this argument that the Holocaust happened, it would be much easier to preserve our arguments against Jews. Therefore, the history of Holocaust occupied an important position in the ideology of anti-Semites. Including the ones in Arab countries – you can buy Mein Kampf translated into the Arab very easily in the larger cities in the Arab world. And these people believe in these claims by Hitler, that Jews are insects, that something must be done in order to remove this constant threat, the Jewish threat, the danger for the German nation, and so on. So, is it possible to find an example of an official institution in Arab countries that combats such crimes by Nazis? No! I never heard about documental initiatives in Egypt or other Arab countries against deniers, against the movies or the internet, YouTube and Facebook activists. There is the question of, of course, politics, and the question of the legacy of the Arab-Israeli wars, etc. In Europe, the situation is different, nevertheless, the result is sometimes the same: The anti-Jewish propaganda, which is in most cases combined with fighting with the history of the Holocaust. 
But, if you ever have the opportunity to go to my archive, to the museum for a couple of days, I’m able, probably, to persuade the most stupid denier that Auschwitz happened. It would require some work, some time, nevertheless, that’s possible. I believe that we have enough arguments, enough documents. The problem is that of course, there is nothing like the one single document, the order written by Hitler saying ``Today, I decided to solve the Jewish problem in Europe, and to kill all Jews, Adolf Hitler''. Such a document does not exist. But, after carefully analysing, say, 200 documents, even the most stupid denier would be persuaded by the force of the arguments. Because, even assuming that the documents were falsified, they would have to recognise that it could not be done in such a way. There is, for instance, a document in which the commander of a certain unit in Auschwitz tried to give his opinion about his sergeant, promoting him before the high military ranks – some guards received military medals of their services in Auschwitz. He explained that the sergeant, was very active and very enthusiastic in collecting clothes during the action of evacuation of Jews. This means that in the result of this action of evacuation of Jews, the Jews were losing their clothes. Or, if we got the orders for the drivers of the truck. The trucks were bringing Zyklon-B from the factory in Erfurt. The problem was that Zyklon-B was used in Germany on a wide scale for delousing purposes, for killing insects. In one order we’ve got a sentence saying that the driver is going to Erfurt and the reason of the trip is delivery of Zyklon-B. Okay, why not. Somebody might say that Zyklon-B was used here, and that is the truth, for delousing of clothes. But there is a second order in which the reason of the trip is that the driver should bring the ``material for special Jew treatment'' or ``\textit{Materialien für die Judenumsiedlung}'', materials for resettlement of Jews. Assuming that everything was okay with the Zyklon and that the Zyklon was always used for delousing purposes here – so why, in other documents, did the German administration use such complicated euphemisms like ``materials for special treatment''? Why they did not simply name this as Zyklon-B, or as material for disinfection? My answer is that because this particular clerk who wrote the document, he knew that actually, a large quantity of Zyklon-B was used not for delousing purposes in Auschwitz, but for killing people. In this way, he wished to hide the truth. \\
So, if you give me, say, ten hours, I’d be able to persuade the strongest denier, I believe. Excuse me, I’m afraid I have to go to for the central ceremony in the city of the International Holocaust Remembrance Day.
