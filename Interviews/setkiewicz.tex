\section{Dr Piotr Setkiewicz}

\textit{Piotr Setkiewicz (born in 1963) is the director of the Centre for Research at the Auschwitz-Birkenau State Museum. He graduated from the Faculty of History at the Jagiellonian University in Kraków. Mr. Setkiewicz received his doctorate in 1999 at the University of Silesia in Katowice with a thesis entitled ``IG Farben - Werk Auschwitz 1941-1945''. He is the editor-in-chief of the scientific publication The Auschwitz Journals} (Zeszyty Oświęcimskie) \textit{as the head historian at the Auschwitz Museum.\\
We met him at Auschwitz Museum and Memorial on January 27th, 2018.}\par 
\vspace*{2em}
\textbf{Have you had any first-hand or second-hand experience with anti-Semitism over the course of your life?}

\textbf{Piotr Setkiewicz:} Over the course of my life? Always, almost every day. Just yesterday, I was called by a man who is anti-Semitic, who is denying the truth. He said that I can be blamed for all mistakes and all the false information that appeared in the media about the Holocaust. He said hat I'm supporting the fake version of history which is based on Jewish, Polish, and communist lies about the Holocaust, that we at the museum here are promoting the survivors who are lying, that we falsified the documents in my archive, and that we are taking money from Jews in Israeli shekels. Many such things. Many people in different countries, not only in Germany, but also in the United States, practically all over Europe, including Poland, surprisingly, try to persuade everyone that the Holocaust never happened. They say that Auschwitz or at least the most important objects here, like the crematoria and the gas chambers, were built only after the War, and that the gas chambers were actually used for delousing purposes, for disinfection.\\ 
My reaction usually is that I don't want to be involved in any sort of discussions with Nazis because as far as I know from my own experience, it's just a waste of my time, and the arguments always stay the same - for instance, that conditions in Auschwitz were relatively good, that it was necessary to isolate the enemies of the \textit{Reich} because it was a war and that, of course, sometimes it might have happened that somebody died here. But, nevertheless, according to them the number of victims is highly exaggerated, the number of one million people who were murdered in Auschwitz, and obviously, it's not true because the capacity of the crematoria was many times lower than we read in books, read in documents, etc.\\ 
Even some strange and surprising arguments are brought up in the discussions: For instance, there is a picture taken by somebody from the staff of \textit{IG Farben} works. \textit{IG Farben} was a huge factory of synthetic rubber; it was situated on the other side of the city. During the War, many thousand prisoners worked at the construction site there. Apart from the prisoners, of course, there was German staff, men and women, about 7,000 people, and they resided in the \textit{Siedlung}, the settlement. They lived somewhat apart from the concentration camp and they got, for instance, a sports club. There is a picture of two people who are fencing and there is a group of people in the background of the picture, and over them, there is an inscription, the name of the sports club in German, \textit{IG Auschwitz}. Now, the Nazis believe that this is a typical representation of the living conditions in Auschwitz, that the people in the picture are prisoners.\\ 
Other than that, one of the most important arguments is the presence of the swimming pool in Auschwitz, which is visible just behind us. We know that it was one of many water tanks for the fire brigade. In 1944, the Allies began to fly over Auschwitz and the SS suspected the threat of mass bombings and a fire, particularly in Auschwitz-Birkenau, so they built a number of such reserves of water for the camp fire brigade. And surprisingly enough, in Auschwitz, some of the prisoners - functionaries, the kapos or block overseers, mostly the German habitual criminal prisoners - organised something like a competitive swimming in this pool. There was something like that, as we know from the testimonies of survivors and from a fan who was taking pictures. Nevertheless, there was a difference between the situation of the functionary prisoners and the fate of regular prisoners at the concentration camp. \\
As I said, there a many such arguments, and many stupid discussions. The most difficult and rather new problem is the huge amount of articles, comments, entries on the internet. On the internet, you can write anything and there are plenty of web pages explaining that Auschwitz never happened. The problem is that recently, we are faced with more and more Holocaust deniers in Poland. For me, this was something I could not believe because many Poles lost their lives in Auschwitz, as well as the Polish Jews, and the knowledge about the German crimes in Auschwitz was obvious for everyone in Poland, for many years after the War. Now, I am afraid that there is a third generation of people who have no personal experience. They probably heard, ``my great-grandfather was in Auschwitz'', but the distance to history is too long. Just a few days ago, the Polish television showed a ceremony of the anniversary of Adolf Hitler's birthday on 20 April, by a group of Neo-Nazis meeting in the forest somewhere in Silesia. They were celebrating the anniversary, they had the \textit{Wehrmacht} uniforms, they baked a cake for the \textit{Führer} with the swastika.\\ 
Perhaps that is still marginal in Poland. But these views are shared by the people who recently protested against immigrants, against attempts from the European Union to persuade the Polish government to accept a certain amount of people from Syria and other countries. It's still not the same as with the Nazis, however, within the relatively large group of Poles who don't want to even hear about immigrants from Arab countries, there is a narrow margin of people, particularly young people – I’d say 20-25 years old, football fans, maybe - who try to link their anti-immigrant views with Nazi ideology. Here in the Museum, we encounter such people only very rarely. It's not a problem for us, but we fear that one day, we might be faced with a group of people with Nazi insignia wishing to visit the Museum - and the problem is, what to do then?\\
It's a good reason for us to do our work, to publish books, for example. I don’t want to be involved in any sort of direct discussions with Nazis, but we are providing arguments for many teachers, arming them with documents. German documents, I'd say, are more useful in discussions between students and teachers, because if we gather the testimony of a survivor, it’s easier for a denier to say: ``Well, it’s just nonsense, what this man is talking about''. For example, a few weeks ago I read a critical review of a book written by a survivor on the internet. The survivor wrote a book with about 300 pages about his experience in Auschwitz. This is basically a good testimony. There are many survivors that are simply trying to tell us their stories. Nevertheless, there are unfortunately, and it's quite normal and natural, some people who believe that they should add something to their stories. For instance, if they were in Auschwitz, they testify about certain happenings that took place in Birkenau. I understand that something like this might happen, that particularly the survivors who used to take part in memorial ceremonies and to meet with schoolchildren, being asked about the War experience, wish to say: ``I was in Auschwitz, I saw everything''.\\
Going back to this book: Although on one side he was generally credible, at one point in his book he said that he remembers that there was a small room in one of the barracks where the Germans, the capos, collected the bodies of those prisoners who died during the nights, and that it was only after three of four days when they collected the bodies and put them in a truck, and the truck was taking the bodies to the crematoria to be burned.\\
The author of this article on the internet made use of the fact that arguments like that are obviously not true because he found a German document in which the commander of Birkenau ordered to remove the bodies every day. For him, this is an obvious argument that this man was lying, and if he was lying on this particular point, on this particular page of the 300-pages book, probably the remaining 300 pages are also not credible.\\
Another category of sources, of course, are the testimonies by the guards themselves. They testified before the military tribunals of the Allied powers in West Germany, immediately after the War before the Polish courts, and then in Germany in the Frankfurt trials, and in many other locations. At least some of them they accepted their guilt and admitted: ``Yes, I saw the people being selected on the ramp. I saw the people being driven to the gas chambers, I saw the thousands of bodies and the smoke over the chimneys of crematoria.'' So, what is an argument of deniers against it? That they were forced to testify in this way, beaten by the guards, by British, Polish, whatever, communist, Jewish guards. That was why they had to give these false descriptions. \\
Then there is the last category - the German documents: The problem is that most of the German files from the chancelleries, from the office administration of Auschwitz, have been burned immediately before the final evacuation of the camp. If you take any testimony of a survivor who was here in January ‘45, you can usually information like: ``I saw the piles of papers, of documents on the crossroads of the camp, being burned by the guards''. And that's true. We've got a very small amount of original documents from German administration, a relatively large amount of documents from the German construction office, the \textit{Bauleitung}, probably because that was a separate unit of the SS-administration. It was situated in several barracks at a distance about 300 metres from here, not inside the camp. For some reason, when the SS escaped from Auschwitz, the just forgot to burn these files. \\
If you take into account all these surviving documents, they are good material for historical research, these plans and construction files of the central construction office of the SS. The amount is about 150,000 pages of documents, a large number of documents, and they are a very valuable source, which is very useful for our indirect discussions with these deniers. There is, for instance, a well-known author who wrote a number of books and in one of them, he tried to summarise the most important arguments of the deniers. One of them was about the first temporary gas chambers in Birkenau called ``the little red house'' and ``the little white house''. The ruins of little white house are still in Birkenau, but nevertheless, he said that it’s impossible that in so many surviving documents there is not any hint about these temporary gas chambers.\\ 
His argument is as follows: That if something like that had existed in Birkenau, then it must have been in the records of the trips of the trucks, especially for delivering bricks and other construction materials to the site of these gas chambers. There must be something, information about the bills, about the amount of money spent for the construction of the gas chambers – but there is nothing. Thus he tried to persuade his readers. But in fact, we have found seven documents that clearly indicate that these gas chambers existed. One of them costed 14,000 \textit{Reichsmark}, for instance, and the SS ordered three wooden barracks for each of these gas chambers, which were used as a temporary store of clothes of the victims. Each of these barracks costed about 17,000 Reichsmark and they were removed from the area in the moment when the Germans completed larger and more modern crematories in Birkenau, the old gas chambers were dismantled and destroyed. We have found requests from the Bauleitung asking for the remaining wooden barracks, because they might be used for some other purposes in the camp. And we’ve got the information that there is a need to send 500 prisoners on a day to dig trenches around the bunkers. That was because one day in July 1942, Heinrich Himmler, during his visit in Auschwitz, ordered to empty the old mass graves and burn the bodies on piles of woods.\\ 

\textbf{What do you think makes people want to deny the fact of Auschwitz?}

\textbf{Piotr Setkiewicz:} There are many reasons. Perhaps some of them believe that the government in Germany in the 30s was ideal, becuase everyone knew his place in the society - but now, what do we see around us? The politicians, thinking only about how to gain more money, are no true patriots. And the \textit{Führer}, he was a man who solved the problem of the economic crisis in Germany. He built the highways. He restored the pride of the German nation. These might be the arguments of the German deniers.\\
In other countries, I think, the reason is mainly anti-Semitism. The Holocaust, as a part of the Jewish history, is identified by most Jewish historians as a marking point in Jewish history and in the history of the modern nation. And for those people who are anti-Semitic, who don’t like the modern state of Israel, Auschwitz is a key issue just as well. There is the famous quotation from an interview with David Irving, the leading British denier, who said: ``If we sunk the battleship of Auschwitz, it would solve all the problems of the 20th century’s history''. Auschwitz is a focus, the most important part of the story. ``If we were able to prove that Auschwitz never happened, it would be easier to destroy the myth of six million of Jewish victims.''

\textbf{How has the number of Holocaust deniers developed over the last twenty-five years?}

\textbf{Piotr Setkiewicz:} Immediately after the War, it was the group of the German veterans mostly, particularly from the unions of the SS, who were deniers. There was a number of guards from Auschwitz even who took part in ceremonies where they solemnly declared that they were in Auschwitz, but never heard about any gas chambers or crematoria. That was the first wave of denial. Then, in the 60s, there was a rise of interest in the Holocaust. On the one hand, we've got the trial of Eichmann in Jerusalem, and on the other hand, a few years later, a series of so-called Auschwitz trials in Frankfurt. It was the time of student’s rebellion in Europe, the leftists began to be interested in he history of their own country. In the leftist press the pictures of leading Nazis who were still living in Germany appeared. The headlines were as follows: ``The assassins, the killers are still among us''. There was this popular belief of the leftist people that something must be done in order to cope with our past and to punish the people who, in many cases, continued their pre-War career after the War.\\ 
All those with a Nazi past whom the judges prosecuted – the technicians, engineers from the companies like \textit{IG Farben} – were renowned chemists and board directors in different companies in Germany. The young people at this time thought that something should be done with this problem of the history of their grandparents. If you observe the rise of the interest in the Holocaust, it was a ``natural'' reaction from the side of the right-wing radicals to deny the Holocaust.\\
Then in the 70s, there were some financial problems, some newspapers and journals published by the Nazi organisations disappeared. Nowadays, I believe, we can observe another wave of Holocaust denial – perhaps because of the internet. If in the 70s, there was a problem with publishing the journals, then now, in the era of internet, its costs almost nothing.\\ 
It is perhaps also a reaction to the idea of a united Europe. As for Brussels, we should think of ourselves as Europeans, to a certain extent, not only as Poles, Czechs, etc., but also members of a European family, who share the same values and the same budget. So, there is a reaction of the people who don’t like this concept of a united Europe, who think: ``Primarily, we are Poles. And we don’t want to be governed from Brussels, we got our own government, our own Polish Złoty, our own history.'' Of course, with the problem of immigrants and the terrorist attacks in other countries, among Poles there is a tendency to talk about it in this way. ``Because our friends – England, Germany – accepted many thousand Arab refugees and among them, there are of course terrorists, they got a problem. We don’t want to have any refugees on the Polish soil, so that we haven’t got the terrorist attacks and bombs in Polish cities.''

\textbf{How is this related to anti-Semitism?} 

\textbf{Piotr Setkiewicz:} If you believe that we are the better ones, that our values are better because we are Christians, or Europeans, it’s just one step to disliking the Jews because the Jews, ``they got too much money, they control the world bank system, they are primarily leftists and promote leftist values, and because we are nationalist to a certain extent, we don’t like to accept these foreign values. There is Mr Soros, who is a Jewish billionaire, and who is supporting the leftist organisations in Poland – we don’t like him because he is leftist, and because he is Jewish.''\\
Nevertheless, Auschwitz is still present somewhere in the background of such discussions as a good illustration of what might happen if we forget about these universal values. When we see the strangers not only as the other people, but also as the representatives of the black forces of international conspiracies. And then it’s very easy to cross the line, to be on the black side of the force. There is the question of reactions to the Nazism of German society in the 30s, here in the camp, the reaction of people who were not Germans, but who accepted the posts of \textit{Blockältester}, the Kapos – many Poles and even Jews here in Auschwitz, who received a stick and tried to exercise force over prisoners, to beat them. Because the prisoners could not work with efficiency, the stick is the best argument.\\ 
These were symbols and pieces of history that might have been used in  discussions about the condition of the modern world and the societies here. What should we do when we hear about atrocities, like a genocide in other countries? Is it possible to do such a thing? We heard about the massacres in Syria, or in Iraq, or in Sudan. What should we do? There’s not a single good answer for it. But if we look at ongoing discussions about human rights, for instance about the need to combat racism in many countries, Auschwitz is always somewhere, it's being used as a symbol, as a turning point, as a milestone in such discussions.
 
\textbf{Do you think that Holocaust denial is the most dangerous form of anti-Semitism nowadays?}

\textbf{Piotr Setkiewicz:} It is a part of the ideology - important, but not the only one. Of course, anti-Semitism is something much wider, it’s the popular belief that the Jews are controlling the world economy, or that the governments are under the constant pressure from the national Jewish organisations. Holocaust denial is a very important part of this ideology because anti-Semites believe that Jews are using the Holocaust as an argument to defend themselves, to blame other nations, or because they wish to control the market in countries like Poland, to get back their property that was confiscated during the War by the Nazis and which was taken over by the Communist government in these countries. That’s a good illustration of the situation in Poland.\\ Taken altogether, ``the Jews are members of the world conspiracy, and they are trying to use the argument of the Holocaust in their propaganda. We are poor guys who were persecuted over centuries and it’s virtually impossible that we can represent a threat for the others. We were the victims, not the oppressors. So if we, the deniers, could refute this argument that the Holocaust happened, it would be much easier to preserve our arguments against Jews.'' \\
I believe that with my archive and the museum, I’m probably able to persuade the most stupid denier that Auschwitz happened. It would require some work and some time, but nevertheless, it’s possible. I believe that we have enough arguments and enough documents. The problem is that of course, there is nothing like the one single document, the order written by Hitler saying: ``Today, I decided to solve the Jewish problem in Europe and to kill all Jews. Adolf Hitler.'' Such a document does not exist. But after carefully analysing, say, 200 documents, even the most stupid denier would be persuaded by the force of the arguments.