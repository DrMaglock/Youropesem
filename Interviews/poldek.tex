\section{Poldek (Leopold Yehuda Maimon)}
\begin{otherlanguage}{polish}
\textit{Leopold Yehuda Maimon, called Poldek, was born in Kraków in 1924. He went to a Hebrew elementary school and later to a Zionist grammar school. After the German invasion in 1939, he joined an underground organisation in the Kraków Ghetto. This organisation carried out an attack on a café visited primarily by Wehrmacht officers. At the age of 18, Poldek was deported to Auschwitz, where he also became a part of the underground resistance. Together with four other inmates, he managed to escape during the death march in 1945.\\
After the liberation, Poldek joined the secret Jewish revenge group Nakam. He is no longer convinced of their deeds today. He emigrated to Palestine illegaly in 1946 and was involved in an Aliyah organisation together with this wife Aviva. Today, Poldek lives in a retirement home in Ramat Gan in the outskirts of Tel Aviv. The interview with him took place there on September 20th, 2016.}\par
\vspace*{2em}
\textbf{Poldek:} Urodziłem się w Krakowie w 1924 roku. Miałem starszego brata. Chodziłem do hebrajskiego gimnazjum. To było normalne gimnazjum, takie jak wszystkie. Matura w naszym gimnazjum miała pełne prawa, była jak matura każdego gimnazjum, nie było żadnych kontroli państwowych – mieliśmy pełne prawa.

\textbf{Jakie ma Pan pierwsze wspomnienia, takie najciekawsze, najpiękniejsze, może wspomnienia właśnie z gimnazjum?}

\textbf{Poldek:} Ja mam tylko piękne wspomnienia, to była wspaniała szkoła. Wczoraj do mnie dzwonili z Krakowa, że postawili pomnik jednemu z naszych nauczycieli, który mnie uczył, był wzorem dla nauczycieli. To profesor Ferdhord, uczył języka polskiego. Pisał książki jako Jan Las i był wykładowcą na uniwersytecie. Jak wchodził do klasy, to była taka cisza, że można było usłyszeć muchę. I nigdy nie podnosił głosu, ale miał taki wpływ na uczniów i słuchać go było tak ciekawie, że nikt się nie odważył zrobić czegoś, co by mu przeszkadzało. 

\textbf{Czy jeśli chodzi o język polski, lubił Pan ten język tylko ze względu na nauczyciela, czy miał Pan jakieś zamiłowania humanistyczne?}
 
\textbf{Poldek:} Ja się już wychowywałem w języku polskim, w kulturze polskiej. Wszystko, co czytałem, wszystkie książki były głównie w języku polskim.
 
\textbf{Pana polszczyzna jest piękna. Jeśli tyle lat Pan pamięta tak dobrze język polski, to tylko pogratulować. Fantastycznie, że miał Pan takich nauczycieli.}

\textbf{Poldek:} Tak. Z wielką miłością wspominam moich nauczycieli – wszystkich, nawet takich, którym przeszkadzałem. Szkoła dała nam wszystko. Bez szkoły nie dało się żyć. Do pół do pierwszej żeśmy się uczyli, a po obiedzie była świetlica i można było uprawiać sport np. ping-pong, można było po dworcu grać w piłkę, był ruch harcerski dozwolony i ja też brałem w nim udział. Ruch harcerski z kierunkiem syjonistycznym, ale głównie to wszystko, co harcerzy cechuje, te same podstawowe wartości. Ja byłem syjonistą, zawsze, od 10 roku życia w tym ruchu harcerskim. Był on w szkole jedynym dozwolonym przez szkołę ruchem młodzieżowym.

\textbf{Czy wszyscy uczniowie byli do niego przekonani, czy w jakiś sposób byli skłaniani by przyjąć takie podejście?}

\textbf{Poldek:} Nie, byliśmy przekonani, to wszystko było dobrowolne, nie było żadnego musu, żeby ktoś brał w tym udział. 
  
\textbf{Jak Pan wspomina kontakty z młodzieżą z innych szkół, z rówieśnikami, którzy nie byli syjonistami, którzy nie podzielali Pana poglądów i poglądów Pana kolegów ze szkoły?}
 
\textbf{Poldek:} Ja nie miałem żadnych kontaktów, ja w ogóle ich nie znałem, ja byłem za młody, żeby znać, miałem piętnaście lat, jak wojna wybuchła. Przebywałem głównie w szkole, tam się też tańczyło i śpiewało.

\textbf{Czy Pana trudności zaczęły się, kiedy wybuchła wojna?}

\textbf{Poldek:} Wtedy wszystko się całkiem zmieniło. Musieliśmy się przystosować, jeszcze mieszkaliśmy w swoim mieszkaniu do 1941 roku, bo getto powstało w marcu 1941 roku. To było jeszcze w Krakowie, część ludzi uciekła na wschód, gdzieś do Rosji, ale większość została. A myśmy się spotykali, mimo zakazów, młodzież jest młodzieżą i zakazy są zakazami, więc spotykaliśmy się dalej i życie jakoś mijało, jeszcze nie było tak źle, jeszcze nie było tych obozów zagłady. One się zaczęły z końcem 1941 roku. Wtedy Kraków i Polska zostały podzielone na mocy paktu. Część poszła do Rosji, część do Niemiec, a część została środkiem kraju. Została środkowa Polska, a Kraków został stolicą tego kraju. 

\textbf{Jakie trudności zaczęły się dla rodziny po 1941 roku? Jak pan to wspomina, z czym to było związane? Czy z jakąś działalnością Polaków, oczywiście poza okupantem?}

\textbf{Poldek:} Trudności były szczególnie ze strony getta, był w sumie jeden pokój w niedużym mieszkaniu, między dwoma rodzinami… Życie się zmieniło. Nie było głodu, wydzielano jedzenie. Można było żyć normalnie, nie tak jak to miało miejsce w Warszawie. W Krakowie nie było trupów na ulicy. Kraków był czysty. My się spotykaliśmy przy bloku. Ja pracowałem. Głód się zaczął, jak byłem we więzieniu. Ale było ciężko, sprzedawało się, co się miało. Były kontakty z ludnością polską i sprzedawało się rzeczy i w ten sposób jakoś rodzinę utrzymywało. Pracowałem, ojciec pracował, brat pracował. Trochę żeśmy zarabiali. Jakoś się żyło. Zacząłem pracować już w 1941 roku. Musiałem, bo były łapanki. Musiałeś pracować, żeby mieć legitymację pracowniczą. Wtedy mogłeś dalej pracować. Dla mnie w życiu trudności się zaczęły właściwie w 1942 roku.
 
\textbf{Na czym polegał największy ucisk w getcie? Co Państwo odczuwali oprócz tej izolacji, piętna?}

\textbf{Poldek:} Prześladowanie było w momencie, kiedy Niemcy weszli. Co tydzień wychodziło jedno rozporządzenie, np. trzeba było nosić opaskę, że się jest Żydem. Nie można było jeździć. Było wiele publicznych łapanek Żydów. Łapano człowieka i musiał iść do pracy, wracał dopiero po dwóch, trzech dniach. Z każdym tygodniem było gorzej. Obelgi były bardzo trudne. Niemcy mieli bardzo wyszukany proces psychiczny.

\textbf{Próbowali zniszczyć Żydów psychicznie?}

\textbf{Poldek:} Tak, dla nich ten, kto znęcał się nad nami psychicznie, bardzo dobrze pracował. Myślę, że obelgi były jeszcze gorsze niż sama śmierć. Bo jak śmierć, to człowiek od razu umiera, a obelgi były stale. Niemcy mogli robić, co chcieli, nie było żadnego sądu. Oni czuli, że wszyscy Niemcy są nadludźmi. Dzisiaj chce się to trochę zamydlić, ale wtenczas wszyscy tak czuli. Widać to na filmach, na których Hitler mówił i wszyscy patrzyli na niego jak na Boga i wszystko, co powiedział, było święte. 
  
\textbf{Niestety, miał charyzmę, prawda? Złą, negatywną... }

\textbf{Poldek:} No tak! Miał kolosalną charyzmę. On wiedział, jak mówić do Niemców.

\textbf{Niemcy byli nad, a Żydzi byli gdzieś poniżej, a czy naród żydowski odczuwał to, że Polacy byli bliżej w stronę Żydów?}
 
\textbf{Poldek:}  Ciężko powiedzieć, to są rzeczy niebadane. Większość Polaków była antyniemiecka, bo Polacy widzieli w Niemcach wroga. To trzeba wiedzieć. Niemcy byli wrogiem Polaków i Polacy o tym wiedzieli. Ta piosenka, Rota polska, „Nie będzie Niemiec pluł nam w twarz”, była w sercach Polaków, ale w stosunku do Żydów niestety Polacy dzielili się na kilka rodzajów. Byli tacy, którzy wydawali Żydów – to była mniejszość – wydawali Żydów za jedzenie albo pieniądze, inni wydawali Żydów z radością, bo byli antysemitami.

\textbf{Czy duży był procent ludzi, którzy wydawali Żydów dla jakiejś satysfakcji, wyższości?}
 
\textbf{Poldek:} Nie powiedziałbym, że było ich dużo. Te rzeczy są jeszcze niebadane. Ale po wojnie Polacy, którzy pomagali Żydom, bali się swoich sąsiadów, że jak się sąsiad dowie, że on pomagał Żydowi… Polak miał karę śmierci za pomoc Żydowi, nie można o tym zapominać.

\textbf{I wielu ludzi na świecie dzisiaj zapomina, że tylko w Polsce ta kara śmierci obowiązywała, tylko Polacy ginęli za to, że pomagali Żydom.}

\textbf{Poldek:} Tego nie można zapomnieć, ale też było dużo Polaków, którzy podchodzili do Żydów z nienawiścią, bo mieli z tego korzyści materialne, na przykład żydowskie domy, które zostały opróżnione przez Niemców. W Polsce był bardzo silny antysemityzm według mnie, kierowany przez Kościół.
   
\textbf{Czy chodziło też o to, że Żydom często lepiej się powodziło? Byli zaradniejsi?}

\textbf{Poldek:} Nie, to nieprawda. Żydzi byli bardzo biedni. To są tylko opowiadania. Znam Żydów sprzed wojny, bo mój ojciec jeździł po tych miasteczkach, był przedstawicielem rozmaitych firm i na wakacjach z nim jeździłem. Żydzi byli bardzo biedni. Byli też bardzo bogaci, było ich od pięciu do dziesięciu procent, ale zamożnych było jeszcze od piętnastu do dwudziestu pięciu procent. Można powiedzieć, że było dwadzieścia pięć procent Żydów, którzy byli zamożni. Siedemdziesiąt pięć procent to byli Żydzi jak z opowiadań żydowskich, widziałem to na własne oczy. Jeśli Żyd był krawcem, to takim, który robił łaty.

\textbf{Wspomniał Pan, że to z Kościoła wyszły pierwsze sygnały antysemityzmu. Jak Pan to rozumie?}

\textbf{Poldek:} Kościół miał bardzo wielki wpływ na ludność polską, która była bardzo religijna. Nie można pominąć problemu analfabetyzmu w Polsce.

\textbf{Także w dużych miastach? W Krakowie?}

\textbf{Poldek:} Wszędzie. I księża to wykorzystywali. I wiadomość, że Żydzi zabili Jezusa, miała bardzo duży wpływ na mentalność rolników. Rolnik polski też był bardzo biedny. Biedota w Polsce była kolosalna.
 
\textbf{Czy spotykał się Pan z tym, że chociaż Kościół w jakiś sposób, bezpośredni lub pośredni, nawoływał do antysemityzmu, to Polacy będąc katolikami pomagali Żydom?}

\textbf{Poldek:} Kardynał Sapieha pomagał Żydom. Nie można powiedzieć, że wszyscy. Ja mówię o Kościele na wsiach, bo właśnie ludzie niewykształceni, analfabeci żyli z tego, co usłyszeli w niedzielę w kościele, co ksiądz proboszcz powiedział, było święte.

\textbf{Ale i wśród tych nieoświeconych, rolników, chłopów, byli tacy, którzy pomagali ryzykując życie, prawda?}

\textbf{Poldek:} Wszędzie byli. Ja sam miałem podczas wojny przyjaciół Polaków. Nie można mówić, że w stu procentach ludność polska była antyżydowska. Ale jeśli mówimy o atmosferze, to była ona antyżydowska.

\textbf{Proszę powiedzieć o początkach buntu w getcie.}
 
\textbf{Poldek:} Myśmy zrobili pierwszy wypad w Europie, nie tylko w Polsce, nie tylko w Krakowie. Pierwszy wypad na Niemców na większą skalę zorganizowała żydowska młodzież w Krakowie 22 grudnia 1932 roku. Byli w kontakcie z Armią Ludową, ale była to czysto żydowska grupa i zrobiliśmy to dwa dni przed Wigilią – pierwszy napad na tę kawiarnię, gdzie byli przed świętami wyżsi oficerowie niemieccy i gestapo. To zrobili Żydzi, ani jednego Polaka tam nie było, tylko napisane na tablicy, że byli Polacy.

\textbf{Jaka była pana rola w tej grupie?}

\textbf{Poldek:} Mieliśmy w kilku miejscach ludzi. Bardzo ciężko było dostać miejsce zamieszkania dla Żyda, bo Żydowi groziła kara śmierci. Podczas tej akcji byłem szefem, byłem odpowiedzialny za tę grupę, która wyszła na główny napad. Na główny napad wyszli z miejsca, gdzie kiedyś był szpital żydowski. To było 22 grudnia. To była główna kawiarnia cyganerii naprzeciwko teatru. Rzuciliśmy flaszki Mołotowa, pełno tam było wyższych oficerów. Część zawiesiła flagi polskie na budynkach administracyjnych, część rozlepiała ulotki, żeby zachęcić Polaków, by zaczęli się bronić, zachęcić do działania. Później zapaliliśmy ogień na ulicach, żeby straż pożarna jechała, żeby był bałagan.
 
\textbf{Jak duża była ta grupa?}

\textbf{Poldek:} Siedemdziesięciu ludzi.

\textbf{W jakim wieku? Czy tylko młodzi ludzie?}
 
\textbf{Poldek:} Ja byłem najmłodszy. Miałem wtedy osiemnaście lat, inni mieli osiemnaście, dwadzieścia lat.

\textbf{Jak zapatrywali się na te działania Polacy, rówieśnicy, młodzi ludzie? Czy współpracowali? Czy w jakiś sposób pomagali?}

\textbf{Poldek:}  Tak. W Polsce były trzy grupy podziemne. Jedna była Gwardia Ludowa, która była pod nadzorem partii komunistycznej, jedna była nacjonalna narodowa, ona mordowała Żydów, nie przyjmowała Żydów, a kto był w AK albo tacy, co mieli papiery jako Polacy, to tam zgodzili się ich przyjąć, ale z zasady nie przyjmowali Żydów. A Gwardia Ludowa była bardziej była skłonna i ona nam pomagała, współpracowaliśmy.
 
\textbf{Jak wyglądało codzienne życie w getcie, gdy zaczął się już bunt młodych ludzi? Jakie akcje podejmowaliście? Z jaką częstotliwością?}
   
\textbf{Poldek:} Byliśmy grupą trzydziestotrzyosobową. Kiedy dowiedzieliśmy się, że mordują Żydów masowo i nie ma właściwie żadnych szans przeżycia, było to bardzo ciężko przeprowadzić wśród inteligencji żydowskiej. Bo żydowska inteligencja w Krakowie była właściwie kulturalnie połączona z kulturą niemiecką. Kraków był pod zaborem austrowęgierskim, wszystko było niemieckie. Językiem był niemiecki, mój ojciec i matka mówili bardzo dobrze po niemiecku. Bardzo ciężko było zrozumieć, że tak kulturalny naród jak Niemcy, który wydał wielkich pisarzy, wielkich uczonych, wielkich kompozytorów, będzie mordował ludzi tylko dlatego, że mają inną wiarę, trudno było w to uwierzyć. Bo to było naprawdę nie do wiary. Myśmy szukali drogi, jak uciec z Polski, żeby wyjechać do Palestyny, bo ta grupa to byli ludzie wychowani w ruchu syjonistycznym. A to się nie udało, bo tak ściśle było wszystko zamknięte. Niemcy byli bardzo dobrze zorganizowani, pod tym względem niestety nie można było wyjść z sideł niemieckich. Przyszły wiadomości z Wilna, że było bardzo mało ludzi. W Krakowie za dużo nie było. W Krakowie było tak dwadzieścia tysięcy Żydów w getcie. Na początku 1942 roku nie było poczty, wiadomości przechodziły przez dziewczęta, które były kurierami. Te grupy syjonistyczne były we wszystkich krajach, bo ruch syjonistyczny był bardzo rozległy przed wojną. Główny ruch syjonistyczny był w Polsce. Tam była kultura żydowska, młodzież żydowska przyjeżdżała tutaj budować kraj, głównie ludzie z Polski budowali ten kraj. Z początkiem 1942 roku dowiedzieliśmy się, że mordują Żydów masowo. Wtedy rozpoczęły się te rozmowy między nami, co powinna młodzież zrobić, czy dać się zamordować bez żadnego oporu, czy stawiać opór. Bardzo łatwo postanowić, że się stawia opór, ale myśmy byli wszyscy studentami, z bronią nie mieliśmy nic 
wspólnego. Nie było rewolweru, nie było młotka nawet. W końcu przystaliśmy do współpracy z Gwardią Ludową, ona nas przyjęła, pod ich kierunkiem mieliśmy współpracować jako organizacja podziemna. I to polegało na tym, że trzeba było zorganizować jakiś wypad mały na fabryki, gdzieś, gdzie były mundury niemieckie. A raz na kolej, która szła na wschód. Pierwszy rewolwer zdobyliśmy tak, że wyszło nas trzech ludzi na plac otoczony plantami, pod wieczór, było ciemno, napadli na policjanta niemieckiego, z siekierą, zabili go i to był pierwszy rewolwer, jaki zdobyliśmy. Także to były kolosalne trudności budować grupę podziemną, bez pieniędzy, bez poparcia ludności.

\textbf{I bez przygotowania wojskowego.}

\textbf{Poldek:} Byliśmy gotowi do działania. I teraz była ta trudność, że musieliśmy wyjąć ludzi z getta. Po pierwsze zaczęliśmy być znani. I to było po dwóch wysiedleniach dziewięciu i pół tysiąca Żydów do Bełżca, gdzie był obóz zagłady. Dwóm ludziom udało się zbiec stamtąd. Wiedzieliśmy, że nie mamy dużo czasu, że nas w końcu złapią. Niemcy byli bardzo dobrze zorganizowani i mieli pomoc donosicieli. Byli też wśród Żydów donosiciele, Niemcy zorganizowali grupę donosicieli żydowskich, którym za pomoc przyrzekali lepsze warunki, ale nie było donosicieli polskich. Także wiedzieliśmy, że nie mamy dużych możliwości przeżycia, że nasz czas jest bardzo krótki, że musimy działać szybko. 22 grudnia był ten wielki wypad. Wzięliśmy sześćdziesiąt osób, była tam jeszcze jedna grupa żydowska, tzw. "`Iskra"', która była całkowicie pod dowództwem komunistycznym, liczyła też jakieś sześćdziesiąt osób. I razem wykonaliśmy pierwszy napad na Niemców, pierwszy w całej Europie na skalę, można powiedzieć, międzynarodową. 
 
\textbf{I jak się zakończył?}

\textbf{Poldek:} Zakończył się tragicznie. Grupa mieszkała głównie w baraku na ulicy Skawińskich i stamtąd wyszliśmy na to działanie. Ja mieszkałem poza gettem u jakiejś pani, która nie wiedziała, że jestem Żydem, bo miałem polskie papiery, fałszywe, ale polskie. Nie wiem, jak się to stało, ale po tym, po tej akcji na cyganerię, oni przyszli do tego baraku. Były dwie wersje tej historii. Jedna, że jeden z tych trzech, który rzucił granatem na tę kawiarnię, nie był Krakowianinem, swoim zachowaniem zwrócił uwagę i poszli za nim, a on nie wiedząc o tym zaprowadził ich do baraku. Według drugiej wersji wśród nas znalazło się przypadkowo dwóch ludzi z obcych szeregów, którzy zostali zaaresztowani z początkiem grudnia i zostali zwolnieni pod warunkiem, że będą donosicielami. I oni donieśli. Nie wiem, co jest prawdą, w każdym razie widziałem, że wszystkich, którzy tam byli, zaaresztowano. Potem robiliśmy jeszcze jeden napad. Getto zostało zlikwidowane 13 marca 1943 roku i potrzebowaliśmy pieniędzy. Byłem jednym z trzech, którzy dokonali napadu pieniężnego na jakąś rodzinę w Bochni. Napad się udał, ale mnie złapali i spędziłem miesiąc w celi śmierci na Montelupich, a później w Auschwitz dwadzieścia dwa miesiące. I tam też byłem w takiej grupie bojowej, bo w Auschwitz też była grupa podziemna. 
  
\textbf{A czy w podziemiu w Krakowie, jeszcze w czasach getta, zanim trafił Pan do Auschwitz, spotkał Pan jakieś przejawy antysemityzmu ze strony Polaków?}

\textbf{Poldek:} Tam nam pomagali, znaczy mieli pomagać więcej, ale można powiedzieć pomagali częściowo, nie dali nam odczuć, że my jesteśmy już straceni, że można nas zamordować.

\textbf{Co bardziej Panu utkwiło w czasach wypadów, działań: czy pomoc Polaków czy niechęć i na przykład donosy, czy jakieś przejawy antysemityzmu?}

\textbf{Poldek:} Osobiście spotykałem się z Polakami, którzy pomagali. Miałem szczęście, bo przez długi okres miałem wygląd nie tak żydowski, można powiedzieć, mówiłem po polsku dosyć dobrze i miałem włosy jaśniejsze, więc nie byłem podobny do Żyda. Miałem kontakt z Polakami, którzy pomagali, miałem przyjaciela. Spotykałem dobrych ludzi, trochę też złych, ale głównie dobrych ludzi tam w Auschwitz.
 
\textbf{Jak w Auschwitz doszło do tych działań, w których brał Pan udział? Czy Pan był ich inicjatorem?}
  
\textbf{Poldek:} Nie, tam nie było inicjatorów, to wszyscy podjęli. Dostaliśmy wiadomość, że będą chcieli zlikwidować obóz w momencie, gdy Niemcy będą musieli go opuścić. Myśmy zbierali broń i przygotowywali różne sposoby obrony na chwilę, gdy będą chcieli zlikwidować ten obóz. Wyszliśmy z założenia, że jak się będziemy bronić, to uratuje się więcej osób, niż gdy nie będziemy się bronić wcale. Myśleliśmy, że gdy dojdzie likwidowania obozu i Niemcy będą chcieli nas zabić, nie będą mieli dużo czasu, więc gdy będziemy walczyć, to część się uratuje, część zabiją, ale więcej nas się uratuje niż gdyby wszystkich zabili. Obóz był kierowany przez więźniów i była tam walka między trzema grupami, była grupa więźniów niemieckich, była grupa Polaków z AK i była grupa żydowska. Między tymi trzema grupami była walka o miejsca, które były ważne w obozie, od których zależało życie ludzi, na przykład kto był kapo. Grupą, która rządziła, właśnie byli Niemcy, skazani głównie za morderstwa, za większe kradzieże. Druga grupa, polska, miała lepsze warunki niż Żydzi, nie mordowali ich masowo, to byli ludzie, których aresztowano i mieli możliwości dostawania później paczek i utrzymywania kontaktów z rodzinami. To im dawało możliwość przeżycia, bo każda kromka chleba była na wagę życia lub śmierci. I była grupa żydowska, która nie miała żadnych możliwości przeżycia, jedynie trzy, cztery miesiące w Oświęcimiu, jak się nie paliło papierosów, bo jak się paliło papierosy, sprzedawało chleb za papierosy, to życie się jeszcze skracało. Jeśli ktoś miał jakiś specjalny, dobry zawód, Niemcy go potrzebowali, miał szansę przeżyć. Tam, gdzie ja byłem przysłany do Auschwitz, budowano miejsca, w których produkowana była benzyna syntetyczna. 
 
\textbf{Z przejawami jakich zachowań ze strony Polaków się Pan spotykał w Auschwitz? A czy Polacy- współwięźniowie w jakiś sposób pomagali lub szkodzili Żydom?}

\textbf{Poldek:} Ja mam bardzo dobre wspomnienia. Byli Polacy lepsi, byli gorsi, głównie byli antyżydowscy.

\textbf{I czym to się przejawiało, wiadomo, pewnie walczyli o swój kawałek chleba, ale czy donosili?}
  
\textbf{Poldek:} Polakom nie brakowało chleba, bo Polacy dostawali paczki, dlatego mogli przeżyć. Wśród Polaków nie było też selekcji. W Auschwitz co miesiąc była selekcja, kto miał, a kto nie miał pójść do krematorium. Żydzi mieli nie dostawać żadnej pomocy i w takich warunkach mogli przeżyć dłużej ci, którzy byli posyłani do roboty przy budowie, dostawali wtedy jedzenie.

\textbf{Bo byli potrzebni?}

\textbf{Poldek:} Tak, tak, byli potrzebni, więc dostawali trochę więcej jedzenia. 

\textbf{To w takim razie, jaki mieli polscy więźniowie interes w tym, żeby przeszkadzać Żydom?}

\textbf{Poldek:} Żaden, tak po prostu było. Niestety ludzie nie kochają się tak bardzo, jak jeden jest słabszy, to silniejszy go wykorzystuje.

\textbf{Jaką pracę wykonywał Pan w Auschwitz?}
 
\textbf{Poldek:} A, roboty głupie, bo przyszedłem tam z wyrokiem, więc byłem w karnej grupie przez pierwsze parę tygodni. Dostałem podwójne zapalenie płuc i opłucnej, znalazłem się w szpitalu i właściwie esesman mnie zanotował na następny dzień do krematorium. Wiedząc o tym, że jestem ostatnim z tych, którzy zrobili ten napad na cyganerię, powiedziałem, że tam, w celi śmierci było nas wielu, ale oni zostali zastrzeleni na Montelupich. I myślałem, że jestem ostatni z tych i że nazajutrz idę do pieca. A kręcił się tam jakiś człowiek, bo to w szpitalu się chodziło bez numerów, więc nie było widać, czy się jest Żydem, czy Polakiem, czy Niemcem. Ale on bardzo ładnie się zachowywał wśród więźniów, w obozie w Auschwitz nie tak było przyjęte. Poprosiłem go, żeby przysiadł obok mnie, bo chcę mu coś opowiedzieć, w końcu miał taką dobrą pracę, to przeżyje może ten obóz. I przyszedł, usiadł, zacząłem mu opowiadać i okazało się, że on był z tej samej organizacji co ja, i że był dziesięć lat starszy, był farmaceutą, złapali go, jak przechodził przez granicę, chciał uciec do Palestyny. Jak się dowiedział, kim jestem, poszedł do szefa tej podziemnej grupy, bo się okazało, że jestem może najmłodszym więźniem politycznym w Auschwitz. Ten człowiek miał wielki wpływ. Kazał lekarzom zniszczyć moją kartę chorobową, w której było napisane, że na drugi dzień miałem iść do pieca i zapisał mnie jako nowego chorego i tak mnie uratowali. 
  
\textbf{Ja się Pan wydostał z obozu?}
 
\textbf{Poldek:} Uciekłem podczas marszu śmierci z grupą, w której nas zaaresztowano razem. Bardzo chcieliśmy uciec i postanowiliśmy, chociaż nie było możliwości uciec z Auschwitz. Wiedzieliśmy, co nas czeka, jeśli spróbujemy. Jedyna możliwość była prawie nielogiczna, ale była. Uciekliśmy drugiego dnia marszu, udało nam się. To było dwudziestego stycznia czterdziestego piątego roku.

\textbf{Co się z Panem działo po tym, jak udało się Panu uciec z Auschwitz?}

\textbf{Poldek:}  Mieliśmy znowu wielkie szczęście. Uciekliśmy w nocy i szliśmy piechotą, nie więcej niż sześćdziesiąt kilometrów od Auschwitz. I przyszedł do nas jakiś żołnierz niemiecki, byłem pewien, że chciał nas zabić. A powiedział do mnie po polsku, nie wiem, dlaczego akurat do mnie, że musimy się tu gdzieś schować, bo tu niedaleko jest front, a tam całe SS i nas na pewno złapią. I znikł. A niedaleko była jakaś wioska, gdzie jeszcze byli Polacy. I około północy poszliśmy do wioski i zaczęliśmy pukać od okna do okna. Ale kto we wsi w nocy otworzy drzwi, kiedy niedaleko front? Aż słyszę odgłos z jednego domu, w którym przebywała siedemnastoletnia panienka, ja miałem wtedy dwadzieścia jeden. Poprosiłem ja, żeby powiedziała, że jej narzeczony uciekł z obozu pracy i szuka schronienia. Ona spojrzała i wpuściła nas do domu, a tam była jej mamusia, starsza pani i dała nam zezwolenie, żebyśmy tam weszli.

\textbf{Czy był Pan wtedy w obozowym ubraniu?}

\textbf{Poldek:} Tak, bo nie miałem nic innego do ubrania.
  
\textbf{I oni się nie bali Pana przyjąć? Uciekiniera z Auschwitz?}

\textbf{Poldek:} Oni nie wiedzieli, co to jest. W obozie myśmy się przygotowali do tej ucieczki, mieliśmy czarne płaszcze z czerwonym pasem, byliśmy długo więźniami, przeszliśmy wszystkie progi, mieliśmy kontakty, to i jakąś dobrą koszulę, dobre buty udało się zdobyć.

\textbf{A od kogo? Skąd te kontakty, skąd te rzeczy? Od Niemców, czy od Polaków, od Żydów?}
 
\textbf{Poldek:} Nie, to więźniowie, którzy pracowali. Ta grupa nazywała się Kanada. Dlaczego Kanada, nie wiem, ale nawet jak się było więźniem, to się miało wszystkie kontakty, bez tego nie można było przeżyć. A to właśnie dzięki podziemiu miałem rozmaite chody, jak się to mówi. Przechowali nas siedem dni i po siedmiu dniach przyszli Rosjanie i poszliśmy do Krakowa na piechotę, przeszliśmy koło Auschwitz i w moje urodziny wszedłem do Krakowa. 
 
\textbf{Ale prezent!}
 
\textbf{Poldek:} Tak.
   
\textbf{Czy potem spotkał się Pan jeszcze z rodziną?}

\textbf{Poldek:} Nie. Moja mamusia została zamordowana w obozie. Przechodziła przez plac, gdzie się odbywały apele i dowódca obozu "`Skarżysko"' badał, czy dobrze celuje… I zastrzelił ją. A ojca wysłali do obozu w Bełżcu.

\textbf{A co z bratem?}

\textbf{Poldek:}  Z bratem spotkałem się później, dostałem wiadomość w 1946 roku, że brat jest w obozie w Niemczech i pojechałem tam. Byłem już w innej grupie, grupie, która brała zemstę na Niemców. Pracowaliśmy, byłem w Paryżu, w Czechosłowacji, miałem tam zadanie, to był obóz więźniów SS i myśmy tam posmarowali chleb arszenikiem, który miał zabić tych esesmanów.

\textbf{A w Krakowie gdzie Pan znalazł miejsce po powrocie z Auschwitz? Jak się toczyło dalej Pana życie?}

\textbf{Poldek:}Przyjechałem do Krakowa 2 lutego roku z zamiarem jak najszybszego wyjazdu z powrotem do Palestyny. Byłem syjonistą i chciałem jak najszybciej wyjechać. Spotkałem się z człowiekiem z Warszawy, który pomagał Żydom wyjechać do Bukaresztu, a stamtąd mieliśmy nadzieję wyjechać do Palestyny.

\textbf{I od razu się udało?}

\textbf{Poldek:} Nie.
 
\textbf{A jak wyglądała ta droga?}

\textbf{Poldek:} Zostałem jeszcze rok w Europie, bo byłem wtedy w grupie dokonującej zemsty na Niemcach.

\textbf{Na czym polegała ta zemsta, Pana działania?}
 
\textbf{Poldek:} Były różne, moim głównym zadaniem było robienie pieniędzy. Byłem od tego, bo myśleli, że ja się na tym znałem. To był obóz Niemców, w którym były wyrabiane szylingi fałszowane, bardzo dobrze, bo to wyrabiało państwo niemieckie, szylingi niemieckie, które miały w ten sposób zniszczyć gospodarstwa niemieckie. To się działo zaraz po wojnie. Te szylingi były bardzo tanie, a we Włoszech, gdzie o tym nie wiedziano, można było sprzedawać drogo, więc myśmy kupowali to w Niemczech, sprzedawali i za te pieniądze utrzymywaliśmy tę grupę.
 
\textbf{Jak spekulanci dawniej w Polsce.}
  
\textbf{Poldek:} Tak, tak spekulanci, ja byłem wielkim spekulantem.
  
\textbf{Jak długo Pan tak pracował?}

\textbf{Poldek:} Rok. W lipcu 1946 roku przypłynąłem nielegalnym okrętem do Izraela.
  
\textbf{Czy ma Pan jeszcze jakieś wspomnienia, jeśli chodzi o Polaków? Pozytywne, negatywne? W tym ostatnim okresie, kiedy wyszedł Pan z Auschwitz, jakie wtedy było nastawienie Polaków do byłych więźniów Auschwitz?}

\textbf{Poldek:} Postanowiłem w Polsce nie zostać, mimo że miałem kolosalną pomoc ze związku partyzanckiego. Raz mnie przyjęli do związku partyzanckiego i dostałem legitymację partyzanta i otworzyli mi drogę na studia, wszystko chcieli mi dać, chcieli mi dać stopień kapitana, chcieli mi naprawdę pomóc, ale jak przyszedłem do Krakowa, to pierwsze, co usłyszałem od tego stróża, gdzieśmy mieszkali "`To tyle Żydów was zostało?"' 

\textbf{Jak Pan myśli, dlaczego cały czas było takie samo nastawienie do Żydów, pomimo że Żydzi tyle przeszli, mimo że Polacy mieli poczucie, że to, co działo się w Auschwitz, było bestialstwem, bo przecież Polaków to też dotykało?}
  
\textbf{Poldek:} Nie mogę na to odpowiedzieć, to dla mnie niezrozumiałe. Nienawiść do Żydów była ogromna po wojnie. Kielce były, Rabka, niebezpieczeństwem było jeździć po Polsce dla Żyda. A to nie było kierowane przez państwo, to nie było zorganizowane przez nikogo, to były pojedyncze grupy, ludzie sami zabijali ludzi. Wiemy, co się działo w Polsce po wojnie, dlaczego się to działo, jest dla mnie największą zagadką, nie potrafię jej rozwiązać, niestety. Jestem związany z kulturą polską, ja się nie wstydzę tego, że byłem w Polsce wychowywany, mam wiele sentymentów do Polski, kocham Kraków, ale z wielkim bólem to mówię, że takie wypadki niestety miały miejsce po wojnie i tego nie można zrozumieć. 

\textbf{Ten ból nosi Pan w sobie jakby w imieniu narodu. A czy po przyjeździe do Izraela utrzymywał Pan jakieś kontakty z Polską? Czy wracał Pan do Polski?}

\textbf{Poldek:} Tak, byłem w Polsce, byłem zaprzyjaźniony z rodziną, która uratowała członka naszej organizacji, przychodziłem tam do nich, mieszkali w Prokocimiu, był tam syn w naszym wieku, byliśmy bardzo zaprzyjaźnieni. 

\textbf{Odwiedzał Pan Kraków i dom, w którym Pan mieszkał?}

\textbf{Poldek:} Do domu mnie nie chcieli wpuścić, nie chcieli mi otworzyć drzwi.

\textbf{Dużo razy był Pan w Polsce?}

\textbf{Poldek:} Tak, robiliśmy film o Krakowie, o tej grupie, w ten sposób się spotykaliśmy. Żydzi, różni Żydzi, szczególnie ci starzy, ale też inni mają wiele pretensji do Polski, do ludności polskiej i to jest dosyć zrozumiałe. To trzeba zrozumieć. Mimo że Polacy nie byli z Niemcami, byli przeciwko Niemcom, ale zachowanie Polski, narodu polskiego nie było… dość to ciężki temat…
 
\textbf{My, Polacy cierpimy teraz z tego powodu, bo często te głosy, że jesteśmy antysemitami, że byliśmy antysemitami, są głośniejsze niż te, że pomagaliśmy. Przecież ogromna część Polaków jest wśród Sprawiedliwych Wśród Narodów Świata.}

\textbf{Poldek:} Tak, to się należy Polsce.

\textbf{A przecież inne narody, na przykład Holendrzy, też wydawali Żydów… My cierpimy, że na naszych terenach było Auschwitz, były obozy.}

\textbf{Poldek:} No, nie, nie tylko. Żydzi europejscy byli głównie w Polsce, trzy i pół miliona Żydów było w Polsce, a w Holandii ilu było Żydów? Trzysta tysięcy? Pięćset tysięcy? Żydzi jako grupa byli w większości w Polsce. Teraz nie ma już tylu Żydów w Polsce, w Krakowie jest osiemdziesięciu ludzi zapisanych jako Żydzi, wcześniej było sześćdziesiąt pięć tysięcy.

\textbf{A czy myśli Pan, że takie zachowanie wobec Żydów to jest cecha narodu polskiego? Czy gdyby padło na jakikolwiek inny naród, na przykład Czechów, Węgrów, Włochów i na ich terenie byłyby obozy koncentracyjne, obozy śmierci, czy zachowywaliby się podobnie? Czy to tkwiło w mentalności Polaków?}

\textbf{Poldek:} Nie, przecież antysemityzm jest na całym świecie, tak nie można powiedzieć, tylko zmiana była w Niemcach taka, że potrafili zmienić kierunek. To bardzo dziwne dla mnie, niezrozumiałe, bo właściwie główna nienawiść powinna być między Żydami a Niemcami. A tak nie jest, Niemcy są dzisiaj największymi przyjaciółmi z Izraelem.
\end{otherlanguage}