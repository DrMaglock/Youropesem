\section{Kalevs Krelins}

\textit{Kalevs Krelins is a rabbi in the Peitav Shul Synagogue in Riga and at the same time chief rabbi in Lithuania and Vilna.\\
He was born in Moscow and studied in Jerusalem. He served as a Rabbi in a school in Copenhagen, Denmark, in Heidelberg, Germany, and for a Young Israel community in the United States. In 2012, he was asked to serve as the Rabbi of the community in Riga. Together with Shimshon Daniel Isaacson, he has been appointed as the chief rabbi of Lithuania in 2016. Besides, he also works as the} Mashgiach \textit{(Hebrew word for supervisor) of the European Council of Kashrut (EEK) in the Baltic states.\\
The interview took place in the Peitav Shul Synagogue on September 28th, 2017. Throughout the conversation, Mr Krelins exclusively expressed his own views, not those of the Jewish community in Riga.} \par 
\vspace*{2em}
\textbf{How did anti-Semitism develop in Latvia after World War II?}  

\textbf{Kalevs Krelins:} It’s hard for me to describe, because I didn’t spend the years after the War here. As far as I know, the Soviet Union had a pretty powerful control over anti-Semitism. They were able to bring it higher or lower according to the international situation. As a child, I was raised in Moscow. For us, Riga was always the top of Jewish development compared to Moscow in terms of civility, and also in terms of religious freedom. There was no religious freedom in the Soviet Union, but in terms of what they could do here compared to what they could do in Moscow and other cities, it was an example of a very well-functioning community. They had kosher meals, they baked matzah and sent it to the whole Soviet Union. They always performed circumcisions. I do circumcisions myself. I travel to other countries and people say: ``We had a circumcision thirty or forty years ago, and the guy came from Riga''. Riga was always the place. For sure, there was the KGB, people were tortured. But if we compare it with Moscow, it was not so bad, and many very knowledgeable people were raised here after the War in Soviet times. That’s what I know from stories. My own experience here is limited to the last five years. I don’t see major issues with anti-Semitism here.  We all know that we cannot judge any place according to strange local people. I believe that the government here is very nationalistic, but in my opinion, their nationalism is related to the conflict between Russians and Latvians, so the Jews are not in the picture. We used to say that we can’t relax, because they’ll finish the conflict with the Russians and then they come after us, but we thank G$\sim$d that we don’t have any problem. There is a joke: ``they finish with them, they’ll start with us.''  

\textbf{Do you think there is some reality behind this joke?}  

\textbf{Kalevs Krelins:} Historically, yes. History shows that Jews were always seen as an enemy. When a new enemy came, the Jews went to the second line. As soon as there is peace with that enemy, they go back to see the Jews as their enemies. I hope I exaggerate. But we have to be aware. 

\textbf{In Germany, synagogues are protected by the police. Is it the same in Riga?} 

\textbf{Kalevs Krelins:} Here we don’t talk about protection. I know how the synagogues in Germany are protected. When I was rabbi in Heidelberg, I once parked at my parking place next to the synagogue. It was September 11th, the year after the attack. The police didn’t recognize me, so they jumped on me and checked me. They take it very seriously. Here people go around the whole country, they feel that nothing happens. It’s not only about synagogues; in the whole country, they're relaxed. They’re only afraid that Russia will invade. In Germany and France, synagogues are much more protected. I personally never had any bad feeling in Germany, but I was in the most intelligent place, Heidelberg, which is a university city. Here in Riga, I don’t feel anti-Semitism.  

\textbf{So you don’t have any problems walking around with a kippah?} 

\textbf{Kalevs Krelins:} First of all, I don't walk around with my kippah. I wear a casquette, for a simple reason. I came here from New York, where it’s easy to walk around with my kippah. I know that there are many people who hate Jews, but if someone who hates Jews and wants to express himself to a Jew meets me with my kippah in New York, he can go further down the street and meet someone else with a kippah. Here in Riga, the guy who wants to say or do something has to wait a couple of years to meet the next Jew. That’s why they hurry up to express themselves. Sometimes on Saturday, I go with a hat, sometimes I go with my kippa, my kids too, but we've never had any issues. Maybe once or twice somebody said something. 

\textbf{What are the differences regarding anti-Semitism be-tween Germany and Latvia?}  

\textbf{Kalevs Krelins:} My personal experience is that Germany recognised and experienced the pain of what it has done, but Latvia never did. Latvia and the Latvian people were extremely cruel with the Jews. They say that the German government gave them free rein and the Latvians did the whole job. The Latvians, Lithuanians and Ukrainians were the most cruel, more cruel than the SS themselves. The people who remember say that the German government was certainly terrible, but as people, Latvians, Lithuanians and Ukrainians were much worse than the Germans.  

\textbf{Is the Holocaust remembered and spoken about in Latvia?} 

\textbf{Kalevs Krelins:} No, they say that there was the occupation. They always say that they were under occupation, ``we’re under German occupation, we’re under Russian occupation, we’re under Swedish occupation 400 years ago, they are always under occupation'', it's not their fault. But they did their best to destroy all the Jews. The terrible thing is that the number one or number two in the destructions was Herberts Cukurs, who is a national hero. He's not officially recognized, but officially praised for his achievements: He built planes, he flew to the Arctic. There was a musical about his greatness here. We see that he has achievements, but these achievements can’t cover up what he did. He was killed by the Mossad during the sixties in Uruguay. He fled from America, from Russia, he fled from everyone. Many war criminals fled to Argentina. The Mossad killed them. They now say that it was a mistake that they didn't bring him to court. 

\textbf{Do you mean that Latvians are not remembering and recognizing their own guilt?} 

\textbf{Kalevs Krelins:} No, they are not recognizing their own guilt. They do say that unfortunately there were some Latvian people who collaborated. They have Waffen SS marching here, you know, every 16th of March. They say it's not exactly against Jews, but against Russians. When they say it's against Russians, everyone is forgiven. 

\textbf{Is the Jewish community offended by these marches?}   

\textbf{Kalevs Krelins:} Definitely. The Jewish community is very offended by these celebrations. We know that some of these people from the Waffen SS participated in mass murder.   

\textbf{Do you think that the government is distancing itself enough from this march?}  

\textbf{Kalevs Krelins:} It’s very hard to follow what the government is doing here. I don’t want to speculate, but certain members of the parliament and of the government are participating in or praising the celebrations. And there are very few demonstrations against it, just expressions. Usually when they catch somebody who expresses himself against it, they say it is an agent of the Kremlin.  For example, when somebody screams ``You are Nazis'', ``You are killers'', they take him aside for disturbing the public order and say it’s inspired by Putin.  

\textbf{Does the Jewish community organise any demonstrations against the celebrations?} 

\textbf{Kalevs Krelins:} No, I don’t think so.  

\textbf{Would they be seen as agents of the Kremlin if they organised demonstrations?} 

\textbf{Kalevs Krelins:} Yes. Latvian Jews were almost all killed during the War, ninety percent. After the War, Latvia became a very developed part of the Soviet Union. It was the Silicon Valley of the Soviet Union. I remember that all sorts of computer parts and cars were made here. Latvia was among the top technologically. So they sent many professionals from Russia here, among them many Jews. A big part of the Jewish population of Latvia today are descendants of Russian Jews that came as professionals to develop the economy of Latvia. So, most Latvian Jews are actually not Latvian. For us, if a person is a Jew, it doesn't matter where they come from. But for them, Latvian Jews and Russian Jews are different. Latvian Jews were exterminated and now, these Russian Jews are associated with Russia.  

\textbf{Do you think that those who are anti-Russian are also anti-Semites and vice versa?} 

\textbf{Kalevs Krelins:} A bit. I have to admit that many Jews who were oppressed in Russia went to the communist movement after the Revolution. Jews were one of the most oppresed nations, so they went against the system and joined the reds, the communist party. 

\textbf{Did they consider themselves as Jews?} 

\textbf{Kalevs Krelins:} They considered themselves as Jews. They were secular, they considered themselves as people who stand against different types of oppression. And oppression of the Jews was one of the oppressions in Russia before the Revolution.   

\textbf{How big is the Jewish community here?} 

\textbf{Kalevs Krelins:} We don’t have official memberships like in Germany, for example. In Germany it works with \textit{Kirchensteuer}, the register. Here we don’t have such numbers. Very roughly, we talk about 8,000 Jews in Latvia. I know that the Jewish community in Riga has 100 members, but that doesn’t mean that 100 people are coming to the services. On a daily basis, about 15 people come. On Saturday, we talk about 40 people, more or less, and on high events it’s packed. It depends on the weather, the time of the year, but it’s a lot, 150-200, sometimes more.  

\textbf{How have the numbers developed since the independence of Latvia?}  

\textbf{Kalevs Krelins:} Many people moved to Israel after the Soviet Union broke apart. The numbers have certainly gone down. But the whole population of Latvia is going down.  

\textbf{Was it illegal to emigrate before independence?} 

\textbf{Kalevs Krelins:} No, it was legal, but the government did not appreciate it. People applied and they got a permission, but only a few. In the begining of the 70s, many people got the permission. After the war in Afghanistan in the early 80s, until 1987, it was very hard to get a permission. After 1987, it got easier again. 

\textbf{Is there a discussion in the Jewish community about moving to Israel?} 

\textbf{Kalevs Krelins:} I think it's alredy settled. Some people come to me and ask why I don't move to Israel. I think if people are here, they have a reason to be here. Young people go to Israel to learn, to go to school or college. There are many people who lived in Israel, experienced Israel, and then came back. They decide to settle here. More or less, everyone came to his or her own conclusion with this question until now. The same is happening in Germany.  

\textbf{Do you see any future for the Jewish community here in Riga?} 

\textbf{Kalevs Krelins:} When someone from the Jewish community asks such a question, I have to say it’s up to you. It’s like a free market. If there is a demand, we will supply the demand. Sometimes there is a rabbi who wants to have a synagogue, so he creates himself the crowd. I think this concept is wrong. If the people need it, if they show that they want to know, they want to learn, I want to teach, but the first step has to be done by the people. Otherwise it’s counterproductive.  