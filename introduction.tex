\section{Introduction}

In 1949, Theodor W. Adorno asked himself and humanity whether one could still write a poem after Auschwitz, thus stating that the delusional and destructive ideology of anti-Semitism that had led to mass extermination had transformed the European culture and society in a permanent manner: It had become morally impossible not to have Auschwitz in mind when thinking about the self-conception of a society or nation and about the way that individualsposition themselves towards it.A widely held belief, especially in Germany, where a reappraisal of the past has always been a central and contentious issue, is that anti-Semitism has been dis-credited and banned from democratic discourse - however, common experience shows that anti-Semitism persists or even increases, an observation that seems to unmask the much-talked of lessons of history as fruitless.From the writings of Adorno and others, and from the fact that the threat of anti-Semitism in Europe persist, one could pose the question of how it is possible to be an anti-Semite after Auschwitz - a question that was essential in the project that lead to this book, which assembles interviews with Jews and experts on Jewish history as well as professionals dealing with anti-Semitism in Central Europe - namely, Germany, Latvia and Poland - after 1945.
\subsection*{The Youth of Europe Against Anti-Semitism} 
We, the collectors of the interviews, are a group of 29 people: High school and university students, apprentices, language teachers and a historian from Aachen, Berlin, Nuremberg, Munich, Riga and Zamość. Peter Zinke, a historian based in Nuremberg, had the initial idea for the project. A few years ago, he had visited Riga and witnessed what he considered to be a fascist demonstration: The March of Veterans of the Latvian Legion, which takes place in Riga each 16th of March. \\
Back in Nuremberg, he became concerned with anti-Semitic views among his friends, whom he had thought to hold an antifascist and open-minded worldview. Alarmed by these experiences, Mr. Zinke developed the idea for a project that should investigate how anti-Semitism had continued to manifest itself since 1945. He began looking for companions. \\
Shortly before, Mr. Zinke had finished two oral history projects, gathering the life stories of Holocaust Survivors together with high school students and teachers from Sderot (Israel), Nuremberg and Zamość. He convinced some of the participants of these projects to sign up for the new one, specifically nine high school students from Nuremberg, who had finished school by that time, as well as Agnieszka Smalej, a high school teacher from Zamość. Together with Beata Chmura, the head teacher of her high school, Mrs. Smalej persuaded nine pupils to take part in the project. Mr. Zinke also wrote to several Latvian institutions about his project idea. This way, he got in touch with Karina Barkane, Executive Director of the Centre for Judaic Studies at University of Latvia. Mrs. Barkane called on university students from Riga to apply for the project. From among the applicants, she eventually selected eight.\\
Thus the group was complete, comprising in alphabetical order the following people: Aleksandra Adamska, Karīna Barkane, Lingita Lina Bopulu, Gabriel Czajka, Janis Dobkevičs, Janis Dreimanis, Magdalena Freckmann, Davids Gurevičs, Lea Himmel, Cathy Hu, Eliza Koprowska, Emilia Kościk, Kamil Kwarciany, Zuzanna Makiel, Edgaars Poga, Johannes Probst, Jonas Röder, Annika Schmidt, Rafael Schütz, Agnieszka Smalej, Anastasija Smirnova, Dagmara Sokołowska, Patrycja Szala, Vilmars Vincans, Myrjam Willberg, Michael Winter, Aleksandra Wodyk, and Peter Zinke. \\
Together, we applied to the European Union for a grant under Programme Erasmus+ Key Action Cooperation for innovation and the exchange of good practices Action Strategic Partnerships Action Type Strategic Partnerships for Youth/Transnational Youth Initiatives Grant Agreement Number 2016-1-DE04-KA205-013927. We were awarded a grant of up to 54,975\euro{} for a project duration of three years from June 2016 to June 2019. \\
Some participants only took part in the first project activities, but most stayed on until the official end of the project on June 7, 2019. We all learned a lot about Jewish life in Central Europe and about the histories of the towns and countries we come from. Apart from the 60 interviews that we conducted, we visited Jewish schools, synagogues, and cemeteries, and various museums and memorial sites, such as the Memorials of the concentration and death camps in Auschwitz, Bełżec and Majdanek. 
\subsection*{Subject of Research in our Project}
The leading question for our project was how anti-Semitism developed after the Second World War in each of the aforementioned countries. For this reason, most of the interviews were centred around the connection between history, politics and anti-Semitism: In what ways is anti-Semitism connected to feelings of national collective guilt and responsibility with regards to the history of the Holocaust, but also to the feeling of being the victim in this historical process? How do the specific roles of the three countries in the Second World War as well as their political development after the War influence the forms that anti-Semitism takes? Can anti-Semitism be combated through raising awareness of history? \\
When we refer to the term anti-Semitism, we are aware of the fact that a broad variety of theoretical approaches towards this phenomenon exists, and that its definition is highly contested. The question of which definition one adheres has far-reaching implications when it comes to investigating the origins of anti-Semitism or the prospects to overcome it. We don't mean to provide a comprehensive overview on or even a positioning in this debate, but we'd like to at least state that we see anti-Semitism as a system of thinking following its own rationale and, in line with Haury (2002, cf. Beyer 2015: 576-582), as a mindset that boasts the following principles: personification of abstract global processes such as capitalism or modernity, Manichaeism - i.e., the dichotomous divide of the world into "good" and "evil", with "the Jews" functioning as a projection surface and representation of all evil - and the construction of homogeneous groups, with Jews being constructed either as "the other" or as a non-group undermining existing group distinctions. These principles operate both on a social or collective and on an individual, psychological level (Ibid.).\\
 As hinted at in our research questions, we were interested in comparing different countries in order to study the connection between the historical development of a country and the expressions of anti-Semitism that can be found there. We believe that the negotiation of a national self-image is at the core of this connection, and that each of the three countries boasts some specifics in the way its national self-images refer to the Holocaust. In the following section, we will, by no means in comprehensive manner, shortly outline these specifics: \par
In Latvia, being under the control of and suffering from a foreign power is an experience that essentially determines the national collective memory. According to the Preamble to the Constitution of the Republic of Latvia the state condemns both Nazi and Communist regimes. However, the Soviet occupation of 1940, that involved massive deportations of Latvians to Siberia, is often referred to as the major national grievance; against this backdrop, the Nazi invasion is 1941 is perceived as a “lesser evil” or even as a liberation. Consequently, Jewish suffering tends to be marginalised, and the issue of Latvian collaboration or bystander inaction tends to be downplayed. The development of a democratic political culture from 1990 onwards has always involved debates on the question of which historical narrative should be privileged over others, resulting in a reluctance or unwillingness to acknowledge the historical suffering of and the present-day discrimination against ethnic groups other than the ethnic Latvians (Misco 2015). In this constellation, little attention is being paid to anti-Semitism in public discourse. Findings from Europe-wide surveys show that both the Jewish and the general Latvian population don't perceive anti-Semitism as a major problem in their country: Out of 200 people of Jewish origin that participated in a 2018 study by the EU Agency for Fundamental Rights (FRA)\footnote{The 2018 online survey "Experiences and perceptions of anti-Semitism - second survey on discrimination and hate crime against Jews in the EU" conducted by the European Union Agency for Fundamental Human Rights (FRA) seeks to provide EU-wide data on present levels of anti-Semitism in order to asses to which extent EU member states are fulfilling their obligation to combat anti-Semitism. Therefore, it "analyses data from the responses of 16,395 self-identified Jewish people (aged 16 or over) in 12 EU Member States – Austria, Belgium, Denmark, France, Germany, Hungary, Italy, the Netherlands, Poland, Spain, Sweden and the United Kingdom. These Member States are home to over 96\% of the EU’s estimated Jewish population" (FRA 2018: 7). As response rates in Latvia were low, recruitment methodology and data collection were adapted in order to reach more respondents. This limits the possibility to compare the results from Latvia with those from the other countries. The size of the Latvian sample was n=200, in Germany n=1,233 and in Poland n=422).}, only 12\% considered anti-Semitism to be very or fairly big problem in Latvia, and 77\% thought that it had stayed the same in the last five years before the survey. 6\% had experienced some form of anti-Semitic harassment in that time, and 8\% reported that this had happened to a family member or close friend (FRA 2018: 79). As for the general Latvian society, the Special Eurobarometer 484 that was carried out in December 2018 and investigates research questions similar to those of the FRA study\footnote{The special Eurobarometer 484, a survey which was carried out in December 2018 in 28 member countries of the European Union based on a request by the European Commission, is comprised of the following research interests: (1) To what extent do Europeans think of anti-Semitism as a problem in their country, and how do they assess its recent development? (2) What are the levels of knowledge of and education about anti-Semitism? This also relates to the awareness of means to combat anti-Semitism, and to adequate Holocaust education. (3) How do "conflicts in the Middle East" and the shift of focus influence the way European Jews are perceived in the EU? Methodology: 27,643 people were surveyed, about 1,000 in each country (Germany: n=1,526, Poland: n=1,011, Latvia: n=1,002). A multi-stage random sample was drawn based on regional administrative units. All interviews were conducted face-to-face in the participant’s home.} finds that 14\% of the study participants think of anti-Semitism as a very or fairly big problem, and 55\% percent are in the sentiment that it stayed the same over the past five years. 64\% percent of the respondents think that people in Latvia are not well informed about the history, customs and practices of Latvian Jews, while 30\% think that people are well informed and 6\% say that they don't know (European Commission 2019). While these data, giving an indication of the perception of anti-Semitism in Latvian society, boast values much lower than in Poland and Germany, the Anti-Defamation League’s Global 100 Survey on anti-Semitism of 2015\footnote{Note that there are some methodological difficulties with this survey. For the purpose of this introduction, it is especially problematic that the survey only asks whether respondents think that a certain statement is "probably true", not giving them the opportunity to answer in a more nuanced way. Nevertheless, the size of the randomly drawn samples (n=500) makes it possible to at least mark some general tendencies within the population.} finds that agreement with anti-Semitic statements in Latvia is in fact stronger than in Germany, and while anti-Semitic conspiracy thinking is stronger in Poland than in Latvia\footnote{The survey includes six statements that can be seen as representations of anti-Semitism as conspiracy thinking. The statements Jews have too much power in the business world (Latvia: 51\%, Poland: 52\%, Germany: 28\%) and Jews have too much power in international financial markets (Latvia: 47\%, Poland: 51\%, Germany 29\%) are those of which the largest share of the study’s participants thinks that they are "probably true", while the sentence Jews are responsible for most of the world's wars meets least approval (Latvia: 12\%, Poland: 14\%, Germany: 9\%)}, some of the statements that represent anti-Semitism as an inter-group-conflict are met with about the same agreement in Latvia as in Poland\footnote{\textit{Jews are more loyal to Israel than to [Germany/Poland/Latvia]}: 49\% approval in Germany, 50\% in Poland, 56\% in Latvia; Jews think they are better than other people – Germany 16\%, Poland 30\%, Latvia 39\%; People hate Jews because of the way Jews behave – Germany 30\%, Poland 34\%, Latvia 21\%). As for anti-Semitism related to history politics, 51\% of Germans, 60\% of Poles and 61\% of Latvians participating in the survey thought that the statement Jews still talk too much about what happened to them in the Holocaust was "probably true”.}  These data certainly must be interpreted with caution; in any case they seem to imply that the significantly lower level of problematization of anti-Semitism in Latvia does not directly correspond to lower levels of anti-Semitic thinking in the Latvian society.  \\
The European Union against Racism and Intolerance (ECRI) published its fifth report on Latvia, that discusses recent manifestations of anti-Semitism in Latvia, both in public life and in state practice, on March 5th, 2019. It states that the Jewish community reported on five cases of vandalism and desecration at the Jewish cemetery in Riga in 2016, and Latvian public media reported on four cases of vandalism at the cemetery in Rezekne in 2017 (ECRI 2019: 19). The report also touches upon annual commemoration ceremonies on 16 March for soldiers who fought in the Latvian Legion of the Waffen SS. The government states that it is unable to prohibit these manifestations as the courts have overturned the ban of the event in previous years. Nonetheless, ECRI expresses concern "about the fact that Members of Parliament belonging to the National Alliance Party, which is part of the governing coalition, have been repeatedly observed attending these commemoration ceremonies" (Ibid.: 15-16) and urges the government to officially condemn the event. ECRI also reiterates its recommendation to the authorities to "make provision for the restitution of the religious and communal property of the Jewish community and dispel any antisemitic sentiment that may stem from such action" (Ibid.: 27), as there are 265 disputed items that were owned by the Jewish community before World War II and the status of which remains unresolved. \\
The Supreme Court of the Republic of Latvia reports on 10 incidents of hate speech against Jews that have been prosecuted in the period from October 2012 until March 2018. \par
In Germany, more than 70 years after the Holocaust, anti-Semitism remains an everyday phenomenon. In the process of dealing with the German past, open expressions of anti-Semitism have become tabooed in mainstream public discourse. However, the ideology continues to fulfil the function of a socially and psychologically relieving interpretation of the world, and it gets reinforced by attempts to reject the collective guilt arising out of history. The tabooing has led to a transformation of anti-Semitism, namely through a shift towards a discourse on Israel loaded with anti-Semitic contents that meets broad acceptance in Germany. At the same time, open expressions of anti-Semitism with "traditional" contents are being condemned, which allows to depict anti-Semitism as a marginal and extremist political phenomenon (Busch et al. 2015: 1-3). However, recently, increases in anti-Semitic "everyday" harassment and in anti-Semitic acts of violence can be observed: The police crime statistics, which are published annually and report on all offences with a clearly anti-Semitic background that legal procedures have been initiated against, list 1,799 anti-Semitic crimes (such as harassment and vandalism) and 69 anti-Semitic acts of violence. Both numbers exceed the numbers of the last ten years. The large majority of crimes is committed with a right-wing ideological background, however, the numbers of offences based on a left-wing, religious or "foreign" political background have increased just as well. \\
In the FRA study, 85\% of the respondents from Germany thought of anti-Semitism as a very or fairly big problem. Anti-Semitism on the internet, on the streets or in public places and in the media were assessed to be the most problematic manifestations of anti-Semitism. The study shows that manifestations of anti-Semitism can severely affect the feeling of security of Jewish people: 29\% (in Poland, by comparison, 32\%) of respondents witnessed other Jewish people being verbally or physically attacked in the last twelve months before the survey. 41\% (59\% in Poland) were worried about being harassed or insulted, 25\% (47\%) about being physically attacked. 30\% (36\%) reported the frequent or permanent avoidance of wearing, carrying or displaying in public things that could identify them as Jewish, with security fears being the most frequently reported reason for this avoidance. 74\% (91\%) thought that the government was not combating anti-Semitism effectively.\\  
64\% of the respondents of the Special Eurobarometer 484 thought of anti-Semitism as a very or fairly big problem, 61\% thought that it has increased during the past five years. 74\% percent of the respondents think that people in Germany are not well informed about the history, customs and practices of German Jews, while 22\% think that people are well informed and 4\% say that they don't know. \par
As a means of grappling with its past and specifically with the Nazi and Soviet occupations, attempts have been made in Polish collective consciousness to restore the national configuration of the Interwar Period that signifies stability and autonomy, a process that involved the revival of social institutions such as the Catholic Church and the family. A part of this restauration process was the tendency to avoid the analysis of the younger past, a tendency that sometimes results in a rejection of any responsibility for the wrongs that occurred during the Holocaust (Grudzinska-Gross 2014: 664-666). This avoidance discourse, as Katrin Stoll, one of the historians that we interviewed, phrases it, has become a breeding ground for anti-Semitism. For sections of the Polish political spectrum, it has become a core ideology that tends to intertwine with conspiracy thinking and other ideologies such as an anti-EU or anti-cosmopolitan resentment (Zuk 2017: 84-85). \\
Like in Germany, 85\% of the respondents of the FRA study thought that anti-Semitism as a very or fairly big problem in Poland. The manifestations of anti-Semitism that most respondents thought of as a problem were anti-Semitism on the internet, in the media and in political life. Respondents were shown eight selected possibly anti-Semitic statements\footnote{Respondents were also asked whether they would consider a person voicing one of these statements to be anti-Semitic. The answers given are not itemised by country in the report. For each of the statements listed here, more than 85\% of all respondents said that they probably or definitely would.} and were asked whether they had heard or seen these being made by non-Jewish people. Out of these, most Polish respondents were confronted with the statement Jews have too much power in Poland (70\% - in Germany, by comparison, 42\%), Jews exploit Holocaust victimhood for their own purposes (67\%, Germany 45\%) and Israelis behave "like Nazis" towards the Palestinians (63\% in Poland and Germany). 41\% of the Polish respondents of the Special Eurobarometer 484 thought of anti-Semitism as a very or fairly big problem, 18\% thought that it has increased during the past five years, while another 18\% thought it had decreased, 23\% thought it had stayed the same and 41\% said that they don’t know. 51\% percent of the respondents think that people in Poland are not well informed about the history, customs and practices of Polish Jews, while 39\% think that people are well informed and 10\% say that they don't know. \\ According to the Hate Crime Reporting by the OSCE Office for Democratic Institutions and Human Rights (ODIHR), the Polish police authorities reported on 78 anti-Semitic hate crimes (including physical attacks, vandalism and verbal harassment) that police investigations had been initiated on in 2017. This number was lower than in 2016 (103 crimes reported on), but significantly higher than in previous years. 
\subsection*{Method and Selection of our Interviews}
We made no attempt to select a representative sample of interviewees. Instead, we tried to talk with people of as diverse backgrounds and perspectives as possible. We spoke with Jews about their personal experiences with anti-Semitism, both youth and nonagenarian Holocaust Survivors, laypeople as well as clerics. We interviewed scientists from a variety of disciplines, historians, sociologists, psychologists, and philosophers. We listened to a police officer combating politically motivated crime, an educator dispelling stereotypes about Jews already held by small children, a volunteer preserving the Jewish heritage of his town that is no longer home to any Jews, to a German-Israeli restaurant owner, to a representative of the Human Rights Office in Nuremberg, and even to German witnesses of the Holocaust holding anti-Semitic vies, we conversed with priests, politicians, and publicists.The places where we met our interview partners were Auschwitz Memorial and Museum, Fürth, Forchheim, Ingolstadt, Kraków, Lublin, Nuremberg, Riga, Tel Aviv, Trier, Warsaw, and Zamość. Unless stated differently, all interviews were conducted face-to-face by a part of our group (minimum two people). \\
As diverse as their backgrounds and professions of our interviewees were their conceptions of anti-Semitism. In line with what the interviewees considered as anti-Semitism, they regarded different approaches to be effective in the combat of anti-Semitism: Some thought that you need to convince people that anti-Semitism is a false projection and inner necessity of modern capitalist society, and that only critical thinking could lift these projections. Others thought that anti-Semitic stereotypes would primarily result from a lack of education and thus disappear if only everybody got to know Jewish people and see that their personalities are as individual as everyone else's. Some had completely resigned themselves. Many were engaged in different activities against anti-Semitism. They directed their efforts at children and youth, for example, or tried to gain political influence, yet others wanted to reach people of all walks of life. Some were only concerned about violence directed against Jews, while others were also worried about all discursive expressions of anti-Semitism, therefore reporting any anti-Semitic post on social media to the platform operators. 
\subsection*{On this Publication}
With publishing some of the interviews we conducted, we intend to present a variety of concepts and views on anti-Semitism. This juxtaposition of different perspectives is not complete in any sense. As the interviews don’t directly refer to each other, it is not necessary to read them in any specific order.\\ 
The interviews should be seen as individual narrations that are not representative for any roles, groups or attributes the interviewees are associated with (e.g. nationality, religion, profession, biographical aspects), even if their statements are naturally influenced by these affiliations. \\
The interviews are truncated and grammatically aligned with more common written language. We tried to edit the texts as little as possible and noted all major amendments that were necessary. \\
Even though we don’t raise a scientific claim as we didn’t follow a specific method when collecting and evaluating the data, the publication can be of scientific use, e.g. for systematisation, further theory formation or exemplary illustration of sociological and historiographical concepts related to anti-Semitism. \\
In any case, we hope that the interviews facilitate it for our readers to enhance their understanding of anti-Semitism in its connection with Central European societies and that it will encourage reflection processes, just as it did for us. 
\subsection*{Acknowledgements}
We’d like to thank all interviewees for sharing their life stories and expertise with us. We were amazed by their openness and by the amount of time that they devoted to answer our question. Also, we'd like to express our gratitude to the Erasmus+ fund making this project possible and to all people who have, via their tax payments, financed this project.
