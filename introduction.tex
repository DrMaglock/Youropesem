\section{Introduction}
\vspace*{3em}
In 1949, Theodor W. Adorno asked himself and humanity whether one could still write a poem after Auschwitz, thus stating that the delusional and destructive ideology of anti-Semitism that had led to mass extermination had transformed the European culture and society in a permanent manner: It had become morally impossible not to think of Auschwitz when thinking about the self-conception of a society or nation and about the way that individuals position themselves towards it.\\
A widely held belief, especially in Germany, where a reappraisal of the past has always been a central and contentious issue, is that anti-Semitism has been discredited and banned from democratic discourse - however, common experience shows that anti-Semitism persists or gets even stronger, an observation that seems to unmask the much talked about lessons of history as fruitless. The quote that we chose for the title of our publication, based on an interview with Kalevs Krelins, rabbi in the Peitav Shul synagogue in Riga, shows that Jews today still perceive anti-Semitism as a threat to their own safety. While many groups face hatred, exclusion, or discrimination in each country and might at times be under attack more strongly than the Jewish minority - such as refugees, Muslims, the Russian minority in Latvia, Ukrainians in Poland - anti-Semitism persists as an ideology deeply rooted in European societies that can be drawn upon at any time.\\
Considering the writings of Adorno and others as well as the fact that the threat of anti-Semitism in Europe persists, one could pose the question how it is possible to be an anti-Semite after Auschwitz - a question that was essential in the project that lead to this publication. This book assembles interviews with Jews and experts on Jewish history as well as professionals dealing with anti-Semitism in Central Europe - namely, Germany, Latvia and Poland - after 1945.
\subsection*{The Youth of Europe against Anti-Semitism} 
We, the collectors of the interviews, are a group of 29 people: high school and university students, apprentices, language \\
teachers, and a historian from Aachen, Berlin, Fürth, Nuremberg, Munich, Riga and Zamość. Peter Zinke, a historian based in Nuremberg, had the initial idea for the project. A few years ago, he had visited Riga and witnessed what he considered to be a fascist demonstration: the March of Veterans of the Latvian Legion, which takes place in Riga each 16th of March. \\
Back in Nuremberg, he became concerned with anti-Semitic views among some of his friends, whom he had thought to hold an antifascist and open-minded worldview. Alarmed by these experiences, Mr. Zinke developed the idea for a project that should investigate how anti-Semitism had continued to manifest itself since 1945. He began looking for companions. \\
Shortly before, Mr. Zinke had finished two oral history projects, gathering the life stories of Holocaust survivors together with high school students and teachers from Sderot (Israel), Nuremberg and Zamość. He convinced some of the participants of these projects to sign up for the new one. These were nine high school students from Nuremberg, who had finished school by that time, and Agnieszka Smalej, a high school teacher from Zamość. Together with Beata Chmura, the head teacher of her high school, Mrs. Smalej persuaded nine pupils to take part in the project. Mr. Zinke also wrote to several Latvian institutions about his project idea. This way, he got in touch with Karīna Barkane, Executive Director of the Centre for Judaic Studies at the University of Latvia. Mrs. Barkane called on university students from Riga to apply for the project. From among the applicants, she eventually selected eight.\\
Thus, the group was complete, comprising in alphabetical order the following people: Aleksandra Adamska, Karīna Barkane, Lingita Lina Bopulu, Gabriel Czajka, Jānis Dobkevičs, Jānis Dreimanis, Magdalena Freckmann, Dāvids Gurevičs, Lea Himmel, Cathy Hu, Eliza Koprowska, Emilia Kościk, Kamil Kwarciany, Zuzanna Makiel, Edgars Poga, Johannes Probst, Jonas Röder, Annika Schmidt, Rafael Schütz, Agnieszka Smalej, Anastasija Smirnova, Dagmara Sokołowska, Patrycja Szala, Vilmārs Vincāns, Myrjam Willberg, Michael Winter, Aleksan-dra Wodyk, and Peter Zinke. \\
Together, we applied to the European Union for a grant under the Programme Erasmus+ - Key Action Cooperation for innovation and the exchange of good practices - Action Strategic Partnerships - Action Type Strategic Partnerships for Youth/Transnational Youth Initiatives.\footnote{We were awarded a grant of up to 54,975\euro{} for a project duration of three years from June 2016 to June 2019 under Grant Agreement Number 2016-1-DE04-KA205-013927.} \\
Some participants only took part in the first project activities, but most stayed on until the official end of the project in June 2019. We all learned a lot about Jewish life in Central Europe and about the histories of the towns and countries we come from. Apart from the 60 interviews that we conducted, we visited Jewish schools, synagogues, and cemeteries, as well as various museums and memorial sites, such as the Memorials of the concentration and death camps in Auschwitz, Bełżec and Majdanek. 
\subsection*{Subject of Research in our Project}
The main question of our project was how anti-Semitism developed after the Second World War in each of the aforementioned countries. For this reason, most of the interviews were centred around the connection between history, politics, and anti-Semitism: In what ways is anti-Semitism connected to feelings of national collective guilt and responsibility with regards to the history of the Holocaust, but also to the feeling of being the victim in this historical process? How do the specific roles of the three countries in the Second World War as well as their political development after the war influence the forms that anti-Semitism takes? Can anti-Semitism be combated through raising awareness of history? \\
When we refer to the term anti-Semitism, we are aware of the fact that a broad variety of theoretical approaches towards this phenomenon exists, and that its definition is highly contested. The question of which definition one adheres to has far-reaching implications when it comes to investigating the origins of anti-Semitism or the prospects to overcome it. We don't mean to provide a comprehensive overview on or even a positioning in this debate. Still, we'd like to state that we see anti-Semitism as a system of thinking following its own rationale and, in line with Haury (2002, cf. Beyer 2015: 576-582), as a mindset that boasts the following principles: personification of abstract global processes such as capitalism or modernity, Manichaeism - i.e., the dichotomous division of the world into ``good'' and ``evil'', with ``the Jews'' functioning as a projection surface and representation of all evil - and the construction of homogeneous groups, with Jews being constructed either as ``the other'' or as a non-group undermining existing group distinctions. These principles operate both on a social or collective and on an individual, psychological level (Ibid.).\\
As hinted at in our research questions, we were interested in comparing different countries in order to study the connection between the historical development of a country and the expressions of anti-Semitism that can be found there. We believe that the negotiation of a national self-image is at the core of this connection and that each of the three countries boasts some specifics in the way its national self-images refer to the Holocaust. In the following section, we will briefly outline these specifics, by no means in a comprehensive manner. \par
In Latvia, being under the control of a foreign power is an experience that essentially determines the national collective memory, the most recent occupations being the Nazi Occupation from 1941 to 1944 and the Soviet Occupation in 1941 and from 1944 to 1991. According to the Preamble to the Constitution of the Republic of Latvia, the state condemns both the Nazi and the Communist regime. However, the Soviet occupation of 1940, that involved massive deportations of Latvians to Siberia, is often referred to as the major national grievance; against this backdrop, the Nazi invasion in 1941 is perceived as a “lesser evil” or even as a liberation. Consequently, Jewish suffering tends to be marginalised and the issue of Latvian collaboration or bystander inaction tends to be downplayed. The development of a democratic political culture from 1990 onwards has always involved debates on the question of which historical narrative should be privileged over others, resulting in a reluctance or unwillingness to acknowledge the historical suffering of and the present-day discrimination against ethnic groups other than ethnic Latvians (Misco 2015). \\
Political and public trends in Latvia substantially depend on the problem of the perception of the events of the Second World War. In June 2019, the liberal party ``Development/For!'' (``Attīstībai/Par!'') intended to submit a law to the Latvian parliament Saeima on compensation to the Latvian Jewish community for the property lost during the Soviet and the Nazi occupations with an amount of 40 million \euro{}.\footnote{\textit{Baltic News Network}, June 12th, 2019} This initiative invoked an ambivalent reaction of the society and caused a new surge in anti-Semitism; particularly, the Jewish community was misrepresented with regard to its board’s connection with political and financial organisations.\footnote{\textit{Pietiek}, June 16th, 2019} As a result of pressure from society, the party was forced to retract its proposal.\footnote{\textit{Baltic News Network}, June 21st, 2019}\\  
Analogously, the topic of the collaboration of Latvians during the Nazi occupation and complicity of certain personalities in the Holocaust is viewed sorely in Latvia. The role of the prominent war pilot Herberts Cukurs, who was killed by the Israeli intelligence agency Mossad in the mid-1960s, is still ambivalently evaluated. The Jewish community is blamed for intentional defamation of Cukurs and falsification of facts of his biography. Moreover, in February 2019, in spite of the objection of the Jewish community, the Prosecutor General's office decided to dismiss the criminal proceedings against Cukurs, since no evidence had been submitted or collected.\footnote{\textit{Public Broadcasting of Latvia}, February 14th, 2019} In addition, it is regularly claimed that during the Second World War, Cukurs saved several Jews.\footnote{Gabre (2019); Neiburgs (2019). See the interview with Ilya Lensky for further information.}\\
In the political constellation sketched above, little attention is being paid to anti-Semitism in public discourse. Findings from Europe-wide surveys show that both the Jewish and the general Latvian population don't perceive anti-Semitism as a major problem in their country: Out of 200 people of Jewish origin that participated in a 2018 study by the EU Agency for Fundamental Rights (FRA)\footnote{The 2018 online survey ``Experiences and perceptions of anti-Semitism - second survey on discrimination and hate crime against Jews in the EU'' conducted by the European Union Agency for Fundamental Human Rights (FRA) seeks to provide EU-wide data on present levels of anti-Semitism in order to asses to which extent EU member states are fulfilling their obligation to combat anti-Semitism. Therefore, it ``analyses data from the responses of 16,395 self-identified Jewish people (aged 16 or over) in 12 EU Member States – Austria, Belgium, Denmark, France, Germany, Hungary, Italy, the Netherlands, Poland, Spain, Sweden and the United Kingdom. These Member States are home to over 96\% of the EU’s estimated Jewish population'' (FRA 2018: 7). As response rates in Latvia were low, recruitment methodology and data collection were adapted in order to reach more respondents. This limits the possibility to compare the results from Latvia with those from the other countries. The size of the Latvian sample was n=200, in Germany n=1,233, and in Poland n=422).}, only 12\% considered anti-Semitism to be a very or fairly big problem in Latvia, and 77\% thought that it had stayed the same in the last five years before the survey. 6\% had experienced some form of anti-Semitic harassment in that time, and 8\% reported that this had happened to a family member or close friend (FRA 2018: 79). As for the general Latvian society, the Special Eurobarometer 484 that was carried out in December 2018 and investigates research questions similar to those of the FRA study\footnote{The special Eurobarometer 484, a survey which was carried out in December 2018 in 28 member countries of the European Union based on a request by the European Commission, covers the following research questions: (1) To what extent do Europeans consider anti-Semitism to be a problem in their country and how do they assess its recent development? (2) What are the levels of knowledge and education about anti-Semitism? This also relates to the awareness of means to combat anti-Semitism and to adequate Holocaust education. (3) How do ``conflicts in the Middle East'' and the shift of focus influence the way European Jews are perceived in the EU? Methodology: 27,643 people were surveyed, about 1,000 in each country (Germany: n=1,526, Poland: n=1,011, Latvia: n=1,002). A multi-stage random sample was drawn based on regional administrative units. All interviews were conducted face-to-face in the participant’s home.} finds that 14\% of the study participants thought of anti-Semitism as a very or fairly big problem and 55\ percent felt that it had stayed the same over the past five years. 64\ percent of the respondents thought that people in Latvia are not well informed about the history, customs and practices of Latvian Jews, while 30\ percent think that people are well informed and 6\% say that they don't know (European Commission 2019). While these data, giving an indication of the perception of anti-Semitism in Latvian society, boast values much lower than in Poland and Germany, the Anti-Defamation League’s Global 100 Survey on anti-Semitism of 2015\footnote{Note that there are some methodological difficulties with this survey. For the purpose of this introduction, it is especially problematic that the survey only asks whether respondents think that a certain statement is ``probably true'', not giving them the opportunity to answer in a more nuanced way. Nevertheless, the size of the randomly drawn samples (n=500) makes it possible to at least track some general tendencies within the population.} finds that agreement with anti-Semitic statements in Latvia is in fact stronger than in Germany. While anti-Semitic conspiracy thinking is stronger in Poland than in Latvia\footnote{The survey includes six statements that can be seen as representations of anti-Semitism as conspiracy thinking. The statements ``Jews have too much power in the business world'' (Latvia: 51\%, Poland: 52\%, Germany: 28\%) and ``Jews have too much power in international financial markets'' (Latvia: 47\%, Poland: 51\%, Germany 29\%) are those of which the largest share of the study’s participants thinks that they are ``probably true'', while the sentence ``Jews are responsible for most of the world's wars'' is met by the least approval (Latvia: 12\%, Poland: 14\%, Germany: 9\%)}. Some of the statements that represent anti-Semitism as an inter-group conflict are met with about the same agreement in Latvia as in Poland.\footnote{\textit{``Jews are more loyal to Israel than to [Germany/Poland/Latvia]''}: 49\% approval in Germany, 50\% in Poland, 56\% in Latvia; ``Jews think they are better than other people'' – Germany 16\%, Poland 30\%, Latvia 39\%; ``People hate Jews because of the way Jews behave'' – Germany 30\%, Poland 34\%, Latvia 21\%). As for anti-Semitism related to history politics, 51\% of Germans, 60\% of Poles, and 61\% of Latvians participating in the survey thought that the statement ``Jews still talk too much about what happened to them in the Holocaust'' was probably true.}  These data certainly must be interpreted with caution; in any case they seem to imply that the significantly lower level of problematisation of anti-Semitism in Latvia does not directly correspond to lower levels of anti-Semitic thinking in the Latvian society.  \\
The European Union against Racism and Intolerance (ECRI) published its fifth report on Latvia on March 5th, 2019. It discusses recent manifestations of anti-Semitism in Latvia, both in public life and in state practice. The report states that the Jewish community reported on five cases of vandalism and desecration at the Jewish cemetery in Riga in 2016 and that Latvian public media reported on four cases of vandalism at the cemetery in Rezekne in 2017 (ECRI 2019: 19).\\
The Supreme Court of the Republic of Latvia reports on 10 incidents of hate speech against Jews that have been prosecuted in the period from October 2012 until March 2018. \par
In Germany, more than 70 years after the Holocaust, anti-Semitism remains an everyday phenomenon. In the process of dealing with the German past, open expressions of anti-Semitism have become tabooed in mainstream public discourse. However, the ideology continues to fulfil the function of a socially and psychologically relieving interpretation of the world. The tabooing has led to a transformation of anti-Semitism, namely through a shift towards a discourse on Israel loaded with anti-Semitic contents that finds broad acceptance in Germany. At the same time, open expressions of anti-Semitism with ``traditional'' contents are being condemned, which allows anti-Semitism to be depicted as a marginal and extremist political phenomenon (Busch et al. 2015: 1-3). Recently however, increases in ``everyday'' anti-Semitic harassment and in anti-Semitic acts of violence can be observed.\footnote{The police crime statistics, which are published annually and report on all offences with a clearly anti-Semitic background that legal procedures have been initiated against, list 1,799 anti-Semitic crimes (such as harassment and vandalism) and 69 anti-Semitic acts of violence. Both numbers exceed those of the last ten years.} The large majority of anti-Semitic incidents is committed with a right-wing ideological background, but the numbers of offences based on a left-wing, religious or ``foreign'' political background have increased just as well. \\
In the FRA study, 85\ percent of the respondents from Germany conceived of anti-Semitism as a very or fairly big problem. Manifestations of anti-Semitism on the Internet, on the streets, in public places, and in the media were assessed to be the most problematic. The study shows that manifestations of anti-Semitism can severely affect the feeling of security of Jewish people: 29\% (in Poland, by comparison, 32\ percent%tic post they saw on social media to the platform operators. 
\subsection*{On this Publication}
By publishing a selection of the interviews we conducted, we intend to present a variety of concepts and views on anti-Semitism. This juxtaposition of different perspectives is not complete in any sense. As the interviews don’t directly refer to each other, it is not necessary to read them in any specific order. We've sorted them chronologically. \\ 
The interviews should be seen as individual narrations that are not representative for any roles, groups, or attributes the interviewees are associated with (e.g. nationality, religion, profession, biographical aspects), even if their statements are naturally influenced by these affiliations. \\
The interviews are shortened and grammatically aligned with common written language. We tried to edit the texts as little as possible and note all major amendments that were necessary. Most of the interviews were conducted in English and we conversed in English within the project group. Nevertheless, some of the interviews were conducted in Polish, German, Latvian, or Russian as it was easiest for all parties involved. For this publication, we decided to publish the interviews in their original languages.  We hope that this way the book will be open for people from backgrounds as various as those of the members of our group, albeit only a few readers may be able to read all the interviews.\\
We don’t raise any scientific claim, as we didn’t follow a specific method when collecting and evaluating the data. Still, the publication can be of scientific use, e.g. for systematizing, elaborating theories, or illustrating sociological and historiographical concepts related to anti-Semitism. In any case, we hope that the interviews allow our readers to enhance their understanding of anti-Semitism in its connection with Central European societies and that it will encourage reflection, just as it did for us. \\
This book is available under a Creative Commons Attribution-ShareAlike 4.0 International Public License. 
\subsection*{Acknowledgements}
We’d like to thank all interviewees for sharing their life stories and expertise with us. We were amazed by their openness and by the amount of time they devoted to answering our questions. We're grateful to Berkan Berkil, Stefanie Dunker, Fabian Fichtner, Emils Metlans, Anja Schoeller, and Dr. Axel Töllner for accompanying us on our journeys and helping us with our work. Also, we'd like to express our gratitude for the Erasmus+ fund that made this project possible and to all people who have, via their tax payments, financed this project. 
\subsection*{Contact Details}
If you'd like to learn more about the project or wish to get access to the original transcripts or other material that we've collected during our journeys, feel free to write to \\ \textit{youthagainstantisemitismeurope\textcircled{a}gmail.com}. \\
We'll be delighted to hear from any person or project that benefits from our work. We're also interested in suggestions for collaboration or further processing of the material.
