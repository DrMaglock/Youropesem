Interview with Bartosz Gajdzik 

Bartosz Gajdzik is an educator at the “Grodzka Gate ‐ NN Theatre” Centre in Lublin since 2012. Through collecting archives of Jews from the Lublin area, composing exhibitions to commemorate particular Jews of Lublin, individuals and families killed in the Holocaust, educating, conducting workshops and publishing, he is participating in ‘repairing the world’, as in the Jewish concept Tikkun olam. He is responsible for organising workshops for secondary school students and also for teachers, often mixed groups of Poles and Israelis. He is focused on propagating Jewish culture among Polish students and also fighting stereotypes in Polish society. He studied philosophy. 

The interview took place at “Grodzka Gate ‐ NN Theatre” Centre on January 27th, 2018 

Bartosz Gajdzik: First of all, we still have a problem with anti-Semitism in Poland. I’d like to say a few words about anti-Semitic attacks against this institution. It was attacked a few times by anti-Semites. They printed some anti-Semitic posters about it, even though all fifty-five people who work here in the full-time jobs are not Jewish. We are in a municipal institution where Polish people, inhabitants of Lublin work to commemorate the Jewish community that lived here before the Second World War. It’s very meaningful that those attacks were prepared by anti-Semites, but against Poles, not against Jews. Our institution exists thanks to local people, I’m trying to show you the two sides of this situation. On one side local people, the local government decided that we cannot understand the history of Lublin without the Jews who lived here and that this is important. If we don’t want to be ignorant, we must remember about the Jews who lived here. Before the Second World War, one third of the population was Jewish, but for example at the beginning of 19th century, Jews made up even more than half of the population. For this reason, we cannot understand history of our town without the history of Jews. It really doesn’t matter that we aren’t Jewish. But for some groups, it’s the reason of being attacked. There were not only posters; the director of this institution, Tomasz Pietrasiewicz, was also attacked, it was a personal attack, someone broke the window of his kitchen in his private house. Those people used bricks with swastikas, it was a very strong symbol. It happened a few years ago. The police started investigation and it took a few months for them to find those people, and one of the men who was in this group was a full-time worker of the printing house of the State Museum at Majdanek. I want to show you this paradoxon: a man who works in a memorial site, in the State Museum at Majdanek, can be also an anti-Semite. Reality is very complex. On the one hand, we have inhabitants of Lublin who are trying to remember Jews in this town. On the other hand, they can be attacked.  
We co-operate with almost every school in Lublin, we conduct workshops. When I come to school as an educator, for example in the lesson of Polish language or history, students usually don’t know what will happen, they are not prepared, and at the beginning I introduce myself and then I’m trying to introduce the topic, ‘We’re talking about Jews’. And when I say this word in Polish, “Żydzi”, sometimes students start to laugh, as if this word was a bit shameful, or some kind of bad word. It shows the many stereotypes which students have about Jews: I just said that we were talking about Jews and they started laughing as it was some funny topic. We can just imagine what kind of stereotypes they have in their minds. 

Have you encountered any anti-Semitic behaviour among students of secondary schools, except for laughing at the word “Jew”? 

Bartosz Gajdzik: It’S not so often about their behaviour. Sometimes I ask, ‘Why are you laughing? What kind of associations do you have with this word?’ Then they explain that they start to laugh because sometimes people use the word ‘Jew’ to imply that someone is very rich and doesn’t want to share their money with others. Usually we talk about the reasons of this kind of behaviour, and we talk about stereotypes. And then I try to show them that, of course, we have stereotypes which can be negative and positive. For example, there is a stereotype about Polish readiness to help or Polish hospitality, or about Germans, that they are very good workers. Of course, it’s again a stereotype. And reality is always much more diverse and complex than stereotypes. 

Can you name any positive stereotype about Jews? 

Bartosz Gajdzik: It’s usually something between positive and negative, there’s this stereotype that they can make a good business. It’s something in between, of course, it’s good to be a good businessman, but on the other hand, there’s a negative side of this stereotype. 

One of the Jewish students in Kraków told us about so-called ‘positive’ anti-Semitism: Some Polish reckon Jews as people who are good lawyers or doctors, but there is a slight negative side in it. They admire them as good workers, good craftsmen, intelligent ones, but at the same time, there is some kind of envy. There is a paradox of ‘positive anti-Semitism’. 

Bartosz Gajdzik: Yes, that’s the paradox. Now, we are in the gallery which is dedicated to the Righteous Among the Nations, and I think this is a good occasion to talk about anti-Semitism in the countryside. We co-operate with the Yad Vashem Institute, the workers of this institute try to collect documents about non-Jews who helped Jews during the Second World War. And all testimonies which we recorded here are given to the Yad Vashem Institute. Sometimes we have some information that someone somewhere in the Lublin Region helped Jews during the Second World War, and we are trying to find those people, we knock on their door, but probably, even if they helped Jews during the war, they don’t want to talk about it. Because of their neighbours, unfortunately. For those people it’s kind of shameful: they helped the Jews during the Second World War, it means that they maybe got money because of that, and they are afraid that their neighbours will say so. Besides, we must remember that it was Nazi law that anyone who helped Jews during the Second World War was treated like a traitor. We don’t know exactly how many people, probably a thousand Polish people were murdered in Poland, because they helped the Jews. And sometimes the Nazis murdered not only the family who helped, but the whole village. It was the risk for the whole village. Maybe those people don’t want to talk about it also because it was the risk not only for them but also for their neighbours. We don’t know exactly. But, at the end, what was wrong about it? Seventy-five years after the Second World War it is still a problem to talk about this. We still have problems with history, with memories.  
Now, let me tell you about the reasons of anti-Semitism. I think the first reason is that in communities and societies there is always a kind of identity, a process of building the community identity for which we need to find an enemy. It was not necessary that those enemies were Jews. But some people can build their identities only if they know who the enemy is. I know my enemy, so I know who I am. Unfortunately, in my opinion, in these days they can be also refugees. The second perspective, in my opinion, is a conflict between Christianity and Judaism, in this case, religion is the reason. I think that this conflict was the most important reason in this part of Europe until the end of 19th century. After the War, the reasons of contemporary anti-Semitism in Poland, in my opinion, are Jewish properties. Many Poles don’t want to remember that Polish people after the War started to use Jewish properties and that before the Second World War, three point five million Jews lived in Poland. All of them had homes and private properties. And usually, during the Second World War and ion the aftermath, some building were destroyed, some not; usually Polish people started to use these properties. And it’s very difficult for Jews to take them back right now. It’s a good solution not to remember that Poles used Jewish properties and that’s the moment when we start to use all these anti-Semitic stereotypes: the Jews were bad people, they were like aliens, and still they want something from us, they want our money, they want our properties, they want our house. We don’t talk about it in public. 

 Do you think that the Jews who emigrated from Poland to Palestine or Israel claim to get back their properties?  

Bartosz Gajdzik: It depends. Usually not. It’s very difficult, it’s even impossible. After the War Poland, in the communistic state, the law didn’t allow to take back private properties, it was possible only for religious communities. For example, it was possible for the Jewish religious community or the Catholic religious community – take, for example, the building of Jewish school Jesziwa here in Lublin. After the War there was a medical university in this building. And in 2002 the Jewish religious community took it back. This is possible just for religious communities. It also works, for example, for Polish farmers. After the War, huge farms were divided into small pieces and it was also impossible to take it back for Polish people. So, this law was not only against Jews, it was sometimes also directed against Polish people who were rich.  
And recently, we are in strained situation. I mean this discussion about the relations between Poland and Israel and the law which is devoted to keeping the good name of Poland. It has different aspects. The official statement is that somewhere in the world, people use the phrase ‘Polish death camps’. Of course, it’s not true, it’s a huge mistake to use it. But in this new law there’s nothing about ‘Polish death camps’. This new law says, more or less, that everyone who will say that some complicity lies with Polish people regarding the Nazi crimes can go to prison for three years. But sometimes the complicity of Poles in some crimes, particular crimes, like Jedwabne for example, is clear - it happened. Israeli people and other people who Survived, they feel afraid if they have had some bad experience during the War because of the Polish people. This law is not only for the Polish but also for people from abroad. It’s a huge mistake, I think that we cannot change history and we cannot change history through a law. We must use educational tools, first of all. 

Do you think there’s a real danger for Polish Jews today because of that law? A few days ago, we made an appointment with a Jewish woman who lives in Kraków, and she was really scared. The day before, she went to Auschwitz, and she heard the speech of the Israeli ambassador, and she was afraid of what would happen to them right now.  

Bartosz Gajdzik: If they feel fear, that’s enough. It was not necessary to start this process. For example, yesterday I spent the half of the day here at my work to explain people from Israel what was going on here, because they had many questions: ‘Can we visit Poland right now?’ They feel really afraid; this law is not so clear; we don’t know exactly what it means. For example, during discussions in parliament, there was a question about the latest book of Jan Gross about Jedwabne. And there was a question: is this book against the law or not? Some vice minister answered that it depended if the author talked about facts or not, if he was a researcher and if he used the correct tools or not. Who will decide about this? No one knows.  
I also want to say that in my opinion, we cannot talk about anti-Semitism without education about different kinds of exclusion and stereotypes. Anti-Semitism is only one kind of exclusion. We must learn not only about the exclusion of Jews, but also about exclusions of different nations, of women, and about homophobia. In my opinion, there are the same reasons that form the basis of all these exclusions. People cannot accept that reality is very complex and diverse, and we cannot change these diversities, we must accept the diversity of reality.  

Usually, visiting groups from Israel have many guards with them. What do you think about it? Is it necessary? 

Bartosz Gajdzik: That’s a good question. Quite often, Polish students ask me why. It looks like they need security because they feel afraid of us. Is it true? Israel is a very untypical country because security is one of the most important matters in Israel, even inside the state. The level of security is much higher than here, in Poland, in Europe. The groups of Israeli students have security guards if they go somewhere as a full class, even inside Israel, when they travel to Galilee or to the Dead Sea from Jerusalem. They have these security measures because they live on the territories which are occupied. And they have them not only in Poland, but also in Germany, in France, in Israel, wherever. So, it’s not because they feel afraid of Poles, but because they feel afraid of terrorism, wherever they are. But of course, it can be misunderstood.  

Getting back to the past, I would like to ask about villagers who obviously had some anti-Semitic attitude because of Catholic priests’ teaching. How would you call a person who is anti-Semitic because of the Catholic Church and at the same time saves Jews’ lives? There were such cases. Would you call the person anti-Semitic or not? 

Bartosz Gajdzik: Yes, I would. Because of their thinking. Many people who were anti-Semites helped Jews during the Second World War, because they felt that it was something obligatory for Christians. Even if I have stereotypes about Jews, even if I hate them, I must do it. And it means that anti-Semitic thoinking and behaviour are not the same thing. It can cross, as in the case of Zofia Kossak-Szczucka. She was a writer, she wrote many anti-Semitic texts, but she helped Jews, many of them survived thanks to her.  

Can we compare it to today’s immigrant crisis? If a person is against keeping refugees in their country but is Catholic?  I can donate some money to support refugees in Syria for example, so I help them. But I don’t want them here.  

Bartosz Gajdzik: Yes. In my opinion, it’s also our Polish contemporary paradox. Again stereotypes of Poles, they consider themselves very helpful and hospitable. They also like to remember that Polish Righteous Among the Nations are the biggest group. Usually, when we talk about the Holocaust in Poland, the mainstream only claimes that Poles are the biggest group among them. It’s the most important thing. Except for that, and the fact that death camps were not Polish, it’s not necessary to know anything. A few years ago, there was a poll in high schools in Warsaw, and there was the question ‘What do you think about relationships between Poles and Jews during the Second World War?’ Eighty percent answered that the Polish people helped Jews often or very often. We must remember that the group of people who helped Jews was really small, probably it was less than one percent. We know about seven thousand Righteous Among the Nations from Poland. On the one hand, it’s twenty five percent of all the Righteous, but on the other hand, this percentage is highest in proportion because here in Poland lived the biggest Jewish community. For example, in Latvia, there are not so many Righteous, firstly because there were not so many Jews there. Around fifteen million Poles lived in Poland before the Second World War. When we compare fifteen million Poles to seven thousand Righteous, what is the percentage? But right now, students in schools, not only in Warsaw, think that during the Second World War Poles, as a whole nation, were very helpful to Jews. And the same group seventy-five percent of students - I don’t know if they want to help. They don’t want to have refugees here. On the one hand, they regard Poles as very helpful in the past, but on the other hand, nowadays it doesn’t work anymore. So that’s also the paradox. It’s a good example of using experiences from the history in contemporary challenges. It’s not only history: anti-Semitism, the Holocaust, Shoah.  

 Do you see any chance to reduce anti-Semitism in the world, and do you have any ideas? 

Bartosz Gajdzik: Of course, education and meetings can help, because many people have anti-Semitic stereotypes even if they haven’t met any Jew. I imagine this Catholic priest1. A few years ago, he told me that he was in hospital, he was standing in line, waiting to see a doctor. And as usually, people were talking to each other and someone said that we have a problem with doctors in this country, with the health service because there are always some Jews in the government and they do all the troubles to us. And he asked: ‘Jews? What do you have against the Jews? I’m a Jew. Did I do anything bad against you?’ This man was confused, because probably he hadn’t met any Jew before. I think that such meetings are the best way to recognise stereotypes and to break them. That’s also the reason why we organise meetings between Polish students and students from Israel. They can understand that in the modern world everyone uses Facebook, everyone listens to the same music, and people more or less are very similar. And it’s also good context to understand that many stereotypes about Jews are just not true.  

And what are the main interests among students when you organise workshops? Is it rather anti-Semitism or the Jewish culture?   

Bartosz Gajdzik: It’s the Jewish culture. There are of course some students who are anti-Semites and want to talk about anti-Semitism, but usually they just know nothing about Judaism, about the Jewish culture, so usually they have questions like “why those kipot on Jews’ heads, why the mezuzas on a front door?”. They need explanations, they have some knowledge, a bit true, but also some false statements or false opinions, and they didn’t have a chance to ask these questions before. Usually, we prepare students before a meeting with Israelis. Having the image of a Jew in black clothes in mind, in a black hat, with those funny elements on the both sides of heads, they ask me if the Jews look like other people, just like me and you.  

 How did your personal interest in this theme develop? 

Bartosz Gajdzik: I don’t know exactly, probably there are few reasons. The deepest is in my personality. But the first reason this priest, I mentioned this priest. He was my professor of philosophy, he talked a lot about Jewish philosophy, about the history of Jews, and I started being interested in Jewish philosophy from the beginning. I graduated from philosophy at the university and I wrote my thesis about the Jewish philosopher Emmanuel Levinas, he was a Jewish Frenchman who lived in 20th century. When I was a student ten years ago, I rented an apartment with my friend in a building more than one hundred years old. One day, I met some people who were just looking around there, as if they were searching for someone. It was the family of this Jewish boy, whom you saw on a photo in the exhibition, Henio Żytomirski, and he had lived in this very building. Of course, I didn’t know about this. I invited this Israeli family to my place. In this building there were four apartments, and they didn’t know exactly in which one they had lived, they just knew the address Szewska 6. And Neta Żytomirska, the lady who was a cousin of Henio, said that she was in this building ten years before, but she was afraid to come inside because of her experience, probably some memories from the past. But when she met me, she felt she could enter. And we became friends. Always when I’m in Israel, I visit this family. But then, ten years ago I just asked myself: What’s going on? I live in the building where Jewish people lived before, I don’t know anything about them, anything about Jews. I know there was a concentration camp at Majdanek, of course, I know that before the Second World War, Jewish people lived in Poland, but actually I was ignorant. I just felt that I should change it. And it was in 2008 when I came to this institution for the first time. I still was a student, I started co-operating as an educator. They trained me how to conduct workshops in mixed groups of Poles and Israelis. And then I started a part-time job, and since 2012 I work here a full-time. For me, it’s the mix of some reflections, some personal experience and history. I just felt that this is something very important. I don’t have any Jewish roots. 

What about Professor Waszkinel2? He is not a priest anymore, is he? 

Bartosz Gajdzik: Officially he is, but he’s out of the Church in Poland, he doesn’t work as a priest. His bishop decided they didn’t need him; it’s actually his problem. The Church in Poland and also in Israel allowed him to mov,e but they also didn’t need him, so he could do whatever he wanted. He doesn’t have a church to be a priest in Israel. It’s very personal for him, if he’s a priest or not, I know he had some internal dialogue between those two identities.  

Do you know perhaps how he is perceived at the Catholic University of Lublin? Especially with his view of John Paul II’s teaching, as he is the patron of the university.  

Bartosz Gajdzik:  I know that he felt bad at the university. It’s hard to discuss about feelings, he started feeling like an alien at the university among his colleagues. He’s a priest but he’s a Jew. They didn’t accept his identity and he decided to retire and they accepted it. In his opinion, they didn’t want him to stay longer at the university, even though he wasn’t forced to leave. But on the other hand, in Israel he is like an alien as well. There are not so many Catholic priests, or people who used to be priests in Israel. I met him in Israel in December, and he told me he decided to stay in Israel because there many people have some trouble with their identity. There are people from all over the world and in this context, in this homogenous society, he feels similar to the majority of Jewish survivors.  

 

 