 Interview with Małgorzata Zajda. 

Małgorzata Zajda (*1948) was born in Kraków, as her parents and grandparents. She is of Polish-Jewish descent. She studies Russian.  

From the beginning of Jewish Community Centre in Kraków (2008), she cooperates with it as a coordinator of the Senior Club. She organises plenty of activities and events for seniors such as Shabbat dinners, choir meetings, yoga, mind training sessions and many other Jewish celebrations and everyday activities for the elderly.  

The interview took place in the Jewish Community Centre in Kraków on January 26th, 2018. 

Małgorzata Zajda (*1948), is working at the Jewish Community Centre in Kraków. She is taking care of elderly people there. 
The interview took place in the Jewish Community Centre in Kraków on January 26th, 2018. We talked about a variety of subjects with Mrs. Zajda; for this publication, the transcript of the interview was shortened to the sections directly related to anti-Semitism and Jewish life. 

Małgorzata Zajda: My parents were Jewish descent. They introduced me into Jewish life as in the 50s, it was almost impossible to talk about Jewish culture. But since I was a child, I was aware that I am Jewish and I was really proud of it, because being a Jew after all those incidents during the War was something amazing. I felt like being better than other people.  

Have you ever been confronted with anti-Semitism? 

Małgorzata Zajda: I came from a really rich family; my father was a lawyer. All people think that Jewish people in Poland after the War were poor- but it is not true when it comes to me. My father told me a lot about Jews, he taught me a lot. He told me about my grandparents, about life before the War, but he never was willing to talk about it. My mother was trying to raise me in catholic life, she didn't know about her origin - probably her mum hid it from her. My father was organising meetings with influential people, every week they were meeting at our house and were resting, talking about art etc. - all of them were Jews. When I looked at them, I thought that every Jew is elegant and clean. When my father died, I got to know that Jews can be all kinds of people, not only intelligent ones, I saw that Jews are normal people and that they take up various jobs. 

Can you tell us about the experiences of your father during the War? 

Małgorzata Zajda: YMy father was in a camp in Janów. They were working by the train tracks. My father was athletic, he never had problems with sport and really liked it, so he could cope with this hard work. Somehow, he managed to escape from the camp and was wandering. He was in Czech Republic; he was a witness of Lidice's pacification in Czech. He was working in a bakery, selling bread. Later, he came to Kraków and hid. Unfortunately, he was caught, and was kept in Montelupich, and there he met a man who helped him to fake documents to prove he was of Aryan origin- that man was one of the Righteous Among the Nations of the World. With those Aryan documents he could lead a normal life in Poland. 
My father told me a story when he was coming back from the prison about how Jews were treated by other people, it was about 45'/46'. When the Jews were coming back home, people were throwing rotten tomatoes at them, they were very disrespectful to them, and it stayed in his psyche. He was scared to be among the people, he was scared to leave his home, he was happy that he changed his surname and could lead a normal life thanks to it. He said that being a Jew isn't as good as some people think. In my childhood, I never encountered any anti-Semitic acts, but I remember when we had a polish housekeeper - her name was Marysia -, once I came back from holidays and Marysia came to me and told me that people don't accept the fact that she is working for Jews. This was my first encounter with an anti-Semitic act- it hurt me so much, because our housekeeper was leading a really good life with our family. The second encounter was when a friend of mine moved to the US and worked as a nanny, she told me that everything would be okay for her if those people who she was working for weren't a Jews. They told her to eat kosher, and she didn't like that. During my young life, I experienced it many times that people painted a Star of David at our house door- but I never took it seriously. They were also painting it on my car, I always treated it as a stupidity of people. Nowadays we encounter less anti-Semitic acts, in Kraków, for example, anti-Semitism almost doesn't exist. In my opinion, nobody has got to love Jews, and everybody has the right to hate Jews and I’m not going to change it, it is their choice. I think we should respect each other even if we don't like somebody. I am not a person who has bad memories of Catholic people if it comes to anti-Semitism. Some of my friends, when they learned that I am Jewish, broke off contact with me. I think we should respect each other even if we don't like it, and I always repeat if when I talk about anti-Semitism. Nowadays, we experience situations in which Jews who were faced with anti-Semitism during the War think that people are acting anti-Semitic to them because they know that they are Jews, for example they say, “when somebody does not hold the doors for me, they don't want to because I am a Jew”. My job is to work it through, to tell that it is just a random. They think like that because it stayed in their psyche after the war.  

Are you able to say how many Jewish people live here in Kraków? 

Małgorzata Zajda:  There are about 100 people. There are much more, but they don't admit to be Jewish. 
Before, people were more against Jews, now they respect us and don't show it, even if they don't like us. If somebody will try to find anti-Semitism, they will find it everywhere, in every single situation. 

Do you think that there is a way to stop anti-Semitism in the future? 

Małgorzata Zajda:  I think that we will never stop it; in my opinion, it will always be there. I am happy that I am from the generation of people who weren’t faced with War so far. There might always be something that can change everything. Before, we never had security guards in the Jewish Community Centre, but now we have some special safety stuff. I can't say that anti-Semitism will not exist in the future; I think "anti-Semitism" is a wide word. If somebody wants to have anti-Semitism, they will get it.  
A few years ago, I had a situation where a Polish priest came to my house and showed me Jewish images and asked me if I am a Jew and left my house, he didn't respect my religion.  

Do you think there is a real danger for Jewish people in Poland?  

Małgorzata Zajda:  Before, I would say, not, but now I think that there is a little danger. There is no day without talking about Jewish issues.  

 People only see how bad are you and don't see how much good you do for people. Maybe instead of talking about anti-Semitism, we should talk about Jewish culture and learn people about it, so that it might change? 

Małgorzata Zajda:  Yes, exactly, it would help. We only talk about anti-Semitism and it might be a reason of this issue.  
There are lots of stereotypes, not only about Jews -"you are mean like a Jew", "you drink as much alcohol as Russian" etc.  

People say that Poland is a country full of anti-Semitism, what is in your opinion? 

Małgorzata Zajda:  It was an anti-Semitic country, but after the War, it was caused by the fact that Poles thought Jews will never come back and took their houses, their shops and their jobs after them. But when the War was over, they came back and wanted their properties back, and in my opinion, it is mainly caused by that. Anti-Semitism always was and always will be, now it does exist, but it is hidden, and we can see it just like that.  
I think that the Jewish Community Centres is a place which shows the Jewish culture and tradition and thanks to that, we reduce anti-Semitism a little.  

Do you think that the problem of the Jews as an affected group is exaggerated? 

Małgorzata Zajda:  No, I think it is not. Jews always were not respected, mainly because of money and success. 
Jewish people are scared of anti-Semitic acts. An example of it is when a Jewish group comes to Poland to visit the camps, they have special securities, and in my opinion, it’s because of that they are more scared by seeing it. It makes them think that they are in danger and it might make them feel scared. People in Israel think that Poland is an anti-Semitic country, and they are scared of it because of history.  

 

 