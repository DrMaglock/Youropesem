Dr Steffen Huber is a research assistant at the Department of Polish Philosophy at the Institute for Philosophy of Jagiellonian University in Krakow since 2005. He is also a member of the Policy Council of the Józef Tischner Institute, an institute that, according to its website, was founded by  pupils and friends of the philosopher and priest Józef Tischner (1931-2000) for the purpose of preserving and spreading knowledge about his works and of continuing research and development about the most important aspects of his philosophy.  
Dr Huber’s research interests are Polish philosophy of the Renaissance, social philosophy and translations of philosophical texts. We talked to Dr Huber and two of his students, Pawel Karpinski and Krysztof Turek, on the 26th of January in the Institute of Philosophy.  

In our project we examine the continuities and fractures of anti-Semitism since 1945. From the interviews we lead with scientists, we hope to get an overview over the country's history in its connection with the anti-Semitism found in this country. In your case, we would be particularly interested in a comparison of Germany and Poland, as you have lived in both countries. So, on the one hand, what you personally have experienced or heard second-hand of anti-Semitic incidents. And then, how anti-Semitism is rooted in the history of ideas in the two countries. 

 

Dr Steffen Huber: That's a big task. I have not dealt scientifically with the subject of anti-Semitism, for me it is a marginal phenomenon which one has to deal with every once in a while. It is also a very contested area, especially the question of what anti-Semitism is, we’ve also had this here at the university. Even today, it is often the case that one encounters anti-Semitism and then has to hear people say that tit would be none. Even if the classical motives occur, of conspiracy and "cultural destruction" and a low level of development of the Jews.  

If I compare Germany and Poland, I would say the major difference is that anti-Semitism in Germany basically has no basis of experience for most people in the twentieth century. People had virtually no experience with Jews who were culturally recognizable as such, and certainly not as groups. There was practically no Jewish cultural group in Germany at the beginning of the 20th century. These were basically phantasies of the extreme right. 

These two parameters are different in Poland at the moment. In other words, Jews in Poland have a much stronger recognizable Jewish identity, even as a group identity, which goes well beyond the religious and becomes tangible as a cultural linguistic, ethnic and economic identity. At first, there were considerably more areas of conflict. One should not only rate conflicts negatively, there are also productive differences in a society that stimulate development. There is a much larger base for this kind of conflict in Poland, also because Jews were not assimilated over the centuries for a much longer period than in Germany. Until the Holocaust, the Jews were clearly defined culturally, linguistically, ethnically and economically. This, in turn, is related to the fact that over the centuries the Jews have been very intensively involved in the political and social history of the country. A very important area that does not exist in Germany, is the history of self-government in Poland, this term still plays a major role in the state or in the universities today. What universities call university autonomy in Germany is called self-administration by universities here; according to this model, local social fragments that have formed, for example, out of religious or ethnic reasons, organise themselves from within. This is also the basis for this Polish history of republicanism. Poland is arguably the only country in Eastern Europe where there have been a large number of edicts comparable to the "de non tolerandis judaeis" written in Western Europe. It was a privilege for cities not to have to tolerate Jews within their walls if they did not want to. In Poland, in turn, there was the "de non tolerandis christianis" for Jewish-managed municipal units. That is, Jewish communities in Poland have a much longer history than parts of society that have developed in parallel, a phenomenon which is also often described in the philosophical and political literature. Western European observers in particular have been interested in how this works. In the course of this observation, critical voices multiply, which then say that they are parallel societies and that it also has dangerous proportions if the social groups develop so differently. For example, this development of Polish society had a major impact on Polish romanticism, which was related to national rebirth in the 19th century and had a great deal of Jewish input. With Adam Bernard Mickiewicze one hears for example a very positive relationship with Judaism when it comes to seeking metaphysical inspiration for how to maintain a Polish identity in this difficult situation without a state of its own. There was a very fruitful and friendly dialogue between, let's call it Christian Romanticism and Jewish traditions in Poland. The second point is that, of course, where there are Jews, there is also anti-Semitism. Anti-Semitism has always existed in Poland, but I would not say that Poland is particularly marked by anti-Semitism. Of course, there have always been tensions. There were pogroms, but these took place more in the east, especially in the Russian area of Poland. But to think of something similar to the Holocaust was something that never happened in Poland. Those who defined themselves as anti-Semites in the first half of the twentieth century have, for the most part, never gone as far as they did in Germany. On the contrary, there were very important people, including in this area, who started from racial nationalism and thought the Jews were a threat to Poland. These people then started to save Jews under the impression of the Holocaust. In that sense, I believe that the two countries cannot be compared directly at all. 

There is a sore spot in Poland: one has the feeling that the Germans have made the Holocaust and now they are moving around, accusing people of anti-Semitism. This is a situation many people do not know how to handle. 

 

Are there regions of which one could say that they were particularly anti-Semitic? Was there a difference between urban and rural areas? 

 

Dr Steffen Huber: There are certain regional hotspots, we can see this even today in the election results. In Łomża there were nationalist groups in the 30s that very much defined themselves through anti-Semitism, they were very active. [Pawel Karpiński enters] 

In this region was more activity by right-wing parties in the, 1920s and 30s than in other regions. In this region the Jedwabne murder took place. But I would not say that this is only a problem of this specific region. 

What I suppose is that in those regions where you find more anti-Semitism, you will also find more cultural contacts between the Christian majority and the Jewish minority. 

 

 You were talking about examples of anti-Semitism that you experience here. What would that be? Can you tell us about some of these experiences? 

 

Dr Steffen Huber: First I have to clarify how I understand anti-Semitism. I think that there is a quite useful definitions by Hannah Arendt, according to which anti-Semitism is based on images of the dangerous Jew, of the Jew who rules, of the Jew who is not rooted in the society and the culture or even in the ethical history of a nation, and of course, you can find this in Poland as you can find this in any other country. I am travelling around in Eastern Europe and I would not say that Poland is a hotspot of anti-Semitism. We have to remember that there was a very intense development of Jewish culture in Poland which later on moved to Israel. As we know that in the first year of the Knesset, the informal second language of the parliament was Polish because everyone from this part of Europe more or less used the “Lingua franca” of Polish, for example also those from the Ukrainian and Belarusian parts. I encountered anti-Semitism in Poland in two ways: First, there is a classical form of anti-Semitism that we find in the writings for example of Feliks Koneczny. He was a historian working here in the University of Krakow and he wrote some books on history from a very specific perspective. He presented a theory of civilizations saying that the highest form of civilization is the Latin, then we have the eastern one, which means Russia, and then we have the Jewish civilization. Of course, the Jewish civilization is the worst and the least and the most dangerous. He argues that Jews who are present in any way in a culture, in the economy or in the society, more or less destroy this higher civilization of the Latin type. He even came to the conclusion that the Holocaust, which he called a crime that was not acceptable in any civilization, even when directed against the “Jewish civilization”, is the outcome of the, he uses Nazi terminology here, “Verjudung” of the German civilization. In his opinion, what happened in the Holocaust was that the Jews made the Germans organize the Holocaust. This is an extreme example of the intrinsic stupidity of anti-Semitism. These texts are used by the right-wing political movement in Poland, they're quite popular. They are even used in some part of the academic discourse and of course, we have very hard conflict over that.  Another example of anti-Semitism from Krakow: I talked to a man whose family owns some flats in the former Jewish part of Krakow, Kazimierz, and when I asked him about the Jews who lived in that house, he stopped talking to me. This does not mean that this man is an anti-Semite. Perhaps this means that his family had some bad experiences with the Germans or with anyone else in this context. This also means that what seems to be anti-Semitism sometimes is just the inability to speak about something what has happened to your own family. Even if this was 70 years ago. And I think that we should also learn to understand what real anti-Semitism is and what a situation is that does not allow you to speak about some experiences that you’ve made. 

  

So, the idea would be that you cannot talk about this experience and then, you switch back to some usual stereotypes that you can easily state and then use it?  

 

Dr Steffen Huber: Yes, we should learn to differ between this and the classical anti-Semitism I saw in Poland. This intrinsic stupidity of anti-Semitism can be exported to other topics and to other ethnic or religious communities. This is what happens in Hungary now. We observed this in Poland as well. I think the technology of exporting the logic of anti-Semitism to other topics, such as refugees from Syria, was invented by the Hungarian government, and now it's being used by the Polish government. It’s a way of copying anti-Semitic patterns and using them in a new political discourse. 

 

But these new discourses, they are directed against different groups, like refugees, for example, but not against Jews, are they?   

 

Dr Steffen Huber: Just observe what the French right-wing Front National is doing. In a way, they argue that they want to protect Jews from being attacked by Muslims, but there is also the argument that the Jews are organizing Muslim immigration to destroy our culture, which is structurally quite close to the argument of Feliks Koneczny. I think we can clearly identify anti-Semitic patterns in the public discourse and they appeared to be directed against Jews sometimes, sometimes they are used to protect you from Muslims and sometimes they put together Jews and Muslims as this dangerous type of civilization, too strange and unacceptable for us. This is what happens in the official political discourse in Poland as well. 

 

Is this limited to the present government? Do politicians from other parties also speak in this manner? 

 

Dr Steffen Huber: I know some people who are really attached to the government of Prawo i Sprawiedliwość, de facto of Kaczyński. And I'm absolutely convinced they are no anti-Semites. They are deeply rooted in that kind of even pro Jewish romantic tradition in Poland and they would never accept any kind of anti-Semitism. And in the last weeks we also observed a process which I personally welcome, that government and the official Media tried to fight anti-Semitism and fight right-wing ideology - this has to be stated as well. On the other hand, I met people who are rather left-wing, or liberal, who are pro-European and so on, and they are anti-Semitic. This shows that there is no clear correlation. If there is a clear correlation of right-wing thought of the Conservative types with the nationalist or even racist type this is the extreme right. But this is not true for the main part of the conservatively thinking people in Poland.  

Pawel Karpiński: The government and the media are trying to fight anti-Semitism and xenophobic ideology, but I doubt that it is a clear intention. For example, our new Minister of Interior once said that he clearly does not tolerate any kind of racism or xenophobia, but when legal procedures run against nationalist parties, nothing happens. These cases are dismissed, and this is a clear sign that it’s being tolerated 

Dr Steffen Huber: Take as an example Mr. Winnicki, he is the leader of Ruch Narodowyis. He is talking about racial separation and he came to Dresden and shouted the Nazi slogan “Deutschland erwache”. If you shout “Deutschland erwache” at a meeting of Pegida in Dresden, this is Nazi ideology. After that, he declared he did not know what it means.  

 

This racial separation, would it also include Jews as a separate race? 

 

Dr Steffen Huber: Try to ask them.  

 

Is there this kind of anti-Semitic stereotype that Jews appear to be keeping the Polish nation on its knees after the War and the Soviet rule? 

 

Dr Steffen Huber: No. 

Krzystof Turek: This anti-Semitic motive doesn’t really exist because we never had bad experiences with Jews in that context in our history for many centuries. Jews were an essential part of our Society. They've been building our economy and our science. They were also Polish citizens. There is only one person which is Jewish and which is also seen as an enemy of Poland by the government, George Soros. He is seen as an enemy because he wants, with his “dirty billions”, to force people around the world to be become atheists and liberals, but this isn’t connected to his Jewishness. 

Pawel Karpiński: I suppose it's not something completely original and George Soros is just an ideal feature of this. I would say one of the pillars of anti-Semitism is the fear from being dominated from abroad - it might be by the Jews, it might be Berlin or Brussels, it might be Russia, and we are willing to do anything to be safe from that because of our history. We had grandfathers who fought wars for Polish independence, and even though they won, we still were not independent. It's so ironic.  

 

Did you ever encounter the stereotype of Judeo-Marxism, is it there? 

 

Dr Steffen Huber: Yes, it is used sometimes, but it's based on the romantic heritage of the 19th century in the fight against the Russians, the Prussians and the Austrians. This was very closely culturally connected to the Jewish experience, so this is a very difficult situation. 

Krzystof Turek: There is another thing connected to Jews. It has a little bit of a different character, because the communist state has openly dismissed Jews in the 60s. Even the communists in Poland, 20 years after World War II, excluded thousands of Jews and people with Jewish routes. They lost their posts in public institutions, they were all removed from the universities, they were removed from the party, even those who were loyal party members. 

  

You have mentioned several times that these people see liberalism and atheism as a big problem. What is the position of the Catholic Church in Poland regarding anti-Semitism? In Germany, the Nazi party was especially strong in Protestant regions, and rather less in Catholic regions. For example, in the region where we are from, it's in Bavaria, the City is Nuremberg, which is a largely protestant region were many prominent protestant reformers of the 19th century lived, and it was very anti-Semitic and very strongly supported the Nazis. Maybe you know Julius Streicher. He held many rallies in Nuremberg and was very successful. Apart from that, you can also look at the election results. Of course, in Poland the Catholic Church has a huge majority – so, what is their position? 

 

 

Dr Steffen Huber: I think there is not just one Catholic church in Poland and I'm wondering why they don't split - which I think would have happened if Poland had a different history. The Church has been the strongest and most endurable institution in Poland for 1000 years, and it is very well trained not to split in a situation of conflict. Nevertheless, the Church was much stronger than the state for centuries. So, you have a part of the Catholic Church which is clearly pro-European; some texts written Pope John Paul II some 30 years ago would be unbearably liberal for a big part of the Polish Society. If you don't tell them that it was written by the Pope, they will say this is liberal ideology from the West. On the other hand, you have a very long tradition of Catholic nationalism of Catholic anti-Semitism in Poland. 

 

 When we are talking about the last three years of economic development, does it influence anti-Semitism anyhow?  For example, could you say that when nations are rising socioeconomically, they develop stronger feelings of anti-Semitism, or that richer people have stronger anti-Semitic attitudes? Is there anything like that? 

 

Dr Steffen Huber: No. 

 

 Is there any connection between the socioeconomic status of a society and the level of anti-Semitism? 

 

Dr Steffen Huber: No. 

Pawel Karpiński: I suppose there might only be a connection between low status and identifying a threat: Some sentiment arising from seeing those who are well situated, who got money, and feeling it's somehow unfair.  

Dr Steffen Huber: That is true, but it's also a stereotype of anti-Semites and racists. And the people with conservative, nationalist or racist convictions in parts of the society were absolutely not bad situated or in a bad economic situation. The rural part of Poland has developed very strongly over the last 10 or 15 years. It is true it has not developed as fast in the 1990s and before the membership of Poland in the European Union, but over the last 15 years you could see a very, very strong development. I rather feel attached to those philosophical theories that say that anti-Semitism is not a political conviction or political instrument. Of course, it happens to be one, but its most substantial element is the need to feel better than someone else.  

Pawel Karpiński: When you are working hard and there is someone else who is working less hard in your opinion, then we might think it is unfair that they are better situated than we are, and this envy is the thing that might create such anti-Semitic views. I’m not saying that all the poor people are anti-Semitic.  

Dr Steffen Huber:  Are you referring to cities or to rural communities? 

Pawel Karpiński: I'm not sure. 

Dr Steffen Huber: In my opinion, in rural Poland, of course they don't like liberals, they don't like the European Union. They use the European Union, but they don't like it. They don't like the liberal elites in Poland. But there are no anti-Semites. The anti-Semites I met where rather well-situated in terms of economy. But that depends on your personal experience... 

 

Krzystof Turek: I want to give a very specific example of anti-Semitism in Krakow, but it is in no way related to the previous topic. There are two main soccer teams in Krakow, Wisła Kraków and KS Cracovia, and the fans don't like each other. Cracovia was founded by Jewish citizens of Krakow 113 years ago. Now for many fans of Wisła, anti-Semitism is a part of their identity as fans of the club. So, in Krakow, when you see anti-Semitic symbols drawn on a wall or a building, 95% were done by football fans because of that. 

 

 This vandalism you're talking about ,is it directed at synagogues or cemeteries which are not related to soccer? 

 

Krzystof Turek: I don't know. 

Dr Steffen Huber: They have to good security around the synagogues. If this happens it would be persecuted for sure. 

 

 This means that they need to have security?  

 

Dr Steffen Huber: Yes, but not for that reason. There is a lot of tourists in Krakow, big parts of them from Israel and the United States, and they're travelling around. For example, Jewish groups that are travelling to Auschwitz. They need some level of security. Yes, there are a lot of anti-Semitic symbols around the city but I would say not close to the synagogues because this is where the police are looking the closest. 

 

And you would say that these symbols are related to the fans of this soccer club? 

 

Krzystof Turek: In Krakow yes. 

 

Is the soccer club doing something against it? 

 

Krzystof Turek: I have no idea. But as far as I know, soccer clubs in Poland accept that there are also radical groups of fans because these are the people who fill the stadiums every week and also the people that make 20 meters flags and bring them to the matches. 

Dr Steffen Huber: If you ask them why they do not do anything about it they could possibly say that this is not Germany where you have a very long tradition of murders. This is what you sometimes here in Poland: There is no tradition of bloody murder in Poland, so you can allow much more aggression in a verbal way which will not become physical aggression. Of course, this is a very risky way to go, and it might at some point not work. But this is how many people look at this problem. 

 

What about the Jewish Communities? Do you personally know any of their members? How do they view the situation of anti-Semitism in Poland? Do you have heard anything about that?  

 

Krzystof Turek: I don’t know any person who is Jewish. After WWII and after the next cleansing made by the Communist Party, they were barely any Jews in Poland.  

Dr Steffen Huber: I met some people. They are living in a quite normal manner and of course they will tell you about anti-Semitism. It’s not an everyday experience. It's not systematic physical aggression, but they will tell you that of course it happens from time to time. 

 

 You were also talking about the role of romanticism. Do you think it only is conducive to anti-Semitism because of its negative stance towards rationality? 

 

Dr Steffen Huber: No, this has to be treated very carefully. You’ll find some roots of aggressive racist nationalism in romanticism. But some authors are clearly pro-Jewish and they take a lot of basics from the Jewish tradition. Those who fought romanticism in the 19th century in Poland belonged to the positivist movement which in the beginning was very liberal but after 30 or 40 years, at the beginning of the 20th century, it turned into the strongest and most serious strongest anti-Semitic force in Poland. The national democratic party is rooted the positivist movement. This is quite strange and you cannot say that anti-Semitism is romantic, anti-rational and so on, and positivism is pro-Western and liberal and rational. It is just not true. I think a substantial difference between Poland and Germany is that the Jewish culture in Poland was much more conservative and much more religious and much more community and family based. This was the experience of the Poles and this is how they tried in 19th century to make some kind of Polish Jewish dialogue which really worked out in a very great manner.  You have great pieces of literature and theater which deal with these common metaphysical feelings. This is really a great part of the Polish literature and culture, and this is also in the writings of Pope John Paul II, which of course are not read by the average Polish reader right now. “Lingua Tertii Imperii” by Klemperer shows it in the clearest way. He said that the Nazi ideology is based on a romantic pattern. But that this is a thing you cannot say about Polish culture; it wouldn't work out here.  

Krzystof Turek: I want to give another example: “Pan Twardowski” is a work by Adam Mickiewicz, the Polish “Faust”. The Jew of the story is one of the most politic characters in the whole complex work. 

The main idea behind this is the idea of fight between good and evil. A Messiah is a person who stands up against evil but for the very cause of standing up against the evil even if the cause is lost, as it was during the many uprisings that also are very important for Polish identity. Even if you know you fail. In 1943, there was the Warsaw uprising in Warsaw ghetto, where thousand Jews rebelled against the good working war machinery of the Third Reich, and even though they knew they would lose, they still made the uprising. This is an obvious thing to do for someone who thinks this way. For both Poles and Jews. This is also the thinking now, about what the polish right-wing government is doing. They are standing against the evil of multiculturalism because it is evil and destroys European culture. 

Dr Steffen Huber: Which is not a romantic message. If you read Mickiewicz, he is never talking like that. He was trying to restitute the ethnically and culturally heterogeneous Poland which existed until late 18th century. This was the romantic model of restitution in Poland, so, yes, Kaczyński is trying to put together two traditions, the romantic and positivist tradition. Yes, of course, there is some anti-Semitism. Of course, there is xenophobic material in romantic traditions, but there is more of it in the positivist tradition if you look at the late works. There was a strong conflict in the Second Republic before WWII between the romantic and the positivist parts of the society, and Kaczyński is trying to put these two traditions together. In a way, he is a pluralist talking about unity but practically, he is putting together very heterogeneous elements. If it helps him to get the effect he wants, he also uses anti-Semitism, but this is just an instrument for him. There is no deeper conviction. His conviction is that the west is bad. 

 