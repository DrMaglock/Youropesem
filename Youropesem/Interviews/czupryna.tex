Interview with Serhii Czupryna  
Serhii Czupryna (*1996) is a member of the Jewish community in Kraków. As such, he is involved in educational activities fostering dialogue between Jewish and non-Jewish people. He was born in Ukraine but is of Polish-Jewish descent, he is living in Kraków since 2013. He has lived in Israel for several years while studying as a kid and later as a university student. He studies at the Institute of the Middle and Far East of the Jagiellonian University. \\

We met him in the Jewish Community Centre (JCC) in Kraków on January 27th, 2018. \\


 

Have you ever experienced anti-Semitism in your life? 

Serhii Czupryna: Yes. Living in Israel, I’ve been to the West Bank a lot of times and I wanted very much to go to Gaza, but that was difficult for an Israeli. In Poland, I’ve never experienced anti-Semitism in its harshness as it may be seen by people from outside of Poland. The first question a lot of Jews, especially in Israel, ask me when I say I live in Poland and that’s where I chose to live, is: “Why? Six million. Why? But I see Poland not only as a place of historical depreciation of Jewish culture, but as a great place for its revival. I feel super comfortable of being Jewish in here, I feel super safe to walk around the city with a kipah on my head. I’ve never seen anybody in Kraków being mean to me just because of the fact that I’m Jewish. 

So, do you think anti-Semitism is a serious problem in Poland? 

Serhii Czupryna: I don’t think of it in terms of the Poles against the Jews and the Jews against the Poles. I see it as a lack of education about who Jews are, what they do. I’m very happy to be with such a group like you and to answer questions, because ialogue is the first step to get to understand another culture.  

 Is the form of anti-Semitism in Kraków more passive than in other cities? 

Serhii Czupryna: It may be passive, as I see Kraków as a big globalised city, in Polish borders it’s one of the biggest cities, there are a lot of universities here, there are a lot of young people, young adults who come here for the outsourcing companies, to work here. From year to year, I see that this city is becoming more and more international and multicultural. In this situation, I don’t see growing hate for any other culture. That’s nice, it's kind of a change that I’ve seen since I came to Poland. Now, it’s not only tourists who come here and see the Old City, go to Auschwitz and go back home, but there are people who don’t even speak Polish yet they live here, work here, do their everyday stuff here. By this, they’re bringing their own culture that works for everybody, also for the Jewish culture which is here, and works positively for Jews as well.  

 Do you think people in Kraków want to know more about Jewish culture? 

Serhii Czupryna: Yes. There is a Jewish Culture Festival annually since twenty-something years. Visiting this event year by year, I see that more and more people are coming and simply because this event exists, I would say that people are definitely interested in getting to know Jewish culture more.  

Do you as a student work with secondary school students, or do you do anything to educate people about the Jewish culture and heritage? 

Serhii Czupryna: I would love to, but as I do my masters right now and I work full-time, I don’t have much time to do so. But as soon as I’m done with my masters, I’m going to educate people about the Jewish culture, the influence of it on the Polish culture. 

Is the Jewish culture matter of the teachings, generally? 

Serhii Czupryna: Yes, there is a whole department of Jewish studies in the Jagiellonian University, so it’s both historical and cultural. Academic studies on Jewish culture exist in the city, they thrive and revive.  
I also know a few NGOs who go to schools and educate teachers. But it’s difficult to teach about the Holocaust, because according to the programme, it’s taught in an inappropriate way, maybe some teachers just go through it or don’t even mention this subject. And maybe some will say, “six million people were murdered here during the Holocaust”, and then full stop. But a lot of people don’t mention that then there was communism, and that in the 60s, a lot of Jews had survived the Holocaust and either stayed or left the country. It’s very important to educate in a correct way about Jews and about the historical context that the Jews are in here. I’m very supportive of those NGOs. I’ve participated in a few of their events.  
Also, here in the Centre there are groups coming from the US, Israel annually, so this kind of impact on teachers is not only on the Polish side. When Israeli groups for example come to Kraków, they see only the history, they see the Holocaust, and a lot of groups just go through this building and say “ok, this is the Jewish Centre, it’s cool that the Jews were here”, but they don’t see the community in here. So, for two years there has been an annual course for the guides of those groups from the US and Israel. They teach them how to raise more awarness for the existence of this community here.  

Let’s continue talking about education. Would you agree that some media say that Polish anti-Semitism operates on a verbal level, mostly? Do you think it’s a chance to decrease anti-Semitism just by educating young people, especially in small cities, small villages and towns? In Kraków, people are better educated, they have lots of possibilities to meet Jews and Jewish culture. What if we started educating people, students in smaller villages and towns? Probably the hate speech and anti-Semitism on the verbal level would decrease. I think it’s the biggest problem that people sometimes say, “I hate Jews even if I haven’t ever met any”. 

Serhii Czupryna: Exactly. I would say that it’s a very important matter because the bigger the city is, the more chances you get to interact with people different from you. But for example, if you’re from a small village, you have one school in this village, you have the homogenous society of Poles, you may have certain stereotypes about other people and other nationalities. It’s like this, “Everybody hates Jews, I also hate Jews, I don’t know why”. I definitely see that there’s a very high importance of coming to smaller cities or villages to educate people on those matters because very often, nobody even knows that before the war the whole village was eighty percent Jewish, while now it’s one hundred percent Polish. A historian would come and say that there was the house of that and that person, there was a cemetery there and there, but locals don’t even know about this. Education would help to change the verbal level of anti-Semitism. A lot of people, for example Polish, would use the word ‘Jew’ in this a little bit pejorative form of ‘Żydek’. There are two different words: the person who believes in Judaism and the person who is born Jewish by nationality, and those are both different from ‘Jew’, but the word itself, ‘Żyd’ comes from the Hebrew three letters, the base of a word Jehud which stands for the person who believes in Judaism. For me, knowing that fact when people use the form of the word ‘Jew’ in Polish to any pejorative meaning, I explain them ‘guys, this is how it works, and stop because it really has nothing to do with this matter’. But the level of the verbal anti-Semitism itself is, from what I’ve noticed, not even that harsh in Poland. It definitely exists, but even comparing to Ukraine, there is way less of verbal anti-Semitism. I’ve came across this a few times in my life, but I also don’t see it as anti-Semitism, because people maybe just didn’t know what they were saying. Education is the key to change all the negative meanings in our lives and our societies. As soon as we educate ourselves and other people on certain matters, the negative aspect of them definitely go away.  

I’m not a history teacher, but my colleagues told me that students are usually not interested in the lessons of the Holocaust and anti-Semitism, it’s a little bit tiring for them. Don’t you think that it would be a better idea to introduce some practical classes about it, I mean instead of two or three lessons about anti-Semitism and the Holocaust, let’s go to Auschwitz, let’s go to your Community to show students what you do.  

Serhii Czupryna: I would totally agree with this. I mean, I would not necessarily agree with the fact that high school students are the best audience to go to Auschwitz. I went to Auschwitz for the first time when I was twenty, that was far away from my high school ages. And being twenty, I realised well that I was not ready for it yet. So, I would not say that visiting Auschwitz is the best way of practical studying about the Holocaust and anti-Semitism, but such things may on the other hand, increase the interest of the students.  

 Do you as a Jewish Community Center organise workshops for students?   

Serhii Czupryna: There is an organisation based in Kraków, they have a project which is called “Alev Bet of Jewish culture”, “the alphabet of Jewish culture”. It takes place annually in the autumn or in winter. They educate people on the whole basis of the Jewish culture, explain the basics of certain traditions and celebrations. 

Are you a religious person, and do you take part in Jewish celebrations? 

Serhii Czupryna: I’m not that religious, but yes, I would definitely say that what I believe in is Judaism. Not the orthodox version of it, but I also live in a kosher home. I don’t put lights on on Shabbat, and there are mezuzahs at the entrances to my rooms in the flat. I keep the kosher kitchen. I live with two more religious friends. From time to time, I go to synagogue.  
There is the possibility of being a religious Jew in the city. There are three more or less full-time functioning synagogues. One is the Isaac synagogue, probably it’s the biggest in a matter of size, and it’s run right now by the movement Chabad-Lubawicz. So, their aim is to teach the Jews how to be a Jew and also to teach other people what the Jews are, and educate about Judaism. So, their main matter here is education, both to Jews and to non-Jews. They also run religious Jewish Sunday school. Here, there is a Jewish preschool, where children also get to know more about Judaism, about the tradition, religion, and certain religious celebrations. Three times a day there is a minjan, just like in the more orthodox Remus synagogue. The Chabad synagogue is more open for everybody. There is a full-time rabbi from an organisation which is called Szewa Israel, who works for a chief rabbi of Poland, Michael Schudrich, and he is a full-time employee of the JCC. He also provides the Thora studies on Shabbat and helps people to convert. There are many different options of being a religious Jew in this city, as well as in other cities in Poland.  

Do you speak Yiddish and, in general, how many people do?  

Serhii Czupryna: I don’t speak Yiddish because of my family’s historical background. There is a certain amount of people who live here, who speak Yiddish. First of all, the more religious, orthodox Americans who chose to live here, they probably would speak Yiddish or at least some standard. Until this summer, we had one of the oldest members of the JCC, Mr Mundek for whom Yiddish was the first language, so he learnt Polish as a second language, not native. And he was born somewhere around Kraków or maybe even in Kraków. He spoke perfect Yiddish, unfortunately, he passed away this summer. There are certainly other members of the community who speak Yiddish. In this building, there is an opportunity of having a private course of Yiddish. Besides, they teach Yiddish in the Jewish studies department of the Jagiellonian Univeristy.  

And what about Hebrew? 

Serhii Czupryna: I do speak Hebrew. Right now, I work as a kind of Hebrew native speaker in one company in the city. There is also the possibility of joining a Hebrew course here, it is provided for ten different levels of Hebrew language skills, so there are people who started learning it since the JCC opened in 2008, and they are still in their groups that continued from 2008, almost for ten years. Hebrew is nowadays very popular in Kraków, I’d say.   

Are there any orthodox Jews in Kraków and in Poland in general? 

Serhii Czupryna: Yes, definitely there are a lot of orthodox members of the community in general, and are few orthodox members of the JCC community. First of all, not every orthodox will have a beard, not every orthodox will have sztrajmł, this huge hat. A lot of young people find themselves feeling way more comfortable in orthodox movement, they look just like us, you can spot only certain things which show that a person may be orthodox. When it comes to interacting with them in everyday life, you would notice that if you say ‘let’s eat sandwich’, this person says ‘oh, but I need to wash my hands’. And there is the whole ritual, it’s complicated, Judaism is complicated.  

Last November there was the demonstration in Warsaw with sixty thousand people, and hundreds of them were shouting ‘Sieg Heil’ and ‘Jews out’. 

Serhii Czupryna:  It’s the question that I often get from the people who ask me, “why do you chose to live in Poland? You’re Jewish, go to Israel”. You’re talking about the Independence Day demonstration. The thing is that there were the far-right neo-Nazi movements, and it may be not very safe to interfere with them in Warsaw on the Independence Day. I was there once, I just had to help my friends. I had two friends from Israel visiting me in Kraków, and we came to Warsaw together on  November 11th. And one of them is of half-Indian, half-Iraqi descent. He was born in Israel. And the second is from Ukraine. The problem was that none of them speaks Polish, and one of them looks definitely not Polish. So, I was just afraid of them going to Warsaw by themselves, especially seeing this in the media. But as I came there with them, I realized that as soon as we just don’t go any close to the demonstration, which is controlled by police to a certain extent, we’re fine, there’s no problem with speaking Hebrew in the centre of Warsaw in the evening of the 11th. I would say that it’s very important that media show this, but I also see it as just a different opinion as long as it doesn’t interfere. They may destroy certain parts of Warsaw, but they don’t hurt people. As long as they don’t hurt people, I’m fine with it. That’s not the majority of Poland. According to me, their ideology is definitely a bad thing, but I see the lack of knowledge about the Jews. Even trying to interact with far-right people or with people who perceive neo-Nazi movements, both in Ukraine and in here, one of the main claims was: Jews go home. Why? Lots of Jews were born here, Poland was pretty much Jewish before the War, so where should we go home to? I can go home to Ukraine, but my ancestor is from Poland. I see both of these countries as my home. So, to which home should I go? And then the conversation goes on, “oh, okay, you were born here, so you’re fine, you speak Polish, it’s a different thing”. No. It’s totally the same thing, start educating those people about what is Jewish, who is a Jewish person. Sometimes, I can open their eyes and they say, “okay, I’ve been doing wrong” or “that’s probably the ideology, maybe it’s not that prefect, so maybe I should switch for something less far-right". I definitely see there’s lack of education, and what I see as a problem is that the Polish government is not intervening in that situation. Until the moment the government bans neo-Nazi organizations, I think that it’s going to be the same, but it’s definitely not these sixty thousand people. 

The whole demonstration consisted of 60,000 people. And some hundreds were shouting. 

Serhii Czupryna: Yes, it’s a small percentage.  

But nobody from the sixty thousand cared about it. Normally, I would throw these people out of my demonstration. 

Serhii Czupryna: Yes, that’s also an organisational problem. What I dislike more about those marches on 11th of November is that each year, there was a symbolical burning of the statue of a rainbow on one of the squares, which stood for diversity and tolerance. If it was removed, so that it’s not going to be burnt annually, that would also be a good step, first of all not to symbolise the intolerance while celebrating the pride of being Polish. I cannot call it patriotism; I call it nationalism. But certain people go there to celebrate patriotism, certain people go there to celebrate nationalism. That’s the problem that maybe the organisers of those marches face. 

Don’t you think that the media, especially the media abroad stress the nationalistic marches and the fact that there are some acts of nationalism, not patriotism? What do they want to stress?  

Serhii Czupryna: Media never show something like ‘the panda was born in this and this zoo, let’s celebrate it’, it’s definitely not interesting for people abroad. But if people hear that something bad or something incredibly good is happening, only then they would be interested in what is shown to them. That’s why I would say that media stress this fact of nationalism, not patriotism. This year in Israel, it was the Independence Day of Poland, and I saw a lot of videos on the internet, on social networks, where were people saying “I go to that demonstration not because of the fact that I’m nationalist, but because I’m patriotic.” And a lot of them were translated to Hebrew or at least had Hebrew subtitles, so the situation about this matter of showing those marches as nationalistic starts to change, I would say. But it takes a certain amount of time to people to understand that there is a difference between people who go there to show some nationalist beliefs and between the people who are patriots of their country, who don’t try to put any other nationality below them, who just celebrate the fact that they are proud of who they are.  

Do you think that teenagers are more anti-Semitic than older people because they frequently use the internet and see a lot of hate about Jews? 

Serhii Czupryna: It’s a really cool question. I think that the middle-aged group is the most anti-Semitic of all age groups. What I like about the globalisation is that if you’ve seen a lot of hate on the internet, you can just google certain things that may disprove this hate, and the availability of the information for teenagers I see as a very good thing. I would definitely not say that teenagers may be more anti-Semitic just because they see some propaganda on the internet or in the media. Now, in my generation, I’m twenty-one, it’s always like “okay, I would look for this, but also I want to find the arguments from the other side”. I would say that a lot of people would choose the peaceful version of perceiving some other minority, not the hateful one, and they would probably want to learn more about the conflict. 

Do you think that the anti-Semitism problem will grow in the future? 

Serhii Czupryna: It’s an interesting thing. It would definitely depend on the region, and what is going to happen there in the future. I would say that in Poland no, because from what I see right now, non-Jewish Poles are getting really interested in the culture, and they start understanding the influence of Jewish culture on Polish culture. It decreases radically, which I like a lot. Maybe in countries like France and Great Britain, which are more globalised, it may grow because of a negative change in the society, connected to other negative factors that are happening with the society. And as the country struggles with certain things, there is a possibility of other negative factors to increase. But the world learnt through the Second World War that anti-Semitism was the depreciation of a certain national minority, and they learnt from the Holocaust that it happened once, so let’s not make it possible to happen anymore, with any group, not only the Jews.  

There’ is a new law in Poland called the law for protection of the good name of Poland. What do you think about this law? 

Serhii Czupryna: I would not agree with a lot of laws and the implementations into the law that the current government brings on, not only in term of this one. From the patriotic side, this matter actually stands in line of the tension between patriotism and nationalism of the current government. It’s really hard topic: Auschwitz, the relations between Israel and Poland, but the whole relations between Poland and Israel and also the exchange of tourists and students and people just coming and going back and forth, these relations also increased in the recent few years. I think that as soon as the government changes, at the point where this law may be partially changed, it’s going to be fine. I don’t think that it may drastically change the relations between Israel and Poland, because diplomatic relations have been built up for a long time, with a lot of effort, so I don’t think that one law like this may change a lot. I certainly see it as something which is really a problem, which depreciates the importance of other nations, national heroes of other countries. The same topic was raised recently in Ukrainian media, also with regard to this law because the national heroes that the Ukrainian government perceives to be national heroes were fighting against Polish forces in a certain period of time, and throughout the history, this whole interfering between the Poles and the Ukrainians took place. It was perceived negatively in Ukrainian media, and also in Israeli media, and it was depicted a anti-Semitic. But I would not say it will drastically change the situation between two countries.  

According to Jan Gross and his book Fear, in Poland solidarity with the Jews during the Second World War was not a mainstream thing. That means in France, people were celebrating when they helped Jews, and in Poland, many people are ashamed and they don’t want to be mentioned to publicity. Is that right? 

Serhii Czupryna: The whole matter of helping Jews during the Second World War is a very touchy subject, because a lot of people say “oh, Poland is this death camp where everything happened”. No, it’s not. It’s just because of the fact that, sorry, Jews were living here, and it was the easiest way from the Nazi point of view to make it all happen there. Most of the Jews in this world were living in Poland. Helping the Jews during the War, you could have got a death penalty, you and your whole family. And probably this fear still remains in certain people. But on the other hand, when Israelis ask me, “Why do you live in Poland? Most of it happened in Poland.”, the other side is that also the biggest amount of Righteous Among the Nations live in Poland, they are Polish and they are proud of it, and they celebrate helping the Jews in Poland during the War. 

Have you experienced any anti-Semitic acts from Muslims when you were in Israel? 

Serhii Czupryna: Actually, no. I feel very safe living in Israel. It’s just a fact that the country exists in the region where it has to be militarized to a certain extent, just to exist. And that’s what I know: That in Poland, you would feel weird if you’d go to a bus and there’s a soldier with a gun. In Israel, it’s totally a different thing. I feel safer when I see a soldier with a gun, just a normal servant who is of my age and who just has to do the service for his country. I have a lot of Palestinian friends, certainly some of them are Muslims, and I’ve never experienced any negative acts from their side. Also, in Poland I have a lot of Muslim friends , and I get along with them very well just because of some common things between their cultures or our religious beliefs.  

How do Polish people behave towards you? Are they curious about your culture and religion? Or are they intolerant and reserved? 

Serhii Czupryna: They’re definitely interested. I participate in few non-governmental organisations, except for the JCC, and it’s interesting to see how people who are not even friends are interested in the fact that I’m a Jew and my friend is a Muslim. How come that we sit at the same table? We explain and say it’s just a religion. Why should we sit at separate tables or not even go to the same café? There is a certain amount of curiosity. People may see me coming from the synagogue with a kipah, and my Muslim friend in hijab, and we’re just coming across the same street, stopping to say hello to each other. Seeing the face of these people with certain stereotypes that they may have about our cultures, and seeing us destroying those stereotypes – it’s really pleasant and funny. But I would say that to curiosity, there also comes a certain amount of stereotyping certain cultures, and we try to together destroy the stereotypes.  

You said that people in Israel are always surprised when you are saying that you chose to live in Poland? Why do you think the do? Is it because of history and media or stereotypes or even something different? 

Serhii Czupryna: I would say because of history and because of stereotypes. The biggest after-War influx of Polish Jews to Israel was in the 60s and in ‘68. That generation already has kids and grand-children, and probably a lot of them have never been to Poland except for Auschwitz. They think, “My family survived the Holocaust, then they were kicked out of Poland”. This is the vision that they have about Poland since their childhood, from their parent and grandparents. I love bringing my Israeli friends to Poland to show them that this country in not black and white, as you see it on the pictures of your granddad. It’s the country which has a lot of beautiful sites, except for Auschwitz, they definitely should visit. And I like to see those people being shocked when we talk Hebrew in public and everybody is fine with it. It’s also a matter of education, destroying the stereotypes, it definitely comes from the stereotypes of Poland, and those stereotypes are created by a certain amount of negative historical background of Poland with regards to the Jewish society.  

Mrs Zajda told us yesterday that these tourists from Israel are always isolated by bodyguards. They only go to Auschwitz,  theydon’t integrate with other people, so they can’t even get to know our culture, our behaviour. 

Serhii Czupryna: It’s true. Because of the fact that most tours from Israel are organised with the help of certain ministries, the ministry of education or the ministry of foreign affairs. They should have bodyguards, it’s fine. The problem comes when these groups go to synagogues. Can you imagine? We have the school trip to Italy and even if you’re not religious, you go with bodyguards to the Sunday mass, and you’re supposed to spend the whole time in the mass. That’s what happens when those groups come to the synagogues and, according to me, it’s like one of the small windows for them to interact with the local community because technically we also go to the same synagogues. Often when I try to enter the synagogue, I’m stopped and asked by Israeli bodyguards “Wait! Stay here. Where are you going?” Obviously, I have a kipah on my head. “Come on, why should a Jewish person go to synagogue on Friday?” I’ve noticed that the same bodyguards come with different groups. And after few situations like these, they start to remember your face and then say “Shabat shalom, shalom”. And certain Israeli teachers, if you get to interfere with them, anyhow talk to them, they start saying “Okay, oh, so you live here? You’re Jewish? Fine. Maybe we can make some meeting of our group with you?” It just needs some time to develop and also to explain the Israeli teens that Poland is not black and white, as on a picture from the War, and people here are smiling, and it’s not bad to talk to them and you don’t have to have bodyguards when coming here.  

I think there is a problem because in most of the classes, there are sixteen or seventeen- year-olds, and when they come to Poland, they visit two extermination camps a day. Of course, it’s too much. And the government of Israel makes a good silence with it. It’s a little bit propaganda.  

Serhii Czupryna: It’s tricky because in the present situation in Israel, both political and military, in different contexts, I see that this propaganda is needed, the propaganda of “let’s not let other people to destroy Jewish nation”. But it’s done inappropriately in the case of Poland, because instead of having this feeling that “yes, we need to stay together as a nation, we should not fight between each other and keep everything as peaceful as possible”, those children probably learn about the fact that all the killings took place in Poland, and that’s the only thing they know about Poland. And that is a problem, it’s inappropriate propaganda. But Still, I see it as a need. I hope that Israeli will come up with change one day.  

Don’t you think that the Jewish trips of students who come to visit Auschwitz or Bełżec and have gunned bodyguards can raise a bad attitude among Polish people? We may feel being perceived hostile. You said that the ministry tells them to do that. 

Serhii Czupryna: It’s actually a requirement by the ministry of education: As soon as you go to any tour, even in Israel, if you’re a group of more than ten people, you should have a bodyguard with a weapon. For them, it’s not exclusively for Poland. But I definitely understand, as a person born here, how people see these groups. It’s weird for Poles, it would be weird for Israelis not to do this, it’s just the conflict of two contemporary cultures. Obviously, they know that is safe in Poland, because the same kids in their free time are just rushing to the shopping mall and those guards don’t go there with them, obviously. That’s just a requirement.  

 Don’t you think that Polish people are most sensitive about it? I mean when Jewish trip goes to e.g. Germany, Germans don’t care about a gunned man. 

Serhii Czupryna: I’d say that Poles see it this way because of the history. In this historical context of Poland, after all that happened in this territory, it’s just unpleasant to see an armed person in any situation. But I don’t think that Germans are less sensitive in this matter.  

What are your thoughts of the people who say that Jews are very influential all over the world? Is it the reason that the people are anti-Semitic? I’m about the Rothschild family.  

Serhii Czupryna: Yes, definitely, in a certain historical context. Let’s start with the different variations of anti-Semitism. Let’s figure out that there are non-scientific terms, a ‘positive’ anti-Semitism and ‘negative’ anti-Semitism (but please, don’t stick to these terms in any other situation). A ‘positive’ anti-Semitism is, according to this understanding, based on the good stereotype about the Jews, but they are still stereotypes. For exampl,e those small figures of Jews holding a coin that you can buy as a souvenir on the main square. It is anti-Semitic, let’s face the truth. But on the other hand, it’s a ‘good’ stereotype, that they are good with money. O the thing that we’re all lawyers, doctors, musicians and maybe scientists, sometimes - three things a Jew can do in their life. It’s a very good stereotype when people say “oh, you’re a good lawyer, it’s because you’re Jewish” - “No, it’s just because I’m smart”. And all those things, they have a negative context but they are not that negative as an actual anti-Semitism.  

How did the Polish society deal with some anti-Semitic, some Catholic anti-Semitic facts like the Madagascar option in the thirties, and the pogroms in Kielce or Jedwabne, was it critically discussed?     

Serhii Czupryna: There is a critical discussion and there is a commemoration of those events by both Jews and Catholic priests. It is like the Holocaust: it happened, somehow the world did allow this to happen, but let’s not make this happen any time again. I would say that currently, the Catholics in Poland and the Catholic Church especially in Poland also stands for the facts, for the same thing in terms of pogroms, ‘pogrom kielecki’, it’s the field that is still studied and is not finished to be studied. It’s a very complicated matter, but I would say that the Catholic Church right now is very active in the field of commemoration, to let the people know this with the message of “let’s not do this again, because it’s bad, let’s not kill people”. That’s one of the bases of the Catholic religion.  

 