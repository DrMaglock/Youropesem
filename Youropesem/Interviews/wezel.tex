Interview mit Dr. Katja Wezel 

Dr Katja Wetzel has been a research assistant in the project "The cosmopolitan city. Riga as a Global Port and International Trade Metropolis (1861-1939)” at the Seminar for Medieval and Modern History of the Georg-August University Göttingen. 

Her research focuses on the history of the Baltic countries, especially Latvia, on history politics and culture of remembrance, nationalism and ethnic conflicts, as well as spatial history and digital history. 

From 1999 she studied history and English in Heidelberg, Aberystwyth and St. Petersburg and completed her Ph.D. thesis titled "History as Politics. Latvia and the Historical Reappraisal after Dictatorship" at the Ruprecht-Karls-University Heidelberg. From 2013 to 2018 she worked in the field of history at the University of Pittsburgh, USA. We thus got in touch with her via Skype for the interview on June 30th, 2017. 

 Inwiefern haben Sie sich mit Antisemitismus in Lettland auseinandergesetzt? 

Dr. Wezel: Ich habe meine Dissertation zur Aufarbeitung des Kommunismus in Lettland geschrieben.1 Dabei habe ich mich ursprünglich gar nicht mit der Frage auseinandergesetzt, wie Lettland mit seiner nationalsozialistischen Vergangenheit umgegangen ist. Aber mir ist relativ schnell klar geworden, dass die beiden Themen immer wieder zusammenfallen, gerade bei der teilweisen Verwendung des Genozid-Begriffs. Es gibt bestimmte Gruppen oder auch Wissenschaftler, die diesen auf die sowjetischen Deportationen in Lettland anwenden, insbesondere die beiden großen Wellen 1941 und 1949. Da habe ich mich natürlich gefragt, inwiefern es eine Auseinandersetzung mit der jüdischen Geschichte und mit dem Holocaust in Lettland gibt. Dabei habe ich herausgefunden, dass der Holocaust als eine Art Vergleichsfolie verwendet wird. Natürlich liegt der Hauptfokus für die Letten auf der Aufarbeitung des Kommunismus und auf der Aufarbeitung der Verbrechen Stalins. Dann erfolgt ein Rückgriff nach dem Motto: Der Holocaust wurde aufgearbeitet und das müssen wir jetzt auch mit den Verbrechen Stalins tun. Dieses Vergleichsmoment birgt natürlich bestimmte Problematiken in sich und das habe ich primär untersucht. Meine These ist, dass in Lettland ein ganz starkes Augenmerk auf den Molotow-Ribbentrop oder Hitler-Stalin-Pakt gelegt wird, der als der Beginn allen Übels betrachtet wird. Dem wohnt dieser Vergleich der Verbrechen schon inne, denn wenn Sie sagen, damit hat alles begonnen, Hitler und Stalin haben sich verbündet gegen uns Kleine – so wird das aus lettischer Sicht wahrgenommen – dann ist dem implizit, dass die Verbrechen von Hitler und von Stalin auf gleich Weise begangen wurden und deswegen auf gleiche Weise aufgearbeitet werden sollten. Weshalb der Hitler-Stalin-Pakt so zentral ist, geht dabei zurück auf den Beginn der lettischen Unabhängigkeitsbewegung. 

Für manche steht Kārlis Ulmanis noch heute für die Unabhängigkeit Lettlands. Was war charakteristisch für seine Diktatur und wie ging es den Minderheiten unter seiner Herrschaft? 

Dr. Wezel: Ulmanis ist eine wichtige Gestalt, um die Zwischenkriegszeit zu verstehen, wobei man da ein bisschen früher ansetzten muss. Ulmanis war mehrfach gewählter Ministerpräsident, bis er 1934 einen Putsch gemacht hat. Nach dem Ersten Weltkrieg ging es sehr stark darum, dass die Letten die Verantwortung für ihren eigenen Staat übernehmen und die Minderheiten aus verantwortungsvollen Positionen entfernt werden. Für den Fall der Universität Lettlands wurde dies vom schwedischen Forscher Per Bolin in einer Detailstudie untersucht.2 Er weist darin nach, dass es immer darum ging, Letten in verantwortungsvolle Positionen zu bekommen und dass Minderheiten zwar akzeptiert wurden, aber nur für den Übergang. 
In der kompletten Zwischenkriegszeit spielten kulturelle Grenzen und die Identifikation mit der Muttersprache eine bedeutende Rolle. Aus lettischer Sicht war man Deutscher, wenn man Deutsch als Muttersprache sprach, obwohl sich die meisten Deutschen als Baltendeutsche und nicht dem Reich zugehörig gefühlt haben. 
Der Coup 1934 ist aus Ulmanis Sicht notwendig, denn Lettland war eine instabile parlamentarische Demokratie, die mit genau solchen Problemen zu kämpfen hatte wie Deutschland und andere Staaten dieser Zeit auch. Lettische Regierungen haben immer nur drei, vier Monate gehalten und brachen dann wieder zusammen. Es gab wahnsinnig viele Parteien und Regierungen im Parlament, weil es keine 5%-Klausel gab wie heute; das machte politisches Arbeiten sehr schwierig. Nach der Wirtschaftskrise wurde es politisch noch schwieriger, was einer der Gründe war, weshalb sich Ulmanis an die Spitze gestellt hat. Eine der wichtigsten Konsequenzen dieses Coups war sein Slogan „Lettland den Letten!“. Es ging darum, vor allen Dingen die ganzen Deutschen loszuwerden. Das hängt damit zusammen, dass Riga sehr stark von der deutschen Kultur geprägt war. Es gab aber relativ viele Juden, die auch deutschsprachige Schulen besuchten und die im deutschsprachigen Milieu zuhause waren. Und es gab gerade unter Händlern, Kaufleuten und Unternehmern relativ viele Juden, gegen die sich das natürlich auch richtete, auch in Form von Enteignung. Ich würde aber sagen, dass dieses Enteignungsvorgehen gegen die jüdische Bevölkerung relativ wenig mit Antisemitismus zu tun hatte und viel mehr mit einem sehr, sehr starken Nationalismus und diesem Verständnis, „Lettland den Letten“. Dieser starke ethnozentrische Nationalismus ist letztendlich ein Problem, das Lettland heute in einer anderen Form immer noch hat, jetzt richtet er sich gegen die Russen. 

Wenn Minderheiten über die Sprache definiert wurden, galten deutschsprachige Juden dann überhaupt als eigene Minderheit? 

Dr. Wezel: Das Problem der jüdischen Bevölkerung in Riga, aber auch in Lettland allgemein war, dass sie sehr stark auseinanderfiel. Jiddisch sprachen eigentlich kaum noch Leute in Lettland. Es gab zwei Gruppen von Juden in Lettland. Die einen hatten sich der deutschen Gemeinschaft mehr oder weniger angeschlossen und schickten ihre Kinder auf deutsche Schulen, die anderen waren aus dem russischen Zarenreich gekommen und sprachen russisch. Zwischen der deutschen und der russischsprachigen jüdischen Gemeinde gab es aufgrund der Sprachbarrieren relativ wenig Miteinander. 
Die Religion war dennoch von Bedeutung. Es ist wichtig festzustellen, dass es im späten neunzehnten, auch noch im frühen zwanzigsten Jahrhundert undenkbar war oder faktisch nicht vorgekommen ist, dass ein Nichtjude einen Juden oder eine Jüdin heiratete, weil die konfessionellen Grenzen sehr bestimmend waren. Die Deutschbalten waren alle Lutheraner. Es gab durchaus intermarriage; es gab Verheiratung zwischen Letten und Deutschen, die den gleichen Glauben hatten, aber mit Juden kam es faktisch nicht vor. In dieser Hinsicht gab es einen großen Unterschied zum deutschen Reich. Dort gab es jüdische Bevölkerung, die sich komplett assimilierte und sogar zum protestantischen Glauben übertrat. Das gab es in Riga oder in Lettland gar nicht. Das führte dazu, dass die Gruppe als solche weiterhin sehr separat lebte, als „die Juden“, selbst wenn sie Deutsch sprachen oder wie einige wenige anfingen, auch lettisch zu sprechen. Sie wurden trotzdem als die Anderen gesehen. 

Gab es wie in Deutschland auch im Mittelalter schon Antisemitismus? Ab wann gab es überhaupt Juden in Lettland? 

Dr. Wezel: Die ersten Juden, die auf dem Gebiet des heutigen Lettlands lebten, waren in Kurland, das zu Polen-Litauen gehörte, bis das komplette Gebiet 1795 von Katharina der Großen mit der polnischen Teilung einverleibt und Teil des russischen Zarenreiches wurde. Das heißt, in Riga gab es im Mittelalter zwei, drei jüdische Händler, die von Kurland kamen, nach Riga gereist sind und ihre Ware vertrieben haben. Aber das waren Handelsreisende und keine wirklich sesshaften Juden. Erst im 18. Jahrhundert fing es an, dass Juden auch sesshaft wurden. Es war ihnen zuerst nicht erlaubt, innerhalb der Stadtmauern zu leben. Die meisten Juden lebten in der Moskauer Vorstadt, wo es ein ursprünglich von der Stadt eingerichtetes und später selbst betriebenes Gebiet ähnlich einem Ghetto gab. Die meisten Juden lebten dort auch weiterhin, selbst als das Verbot, in der Stadt zu leben, aufgehoben war. In der Rigaer Altstadt wurde erst relativ spät eine große Synagoge gebaut, die 1905 fertiggestellt wurde. Vorher gab es noch keine Synagoge in der Altstadt, sondern nur in den Gebieten, in denen sie lebten. 
Juden waren in Lettland nie eine sehr große Gruppe. Das unterscheidet zum Beispiel eine Stadt wie Riga stark von Vilnius. Juden machten vielleicht sechs, sieben Prozent der Bevölkerung aus und auch in der Zwischenkriegszeit waren Juden mit bis zu 5% in der Bevölkerung eine relativ kleine Gruppe. Diesen historischen Antisemitismus findet man in der Region des heutigen Lettlands in der Form daher nicht. 
Wie haben die Letten die Juden gesehen? Mein Eindruck ist, dass man sie distanziert als die Anderen betrachtet hat. „Sie interessieren uns nicht, solange sie uns nichts tun und solange wir nicht das Gefühl haben, sie nehmen uns etwas weg.“ Wenn es Ausschreitungen gab, die man als antisemitisch aufgeladen sehen kann, was gerade an der Universität Lettlands in der Zwischenkriegszeit vorgekommen ist, ging das von einzelnen radikal-konservativen, nationalistischen Studentengruppierungen aus.3 Die Burschenschaften in Lettland sind in der Zwischenkriegszeit relativ stark gewesen, von denen ging das aus. Und aus den Burschenschaften speisen sich auch später diejenigen, die das sogenannte Arbeitskommando bilden, das Mörderkommando, das dann für die Deutschen während der nationalsozialistischen Besatzung Lettlands letztlich die Arbeit macht. Das war eine Gruppe von sehr radikalen nationalistisch gesinnten Menschen. Es ist immer wieder in Forschungsarbeiten nachgewiesen worden, etwa von Kathrin Reichelt, dass das eher eine Art des Opportunismus war.4 Sie wollten davon profitieren, dass die Juden weg sind und waren weniger von antisemitischen Grundüberzeugungen geprägt. 

Wie hat die Zusammenarbeit zwischen lettischer Bevölkerung und Nationalsozialisten ausgesehen? 

Sehr wichtig ist das sogenannte Arbeitskommando. Das war eine Gruppe von insgesamt ca. 3000 Männern, wobei nicht alle die ganze Zeit dabei waren. Es kam ja erst zur sogenannten sowjetischen Besatzung von 1940 bis ‘41 und dann zur nationalsozialistischen. Arājs ist ein absoluter Karrierist und Opportunist, der erst versuchte, bei den Kommunisten anzuheuern. Als diese dann weg waren, ist er umgeschwenkt und hat sich bei den Nationalsozialisten in Stellung gebracht und wollte für sie arbeiten. Er hat versucht persönlich zu profitieren, was bei vielen der Fall war, die mitgemacht haben. Das waren häufig Männer, die eine sehr starke nationalistische Überzeugung hatten, die sich auch aus der Ulmanis-Zeit speiste, „Lettland den Letten und alle anderen sollen weg“. Das war allerdings eine Gruppe von sehr Radikalen und betraf nicht das Gros der Bevölkerung. Dieses Arbeitskommando funktionierte dadurch, dass sie persönlich profitiert haben: Wenn sie Juden verhaftet haben, konnten sie hinterher in deren Wohnungen einziehen. „Meine Familie braucht eine Wohnung, also mach ich da jetzt eine Weile mit, weil dann bekomme ich eine Wohnung“, so wurde häufig sehr einfach gedacht. Dabei handelte es sich um ein wirkliches Mörderkommando. Sie hatten ihre Zentrale in der Valdemāra Straße und wenn die Deutschen angerufen haben, sind sie aktiv geworden. Sie hatten einen Fahrer, der hat sie dann dort hingefahren. Letztendlich hat das Arbeitskommando die Ermordung der jüdischen Bevölkerung Lettlands in den Kleinstädten innerhalb von wenigen Monaten übernommen. Die Großstädte waren zunächst ausgenommen, in Riga, Liepāja und in Daugavpils hat man Ghettos eingerichtet. Aber in den Kleinstädten, in denen häufig nur vielleicht zwei jüdische Familien wohnten, haben die Deutschen Arājs beauftragt und dieser ist mit seinen Leuten losgezogen und hat die Leute einfach ermordet. Da kann man sich natürlich fragen: Warum haben die Nachbarn nicht geholfen? Um das zu verstehen, muss man wiederum die lettische Geschichte verstehen und die nationalsozialistische Besatzungszeit einordnen. 

Im Jahr davor war das erste Jahr der sowjetischen Besatzung, die mit den großen Deportationen vom 14. Juli 1941 endete und zwei Wochen später kamen die Deutschen. Das heißt, viele Letten waren sozusagen in Schockstarre, weil Familienmitglieder oder Freunde verhaftet und deportiert worden waren, das heißt sie wussten gar nicht, wo diese sind – und jetzt kommen die Nächsten. Viele Letten dachten erst mal, dass es ihnen unter den Deutschen besser gehen würde, im Sinne: „Mit den Deutschen haben wir schon eine Weile Erfahrung gemacht, früher und im Ersten Weltkrieg war es auch nicht so schlimm.“ Das haben übrigens auch viele Juden gedacht. Viele Juden sind nicht geflohen, obwohl sie die Möglichkeit gehabt hätten, mit der abrückenden sowjetischen Armee zu fliehen, weil sie nicht gedacht haben, dass es so schlimm kommen würde. Denn das, was in Lettland berichtet wurde, war gerade zur Zeit von Ulmanis sehr begrenzt. Das heißt, in Lettland wusste man nur sehr begrenzt Bescheid, was in Deutschland abging. Wie stark die Nürnberger Gesetze die Juden entrechteten, wusste man in Lettland nur bedingt. Die These, die manche lettische Wissenschaftler aufgestellt haben, dass den Letten erstens die sechs Jahre Ulmanis und dann das Jahr Sowjetzeit zivilgesellschaftliche Verhaltensweisen abgewöhnt haben, dass man sich lieber auf sich selbst und seine Familie und nicht auf das Ganze konzentrierte und man, wenn die Nachbarn abgeholt wurden, lieber nicht hinschaute, erscheint mir doch sehr glaubwürdig. Man muss auch noch dieses Verständnis mitbedenken, dass die Juden nicht wirklich zum Staat dazugehören, weil in der gesamten Ulmanis-Zeit „Lettland den Letten“ gepredigt wurde. Auf der anderen Seite ist es wichtig, dass es auch Fälle von einigen österreichischen auch deutschen Juden gab, die in den späten 30er Jahren nach dem Anschluss von Österreich in Lettland Zuflucht fanden. Vestermanis hat dazu geforscht. Sie wurden eigentlich ganz positiv aufgenommen oder hatten zumindest das Gefühl, dass sie hier existieren und überleben können. Solche Einzelfälle gab es auch. Aber der Einschnitt der sowjetischen Besatzung 1940/41 ist wirklich zentral, um zu verstehen, wie sich die Letten verhielten, als die Nazis einzogen. 

Gab es die Bewertung, dass es unter der deutschen Besetzung besser gewesen sei, auch rückblickend von Letten, und gibt es das noch heute? 

Dr. Wezel: Kaum, ich habe immer den Eindruck, die Letten interessieren sich relativ wenig für die nationalsozialistische Besatzungszeit. Während der 50 Jahre Sowjetherrschaft wurden die Deutschen natürlich als die Faschisten und die Schlimmsten dargestellt und heutige Letten sind natürlich durch diese Schulbildung gegangen. Dass es heute noch ganz alte Leute gibt, die sagen, so schlimm war das gar nicht unter den Deutschen, halte ich zahlenmäßig nicht für sehr relevant. Die sowjetische Periode, in der das Thema auf eine ganz andere Weise unterrichtet wurde, war da prägender. 

Immer am 16. März wird in Riga die SS geehrt. Das spricht doch dafür, dass es auch heute noch Letten gibt, die das irgendwie verherrlichen. 

Dr. Wezel: Das kann man so nicht sehen. Ich weiß, dass das von Deutschen gerne so wahrgenommen wird. Denn aus deutscher Sicht ist die SS natürlich gleichbedeutend mit den schlimmsten Naziverbrechern, weil sie die Konzentrationslager überwacht haben und so weiter. Man muss aber auch verstehen, dass die Letten die SS gar nicht so sehen. Es liegt schon am Namen, dass es aus lettischer Sicht nicht die SS ist, die da marschiert, sondern die sogenannte lettische Legion. Diese wird quasi völlig losgelöst betrachtet von der SS. Viele der Letten, die das positiv sehen und sagen, denen müsse man doch gedenken, haben überhaupt gar keine Ahnung, was die SS war. Aus lettischer Sicht war die lettische Legion eine verkappte Nationalarmee, die versucht hat, Lettland zu befreien oder zumindest dafür zu sorgen, dass Lettland nicht erneut von den Sowjets besetzt wird, denn das war das Schlimmste. Aus lettischer Sicht ist es tatsächlich so, dass es den Letten unter der nationalsozialistischen Besatzung besser ging, denn es wurden keine deportiert oder verhaftet, es sei denn man war jüdisch. Wenn man ethnisch lettisch war, dann ging es einem unter der nationalsozialistischen Besatzung auf jeden Fall besser und man war sicherer vor Deportationen, Verhaftungen, etc. als unter sowjetischer Besatzung. Als diese lettische Legion 1943 gebildet wurde, haben die Letten das so interpretiert: „Wir versuchen, unser Land vor den Bolschewisten zu schützen.“ Davon gibt es Poster. Es ging also gegen die Bolschewisten. Ob auf der Uniform das SS-Zeichen war, war für die Letten völlig irrelevant. Wichtig war für sie, dass auf dem Gewehr trotzdem das Zeichen Lettlands klebte. Es handelte sich um eine Fremdenlegion, die die Nazis ausgenutzt haben, indem sie sagten: „Wir brauchen Kanonenfutter, wir brauchen Leute, die für uns kämpfen.“ Sie haben einen Deal mit den lettischen Militärs gemacht. Diese haben aber von vornherein gesagt, wir kämpfen nicht an der Westfront. Das war eine Kampftruppe, die auch nicht in Konzentrationslagern eingesetzt wurde, der kämpfende Teil der SS und letztendlich nichts anderes als eine Fremdenlegion, ein Teil der Wehrmacht. Nur da man keine Ausländer in die Wehrmacht aufgenommen hat, hat man diese ausländischen SS-Legionen gebildet. Sie haben also von vornherein gesagt, wir kämpfen nicht gegen die Amerikaner und nicht gegen Großbritannien, wir kämpfen nur an der Ostfront gegen die Sowjets und das haben sie dann auch gemacht. Die lettische Legion hat mit dafür gesorgt, dass Kurland in sechs Schlachten nicht besiegt und besetzt wurde. Kurland war bis zum Schluss unter nationalsozialistischer Herrschaft. 
Nach der Kapitulation am 9. Mai hatten viele der Letten, die da gekämpft hatten, im Prinzip nur zwei Optionen. Sie konnten versuchen, noch übers Meer nach Schweden zu fliehen, das haben auch ein paar gemacht. Dann konnten sie versuchen, in die Wälder zu gehen und sich zu verstecken, was auch manche gemacht haben. Das war sozusagen der Ursprung von den sogenannten Waldbrüdern, die noch bis in die 50er Jahre als Partisanen gegen die Sowjets gekämpft haben. Oder sie sind in Gefangenschaft geraten und meistens im Gulag geendet, nur wenige davon haben überlebt. 
Das war die eine lettische Legion, es gab zwei. Der andere Teil der lettischen Legion hatte es ein bisschen besser. Die andere lettische Legion kämpfte in Vorpommern, geriet letztendlich in britische Gefangenschaft und wurde auf den Nürnberger Prozessen freigesprochen, weil man gesagt hat, die kann man eigentlich nicht als SS-Truppen betrachten, sondern muss sie wirklich separat als Fremdenlegion behandeln. Sie waren nur partiell freiwillig beigetreten; junge Männer hatten Einweisungsbefehle erhalten, für die Legion zu kämpfen. Infolge dieses Urteils in den Nürnberger Prozessen konnten sie auswandern und haben eine Aufnahme beispielsweise in den USA und in anderen Staaten gefunden. 
Diejenigen, die am 16. März marschieren, sind natürlich nur noch ganz wenige Veteranen, da sind ja kaum noch welche übrig. Aber es gibt eben nationalistische Letten, die sagen, wir müssen an sie erinnern, weil sie für unser Land gestorben sind. Der Heldenstatus erklärt sich auch daraus, dass einige von ihnen im Gulag geendet sind oder als Waldbrüder und als Partisanen gekämpft haben. Das erklärt, weshalb man verschiedene Namen in einem Topf hat. 
Nun zur Beteiligung an den Verbrechen der Nationalsozialisten. Man kann die Anführer der lettischen Legion nicht völlig davon freisprechen, weil ungefähr 15% tatsächlich freiwillig beigetreten sind und unter diesen 15%, gab es auch welche, die tatsächlich vorher während des Holocausts dem Sicherheitsdienst geholfen haben. Aber das sind einzelne, es ist definitiv nicht die Mehrheit. Das ist wichtig zu unterscheiden, denn der Holocaust war in Lettland 1943, als die Legion gebildet wurde, abgeschlossen. Es gab noch ein paar Ghettos, aber der Großteil der Verbrechen war vorbei. Insofern muss man das trennen und muss die lettische Seite verstehen, für die die lettische Legion nichts mit der SS zu tun hat, auch wenn es aus deutscher Sicht paradox klingt. 

Die lettischen Kollaborateure im Holocaust waren also eine von der lettischen Legion größtenteils verschiedene Gruppe? 

Dr. Wezel: Ja. Denn die wenigen, die wirklich kollaboriert hatten, wurden von den Nazis dann gerne als Hilfspolizeitruppen weiter eingesetzt. Diese Mördertrupps sind dann weiter nach Weißrussland gezogen und in der Ukraine eingesetzt worden. Unter den lettischen Legionären gab es bestimmt den einen oder anderen, der auch darunter war, aber man kann nicht sagen, dass es die Mehrheit war. 

Die Judenverfolgung in Lettland war unter den Nazis am größten. Aber trotzdem gab es Antisemitismus und Judenverfolgung auch unter Stalin. Sind der Nationalsozialismus und der Kommunismus in der UdSSR zwei gegensätzliche Ideologien, die dennoch denselben Feind in den Juden haben? 

Dr. Wezel: Nein, weil der Ausgangspunkt ein anderer ist. Aber was stimmt ist, dass gerade von den ersten sowjetischen Deportationen und dem 14. Juli 1941 Juden prozentual stärker betroffen waren, als sie in der Gesamtbevölkerung repräsentiert waren. Das erklärt sich wiederum daraus, dass relativ viele Juden als Händler und Kaufleute in wichtigen Positionen ein bisschen bessergestellt waren. Sie galten dann aus dieser Perspektive als bürgerlich und die Kommunisten sind gegen alle Bürgerlichen vorgegangen. Sie wurden also nicht verhaftet und deportiert, weil sie Juden waren, sondern weil sie angeblich zu viel Geld hatten, bürgerlich waren, zu erfolgreich waren und so eingestuft wurden, dass sie sich gegen den Kommunismus wenden würden. 

Wurde das dann auf alle Juden pauschalisiert, dass sie bürgerlich und wohlhabend seien oder wurden vorrangig die Juden verfolgt, die tatsächlich wohlhabender waren? 

Dr. Wezel: Gerade unter den lettischen Sozialdemokraten gab es auch sehr viele Juden und die waren natürlich “die Guten”. Die haben die sowjetische Besatzung dann auch mitgetragen und unterstützt. Der Gegenpol zur lettischen Legion während des Zweiten Weltkriegs waren die sogenannten lettischen Einheiten in der Roten Armee. Darunter waren zeitweise bis zu 17% jüdische Letten5, die entweder noch geflüchtet sind, kurz bevor die Nazis kamen und sich dann der Roten Armee angeschlossen haben, oder die eh schon in Russland waren und sich dann angeschlossen haben. Daher kann man das nicht pauschalisieren. 

Wie aufgeklärt ist heute das Geschichtsbild, etwa in den Schulen oder auch in der lettischen Forschung, zum einen allgemein gegenüber den Besatzungszeiten und zum anderen speziell auf die Judenverfolgung bezogen? 

Dr. Wezel: Zum einen ist es wichtig zu sagen, dass Lettland diverse Abkommen unterzeichnet hat, z.B. in der Stockholmkonferenz 20006. Lettland hat bereits 1991 einen Gedenktag für die Ermordung der Juden eingerichtet. Er ist immer am 4. Juli, weil dieser Tag den Startschuss für die Entrechtung der Juden in Lettland war. Am 1. Juli sind die Nationalsozialisten in Riga einmarschiert und am 4. Juli haben die Synagogen gebrannt. Der Gedenktag ist deswegen ein vor allem lokal bedeutsamer Termin. Von Seiten der Politik gibt es eine Gedenkveranstaltung, aber es ist immer die Frage, inwiefern das den Großteil der Bevölkerung betrifft. Als ich die ersten Male in Lettland war und dann gefragt habe, wieso die lettischen Flaggen am 4. Juli mit Trauerflor sind, konnten vielleicht 80% der Leute das nicht sagen. Natürlich gibt es gut Informierte, aber das Gros der Bevölkerung weiß es nicht. Da kann man natürlich schon sagen, dass die Schulbildung offensichtlich ein bisschen versagt hat. 
Der Holocaust wird definitiv unterrichtet. Es gibt auch sehr gute Projekte wie das neue Haus von Jānis Lipke, der über 50 Juden gerettet hat. Das ist jetzt ein Museum, das gut angenommen wird. Es wirkt vielleicht erstmal ein bisschen merkwürdig, dass man das Thema aus dieser Sicht betrachtet, indem man sich denjenigen anschaut, der Juden gerettet hat, denn es gab nur eine Handvoll Menschen, die das gemacht haben. Aber trotzdem ist die Bildungsarbeit, die gerade dieses Haus leistet, sehr wichtig, weil sehr viele Schulgruppen hingehen und dann auch gefragt wird: „Was hättest du in der Situation gemacht?“ Man kann sich in diese Person Jānis Lipke sehr schön reinfinden und verstehen: Die Situation war sehr schwierig, und er hat trotzdem Juden gerettet. Wenn man dieses Programm durchlaufen hat, ist relativ klar, dass die meisten nicht so gehandelt haben, weil sie nicht diese Courage besaßen. Es gibt also aus meiner Sicht sehr gute Bildungsprogramme. Letztendlich hängt es aber sehr viel von der Eigeninitiative des einzelnen Lehrers ab, was er mit seiner Schulklasse macht. 
Was ich in Bezug auf die Wissenschaftler sagen kann: Es gab in den letzten Jahren zahlreiche Konferenzen, die das Thema Holocaust in Lettland bearbeitet haben, aber es ist letztendlich eine Handvoll von Wissenschaftlern, die sich wirklich damit beschäftigen. Das ist ein bisschen meine Kritik, wobei ich auch wenige Lösungsansätze habe, denn es ist sehr schwer, das in die Bevölkerung zu tragen, weil viele Letten der Meinung sind, „erstmal müssen wir darüber reden, was uns alles widerfahren ist, der Holocaust ist doch schon gut erforscht und die Juden sind doch überall präsent, aber von uns weiß keiner“. Es gibt auch die Ansicht, die ich teilweise verstehen kann, dass in Westeuropa die Verbrechen Stalins nicht sehr bekannt sind. 

Wie ist die Situation der Juden, die heute in Lettland leben? Wie viele von ihnen haben die lettische Staatsbürgerschaft und sind gut integriert? 

Dr. Wezel: Das Problem ist, dass es nur eine Handvoll Holocaust-Überlebende gab, die tatsächlich nach Tallinn, Riga oder in andere Städte zurückgekommen sind. Die jetzige jüdische Gemeinde in Riga und auch in anderen Städten speist sich hauptsächlich aus Zugezogenen, die aus anderen Teilen der ehemaligen Sowjetunion gekommen sind, größtenteils aus Russland. Sie werden primär nicht als Juden wahrgenommen, sondern als Russen, weil sie russisch sprechen, russisch assimiliert sind, auf russische Schulen gehen und so weiter. Sie haben dann die gleichen Probleme, die teilweise auch andere Russen haben, dadurch dass sie eben keine lettische Staatsbürgerschaft haben. Es hängt auch damit zusammen, dass die jüdische Bevölkerung häufig erst in den 80er Jahren zugewandert ist. Das heißt, sie leben häufig noch gar nicht so lang in Lettland. Diejenigen, die dort seit längerem leben, also vor allem die Nachkommen der historischen jüdischen Bevölkerung, haben natürlich auch alle Bürgerschafts- und Staatsbürgerschaftsrechte, weil das Staatsbürgerschaftsrecht Lettlands darauf fußt, dass man Bürger Lettlands in der Zwischenkriegszeit war. Alle, die vor 1940 Staatsbürger waren, haben 1990 automatisch die Staatsbürgerschaft gekriegt. Daher gilt die Staatsbürgerschaftsfrage nur für diejenigen, die als sowjetische Immigranten später dazugekommen sind. 

 Inwiefern gibt es heute in der Bevölkerung antisemitische Klischees? 

Dr. Wezel: Ich würde sagen, dass die meisten Letten sich mit dieser Frage nicht beschäftigen und auch überhaupt keine Meinung über Juden haben. Das große Problem, das es meiner Ansicht nach in Lettland gibt, ist die Schwierigkeit, Leute überhaupt für das Thema zu interessieren. Die jüdische Minderheit ist aus lettischer Sicht immer eine kleine Gruppe gewesen. Letztendlich ging es Jahrhunderte lang darum, sich mit den Deutschen und dann mit den Russen auseinanderzusetzen; die jüdische Minderheit ist irgendwie immer untergegangen. Die Feindbilder bauen sich also nicht primär gegen Juden auf. Das heißt nicht, dass es keinen Antisemitismus gibt; es gibt auf jeden Fall Nationalisten, die auch antisemitische Feindbilder integrieren, aber das ist nicht der Hauptfokus der Xenophobie in Lettland. 

 Inwiefern ist Antisemitismus in Lettland überhaupt vorhanden? Was müsste man ändern an Geschichtsaufarbeitung und allgemeinem Umgang mit dem Judentum in Lettland? 

Dr. Wezel: Also ich glaube, dass man mit dem Antisemitismus-Begriff nicht sehr weit kommt, weil der nicht wirklich geeignet ist, um diese sehr spezifische Konstellation, die in Lettland vorherrscht, zu erklären und um auch zu erklären, warum es den Holocaust in Lettland in dieser spezifischen Form gegeben hat. Zweitens denke ich, dass man in Ansätzen schon genau das Richtige macht, indem man versucht, die Bevölkerung für Sachen zu interessieren, die tatsächlich in der eigenen Stadt passiert sind. Zum Beispiel gibt es jetzt an der ehemaligen großen Synagoge in Riga ein Denkmal. Schülergruppen gehen da wirklich hin, wie auch zu solchen Museen wie dem Jānis-Lipke-Haus. Das ist aus meiner Sicht das Wichtigste, dass man das lokalgeschichtlich einbettet, weil ich glaube, dass das auch in Deutschland Sachen wie die Stolpersteine am ergiebigsten sind. Das gibt es in Riga jetzt auch verstärkt, aber noch nicht sehr weit verbreitet. Das wäre auf jeden Fall schön, wenn es noch mehr davon gäbe und wenn die Leute dann auch wüssten, was das eigentlich ist und was da dahintersteckt. Es geht darum, diese Leerstelle aufzuzeigen, dass die Juden einst einen wichtigen Teil der Bevölkerung ausgemacht haben und jetzt in dieser Form nicht mehr da sind, weil die Juden, die jetzt in Lettland leben, größtenteils keine Nachfahren von den im Holocaust Verfolgten sind. 