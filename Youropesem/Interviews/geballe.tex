 

Interview mit David Geballe 

David Geballe (* 1981) grew up in Hamburg. At the age of 16, he began to work as a youth leader with Jewish youth. He studied in Berlin, New York and Jerusalem, in 2006 he received the rabbinic dignity. Since 2011 he has worked as a rabbi in Germany, first in Munich and then in Fürth, where he was in charge of the Israelite religious community at the time of the interview. In addition to the rabbinical activity, he was involved among others in the Jewish fraternity of students (Jüdischer Studentenverbund Franken) and the board of the Society for Christian-Jewish Cooperation (Gesellschaft für für christlich-jüdische Zusammenarbeit). Since September 2017 he is in charge of the Jewish Community of Duisburg-Mülheim / Ruhr-Oberhausen as Chief Rabbi. 

The interview took place in Fürth in 2016. 

Könnten Sie sich zu Beginn kurz vorstellen? 

David Geballe: Mein Name ist David Geballe, geboren bin ich im fernen Hamburg, bin jetzt seit nicht ganz sechs Jahren Rabbiner hier in der Gemeinde und habe dementsprechend viel Kontakt mit den Gemeindemitgliedern, aber auch mit den anderen Gemeinden hier in der Umgebung, und dadurch auch einen relativ guten Einblick, wie diese momentan stehen. 

Wie sieht dieser Einblick denn aus? 

David Geballe: Dass es seit der Syrienkrise schwieriger geworden ist, dadurch, dass die hunderttausende Leute mit einem muslimischen Erziehungshintergrund, die nach Deutschland gekommen sind, von klein an durch die Eltern und Medien in den arabischsprachigen Ländern nicht gerade zu Judenliebe erzogen worden sind und ein ganz anderes Bild haben, was Juden oder Nicht-Muslime angeht. In den letzten vier Jahren sind, nach meiner Erfahrung und nach dem was man hört, die allermeisten Probleme aus diesem Klientel. Das soll nicht heißen, dass es unter Deutschen oder anderen Europäern keinen Antisemitismus gibt oder gegeben hat – den gibt es leider immer noch, aber der wird heutzutage als „Israelkritik“ verkleidet oder damit gerechtfertigt, dass man „so etwas unter Freunden ja noch sagen darf“. Muslimisch geprägter Antisemitismus ist oftmals noch direkter Antisemitismus oder teilweise noch Antijudaismus – was es in Europa in dieser Form und Ausprägung seit 150 Jahren nicht mehr gibt. Aber Antisemitismus bleibt Antisemitismus, egal, wie er sich verkleidet. 

 Gibt es mehrere Formen des Antisemitismus? Welche würden Sie da aufführen? 

David Geballe: Es gibt die drei Schulmeinungsansätze, angefangen vom Antijudaismus, der sich gegen die Religion selbst richtet und vor 150 bis 200 Jahren zum reinen Antisemitismus weiterentwickelt hat. Dieser wurde nicht primär auf die Religion, sondern auf das Volk bezogen. Nach dem Sechstagekrieg 1967 wurde das dann schleichend zu einem Antizionismus oder Antiisraelismus, bei dem die Israelis nicht mehr als Opfer des zweiten Weltkriegs, sondern als Täter angesehen wurden und damit das Feindbild waren. Das ist die Entwicklung der letzten Jahrzehnte. 

 Und glauben Sie, dass sich das in allen gesellschaftlichen Gruppen in Deutschland gleich entwickelt hat, oder gab es da Unterschiede? Sie haben ja zum Beispiel den muslimischen Hintergrund angesprochen. 

David Geballe: Klar, man kann die verschiedenen Gruppen kaum miteinander vergleichen. Jemand, der in einem muslimisch geprägten Land aufwächst, wo im Fernsehen offiziell Kinderserien laufen, in denen Juden als Nichtmenschen dargestellt werden - wenn dieses Kind einmal erwachsen wird, hat es gar nicht die freie Wahl, Juden nicht zu hassen. Es ist so in diese Person und ihre Psyche eingebaut, dass es schon wirklich einen sehr besonderen Menschen braucht, um diese Kette zu durchbrechen. 

 Meinen Sie, das kommt vor? 

David Geballe: Es gibt durchaus ein paar berühmte Beispiele. Zum Beispiel den Sohn von einem der Hamas-Führer, der diese Kette durchbrochen hat. Der tritt auch auf, um diese Meinung nach draußen zu bringen1. Dann gibt es auch in Deutschland einen, ich glaube er ist Historiker. Auf jeden Fall gibt es ein paar Leute, die diese Kette durchbrechen, das sind leider die Ausnahmen und nicht die Regel. Ich würde jetzt auch nicht unterstellen, dass automatisch jeder Muslim ein Antisemit sei, um Gottes Willen. Aber es sind bestimmte Vorurteile, die vor und insbesondere nach 1967 durch diese Kultur geprägt wurden, mit den “drei Neins” gegen Israel, kein Frieden, keine Anerkennung, keine Verhandlungen. Das hat sich zu einem politisch geprägten Antisemitismus entwickelt. 

Und abgesehen von den Leuten, die aus muslimisch geprägten Ländern kommen - wie hat sich das nach Ihrer Einschätzung in Deutschland nach dem Krieg entwickelt? 

David Geballe: Es gab hier in Fürth und auch anderswo Fälle, in denen man auch nach dem Krieg Leute mit dem offiziellen Persilschein angepöbelt hat. Der Persilschein war auf dem Papier und nicht in den Köpfen der Leute. Vorstände der jüdischen Gemeinde durften sich dann Dinge anhören wie "Schade, dass wir dich nicht auch noch bekommen haben" und so etwas. Klar, das gab es immer nach dem Krieg, durch die ganze Geschichte durch, teilweise auch in der Justiz, weil die hohen Beamten irgendwo herkommen mussten. Es war so, dass Altbeamte mit übernommen wurden, die auch in der Nazizeit eine nicht unwichtige Rolle gespielt haben. 

Glauben Sie, dass diese Übernahmen tatsächlich notwendig waren? 

David Geballe: Klar ist, dass Anfang der Fünfziger einfach Polizisten, Richter und sowas gebraucht wurden. In der DDR war das anders, weil dort Leute aus der Sowjetunion hingeschickt und eingesetzt wurden. Inzwischen gibt es historische Nachforschungen, laut denen auch dort nicht alle Beamten und Parteifunktionäre während der Nazizeit eine wirklich weiße Weste hatten. Aber es war in beiden Deutschlands eigentlich so, dass Leute mit übernommen wurden. Mussten sie, mussten sie nicht? Hinterher ist man immer klüger, aber bei manchen stellt sich die Frage, hätten die wirklich in so eine wichtige Position kommen sollen oder nicht? 

Und diese Anpöbelungen gegen Vorstände der jüdischen Gemeinde, war das in der unmittelbaren Nachkriegszeit oder zieht sich das bis heute hin? 

David Geballe: In den Fünfziger und Sechziger Jahren gab es das alles, teilweise auch heute noch. Natürlich sind die ganzen Altnazis inzwischen nicht mehr in der Lage, so zu pöbeln. Wenn sie noch am Leben sind, sind sie in Alters- oder Pflegeheimen. Aber es vergeht keine Woche, in der jüdische Gemeinden in Deutschland nicht mit Drohbriefen – sowohl mit als auch ohne Namen – angeschrieben werden, die teilweise sehr deutlich bestimmte Gesetze brechen und nicht wirklich angenehm zu lesen sind. Zum Beispiel, als die sogenannte Beschneidungsdebatte war, war es sehr, sehr schlimm, da haben Leute mit hohem Bildungsgrad – Doktoren, Rechtsanwälte, teilweise auch Lehrer und Direktoren an Schulen – Briefe geschrieben die unterhalb jeder Gürtellinie sind. Diese Leute hatten auch keine Probleme, ihre Namen, Adressen und Titel in voller Gänze auf ihre Briefe zu schreiben. Nicht so wie früher, wo so etwas anonym und aus Zeitungsschnipseln gemacht war, das ist inzwischen selten geworden. Heute ist es so, dass die meisten Briefe anfangen mit "Unter Freunden wird man das ja wohl noch mal sagen dürfen..." oder „Das sagen wir nur, weil wir ja euch so mögen“ und so etwas wie „Aber was ihr mit den armen Palästinensern oder mit euren armen Kindern macht, das darf doch wohl nicht sein“. Also wenn ein Brief so anfängt, dann ist man sicher, entweder unten oder oben stehen der volle Name und die Adresse dieser Person. Es hat sich ein bisschen eingebürgert: Wenn man diesen magischen Satz sagt, "unter Freunden wird man das ja noch sagen dürfen", dann ist alles, was danach kommt, nicht mehr antisemitisch, sondern nur freundlich gemeint. 
Ich denke, es ist so, dass Antisemitismus ein Krebs ist, der niemals ganz geheilt werden kann. Es wird immer Antisemitismus geben, es hat ihn auch schon immer gegeben. Die Frage ist nur, ist es gesellschaftsfähig oder nicht? Während der Nazizeit war er sogar Staatsraison, was er Gott sei Dank seitdem nicht mehr ist, aber heute wird keiner mehr öffentlich aufstehen und sagen "I)hr Juden seid so und so", sondern es ist eher Israel, das man angreifen kann. Genauso wie jetzt Anfang des Jahres, wo ein deutsches Gericht entschieden hat, dass ein versuchter Brandanschlag auf eine Synagoge keine antisemitische Straftat, sondern nur eine politische Aussage gegen den Staat Israel im Gaza-Krieg ist, was für mich eine absolute Schweinerei war. Was soll man dazu sagen? Natürlich fällt diese Straftat dann nicht in die Kategorie von Antisemitismus und dadurch kann man auch diese ganzen Listen und Szenen von antisemitischen Straftaten klein halten. 

Gab es in den letzten Jahren auch Antisemitismus in Fürth, in Bezug auf den Gaza-Krieg zum Beispiel? 

David Geballe: Zum Glück gab es in den letzten Jahren keinen gewalttätigen Antisemitismus in Fürth, aber mehr oder weniger verdeckten, in Form von Briefen, Anrufen oder E-Mails. Neulich erst kam eine E-Mail, die an mehrere Rabbiner und Gemeindevorstände und berühmte Juden gerichtet war und diese aufgefordert hat, zu einem Kriegstribunal vorstellig zu werden wegen Verbrechen gegen die Menschlichkeit und Sonstiges. Diese stammte von Leuten die nicht dem rechten Spektrum zuzuordnen sind, sondern eher dem linken. 

Nochmal zu den Medien - Sie bekommen also Briefe, E-Mails, Anrufe? 

David Geballe: Alle Möglichkeiten der Telekommunikation werden da ausgeschöpft. 

Ich nehme mal an, in den sozialen Netzwerken ist der Antisemitismus weniger verkleidet oder versteckt, weil er dort anonym geäußert werden kann. 

David Geballe: Man braucht nur auf YouTube gehen und irgendein Video, das mit Judentum zu tun hat, anzuklicken, und dann sieht man gleich zweierlei Arten von Kommentaren. Einmal die muslimisch geprägten, die dementsprechend anfangen, und dann die nichtmuslimischen, die dann mit Beschneidung oder sonst etwas anfangen. So etwas wird man überall finden, auf Facebook muss ich gar nicht eingehen. Als vor etwa drei Jahren die Beschneidungsdebatte war, war es dort sehr, sehr schlimm. Sobald irgendein Bericht in der Zeitung erschien, der auch online zu lesen war, gab es darauf natürlich Kommentare. Es war teilweise nicht mehr lustig, das mitzulesen. Von irgendwelchen Pädophilie-Vorwürfen gegenüber allen Juden und so weiter. 

Wie sieht es mit antisemitischen Äußerungen von einer eher esoterisch-tierschützenden Bewegung zum Schächten aus, zum Beispiel? 

David Geballe: Gibt es auch, zum Glück sind die in Deutschland noch nicht so stark. In anderen Ländern, z. B. in der Schweiz, ist das mit dem Schächten verboten, was auch zum großen Teil vom Tierschutz mitgetragen wurde. Das Interessante ist, wenn man einmal mit diesen Leuten spricht und sie nach wissenschaftlichen Beweisen fragt, dann kommt eigentlich nur warme Luft zurück. Also, es ist nur ein Nebenthema, diese klassischen Angriffe gegen das Judentum sind genauso substanzlos wie sie immer waren. Das Krasseste, was ich einmal erlebt habe, ist, dass ein Lehrer mich bei einer Schulführung mit einer katholischen Religionsklasse gefragt hat, was für ein Blut heutzutage für Matzen benutzt wird. Am Anfang dachte ich, es wäre ein schlechter Scherz, aber das war es leider nicht. 

Ich bin Historiker, ich habe ein Buch über Antisemitismus in Franken während der Weimarer Republik geschrieben und dabei ist mir aufgefallen, dass es in Fürth einen ziemlich starken zivilgesellschaftlichen Widerstand gegen die Nazis und gegen den Antisemitismus gab – ganz im Gegensatz zu Nürnberg, wo sich die Arbeiterklasse zurückgezogen und nichts dagegen gemacht hat. Ist jetzt noch etwas in Fürth zu merken von einem zivilgesellschaftlichen Widerstand gegen Rechte Gruppierungen? 

David Geballe: Jein. Bei den Fürthern, die wirklich noch ein bisschen die Geschichte kennen, ist es immer noch ein bisschen im Kopf drinnen, dass Fürth eine Art fränkisches Jerusalem ist und so etwas. Das ist durchaus bewusst, aber weil die Gesellschaft heute allgemein viel schneller umzieht, stellt sich die Frage, wie viele von den in Fürth lebenden Menschen wirklich Fürther oder Franken in diesem Sinne sind. Ich meine, ich bin selbst keiner, und wird eine Person, die aus Berlin, aus Stuttgart oder ich weiß nicht was herkommt, wirklich wissen, was Fürth für eine Geschichte hat? Vielleicht nicht, vielleicht ja, vielleicht schafft sie es mal, hier ins jüdische Museum oder so etwas, aber das war es auch schon. Das Wissen ist in dem Sinne heute nicht mehr so stark ausgeprägt, wie es früher mal einmal war. Man muss natürlich sagen, dass die Stadt an sich, also beispielsweise der Oberbürgermeister, sich dessen voll und ganz bewusst sind. Es gibt daher auch ein sehr gutes Verhältnis zwischen der Stadt und der Gemeinde, in beide Richtungen. Die Stadt hilft uns, wo immer sie kann, wir helfen der Stadt, wo immer wir können. Das ist schon da, auf jeden Fall gibt es also Überreste davon. 

Können Sie sich bei der aktuellen Erweiterung des Jüdischen Museums Franken in Fürth einbringen? 

David Geballe: Das Museum hat ja einen Trägerverein und auch eine wirkliche wissenschaftliche Leitung, die dort natürlich hauptsächlich dort bestimmt. Natürlich gibt es auch eine Zusammenarbeit zwischen dem Museum und der Gemeinde, aber Museum und Gemeinde sind getrennt. Ich bin mir ziemlich sicher, wenn ich einen Vorschlag machen würde, würde man darüber nachdenken und darüber sprechen, aber im Großen und Ganzen bin ich mit dem alten Teil zufrieden. Ich denke daher, dass auch der neue Teil gut ausgestattet werden wird. 

Was halten Sie generell von der Bildung, die in Deutschland über das Judentum vermittelt wird, also in den Schulen oder auch anderswo? 

David Geballe: Ich erweitere das mal ein bisschen mit der Bildung über den Zweiten Weltkrieg. Auf der einen Seite zu viel, auf der anderen Seite zu schlecht. Das soll heißen, dass die meisten Schüler, wenn wieder das Thema Holocaust oder Zweiter Weltkrieg in der Schule hochkommt, sich denken „ah, nicht schon wieder“. Das ist natürlich genau das, was man nicht erreichen möchte. Und an einer Schule, an der 80 % der Klasse aus Muslimen besteht, muss man natürlich ganz anders unterrichten. Es gibt Berichte, ich habe von einer Lehrerin aus Berlin gelesen, am deren Schule wirklich 90% der Schüler*innen Muslime sind. Als dann Fotos von KZs gezeigt wurden, haben die angefangen zu klatschen. Das ist natürlich genau das Gegenteil von dem, was man damit erreichen möchte. Von daher denke ich, es wäre sinnvoll, sich neue Konzepte zu erarbeiten, die dann für die Lehrer quasi verpflichtend sind. Vielleicht nicht ganz so oft, aber dafür wirklich besser. 

 Was würden Sie sich da vorstellen? 

David Geballe: Natürlich kann man eine Klasse von einem Gymnasium in Bayern auf dem Land nicht mit einer Gesamtschule in Berlin vergleichen. Das Klientel ist ein ganz anderes, das muss natürlich auch dementsprechend angepasst sein. Es gibt keine magische Lösung, die immer funktioniert, aber es gibt Gott sei Dank genügend Leute, die von der Erarbeitung von Konzepten und Bildungsmaterialien mehr Ahnung haben als ich. Wie das genau aussieht, weiß ich nicht, darüber habe ich noch nicht wirklich nachgedacht, aber dass Handlungsbedarf besteht, da bin ich mir ziemlich sicher. 

 Ich war ein paarmal bei den Wochen der Brüderlichkeit dabei und mein persönlicher Eindruck war, dass das Verhältnis zwischen deutschen Christen und deutschen Juden etwas ritualisiert und bemüht ist. Ist dieser Eindruck richtig? 

David Geballe: Die Woche der Brüderlichkeit an sich ist etwas sehr Gutes, sehr Tolles, es ist ja auch geschichtlich gesehen eine Abstammung vom reeducation programme der Amerikaner. Das Problem ist, wenn man sich jede Veranstaltung in der Woche der Brüderlichkeit anschaut und das Durchschnittsalter der Zuhörer oder Mitwirkenden berechnet, wird man jenseits des Rentenalters sein. Immer wenn ich eine Schulklassenführung in der Woche der Brüderlichkeit habe, frage ich die Schüler*innen, ob sie wissen, was für eine Woche gerade ist. Ich habe einmal bisher eine Antwort bekommen, aber auch nur, weil der Vater von einer der Schülerinnen evangelischer Pfarrer und deswegen bemüht ist. Deswegen wusste die Tochter zufällig, was das ist, aber sonst kamen dann Fragen zurück, vielleicht Champions League-Woche oder so etwas. Obwohl es doch eigentlich prädestiniert wäre, in den Schulen dazu etwas zu machen, passiert das in den allermeisten Fällen nicht. Es gibt hier in Fürth zum Glück eine sehr gute Ausnahme, das Helene-Lange-Gymnasium, wo es schon wirklich Tradition ist, am Donnerstag in der Woche der Brüderlichkeit eine Veranstaltung zu machen, die auch für die höheren Klassen verpflichtend ist. Das vordere Viertel der Stuhlreihen ist für Ehrengäste reserviert, aber der Rest sind wirklich Schüler. Ich finde, dazu ist die Woche der Brüderlichkeit da und nicht dieses gegenseitige „piep, piep, piep, wir haben uns alle lieb“, das ist etwas stilisiert und schon fast eingeübt und nicht wirklich der Sinn und Zweck des Ganzen. 

Was genau wird am Helene-Lange-Gymnasium an diesem Tag gemacht? 

David Geballe: Es gibt jedes Jahr einen Hauptsprecher von verschiedenen Quellen, vor ein paar Jahren war z. B. jemand vom israelischen Konsulat in München hier und hat eine Rede gehalten. Dieses Jahr war es ein wissenschaftlicher Mitarbeiter der Uni Frankfurt, der über die deutschen Nazi-Prozesse berichtet hat. Es gibt verschiedene Vorträge, die ein Thema wirklich vertiefen anstatt oberflächlich zu bleiben, also wirklich ein reeducation programme. Nicht einfach Larifari, wir haben uns alle lieb, sondern wirklich Wissen, das vermittelt wird. Gott sei Dank ist Dummheit eine Krankheit, die geheilt werden kann. Ich bin der festen Überzeugung, dass Vorurteile nur dort wachsen können, wo Wissen fehlt. Das macht es nur umso schlimmer, dass die Woche der Brüderlichkeit nicht viel mehr in den Schulen thematisiert wird. 

Und was bieten Sie persönlich in der Gemeinde an? Führungen für Schulklassen haben sie schon erwähnt, gibt es noch andere Sachen? 

David Geballe: m Durchschnitt mache ich zwei Führungen in der Woche. Meistens von Neuntklässlern, weil in der neunten Klasse andere Religionen sowie Traditions- und Toleranzbegriffe im Lehrplan stehen. Da sind eigentlich die meisten Führungen, es gibt auch ein paar andere. Manche machen es schon in der vierten Klasse, obwohl ich das persönlich nicht so gerne mache, wegen des Alters und der Aufmerksamkeitsspanne. Wenn ich wirklich jede Anfrage annehmen würde, was rein zeitlich leider gar nicht mehr geht, würde ich manche Wochen mit zehn Führungen haben, aber es gibt Gott sei Dank auch genügend andere Dinge zu tun in einer Gemeinde, für die ebenfalls gesorgt werden muss. 

Wie sind Ihre Erfahrungen aus den Führungen, sind die Schüler interessiert oder muss man sie wirklich dafür begeistern? 

David Geballe: Es kommt darauf an. Meiner Erfahrung nach hängt es davon ab, wie gut die Kinder darauf vorbereitet wurden. Kinder in der neunten Klasse sind ja Jugendliche, meistens so um die 15 Jahre alt. Was ich eigentlich allen Lehrern mitgebe, wenn sie nach einer Führung fragen, ist „ja, aber nur wenn Fragen auch im Unterricht vorbereitet werden“, ich muss hier ja keinen Frontalunterricht durchführen. Dafür ist mir meine Zeit viel zu schade. Wenn die Schüler einfach nur die Augen verdrehen und sich fragen, wann es endlich zu Ende ist, dann ist das eine Zeitverschwendung für sie und für mich. Das heißt, ich will diese Stunde sinnvoll nutzen und das funktioniert meistens dann gut, wenn die Schüler gut vorbereitet wurden. 

Welche Fragen stellen Schüler Ihnen, wenn sie welche vorbereitet haben? 

David Geballe: Es gibt Verschiedenstes. Also es gibt Fragen, die sich aus dem Unterricht ergeben haben, auf die der Lehrer oder die Lehrerin keine Antwort wussten. Meistens gebe ich auch eine Einführung in das Judentum. Ich habe es leider oft gesehen, dass Lehrer bei dem Thema eine kleine Nachbildung gebrauchen könnten. Es kommt nicht nur vor, dass Informationen fehlen, sondern auch, dass teilweise wirklich falsche Dinge beigebracht wurden. Ich erläutere die wichtigsten Dinge im Judentum, wie eine Kurzfassung, und dann bleiben wir bei ein paar Themen stehen, die für Jugendliche vielleicht einen Anreiz haben oder sie vielleicht selbst bewegen. Dabei entwickeln sich meistens auch sehr viele Fragen. 

Das kann ich nur bestätigen, ich war ein halbes Jahr in der Lehrerfortbildung bei Berufsschullehrer*innen tätig. Das war mein härtester Job, weil alles, was Lehrer nicht wollen, ist belehrt zu werden. Und das Wissen war wirklich gleich Null, also erschreckend schwach. Haben Sie das bei Religions- oder Geschichtslehrern erlebt? 

David Geballe: Also die meisten, die kommen, sind Religionslehrer, weil es dort im Lehrplan verankert ist. Geschichtslehre hatte ich noch nicht, wenn ich mich nicht täusche. Es gab mal einen Philosophiekurs, aber Geschichte noch nicht. 

 Und was ist mit anderen Gruppen, also Nicht-Schüler? 

David Geballe: Sind meistens das andere Spektrum, das heißt Rentengruppen, meistens überwiegend von Kirchen oder so etwas. Ich hatte auch schon einmal den Rentnerclub der Siemenswerke in Erlangen. Bis dahin wusste ich nicht, dass es so etwas gibt, aber das ist ein Club von ehemaligen Siemens-Angestellten, die verschiedene Ausflüge anbieten. 

Wie sind da Ihre Erfahrungen? 

David Geballe: Die Erwachsenengruppen sind meist doch etwas fragaktiver, um es so auszudrücken. Es gibt aber auch dort Leute, die – wie sie eben gesagt haben – nicht belehrt werden wollen. Es gibt auch einige, die mit einer festen gebildeten Meinung über Juden oder das Judentum herkommen und diese eigentlich nur bestätigt haben wollen. Ich hatte es schon ein- oder zweimal, dass diese wirklich ausfallend wurden und meinten, ich solle endlich mal die Wahrheit erzählen und nicht irgendetwas Geschöntes. 

Wie reagieren Sie in so einer Situation? 

David Geballe: Die Organisation muss angepasst werden. Man muss schauen, wie der Rest der Gruppe reagiert. Wenn der Rest der Gruppe nickend zustimmt, ist es etwas anderes als wenn die Gruppe diese Person komisch anschaut und fragt, was da jetzt los ist. Natürlich reagiere ich in beiden Fällen anders. 
In der beschriebenen Situation habe ich, wenn zeitlich die Möglichkeit bestand, quasi jede einzelne Aussage dieser Person auseinandergenommen und gezeigt, wie schwach das doch eigentlich ist. Zum Beispiel, dass Matzen eigentlich aus Blut gemacht und nur heutzutage, weil es verboten ist, aus Getreide hergestellt werden. Das war dann auch wirklich einfach: „Gibt es hier einen Chemiker?“. Zum Glück war jemand da, der sogar Lebensmittelchemiker war von Beruf. Den habe ich gefragt, ob er einen Prozess kennt, mit dem man Blut in eine weiße Masse transformieren kann, die auch noch essbar ist. „Nein, kenne ich nicht“ – und wenn selbst der Lebensmittelchemiker das nicht weiß… 

Das was eine Anspielung auf die Ritualmordlegende? 

David Geballe: Ja, das gibt es Verschiedenes. Zum Beispiel noch, dass Frauen im Judentum unterdrückt oder schwach seien, viele absurde Schwachsinnigkeiten. Falsches oder schlechtes Wissen gibt es leider auch sehr häufig. Hier in Fürth gibt es ein Weiterbildungszentrum des Deutschen Zolls, und die bieten in den verschiedenen Bereichen des Zolls, die mit anderen Leuten zu tun haben, Kurse an. Da kann man sich z. B. mal mit Amerikanern austauschen oder so etwas. Und jetzt hat es sich seit etwa vier Jahren eingebürgert, dass auch mal Gruppen des Deutschen Zolls in die Synagoge kommen und hier zollrelevante Dinge besprochen werden, zusammen mit einer Einführung in das Judentum. Das ist die dritte große Gruppe. Für eine Schulgruppe nehme ich ungefähr eine Stunde Zeit, für die Zollgruppen sind es eigentlich immer zwei oder zweieinhalb Stunden, weil meistens so viele Fragen kommen und so viel Praktisches für den Zoll-Alltag, was die Leute besprechen wollen. Sie erzählen mir z. B. von einer Situation, die aus dem Ruder gelaufen ist und fragen, was sie anders hätten machen sollen oder was besser gewesen wäre. Das sind auch meistens die interessantesten Führungen für mich persönlich. 

 Welche praktische Bedeutung hat das für den Zoll? 

David Geballe: Ein klassischer Fall wäre, dass diese Leute z. B. in München oder Frankfurt am Flughafen beschäftigt sind und eine Maschine mit israelischen Fluggästen reinkommt, die einer Routineinspektion vom Zoll unterzogen werden soll, jeder kennt das. Und es kann mal passieren, dass dann einer der Reisenden etwas forsch antwortet, zum Beispiel mit „du durchsuchst mich doch nur, weil ich Jude bin, bist du etwa ein Nazi?“. Solche Sprüche, die völlig aus dem Nichts hergeholt sind, aber leider von manchen benutzt werden, nach dem Motto „Angriff ist die beste Verteidigung, und wenn ich ihn jetzt als Nazi abstempele, wird er sich nicht mehr trauen, mich zu durchsuchen“. Natürlich ist das nicht der Fall. Man kann sich damit auch eine Anzeige wegen Beleidigung einheimsen, und es gibt Tipps, wie man als Zöllner mit so einer Situation umgeht und was man sagen oder tun kann, um sie ein wenig zu entschärfen. 

 Interessant. Noch einmal zu den ganzen antisemitischen Angriffen, die Sie bekommen: setzen Sie sich mit denen auseinander oder versuchen Sie lieber, sie zu ignorieren? 

David Geballe: Je nachdem. Ich beschäftige mich mit allen Briefen, E-Mails, etc., die reingehen, und das nicht nur theoretisch besprechen, sondern z. B. sagen „Was macht ihr, ich werde euch dafür büßen lassen“. Solche Dinge werden natürlich an den Staatsschutz weitergeleitet, um akute Gefahren erkennen zu können und damit etwas dagegen getan werden kann. Gerade da muss man ganz klar unterscheiden zwischen denen, die vergleichsweise harmlos sind, mit theoretischen Drohungen, und denen, die wirklich Gewalttaten oder sonstiges androhen. 

Archivieren Sie das selbst, oder leiten Sie nur bestimmte Sachen weiter? 

David Geballe: Bestimmte Sachen werden weitergeleitet. Es gibt auch welche, die schon dermaßen außerhalb der Realität sind, dass sie eigentlich schon fast wieder komisch sind. Manche meiner Kollegen sammeln die einfach nur, damit sie nur ihre Mappe aufschlagen müssen, um wirklich gut darüber lachen zu können. Da denkt man sich manchmal, der Verfasser gehört eigentlich in die Klapse. Klar, zum Teil landet so etwas auch einfach gleich im Mülleimer. 

Haben Sie eigentlich mitbekommen, dass es vor sechs, sieben Jahren wieder eine antijudaistische Prozession nach Heiligenblut am Brombachsee gab? 
In Heiligenblut, so die Legende, hat ein böser Jude einen armen Bauern angestiftet, eine Hostie zu klauen. Der böse Jude hat dann hineingestochen und die Hostie fing natürlich wieder zu bluten an, weil die Hostie ja der Jesus ist. Der böse Jude hat den Jesus also ein zweites Mal erstochen. Dann wollte der Jude seiner Strafe entgehen und konvertieren, in der Kirche von Spalt hat ihn dann aber bei der Konvertierung der Blitz erschlagen. Daraufhin ist Heiligenblut erbaut worden, das war einmal ein Kloster und jetzt ist es noch eine Gedenkstätte. Die Kolping-Gemeinde von Spalt hat vor etwa sieben Jahren diese antijudaistische Prozession, die schon 50 Jahre lang aufgehört hatte, erneut ins Leben gerufen. Da sind dann ein paar hundert Leute mit dem Boot nach Heiligenblut gefahren. Ich bin mit der Prozession mitgelaufen, habe dann einen Artikel darüber auf HaGalil geschrieben und da gab es eine große Diskussion. Seit der Artikel erschienen ist, ist auch die Prozession wieder gestorben. Da gab es ja noch eine ganz interessante Debatte, weil mir der Pfarrer von Spalt (ein Philosemit, der eine Mesusa an seiner Tür hat und diese zweimal täglich küsst als Christ) gesagt hat, ich sei ein Antisemit, weil ich diese Prozession veranstaltet habe. Da kann man gut nachvollziehen, dass Philosemiten, wenn sie von den Juden nicht geliebt werden, im Nu zu Antisemiten werden. Und der Pfarrer dachte, ich sei Jude und habe ihn kritisiert. Ist das auch Ihre Einschätzung?  

David Geballe: Es gibt diesen Typus von Menschen durchaus. Es ist eigentlich so, dass diejenigen, die am lautesten schreien, dass sie keine Hilfe brauchen. diejenigen sind, die am meisten Hilfe benötigen. Genauso gibt es diejenigen, die sagen, sie lieben alle Juden, nur um das zu übertünchen, was sie im Inneren wirklich fühlen. Da kann man dann beobachten, wie das sehr schnell ins Gegenteil übergeht. Das beste Beispiel ist Martin Luther. 

Wie finden Sie jetzt im Lutherjahr den Umgang mit Martin Luther? 

David Geballe: Es gibt durchaus gute und ehrliche Ansätze, aber natürlich auch welche, die verschönern. Gute Ansätze gibt es z. B. beim evangelischen Bildungswerk hier in Fürth, das eine Veranstaltung mit dem Thema "Luther als Antisemit” gemacht hat. Die spricht das wirklich aus und erklärt dann im Vortrag, wie es ist und was man nicht schönreden kann, und dass Luther nun mal nicht unfehlbar war. Dann gibt es andere Artikel, die das relativieren und entweder sagen, dass es nicht so gemeint war oder dass man es im Zusammenhang sehen muss und Luther kein Antisemit war. 

 Wie gehen Sie innerhalb Ihrer, Ihrer Religion mit Reformation um, z. B. dass Frauen Rabbinerinnen sind, so wie Frau Antje Yael Deusel aus Bamberg? 

David Geballe: Es ist ein kompliziertes Thema in einem Sinne, aber auf der anderen Seite auch ein sehr einfaches. Es gibt hunderte von Richtungen im Judentum, von Reform, liberal, orthodox, ultra-orthodox, ultra-ultra-orthodox, neo-orthodox, man kann damit nicht mehr wirklich mitkommen. Eigentlich aus der jüdischen Sicht völliger Schwachsinn. Es gibt eigentlich eine einzige Frage, die in dem Sinne wichtig ist, und zwar ob man sagt, dass das Religionsgesetz bindend ist. 
Wenn man sagt, das Religionsgesetz ist bindend, dann ist der Rest nur Kosmetik. Wenn man sagt, dass Religionsgesetz ist nicht bindend, dann ist es ganz egal, aus was für philosophischen, theologischen Gründen man das tut. Oder als Ableger davon, wenn man sagt, das Religionsgesetz ist bindend, aber man kann selbst bestimmen, was das Religionsgesetz sagt, dann ist das natürlich am Ende nur eine Farce und man bindet sich nicht daran. Es gibt bestimmte religionsgesetzliche Gründe, warum eine Frau keine Rabbinerin sein kann. Dementsprechend sehen das auch diejenigen, für die das Religionsgesetz bindend ist. 

 Wie gehen Sie dann mit der Bamberger Gemeinde um? Die hatte sich ja eine Zeitlang dafür entschieden, aber ich glaube, sie wurde jetzt abgesetzt. 

David Geballe: Sie wurde vor ungefähr eineinhalb Jahren entlassen. Es gab, soweit ich weiß, einfach Unstimmigkeiten zwischen dem Vorstand der Gemeinde und Frau Deusel. Interna kenne ich nicht, aber sie wurde nicht gefeuert, weil sie eine Rabbinerin war oder ist. 

Zur Struktur Ihrer Gemeinde: Es sind wahrscheinlich meistens Juden aus Osteuropa, schätze ich? 

David Geballe: Es sind zum großen Teil sogenannte Kontingentflüchtlinge aus der ehemaligen Sowjetunion gewesen, die sind aber meistens schon in den Neunziger Jahren gekommen. Also sind es auch schon über zwanzig Jahre, die die meisten hier sind. Klar, bei denen, die schon in einem gewissen Alter hergekommen sind, war die Integration nicht ganz so erfolgreich, aber die jüngeren Leute sind voll und ganz integriert. 

Haben Sie eigentlich Kontakt zur Chabad-Gemeinde in Nürnberg? 

David Geballe: Die Gemeinde Fürth ist Körperschaft des öffentlichen Rechtes, das heißt völlig autark. Es gibt zwischen Nürnberg und Fürth eine U-Bahn-Station, die heißt Stadtgrenze, und die heißt nicht einfach so Stadtgrenze, das heißt genauso wie die Jüdische Gemeinde in Nürnberg autark ist, sind auch alle anderen Gruppen autark. Natürlich gibt es eine Zusammenarbeit zwischen einzelnen Gemeinden, aber technisch gesehen ist Chabad Nürnberg ein Verein, wenn man so möchte. Wir machen nichts zusammen, aber es ist nicht so, weil wir sagen „oh mein Gott, die mögen wir nicht“, sondern es ist einfach getrennt. 

Arbeiten Sie mit der Israelitischen Kultusgemeinde in Nürnberg oder anderen Gemeinden in der Region zusammen? 

David Geballe: Dort, wo es Sinn macht, z. B. in der Arbeit mit jungen Leuten. Viele Studenten ziehen von außerhalb her und werden in den paar Studienjahren nicht hier Gemeindemitglied, sondern nehmen das Beste aus allen Welten mit, indem sie ausnutzen, dass es hier im Umkreis von fünfzehn Kilometern drei Gemeinden gibt, in Fürth, Nürnberg und Erlangen. 

Werden die antisemitischen Angriffe, die sie miterleben, gegen Gemeinde insgesamt gerichtet, oder gibt es auch persönliche Angriffe gegen ihre Gemeindemitglieder? 

David Geballe: Briefe sind mir nicht bekannt. Es gibt sozusagen Spontan-Antisemitismus, das heißt man man sieht z. B. einen Juden auf der Straße und sagt ihm etwas oder spuckt ihm vor die Füße oder sowas in der Richtung. Das ist aber nicht organisiert. Vor zwei Jahren, als der Gaza-Krieg wieder sehr heiß war, war es zum Beispiel so, dass Leute mich gefragt haben, ob sie lieber vor und nach der Synagoge die Kippa abnehmen sollen. Die hatten wirklich Angst, dass etwas passiert – Gott sei Dank passierte nichts. 

 Wie sieht es mit den ultraorthodoxen Juden aus, also mit denen, die Schläfenlocken tragen? 

David Geballe: Davon gibt es ja nicht viele in Deutschland. Es gibt durchaus ein paar, aber auch da kann man es so machen, dass nicht gleich offensichtlich ist, dass man jüdisch ist. Aus Sicherheitsgründen ist das vielleicht nicht die schlechteste aller Ideen. 

Haben Sie das Ihren Mitgliedern geraten? 

David Geballe: Ja. Wir haben anfangs darüber gesprochen, bei den hunderttausenden Muslimen, die nach Deutschland gekommen sind in den letzten zwei Jahren. Sicherlich würde keiner bei Verstand sagen, dass alle von denen gewalttätige Antisemiten sind oder sonstiges, um Gottes Willen. Selbst wenn man aber sagt, dass nur ein Prozent oder 0,1 Prozent davon gewalttätige Antisemiten sein würden, sind das bei fast einer Million, die ins Land gekommen sind, immer noch sehr viele. Und es braucht nicht viele, um etwas zu tun. Es braucht nur ein oder zwei Personen, die etwas machen wollen, und dann ist es getan. 

 Würden Sie den spontanen Antisemitismus nur Leuten, die als Flüchtlinge gekommen sind, zuordnen? 

David Geballe: Nein. Man muss nur auf den Fußballplatz gehen, wo „du Jude“ eine typische Beleidigung ist. Das ist heute leider in der Jugendkultur und teilweise auch im Sport sehr tief verankert. Die Schule ist nun mal ein Schmelztiegel, in dem viele Dinge zusammenkommen. 

Wissen Sie etwas von Initiativen oder Zusammenarbeit speziell mit Sportvereinen, um den Antisemitismus der Fans zu bekämpfen? 

David Geballe: Berühmt ist in größeren jüdischen Gemeinden der sogenannte Sportverein Maccabi, der vor allem im Fußball immer große Probleme hat. Egal, in welcher Liga sie spielen, es kommt oft vor, dass im gegnerischen Team z. B. sehr viele Muslime sind und die Fans, also deren Familie, Freunde, etc. teilweise negativ auffallen. Das hat auch schon zu Sperren und Strafen geführt. Es gibt in dieser Hinsicht auch Versuche, das ein bisschen einzuschränken und dagegen zu wirken, aber Fußball ist nun mal der Breitensport schlechthin und nicht in jedem kleinen Dorf gibt es eine jüdische Gemeinde, die das Know-How, die Zeit und auch die Manpower dazu hätte, da wirklich groß einzugreifen. 

Machen die Spielvereinigung in Fürth oder der Fußballclub Nürnberg da etwas? 

David Geballe: Fürth war schon immer als judenfreundliche Stadt bekannt, und das gilt auch heute noch. Vor dem Krieg waren viele Spieler von Greuther Fürth jüdisch. Zur Veranstaltung am 9. November sind bei uns auch immer ein paar Spieler von Greuther Fürth dabei. Selbst wenn sie keine offizielle Ansprache halten, sind sie trotzdem da, um ein Zeichen zu setzen. Also da gibt es durchaus Vereine, die auch dagegen aktiv sind. 

 

 