Interview with Sonia Ruszkowska

Sonia Ruszkowska is an educator at the Museum of the History of Polish Jews POLIN since 2013 where she is responsible for the educational programme of the Museum directed at schools. She participated in the educational program of the main exhibition from the beginning. She is specialised in drama theatre workshops and anti-discriminatory education with a special focus on civil rights and the fight against Anti-Semitism. Her educational programmes are mainly aimed at children, youth and teachers. She studied philosophy, and wrote her PhD about the Holocaust, the death in the gas chambers and the possibility of bringing back subjectivity for victims of the mass deaths. Before she started her work at the Museum, she worked as a schoolteacher for philosophy and ethics and participated in a foundation called ‘Forum for Dialogue’, that projects with school children in different towns and cities in Poland, raising awareness for the Jewish history of the places. We met her in the POLIN Museum on January 29th, 2018.  

Introduction: Sonia Ruszkowska is an educator at the Museum of the History of Polish Jews POLIN since 2013 where she is responsible for the educational programme of the Museum directed at schools. She participated in the educational program of the main exhibition from the beginning. She is specialised in drama theatre workshops and anti-discriminatory education with a special focus on civil rights and the fight against Anti-Semitism. Her educational programmes are mainly aimed at children, youth and teachers. 
She studied philosophy, and wrote her PhD about the Holocaust, the death in the gas chambers and the possibility of bringing back subjectivity for victims of the mass deaths. Before she started her work at the Museum, she worked as a school teacher for philosophy and ethics and participated in a foundation called ‘Forum for Dialogue’, that projects with school children in different towns and cities in Poland, raising awareness for the Jewish history of the places. 
We met her in the POLIN Museum on January 29th, 2018. We were especially interested in her perspective on how education can function as a means to fight Anti-Semitism. 

 

I: You told us about a Forum for Dialogue. Is it only devoted to small towns and villages in Poland, or also to bigger cities? As an educator, do you see a greater need of educating in smaller towns? 

 

SR: There are programs of Forum for Dialogue in bigger cities like Białystok or Warsaw, but generally the project is run in smaller towns. I think the need is the same in smaller and bigger cities but in bigger cities there are more educational programs that are available for students. 

 

This kind of programs are very necessary because kids do not know the Jewish history of their towns at all, they have no idea that Jews were living there and if they know, they do not know what it means because they know nothing about Jewish culture. But they are very interested, if we go there and we start the topic they are really fascinated by it. 
In their homes, many times they encounter very anti-Semitic perspectives, so it’s harder for them to really go beyond that, if they will never meet somebody for whom Jews are positively connoted. 

 

I: What do you think, where do these anti-Semitic things come? Are the parents anti-Semitic because their parents were anti-Semitic? 

 

SR: That’s a big topic. Generally, what we know from history, antisemitism comes for example (and we show it on our exhibition) from Christian-Jewish relationship, but also from economic rivalry...it’s a big topic… there are so many books about it... But from an individual perspective, I meet mostly pupils and they say ― that’s a good example: “My grandma heard that there’s a Jewish workshop in the museum and she told me not to bring my wallet.” And this kid, he was laughing, like: “Oh, what is grandma saying”, so this is already okay, because he trusted us (assuming that we are Jewish if we make the Jewish workshops which is not the case, for example, for me, but there is this assumption). They hear it at home, from their grandparents and parents. And very small children, who do not really think in a conscious way, they have very bad assumptions with the word “Jew”. At the beginning of the workshop in the Museum I normally ask them: What does it mean that somebody is a Jew, what does this word mean?”, and many times they said “Oh, a Jew is a person who does not want to share with others”. In Polish it’s “skąpiec” ― “greedy”. I always ask myself: “How does this six-year-old child know these stereotypes already?”, and this is in language, because in Polish żydzić, it’s a verb, it comes from “Żyd“ ― „Jew“, and means “being stingy”. So it’s basically the language that gives them these stereotypes, but also, I have the impression that generally, in Polish society, there is a very bad association with Jewish culture, with Jews, and, of course, there’s always people who are fascinated by Jewish culture. 

 

I: How we can fight these stereotypes, in your opinion? 

 

SR: In my workshops, I use this method called Non-Violent Communication (NVC), I try to give empathy to a person and not to judge this person, like “You’re stupid, you’re anti-Semitic, how can you say these things?” ― I try to understand why this person is saying this, and mostly it is in the end that people want to feel safe. And if they don’t know something, they have this association of danger, for example, that their values are in a way endangered. So, I try to understand this person, what this person is saying, and then convert it into a language of needs ― what does this person need? And then when this person really feels that I’m listening to him or her and is calm, then I can say how it is for me. For example I can say that I know many Jews, many of them are my friends and I would like people to respect them. Of course, I can also say that some Jews can do negative things as everybody else, but generally, for me, the respect is important. Or, for example, I can say that I wrote a book about the Holocaust and know a lot of testimonies, so it’s really important for me not to make jokes about the Holocaust. So, I try to make it personal, so that’s one thing – other thing: people need knowledge, because they don’t know so many things and they have only some images. For example, this topic that is always coming back that Poles were so great for Jews during the War and how Jews could say that Poles were murdering Jews. So, if people do not know that before the War there was a lot of discrimination of Jews in Poland and a lot of tension, they will never understand why Poles reacted the way they did during the War, why this tensions were even greater because of the occupation. So, if people learn things and understand things, they are more open.  
Another thing: I believe in meetings. If people have only images and they have never seen any Jewish person, they have really only an image. So, when I see these, for example, Israeli-Polish meetings, even if they last only for one hour, they can really change something, and I really believe, not that it will change everything but as an element of a project, it gives a lot of change. And I think that for example ― what we do here is that we want to give a positive association with Jews and the Jewish culture. Not like: “Jews means only Holocaust and Antisemitism – everything which is negative or difficult”, but also to show life, to show joy in Jewish culture, tradition and all these positive things. We want to show both. 

 

I: So, many interview partners have said that with like the internet, there is much more Anti-Semitism, and it’s spreading, there’s also hate speech in the internet. What do you think that normal people could do against that? Writing back comments – it’s usually not that helpful. 

 

SR: I invented an anti-discriminatory workshop for youth, it is about hate speech, for example on Facebook and other platforms. It is a Drama workshop, so there is a hero which is the target of hate speech, and others who use hate speech, and people have to play both parts, and then we discuss and try to invent a solution: what they could do in this situation. So, I believe in transferring the online situation in real situation between people because this way I feel that we can do more. And, in both cases the conclusion was that it’s always good to search the people who can support you, who can help you, that the person who is using hate speech, it’s not that he or she have the whole power, it’s mostly like one or two people, and there’s a lot of people in the class or in the group that are bystanders. And if you can take them on your side or ask them for help, in many cases they will join you, in a way. I make this kind of workshop to work with people, to show them how is it to be this person against which hate speech is being used. 

 

I: How common, do you think, are anti-Semitic stereotypes in Poland? 

 

SR: I think everybody knows them. Every little child knows this anti-Semitic perspective, somehow. I’m not saying that children consciously think like this. So, everybody knows the stereotypes, of course not everybody says that they are true, but the knowledge is common. I think, in a smaller towns, a lot of people really think in this way, also because of trauma after the War. In many little towns people live in so-called ‘Post-Jewish houses’,  and it is too hard for them to really confront with this story, to really work with it. It’s easier to forget about this, but this conflict stays. It always makes you frustrated, so anti-Semitism is maybe the way to express this frustration against Jews. So, I think anti-Semitic stereotypes exists, but it’s not necessarily the case, that  when Jews are in public space in Poland, there would be some act of violence against them – it is more about saying to other people that Jews are… for example greedy. In big cities, I think, there are some small communities that are  fascinated by Jewish culture. There are people coming to our Museum on every festival, every concert or event – many of them are well educated or interested in culture in a general way. There are also people who just don’t care about Jews or Jewish culture. But I would say that Anti-Semitism is really strong. I didn’t do any studies or research about it, it’s more like a general feeling. 

 

I: I was wondering... as you said, pupils do not know much about Jews or how the actually are – is that part of the compulsory education? Because I think it is part of the Christian history as well. 

 

SR: You can teach something in school, but children do not know everything that’s in their books. And it’s only a small part of the program, and it depends on the teacher. Of course, there are teachers who are interested in the Jewish history and so it is possible to talk a lot about this history during lessons. But if you don’t want to, you can do only one lesson about this, and then there is Holocaust. And the Holocaust is really present, but you can also teach it in different ways: is it about this Polish heroism or is it more about the Holocaust? Mostly, I think, Jewish history in school program is connected to War. So, it creates such an image: – there are no Jewish people, and then we see, out of the blue, there are a lot of them, and then they are murdered. So, that is basically the image that children have after the school education. But now, there is also the new thing that there is a lot of Islamophobia in Poland. And these two things are combined, that if I ask children about Jews, they often say that the book of Jews is Koran, and that Jews are Muslims. Really, it’s very common now. The Jew is the alien, now, Muslim is an alien, so – Jews are Muslims.  

 

I: During the Holocaust, there were Polish people who were helping Jews to survive. And you can read a lot about that and there are whole exhibitions in the museums about that. But there were also collaborators, right? Polish collaborators. So, my question is: is it also well-known, is it also documented, easy to do research on that topic? 

 

SR: It’s a very interesting story because there are a lot of scientific books about it. And if you want to know about it, it’s not a problem. But no, people don’t know about this. Students don’t know about it at all, and we have the workshop about Jewish-Polish relationships during the War, and I made a film with material from the USC Shoah Foundation Archives, and there are people talking about their deportation to Ghettos. And during these deportations, when they were entering the Ghetto, they told what the reaction of Poles was. And they said that people were laughing, people were doing bad things to them, they were cruel. And we show this short film to students, and they are always shocked. They had no idea. Of course, it’s not that every Pole during the War was doing such a thing. There were a lot of Poles that were sad and full of compassion during the deportations of Jews to the ghettos. But there were a lot of people reacting with cruelty. In fact, I think I realized that only during my studies. Nobody ever told me about this in my whole school education. Personally, I remember this moment of realization very well and I felt that they lied to me all my life. That I finished school in Warsaw, humanistic classes, and nobody ever told me about this. Especially in primary school it was always: “Poles ― the heroes, great” and then I discovered it’s not true, it’s a lot more complicated. This positive image of Polish people, it’s everywhere, and the truth is not so beautiful, it’s kind of hidden in the consciousness of society – but the knowledge, it’s really there, if you want to see it. 

 

I: Do you think the government wants to sort of, well, lead it in the direction that everybody thinks Poles have helped the Jews, but no Poles have helped the nazis killing the Jews? And is difficult to do research in that direction, does the government control what is shown in the museums, something like that? 

 

SR: Yes, I think so. I think there is a danger of it because people who make research need grants, and it is the government that has influence on who will receive these grants. So, there is always this question: how much the researchers will be dependent, which university will get more money, which less, and so on. So – it’s never really objective, there’s always some interest in there. 

 

I: Let’s not speak about universities. Why do you think do normal people don’t want to admit that Poles didn’t help enough? 

 

SR: It’s not nice to see yourself like this… 

 

I: But why do you see yourself immediately as one of them? It’s just a fact, nothing personal. 

 

SR: It is, because they identify with a nation. And you want to have the good image of your nation. So many people say: “It is not true that Poles were not helping enough, it’s only Jews who say this”. 

You know, there is a good book about this, it’s called “Polski Teatr Zagłady” (Polish Theatre of Holocaust) written by Grzegorz Niziołek. It’s about the theatre spectacles about the War, made just after the War. He writes from a psychoanalytical point of view. He claims that the bystander position, is very difficult, because it’s somebody who sees what is happening, who feels guilty that he or she didn’t do anything, and who is afraid because he thinks “I can be the next one”. He is also happy that it’s not him who was killed, and then he feels guilty because of this joy. There are many emotional layers. And it’s a very difficult position. There is a big, big guilt. Situation is not black and white. With Jews and Germans, it’s somehow black and white. Germans started the War, they were murderers, and Jews were victims. And Poles are kind of in between. They are victims, but they are also murderers, and they are bystanders. The second thing is that many Jews, have also no true vision of Polish history of the War. For example, many of them never heard about the Warsaw uprising, and about the fact that Poles were actually fighting with Germans, that so many Poles – non-Jewish Poles – were killed. These two narrations about the war (Polish and Jewish) are so different, and neither of them is not entirely true. I think the truth is very complex, it’s not at all black and white. Poles want to see themselves as total heroes, total purified, people who only helped Jews. Jews often think that Poles didn’t helped them, that they were perpetrators. I think the truth is somewhere in between, but somehow people prefer to have a clear image. And why people simplify the truth? That’s a philosophical question, it’s very interesting. Psychologists say that ambivalence is the most difficult feeling/state of mind. We prefer clarity. 

 

I: I can say from the perspective of a teacher that in Polish education there is still not enough time to teach about everything in detail. So, if you only got several hours to devote on the Holocaust, you would prefer to talk about ‘good Poles’, not about ‘bad ones’. Probably, that’s why students are not well educated about the bad sides of Poles. Do you agree that it might get uncomfortable for teachers because of the lack of time and the comfort? 

 

SR: I don’t really agree, it’s an excuse. I think it’s just difficult topic. Teachers maybe do not feel really prepared to teach about it because they would have to really think about it personally, make peace with it in themselves. It took me many years to really dig in and to really feel acceptance with this topic. It’s difficult topic and teachers also need education for them about it. In the Museum we make a lot of workshops for teachers also about the Holocaust. If you have only one hour to teach about the Holocaust and you say that Poles were so great and in fact, they weren’t… Maybe you could change it and devote half an hour to tell about polish heroism, and half an hour to tell about Poles murdering Jews. 

 

I: As you said, it’s a good excuse. I wouldn’t like to do it. I would avoid being accused by parents of my students that I said something about bad Polish people during the War. 

 

SR.: Yes, of course. We in the Museum are in a privileged position in this aspect because we don’t have contact with children’s parents but sometimes teachers also say that we show too much of Poles’ ‘bad side’. So, we always try to say that it’s only one perspective and that there is also different one. But I think that children will hear about this positive side of Poles anyway ― about the righteous among the nations, it’s a hundred percent certain. So, I would prefer to say other things which they might not know of. 

 

I: What would you say about the development of anti-Semitism? Would you say that maybe it gets less because people become more educated, or maybe it gets stronger because, well, the Holocaust, has happened seventy years ago? Are people becoming more patriotic and nationalist against other groups? 

 

SR: I think there is a possibility that anti-Semitism will grow because if there is more Islamophobia and Homophobia and all this kind of attitudes, it is always connected with anti-Semitism. I don’t know why, but it happens. I think it will grow. And besides, anti-Semitism is now mixed with anti-Israeli reactions, and this is also very complex topic. And I think the situation in Israel will not be very calm in the near future, I fear, so it will also not be the element that can help. I don’t see anything in the big scale that would really help. 
 

I: I had a discussion with a roommate, and he told me that he doesn’t like it that little kids learn about the Holocaust, and he thinks that it will do some harm to them. At what age, do you think, can we tell a child about the Holocaust, and how can you start telling these things? 

 

SR: In the Museum, we start these workshops in fourth class, so the kids are around ten years old, and we do it with literature, not with images. I think it’s more about the question of how to teach, not if we teach it. But I also had an interesting discussion about this with friends from Israel, and they said, “We teach about Holocaust very early, very little children.” But in Israel there are a lot of days during the school year when there are some institutionally organised days about the Holocaust, so these children will hear about it anyway. So, it’s only the question if they do it in appropriate way, to prepare them for it, or they will just imagine things. I think in Poland it’s a bit the same, that, if you walk around in Warsaw, for example, each few metres there’s a plaque saying, like, “Here, fifteen people were shot by Germans” etc. So you cannot really avoid it, the memory of war, it’s really everywhere. Children know it very early – probably earlier than in other countries, and so, I think it’s okay to teach them, but in a wise way, not with images of naked corpses or other horror things. Even if they say they want to see it.  In fact whenever we go to the Interwar Period Gallery with small children, they say: “Oh, we want to go to the war, we want to go to the war!”, and we say: “No, you will do it when you’re older”, and they react: “No, we want now, we want to see it!” and it’s like a computer war game for them – of course, they don’t understand what it really was ― the Second World War. So, I think it should be the process, that first you talk about values, you tell children about individual stories, and then there is more and more details. But I think, for example, to take people to Auschwitz – you should do it at end of high school. For me, it shouldn’t be too early, to talk about gas chambers and the mass death. It’s so horrible that people should be prepared for to really understand it.  