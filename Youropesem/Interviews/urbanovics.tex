Interview with Jānis Urbanovičs 

Jānis Urbanovičs (*1959) is a Latvian politician and member of the Saeima (Latvian parliament) since 1994. In 1998, he became one of the founders of the Baltic Forum, one of whose aims it is to maintain good relations with Russia. The Forum has gradually become a dialogue platform for post-Soviet states. As part of his work for the Baltic Forum, Jānis Urbanovičs has published several books on culture, history and economics. From 2005 to 2010, he was chairman of the National Harmony Party. He was chairman of Harmony from 2010 to 2014. 

Jānis Urbanovičs’ father fought Nazi Germany in the Red Army, while his uncle was in the Latvian Waffen-SS Legion.  

We met him in Rīga in the end of September 2017, in the company of an English-Latvian translator. 

How was it for your uncle who fought for the Legion, was he conscripted or did he go there as a volunteer? 

Translator: Vai Jūsu onkulis gribēja brīvprātīgi iet leģionā vai viņu piespieda pierakstīties?  

Jānis Urbanovičs: Es patiesībā zinu droši, ka tas bija piespiedu kārtā. 

Translator: He says that he was pressured to go into this Legion. 

Jānis Urbanovičs: Vispār par Latviju, par Latvijas ļaudīt, Otrā pasaules kara kontekstā, par brīvprātīgajiem runāt var ar ļoti, ļoti lielu izņēmumu.  

Translator: And that people who went from their own will were very rare. They didn’t want to. 

Jānis Urbanovičs: Latvijā bija pavisam maz tikai Hitlera Vācijas vai Staļina Krievijas patrioti, kas gāja brīvprātīgi kauties. 

Translator: There were very small part of people who went willingly on the side of the Nazis or the SSSR. 

Jānis Urbanovičs: Jā, ja iet runa par SS Waffen leģionāriem, tad arī tur, vai Arāja slaveno brigādi, tad nu tas drīzāk ir izņēmums. 

Translator: Well, there were exceptions, for example famous Arajs brigade. 

Every 16th of March there is a celebration of the Waffen SS in Riga. What do you think about it? 

Jānis Urbanovičs: Tā ir izrādīšanās, nevis leģionāru piemiņas diena. 

Translator: He says that it’s just showing off, and not a day for remembering people. 

Is it also anti-Semitic?  

Jānis Urbanovičs: Visi antisemīti neieredz arī krievus. 
Translator: Everyone who is anti-Semitic is against to Russians, too. 
ānis Urbanovičs: Otrādi nav.  
Translator: It doesn't work the other way around.  

So, it’s a similar thing, anti-Semitism and anti-Russian racism?  

Translator: What he wanted to say was those who are russophobes and anti-Semites are the same people. There are no people who thinks, I’m an anti-Semite, I’m not russophob, or vice versa. There is one small group that wants to provoke Russian people, those are the people who are going to this thing on March, you know, on the demonstration. 
Jānis Urbanovičs: Latvijas neatkarības atjaunošanai bija par atdalīšanos no krieviem un visa krieviskā, un mēs turpinām šo atdalīšanos.  

Translator: And that the declaration of independence here in Latvia was about the separation from Russia and Russian people and we are still continuing to do so. 
Jānis Urbanovičs: Un no visām krieviskajām izpausmēm, mēs tā kā atdalāmies, līdz ar to arī pret saviem līdzpilsoņiem.  
Translator: And we are still separating from traditions and everything connected with Russia. We are still acting like that year in Latvia with our own people. 

Are there any other a signs or demonstrations of anti-Semitism today in Latvia?  

Translator: Vai tagad ir kaut kādas antisemītisma izpausmes?  
Jānis Urbanovičs: Šī ir raksturīgākā.  
Translator: This is the most typical one.  
Jānis Urbanovičs: Ebreju tautības pārstāvji bija aktīvākie padomju varas ieviesēji šeit. Viņi runāja krieviski.  
Translator: Jewish people were the most active Russian supporters at that time. 

That is the truth, or that is what many people believe?  

Translator: Tā bija vai tika domāts, ka tā bija?  
Translator: Both. 
Jānis Urbanovičs: Bija, bija tādi personāži, bet bija arī tiražēts. 
Translator: There were people in this communist party, but it was also talked about in a way not necessarily true. 

In Germany and in Poland, we have a very strong tradition of the anti-Semitism. And in Latvia, according to my knowledge, there is not a strong traditional anti-Semitism, but there were some collaboration during the Nazi time, is it right? 

Jānis Urbanovičs: Latvija ir mazāka par Poliju un Vāciju, mums viss ir mazāks.  
Translator: Latvia is smaller than Germany and Poland, so everything is smaller. [laughs] 
Jānis Urbanovičs: Mēs esam arī nedaudz perifērijā. Krievijā ir aizgājusi arī ziņa par tirazēšanā par Latviju. Viņi mazāk zina, tāpēc ka par mums mazāk runā kā par Poliju.  
Translator: We are also in the periphery of European Union, that’s maybe the reason why you know a little less than about other countries, maybe people are not talking too much about Latvia. 
Jānis Urbanovičs: Bet Latvijas iedzīvotāji ir bijuši gan vācu pusē, gan Hitlera, gan Staļina pusē.  
Translator: Latvian citizens had collaborations with the Hitler German and the Stalin –Russian side. 

Do the Jewish community or Israeli diplomats complain a lot about anti-Semitism in Latvia? Are there any complains?  

Jānis Urbanovičs: No, there are none. 

What do they say?  

Jānis Urbanovičs: Viņi satraucās par šo pusi.  
Translator: They are concerned about it.  
Jānis Urbanovičs: Jā, concerned is labākais vārds, lai to aprakstītu. Viņi ir bažīgi, bet viņi nav agresīvi. Translator: They are not aggressive, they are just concerned.  

Is it a lot of concern about the, say, anti-Semitism on the internet, or more about anti-Semitism in real life?  

Translator: Vai antisemītisms ir vairāk internetā vai reālajā dzīvē?  
Jānis Urbanovičs: Patiesībā es varu vairāk runāt par saviem vai tuvāko novērojumiem. Nav viņa daudz ne internetā, ne sadzīvē. Viņš ir vairāk politikā.  
Translator: He can talk only about his experience and he says that it's not there in the internet, not there in the real life it’s just there in politics.  
Jānis Urbanovičs: Piemēram, holokausta upuru īpašumu nodošana ebreju draudzēm radīja atkārtotu antisemītisma vilni.  
Translator: The government just gave back to the Jewish society a few buildings, which belonged to the of Jewish community, and of course it raised some wave of anti-Semitism.  

Are there any programs by the government to combat anti-Semitism or racism in general?  

Jānis Urbanovičs: Nē, nav tādu programmu. 

Translator: No, there is not programme like this.  

Are there any Jews in the parliament or in the Harmony party?  

Jānis Urbanovičs: In parliament and in Harmony party.  
Jānis Urbanovičs: Jewish is… 
Translator: …everywhere. 

Also in other parties?  

Translator: Yes. 

Does it happen that politicians of the right wing in the parliament say anti-Semitic things?  

Jānis Urbanovičs: No.  

It doesn't happen? 

Jānis Urbanovičs: Ir tikai daži izņēmumi.  
Translator: There are only a few exceptions.  

It's a taboo? 

Jānis Urbanovičs: It is a bad tradition. Not Latvian parliament tradition. 

And outside the parliament?  

Jānis Urbanovičs: Mēs esam ļoti kulturāls parlaments un uz āru mēs esam visi ļoti kulturāli - gan politiķi, gan sabiedrība. Uz āra.  
Translator: On the outside the parliament is very polite. [laughs] 

If you look at a museum, for example, or at the historiography: Is it more scientific or more folkloristic -only heroes, only victims?  

Jānis Urbanovičs: Mums ir, mums ir visāda veida vēstures sacerējumi, tajā laikā arī antisemītiski ļoti. Ir. 
Translator: We have different kinds of these books and they have different thoughts of them.  
Jānis Urbanovičs: Ir, piemēram, pētījums par Cukuru, ļoti komplimentārs viņam. Lidotājs Cukurs, kurš bija Arāja..., nu, kas nodarbojās ar ebreju...em...antisemītismu kara laikā.  
Translator: There is a book about Herberts Cukurs, who was fighting against the Jewish people during the Nazi time. And there is a book very complimentary about him. 

We heard that it's possible to Jewishness as a nationality put in the passport or ID document, like in the Soviet times. Is that still possible in Latvia?  

Jānis Urbanovičs: Nav obligāti, bet var, ja grib. 
Translator: If you want, you can put it. 
Jānis Urbanovičs: Tas nav ar likumu norādīts, var neko nenorādīt pasē.  
Translator: But the law doesn't say that you have to do it.  

But do you have any advantages if you do it, for coming to Israel or something alike? Does it have any consequences if you do it?  

Translator: It’s not the same as citizenship. You still have the Latvian citizenship, all rights are the same, you can just put your nationality, if you don't want to put it, just leave it out.  

How is the relation between Latvia and Israel? And how do most of the Latvian population think about the state Israel? 

Jānis Urbanovičs: Konfliktā ar arābu valstīm mums lielākā daļa sabiedrības ir Izraēlas pusē.  
Translator: In the conflict in the Arabic states we are in the Israel side, supporting them.  

And so is the population? 

Jānis Urbanovičs: Yeah. 

How is the government supporting the Jewish community? Are there any measures taken? 

Translator: Vai valdība kaut kā atbalsta ebreju kopienu? 
Jānis Urbanovičs: Caur vēstniecību. Vēstniecībā tur viņi sadarbojas. Ir vēl savas programmas bijušajiem Latvijas ļaudīm, īpaši tiem, kuri ir saglabājuši pilsonību. Ar Izraēlu mums var būt dubultpilsonība. Translator: There are some programmes in Latvian embassy in Israel and in the country. There are few programs and one of these says that you can have double citizenships with Israel. It’s not allowed with other countries, but you can be a citizen of Latvia and Israel.  

We read that you have participated in the publication of a compendium of life stories of Latvians since 1934? Are there also life stories of Jewish people from Latvia in this book?  

Jānis Urbanovičs: Es esmu rakstījis par pēckara laiku. Jā, protams, ka ir, skaidrs, ka tur ir, īpaši pirmajā grāmatā, nozīmīgi fakti par Izraēlas kopienu Latvijā.  
Translator: Of course, there are some, especially in the first book there are some very important facts about the Jewish/Israeli community.  

And is it documents from archives that are published?  

Jānis Urbanovičs: Mēs rakstījām 3 autori, lasot arhīva dokumentus, tā laika presi, izdevumus, kā arī hronikas, un apmainījāmies ar viedokļiem. Tā grāmata ir viedokļu apmaiņa.  
Translator: This book is an exchange of opinions and it was based on the facts from the archives, from publications, from chronicles. Everybody read them and then there was an exchange of opinions that is printed the book. 

Are there any educational programmes about the Jewish religion, about Jews in Latvia, about the Holocaust, about anti-Semitism? Is it part of the school schedule, is it in the history lessons, in the religious education, maybe?  

Jānis Urbanovičs: Tas viss ir skolā, bet, cik dziļi - nezinu, sen neesmu bijis skolā.  
Translator: He says that these topics are taught in school, but how deep, he doesn't know, he ended the school a long time ago. 

Do you have anything to add what could be done to improve the situation of Jews in Latvia, especially in politics?  

Jānis Urbanovičs: Politikā tam nav nozīme, tas tiek izmantots tikai, lai kādreiz iekostu otram.  
Translator: It doesn't matter in the politics, it’s just a thing with which you can bite someone.  
Jānis Urbanovičs: Kad trūkst argumentu, tad sākās kaut kādas piesaukšanas. Piemēram, valodas prasmes vai izcelšanās, tas jau ir tad, kad politiķiem trūkst argumentu.  
Translator: When the politicians don't have enough arguments, then they just pick on things like language skills or nationality. 

How big is the anxiety because of Russia? Today, we had an interview with a scientist, and she was saying there is a big anxiety that Russia can invade in Latvia like in Ukraine? Is there a big fear?  

Translator: Vai ir bailes no Krievijas?  
Jānis Urbanovičs: (Nopūta). Blakus Latvijai ir Krievija.  
Translator: Russia is next to Latvia.  
Jānis Urbanovičs: Latvijā ir daudz krievu. 
Translator: There are many Russians in Latvia.  

That's why I asked this question. 

Jānis Urbanovičs: Ja Latvijā krievi un latvieši dzīvo draudzīgi, atrod savu kopdzīves formulu, kur abiem labi.  
Translator: And if the Latvians and Russians find a way to live together peacefully,…  
Jānis Urbanovičs: Tad mums ir vienalga, kas notiek Kremlī. 
Translator: …then it doesn't matter what happens in the Kremlin.  
Jānis Urbanovičs: Tad mums tik un tā būs labas attiecības ar Krieviju. 
Translator: We will still have a good relationship with Russia.  
Jānis Urbanovičs: Ja Latvijā latvieši un krievi dzīvos naidā un aizdomīgumā, tad arī mums vienalga, kas būs Kremlī, mums būs slikti, sliktas attiecības ar krieviem.  
Translator: And if the Latvians and the Russians have bad relationships here, then it doesn't matter who holds power there in Russia, we will still have bad relationships.  
Jānis Urbanovičs: Iespējams, ja Krievija būtu tālāk, tad latvieši un krievi sadzīvotu ātrāk labāk.  
Translator: And maybe if Russia would be somewhere further way, then the relationship would be easier. 
Jānis Urbanovičs: Bet es domāju, ka ekonomiski krievi ir ļoti izdevīgi tuvu, ļoti izdevīgi tuvu. 
Translator: But economically it's good that Russia is nearby.  
Jānis Urbanovičs: Atliek vien tikai mums mājās sakārtot attiecības starp 2 etniskajām grupām - lielo latviešu un mazo krievu.  
Translator: Yeah, the relationships between these two ethnic groups should be softened, and it would be better for both sides. 

In history, the Baltic Germans were a ruling class in the Baltics. Is there still a anti-German resentment? 

Jānis Urbanovičs: Līdz Pirmajam pasaules karam mēs neieredzējām vāciešus, pēc Otrā pasaules kara mēs neieredzējām krievus.  
Translator: Till the Second World War, we hated the Germans, now we hate Russians. [laughter]  

What about the Polish people? [laughter]  

Jānis Urbanovičs: Viņi arī ir Baltijas valsts. 
Translator: They also are Baltics. [laughter] 