Interview with Vitaly Liberov 

Mr Liberov will introduce himself in the course of the interview, which took place in Nuremberg on March 8th, 2017. 

Could you please tell us something about yourself first? 

Vitaly Liberov: My name is Vitaly Liberov. I am forty years old. I am an entrepreneur. I was born in Bryansk, Russia. I was raised as usual Soviet kid, I was a Young Pioneer, then I joined Komsomol, so, I was not familiar with religion. But as I was growing up, I remember men in our family going together somewhere several times a year, whispering about something, coming home with big slices of Matzah wrapped up in paper, and things like that. But it all was so strange to me that I did not pay attention. The first moment I realised I was a Jew was when I came to government office to receive my passport at age of sixteen. The official asked me: “Kid, do you really want to be registered as a Jew?”. I naively answered: “How could that be otherwise?”. It was the time of Perestroika, 1989 or 1990. The first branches of the Jewish Agency Sokhnut1 started to appear in the Soviet Union, and that was the reason why I decided to discover my origin. I discovered that my grandparents spoke Yiddish in their youth, my grandmother’s original name was not Galina as I thought, but Golda. The fear of the Soviet system was so great, that everybody in our family was afraid of talking about their Jewish origin. That is why I knew only a few words in Yiddish, as elderly people in our family used them occasionally. And there I was standing with my passport, in which the word “Jew” was written down, and I knew nothing about that. That is how I became an extremely active member of the Sokhnut Bryansk branch. I started working as a Madrich2, and I liked it.  It was really amusing, because we had to organise camps for children without any knowledge of Judaism and traditions. So, there were evenings and nights before the events when we had to learn ourselves. That was quite an interesting time. Then my parents decided to emigrate to Germany. It was quite a difficult decision for me personally, I was not fond of my parents’ decision because I wanted to go to Israel. But it was not for me to decide, that is why I in 1996 I came to Germany, and now I live in Germany for a longer time than I lived in Bryansk. When I came to Germany, I directly applied to the Center Council of Jewish Communities of Germany. That is how I became a member of Jewish community.  

Is it correct that you realised that you are a Jew at the age of sixteen? 

Vitaly Liberov: I always knew I was a Jew. Even as a kid I had to fight with other kids who were mocking me because I was a Jew. But until the age of sixteen, I never considered myself as a Jew religiously. It is hard to explain, that there is a difference between being a Jew as ethnicity and being a Hebrew as religion. But at age of sixteen, I decided not to distinguish these two meanings – I accepted both ethnicity and religion.  

Were your relationships with other children good, before the age of sixteen? 

Vitaly Liberov: Well, I do not want to say that I was an unhappy child. In Soviet school it was officially not acceptable to distinguish people by ethnicity, because all were Little Octobrists, then Young Pioneers and so forth. And if someone was trying to say something bad against Jews in school, then the teacher had to punish this kid. State propaganda showed the Soviet Union as a multi-national state of fraternal friendship among the peoples, but in fact there was a huge control over the number of Jews accepted to universities, so-called quotas. In the streets, of course, there was anti-Semitism among the children.  

 How did your family preserve Jewish traditions? 

Vitaly Liberov: In a very rudimental form. For example, my grandparents used no to eat pork, not to eat bread during Passover or to restrain themselves from food during Yom-Kippur. It was not the question of preserving traditions, but rather some little nuances. And, of course, no Yiddish could be spoken. But, by the way, when I came to Germany, I surprisingly discovered that German people, especially elderly in Bavaria, use some Yiddish words without realising it. For example, “Masel”, which means “happiness”, or “Tachlis”, which means being honest or speaking the truth.  

How did you become a member of the Jewish Community of Nuremberg? 

Vitaly Liberov: I just came to the synagogue for a daily prayer, and I got acquainted with other members. I wanted to keep doing what I was doing in Bryansk. Luckily, the Jewish emigration from post-Soviet countries started, and I got a lot of friends for whom Russian was their native language. And of course, there is a peculiarity of Judaism, that whenever you pray, whenever you celebrate Shabbat or any other holiday, there are people in every country of the world that are doing the same. That is why you cannot be alone. I was never alone when I became a member of Nuremberg community.  

Is the Jewish community of Nuremberg different today from what it was like when you came? 

Vitaly Liberov: Tremendously. I came before the large wave of emigration started. Before that the people at the synagogue could not gather for the Minyan to read the prayer. But in the end of the 1990s, the community hugely enlarged. There are a lot of members of community that regularly attend ceremonies. We have a nursing home, a big hall for events, classes for youth, we even think about opening a kindergarten. Sometimes hundreds of people attend celebrations.  But, of course, we have also other kind of events, for example lectures, classes, conferences. We invite people from other organisations, everybody is welcome. I think that Nuremberg is getting a new “face” thanks to the growing Jewish community, because German people are interested in coming to us.    

How many Jews live in Nuremberg? 

Vitaly Liberov: I know that there are two thousand members of our community, and I think it is forty percent of all Jews in Nuremberg. Actually, our community is not the only one. It just happened historically, that there is also Chabad community and several alternative groups. These groups even have their own rabbis, who are less conservative. Our community is a classic orthodox community. 
Our liturgy was written in the middle of 19th century, specifically for the needs of community.   

Does your community have contacts with Israel? 

Vitaly Liberov: We have Hebrew classes. Nuremberg’s sister city is Hadera, that is why we have exchange programmes with Israeli schools. Many Israeli students come to Nuremberg and especially to our community. 

Do you think that there is anti-Semitism in German society? 

Vitaly Liberov: My answer would be very subjective. I just want to remember summer of 2014, when there was a military operation in Gaza. At this time, I was really afraid, because everyone – far-right, far-left, Muslim organisations – united in hatred for the Jews. People were out on the streets shouting anti-Semitic things like “Hamas, Hamas, Juden ins Gas”, and the Police did not do a thing to stop them. A huge crowd of young men broke into the main station building because they thought the owners of Burger King and McDonald’s are Jews, whereas in Nuremberg, these stores belong to Muslims. They vandalised the building because of hatred for the Jews. That is why I can definitely say that there is anti-Semitism in Germany. I think there is a huge gap in the educational system. After the War, the topic of the Holocaust was broadly discussed at schools in Germany. Moreover, there were people who survived during the War, and they could tell a lot. Nowadays, there are a lot of children at schools to whom the topic of War and the Holocaust is not as close as to people of my generation. Unfortunately, to children from some Muslim families, the topic of the Holocaust is irrelevant, or they consider it to be fake. In my opinion, this is a problem or the educational system. There is no unified system of teaching history among Federal Lands in Germany. There is no unified pedagogic plan. Of course, there are standards set by the Ministry of Education, but in the end, everything depends on the teacher, who are afraid to have a discussion. That is why many teachers just tell pupils to write an essay on Spielberg’s Schindler’s list, which does not show the Holocaust. Speaking about reasons of anti-Semitism, I would like to refer to director Sokhnut, Nathan Sharansky. He developed the so-called “Three D” test of anti-Semitism: deligitimisation, demonisation and double standards. Each of these “three Ds” is a basic of anti-Semitism. Some people say, “Yes, Israelis are entitled to protect themselves, but they at least have automatic rifles, Palestinians do not”, or “Yes, Jews were killed during the War, but other peoples were killed, too”. Nowadays, anti-Semitism is not as primitive as before, because some people tend to hide anti-Semitism under a leftist fight against capitalism and globalism. Some tend to use euphemisms like “antizionism” or “critique of Israel”, which is a pure form of anti-Semitism. Sometimes, these are the same people that attack refugees, both in reality and in social media. People do not realise that anti-Semitism is not only a problem of Jews – this is a problem of society as a whole. Even when I speak with my German friends, they admit that they do not consider anti-Semitism as their problem. I remember that when I was a Madrich in a Jewish camp in Frankfurt in 1998, some German kids sneaked to the venue to see, as they said, how the Jews drink blood. This example shows how deeply anti-Semitism is rooted in society, and how different the forms are that it can take.  

What can be done to fight with anti-Semitism? 

Vitaly Liberov: First of all, we have to work with children. It is necessary to show at school the consequences of hatred and intolerance. Secondly, we must make our politicians listen. Of course, in the end summer 2015 there was a huge manifestation in Berlin near the Brandenburg gate organised by the Center Council of Jewish Communities of Germany. Even Angela Merkel attended. Some politicians came and said right things, but most of them do not care. There are still fascist parties in Germany, which are not prohibited. It just changes its name from “German National Union” to “Gathering of German Nationalists” and then to “National Party of Germany”. They pretend to be democratic, but in reality, everyone knows its nature, despite that the Construction Protection Bureau inspect them. But there were even underground groups who murder people, vandalise shop and businesses. Unfortunately, the police are not efficient in catching them. 

We would like to ask you about the Forum of Jewish Culture in Germany. Could you please tell us about this organisation?   

Vitaly Liberov: This organisation is based in Nuremberg. I am a member of the managing board. It is a society which consists of Jewish and non-Jewish members. Our purpose is to tell about the history and traditions of Jews in Germany and in Nuremberg, particularly. We organise some events, lectures, we invite representatives of different confessions to discuss some topics, for example marriage, raising children and so on. We also organise excursions and seminars. People like it and the number of participants increases.  

Does the Forum speak about Reformation and Martin Luther’s attitude toward the Jews, about his anti-Semitism? 

Vitaly Liberov: Yes, and this topic was especially interesting for Catholics. But Nuremberg is mainly a Protestant city, that is why it is not so easy to speak about this. Unfortunately, in some churches in Germany there is still such an anti-Semitic thing as the Judensau, “Jewish pig”. Just a few years ago, stones from a Jewish cemetery were found in the floor of the southern tower of the Lorenzkirche in Nuremberg. The former chairman of the Community, Arno Hamburger, was working on taking these stones out of church, there were many obstacles imposed by the Lutheran church. But we managed it, and now these stones are back on the cemetery. The history of Jews in Nuremberg is very rich. One of the oldest Jewish cemeteries is in Nuremberg. Actually, Nuremberg is tragically connected to Riga, since many Nuremberg Jews were deported to Riga and shot in Biķernieki forest during the War. We have also so-called Riga Committee, which is working on commemorating those people.  