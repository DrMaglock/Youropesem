\section{Michal Bilewicz}
\textit{Michal Bilewicz (*1980) is an Assistant Professor at the Faculty of Psychology of the University of Warsaw since 2008. \\
He studied sociology and philosophy in Warsaw between 1999 and 2003. His Ph.D. thesis, completed in 2007, carries the title `'Between self-verification and social identity processes: Social psychology of threatened ingroup status'. 
He has worked in New York and Delaware in the USA and in Jena, Germany. \\
His research interests, among others, are conspiracy theories, prejudices, intergroup conflict, the threat of positive social identity and dehumanization, mainly on the example of xenophobia, anti-Semitism and ethnic conflicts, reconciliation mechanisms after genocides, the influence of cognitive mechanisms and public language on the exclusion of minorities.
He is the vice president of the Forum Dialogue Foundation, that is dedicated to fostering the relation between contemporary Poland and the Jewish people. \\
The interview took place in the Faculty of Psychology of the University of Warsaw on January 30th, 2018.}

Parts of the transcript as well as the edition are still missing!