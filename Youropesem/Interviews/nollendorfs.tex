Interview with Valters Nollendorfs

Valters Nollendorfs was born in Riga in 1931. At the age of 13, he fled to Westphalia in Germany with his family. He emigrated to the United States of America in 1950, where he became professor of German language and literature at the University of Wisconsin-Madison. In 1988, he first returned to his home country and became a member of President of Latvia's Historians Commission and the Latvian Academy of Science. Until today, he is Chairman of the Board of the Museum of the Occupation of Latvia in Riga and offers tours through the museum. We met him in the museum in September 2017. Before we started the recording of the interview, we talked about the historical development of Latvia with him, namely what role occupations through foreign powers played in this history.

VN: So with few exceptions, those countries did not accept the Soviet occupation by military Force, by threat and of course they didn't accept the Nazi occupation or the continued Soviet occupation, which means that there were military units in Latvia from 1939 to 1994, which could have overpowered any resistance. The government was dissolved, but in international recognition by law Latvia continued existing although it was occupied and de facto was not sovereign. And when Latvia redeclared independence in 1991, most of the countries wrote, we are willing to take up diplomatic relations again, so the first foreign minister who wrote to Latvian government after the coup in Moscow and after Latvia declared full independence in 1991 was the Foreign Minister of Iceland, and in his letter he wrote, we have never recognised the military occupation and the illegal annexation of Latvia by the Soviet Union, more or less, we are ready to take up relations again if you are ready. That has repercussions, that means that immigration at the time from other countries was illegal. That is one reason why people did not get immediate citizenship here, unless they were citizens or children of citizens, and second and even more important, that resistance was in international law legitimate with one exception: crimes committed during resistance were not. The Nuremberg Trials said very clearly, the SS is an illegal, a criminal organisation, but that doesn't mean each member is guilty until this member has committed crime against humanity or a war crime, that's individual. That means to some extent that the drafting of Latvian citizens in the Soviet army and in the Nazi army was illegal. When you saw that there were two Latvians serving in the two armies, Hitler saying, this is a volunteer operation, but the note says, this is a draft notice, you have to appear. So, it's the coercion of the local population into acts which otherwise could be criminal, or which are forbidden by international conventions. And that means that people who after the war went into the woods and carried out resistance until 1956, unless they committed crimes against humanity or war crimes, were legitimate fighters for Latvian independence, which they declared. Now, the Russian Foreign Ministry right now does not recognise this doctrine of continuity. They say, you wanted to join us in 1940, you saw how that was done, Crimea is a good example, and in 1991 you left. That's a different doctrine. They say, people who came here were legitimately here, but in international law there is no question, international law has recognised Latvia as a continuation state with some exceptions. 

I: If it's ok for you, we would like to ask some general questions about you. How exactly did you get here, how did it happen that you just changed your profession and everything and just came here?

VN: Well, I was born here. I was 13 years old when I left because my father had served in the police force. When the Soviets came, and it was very clear that we either had to take the ship to Germany, hoping that and well knowing that Germany has lost the war, rather than taking a free train ride to Siberia. so that was clear. I was in a refugee camp, displaced persons camp Greven, Westphalia until 1950, when my family went to the United States. And that's where I got my academic education. And then I became professor, and in 1996 I was bold enough to retire, they gave me emeritus status, which means that they recognised that I could use the lavatory, for example, and so, and since I saw what can be done, I was involved in academic reform. Then the museum had been founded, 1993, and I started volunteering, I didn't want to work here. They don't pay me. 
They say I’m too expensive to be paid. Anyway, I do this, this is not a state museum, keep in mind, this is a museum that is run by an organisation, and I'm the chairman of the board of this organisation. We own and we administer, basically, the museum, but the museum is day-to-day operated by the museum director. I work in the museum and then I'm under the director and when I'm not working in the museum, the director is under me.

I: And why did you choose to educate people about the history instead of searching for more information and just do research?

VN: Why I'm not in academic life anymore? I think freer. In academic life, there are a lot of things that you have to do, and I thought I have done enough for German literature as professor, as publisher of a professional journal, as a reformer, in effect, in America for German studies. So, I had to prove myself to get my salary increase and everything else. Now, I was 65, they said bye-bye, well, it was not a sad parting. I just felt, now I can be free and work for my country and in my country and be home. So, my children and my grandchildren are in America, but they understand that daddy and granddaddy have things to do in Latvia. And I can say, the Latvian anthem, it’s very simple, God save Latvia, God bless Latvia, God bless Latvia, God bless our fatherland, oh God, bless her. Where the Latvian maids are blooming, where the Latvian boys, where the Latvian sons are singing, let us be happy here in Latvia. And I'm happy there in Latvia. So, that much is for me. Yes, but I feel that in the situation that the world finds itself in these days, actually the museum is performing a very crucial task by telling the best we can and the most correctly we can what actually happened, and try not to gloss over, no fake news . Of course, Russia is accusing us of rewriting history, and indeed, we are rewriting history because the Soviets first started rewriting history and the Nazis rewrote history, so we are trying to put the history back where it should have been and trying to tell the truth, so that is very important, and this mission becomes certainly much more important because of what is going on in the internet, something that we didn’t know 50 years ago . The internet is full of everything. We’ll be trying, we’ll try to be honest, we’ll try to tell our story. For the most part, we’ll also want to convince the world that we have a story to tell and that is not simply something that can be, “ah, yes, the same was in Estonia”. Yes, and no, the same in Lithuania, yes and no. So, the Lithuanians, when the Germans said, hey, the Estonians have a Legion, which of course was no Legion, it was not a fighting unit, Latvians have Legion, how about you, Lithuanians, Lithuanians said no. And the Lithuanians had more Forest Brethren after the War than the Latvians and Estonians because they had not decimated their young people in the War. 
You know the Polish story, horrible. If anybody that tells me about the Poles, I say Poles were the ones that got it from both sides first and who showed what Germans, Nazis, and the Soviets thought about Polish interests in being independent, being Poles in Poland after so many years of being spread out . I say, Poles are very proud people. And I think that Germans in the nineteenth century under the Kaiser finally became a unified country. Germans deserved to be a unified country, but they are not entitled to be a country that imposes its will on others. Latvia can't impose its will on others, so there's a difference between big countries and small countries, but we like this Germany much better. And I was in Wrocław a couple of years ago when they celebrated the famous bishops’ letter of coming rapprochement, coming to terms with the past, I think it is very significant. I think it is very significant that Germany accepted what certainly was not by any measure a just thing, making so many millions emigrate, which was caused by the Soviets pushing the borders of Belarus and Ukraine west, so you pushed the whole thing west, the question is whether that had to be achieved by force. In other words, so many injustices have been done there, but that if we dwell on those injustices these days in the world, we will never have the end of it, so the best thing is that we sometimes just simply say, forgive us. Let's not forget it, but let the past not stand in the way of the present and, especially, of the future. . And time and again, we still have this going back, the history to prehistoric times. So, again I say, democracy, the free exchange of opinions and ideas, but especially facts, trying to come near to truth, we will never get the truth, people who are religious would say, truth is up there, only God knows the truth, we can only approach it. I am a literary person. Goethe says in one poem that the poet’s words are only trying to knock at the gates of Heaven asking to be let in. They are trying to be true, but as in poetry, so in life, we can only strive for the absolute truth, and the striving effort is what is important. 
But now ask me, you certainly want to know more about Holocaust because otherwise, Dr Zinke will be angry at you. 
When the Jews were persecuted in Germany, Latvia allowed Jews to come into Latvia and gave them passports, so they could emigrate. This is a little-known story. There is also my story about how the Latvian exile dealt with the Holocaust or rather did not deal with the Holocaust. We are not dealing with the exile as such, but I've been involved a little bit in questions concerning Holocaust and questions of the guilt and the inability to come to terms with the Holocaust in Latvian society, here and in exile. In exile, there were 100,000, 150,000 Latvians, organisations and so on, activities. We kept our Latvian identity alive in the States, in Germany, in Great Britain, in Canada and all over the world. But there were certain things that were not discussed and one of those things that was not discussed was the Holocaust. In Latvia, it was well known that there had been a murder of the Jews, but again it was not a topic of discussion. In the Soviet Union, you probably know, Jews were a hot topic, so they were included in the total losses of civilian population, friedliche Einwohner, as it says in Soviet parlance. “Oh yes, they were Jews”, but the Holocaust was not taken out of the framework. It was not dismissed. In the west, discussions, real discussions about the Holocaust came about late 1960s, 70s. There is an American book by historian Peter Novick, “Holocaust in American Life”, where he points out that Israel was very ambivalent about the Jews that were killed, that allowed themselves to be killed, because they were not the fighters, not the warriors they needed. And only when Israel was threatened, when they felt that they are surrounded by enemy forces, did the Holocaust play a major role: Films, remember, Holocaust on television, Schindler's List and so on. The Holocaust as public discussion became great in the 70s and 80s while of course, there was no discussion and so the consciousness in the west is such, the Holocaust was the only crime against humanity, forgetting that another crime was committed here. And German historians were so politically correct that they were sometimes afraid to raise the question of Stalin's crimes because that would be interpreted as trying to relativise, we don’t have to relativise, we could call all things with the right name, but this is the context of the Holocaust discussion and that's why people here, having suffered under the Soviet regime, don't want to be bothered by it to some extent. To some extent it was not raised officially. I regret that. But it is possible to raise such questions only when the context becomes clearer, and here you will find in the public oftentimes a “Yes, but”. Oh yes, there was Holocaust, but we also suffered. And if only we could get rid of the “yes, but” and say Holocaust, yes, Soviet atrocities yes, but not put the “but” in between them. That any suffering is equal to any other suffering. That is not only I, my suffering, that is important, it is important that other people suffered. And when we are talking to people who were in the Gulag, who suffered in war, who were tortured by the KGB and so on, oftentimes they say, I suffered so much, I cannot forgive. But then you meet people who say, yes, I suffered, but I understand that those who were in the camp with me suffered as well, and those were not only Latvians. A poet whose picture is there in the exhibition, [possibly Latvian, maybe Scandinavian name, 10000.MTS, 25:18], a good friend of mine, I met him first time in Münster at the Latvian Gymnasium when I was playing director for a year in 1989. I took him in my office, and we started talking, and one of the first thing he said was, I was kept for 8 years, and the one thing I learnt, I was not the only one who suffered. And Western Europeans should learn a little bit more about Polish suffering, Latvian suffering, Estonian suffering, German suffering, all those who were suffering under these regimes, who died in Siberia, who were executed, who were killed. And yes, there is one thing that the Holocaust stands out for, that is, it is directed against one specific group with total annihilation. That was Hitler's aim, total annihilation. Latvians, if they had not liberated themselves with Western and other assistances, one more generation, we would still be Latvians here, we would still speak Latvian language, but we would be part of the Soviet Union, we would be as post-Soviet as Belarus, for example. It's the annihilation of national identity, which is the ultimate, not the physical annihilation, and that has been averted, but we have difficult time coming to terms with the difficult past we have, let alone coming to terms with the whole world. We are 50 years behind. And I have sometimes difficulties talking to my compatriots here, look, don't concentrate just on yourself, concentrate on the Syrians right now, concentrate on the Jews, concentrate on the Africans, and if you don’t concentrate, at least keep in mind that you're not the only ones who have the privilege of suffering. It could be a little bit more of this general understanding, but this coming out of yourself is very difficult after you have been into yourself. 

I: I would like to ask just one more question about the museum. Of course, the mission of museum is really ambitious, but have you got a place here for the Jewish community?

VN: The Jewish community has a building and the museum “Jews in Latvia” is located in that building. We have very good cooperation with the founding director of the museum, Marģers Vestermanis, who for example is spending his life concentrating on those people who rescued Jews. He himself escaped from the concentration camp in the last year of War, went to the woods, fought against the Germans with the partisans, and he has spent his lifetime on that, and we have a very good relationship. We have every year an Austrian coming here, a young Austrian in the alternate service, you have that in Germany as well, if you don't want to go into the obligatory, then you can go into the alternate service, and every year, a young Austrian comes here and works half-time in the Jewish museum and half-time in the Occupation Museum. So, the answer is yes, we have good relationships. The Jews do not meet, but the people who were deported, for example, people who publish a book, for example, when they hold a book presentation, it is open to those people who are involved and whose speeches you sometimes see in our exhibition or whose stories are told. We gather life stories, we gather testimonials, Zeitzeugen. We have close to 2,500 on video. You maybe saw those who were young people at the time when they were deported, they are in our collection, they are recollecting how it happened, how they were taken prisoner. And then you put one, two, three, four, five together, and you get a total picture, you get what these young people at that time experienced. And that ends up to the true story of the experience, not only to historical facts, but also of personal experience. And since these stories complement each other, we can come closer to the true psychology of how that, what that meant for these people. So, yes, the former deportees time and again come here, we talk to them. These are the sad stories, there are people whose minds have been affected. For example, once a month we had a lecture, and for seven years, one woman showed up always. And then there was discussion. Her hand always went up. She had only one question: “When will I get my apartment back?” When will I get my apartment back – of course, we couldn’t answer that question. See what the whole experience had done to this woman. She’s not the only one. And then you can admire those who come back and say, “I understand I was not the only one who suffered.” So, I sometimes call those people saints because they use this experience to become better people. And there are some people who couldn’t, who were broken. There are too many of those broken people who don’t want to talk about it. They keep it in themselves, they never tell their children. And then there are those who talk, and I think those who talk and who are willing to share are the ones… The Holocaust question, as I say, has been so long swept under the rug that it is difficult to talk about it. If there was somebody who was involved in the actual killings of the Holocaust -we don’t know, there may be some still, probably old people. It’s difficult to bring up such old items which are not alive, but we have to keep reminding this society. And we certainly avoid doing it, but we will continue with it. We are also the first ones who did the unimaginable thing, put these events side by side in 1941. Imagine, mass deportation, mass killings, and then comes the Holocaust, within one-month time. Psychologically, you can imagine shock after shock, and I would be telling you the untruth id I said that Latvians did not have certain anti-Semitic bias. But they were never violent, they have been no pogroms here. My father had two images of the Jew: Next to the police station at the street was a Jewish tailor, people who made suits, dresses and so on, there were little tailor shops there. He was a policeman; he was walking the beat. And he says, “Oh, where is this Jewish tailor?” And he says, “Officer come on in, I can put cloth, I’ll make you a nice suit”, and he says, “Such wonderful people”. And then there is “the Jew” with emphasis, big, now where does this image of the Jew come from, the Jewish image comes from the Protocols of the Elders of Zion. That’s an insidious document which has left huge impression because we are so apt in thinking in terms of conspiration - unfortunately, but that is a world-wide problem. Look at what is happening in America, whose citizen I am, too. I cannot imagine, Neo-Nazis walking in the city of Thomas Jefferson, shouting anti-Semitic slogans. Where are we coming to? So, it is a world-wide problem and therefore, I think it’s important in combating anti-Semitism of Latvians here, to keep that question alive. I keep reminding the people. But also keep reminding that the other side is not far away, that it’s right across the border. And so, people are worried and must be worried. There could be a resurrection of similar attitudes. So, let’s keep democracy alive, let’s keep asking questions, and let’s insist on getting honest answers.

I: Was there any member of your family who was forced to serve at the Nazi or Soviet army?

VN: No. Not my immediate family, and not the closes relatives that I have. My father during the German occupation served the so-called German self-administration. I am unhappy that at the time when my father was nearing his last days and I interviewed him I didn’t ask him such crucial questions, so I am not sure of how he was involved in the take-over. I know that during the Soviet occupation, he resigned, left the police force, and we lived in our summer home all through winter. I was in Riga with my aunt, I went to school, but he, his wife and my little brother were in this summer home, during the winter, and we were not arrested, we were not taken away in the deportation. So, when the Germans came, I know he worked for the so-called self-administration, and I know he inspected stores to make sure that the supplies are not stolen and so on, and that the book are in order. But obviously, in those terms he participated – as most people had to under military occupation. But I don’t know whether he was involved in any other capacity, so, I also have not really looked for documents, but right now, I feel it’s more important to do my job here and to simply say that I don’t know. But not in a military force – my father was too old, and I was too young. I was 13, if I would have been 16, I could have been called up as the Flagggehilfe, in the German air force as an anti-aircraft operator, supposedly. But those were 16, 17-year-olds. Towards the end of the War, even German young kids went into the army, which is also a crime, of course. Now, there are too many crimes which we should do the best to keep the world honest about. 

I: I have one more question for you. Just imagine, if Holocaust never existed, how would the world look nowadays? Would it be better or not?

VN: If the Holocaust had not occurred? I think it would be much better. For the Holocaust, in many ways, revealed that in the world – You have to imagine, when did we first find out that there was a world? In the 15th century, 16th century! And when did the world become more or less accessible, when was Australia discovered? In the late 1700s. Now, let’s say, 19th century, beginning of 20th century, we were already starting to understand the world, communicate around the world with telegraph and so on, but not the way we understand the world today. The Holocaust that the Nazis instigated broke down the basic moral structure by saying that there are people whom we can eliminate because they are not fit to be part of humanity. There were theories about it, as I heard, the Soviets did it in Holodomor and so on, but concentrated mass murder of one specific group of people to such an extent, it was something, in my opinion, unprecedented, and put in question the moral order. Whether your Christian or belonging to another religion anybody would say, probably, that killing is a crime, killing is a moral transgression – but here, a state, as state policy, was applying these methods. Gas chambers were not used in Latvia. It was only murder by bullet. As for the Polish Jews, Treblinka was opened in 1943, and the Germans didn’t want to kill the Jews on their land, so they sent them to Poland, they sent them here. And now of course, the Poles have to answer for the Nazi murders on their soil. That’s also a crime, in my opinion. But yes, I think that if the Holocaust had never happened, I hope we wouldn’t have come up with the idea that we could simply eliminate a group of people just because they are. But the communists did that, too, because they were trying to eliminate in the so-called Gulags. That’s the Holodomor in Ukraine. But they did it mostly by deportation, sending them away. So, the Holodomor is certainly a crime, but it is questionable to what extent it’s as central as the Holocaust. 
The first book about the Holocaust in Latvia came out in 1995. That’s by Andrew Ezergailis, who is a good friend, historian in America. And he had a great deal of difficulty; he had to take a lot of flak from certain Latvian circles for raising the question and raising the question of the involvement of our people in the Holocaust. I am a member of the historian’s commission - actually, the historian commission has done a lot of research on the Holocaust, unfortunately, most of it in Latvian. We know exactly what happened in small towns, for example, how it was carried out. There was a method to it. It was not a spontaneous action, it was certainly always an organised action, and behind the organisation was Stahlecker with his Einsatzgruppe which did not issue orders, there are no orders. Orders were issued by word of mouth or by telephone. That’s how we understood it. No written orders, because they said, “Do it so they cannot trace it back to you, and make it look as if it was the local populations doing it. But then, in November 1941 came the man whose picture is also there, Friedrich Jeckeln. Friedrich Jeckeln came from Ukraine. He was involved in Babi Yar Massacre, and he did the final killings in November/December with this SS men. But he, of course, involved again support groups, police and so on. If you take 25,000 people out of a ghetto, that’s a huge number of people. They killed them on two separate days. So, for taking them out of the ghetto, making them march kilometres, you have to have an organisation, but the killing was not by Jeckeln’s men. So, to me, the Holocaust is very significant, despite everything else that our people might say. But we, too – I say yes, we, too, but let’s keep in mind that there is a state involved in killing its own people, people who belong to other states, just because they are Jews. But you have to first eliminate the state structures, you have to make the field open and then, anything can happen. Shoot at will. But they didn’t shoot at will, even here, it was very much organised. And so it was, actually, in Poland. Our historians don’t quite believe the Jedwabne story. Cause there has been this sort of picture that the Jedwabne people were murderers and so on. Our people think that was also German-organised, German people certainly weren’t [??, MTS00011, 20:57] present. And if you have military presence, well – don’t trust that what happened. The same in Lithuania, for example, infamous massacre in the Lietūkis Garage. You saw, in pictures, you see German militaries standing around. And according to the Hague Convention of 1907, according to the Hague Convention, the occupying power has to keep order. In other words, the Nazis at that time shouldn’t have stood around, but should have prevented. And so, I’m sorry for those people – Latvians, Lithuanians, Poles, Ukrainians, Belarusians, who participated then. Because they were in such conditions that they may be felt forced, but certainly, the Germans would have been, under international convention, obliged to prevent it, and they didn’t, of course, because that was their policy. And that’s why I think that this breakdown of this very basic order by state is very important and allowed for other things because, well, if you once have started killing, you might as well kill some other people. Kill the Tutsi for instance. That was a genocide, wasn’t it? Well, kill all these people with dark hair. Let’s live, let’s help people live. Angela Merkel is now suffering before allowing the refugees to come in, but it’s in German constitution. And maybe, the refugees came in and that was like breaking down the doors, of course. But lo and behold, if the ultimate right here in Germany comes to power. I like the European Union; we have certain peaceful ideals. It’s difficult to live up to the ideals, but we can strive to approach the ideal, and not leave it out of sight. But I don’t see any other way for the world to survive and for humanity to survive except this way. And the ultimate right answer is, “Roll it all up”, and maybe we will someday. But I hope not. We still have children, we have grandchildren, we have great-grandchildren. Generations are coming, and I hope they will be better, and they will learn a little bit from the past. My regards to Mr Zinke – it’s a great thing that you’re doing, that you’re going after these stories, learning about it. Keep in mind that it’s for the present and future that you are doing it. And that it is not only the task that you’re doing that all of us should be doing.

I: I have a last question. This future generations which you have mentioned a moment ago, do you think that they will be full of grief or hatefulness to those crimes, or to the orders?

VN: They should know that we should first of all intellectually understand that it happened and be convinced that it did happen. Now, there are people who deny it. That’s number one. You have to understand it, and then comes empathy. And then comes enlightenment. You should also be able to feel it, just to know it intellectually is not enough. You read that6 million Jews died in the Holocaust. This numbers game is no good. Peter Novick says, first of all, the number was 11 million victims of Nazis, of all kinds, that was the first. And when the Jewish question, the Holocaust came up, they decided what the figure is. So, they said, dividing sort of fifty-fifty is also is not possible. Let’s make it six million. And then, the rest were forgotten. Six million may be very close to truth. Counting that in Latvia, we had about 90,000, 70,000 of them were killed, and you start adding up the Polish Jews, the German Jews, the Hungarian Jews, the Jews in Russia, the Jews in Belarus, the Ukrainian Jews, and so on. And you’ll probably come up with millions, I don’t know whether anyone has done it the other way, from this general number that is being thrown around. In Latvia, we have been discussing the numbers, but 70,000 seems to be where it is all centred. And that’s a huge number, a huge number. You have to understand it. Then, you start to understand it in human terms, and then you can become a saint, as I say it. A poet, friend of mine said, you suffer, but you understand you’re not the only one who are suffering. And only when we develop that sense, then we can move forward. 
