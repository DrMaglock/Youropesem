\documentclass{book}
\usepackage[utf8]{inputenc}
\usepackage[T1]{fontenc}
\usepackage[gen]{eurosym}
\usepackage{biblatex}
\usepackage{xcolor}
\addbibresource{./bibliography.bib} %something wrong with this fucking bibliography
%information for titlepage
\title{``When the refugees are gone, they'll come after us''\\ Experiences with Anti-Semitism in Central Europe after 1945}
\author{The Youth of Europe Against Anti-Semitism}
\date{}
%numbering and table of contents
\setcounter{tocdepth}{1}
\setcounter{secnumdepth}{0}
%footers and headers
\usepackage{fancyhdr}
\pagestyle{fancy}
\fancyhf{}
\fancyhead[LE]{EXPERIENCES WITH ANTI-SEMITISM IN CENTRAL EUROPE}
\fancyhead[RO]{\rightmark}
\fancyfoot[RO]{\thepage}
\fancyfoot[LE]{\thepage}
%\DeclareUnicodeCharacter{202F}{FIX ME!!}
%begin document
\begin{document}
	\setlength{\parindent}{0pt}
	\setlength{\parskip}{1em}
\maketitle 
%make more beautiful title
%get rid of empty page
\thispagestyle{empty}
\tableofcontents
% implementing the child files: rename your own files and folder according to these commands.
%move heading upwards so the table fits on one page
\section{Introduction}
\vspace*{3em}
In 1949, Theodor W. Adorno asked himself and humanity whether one could still write a poem after Auschwitz, thus stating that the delusional and destructive ideology of anti-Semitism that had led to mass extermination had transformed the European culture and society in a permanent manner: It had become morally impossible not to have Auschwitz in mind when thinking about the self-conception of a society or nation and about the way that individualsposition themselves towards it.A widely held belief, especially in Germany, where a reappraisal of the past has always been a central and contentious issue, is that anti-Semitism has been dis-credited and banned from democratic discourse - however, common experience shows that anti-Semitism persists or even increases, an observation that seems to unmask the much-talked of lessons of history as fruitless.From the writings of Adorno and others, and from the fact that the threat of anti-Semitism in Europe persist, one could pose the question of how it is possible to be an anti-Semite after Auschwitz - a question that was essential in the project that lead to this book, which assembles interviews with Jews and experts on Jewish history as well as professionals dealing with anti-Semitism in Central Europe - namely, Germany, Latvia and Poland - after 1945.
\subsection*{The Youth of Europe Against Anti-Semitism} 
We, the collectors of the interviews, are a group of 29 people: High school and university students, apprentices, language teachers and a historian from Aachen, Berlin, Nuremberg, Munich, Riga and Zamość. Peter Zinke, a historian based in Nuremberg, had the initial idea for the project. A few years ago, he had visited Riga and witnessed what he considered to be a fascist demonstration: The March of Veterans of the Latvian Legion, which takes place in Riga each 16th of March. \\
Back in Nuremberg, he became concerned with anti-Semitic views among his friends, whom he had thought to hold an antifascist and open-minded worldview. Alarmed by these experiences, Mr. Zinke developed the idea for a project that should investigate how anti-Semitism had continued to manifest itself since 1945. He began looking for companions. \\
Shortly before, Mr. Zinke had finished two oral history projects, gathering the life stories of Holocaust Survivors together with high school students and teachers from Sderot (Israel), Nuremberg and Zamość. He convinced some of the participants of these projects to sign up for the new one, specifically nine high school students from Nuremberg, who had finished school by that time, as well as Agnieszka Smalej, a high school teacher from Zamość. Together with Beata Chmura, the head teacher of her high school, Mrs. Smalej persuaded nine pupils to take part in the project. Mr. Zinke also wrote to several Latvian institutions about his project idea. This way, he got in touch with Karina Barkane, Executive Director of the Centre for Judaic Studies at University of Latvia. Mrs. Barkane called on university students from Riga to apply for the project. From among the applicants, she eventually selected eight.\\
Thus the group was complete, comprising in alphabetical order the following people: Aleksandra Adamska, Karīna Barkane, Lingita Lina Bopulu, Gabriel Czajka, Janis Dobkevičs, Janis Dreimanis, Magdalena Freckmann, Davids Gurevičs, Lea Himmel, Cathy Hu, Eliza Koprowska, Emilia Kościk, Kamil Kwarciany, Zuzanna Makiel, Edgaars Poga, Johannes Probst, Jonas Röder, Annika Schmidt, Rafael Schütz, Agnieszka Smalej, Anastasija Smirnova, Dagmara Sokołowska, Patrycja Szala, Vilmars Vincans, Myrjam Willberg, Michael Winter, Aleksandra Wodyk, and Peter Zinke. \\
Together, we applied to the European Union for a grant under Programme Erasmus+ Key Action Cooperation for innovation and the exchange of good practices Action Strategic Partnerships Action Type Strategic Partnerships for Youth/Transnational Youth Initiatives Grant Agreement Number 2016-1-DE04-KA205-013927. We were awarded a grant of up to 54,975\euro{} for a project duration of three years from June 2016 to June 2019. \\
Some participants only took part in the first project activities, but most stayed on until the official end of the project on June 7, 2019. We all learned a lot about Jewish life in Central Europe and about the histories of the towns and countries we come from. Apart from the 60 interviews that we conducted, we visited Jewish schools, synagogues, and cemeteries, and various museums and memorial sites, such as the Memorials of the concentration and death camps in Auschwitz, Bełżec and Majdanek. 
\subsection*{Subject of Research in our Project}
The leading question for our project was how anti-Semitism developed after the Second World War in each of the aforementioned countries. For this reason, most of the interviews were centred around the connection between history, politics and anti-Semitism: In what ways is anti-Semitism connected to feelings of national collective guilt and responsibility with regards to the history of the Holocaust, but also to the feeling of being the victim in this historical process? How do the specific roles of the three countries in the Second World War as well as their political development after the War influence the forms that anti-Semitism takes? Can anti-Semitism be combated through raising awareness of history? \\
When we refer to the term anti-Semitism, we are aware of the fact that a broad variety of theoretical approaches towards this phenomenon exists, and that its definition is highly contested. The question of which definition one adheres has far-reaching implications when it comes to investigating the origins of anti-Semitism or the prospects to overcome it. We don't mean to provide a comprehensive overview on or even a positioning in this debate, but we'd like to at least state that we see anti-Semitism as a system of thinking following its own rationale and, in line with Haury (2002, cf. Beyer 2015: 576-582), as a mindset that boasts the following principles: personification of abstract global processes such as capitalism or modernity, Manichaeism - i.e., the dichotomous divide of the world into "good" and "evil", with "the Jews" functioning as a projection surface and representation of all evil - and the construction of homogeneous groups, with Jews being constructed either as "the other" or as a non-group undermining existing group distinctions. These principles operate both on a social or collective and on an individual, psychological level (Ibid.).\\
 As hinted at in our research questions, we were interested in comparing different countries in order to study the connection between the historical development of a country and the expressions of anti-Semitism that can be found there. We believe that the negotiation of a national self-image is at the core of this connection, and that each of the three countries boasts some specifics in the way its national self-images refer to the Holocaust. In the following section, we will, by no means in comprehensive manner, shortly outline these specifics: \par
In Latvia, being under the control of and suffering from a foreign power is an experience that essentially determines the national collective memory. According to the Preamble to the Constitution of the Republic of Latvia the state condemns both Nazi and Communist regimes. However, the Soviet occupation of 1940, that involved massive deportations of Latvians to Siberia, is often referred to as the major national grievance; against this backdrop, the Nazi invasion is 1941 is perceived as a “lesser evil” or even as a liberation. Consequently, Jewish suffering tends to be marginalised, and the issue of Latvian collaboration or bystander inaction tends to be downplayed. The development of a democratic political culture from 1990 onwards has always involved debates on the question of which historical narrative should be privileged over others, resulting in a reluctance or unwillingness to acknowledge the historical suffering of and the present-day discrimination against ethnic groups other than the ethnic Latvians (Misco 2015). \\
Political and public trends in Latvia substantially depend on problem of perception of the events of the Second World War. As such, in June 2019 the liberal party ``Development/For!'' (``\textit{Attīstībai/Par}'') intended to submit a law to Saeima on compensation to the Latvian Jewish community for the property lost during the Soviet and the Nazi occupations of amount of 40 million euro.\footnote{\textit{Baltic News Network}, June 12th, 2019} This initiative invoked ambivalent reaction of the society and caused a new surge in anti-Semitism; particularly, the Jewish community was misrepresented in regard of its board’s connection with political and financial organisations.\footnote{\textit{Pietiek}, June 16th, 2019} In result, under pressure from society, the party was forced to retract its proposal.\footnote{\textit{Baltic News Network}, June 21st, 2019}\\  
Analogically, the topic of the collaboration of Latvian people during the Nazi occupation and complicity of certain personalities in the Holocaust is sorely viewed in Latvia. The role of a prominent war pilot Herberts Cukurs, who was killed by Israeli intelligence Mossad in mid 1960s, is still ambivalently evaluated. The Jewish community is blamed for intentional defamation of Cukurs and falsification of facts of his biography. Moreover, in the beginning of June 2019, in spite of objection of the Jewish community, the Office of Prosecutor General decided to dismiss the criminal proceedings against Cukurs, since no evidence had been submitted or collected.\footnote{\textit{Public Broadcasting of Latvia}, February 14th, 2019} In addition, it is regularly claimed that during the Second World War Cukurs saved several Jews.\footnote{Gabre (2019); Neiburgs (2019)}\\
In the political constellation sketched above, little attention is being paid to anti-Semitism in public discourse. Findings from Europe-wide surveys show that both the Jewish and the general Latvian population don't perceive anti-Semitism as a major problem in their country: Out of 200 people of Jewish origin that participated in a 2018 study by the EU Agency for Fundamental Rights (FRA)\footnote{The 2018 online survey "Experiences and perceptions of anti-Semitism - second survey on discrimination and hate crime against Jews in the EU" conducted by the European Union Agency for Fundamental Human Rights (FRA) seeks to provide EU-wide data on present levels of anti-Semitism in order to asses to which extent EU member states are fulfilling their obligation to combat anti-Semitism. Therefore, it "analyses data from the responses of 16,395 self-identified Jewish people (aged 16 or over) in 12 EU Member States – Austria, Belgium, Denmark, France, Germany, Hungary, Italy, the Netherlands, Poland, Spain, Sweden and the United Kingdom. These Member States are home to over 96\% of the EU’s estimated Jewish population" (FRA 2018: 7). As response rates in Latvia were low, recruitment methodology and data collection were adapted in order to reach more respondents. This limits the possibility to compare the results from Latvia with those from the other countries. The size of the Latvian sample was n=200, in Germany n=1,233 and in Poland n=422).}, only 12\% considered anti-Semitism to be very or fairly big problem in Latvia, and 77\% thought that it had stayed the same in the last five years before the survey. 6\% had experienced some form of anti-Semitic harassment in that time, and 8\% reported that this had happened to a family member or close friend (FRA 2018: 79). As for the general Latvian society, the Special Eurobarometer 484 that was carried out in December 2018 and investigates research questions similar to those of the FRA study\footnote{The special Eurobarometer 484, a survey which was carried out in December 2018 in 28 member countries of the European Union based on a request by the European Commission, is comprised of the following research interests: (1) To what extent do Europeans think of anti-Semitism as a problem in their country, and how do they assess its recent development? (2) What are the levels of knowledge of and education about anti-Semitism? This also relates to the awareness of means to combat anti-Semitism, and to adequate Holocaust education. (3) How do "conflicts in the Middle East" and the shift of focus influence the way European Jews are perceived in the EU? Methodology: 27,643 people were surveyed, about 1,000 in each country (Germany: n=1,526, Poland: n=1,011, Latvia: n=1,002). A multi-stage random sample was drawn based on regional administrative units. All interviews were conducted face-to-face in the participant’s home.} finds that 14\% of the study participants think of anti-Semitism as a very or fairly big problem, and 55\% percent are in the sentiment that it stayed the same over the past five years. 64\% percent of the respondents think that people in Latvia are not well informed about the history, customs and practices of Latvian Jews, while 30\% think that people are well informed and 6\% say that they don't know (European Commission 2019). While these data, giving an indication of the perception of anti-Semitism in Latvian society, boast values much lower than in Poland and Germany, the Anti-Defamation League’s Global 100 Survey on anti-Semitism of 2015\footnote{Note that there are some methodological difficulties with this survey. For the purpose of this introduction, it is especially problematic that the survey only asks whether respondents think that a certain statement is "probably true", not giving them the opportunity to answer in a more nuanced way. Nevertheless, the size of the randomly drawn samples (n=500) makes it possible to at least mark some general tendencies within the population.} finds that agreement with anti-Semitic statements in Latvia is in fact stronger than in Germany, and while anti-Semitic conspiracy thinking is stronger in Poland than in Latvia\footnote{The survey includes six statements that can be seen as representations of anti-Semitism as conspiracy thinking. The statements Jews have too much power in the business world (Latvia: 51\%, Poland: 52\%, Germany: 28\%) and Jews have too much power in international financial markets (Latvia: 47\%, Poland: 51\%, Germany 29\%) are those of which the largest share of the study’s participants thinks that they are "probably true", while the sentence Jews are responsible for most of the world's wars meets least approval (Latvia: 12\%, Poland: 14\%, Germany: 9\%)}, some of the statements that represent anti-Semitism as an inter-group-conflict are met with about the same agreement in Latvia as in Poland\footnote{\textit{Jews are more loyal to Israel than to [Germany/Poland/Latvia]}: 49\% approval in Germany, 50\% in Poland, 56\% in Latvia; Jews think they are better than other people – Germany 16\%, Poland 30\%, Latvia 39\%; People hate Jews because of the way Jews behave – Germany 30\%, Poland 34\%, Latvia 21\%). As for anti-Semitism related to history politics, 51\% of Germans, 60\% of Poles and 61\% of Latvians participating in the survey thought that the statement Jews still talk too much about what happened to them in the Holocaust was "probably true”.}  These data certainly must be interpreted with caution; in any case they seem to imply that the significantly lower level of problematization of anti-Semitism in Latvia does not directly correspond to lower levels of anti-Semitic thinking in the Latvian society.  \\
The European Union against Racism and Intolerance (ECRI) published its fifth report on Latvia, that discusses recent manifestations of anti-Semitism in Latvia, both in public life and in state practice, on March 5th, 2019. It states that the Jewish community reported on five cases of vandalism and desecration at the Jewish cemetery in Riga in 2016, and Latvian public media reported on four cases of vandalism at the cemetery in Rezekne in 2017 (ECRI 2019: 19). \\
The Supreme Court of the Republic of Latvia reports on 10 incidents of hate speech against Jews that have been prosecuted in the period from October 2012 until March 2018. \par
In Germany, more than 70 years after the Holocaust, anti-Semitism remains an everyday phenomenon. In the process of dealing with the German past, open expressions of anti-Semitism have become tabooed in mainstream public discourse. However, the ideology continues to fulfil the function of a socially and psychologically relieving interpretation of the world, and it gets reinforced by attempts to reject the collective guilt arising out of history. The tabooing has led to a transformation of anti-Semitism, namely through a shift towards a discourse on Israel loaded with anti-Semitic contents that meets broad acceptance in Germany. At the same time, open expressions of anti-Semitism with "traditional" contents are being condemned, which allows to depict anti-Semitism as a marginal and extremist political phenomenon (Busch et al. 2015: 1-3). However, recently, increases in anti-Semitic "everyday" harassment and in anti-Semitic acts of violence can be observed: The police crime statistics, which are published annually and report on all offences with a clearly anti-Semitic background that legal procedures have been initiated against, list 1,799 anti-Semitic crimes (such as harassment and vandalism) and 69 anti-Semitic acts of violence. Both numbers exceed the numbers of the last ten years. The large majority of crimes is committed with a right-wing ideological background, however, the numbers of offences based on a left-wing, religious or "foreign" political background have increased just as well. \\
In the FRA study, 85\% of the respondents from Germany thought of anti-Semitism as a very or fairly big problem. Anti-Semitism on the internet, on the streets or in public places and in the media were assessed to be the most problematic manifestations of anti-Semitism. The study shows that manifestations of anti-Semitism can severely affect the feeling of security of Jewish people: 29\% (in Poland, by comparison, 32\%) of respondents witnessed other Jewish people being verbally or physically attacked in the last twelve months before the survey. 41\% (59\% in Poland) were worried about being harassed or insulted, 25\% (47\%) about being physically attacked. 30\% (36\%) reported the frequent or permanent avoidance of wearing, carrying or displaying in public things that could identify them as Jewish, with security fears being the most frequently reported reason for this avoidance. 74\% (91\%) thought that the government was not combating anti-Semitism effectively.\\  
64\% of the respondents of the Special Eurobarometer 484 thought of anti-Semitism as a very or fairly big problem, 61\% thought that it has increased during the past five years. 74\% percent of the respondents think that people in Germany are not well informed about the history, customs and practices of German Jews, while 22\% think that people are well informed and 4\% say that they don't know. \par
As a means of grappling with its past and specifically with the Nazi and Soviet occupations, attempts have been made in Polish collective consciousness to restore the national configuration of the Interwar Period that signifies stability and autonomy, a process that involved the revival of social institutions such as the Catholic Church and the family. A part of this restauration process was the tendency to avoid the analysis of the younger past, a tendency that sometimes results in a rejection of any responsibility for the wrongs that occurred during the Holocaust (Grudzinska-Gross 2014: 664-666). This avoidance discourse, as Katrin Stoll, one of the historians that we interviewed, phrases it, has become a breeding ground for anti-Semitism. For sections of the Polish political spectrum, it has become a core ideology that tends to intertwine with conspiracy thinking and other ideologies such as an anti-EU or anti-cosmopolitan resentment (Zuk 2017: 84-85). \\
Like in Germany, 85\% of the respondents of the FRA study thought that anti-Semitism as a very or fairly big problem in Poland. The manifestations of anti-Semitism that most respondents thought of as a problem were anti-Semitism on the internet, in the media and in political life. Respondents were shown eight selected possibly anti-Semitic statements\footnote{Respondents were also asked whether they would consider a person voicing one of these statements to be anti-Semitic. The answers given are not itemised by country in the report. For each of the statements listed here, more than 85\% of all respondents said that they probably or definitely would.} and were asked whether they had heard or seen these being made by non-Jewish people. Out of these, most Polish respondents were confronted with the statement Jews have too much power in Poland (70\% - in Germany, by comparison, 42\%), Jews exploit Holocaust victimhood for their own purposes (67\%, Germany 45\%) and Israelis behave "like Nazis" towards the Palestinians (63\% in Poland and Germany). 41\% of the Polish respondents of the Special Eurobarometer 484 thought of anti-Semitism as a very or fairly big problem, 18\% thought that it has increased during the past five years, while another 18\% thought it had decreased, 23\% thought it had stayed the same and 41\% said that they don’t know. 51\% percent of the respondents think that people in Poland are not well informed about the history, customs and practices of Polish Jews, while 39\% think that people are well informed and 10\% say that they don't know. \\ According to the Hate Crime Reporting by the OSCE Office for Democratic Institutions and Human Rights (ODIHR), the Polish police authorities reported on 78 anti-Semitic hate crimes (including physical attacks, vandalism and verbal harassment) that police investigations had been initiated on in 2017. This number was lower than in 2016 (103 crimes reported on), but significantly higher than in previous years. 
\subsection*{Method and Selection of our Interviews}
We made no attempt to select a representative sample of interviewees. Instead, we tried to talk with people of as diverse backgrounds and perspectives as possible. We spoke with Jews about their personal experiences with anti-Semitism, both youth and nonagenarian Holocaust Survivors, laypeople as well as clerics. We interviewed scientists from a variety of disciplines, historians, sociologists, psychologists, and philosophers. We listened to a police officer combating politically motivated crime, an educator dispelling stereotypes about Jews already held by small children, a volunteer preserving the Jewish heritage of his town that is no longer home to any Jews, to a German-Israeli restaurant owner, to a representative of the Human Rights Office in Nuremberg, and even to German witnesses of the Holocaust holding anti-Semitic vies, we conversed with priests, politicians, and publicists.The places where we met our interview partners were Auschwitz Memorial and Museum, Fürth, Forchheim, Ingolstadt, Kraków, Lublin, Nuremberg, Riga, Tel Aviv, Trier, Warsaw, and Zamość. Unless stated differently, all interviews were conducted face-to-face by a part of our group (minimum two people). \\
As diverse as their backgrounds and professions of our interviewees were their conceptions of anti-Semitism. In line with what the interviewees considered as anti-Semitism, they regarded different approaches to be effective in the combat of anti-Semitism: Some thought that you need to convince people that anti-Semitism is a false projection and inner necessity of modern capitalist society, and that only critical thinking could lift these projections. Others thought that anti-Semitic stereotypes would primarily result from a lack of education and thus disappear if only everybody got to know Jewish people and see that their personalities are as individual as everyone else's. Some had completely resigned themselves. Many were engaged in different activities against anti-Semitism. They directed their efforts at children and youth, for example, or tried to gain political influence, yet others wanted to reach people of all walks of life. Some were only concerned about violence directed against Jews, while others were also worried about all discursive expressions of anti-Semitism, therefore reporting any anti-Semitic post on social media to the platform operators. 
\subsection*{On this Publication}
With publishing some of the interviews we conducted, we intend to present a variety of concepts and views on anti-Semitism. This juxtaposition of different perspectives is not complete in any sense. As the interviews don’t directly refer to each other, it is not necessary to read them in any specific order.\\ 
The interviews should be seen as individual narrations that are not representative for any roles, groups or attributes the interviewees are associated with (e.g. nationality, religion, profession, biographical aspects), even if their statements are naturally influenced by these affiliations. \\
The interviews are truncated and grammatically aligned with more common written language. We tried to edit the texts as little as possible and noted all major amendments that were necessary. \\
Even though we don’t raise a scientific claim as we didn’t follow a specific method when collecting and evaluating the data, the publication can be of scientific use, e.g. for systematisation, further theory formation or exemplary illustration of sociological and historiographical concepts related to anti-Semitism. \\
In any case, we hope that the interviews facilitate it for our readers to enhance their understanding of anti-Semitism in its connection with Central European societies and that it will encourage reflection processes, just as it did for us. 
\subsection*{Acknowledgements}
We’d like to thank all interviewees for sharing their life stories and expertise with us. We were amazed by their openness and by the amount of time that they devoted to answer our question. Also, we'd like to express our gratitude to the Erasmus+ fund making this project possible and to all people who have, via their tax payments, financed this project.
\subsection*{Contact Details}
If you would like to learn more about the project or wish to get access to the original transcripts or other material that we have collected during our journeys, feel free to write to \textit{youthagainstantisemitismeurope\textcircled{a}gmail.com}. We will be delighted to hear from any person or project that benefits from our work, and we are also interested in suggestions for collaborations or further processing of the material.

%shall we include the bibliography here, i.e. at the end of each section that contains citiations? Or rather at the very end of the book?
\newpage
\thispagestyle{empty}
\vspace*{15em}
 \textbf{\Huge{THE INTERVIEWS}} 
 %further formatting needed
\section{Poldek (Leopold Yehuda Maimon)}
\begin{otherlanguage}{polish}
\textit{Leopold Yehuda Maimon, called Poldek, was born in Kraków in 1924. He went to a Hebrew elementary school and later to a Zionist grammar school. After the German invasion in 1939, he joined an underground organisation in the Kraków Ghetto. Among other activities, this organisation carried out an attack on a café visited primarily by Wehrmacht officers. At the age of 18, Poldek was deported to Auschwitz, where he also became a part of the underground resistance. Together with four other inmates, he managed to escape during the death march in 1945.\\
After the liberation, Poldek joined the secret Jewish revenge group Nakam. He is no longer totally convinced of all their deeds today. He emigrated to Palestine illegaly in 1946 and was involved in an Aliyah organisation together with his wife Aviva. Today, Poldek lives in a retirement home in Ramat Gan in the outskirts of Tel Aviv. The interview with him took place there on September 20th, 2016.}\par
\vspace*{2em}
\textbf{Poldek:} Urodziłem się w Krakowie w 1924 roku. Miałem starszego brata. Chodziłem do hebrajskiego gimnazjum. To było normalne gimnazjum, takie jak wszystkie. Matura w naszym gimnazjum miała pełne prawa, była jak matura każdego gimnazjum, nie było żadnych kontroli państwowych – mieliśmy pełne prawa.

\textbf{Jakie ma Pan pierwsze wspomnienia, takie najciekawsze, najpiękniejsze, może wspomnienia właśnie z gimnazjum?}

\textbf{Poldek:} Ja mam tylko piękne wspomnienia, to była wspaniała szkoła. Wczoraj do mnie dzwonili z Krakowa, że postawili pomnik jednemu z naszych nauczycieli, który mnie uczył, był wzorem dla nauczycieli. To profesor Ferdhord, uczył języka polskiego. Pisał książki jako Jan Las i był wykładowcą na uniwersytecie. Jak wchodził do klasy, to była taka cisza, że można było usłyszeć muchę. I nigdy nie podnosił głosu, ale miał taki wpływ na uczniów i słuchać go było tak ciekawie, że nikt się nie odważył zrobić czegoś, co by mu przeszkadzało. 

\textbf{Czy jeśli chodzi o język polski, lubił Pan ten język tylko ze względu na nauczyciela, czy miał Pan jakieś zamiłowania humanistyczne?}
 
\textbf{Poldek:} Ja się już wychowywałem w języku polskim, w kulturze polskiej. Wszystko, co czytałem, wszystkie książki były głównie w języku polskim.
 
\textbf{Pana polszczyzna jest piękna. Jeśli tyle lat Pan pamięta tak dobrze język polski, to tylko pogratulować. Fantastycznie, że miał Pan takich nauczycieli.}

\sloppy
\textbf{Poldek:} Tak. Z wielką miłością wspominam moich nauczycieli – wszystkich, nawet takich, którym przeszkadzałem. Szkoła dała nam wszystko. Bez szkoły nie dało się żyć. Do pół do pierwszej żeśmy się uczyli, a po obiedzie była świetlica i można było uprawiać sport np. ping-pong, można było po dworcu grać w piłkę, był ruch harcerski dozwolony i ja też brałem w nim udział. Ruch harcerski z kierunkiem syjonistycznym, ale głównie to wszystko, co harcerzy cechuje, te same podstawowe wartości. Ja byłem syjonistą, zawsze, od 10 roku życia w tym ruchu harcerskim. Był on w szkole jedynym dozwolonym przez szkołę ruchem młodzieżowym.

\textbf{Czy wszyscy uczniowie byli do niego przekonani, czy w jakiś sposób byli skłaniani by przyjąć takie podejście?}

\textbf{Poldek:} Nie, byliśmy przekonani, to wszystko było dobrowolne, nie było żadnego musu, żeby ktoś brał w tym udział. 

\sloppy  
\textbf{Jak Pan wspomina kontakty z młodzieżą z innych szkół, z rówieśnikami, którzy nie byli syjonistami, którzy nie podzielali Pana poglądów i poglądów Pana kolegów ze szkoły?}
 
\textbf{Poldek:} Ja nie miałem żadnych kontaktów, ja w ogóle ich nie znałem, ja byłem za młody, żeby znać, miałem piętnaście lat, jak wojna wybuchła. Przebywałem głównie w szkole, tam się też tańczyło i śpiewało.

\textbf{Czy Pana trudności zaczęły się, kiedy wybuchła wojna?}

\textbf{Poldek:} Wtedy wszystko się całkiem zmieniło. Musieliśmy się przystosować, jeszcze mieszkaliśmy w swoim mieszkaniu do 1941 roku, bo getto powstało w marcu 1941 roku. To było jeszcze w Krakowie, część ludzi uciekła na wschód, gdzieś do Rosji, ale większość została. A myśmy się spotykali, mimo zakazów, młodzież jest młodzieżą i zakazy są zakazami, więc spotykaliśmy się dalej i życie jakoś mijało, jeszcze nie było tak źle, jeszcze nie było tych obozów zagłady. One się zaczęły z końcem 1941 roku. Wtedy Kraków i Polska zostały podzielone na mocy paktu. Część poszła do Rosji, część do Niemiec, a część została środkiem kraju. Została środkowa Polska, a Kraków został stolicą tego kraju. 

\textbf{Jakie trudności zaczęły się dla rodziny po 1941 roku? Jak pan to wspomina, z czym to było związane? Czy z jakąś działalnością Polaków, oczywiście poza okupantem?}

\textbf{Poldek:} Trudności były szczególnie ze strony getta, był w sumie jeden pokój w niedużym mieszkaniu, między dwoma rodzinami… Życie się zmieniło. Nie było głodu, wydzielano jedzenie. Można było żyć normalnie, nie tak jak to miało miejsce w Warszawie. W Krakowie nie było trupów na ulicy. Kraków był czysty. My się spotykaliśmy przy bloku. Ja pracowałem. Głód się zaczął, jak byłem we więzieniu. Ale było ciężko, sprzedawało się, co się miało. Były kontakty z ludnością polską i sprzedawało się rzeczy i w ten sposób jakoś rodzinę utrzymywało. Pracowałem, ojciec pracował, brat pracował. Trochę żeśmy zarabiali. Jakoś się żyło. Zacząłem pracować już w 1941 roku. Musiałem, bo były łapanki. Musiałeś pracować, żeby mieć legitymację pracowniczą. Wtedy mogłeś dalej pracować. Dla mnie w życiu trudności się zaczęły właściwie w 1942 roku.
 
\textbf{Na czym polegał największy ucisk w getcie? Co Państwo odczuwali oprócz tej izolacji, piętna?}

\textbf{Poldek:} Prześladowanie było w momencie, kiedy Niemcy weszli. Co tydzień wychodziło jedno rozporządzenie, np. trzeba było nosić opaskę, że się jest Żydem. Nie można było jeździć. Było wiele publicznych łapanek Żydów. Łapano człowieka i musiał iść do pracy, wracał dopiero po dwóch, trzech dniach. Z każdym tygodniem było gorzej. Obelgi były bardzo trudne. Niemcy mieli bardzo wyszukany proces psychiczny.

\textbf{Próbowali zniszczyć Żydów psychicznie?}

\textbf{Poldek:} Tak, dla nich ten, kto znęcał się nad nami psychicznie, bardzo dobrze pracował. Myślę, że obelgi były jeszcze gorsze niż sama śmierć. Bo jak śmierć, to człowiek od razu umiera, a obelgi były stale. Niemcy mogli robić, co chcieli, nie było żadnego sądu. Oni czuli, że wszyscy Niemcy są nadludźmi. Dzisiaj chce się to trochę zamydlić, ale wtenczas wszyscy tak czuli. Widać to na filmach, na których Hitler mówił i wszyscy patrzyli na niego jak na Boga i wszystko, co powiedział, było święte. 
  
\textbf{Niestety, miał charyzmę, prawda? Złą, negatywną... }

\textbf{Poldek:} No tak! Miał kolosalną charyzmę. On wiedział, jak mówić do Niemców.

\textbf{Niemcy byli nad, a Żydzi byli gdzieś poniżej, a czy naród żydowski odczuwał to, że Polacy byli bliżej w stronę Żydów?}
 
\textbf{Poldek:}  Ciężko powiedzieć, to są rzeczy niebadane. Większość Polaków była antyniemiecka, bo Polacy widzieli w Niemcach wroga. To trzeba wiedzieć. Niemcy byli wrogiem Polaków i Polacy o tym wiedzieli. Ta piosenka, Rota polska, „Nie będzie Niemiec pluł nam w twarz”, była w sercach Polaków, ale w stosunku do Żydów niestety Polacy dzielili się na kilka rodzajów. Byli tacy, którzy wydawali Żydów – to była mniejszość – wydawali Żydów za jedzenie albo pieniądze, inni wydawali Żydów z radością, bo byli antysemitami.

\textbf{Czy duży był procent ludzi, którzy wydawali Żydów dla jakiejś satysfakcji, wyższości?}
 
\textbf{Poldek:} Nie powiedziałbym, że było ich dużo. Te rzeczy są jeszcze niebadane. Ale po wojnie Polacy, którzy pomagali Żydom, bali się swoich sąsiadów, że jak się sąsiad dowie, że on pomagał Żydowi… Polak miał karę śmierci za pomoc Żydowi, nie można o tym zapominać.

\textbf{I wielu ludzi na świecie dzisiaj zapomina, że tylko w Polsce ta kara śmierci obowiązywała, tylko Polacy ginęli za to, że pomagali Żydom.}

\textbf{Poldek:} Tego nie można zapomnieć, ale też było dużo Polaków, którzy podchodzili do Żydów z nienawiścią, bo mieli z tego korzyści materialne, na przykład żydowskie domy, które zostały opróżnione przez Niemców. W Polsce był bardzo silny antysemityzm według mnie, kierowany przez Kościół.
   
\textbf{Czy chodziło też o to, że Żydom często lepiej się powodziło? Byli zaradniejsi?}

\textbf{Poldek:} Nie, to nieprawda. Żydzi byli bardzo biedni. To są tylko opowiadania. Znam Żydów sprzed wojny, bo mój ojciec jeździł po tych miasteczkach, był przedstawicielem rozmaitych firm i na wakacjach z nim jeździłem. Żydzi byli bardzo biedni. Byli też bardzo bogaci, było ich od pięciu do dziesięciu procent, ale zamożnych było jeszcze od piętnastu do dwudziestu pięciu procent. Można powiedzieć, że było dwadzieścia pięć procent Żydów, którzy byli zamożni. Siedemdziesiąt pięć procent to byli Żydzi jak z opowiadań żydowskich, widziałem to na własne oczy. Jeśli Żyd był krawcem, to takim, który robił łaty.

\textbf{Wspomniał Pan, że to z Kościoła wyszły pierwsze sygnały antysemityzmu. Jak Pan to rozumie?}

\textbf{Poldek:} Kościół miał bardzo wielki wpływ na ludność polską, która była bardzo religijna. Nie można pominąć problemu analfabetyzmu w Polsce.

\textbf{Także w dużych miastach? W Krakowie?}

\textbf{Poldek:} Wszędzie. I księża to wykorzystywali. I wiadomość, że Żydzi zabili Jezusa, miała bardzo duży wpływ na mentalność rolników. Rolnik polski też był bardzo biedny. Biedota w Polsce była kolosalna.
 
\textbf{Czy spotykał się Pan z tym, że chociaż Kościół w jakiś sposób, bezpośredni lub pośredni, nawoływał do antysemityzmu, to Polacy będąc katolikami pomagali Żydom?}

\textbf{Poldek:} Kardynał Sapieha pomagał Żydom. Nie można powiedzieć, że wszyscy. Ja mówię o Kościele na wsiach, bo właśnie ludzie niewykształceni, analfabeci żyli z tego, co usłyszeli w niedzielę w kościele, co ksiądz proboszcz powiedział, było święte.

\textbf{Ale i wśród tych nieoświeconych, rolników, chłopów, byli tacy, którzy pomagali ryzykując życie, prawda?}

\textbf{Poldek:} Wszędzie byli. Ja sam miałem podczas wojny przyjaciół Polaków. Nie można mówić, że w stu procentach ludność polska była antyżydowska. Ale jeśli mówimy o atmosferze, to była ona antyżydowska.

\textbf{Proszę powiedzieć o początkach buntu w getcie.}
 
\textbf{Poldek:} Myśmy zrobili pierwszy wypad w Europie, nie tylko w Polsce, nie tylko w Krakowie. Pierwszy wypad na Niemców na większą skalę zorganizowała żydowska młodzież w Krakowie 22 grudnia 1932 roku. Byli w kontakcie z Armią Ludową, ale była to czysto żydowska grupa i zrobiliśmy to dwa dni przed Wigilią – pierwszy napad na tę kawiarnię, gdzie byli przed świętami wyżsi oficerowie niemieccy i gestapo. To zrobili Żydzi, ani jednego Polaka tam nie było, tylko napisane na tablicy, że byli Polacy.

\textbf{Jaka była pana rola w tej grupie?}

\textbf{Poldek:} Mieliśmy w kilku miejscach ludzi. Bardzo ciężko było dostać miejsce zamieszkania dla Żyda, bo Żydowi groziła kara śmierci. Podczas tej akcji byłem szefem, byłem odpowiedzialny za tę grupę, która wyszła na główny napad. Na główny napad wyszli z miejsca, gdzie kiedyś był szpital żydowski. To było 22 grudnia. To była główna kawiarnia cyganerii naprzeciwko teatru. Rzuciliśmy flaszki Mołotowa, pełno tam było wyższych oficerów. Część zawiesiła flagi polskie na budynkach administracyjnych, część rozlepiała ulotki, żeby zachęcić Polaków, by zaczęli się bronić, zachęcić do działania. Później zapaliliśmy ogień na ulicach, żeby straż pożarna jechała, żeby był bałagan.
 
\textbf{Jak duża była ta grupa?}

\textbf{Poldek:} Siedemdziesięciu ludzi.

\textbf{W jakim wieku? Czy tylko młodzi ludzie?}
 
\textbf{Poldek:} Ja byłem najmłodszy. Miałem wtedy osiemnaście lat, inni mieli osiemnaście, dwadzieścia lat.

\textbf{Jak zapatrywali się na te działania Polacy, rówieśnicy, młodzi ludzie? Czy współpracowali? Czy w jakiś sposób pomagali?}

\textbf{Poldek:}  Tak. W Polsce były trzy grupy podziemne. Jedna była Gwardia Ludowa, która była pod nadzorem partii komunistycznej, jedna była nacjonalna narodowa, ona mordowała Żydów, nie przyjmowała Żydów, a kto był w AK albo tacy, co mieli papiery jako Polacy, to tam zgodzili się ich przyjąć, ale z zasady nie przyjmowali Żydów. A Gwardia Ludowa była bardziej była skłonna i ona nam pomagała, współpracowaliśmy.
 
\textbf{Jak wyglądało codzienne życie w getcie, gdy zaczął się już bunt młodych ludzi? Jakie akcje podejmowaliście? Z jaką częstotliwością?}
   
\textbf{Poldek:} Byliśmy grupą trzydziestotrzyosobową. Kiedy dowiedzieliśmy się, że mordują Żydów masowo i nie ma właściwie żadnych szans przeżycia, było to bardzo ciężko przeprowadzić wśród inteligencji żydowskiej. Bo żydowska inteligencja w Krakowie była właściwie kulturalnie połączona z kulturą niemiecką. Kraków był pod zaborem austrowęgierskim, wszystko było niemieckie. Językiem był niemiecki, mój ojciec i matka mówili bardzo dobrze po niemiecku. Bardzo ciężko było zrozumieć, że tak kulturalny naród jak Niemcy, który wydał wielkich pisarzy, wielkich uczonych, wielkich kompozytorów, będzie mordował ludzi tylko dlatego, że mają inną wiarę, trudno było w to uwierzyć. Bo to było naprawdę nie do wiary. Myśmy szukali drogi, jak uciec z Polski, żeby wyjechać do Palestyny, bo ta grupa to byli ludzie wychowani w ruchu syjonistycznym. A to się nie udało, bo tak ściśle było wszystko zamknięte. Niemcy byli bardzo dobrze zorganizowani, pod tym względem niestety nie można było wyjść z sideł niemieckich. Przyszły wiadomości z Wilna, że było bardzo mało ludzi. W Krakowie za dużo nie było. W Krakowie było tak dwadzieścia tysięcy Żydów w getcie. Na początku 1942 roku nie było poczty, wiadomości przechodziły przez dziewczęta, które były kurierami. Te grupy syjonistyczne były we wszystkich krajach, bo ruch syjonistyczny był bardzo rozległy przed wojną. Główny ruch syjonistyczny był w Polsce. Tam była kultura żydowska, młodzież żydowska przyjeżdżała tutaj budować kraj, głównie ludzie z Polski budowali ten kraj. Z początkiem 1942 roku dowiedzieliśmy się, że mordują Żydów masowo. Wtedy rozpoczęły się te rozmowy między nami, co powinna młodzież zrobić, czy dać się zamordować bez żadnego oporu, czy stawiać opór. Bardzo łatwo postanowić, że się stawia opór, ale myśmy byli wszyscy studentami, z bronią nie mieliśmy nic 
wspólnego. Nie było rewolweru, nie było młotka nawet. W końcu przystaliśmy do współpracy z Gwardią Ludową, ona nas przyjęła, pod ich kierunkiem mieliśmy współpracować jako organizacja podziemna. I to polegało na tym, że trzeba było zorganizować jakiś wypad mały na fabryki, gdzieś, gdzie były mundury niemieckie. A raz na kolej, która szła na wschód. Pierwszy rewolwer zdobyliśmy tak, że wyszło nas trzech ludzi na plac otoczony plantami, pod wieczór, było ciemno, napadli na policjanta niemieckiego, z siekierą, zabili go i to był pierwszy rewolwer, jaki zdobyliśmy. Także to były kolosalne trudności budować grupę podziemną, bez pieniędzy, bez poparcia ludności.

\textbf{I bez przygotowania wojskowego.}

\textbf{Poldek:} Byliśmy gotowi do działania. I teraz była ta trudność, że musieliśmy wyjąć ludzi z getta. Po pierwsze zaczęliśmy być znani. I to było po dwóch wysiedleniach dziewięciu i pół tysiąca Żydów do Bełżca, gdzie był obóz zagłady. Dwóm ludziom udało się zbiec stamtąd. Wiedzieliśmy, że nie mamy dużo czasu, że nas w końcu złapią. Niemcy byli bardzo dobrze zorganizowani i mieli pomoc donosicieli. Byli też wśród Żydów donosiciele, Niemcy zorganizowali grupę donosicieli żydowskich, którym za pomoc przyrzekali lepsze warunki, ale nie było donosicieli polskich. Także wiedzieliśmy, że nie mamy dużych możliwości przeżycia, że nasz czas jest bardzo krótki, że musimy działać szybko. 22 grudnia był ten wielki wypad. Wzięliśmy sześćdziesiąt osób, była tam jeszcze jedna grupa żydowska, tzw. "`Iskra"', która była całkowicie pod dowództwem komunistycznym, liczyła też jakieś sześćdziesiąt osób. I razem wykonaliśmy pierwszy napad na Niemców, pierwszy w całej Europie na skalę, można powiedzieć, międzynarodową. 
 
\textbf{I jak się zakończył?}

\textbf{Poldek:} Zakończył się tragicznie. Grupa mieszkała głównie w baraku na ulicy Skawińskich i stamtąd wyszliśmy na to działanie. Ja mieszkałem poza gettem u jakiejś pani, która nie wiedziała, że jestem Żydem, bo miałem polskie papiery, fałszywe, ale polskie. Nie wiem, jak się to stało, ale po tym, po tej akcji na cyganerię, oni przyszli do tego baraku. Były dwie wersje tej historii. Jedna, że jeden z tych trzech, który rzucił granatem na tę kawiarnię, nie był Krakowianinem, swoim zachowaniem zwrócił uwagę i poszli za nim, a on nie wiedząc o tym zaprowadził ich do baraku. Według drugiej wersji wśród nas znalazło się przypadkowo dwóch ludzi z obcych szeregów, którzy zostali zaaresztowani z początkiem grudnia i zostali zwolnieni pod warunkiem, że będą donosicielami. I oni donieśli. Nie wiem, co jest prawdą, w każdym razie widziałem, że wszystkich, którzy tam byli, zaaresztowano. Potem robiliśmy jeszcze jeden napad. Getto zostało zlikwidowane 13 marca 1943 roku i potrzebowaliśmy pieniędzy. Byłem jednym z trzech, którzy dokonali napadu pieniężnego na jakąś rodzinę w Bochni. Napad się udał, ale mnie złapali i spędziłem miesiąc w celi śmierci na Montelupich, a później w Auschwitz dwadzieścia dwa miesiące. I tam też byłem w takiej grupie bojowej, bo w Auschwitz też była grupa podziemna. 
  
\textbf{A czy w podziemiu w Krakowie, jeszcze w czasach getta, zanim trafił Pan do Auschwitz, spotkał Pan jakieś przejawy antysemityzmu ze strony Polaków?}

\textbf{Poldek:} Tam nam pomagali, znaczy mieli pomagać więcej, ale można powiedzieć pomagali częściowo, nie dali nam odczuć, że my jesteśmy już straceni, że można nas zamordować.

\textbf{Co bardziej Panu utkwiło w czasach wypadów, działań: czy pomoc Polaków czy niechęć i na przykład donosy, czy jakieś przejawy antysemityzmu?}

\textbf{Poldek:} Osobiście spotykałem się z Polakami, którzy pomagali. Miałem szczęście, bo przez długi okres miałem wygląd nie tak żydowski, można powiedzieć, mówiłem po polsku dosyć dobrze i miałem włosy jaśniejsze, więc nie byłem podobny do Żyda. Miałem kontakt z Polakami, którzy pomagali, miałem przyjaciela. Spotykałem dobrych ludzi, trochę też złych, ale głównie dobrych ludzi tam w Auschwitz.
 
\textbf{Jak w Auschwitz doszło do tych działań, w których brał Pan udział? Czy Pan był ich inicjatorem?}
  
\textbf{Poldek:} Nie, tam nie było inicjatorów, to wszyscy podjęli. Dostaliśmy wiadomość, że będą chcieli zlikwidować obóz w momencie, gdy Niemcy będą musieli go opuścić. Myśmy zbierali broń i przygotowywali różne sposoby obrony na chwilę, gdy będą chcieli zlikwidować ten obóz. Wyszliśmy z założenia, że jak się będziemy bronić, to uratuje się więcej osób, niż gdy nie będziemy się bronić wcale. Myśleliśmy, że gdy dojdzie likwidowania obozu i Niemcy będą chcieli nas zabić, nie będą mieli dużo czasu, więc gdy będziemy walczyć, to część się uratuje, część zabiją, ale więcej nas się uratuje niż gdyby wszystkich zabili. Obóz był kierowany przez więźniów i była tam walka między trzema grupami, była grupa więźniów niemieckich, była grupa Polaków z AK i była grupa żydowska. Między tymi trzema grupami była walka o miejsca, które były ważne w obozie, od których zależało życie ludzi, na przykład kto był kapo. Grupą, która rządziła, właśnie byli Niemcy, skazani głównie za morderstwa, za większe kradzieże. Druga grupa, polska, miała lepsze warunki niż Żydzi, nie mordowali ich masowo, to byli ludzie, których aresztowano i mieli możliwości dostawania później paczek i utrzymywania kontaktów z rodzinami. To im dawało możliwość przeżycia, bo każda kromka chleba była na wagę życia lub śmierci. I była grupa żydowska, która nie miała żadnych możliwości przeżycia, jedynie trzy, cztery miesiące w Oświęcimiu, jak się nie paliło papierosów, bo jak się paliło papierosy, sprzedawało chleb za papierosy, to życie się jeszcze skracało. Jeśli ktoś miał jakiś specjalny, dobry zawód, Niemcy go potrzebowali, miał szansę przeżyć. Tam, gdzie ja byłem przysłany do Auschwitz, budowano miejsca, w których produkowana była benzyna syntetyczna. 

\sloppy 
\textbf{Z przejawami jakich zachowań ze strony Polaków się Pan spotykał w Auschwitz? A czy Polacy - współwięźniowie - w jakiś sposób pomagali lub szkodzili Żydom?}

\textbf{Poldek:} Ja mam bardzo dobre wspomnienia. Byli Polacy lepsi, byli gorsi, głównie byli antyżydowscy.

\textbf{I czym to się przejawiało, wiadomo, pewnie walczyli o swój kawałek chleba, ale czy donosili?}
  
\textbf{Poldek:} Polakom nie brakowało chleba, bo Polacy dostawali paczki, dlatego mogli przeżyć. Wśród Polaków nie było też selekcji. W Auschwitz co miesiąc była selekcja, kto miał, a kto nie miał pójść do krematorium. Żydzi mieli nie dostawać żadnej pomocy i w takich warunkach mogli przeżyć dłużej ci, którzy byli posyłani do roboty przy budowie, dostawali wtedy jedzenie.

\textbf{Bo byli potrzebni?}

\textbf{Poldek:} Tak, tak, byli potrzebni, więc dostawali trochę więcej jedzenia. 

\textbf{To w takim razie, jaki mieli polscy więźniowie interes w tym, żeby przeszkadzać Żydom?}

\textbf{Poldek:} Żaden, tak po prostu było. Niestety ludzie nie kochają się tak bardzo, jak jeden jest słabszy, to silniejszy go wykorzystuje.

\textbf{Jaką pracę wykonywał Pan w Auschwitz?}
 
\textbf{Poldek:} A, roboty głupie, bo przyszedłem tam z wyrokiem, więc byłem w karnej grupie przez pierwsze parę tygodni. Dostałem podwójne zapalenie płuc i opłucnej, znalazłem się w szpitalu i właściwie esesman mnie zanotował na następny dzień do krematorium. Wiedząc o tym, że jestem ostatnim z tych, którzy zrobili ten napad na cyganerię, powiedziałem, że tam, w celi śmierci było nas wielu, ale oni zostali zastrzeleni na Montelupich. I myślałem, że jestem ostatni z tych i że nazajutrz idę do pieca. A kręcił się tam jakiś człowiek, bo to w szpitalu się chodziło bez numerów, więc nie było widać, czy się jest Żydem, czy Polakiem, czy Niemcem. Ale on bardzo ładnie się zachowywał wśród więźniów, w obozie w Auschwitz nie tak było przyjęte. Poprosiłem go, żeby przysiadł obok mnie, bo chcę mu coś opowiedzieć, w końcu miał taką dobrą pracę, to przeżyje może ten obóz. I przyszedł, usiadł, zacząłem mu opowiadać i okazało się, że on był z tej samej organizacji co ja, i że był dziesięć lat starszy, był farmaceutą, złapali go, jak przechodził przez granicę, chciał uciec do Palestyny. Jak się dowiedział, kim jestem, poszedł do szefa tej podziemnej grupy, bo się okazało, że jestem może najmłodszym więźniem politycznym w Auschwitz. Ten człowiek miał wielki wpływ. Kazał lekarzom zniszczyć moją kartę chorobową, w której było napisane, że na drugi dzień miałem iść do pieca i zapisał mnie jako nowego chorego i tak mnie uratowali. 
  
\textbf{Ja się Pan wydostał z obozu?}
 
\textbf{Poldek:} Uciekłem podczas marszu śmierci z grupą, w której nas zaaresztowano razem. Bardzo chcieliśmy uciec i postanowiliśmy, chociaż nie było możliwości uciec z Auschwitz. Wiedzieliśmy, co nas czeka, jeśli spróbujemy. Jedyna możliwość była prawie nielogiczna, ale była. Uciekliśmy drugiego dnia marszu, udało nam się. To było dwudziestego stycznia czterdziestego piątego roku.

\textbf{Co się z Panem działo po tym, jak udało się Panu uciec z Auschwitz?}

\textbf{Poldek:}  Mieliśmy znowu wielkie szczęście. Uciekliśmy w nocy i szliśmy piechotą, nie więcej niż sześćdziesiąt kilometrów od Auschwitz. I przyszedł do nas jakiś żołnierz niemiecki, byłem pewien, że chciał nas zabić. A powiedział do mnie po polsku, nie wiem, dlaczego akurat do mnie, że musimy się tu gdzieś schować, bo tu niedaleko jest front, a tam całe SS i nas na pewno złapią. I znikł. A niedaleko była jakaś wioska, gdzie jeszcze byli Polacy. I około północy poszliśmy do wioski i zaczęliśmy pukać od okna do okna. Ale kto we wsi w nocy otworzy drzwi, kiedy niedaleko front? Aż słyszę odgłos z jednego domu, w którym przebywała siedemnastoletnia panienka, ja miałem wtedy dwadzieścia jeden. Poprosiłem ja, żeby powiedziała, że jej narzeczony uciekł z obozu pracy i szuka schronienia. Ona spojrzała i wpuściła nas do domu, a tam była jej mamusia, starsza pani i dała nam zezwolenie, żebyśmy tam weszli.

\textbf{Czy był Pan wtedy w obozowym ubraniu?}

\textbf{Poldek:} Tak, bo nie miałem nic innego do ubrania.
  
\textbf{I oni się nie bali Pana przyjąć? Uciekiniera z Auschwitz?}

\textbf{Poldek:} Oni nie wiedzieli, co to jest. W obozie myśmy się przygotowali do tej ucieczki, mieliśmy czarne płaszcze z czerwonym pasem, byliśmy długo więźniami, przeszliśmy wszystkie progi, mieliśmy kontakty, to i jakąś dobrą koszulę, dobre buty udało się zdobyć.

\textbf{A od kogo? Skąd te kontakty, skąd te rzeczy? Od Niemców, czy od Polaków, od Żydów?}
 
\textbf{Poldek:} Nie, to więźniowie, którzy pracowali. Ta grupa nazywała się Kanada. Dlaczego Kanada, nie wiem, ale nawet jak się było więźniem, to się miało wszystkie kontakty, bez tego nie można było przeżyć. A to właśnie dzięki podziemiu miałem rozmaite chody, jak się to mówi. Przechowali nas siedem dni i po siedmiu dniach przyszli Rosjanie i poszliśmy do Krakowa na piechotę, przeszliśmy koło Auschwitz i w moje urodziny wszedłem do Krakowa. 
 
\textbf{Ale prezent!}
 
\textbf{Poldek:} Tak.
   
\textbf{Czy potem spotkał się Pan jeszcze z rodziną?}

\textbf{Poldek:} Nie. Moja mamusia została zamordowana w obozie. Przechodziła przez plac, gdzie się odbywały apele i dowódca obozu "`Skarżysko"' badał, czy dobrze celuje… I zastrzelił ją. A ojca wysłali do obozu w Bełżcu.

\textbf{A co z bratem?}

\textbf{Poldek:}  Z bratem spotkałem się później, dostałem wiadomość w 1946 roku, że brat jest w obozie w Niemczech i pojechałem tam. Byłem już w innej grupie, grupie, która brała zemstę na Niemców. Pracowaliśmy, byłem w Paryżu, w Czechosłowacji, miałem tam zadanie, to był obóz więźniów SS i myśmy tam posmarowali chleb arszenikiem, który miał zabić tych esesmanów.

\textbf{A w Krakowie gdzie Pan znalazł miejsce po powrocie z Auschwitz? Jak się toczyło dalej Pana życie?}

\textbf{Poldek:}Przyjechałem do Krakowa 2 lutego roku z zamiarem jak najszybszego wyjazdu z powrotem do Palestyny. Byłem syjonistą i chciałem jak najszybciej wyjechać. Spotkałem się z człowiekiem z Warszawy, który pomagał Żydom wyjechać do Bukaresztu, a stamtąd mieliśmy nadzieję wyjechać do Palestyny.

\textbf{I od razu się udało?}

\textbf{Poldek:} Nie.
 
\textbf{A jak wyglądała ta droga?}

\textbf{Poldek:} Zostałem jeszcze rok w Europie, bo byłem wtedy w grupie dokonującej zemsty na Niemcach.

\textbf{Na czym polegała ta zemsta, Pana działania?}
 
\textbf{Poldek:} Były różne, moim głównym zadaniem było robienie pieniędzy. Byłem od tego, bo myśleli, że ja się na tym znałem. To był obóz Niemców, w którym były wyrabiane szylingi fałszowane, bardzo dobrze, bo to wyrabiało państwo niemieckie, szylingi niemieckie, które miały w ten sposób zniszczyć gospodarstwa niemieckie. To się działo zaraz po wojnie. Te szylingi były bardzo tanie, a we Włoszech, gdzie o tym nie wiedziano, można było sprzedawać drogo, więc myśmy kupowali to w Niemczech, sprzedawali i za te pieniądze utrzymywaliśmy tę grupę.
 
\textbf{Jak spekulanci dawniej w Polsce.}
  
\textbf{Poldek:} Tak, tak spekulanci, ja byłem wielkim spekulantem.
  
\textbf{Jak długo Pan tak pracował?}

\textbf{Poldek:} Rok. W lipcu 1946 roku przypłynąłem nielegalnym okrętem do Izraela.
  
\textbf{Czy ma Pan jeszcze jakieś wspomnienia, jeśli chodzi o Polaków? Pozytywne, negatywne? W tym ostatnim okresie, kiedy wyszedł Pan z Auschwitz, jakie wtedy było nastawienie Polaków do byłych więźniów Auschwitz?}

\textbf{Poldek:} Postanowiłem w Polsce nie zostać, mimo że miałem kolosalną pomoc ze związku partyzanckiego. Raz mnie przyjęli do związku partyzanckiego i dostałem legitymację partyzanta i otworzyli mi drogę na studia, wszystko chcieli mi dać, chcieli mi dać stopień kapitana, chcieli mi naprawdę pomóc, ale jak przyszedłem do Krakowa, to pierwsze, co usłyszałem od tego stróża, gdzieśmy mieszkali "`To tyle Żydów was zostało?"' 

\textbf{Jak Pan myśli, dlaczego cały czas było takie samo nastawienie do Żydów, pomimo że Żydzi tyle przeszli, mimo że Polacy mieli poczucie, że to, co działo się w Auschwitz, było bestialstwem, bo przecież Polaków to też dotykało?}
  
\textbf{Poldek:} Nie mogę na to odpowiedzieć, to dla mnie niezrozumiałe. Nienawiść do Żydów była ogromna po wojnie. Kielce były, Rabka, niebezpieczeństwem było jeździć po Polsce dla Żyda. A to nie było kierowane przez państwo, to nie było zorganizowane przez nikogo, to były pojedyncze grupy, ludzie sami zabijali ludzi. Wiemy, co się działo w Polsce po wojnie, dlaczego się to działo, jest dla mnie największą zagadką, nie potrafię jej rozwiązać, niestety. Jestem związany z kulturą polską, ja się nie wstydzę tego, że byłem w Polsce wychowywany, mam wiele sentymentów do Polski, kocham Kraków, ale z wielkim bólem to mówię, że takie wypadki niestety miały miejsce po wojnie i tego nie można zrozumieć. 

\textbf{Ten ból nosi Pan w sobie jakby w imieniu narodu. A czy po przyjeździe do Izraela utrzymywał Pan jakieś kontakty z Polską? Czy wracał Pan do Polski?}

\textbf{Poldek:} Tak, byłem w Polsce, byłem zaprzyjaźniony z rodziną, która uratowała członka naszej organizacji, przychodziłem tam do nich, mieszkali w Prokocimiu, był tam syn w naszym wieku, byliśmy bardzo zaprzyjaźnieni. 

\textbf{Odwiedzał Pan Kraków i dom, w którym Pan mieszkał?}

\textbf{Poldek:} Do domu mnie nie chcieli wpuścić, nie chcieli mi otworzyć drzwi.

\textbf{Dużo razy był Pan w Polsce?}

\textbf{Poldek:} Tak, robiliśmy film o Krakowie, o tej grupie, w ten sposób się spotykaliśmy. Żydzi, różni Żydzi, szczególnie ci starzy, ale też inni mają wiele pretensji do Polski, do ludności polskiej i to jest dosyć zrozumiałe. To trzeba zrozumieć. Mimo że Polacy nie byli z Niemcami, byli przeciwko Niemcom, ale zachowanie Polski, narodu polskiego nie było… dość to ciężki temat…
 
\textbf{My, Polacy cierpimy teraz z tego powodu, bo często te głosy, że jesteśmy antysemitami, że byliśmy antysemitami, są głośniejsze niż te, że pomagaliśmy. Przecież ogromna część Polaków jest wśród Sprawiedliwych Wśród Narodów Świata.}

\textbf{Poldek:} Tak, to się należy Polsce.

\textbf{A przecież inne narody, na przykład Holendrzy, też wydawali Żydów… My cierpimy, że na naszych terenach było Auschwitz, były obozy.}

\textbf{Poldek:} No, nie, nie tylko. Żydzi europejscy byli głównie w Polsce, trzy i pół miliona Żydów było w Polsce, a w Holandii ilu było Żydów? Trzysta tysięcy? Pięćset tysięcy? Żydzi jako grupa byli w większości w Polsce. Teraz nie ma już tylu Żydów w Polsce, w Krakowie jest osiemdziesięciu ludzi zapisanych jako Żydzi, wcześniej było sześćdziesiąt pięć tysięcy.

\textbf{A czy myśli Pan, że takie zachowanie wobec Żydów to jest cecha narodu polskiego? Czy gdyby padło na jakikolwiek inny naród, na przykład Czechów, Węgrów, Włochów i na ich terenie byłyby obozy koncentracyjne, obozy śmierci, czy zachowywaliby się podobnie? Czy to tkwiło w mentalności Polaków?}

\textbf{Poldek:} Nie, przecież antysemityzm jest na całym świecie, tak nie można powiedzieć, tylko zmiana była w Niemcach taka, że potrafili zmienić kierunek. To bardzo dziwne dla mnie, niezrozumiałe, bo właściwie główna nienawiść powinna być między Żydami a Niemcami. A tak nie jest, Niemcy są dzisiaj największymi przyjaciółmi z Izraelem.
\end{otherlanguage}
\section{Jakub (Yaakov)}
\begin{otherlanguage}{polish}
\textit{Jakub (Yaakov) was born in Frampol, Poland, in 1928. Together with his family, he spent three years (1939-1941) in Biłgoraj Ghetto in Poland. He used to get out of the ghetto to get some food and got caught. He thought his days were numbered, but the German who caught him agreed to let him live in his house. Jakub escaped to Russia in 1941 and later to Palestine. During his life, he learned and worked in many different jobs. He speaks six languages. Today, he lives in a retirement home in Ramat Gan in the outskirts of Tel Aviv, Israel. The interview took place there on September 21st, 2016.}\par
\vspace*{2em}
\textbf{Jakub:} Urodziłem się we Frampolu w 1928 roku. Kiedy w trzydziestym dziewiątym roku zaczęła się wojna Niemiec z Polską, przyleciały samoloty, słyszałem je, miałem dziesięć lat. Zdążyliśmy uciec z domu, ja, ojciec, matka, siostra i brat. We Frampolu, zaraz za miastem jest taka góra i tam się schowaliśmy, może ze dwie godziny żeśmy tam siedzieli, aż ustał szum bomb. Poszliśmy zobaczyć, co się dzieje w mieście. Niewiele zostało, rozbili całe miasto, zostały tylko synagoga, kościół i szkoła, w której się uczyłem. Wszyscy, którzy nie zdążyli uciec, zostali zabici. Nie wiedzieliśmy, co robić.  

\sloppy
\textbf{Proszę opowiedzieć więcej o swoim dzieciństwie, o szkole.} 

\textbf{Jakub:} Długo się nie uczyłem, skończyłem dwie klasy we Frampolu, wybuchła wojna i już więcej nie chodziłem do szkoły. Jak rozbili nam miasto, nie było gdzie szukać jedzenia i miejsca do spania. Mój ojciec znał ludzi, którzy mieszkali za miastem. Chodziliśmy tu i tam, by trochę zjeść. W sadzie szukaliśmy jabłek, w jednej chałupie dali mleko, w innej mięso. Nie mieliśmy gdzie iść, liczyliśmy na to, że może w mieście coś zorganizują. Poszliśmy do miasta, do szkoły, tam, gdzie ostały się jeszcze domy. Niemcy byli we Frampolu i wzięli nas do Biłgoraja. Biłgoraj to większe miasto. Tam była granica z Rosją, tam jeździły pociągi. Rozdzielili nas w mieszkaniach, stworzyli getto. Jeść dawali - trochę zupy i chleba. Ale ludzie umierali z głodu. Jeden Niemiec, naczelnik obozu, Hans powiedział do mnie "`idź pracować"'. No co zrobić, poszedłem do niego pracować. Ciężko było.

\textbf{Czy oprócz jedzenia Polacy udzielali Wam pomocy, gdy chodziliście po wioskach? Nie byli wrogo nastawieni?} 

\textbf{Jakub:} Dali nam jeść, ale nie pomagali więcej. Bali się Niemców, za ukrywanie Żyda mogli zostać zabici. Żyliśmy w tym obozie w Biłgoraju trzy lata. Ludzie dorośli byli brani do pracy, zbierali kamienie i wywozili do Niemiec. Ja pracowałem u tego Niemca.

\textbf{Co się później z Państwem stało?} 

\textbf{Jakub:} Byłem trzy lata w getcie. A kiedy już złota nie było, trzeba było kraść. Poszedłem w nocy, wiedziałem kiedy Niemiec śpi, przekradłem się przez druty i udałem się do wioski. Było bardzo ciemno, wszedłem w siano, ale do jedzenia nic nie znalazłem. To było chyba w tysiąc dziewięćset czterdziestym pierwszym roku, zima była ciężka. Wyszła Ukrainka, bo usłyszała szum. Poznała, że ktoś rozrzucił siano. Wzięła widły i zaczęła mnie nimi kłuć. Mam jeszcze dziury na nogach. Dźgała mnie i krzyknęła "`Antek, złapałam Żyda"'... Ledwo uciekłem z tego siana. Ona zawołała tego Antka, żeby pojechał po Niemca do miasta, bo w tej miejscowości nie było Niemców, bali się partyzantów, bandytów. Ten wziął wóz i pojechał po Niemca, żeby mnie zabrał. Trwało to jakieś pół godziny. Przyjechał ten Niemiec, spojrzał na mnie i mówi do mnie "`uciekaj"'. I uciekłem. Matka z ojcem stali w oknie całą noc, nie spali. Przybiegłem do domu, matka płakała. Nie chcieliśmy tak dalej żyć. Rozpoczęła się wojna Niemiec z Rosją. Uciekliśmy do Rosji. Wykradliśmy się i pojechaliśmy pociągiem. Jechaliśmy dwie niedziele. Przyszedł jakiś rosyjski oficer i spytał: "`Czy wiecie, dokąd jedziecie? Wy jedziecie do Rosji"'. Dał jakieś dokumenty i podpisał, zapytał o obywatelstwo. Powiedział, że pojedziemy na Sybir. Nie było widać słońca, same lasy. Razem z nami były inne rodziny. Dali nam siekiery do rąbania drzew. Pracowaliśmy tam i było ciężko, jeszcze gorzej niż w getcie. Brakowało jedzenia, musieliśmy kraść, ale jakoś przeżyliśmy. W czterdziestym szóstym roku przyjechaliśmy na Dolny Śląsk i tam zacząłem pracować w kopalni. Szkoły nie było, uczyć się nie było czasu, więc poszedłem, żeby pomóc ojcu. Zrobili ze mnie stachanowca, wybrali mnie, robiłem sto dwadzieścia procent i poszedłem do kibucu. Przyjechali z Izraela uczyć nas, przygotowywać do życia w kibucu, bo miał powstać Izrael w czterdziestym ósmym roku. Byliśmy jeszcze młodzi. Kiedy wyjeżdżaliśmy do Izraela, chcieliśmy przejść granicę z Czechosłowacją, ale nie przepuścili nas. Wróciliśmy. Jeszcze dziesięć lat zostałem na Dolnym Śląsku i nadal pracowałem w kopalni. Jeździłem do Warszawy, byłem delegatem jako jedyny Żyd, który pracował w kopalni. W końcu przyjechaliśmy do Izraela, to znaczy nie dali nam wyjechać, aż się zamienił rząd. Gomułka doszedł do władzy i pozwolił. Tak to było. I przyjechaliśmy tutaj, szukałem pracy, ale nie było łatwo, jedzenia też nie było, bo dawali na kartki.

\textbf{Czy utrzymuje Pan kontakty z Polską, ma Pan tam znajomych, przyjaciół?} 

\textbf{Jakub:} Mam przyjaciół w wielu miastach w Polsce. Byli tutaj nieraz, Polacy z Warszawy robili film, nagrywali, chodziliśmy nad morze razem, byliśmy u mnie w mieszkaniu. Żyliśmy jak dobrzy koledzy, a kiedy wyjeżdżali, płakałem. Miałem wielu kolegów, ale już minęło tyle lat. Ale jak żyła moja żona, to myśmy rozmawiali w domu po polsku. Polska jest mi jeszcze winna, ponieważ ja dziesięć lat pracowałem w kopalni po wojnie. Były delegacje z Ameryki, chcieli mnie zabrać do Ameryki, ale ja nie lubiłem tak jeździć. 

\textbf{Jakie jest Pana ogólne wspomnienie Polakach?}

\textbf{Jakub:} Są różni ludzie na świecie, dobrzy i niedobrzy. Czasem się zdarzy pośród tysiąca ludzi jeden, co będzie zły. W zasadzie to nie mam złych wspomnień, dawali nam jedzenie. 
\end{otherlanguage}
\section{Miriam Linyal}

\textit{Miriam Linyal (*1922) was born in Poland, in a village near Poznań. In 1942 she was transported to the ghetto in Łódź, from there she was taken to a work camp where she was carrying rocks. She returned to the Łódź ghetto and in 1944 she was sent to Auschwitz. All her family died there. After liberating the camp, she worked in a factory in Germany, later she came back to Poland. After a short time, she emigrated illegally to Palestine. Today, she lives in an Elderly’s home in Ramat Gan in the outskirts of Tel Aviv, many of whose residents are German-speaking Survivors. The interview took place in there on September 21st, 2016.}\par
\vspace*{2em}
\textbf{Gdzie się Pani urodziła i gdzie mieszkała?} 

\textbf{Miriam:} Urodziłam się i mieszkałam w Koźminku, niedaleko Kalisza. Miałam ładne życie, bardzo spokojne, przyjemne.

\textbf{Miała Pani rodzeństwo?} 

\textbf{Miriam:} Tak, siedmioro, ale ja byłam najstarsza w domu. Czworo z nich zginęło, a troje żeśmy przyjechali tutaj do Izraela. Mieszkaliśmy razem z rodzicami. Chodziło się do szkoły, było tam dużo kolegów. Byli Polacy. W ogóle nie czuło się, że to małe miasteczko. Byliśmy bardzo zadowoleni, dopóty nie wybuchła wojna.

\textbf{ Co wtedy się stało? Jakie są Pani wspomnienia?} 

\textbf{Miriam:} Jak Niemcy weszli, była katastrofa. Wojna wybuchła pierwszego września, trzeciego już byli u nas. Niemcy weszli, byli po drodze do Warszawy. Tej samej nocy już zabili pierwszego Żyda. Szukali, co robić z nami. Dzień w dzień było gorzej. Jedzenia nie było. Każdy się bał, bał wizyt w domu. Ja rozumiem dlaczego. Bo jest wojna, to się boję. Wzięło się nas zaraz do pracy i tak dzień w dzień coś innego.

\textbf{Czyli nie zabrano Was do jednego miejsca, by pracować, tylko codziennie jakieś inne zajęcia?} 

\textbf{Miriam:} Nie, na początku codziennie do innej pracy. Szukali tylko, co z nami robić złego. Zabrali nas do pracy. Jak była praca w polu, to było w porządku, ale zabrali nas do jakiejś pracy, której w ogóle się nie potrzebowało. Żyło się w strachu. Żyło się w bardzo niedobrej sytuacji, bo jeden dzień nie był podobny do drugiego dnia. 

\textbf{Czy w tej miejscowości, z której Pani pochodzi, mieszkało dużo żydowskiej ludności?} 

\textbf{Miriam:} Myśmy byli w domu aż do czterdziestego roku. W czterdziestym roku posłali nas do Łodzi. Łódź była pełna. Już było getto w Łodzi, nie było miejsca dla nas. Zaprowadzili nas z powrotem z Łodzi do Koźminka. W czterdziestym roku otworzyli getto. Myśmy nawet nie wiedzieli, co to znaczy getto. I byliśmy w Koźminku dwa lata, od czterdziestego do czterdziestego drugiego roku. Dzień w dzień coś innego, dzień w dzień wysyłali nas do innej pracy. Na przykład jednego dnia prosili o listę dzieci od jedenastu lat do czternastu. Wśród tych dzieci był mój brat. Zabrali ich do Niemiec do pracy i wiem o tym, że on potem zginął w Oświęcimiu. Możecie sobie wyobrazić, jak jedenastoletnie dziecko bierze się do pracy? Do jakiej pracy to dziecko było przyzwyczajone? Zabrali też dzieci za małe, nie mogły pracować w getcie, nie było tam dla nich pracy.  

\textbf{A jak wyglądały warunki w getcie?} 

\textbf{Miriam:}  Życie było ze strachem. Przypadkowo dostałam dobrą pracę, pracowałam na żandarmerii, czyściłam i to była najlepsza robota, bo czasami, jak oddałam wiadro i skończyłam pracę, to kobiety dawały nam kawałek chleba. I tak się pracowało codziennie. Dzień w dzień napady, dzień w dzień szukali Żydów, wyjść nie wolno 

\textbf{Ile miała Pani wtedy lat? } 

\textbf{Miriam:}  Jak wojna wybuchła, miałam szesnaście i pół roku. Dwa lata później była zgroza. Były płacze, były krzyki dzieci. Co chcieli, robili z nami. Nie wolno było wyjść z getta. Było strasznie. I siedzieliśmy w getcie przez dwa lata do czterdziestego drugiego roku. I później wysłali nas z powrotem do getta łódzkiego. Wówczas już było miejsce, bo w łódzkim getcie bardzo dużo ludzi umarło. Nie było nic do jedzenia. Do pracy się chodziło. Praca to nie była najgorsza rzecz, ale że mordowano dzień w dzień kogoś innego. Z rana mam rodziców, a wieczorem już nie mam albo z rana matka miała jeszcze dziecko, a wieczorem już dziecka nie było. Nie wiadomo, dokąd ich wysłali, nie wiadomo, co z nimi zrobili, ale jak się zna historię, to się wie, że zginęli w jakimś obozie. W łódzkim getcie na początku pracowałam w kuchni. Później szyliśmy buty dla żołnierzy niemieckich.

\textbf{A czy do getta w Łodzi trafiła cała Pani rodzina?} 

\textbf{Miriam:} Przyjechałam do getta łódzkiego z rodziną. Moja cała rodzina zginęła. My troje żeśmy zostali przy życiu.  

\textbf{Czy kiedy byli Państwo w getcie, mieli Państwo kontakty z polską ludnością?} 

\textbf{Miriam:} W ogóle nie było kontaktu. Nie wolno było wyjść z getta ani wejść do getta. Życie się nasze skończyło z tym.

\textbf{A czy Państwo w getcie stawiali jakiś opór, buntowali się, czy przyjmowali los, jaki zgotowali okupanci?} 

\textbf{Miriam:} A co mieliśmy robić? Nie wolno nam wyjść, my nic nie mamy do mówienia. Nie ma gazety, nie ma niczego. Raz na miesiąc dostajesz jedzenie, chleb raz na tydzień. Życie było okropne, ale starało się być zadowolonym. Chodziło się do pracy, siedziało się z rodzicami, płakało się razem, śmiało się razem. Życie się toczyło, nie wiadomo było, jak długo to jeszcze potrwa, ale jednak było życie i to się jednego dnia także skończyło... 

\textbf{Kiedy zostali Państwo przewiezieni do Auschwitz? Czy bezpośrednio z łódzkiego getta? }

\textbf{Miriam:} W czterdziestym czwartym roku. Myśmy w ogóle nie wiedzieli, co to znaczy Oświęcim. Słyszało się, że Rosjanie nadchodzą, ale nie wiedziało się dokładnie. Były różne wersje, może Rosjanie jeszcze wejdą, może zostaniemy przy życiu. To nie była droga, taka normalna, jak wejść do pociągu i przyjechać. Myśmy weszli do pociągu, ale nie wiedziało się, dokąd nas wiozą. To była zgroza. Nie ma człowieka, nie ma malarza, żeby mógł wyobrazić, co znaczy chwila w Oświęcimiu. W ogóle nie wiedzieliśmy, że jest Oświęcim. W ogóle nie wiedzieliśmy, że jest taka zagłada. Nikt nie wiedział, co się z nimi dzieje… Nie wierzę, że można przejść Oświęcim i jeszcze żyć i być człowiekiem. Kiedy przyjechaliśmy, mężczyźni zostali oddzieleni od kobiet. Widziałam ojca z daleka, "`Szalom"' nawet nie powiedziałam, bo nie było można dojść. Matka trzymała jedno dziecko przy ręce. Siostra poleciała do niej, żeby być z nią razem, żeby ktoś mógł pomóc matce… Nie udało się, ona została z dzieckiem, z siostrą, która miała pięć lat. Więcej nie widzieliśmy mamy… Zabrali nas, obcięli nam włosy we wszystkich miejscach. Jeden nie poznał drugiego. Chodziło się, patrzyło na twarze. Jak patrzyłam na swoją siostrę, nie mogłam w żaden sposób poznać, czy to ona. Wyglądaliśmy jak małpy. Tak nas zostawili na kilka godzin. Później zabrali nas do sauny. Nikt z nas nie wiedział, co to jest sauna. Myśleliśmy, że to ostatnie chwile naszego życia. Krzyki "`Szma Israel"'. Strasznie płakaliśmy. Godziny mijały za godzinami. Dali nam duże buty, nikt nie mógł ich nosić, bo były za duże. Trzymało się je w rękach. Dostałam spódniczkę, która była dla mnie kilka numerów za duża. Godziny trwało, zanim nas wysłali do baraków, a na podłodze nie mieliśmy pryczy. Pięć nas było, trzy siostry i dwie kuzynki. Cały czas nas liczyli. 

\textbf{A czy w obozie Państwo kontaktowali się z Polakami, czy były możliwości spotkania?}

\textbf{Miriam:} Nie, skąd! Nawet z siostrą nie mogłam mówić, to było okropne. Nie można było komu powiedzieć, że nam ciężko. 

\textbf{Kiedy Pani wydostała się z obozu?} 

\textbf{Miriam:} W tym samym dniu, kiedy się skończyła wojna. Później nas posłali znowu do pracy.  

\textbf{Dokąd?} 

\textbf{Miriam:} W Sudetach, w Niemczech. Ja byłam pięć i pół roku. Pracowało się bardzo ciężko, jako młode dziewczynki, może od trzeciej nad ranem, czasami do dwunastej w nocy. 

\textbf{A z Niemiec wróciła Pani do Polski?} 

\textbf{Miriam:} Polacy zabrali nas do Polski, do Łodzi. Wyjechało się z Łodzi, to i wróciło się. A tam dokąd masz pójść? Przecież nie jestem łodzianką, a jeśli nawet, to ktoś już mieszka w moim domu, nie wiesz kto. Na ziemi się leżało, na stacji kolejowej. Ale byliśmy szczęśliwi, że dostawaliśmy kawałek chleba do jedzenia. Nie można sobie wyobrazić, co to jest kawałek chleba. Czekało się, dokąd pójść, gdzie nocować, co dalej będzie. W nasze grupie była jakaś dziewczynka, co wyszła na ulicę i spotkała siostrę, która była w innym obozie. Uciecha! Siostra miała kolegę, który słyszał, że w jakimś miejscu na Północnej jest mieszkanie puste. Dach tam trzeba było zreperować. A kto to robi? Nie ma forsy, nie ma niczego. I tak się zaczęło to życie. Słyszałam, że ciotka moja żyje, moja mama była z rodziny z trzynaściorgiem dzieci, tylko ciotka jedna jedyna została przy życiu. Miała dwoje dzieci. Całe rodziny tam zginęły.

\textbf{A czy po powrocie do Łodzi dostaliście jakąś pomoc od Polaków? } 

\textbf{Miriam:}  Nie, nikt nam nie pomógł. 

\textbf{Dlaczego?} 

\textbf{Miriam:} Nie wiedzieliśmy, do kogo pójść, nikt się nami nie interesował. Rosjanie jeszcze tam byli. Szłam do miasta, szukałam pracy. Spotkałam kolegę, z którym pracowałam w kuchni. On mnie nie poznał, byłyśmy bez włosów, suknie jeszcze były z numerem. Powiedział, że przed wojną mieli fabrykę pończoch i rodzice dali to Polakom, ale jak ktoś przeżyje, to mieli mu oddać. I Polacy mu oddali tę fabrykę i dał mi pracę. Jak już zaczęłam pracować, mogłam pojechać do ciotki do Kalisza. Ja i moje dwie siostry żeśmy pojechałyśmy do ciotki. Uciecha! Uciecha i płacz. Tuliło się do ciotki jak do matki. 

\textbf{Czy dużo Żydów wróciło do Łodzi? }

\textbf{Miriam:} Bardzo mało, bardzo. Było nas tysiące, ale ile wróciło? Trzynaścioro rodzeństwa w rodzinie mojej matki i prawie nikt nie został… W każdej rodzinie było tak samo, nie było jednej rodziny całej.  

\textbf{Jak długo zostaliście w Łodzi?} 

\textbf{Miriam:} Niedługo. Wyjechaliśmy zaraz.

\textbf{Dokąd? Do Izraela?}

\textbf{Miriam:}  Ja przyjechałam do Izraela. Polacy nas po drodze złapali, siedziałyśmy we więzieniu. Nie tak szybko było. Później żeśmy postanowiły, że nie ma dla nas miejsca i wyjeżdżamy do Izraela. Przyjechałyśmy nielegalnie, ale to jest nieważne. Każdy z nas chciał tylko, żebyśmy mieli swój własny kraj. 

\textbf{Czy ktoś wam pomagał?} 

\textbf{Miriam:}  Była pomoc, z wyjściem to już była pomoc. Była organizacja, wszystko przez organizacje. I zaczęło się życie tu w kraju, bardzo ciężkie życie. Tu wojny się zaczęły, przecież nie było państwa. Dopiero teraz powstało państwo, ale byliśmy szczęśliwi, że będzie kraj. 

\textbf{Czy utrzymywaliście jakieś kontakty z Polską?} 

\textbf{Miriam:} Ja już byłam trzy razy. Pojechałam do Oświęcimia z dziećmi, chciałam, żeby widzieli Oświęcim. Byłam w Warszawie, w Krakowie, byłam w różnych miejscach. Ale o Polsce nic złego nie mówię. Ja na początku bardzo tęskniłam za Polską. Polacy w Kielcach zabili Żydów, co wrócili z Rosji. Jest tablica, że tu leżą ci, którzy przeszli wojnę i ich zabili, bo byli Żydami. Ale jednak ja tęskniłam strasznie za Polską. 

\textbf{Czyli te dobre wspomnienia zostały w Pani bardziej?} 

\textbf{Miriam:}  Nie odczuwałam nic złego, było mi dobrze. Pojechałam do naszego miasteczka. Przeszłam kilka razy przed naszym domem, był zamknięty. Nasz dom… Nie mogłam wejść i zobaczyć co się tam dzieje. 

\textbf{Dlaczego?} 

\textbf{Miriam:} Nie wiem. Nie weszłam, nie widziałam nic. Może jak bym poszła do gminy, to by mi otworzyli, ale nie mogłam… 

\textbf{A tam nikt nie mieszkał? } 

\textbf{Miriam:} Teraz mieszkają. A jedna rzecz mi się nie podobała w Polsce. Powiedziałam to burmistrzowi w Koźminku, że nie mogę zrozumieć tego, że wsadzili Polaków do synagogi. Teraz tam Polacy mieszkają! W synagodze. Powiedziałam "`Nie wstydzicie się?"' Tego do dzisiejszego dnia nie mogę zapomnieć. Dobrze znam burmistrza, przyjeżdżają czasami do mnie i wtedy ich przyjmuję i wszystko dobrze, ale uraz mam tylko o to. Ale co można zrobić? Jak jest wojna, to wszystko jest odwrotnie. 

\textbf{Czy więcej dobra czy więcej zła doświadczyła Pani od Polaków?} 

\textbf{Miriam:} Nie wiem. Pamiętam wszystko dobre, nie pamiętam złego. Myśmy żyli bardzo dobrze z Polakami. Nasi sąsiedzi to byli Polacy. Żyliśmy bardzo dobrze, dopóty wojna nie wybuchła, później to zupełnie coś innego. Po wojnie także byłam kilka razy, było wszystko w porządku. 

\textbf{A w czasie wojny Polacy nie pomagali Żydom?}

\textbf{Miriam:}  Nie mogli, nie można tego wymagać. Niemcy nie pozwolili. 

\textbf{A czy zdarzały się sytuacje, że ktoś pomimo tego ryzykował życie i pomagał Państwu?} 

\textbf{Miriam:}  Nie. Ale ja nie mam pretensji. Mam pretensje tylko do Niemców, a do nikogo więcej nie mam. Antysemityzm zawsze był i zostanie jeszcze, tak mi się zdaje. 

\textbf{A czy ma Pani pomysł, skąd on się wziął w Polsce? Bo często się mówi, że Polacy są antysemitami.} 

\textbf{Miriam:} Nie mogę tego powiedzieć. Myśmy bardzo ładnie żyli z sąsiadami Polakami. Chodziłam do polskiej szkoły, uczyłam się z Polkami. Jeden jest taki, drugi jest taki i podczas wojny… Przeszło i żeby więcej nie było takich wojen. 
\section{David Geballe} 
\begin{otherlanguage}{ngerman}
\textit{David Geballe (*1981) grew up in Hamburg. At the age of 16, he began to work as a youth leader with Jewish youth. He studied in Berlin, New York and Jerusalem, in 2006 he received the rabbinic dignity. Since 2011 he has worked as a rabbi in Germany, first in Munich and then in Fürth, where he was in charge of the Israelite religious community at the time of the interview. In addition to the rabbinical activity, he was involved among others in the Jewish fraternity of students }(Jüdischer Studentenverbund Franken)\textit{and the board of the Society for Christian-Jewish Cooperation }(Gesellschaft für für christlich-jüdische Zusammenarbeit)\textit{. Since September 2017, he is in charge of the Jewish Community of Duisburg-Mülheim / Ruhr-Oberhausen as Chief Rabbi.\\
The interview took place in Fürth in 2016, it was conducted in German.}\par
\vspace*{2em}
\textbf{Könnten Sie sich zu Beginn kurz vorstellen?} 

\textbf{David Geballe:} Mein Name ist David Geballe, geboren bin ich im fernen Hamburg, bin jetzt seit nicht ganz sechs Jahren Rabbiner hier in der Gemeinde und habe dementsprechend viel Kontakt mit den Gemeindemitgliedern, aber auch mit den anderen Gemeinden hier in der Umgebung, und dadurch auch einen relativ guten Einblick, wie diese momentan stehen. 

\textbf{Wie sieht dieser Einblick denn aus?} 

\textbf{David Geballe:} Dass es seit der Syrienkrise schwieriger geworden ist, dadurch, dass die hunderttausende Leute mit einem muslimischen Erziehungshintergrund, die nach Deutschland gekommen sind, von klein an durch die Eltern und Medien in den arabischsprachigen Ländern nicht gerade zu Judenliebe erzogen worden sind und ein ganz anderes Bild haben, was Juden oder Nicht-Muslime angeht. In den letzten vier Jahren sind, nach meiner Erfahrung und nach dem was man hört, die allermeisten Probleme aus diesem Klientel. Das soll nicht heißen, dass es unter Deutschen oder anderen Europäern keinen Antisemitismus gibt oder gegeben hat – den gibt es leider immer noch, aber der wird heutzutage als „Israelkritik“ verkleidet oder damit gerechtfertigt, dass man „so etwas unter Freunden ja noch sagen darf“. Muslimisch geprägter Antisemitismus ist oftmals noch direkter Antisemitismus oder teilweise noch Antijudaismus – was es in Europa in dieser Form und Ausprägung seit 150 Jahren nicht mehr gibt. Aber Antisemitismus bleibt Antisemitismus, egal, wie er sich verkleidet. 

\textbf{Gibt es mehrere Formen des Antisemitismus? Welche würden Sie da aufführen?}

\textbf{David Geballe:} Es gibt die drei Schulmeinungsansätze, angefangen vom Antijudaismus, der sich gegen die Religion selbst richtet und vor 150 bis 200 Jahren zum reinen Antisemitismus weiterentwickelt hat. Dieser wurde nicht primär auf die Religion, sondern auf das Volk bezogen. Nach dem Sechstagekrieg 1967 wurde das dann schleichend zu einem Antizionismus oder Antiisraelismus, bei dem die Israelis nicht mehr als Opfer des zweiten Weltkriegs, sondern als Täter angesehen wurden und damit das Feindbild waren. Das ist die Entwicklung der letzten Jahrzehnte. 

\textbf{Und glauben Sie, dass sich das in allen gesellschaftlichen Gruppen in Deutschland gleich entwickelt hat, oder gab es da Unterschiede? Sie haben ja zum Beispiel den muslimischen Hintergrund angesprochen.}

\textbf{David Geballe:} Klar, man kann die verschiedenen Gruppen kaum miteinander vergleichen. Jemand, der in einem muslimisch geprägten Land aufwächst, wo im Fernsehen offiziell Kinderserien laufen, in denen Juden als Nichtmenschen dargestellt werden - wenn dieses Kind einmal erwachsen wird, hat es gar nicht die freie Wahl, Juden nicht zu hassen. Es ist so in diese Person und ihre Psyche eingebaut, dass es schon wirklich einen sehr besonderen Menschen braucht, um diese Kette zu durchbrechen. 

\textbf{Meinen Sie, das kommt vor?} 

\textbf{David Geballe:} Es gibt durchaus ein paar berühmte Beispiele. Zum Beispiel den Sohn von einem der Hamas-Führer, der diese Kette durchbrochen hat. Der tritt auch auf, um diese Meinung nach draußen zu bringen1. Dann gibt es auch in Deutschland einen, ich glaube er ist Historiker. Auf jeden Fall gibt es ein paar Leute, die diese Kette durchbrechen, das sind leider die Ausnahmen und nicht die Regel. Ich würde jetzt auch nicht unterstellen, dass automatisch jeder Muslim ein Antisemit sei, um Gottes Willen. Aber es sind bestimmte Vorurteile, die vor und insbesondere nach 1967 durch diese Kultur geprägt wurden, mit den “drei Neins” gegen Israel, kein Frieden, keine Anerkennung, keine Verhandlungen. Das hat sich zu einem politisch geprägten Antisemitismus entwickelt. 

\textbf{Und abgesehen von den Leuten, die aus muslimisch geprägten Ländern kommen - wie hat sich das nach Ihrer Einschätzung in Deutschland nach dem Krieg entwickelt?} 

\textbf{David Geballe:} Es gab hier in Fürth und auch anderswo Fälle, in denen man auch nach dem Krieg Leute mit dem offiziellen Persilschein angepöbelt hat. Der Persilschein war auf dem Papier und nicht in den Köpfen der Leute. Vorstände der jüdischen Gemeinde durften sich dann Dinge anhören wie "Schade, dass wir dich nicht auch noch bekommen haben" und so etwas. Klar, das gab es immer nach dem Krieg, durch die ganze Geschichte durch, teilweise auch in der Justiz, weil die hohen Beamten irgendwo herkommen mussten. Es war so, dass Altbeamte mit übernommen wurden, die auch in der Nazizeit eine nicht unwichtige Rolle gespielt haben. 

\textbf{Glauben Sie, dass diese Übernahmen tatsächlich notwendig waren?} 

\textbf{David Geballe:} Klar ist, dass Anfang der Fünfziger einfach Polizisten, Richter und sowas gebraucht wurden. In der DDR war das anders, weil dort Leute aus der Sowjetunion hingeschickt und eingesetzt wurden. Inzwischen gibt es historische Nachforschungen, laut denen auch dort nicht alle Beamten und Parteifunktionäre während der Nazizeit eine wirklich weiße Weste hatten. Aber es war in beiden Deutschlands eigentlich so, dass Leute mit übernommen wurden. Mussten sie, mussten sie nicht? Hinterher ist man immer klüger, aber bei manchen stellt sich die Frage, hätten die wirklich in so eine wichtige Position kommen sollen oder nicht? 

\textbf{Diese Anpöbelungen gegen Vorstände der jüdischen Gemeinde, war das in der unmittelbaren Nachkriegszeit oder zieht sich das bis heute hin?} 

\textbf{David Geballe:} In den Fünfziger und Sechziger Jahren gab es das alles, teilweise auch heute noch. Natürlich sind die ganzen Altnazis inzwischen nicht mehr in der Lage, so zu pöbeln. Wenn sie noch am Leben sind, sind sie in Alters- oder Pflegeheimen. Aber es vergeht keine Woche, in der jüdische Gemeinden in Deutschland nicht mit Drohbriefen – sowohl mit als auch ohne Namen – angeschrieben werden, die teilweise sehr deutlich bestimmte Gesetze brechen und nicht wirklich angenehm zu lesen sind. Zum Beispiel, als die sogenannte Beschneidungsdebatte war, war es sehr, sehr schlimm, da haben Leute mit hohem Bildungsgrad – Doktoren, Rechtsanwälte, teilweise auch Lehrer und Direktoren an Schulen – Briefe geschrieben die unterhalb jeder Gürtellinie sind. Diese Leute hatten auch keine Probleme, ihre Namen, Adressen und Titel in voller Gänze auf ihre Briefe zu schreiben. Nicht so wie früher, wo so etwas anonym und aus Zeitungsschnipseln gemacht war, das ist inzwischen selten geworden. Heute ist es so, dass die meisten Briefe anfangen mit "`Unter Freunden wird man das ja wohl noch mal sagen dürfen..."' oder "`Das sagen wir nur, weil wir ja euch so mögen"' und so etwas wie "`Aber was ihr mit den armen Palästinensern oder mit euren armen Kindern macht, das darf doch wohl nicht sein"'. Also wenn ein Brief so anfängt, dann ist man sicher, entweder unten oder oben stehen der volle Name und die Adresse dieser Person. Es hat sich ein bisschen eingebürgert: Wenn man diesen magischen Satz sagt, "`unter Freunden wird man das ja noch sagen dürfen"', dann ist alles, was danach kommt, nicht mehr antisemitisch, sondern nur freundlich gemeint. 
Ich denke, es ist so, dass Antisemitismus ein Krebs ist, der niemals ganz geheilt werden kann. Es wird immer Antisemitismus geben, es hat ihn auch schon immer gegeben. Die Frage ist nur, ist es gesellschaftsfähig oder nicht? Während der Nazizeit war er sogar Staatsraison, was er Gott sei Dank seitdem nicht mehr ist, aber heute wird keiner mehr öffentlich aufstehen und sagen "`Ihr Juden seid so und so"', sondern es ist eher Israel, das man angreifen kann. Genauso wie jetzt Anfang des Jahres, wo ein deutsches Gericht entschieden hat, dass ein versuchter Brandanschlag auf eine Synagoge keine antisemitische Straftat, sondern nur eine politische Aussage gegen den Staat Israel im Gaza-Krieg ist, was für mich eine absolute Schweinerei war. Was soll man dazu sagen? Natürlich fällt diese Straftat dann nicht in die Kategorie von Antisemitismus und dadurch kann man auch diese ganzen Listen und Szenen von antisemitischen Straftaten klein halten. 

\textbf{Gab es in den letzten Jahren auch Antisemitismus in Fürth, in Bezug auf den Gaza-Krieg zum Beispiel?} 

\textbf{David Geballe:} Zum Glück gab es in den letzten Jahren keinen gewalttätigen Antisemitismus in Fürth, aber mehr oder weniger verdeckten, in Form von Briefen, Anrufen oder E-Mails. Neulich erst kam eine E-Mail, die an mehrere Rabbiner und Gemeindevorstände und berühmte Juden gerichtet war und diese aufgefordert hat, zu einem Kriegstribunal vorstellig zu werden wegen Verbrechen gegen die Menschlichkeit und Sonstiges. Diese stammte von Leuten die nicht dem rechten Spektrum zuzuordnen sind, sondern eher dem linken. 

\textbf{Nochmal zu den Medien - Sie bekommen also Briefe, E-Mails und Anrufe?} 

\textbf{David Geballe:} Alle Möglichkeiten der Telekommunikation werden da ausgeschöpft. 

\textbf{Ich nehme mal an, in den sozialen Netzwerken ist der Antisemitismus weniger verkleidet oder versteckt, weil er dort anonym geäußert werden kann.} 

\textbf{David Geballe:} Man braucht nur auf YouTube gehen und irgendein Video, das mit Judentum zu tun hat, anzuklicken, und dann sieht man gleich zweierlei Arten von Kommentaren. Einmal die muslimisch geprägten, die dementsprechend anfangen, und dann die nichtmuslimischen, die dann mit Beschneidung oder sonst etwas anfangen. So etwas wird man überall finden, auf Facebook muss ich gar nicht eingehen. Als vor etwa drei Jahren die Beschneidungsdebatte war, war es dort sehr, sehr schlimm. Sobald irgendein Bericht in der Zeitung erschien, der auch online zu lesen war, gab es darauf natürlich Kommentare. Es war teilweise nicht mehr lustig, das mitzulesen. Von irgendwelchen Pädophilie-Vorwürfen gegenüber allen Juden und so weiter. 

\textbf{Wie sieht es mit antisemitischen Äußerungen von einer esoterisch-tierschützenden Bewegung zum Schächten aus, zum Beispiel?}

\textbf{David Geballe:} Gibt es auch, zum Glück sind die in Deutschland noch nicht so stark. In anderen Ländern, z. B. in der Schweiz, ist das mit dem Schächten verboten, was auch zum großen Teil vom Tierschutz mitgetragen wurde. Das Interessante ist, wenn man einmal mit diesen Leuten spricht und sie nach wissenschaftlichen Beweisen fragt, dann kommt eigentlich nur warme Luft zurück. Also, es ist nur ein Nebenthema, diese klassischen Angriffe gegen das Judentum sind genauso substanzlos wie sie immer waren. Das Krasseste, was ich einmal erlebt habe, ist, dass ein Lehrer mich bei einer Schulführung mit einer katholischen Religionsklasse gefragt hat, was für ein Blut heutzutage für Matzen benutzt wird. Am Anfang dachte ich, es wäre ein schlechter Scherz, aber das war es leider nicht. 

\textbf{Ich bin Historiker, ich habe ein Buch über Antisemitismus in Franken während der Weimarer Republik geschrieben und dabei ist mir aufgefallen, dass es in Fürth einen ziemlich starken zivilgesellschaftlichen Widerstand gegen die Nazis und gegen den Antisemitismus gab – ganz im Gegensatz zu Nürnberg, wo sich die Arbeiterklasse zurückgezogen und nichts dagegen gemacht hat. Ist jetzt noch etwas in Fürth zu merken von einem zivilgesellschaftlichen Widerstand gegen Rechte Gruppierungen?} 

\textbf{David Geballe:} Jein. Bei den Fürthern, die wirklich noch ein bisschen die Geschichte kennen, ist es immer noch ein bisschen im Kopf drinnen, dass Fürth eine Art fränkisches Jerusalem ist und so etwas. Das ist durchaus bewusst, aber weil die Gesellschaft heute allgemein viel schneller umzieht, stellt sich die Frage, wie viele von den in Fürth lebenden Menschen wirklich Fürther oder Franken in diesem Sinne sind. Ich meine, ich bin selbst keiner, und wird eine Person, die aus Berlin, aus Stuttgart oder ich weiß nicht was herkommt, wirklich wissen, was Fürth für eine Geschichte hat? Vielleicht nicht, vielleicht ja, vielleicht schafft sie es mal, hier ins jüdische Museum oder so etwas, aber das war es auch schon. Das Wissen ist in dem Sinne heute nicht mehr so stark ausgeprägt, wie es früher mal einmal war. Man muss natürlich sagen, dass die Stadt an sich, also beispielsweise der Oberbürgermeister, sich dessen voll und ganz bewusst sind. Es gibt daher auch ein sehr gutes Verhältnis zwischen der Stadt und der Gemeinde, in beide Richtungen. Die Stadt hilft uns, wo immer sie kann, wir helfen der Stadt, wo immer wir können. Das ist schon da, auf jeden Fall gibt es also Überreste davon. 

\textbf{Können Sie sich bei der aktuellen Erweiterung des Jüdischen Museums Franken in Fürth einbringen?}

\textbf{David Geballe:} Das Museum hat ja einen Trägerverein und auch eine wirkliche wissenschaftliche Leitung, die dort natürlich hauptsächlich dort bestimmt. Natürlich gibt es auch eine Zusammenarbeit zwischen dem Museum und der Gemeinde, aber Museum und Gemeinde sind getrennt. Ich bin mir ziemlich sicher, wenn ich einen Vorschlag machen würde, würde man darüber nachdenken und darüber sprechen, aber im Großen und Ganzen bin ich mit dem alten Teil zufrieden. Ich denke daher, dass auch der neue Teil gut ausgestattet werden wird. 

\textbf{Was halten Sie generell von der Bildung, die in Deutschland über das Judentum vermittelt wird, also in den Schulen oder auch anderswo?} 

\textbf{David Geballe:} Ich erweitere das mal ein bisschen mit der Bildung über den Zweiten Weltkrieg. Auf der einen Seite zu viel, auf der anderen Seite zu schlecht. Das soll heißen, dass die meisten Schüler, wenn wieder das Thema Holocaust oder Zweiter Weltkrieg in der Schule hochkommt, sich denken „ah, nicht schon wieder“. Das ist natürlich genau das, was man nicht erreichen möchte. Und an einer Schule, an der 80\% der Klasse aus Muslimen besteht, muss man natürlich ganz anders unterrichten. Es gibt Berichte, ich habe von einer Lehrerin aus Berlin gelesen, am deren Schule wirklich 90\% der Schüler*innen Muslime sind. Als dann Fotos von KZs gezeigt wurden, haben die angefangen zu klatschen. Das ist natürlich genau das Gegenteil von dem, was man damit erreichen möchte. Von daher denke ich, es wäre sinnvoll, sich neue Konzepte zu erarbeiten, die dann für die Lehrer quasi verpflichtend sind. Vielleicht nicht ganz so oft, aber dafür wirklich besser. 

\textbf{Was würden Sie sich da vorstellen?} 

\textbf{David Geballe:} Natürlich kann man eine Klasse von einem Gymnasium in Bayern auf dem Land nicht mit einer Gesamtschule in Berlin vergleichen. Das Klientel ist ein ganz anderes, das muss natürlich auch dementsprechend angepasst sein. Es gibt keine magische Lösung, die immer funktioniert, aber es gibt Gott sei Dank genügend Leute, die von der Erarbeitung von Konzepten und Bildungsmaterialien mehr Ahnung haben als ich. Wie das genau aussieht, weiß ich nicht, darüber habe ich noch nicht wirklich nachgedacht, aber dass Handlungsbedarf besteht, da bin ich mir ziemlich sicher. 

\textbf{Ich war ein paarmal bei den Wochen der Brüderlichkeit dabei und mein persönlicher Eindruck war, dass das Verhältnis zwischen deutschen Christen und deutschen Juden etwas ritualisiert und bemüht ist. Ist dieser Eindruck richtig?} 

\textbf{David Geballe:} Die Woche der Brüderlichkeit an sich ist etwas sehr Gutes, sehr Tolles, es ist ja auch geschichtlich gesehen eine Abstammung vom \textit{reeducation programme} der Amerikaner. Das Problem ist, wenn man sich jede Veranstaltung in der Woche der Brüderlichkeit anschaut und das Durchschnittsalter der Zuhörer oder Mitwirkenden berechnet, wird man jenseits des Rentenalters sein. Immer wenn ich eine Schulklassenführung in der Woche der Brüderlichkeit habe, frage ich die Schüler*innen, ob sie wissen, was für eine Woche gerade ist. Ich habe einmal bisher eine Antwort bekommen, aber auch nur, weil der Vater von einer der Schülerinnen evangelischer Pfarrer und deswegen bemüht ist. Deswegen wusste die Tochter zufällig, was das ist, aber sonst kamen dann Fragen zurück, vielleicht Champions League-Woche oder so etwas. Obwohl es doch eigentlich prädestiniert wäre, in den Schulen dazu etwas zu machen, passiert das in den allermeisten Fällen nicht. Es gibt hier in Fürth zum Glück eine sehr gute Ausnahme, das Helene-Lange-Gymnasium, wo es schon wirklich Tradition ist, am Donnerstag in der Woche der Brüderlichkeit eine Veranstaltung zu machen, die auch für die höheren Klassen verpflichtend ist. Das vordere Viertel der Stuhlreihen ist für Ehrengäste reserviert, aber der Rest sind wirklich Schüler. Ich finde, dazu ist die Woche der Brüderlichkeit da und nicht dieses gegenseitige „piep, piep, piep, wir haben uns alle lieb“, das ist etwas stilisiert und schon fast eingeübt und nicht wirklich der Sinn und Zweck des Ganzen. 

\textbf{Was genau wird am Helene-Lange-Gymnasium an diesem Tag gemacht?} 

\textbf{David Geballe:} Es gibt jedes Jahr einen Hauptsprecher von verschiedenen Quellen, vor ein paar Jahren war z. B. jemand vom israelischen Konsulat in München hier und hat eine Rede gehalten. Dieses Jahr war es ein wissenschaftlicher Mitarbeiter der Uni Frankfurt, der über die deutschen Nazi-Prozesse berichtet hat. Es gibt verschiedene Vorträge, die ein Thema wirklich vertiefen anstatt oberflächlich zu bleiben, also wirklich ein \textit{reeducation programme}. Nicht einfach Larifari, wir haben uns alle lieb, sondern wirklich Wissen, das vermittelt wird. Gott sei Dank ist Dummheit eine Krankheit, die geheilt werden kann. Ich bin der festen Überzeugung, dass Vorurteile nur dort wachsen können, wo Wissen fehlt. Das macht es nur umso schlimmer, dass die Woche der Brüderlichkeit nicht viel mehr in den Schulen thematisiert wird. 

\textbf{Und was bieten Sie persönlich in der Gemeinde an? Führungen für Schulklassen haben sie schon erwähnt, gibt es noch andere Sachen?} 

\textbf{David Geballe:} Im Durchschnitt mache ich zwei Führungen in der Woche. Meistens von Neuntklässlern, weil in der neunten Klasse andere Religionen sowie Traditions- und Toleranzbegriffe im Lehrplan stehen. Da sind eigentlich die meisten Führungen, es gibt auch ein paar andere. Manche machen es schon in der vierten Klasse, obwohl ich das persönlich nicht so gerne mache, wegen des Alters und der Aufmerksamkeitsspanne. Wenn ich wirklich jede Anfrage annehmen würde, was rein zeitlich leider gar nicht mehr geht, würde ich manche Wochen mit zehn Führungen haben, aber es gibt Gott sei Dank auch genügend andere Dinge zu tun in einer Gemeinde, für die ebenfalls gesorgt werden muss. 

\textbf{Wie sind Ihre Erfahrungen aus den Führungen, sind die Schüler interessiert oder muss man sie wirklich dafür begeistern?} 

\textbf{David Geballe:} Es kommt darauf an. Meiner Erfahrung nach hängt es davon ab, wie gut die Kinder darauf vorbereitet wurden. Kinder in der neunten Klasse sind ja Jugendliche, meistens so um die 15 Jahre alt. Was ich eigentlich allen Lehrern mitgebe, wenn sie nach einer Führung fragen, ist „ja, aber nur wenn Fragen auch im Unterricht vorbereitet werden“, ich muss hier ja keinen Frontalunterricht durchführen. Dafür ist mir meine Zeit viel zu schade. Wenn die Schüler einfach nur die Augen verdrehen und sich fragen, wann es endlich zu Ende ist, dann ist das eine Zeitverschwendung für sie und für mich. Das heißt, ich will diese Stunde sinnvoll nutzen und das funktioniert meistens dann gut, wenn die Schüler gut vorbereitet wurden. 

\textbf{Welche Fragen stellen Schüler Ihnen, wenn sie welche vorbereitet haben?} 

\textbf{David Geballe:} Es gibt Verschiedenstes. Also es gibt Fragen, die sich aus dem Unterricht ergeben haben, auf die der Lehrer oder die Lehrerin keine Antwort wussten. Meistens gebe ich auch eine Einführung in das Judentum. Ich habe es leider oft gesehen, dass Lehrer bei dem Thema eine kleine Nachbildung gebrauchen könnten. Es kommt nicht nur vor, dass Informationen fehlen, sondern auch, dass teilweise wirklich falsche Dinge beigebracht wurden. Ich erläutere die wichtigsten Dinge im Judentum, wie eine Kurzfassung, und dann bleiben wir bei ein paar Themen stehen, die für Jugendliche vielleicht einen Anreiz haben oder sie vielleicht selbst bewegen. Dabei entwickeln sich meistens auch sehr viele Fragen. 

\textbf{Das kann ich nur bestätigen, ich war ein halbes Jahr in der Lehrerfortbildung bei Berufsschullehrer*innen tätig. Das war mein härtester Job, weil alles, was Lehrer nicht wollen, ist belehrt zu werden. Und das Wissen war wirklich gleich Null, also erschreckend schwach. Haben Sie das bei Religions- oder Geschichtslehrern erlebt?} 

\textbf{David Geballe:} Also die meisten, die kommen, sind Religionslehrer, weil es dort im Lehrplan verankert ist. Geschichtslehre hatte ich noch nicht, wenn ich mich nicht täusche. Es gab mal einen Philosophiekurs, aber Geschichte noch nicht. 

\textbf{Und was ist mit anderen Gruppen, also Nicht-Schüler?}\footnote{This question, as well as the foloowing one, was originally asked at a later stage of the interview; for this publication, the passage was inserted here in order to avoid thematic breaks.} 

\textbf{David Geballe:} Sind meistens das andere Spektrum, das heißt Rentengruppen, meistens überwiegend von Kirchen oder so etwas. Ich hatte auch schon einmal den Rentnerclub der Siemenswerke in Erlangen. Bis dahin wusste ich nicht, dass es so etwas gibt, aber das ist ein Club von ehemaligen Siemens-Angestellten, die verschiedene Ausflüge anbieten. 

\textbf{Wie sind da Ihre Erfahrungen?} 

\textbf{David Geballe:} Die Erwachsenengruppen sind meist doch etwas fragaktiver, um es so auszudrücken. Es gibt aber auch dort Leute, die – wie sie eben gesagt haben – nicht belehrt werden wollen. Es gibt auch einige, die mit einer festen gebildeten Meinung über Juden oder das Judentum herkommen und diese eigentlich nur bestätigt haben wollen. Ich hatte es schon ein- oder zweimal, dass diese wirklich ausfallend wurden und meinten, ich solle endlich mal die Wahrheit erzählen und nicht irgendetwas Geschöntes. 

\textbf{Wie reagieren Sie in so einer Situation?} 

\textbf{David Geballe:} Die Organisation muss angepasst werden. Man muss schauen, wie der Rest der Gruppe reagiert. Wenn der Rest der Gruppe nickend zustimmt, ist es etwas anderes als wenn die Gruppe diese Person komisch anschaut und fragt, was da jetzt los ist. Natürlich reagiere ich in beiden Fällen anders. 
In der beschriebenen Situation habe ich, wenn zeitlich die Möglichkeit bestand, quasi jede einzelne Aussage dieser Person auseinandergenommen und gezeigt, wie schwach das doch eigentlich ist. Zum Beispiel, dass Matzen eigentlich aus Blut gemacht und nur heutzutage, weil es verboten ist, aus Getreide hergestellt werden. Das war dann auch wirklich einfach: „Gibt es hier einen Chemiker?“. Zum Glück war jemand da, der sogar Lebensmittelchemiker war von Beruf. Den habe ich gefragt, ob er einen Prozess kennt, mit dem man Blut in eine weiße Masse transformieren kann, die auch noch essbar ist. „Nein, kenne ich nicht“ – und wenn selbst der Lebensmittelchemiker das nicht weiß… 

\textbf{Das was eine Anspielung auf die Ritualmordlegende?} 

\textbf{David Geballe:} Ja, das gibt es Verschiedenes. Zum Beispiel noch, dass Frauen im Judentum unterdrückt oder schwach seien, viele absurde Schwachsinnigkeiten. Falsches oder schlechtes Wissen gibt es leider auch sehr häufig. Hier in Fürth gibt es ein Weiterbildungszentrum des Deutschen Zolls, und die bieten in den verschiedenen Bereichen des Zolls, die mit anderen Leuten zu tun haben, Kurse an. Da kann man sich z. B. mal mit Amerikanern austauschen oder so etwas. Und jetzt hat es sich seit etwa vier Jahren eingebürgert, dass auch mal Gruppen des Deutschen Zolls in die Synagoge kommen und hier zollrelevante Dinge besprochen werden, zusammen mit einer Einführung in das Judentum. Das ist die dritte große Gruppe. Für eine Schulgruppe nehme ich ungefähr eine Stunde Zeit, für die Zollgruppen sind es eigentlich immer zwei oder zweieinhalb Stunden, weil meistens so viele Fragen kommen und so viel Praktisches für den Zoll-Alltag, was die Leute besprechen wollen. Sie erzählen mir z. B. von einer Situation, die aus dem Ruder gelaufen ist und fragen, was sie anders hätten machen sollen oder was besser gewesen wäre. Das sind auch meistens die interessantesten Führungen für mich persönlich. 

\textbf{Welche praktische Bedeutung hat das für den Zoll?}

\textbf{David Geballe:} Ein klassischer Fall wäre, dass diese Leute z. B. in München oder Frankfurt am Flughafen beschäftigt sind und eine Maschine mit israelischen Fluggästen reinkommt, die einer Routineinspektion vom Zoll unterzogen werden soll, jeder kennt das. Und es kann mal passieren, dass dann einer der Reisenden etwas forsch antwortet, zum Beispiel mit „du durchsuchst mich doch nur, weil ich Jude bin, bist du etwa ein Nazi?“. Solche Sprüche, die völlig aus dem Nichts hergeholt sind, aber leider von manchen benutzt werden, nach dem Motto „Angriff ist die beste Verteidigung, und wenn ich ihn jetzt als Nazi abstempele, wird er sich nicht mehr trauen, mich zu durchsuchen“. Natürlich ist das nicht der Fall. Man kann sich damit auch eine Anzeige wegen Beleidigung einheimsen, und es gibt Tipps, wie man als Zöllner mit so einer Situation umgeht und was man sagen oder tun kann, um sie ein wenig zu entschärfen. 

\textbf{Interessant. Noch einmal zu den ganzen antisemitischen Angriffen, die Sie bekommen: setzen Sie sich mit denen auseinander oder versuchen Sie lieber, sie zu ignorieren?} 

\textbf{David Geballe:} Je nachdem. Ich beschäftige mich mit allen Briefen, E-Mails, etc., die reingehen, und das nicht nur theoretisch besprechen, sondern z. B. sagen „Was macht ihr, ich werde euch dafür büßen lassen“. Solche Dinge werden natürlich an den Staatsschutz weitergeleitet, um akute Gefahren erkennen zu können und damit etwas dagegen getan werden kann. Gerade da muss man ganz klar unterscheiden zwischen denen, die vergleichsweise harmlos sind, mit theoretischen Drohungen, und denen, die wirklich Gewalttaten oder sonstiges androhen. 

\textbf{Archivieren Sie das selbst, oder leiten Sie nur bestimmte Sachen weiter?} 

\textbf{David Geballe:} Bestimmte Sachen werden weitergeleitet. Es gibt auch welche, die schon dermaßen außerhalb der Realität sind, dass sie eigentlich schon fast wieder komisch sind. Manche meiner Kollegen sammeln die einfach nur, damit sie nur ihre Mappe aufschlagen müssen, um wirklich gut darüber lachen zu können. Da denkt man sich manchmal, der Verfasser gehört eigentlich in die Klapse. Klar, zum Teil landet so etwas auch einfach gleich im Mülleimer. 

\textbf{Haben Sie eigentlich mitbekommen, dass es vor sechs, sieben Jahren wieder eine antijudaistische Prozession nach Heiligenblut am Brombachsee gab? \\
In Heiligenblut, so die Legende, hat ein böser Jude einen armen Bauern angestiftet, eine Hostie zu klauen. Der böse Jude hat dann hineingestochen und die Hostie fing natürlich wieder zu bluten an, weil die Hostie ja der Jesus ist. Der böse Jude hat den Jesus also ein zweites Mal erstochen. Dann wollte der Jude seiner Strafe entgehen und konvertieren, in der Kirche von Spalt hat ihn dann aber bei der Konvertierung der Blitz erschlagen. Daraufhin ist Heiligenblut erbaut worden, das war einmal ein Kloster und jetzt ist es noch eine Gedenkstätte. Die Kolping-Gemeinde von Spalt hat vor etwa sieben Jahren diese antijudaistische Prozession, die schon 50 Jahre lang aufgehört hatte, erneut ins Leben gerufen. Da sind dann ein paar hundert Leute mit dem Boot nach Heiligenblut gefahren. Ich bin mit der Prozession mitgelaufen, habe dann einen Artikel darüber auf \textit{HaGalil} geschrieben und da gab es eine große Diskussion. Seit der Artikel erschienen ist, ist auch die Prozession wieder gestorben. Da gab es ja noch eine ganz interessante Debatte, weil mir der Pfarrer von Spalt (ein Philosemit, der eine Mesusa an seiner Tür hat und diese zweimal täglich küsst als Christ) gesagt hat, ich sei ein Antisemit, weil ich diese Prozession veranstaltet habe. Der Pfarrer dachte, ich sei Jude und habe ihn kritisiert. Daran kann man gut nachvollziehen, dass Philosemiten, wenn sie von den Juden nicht geliebt werden, im Nu zu Antisemiten werden. Ist das auch Ihre Einschätzung?}  

\textbf{David Geballe:} Es gibt diesen Typus von Menschen durchaus. Es ist eigentlich so, dass diejenigen, die am lautesten schreien, dass sie keine Hilfe brauchen. diejenigen sind, die am meisten Hilfe benötigen. Genauso gibt es diejenigen, die sagen, sie lieben alle Juden, nur um das zu übertünchen, was sie im Inneren wirklich fühlen. Da kann man dann beobachten, wie das sehr schnell ins Gegenteil übergeht. Das beste Beispiel ist Martin Luther. 

\textbf{Wie finden Sie jetzt im Lutherjahr den Umgang mit Martin Luther?} 

\textbf{David Geballe:} Es gibt durchaus gute und ehrliche Ansätze, aber natürlich auch welche, die verschönern. Gute Ansätze gibt es z. B. beim evangelischen Bildungswerk hier in Fürth, das eine Veranstaltung mit dem Thema "Luther als Antisemit” gemacht hat. Die spricht das wirklich aus und erklärt dann im Vortrag, wie es ist und was man nicht schönreden kann, und dass Luther nun mal nicht unfehlbar war. Dann gibt es andere Artikel, die das relativieren und entweder sagen, dass es nicht so gemeint war oder dass man es im Zusammenhang sehen muss und Luther kein Antisemit war. 

\textbf{Wie gehen Sie innerhalb Ihrer, Ihrer Religion mit Reformation um, z. B. dass Frauen Rabbinerinnen sind, so wie Frau Antje Yael Deusel aus Bamberg?} 

\textbf{David Geballe:} Es ist ein kompliziertes Thema in einem Sinne, aber auf der anderen Seite auch ein sehr einfaches. Es gibt hunderte von Richtungen im Judentum, von Reform, liberal, orthodox, ultra-orthodox, ultra-ultra-orthodox, neo-orthodox, man kann damit nicht mehr wirklich mitkommen. Eigentlich aus der jüdischen Sicht völliger Schwachsinn. Es gibt eigentlich eine einzige Frage, die in dem Sinne wichtig ist, und zwar ob man sagt, dass das Religionsgesetz bindend ist. 
Wenn man sagt, das Religionsgesetz ist bindend, dann ist der Rest nur Kosmetik. Wenn man sagt, dass Religionsgesetz ist nicht bindend, dann ist es ganz egal, aus was für philosophischen, theologischen Gründen man das tut. Oder als Ableger davon, wenn man sagt, das Religionsgesetz ist bindend, aber man kann selbst bestimmen, was das Religionsgesetz sagt, dann ist das natürlich am Ende nur eine Farce und man bindet sich nicht daran. Es gibt bestimmte religionsgesetzliche Gründe, warum eine Frau keine Rabbinerin sein kann. Dementsprechend sehen das auch diejenigen, für die das Religionsgesetz bindend ist. 

\textbf{Wie gehen Sie dann mit der Bamberger Gemeinde um? Die hatte sich ja eine Zeitlang dafür entschieden, aber ich glaube, sie wurde jetzt abgesetzt.} 

David Geballe: Sie wurde vor ungefähr eineinhalb Jahren entlassen. Es gab, soweit ich weiß, einfach Unstimmigkeiten zwischen dem Vorstand der Gemeinde und Frau Deusel. Interna kenne ich nicht, aber sie wurde nicht gefeuert, weil sie eine Rabbinerin war oder ist. 

\textbf{Zur Struktur Ihrer Gemeinde: Es sind wahrscheinlich meistens Juden aus Osteuropa, schätze ich?}

\textbf{David Geballe:} Es sind zum großen Teil sogenannte Kontingentflüchtlinge aus der ehemaligen Sowjetunion gewesen, die sind aber meistens schon in den Neunziger Jahren gekommen. Also sind es auch schon über zwanzig Jahre, die die meisten hier sind. Klar, bei denen, die schon in einem gewissen Alter hergekommen sind, war die Integration nicht ganz so erfolgreich, aber die jüngeren Leute sind voll und ganz integriert. 

\textbf{Haben Sie eigentlich Kontakt zur Chabad-Gemeinde in Nürnberg?}

\textbf{David Geballe:} Die Gemeinde Fürth ist Körperschaft des öffentlichen Rechtes, das heißt völlig autark. Es gibt zwischen Nürnberg und Fürth eine U-Bahn-Station, die heißt Stadtgrenze, und die heißt nicht einfach so Stadtgrenze, das heißt genauso wie die Jüdische Gemeinde in Nürnberg autark ist, sind auch alle anderen Gruppen autark. Natürlich gibt es eine Zusammenarbeit zwischen einzelnen Gemeinden, aber technisch gesehen ist Chabad Nürnberg ein Verein, wenn man so möchte. Wir machen nichts zusammen, aber es ist nicht so, weil wir sagen „oh mein Gott, die mögen wir nicht“, sondern es ist einfach getrennt. 

\textbf{Arbeiten Sie mit der Israelitischen Kultusgemeinde in Nürnberg oder anderen Gemeinden in der Region zusammen?} 

\textbf{David Geballe:} Dort, wo es Sinn macht, z. B. in der Arbeit mit jungen Leuten. Viele Studenten ziehen von außerhalb her und werden in den paar Studienjahren nicht hier Gemeindemitglied, sondern nehmen das Beste aus allen Welten mit, indem sie ausnutzen, dass es hier im Umkreis von fünfzehn Kilometern drei Gemeinden gibt, in Fürth, Nürnberg und Erlangen. 

\textbf{Werden die antisemitischen Angriffe, die sie miterleben, gegen Gemeinde insgesamt gerichtet, oder gibt es auch persönliche Angriffe gegen ihre Gemeindemitglieder?} 

\textbf{David Geballe:} Briefe sind mir nicht bekannt. Es gibt sozusagen Spontan-Antisemitismus, das heißt man man sieht z. B. einen Juden auf der Straße und sagt ihm etwas oder spuckt ihm vor die Füße oder sowas in der Richtung. Das ist aber nicht organisiert. Vor zwei Jahren, als der Gaza-Krieg wieder sehr heiß war, war es zum Beispiel so, dass Leute mich gefragt haben, ob sie lieber vor und nach der Synagoge die Kippa abnehmen sollen. Die hatten wirklich Angst, dass etwas passiert – Gott sei Dank passierte nichts. 

\textbf{Wie sieht es mit den ultraorthodoxen Juden aus, also mit denen, die Schläfenlocken tragen?} 

\textbf{David Geballe:} Davon gibt es ja nicht viele in Deutschland. Es gibt durchaus ein paar, aber auch da kann man es so machen, dass nicht gleich offensichtlich ist, dass man jüdisch ist. Aus Sicherheitsgründen ist das vielleicht nicht die schlechteste aller Ideen. 

\textbf{Haben Sie das Ihren Mitgliedern geraten?} 

\textbf{David Geballe:} Ja. Wir haben anfangs darüber gesprochen, bei den hunderttausenden Muslimen, die nach Deutschland gekommen sind in den letzten zwei Jahren. Sicherlich würde keiner bei Verstand sagen, dass alle von denen gewalttätige Antisemiten sind oder sonstiges, um Gottes Willen. Selbst wenn man aber sagt, dass nur ein Prozent oder 0,1\% davon gewalttätige Antisemiten sein würden, sind das bei fast einer Million, die ins Land gekommen sind, immer noch sehr viele. Und es braucht nicht viele, um etwas zu tun. Es braucht nur ein oder zwei Personen, die etwas machen wollen, und dann ist es getan. 

\textbf{Würden Sie den spontanen Antisemitismus nur Leuten, die als Flüchtlinge gekommen sind, zuordnen?} 

\textbf{David Geballe:} Nein. Man muss nur auf den Fußballplatz gehen, wo „du Jude“ eine typische Beleidigung ist. Das ist heute leider in der Jugendkultur und teilweise auch im Sport sehr tief verankert. Die Schule ist nun mal ein Schmelztiegel, in dem viele Dinge zusammenkommen. 

\textbf{Wissen Sie etwas von Initiativen oder Zusammenarbeit speziell mit Sportvereinen, um den Antisemitismus der Fans zu bekämpfen?} 

\textbf{David Geballe:} Berühmt ist in größeren jüdischen Gemeinden der sogenannte Sportverein Maccabi, der vor allem im Fußball immer große Probleme hat. Egal, in welcher Liga sie spielen, es kommt oft vor, dass im gegnerischen Team z. B. sehr viele Muslime sind und die Fans, also deren Familie, Freunde, etc. teilweise negativ auffallen. Das hat auch schon zu Sperren und Strafen geführt. Es gibt in dieser Hinsicht auch Versuche, das ein bisschen einzuschränken und dagegen zu wirken, aber Fußball ist nun mal der Breitensport schlechthin und nicht in jedem kleinen Dorf gibt es eine jüdische Gemeinde, die das Know-How, die Zeit und auch die Manpower dazu hätte, da wirklich groß einzugreifen. 

\textbf{Machen die Spielvereinigung in Fürth oder der Fußballclub Nürnberg da etwas?} 

\textbf{David Geballe:} Fürth war schon immer als judenfreundliche Stadt bekannt, und das gilt auch heute noch. Vor dem Krieg waren viele Spieler von Greuther Fürth jüdisch. Zur Veranstaltung am 9. November sind bei uns auch immer ein paar Spieler von Greuther Fürth dabei. Selbst wenn sie keine offizielle Ansprache halten, sind sie trotzdem da, um ein Zeichen zu setzen. Also da gibt es durchaus Vereine, die auch dagegen aktiv sind. 
\end{otherlanguage}
\section{Dr Svetlana Bogojavlenska}

\textit{Dr. Svetlana Bogojavlenska studied history from 1995 to 2002 at the University of Latvia in Riga. Between 1997 and 2002 she worked in the Museum ``Jews in Latvia''. From 2003 to 2008 she worked as a doctoral student and lecturer at Johannes Gutenberg University Mainz. Her doctoral thesis is entitled ``The Formation and Position of Jewish Society in Riga and in the Government Courland 1795-1915''. From 2011 to 2016, she worked in the DFG project ``Russian in the Latvian Context: Russian Identity Formation in Latvia 1914-1940'' at the Department of History of the University of Mainz. Since 2018 she is Research Associate in the project ``Before the Cultural History: Functions and Dynamics Russian Historiography in the European Context (1750-1830)''. Her research interests include Jewish culture and history, the history and theology of the Orthodox Church, sacred music and art, comparative religion, minorities in Russia and the Baltic States, and cultural transfer.
For the interview she visited us in Nuremberg, the interview took place on September 5th, 2017 and was conducted in German.}\par
\vspace*{2em}
\textbf{Vielen Dank, dass Sie sich Zeit nehmen, uns ein Interview zu geben. Ich würde jetzt einmal mit der Frage beginnen, wann Sie, wenn Sie sich noch erinnern können, zum ersten Mal von dem Begriff "`Jude"' oder "`jüdisch"' gehört haben.}

\textbf{Dr. Svetlana Bogojavlenska:} In der Kindheit. Ich glaube, ich habe es zum ersten Mal von meiner Großmutter gehört, und der Begriff war schon mit Vorurteilen beladen. Es war so, dass ich das Gefühl hatte, dass das etwas Fremdartiges ist. Was vielleicht ein bisschen gefährlich ist, aber noch gefährlicher waren "`Zigeuner"'. In der Gegend, in der ich gewohnt habe, war das die Wahrnehmung der Bevölkerung. Oder die Wahrnehmung meiner Großmutter. Seitdem war der Begriff eigentlich immer wieder irgendwie da, aber ich habe mich dann nicht so viel damit beschäftigt. Bis ich dann selbst auf die Idee kam, dass ich eigentlich viele Klassenkameraden hatte, die plötzlich nach Israel ausgewandert waren.
Vorher wusste man gar nicht, dass sie etwas mit dem Judentum zu tun hatten. Man hätte es auch gar nicht vermuten können. Sie haben sich nicht speziell als Juden ausgegeben. Das waren ganz normale Klassenkameraden wie jede und das wurde gar nicht thematisiert. Dann waren viele plötzlich in Israel. Und dann habe ich angefangen, mich dafür zu interessieren, welche Geschichte dieses Volk hat - was haben sie in Lettland gemacht und wie kam es dazu, dass wir eigentlich so wenig über sie wissen? Über ihre Geschichte und warum sie so unscheinbar da sind. Und plötzlich wandern sie nach Israel aus, weil sie sagen: "`Ich möchte Jude sein"'. Warum kann man das nicht in Lettland sein, zum Beispiel? Auch das war meine Frage.

\textbf{Also das haben sie gesagt? Sie könnten in Lettland keine Juden sein?}

\textbf{Dr. Svetlana Bogojavlenska:} Zumindest ein Klassenkamerad von mir, mit dem ich Schriftverkehr hatte über Jahre. Seine Argumentation war: "`In Israel habe ich zu mir zurückgefunden"'. Weil es keine Hindernisse mehr gibt, weil man sich frei zur eigenen Identität äußern kann. Und das war schon eine Überraschung für mich, weil ich ihn vorher nicht als Juden wahrgenommen habe und er es auch nie in der Schule thematisiert hat.

\textbf{Sind ihre Klassenkameraden alleine ausgewandert oder mit ihren Familien?}

\textbf{Dr. Svetlana Bogojavlenska:} Mit den Familien zusammen. Es gab eine Welle an Auswanderung in Lettland, das hat wahrscheinlich '91/'92 angefangen. Und in meiner Erinnerung ist es so geblieben, dass Berge guter Bücher irgendwo in den Höfen lagen, zum Beispiel, oder einfach in die Schule gebracht worden sind von den Familien, die auswandern.  Das war dann um '94/'95.

\textbf{Abgesehen davon, wurde das Thema Judentum, jüdische Religion und jüdische Geschichte in der Schule behandelt?}

\textbf{Svetlana Bogojavlenska:} Überhaupt nicht, nein. Darüber haben wir nie geredet. Ich glaube mich erinnern zu können, dass im Geschichtsunterricht, als es um den Zweiten Weltkrieg ging, die Judenvernichtung erwähnt wurde, aber nur erwähnt. Und sonst nicht. Der Begriff Holocaust ist gar nicht erwähnt worden, er war völlig unbekannt. Dass Juden als ein Volk tatsächlich als Zielscheibe des Nationalsozialismus ausgewählt wurden und dann ausgelöscht wurden, das wurde schon thematisiert, aber wie gesagt sehr kurz. Mit ein paar Sätzen wurde das gesagt: Die Juden waren ein besonderes Ziel der Vernichtung, aber Hitler hatte vor ganz Osteuropa auszulöschen. - ungefähr in diesem Sinne.

\textbf{Vor den Holocaust zurück bzw. vor die Okkupation zurück. Wie war die Stellung der Juden in der Zwischenkriegszeit in Lettland?}

\textbf{Svetlana Bogojavlenska:} Juden waren gleichberechtigte Bürger, zum ersten Mal in ihrer Geschichte auf diesem Territorium. Sie wurden 1918 mit der Gründung Lettlands zu gleichberechtigten Bürgern, die sich auch vollständig in der Politik frei entfalten konnten, die sich auch tatsächlich in der Verwaltung betätigt haben, auch bei der Regierungsbildung. Bis 1934, bis zum Regierungsumsturz und bis zur Etablierung des Autoritarismus von Ulmanis in Lettland kann man tatsächlich von der vollständigen politischen Gleichberechtigung sprechen.
1934 hat sich die allgemeine nationale Politik des Staates geändert. Wenn bei der Regierungsdeklaration 1918 verkündet wurde, dass Lettland ein Staat für alle Ethnien ist, die auf diesem Territorium leben, dann wurde nach dem Regierungsumsturz 1934 propagiert, dass zwar alle bleiben und ihren Teil daran haben dürfen, aber Lettland doch für die Letten da ist. Das war sehr nationalistisch, aber das war nicht explizit gegen Juden gerichtet, sondern gegen alle ethnischen Minderheiten, die in Lettland gewohnt haben. Und da musste man schon zusehen, wie man sich mit dem Regime arrangiert.
Mit dem Bildungsgesetz von 1919 war es allen Minderheiten in Lettland erlaubt, eigene Schulen zu gründen und die Eltern konnten frei entscheiden, in welche Schule ihr Kind geschickt wird. Es kam deswegen sehr häufig vor, dass z.B. Kinder aus russischsprachigen jüdischen Familien in die russischen Schulen geschickt worden sind. Jüdisches Schulwesen existierte in 4 Sprachen: Es gab russische jüdische Schulen, deutsche jüdische Schulen, jiddische jüdische Schulen und einige hebräische jüdische Schulen. Und jede Familie durfte selbst entscheiden wohin das Kind geht. Nach '34 hat sich die Situation entschieden gewandelt, indem es dann vom Staat vorgeschrieben war, dass jede Ethnie ihre Kinder in die entsprechende Schule schicken muss. Wenn sie sich gegen diese Schule entscheiden, dann müssen sie ihre Kinder in die lettische Schule schicken, und die Sprache der Familie spielte dann keine Rolle mehr. Man konnte dann einfach nur noch sagen: "`Ich bin Jude."' - "`Wenn du Jude bist, dann gehst du halt in die jiddische oder hebräische Schule"'. Es gab keine russisch- oder deutschsprachigen jüdischen Schulen mehr. Und dann wurde es komplizierter sich frei zu entscheiden und sich frei einer Kultur anzuschließen. Vorher war das viel einfacher und auch z.B. von den Russen war das akzeptiert.

\textbf{Wie ungefähr sah die Zusammensetzung der jüdischen Bevölkerung aus? Sie haben ja schon erwähnt, dass es diese Schulen gab und dann dementsprechend auch die unterschiedlichen Sprachgruppen.}

\textbf{Svetlana Bogojavlenska:} Die überwiegende Mehrheit hat Jiddisch gesprochen. Bei den Volkszählungen, die damals durchgeführt worden sind, davon gab es drei, wurde Hebräisch fast gar nicht erwähnt. Vielleicht nur bei ein paar Leuten, weil nach der Muttersprache oder nach der Familiensprache gefragt wurde und Hebräisch damals noch keine Muttersprache der Juden in Lettland war.
Von 93 000 Juden, die in Lettland 1930 erfasst wurden, haben ca. 10\% Deutsch als Familiensprache angegeben. In Riga haben 15\% der Juden im Alltag Deutsch gesprochen, in den Städten Kurlands bis zu 21\%. Russisch haben 1930 etwa 6\% der Juden in ganz Lettland als Alltagssprache angegeben.\footnote{Skujenieks 1930: 456, 466}  Es kann davon ausgegangen werden, dass mehrere Sprachen in der Familie gesprochen wurden. Jiddisch zum Beispiel parallel zum Deutschen oder zum Russischen. 
Lettisch als Familiensprache haben nur wenige angegeben, 652 Personen.\footnote{Ibid: 456}  Lettisch als Muttersprache war unter den Juden nicht so verbreitet. Aber wenn man sie dann nach der Sprachenkenntnis gefragt hat, war unter den Juden Prozent derer, die Lettisch beherrschten ziemlich hoch: 62\% im Jahr 1930.\footnote{Salnītis 1938: 103}

\textbf{In der Zwischenkriegszeit erstarkte die antisemitische Bewegung in vielen europäischen Ländern, in Polen oder in Deutschland. Gab es solche Tendenzen auch in Lettland?}

\textbf{Svetlana Bogojavlenska:} Tendenzen gab es, es gab sogar eine offen antisemitische Organisation, die hieß Pērkonkrusts. Es gab auch antijüdische Übergriffe, aber das waren tatsächlich nur sehr vereinzelte Übergriffe. Das war keine Massenbewegung, und 1934 hat Ulmanis diese Organisation sogar verboten, als staatsfeindlich und als eine, die dem lettischen Geist gar nicht entspreche, weil sie fremdenfeindlich ist. Einige Mitglieder dieser Organisation wurden später, während des Krieges, zu Tätern. Sie haben sich an der Ermordung der Juden beteiligt.

\textbf{Wie stark war der Zuspruch in der lettischen Bevölkerung für diese Organisation?}

\textbf{Svetlana Bogojavlenska:} Nicht so stark. Man kann nicht sagen, dass die Bevölkerung das getragen oder unterstützt hätte. Das war schon eine sehr begrenzte Gruppe.
 
\textbf{Also nicht zu vergleichen mit Polen oder Deutschland?}

\textbf{Svetlana Bogojavlenska:} Nein, überhaupt nicht. Antisemitismus war in Lettland, das weiß ich aus meiner Forschung über das 19. Jahrhundert, keine alltägliche Sache. Ich habe beinahe alle Archivbestände durchgeforstet und die ganze lettische Presse durchgelesen. Ab den 1880er Jahren mit der Pogromwelle in Russland kommt es zu zwei oder drei Zwischenfällen in ganz Lettland, wo vielleicht Stände, Obststände der jüdischen Marktkaufleute entweder zerstört wurden oder einige Juden verprügelt wurden, aber es kam zu keinem Totschlag. Es kam zu keiner Massenbewegung gegen die Juden, und man kann nicht sagen, dass Letten besonders antisemitisch eingestellt wären. Ich habe auch die lettische Literatur, die als Quelle dienen kann, analysiert und da gab es einige Erzählungen, in denen Juden sogar positiv dargestellt werden. Und es gibt ein Theaterstück, „Schneidertage in Silmači“, das immer noch jedes Jahr vor dem lettischen Johannisfest aufgeführt wird,\footnote{\textit{Skroderdienas Silmačos} – Schneidertage in Silmači von Rudolfs Blaumanis, 1902 in Riga uraufgeführt} was sicherlich in Deutschland nicht durchgehen würde. In Lettland ist es aber kein Problem, es gehört zum lettischen Kulturerbe. Da ist eine jüdische Gestalt zu sehen, ein jüdischer Hausierer. Das ist eine ironische Gestalt, könnte man sagen, sein Lettisch ist gebrochen und er sorgt dann immer für einen Lacher im Saal, weil er alles andere als normal ist. Aber auch dieses Stück zeigt, denke ich, dass die Juden einfach dazugehörten, sie waren im Alltag der Letten immer da. Der jüdische Hausierer war nicht aus Lettland wegzudenken am Ende des 19., Anfang des 20. Jahrhunderts, und die Kontakte bestanden schon. Man beneidete die Juden für ihre wirtschaftlichen Erfolge, und es lässt sich bestätigen, dass die Letten die in Folge der lettischen nationalen Bewegung versuchten, sich in der Wirtschaft zu etablieren, auf die Juden trafen, die da schon ihre Nischen besetzt hatten. Dann kam es tatsächlich zu ökonomisch motiviertem Antisemitismus, der in den Zeitungen zu sehen ist. Aber man kann z.B. von keinem Antijudaismus sprechen, von keinen Ritualmordvorwürfen usw. Man hat es dann eben im Alltag gesehen - "`Das sind unsere Konkurrenten, und die gilt es zu vertreiben"', aber nicht, weil sie Juden sind, sondern weil sie Konkurrenten sind. Wäre da ein anderes Volk gewesen, gehe ich davon aus, dass die Reaktion ebenso ausfallen würde.

\textbf{Der Antisemitismus und Antijudaismus Luthers hat sich also nicht niedergeschlagen?}

\textbf{Svetlana Bogojavlenska:} Ich habe keine Beweise dafür gesehen. Eine der ersten Zeitungen in lettischer Sprache war Mājas Viesis, diese wurde von den lutherischen Pastoren für die Letten herausgegeben. Und erstaunlicherweise wurden darin ab den 1860er Jahren viele Geschichten aus dem Alten Testament wiedergegeben. Aber nicht nur das. Jüdische Feste wurden beschrieben, zum Beispiel Sukkot. Das hätte ich von einer lettischen Zeitung gar nicht erwartet. Da wurden Hintergründe beschrieben, "`Warum feiern die Juden es?"' usw. Es war sehr interessant zu lesen, dass die Letten darüber unterrichtet wurden, wie die Juden ihre Feste feiern ohne da irgendwie etwas Negatives draus zu ziehen. Sie waren eben Nachbarn, sie haben zusammen gelebt, vor allem in Ostlettland, denn östlicher Teil Lettlands, Lettgallen, gehörte zum Ansiedlungsrayon.\footnote{Als Ansiedlungsrayon der Juden (Russisch: \textit{čerta osedlosti}) bezeichnet man die Teile des russischen Reiches, überwiegend in den Gebieten der heutigen Ukraine, Weißrussland, Litauen und Lettland, in denen Juden ohne eine Sondergenehmigung leben und arbeiten durften. Der östliche Teil Lettlands, ethnographisches Gebiet Lettgallen, gehörte zum Gouvernement Vitebsk, das zum Ansiedlungsrayon gehörte. Der Ansiedlungsrayon wurde erst 1915 offiziell abgeschafft. Seit den Liberalisierungsreformen des Zaren Alexander II. durften verschiedene Gruppen von Juden, z.B. Handwerker mit ihren Familien auch außerhalb des Ansiedlungsrayons wohnen. Auch davor gab es schon jüdische Berufsgruppen, denen es erlaubt war mit Sondergenehmigung außerhalb des Rayons zu wohnen: so z.B. in Kurland, Riga, St. Petersburg usw. Vgl. z.B Kotowski et al. (2001): 180-195.} Da sich die Juden laut Gesetz in den Städten registrieren mussten, kam es tatsächlich dazu, dass bis zu 90\% der Stadtbevölkerung Juden waren, die mit den Letten immer in Kontakt waren, die in der ländlichen Umgebung gewohnt haben. Die Synagogen wurden errichtet, es ist kein einziger Fall bekannt, wo irgendein Anschlag gegen eine Synagoge verübt worden wäre.

\textbf{Sie sagen "`miteinander in Kontakt"'? Gab es auch familiäre Verbindungen, zum Beispiel? Oder nur Freundschaften und so etwas?}

\textbf{Svetlana Bogojavlenska:} Familiäre Verbindungen gab es eher nicht. Es gibt statistische Erhebungen, auch aus der Periode in der das Territorium von Lettland ans Russische Reich angegliedert war. Und da hat man auch kontrolliert, wer mit wem geheiratet hat. Im Allgemeinen gab es sehr wenige religiös gemischte Ehen. Es gab unter den Juden auch einige Übertreter zum Christentum, nicht aber umgekehrt, denn das war auch verboten. Es gab einen Versuch der deutschen, evangelischen Mission, beispielsweise Juden in Kurland zu christianisieren, sie evangelisch-lutherisch zu machen. Die Mission konnten allerdings keine großen Erfolge feiern. Die Übertritte beliefen sich auf 10 Personen pro Jahr. Insgesamt waren die konfessionellen Gruppen stark voneinander abgegrenzt.

\textbf{Wie ist heute in der lettischen Bevölkerung die mehrheitliche Einschätzung von Karlis Ulmanis? Wie bewertet man ihn heute, welchen Ruf hat er?}

\textbf{Svetlana Bogojavlenska:} Es gibt ein Denkmal für Ulmanis in Riga. Dieses Denkmal ist sehr umstritten. Es gibt einen Teil der Bevölkerung der glaubt, dass diese Zeit die Beste war. Man glaubt ja auch immer, dass alles besser war, als man jung war. Die Zeit des Autoritarismus haben sehr viele Letten, die heute noch leben, miterlebt, die Zeit des demokratischen Lettlands jedoch kaum. Diejenigen, die diese Unabhängigkeitszeit erlebt haben, die haben schon in dieser autoritären Zeit ihr Leben verbracht. Und was kam danach? Danach kam das sogenannte "`schreckliche Jahr"', das Jahr der sowjetischen Besatzung, das so schlimm war, dass in der Erinnerung der Letten die Herrschaftszeit von Ulmanis, die damals auch nicht unumstritten war, tatsächlich als eine paradiesische Periode die Zeit überdauert hat. Sie meinen, die Zeit von Ulmanis war die Blütezeit. Wobei das nicht so stimmt, da war ja auch die ökonomische Krise, man blendet das Ganze also aus. Denn im Vergleich zu dem, was danach kam, war das natürlich ein Paradies. Das würde ich auch vielleicht sagen, aber eine andere Sache ist, wie geht man damit um, dass man so einer Persönlichkeit, die die Demokratie abgeschafft hat, ein Denkmal im Zentrum von Riga aufstellt? Dieses Denkmal ist in seiner Darstellung kein heroisches Denkmal, deswegen gab es kaum Widerstand dagegen. Es steht da, in einem Park, da kommen ein paar alte Omas vielleicht ab und zu hin, stellen Blümchen hin. Sein Todestag, der noch nicht ganz geklärt ist, wird erinnert. Er wurde von den Sowjets verhaftet und ist in einem sowjetischen Gefängnis 1942 gestorben. Obwohl ihm versprochen wurde, dass er auf freien Fuß gesetzt wird, wenn er alles unterschreibt und nicht gegen die Sowjets kämpft, und die Erlaubnis erhält in die Schweiz auszureisen. Im Endeffekt endete er in einem Gefängnis in Turkmenistan. Er war einer der Gründer Lettlands und einer der aktivsten Politiker, auch in der demokratischen Periode. Deshalb kann man seine Persönlichkeit so oder so sehen. Natürlich ist es ein Verbrechen seinerseits, die Demokratie abzuschaffen. Unter Historikern herrscht die einhellige Meinung, dass er das nur gemacht hat, weil er um seine politische Karriere bemüht war und Angst hatte, dass bei der nächsten Wahl des Parlaments seine Partei nicht mehr ins Parlament kann und er dann politisch ausgeschaltet wird. Politik war sein ganzes Leben. Und deshalb ist er wirklich eine sehr widersprüchliche Persönlichkeit. Aber von der Mehrheit der Bevölkerung und ihrer Einstellung kann man gar nicht sprechen. Die Meinungen in der Bevölkerung sind auch sehr widersprüchlich.

\textbf{Wie hat sich dann die Situation der Juden mit der sowjetischen Besatzung verändert?}

\textbf{Svetlana Bogojavlenska:} Es hat sich alles geändert mit der sowjetischen Besatzung. Erstens gab es den lettischen Staat nicht mehr, zweitens wurde eine Deportation vorbereitet und durchgeführt. Bei dieser Deportation wurden auch ca. 5000 Juden deportiert. Prozentuell gesehen war die jüdische Bevölkerung die, die am meisten unter dieser Deportation gelitten hat. Bei der ersten Deportationswelle wurden Vertreter der Intelligenz, nicht nur der lettischen, sondern auch der jüdischen und der russischen aus Lettland nach Sibirien gebracht, und dieses eine Jahr der sowjetischen Besatzung bedeutete für die Juden, sowie für alle anderen, Enteignung. Und eine Verschlechterung ihrer Lage und ein Verschwinden der politisch Rechte. 
Aber es gibt auch die Legende, dass der Rabbiner Dubin\footnote{Mordehai Dubin, 1889-1956, Politiker und einflussreicher spiritueller Führer} vor der sowjetischen Besatzung gefragt wurde, "`Was sollen wir jetzt machen?"', und dass er zu der jüdischen Gemeinde gesagt habe- er war auch politischer Vertreter der Gemeinde: "`Lieber gebt die Schlüssel von euren Geschäften an Stalin, als eure Leben an Hitler."' Dadurch lässt sich auch erklären, dass viele Juden sich dazu fast gezwungen sahen, die sowjetische Macht zu begrüßen. Es gibt sehr viele Fotos von den Demonstrationen, die die Sowjetmacht in Lettland begrüßten. Die organisierten sich nicht spontan, sondern wurden von den Sowjets organisiert. Die Beteiligung an diesen Demonstrationen ist auch das, was von der lettischen Seite den Juden gegenüber vorgehalten wird. "`Ihr wart diejenigen, die die Sowjets damals mit Blumen empfangen haben!"' Es gibt tatsächlich auch Fotos auf denen man sieht, dass da überwiegend, oder sehr viele Juden sind in der Menge. Und das ist unter anderem damit zu erklären, dass sie zwischen Pest und Cholera auswählen mussten. 

\textbf{Waren die Juden die einzigen Demonstrationsteilnehmer?}

\textbf{Svetlana Bogojavlenska:} Nein, natürlich nicht. Nein, es waren von der sowjetischen Besatzungsmacht organisierte Demonstrationen, aber wenn man eine lettgallische Kleinstadt nimmt in der 90\% der Bevölkerung Juden sind, dann sind auf dem Bild natürlich nur Juden. Dann sieht man da kaum Letten oder andere. Andere sind in den Dörfern, lettische Bauern, russische Bauern, die werden dann auch natürlich zum Demonstrieren angehalten, fotografiert wird aber in der Stadt.

\textbf{Kann man sagen, warum es prozentual vor allem Juden waren, die von diesen Deportationen betroffen waren?}

\textbf{Svetlana Bogojavlenska:} Weil sie viel Eigentum hatten, wirtschaftlich erfolgreich waren und weil sie gefürchtet wurden. Man fürchtete, wenn man die reichen und gebildeten Juden dalässt, organisieren sie einen Widerstand. Es wurde von der jüdischen Gesellschaft kurz darauf befürchtet, dass auch eine zweite Deportationswelle bevorstand, in der dann noch mehr Juden deportiert worden wären, aber das ist dann nicht geschehen, weil Lettland von Deutschland überfallen wurde.

\textbf{Wie wurden diese Listen zusammengestellt? Woher wusste man, wer gefährlich sein konnte?} 

\textbf{Svetlana Bogojavlenska:} Ich kann das nicht ganz genau sagen, ich weiß nur, dass es schon vorher vorbereitet wurde. Die Besetzung Lettlands passierte ja nicht einfach so von einem Tag auf den anderen, sondern wurde von der sowjetischen Seite vorbereitet, es wurden Agenten nach Lettland geschickt. Die lettische Polizei verzeichnete auch den plötzlichen Anstieg der Exilletten, die im letzten Jahr der Unabhängigkeit aus der Sowjetunion nach Lettland gekommen waren. Die Listen der verdächtigen Personen wurden seit Herbst 1940 zusammengestellt.\\
An die Informationen zu kommen war nicht so kompliziert. Es waren auch wirklich bekannte Familien.

\textbf{Gab es unter den Sowjets und vorher Antisemitismus in Russland? Welche Art von Antisemitismus gab es?}

\textbf{Svetlana Bogojavlenska:} Ja, eindeutig. Es gab stark ausgeprägten Antisemitismus, der auch nach dem Krieg Durchschlag hatte.
Z.B. ging die Pogromwelle 1881 von Russland aus. 1880 hat es schon angefangen, '81 mit der Ermordung des Zaren wurden die Juden der Ermordung beschuldigt, obwohl es Anarchisten waren, die sich in der Wohnung eines Juden versammelt hatten, als sie diesen Anschlag auf den Zaren planten. Und der jüdische Historiker Simon Dubnow, der 1941 in Riga bei einer der Massenerschießungen ums Leben kam, ging davon aus, dass diese Pogromwelle sogar vom Staat organisiert wurde. Das wurde durch spätere Forschungen widerlegt, was aber nur das Zeugnis dafür abgibt, das die Bevölkerung eher antisemitisch eingestellt war und aus verschiedenen Gründen auch selbst Pogrome verübt hat. Was die Regierung sicher nicht gemacht hat, war die Pogrome rechtzeitig zu stoppen. Die Gendarmerie hat immer ein paar Tage gewartet, bis da nichts mehr zu retten war.
1905, während der Revolution, gab es auch Pogrome, und die gingen auch nicht von staatlicher Seite aus, aber sie beschränkten sich auf den Ansiedlungsrayon der Juden. Dort, wo es eine sehr hohe Konzentration jüdischer Bevölkerung gab, da kam es auch zu den Pogromen. 1905 gab es nur ein Pogrom in Lettgallen, der dokumentiert ist, in Ludsen, heute Ludza. Ansonsten nicht. 

\textbf{Kann man dann sagen, dass der Antisemitismus in der Zwischenkriegszeit in allen ethnischen Gruppen in Lettland gleich stark, bzw. gleich schwach vertreten war?}

\textbf{Svetlana Bogojavlenska:} Vom Antisemitismus in der Zwischenkriegszeit halte ich nicht viel. Ich habe mich mit der Zwischenkriegszeit und der russischen Bevölkerung beschäftigt. Die größte russische Zeitung wurde von einem Juden herausgegeben und hat eigentlich für beide Bevölkerungsgruppen gearbeitet und für verschiedene Parteien. Die Zeitung hat sowohl für die russischen als auch die jüdischen Parteien Wahlkampf gemacht und Probleme beider Minderheiten gemeinsam besprochen. In der russischen Bevölkerung war jeder Jude, der sich zur russischen Kultur bekannte, als Russe angenommen. Die haben verstanden, "`Wir sind hier alle fremd, aber wir haben alle eine gemeinsame Kultur."' Juden, die im russischen Imperium von Russen nicht als Russen akzeptiert wurden, wie sie sich auch bemühten, wurden in Lettland plötzlich anerkannt als russische Angehörige. Ohne dass sie sich von ihrer Religion oder ihrer ethnischen Abstammung lossagen mussten. Jeder Jude konnte sagen: "`Ich fühle mich der russischen Kultur angehörig."' Was die lettische Bevölkerung betrifft, ich habe das ja auch schon angesprochen, war Pērkonkrusts die wichtigste antisemitische Organisation. Man kann vielleicht noch die Universität Lettlands erwähnen, wo die studentischen Korporationen gegründet worden sind. Da gab es lettische Korporationen, in denen kein einziger Jude vertreten war. Aber die waren auch anderen gegenüber nicht aufgeschlossen, auch Russen wurden da nicht aufgenommen. In den russischen studentischen Organisationen, oder übrigens auch in den polnischen, waren Juden drin. Wenn sie gesagt haben, ich fühle mich hier angehörig, wurden sie auch aufgenommen. 

\textbf{Wie war das Verhältnis der Letten zu den Deutschen 1941?}

\textbf{Svetlana Bogojavlenska:} Ich würde nicht pauschalisieren und sagen, alle Letten hätten Deutsche mit Begeisterung empfangen. Aber ich stimme dem Argument zu, dass das eine sowjetische Jahr so ein tiefer Einschnitt in die Geschichte Lettlands war, dass die Feindschaft gegenüber den Deutschen durch die Erfahrung 1940/41 mit der Sowjetmacht weg war. Dass die Deutschen tatsächlich als Befreier von der sowjetischen Besatzung angesehen worden sind. Als eine Chance, Lettland vielleicht wieder unabhängig zu machen. Es gab vom lettischen Zentralrat (\textit{Latvijas centrālā padome}), das ist eine Widerstandsorganisation, die gegen beide, d.h. gegen die deutsche und gegen die sowjetische Seiten gekämpft hat, Versuche, Lettland wieder als unabhängigen Staat herzustellen. Die lettische Selbstverwaltung dagegen, die von den deutschen gebilligt wurde, wurde am Ende zur rechten Hand der deutschen Besatzungsmacht. Die nationalsozialistische Besatzung hat eine kluge Politik in diesem Sinne durchgeführt. So hat sie zum Beispiel die Gottesdienste in den Kirchen wieder erlaubt, die in dem Jahr der sowjetischen Besatzung verboten waren. Die haben die sog. Pskower Orthodoxe Mission gegründet und die orthodoxen Kirchen in den besetzten Territorien wieder aufgemacht, auch in Lettland. Das hat natürlich die Sympathien derjenigen gebracht, die nicht direkt vom nationalsozialistischem Terror und Mord betroffen waren.\\
Es gab auch eine Zeitung in russischer Sprache, die hieß \textit{Za Rodinu}, für das Vaterland, und man hat so versucht, die russische Bevölkerung zu instrumentalisieren. Es war sehr interessant, dass die russische Intelligenz und die russischen Bauern darauf angesprochen wurden, wie schlimm es der russischen Bevölkerung während der Zeit der sowjetischen Besatzung ging und dass der große Hitler natürlich alles besser machen würde. Es wurden sogar verschiedene Beiträge in dieser Zeitung veröffentlicht, die in der Zwischenkriegszeit in der russischen Zeitung \textit{Segodnya} publiziert wurden. Dabei wurde natürlich verschwiegen, dass diese Zeitung von einem jüdischen Herausgeber stammte. 
Die antisemitische Propaganda der Nazis wurde sehr sorgfältig vorbereitet und der wichtigste Punkt war, dass die Repressionen, die die lettische Bevölkerungen 1940/41 durch die Sowjets erlitten hatte, von Juden organisiert und durchgeführt worden seien. Als im Rigaer Zentralgefängnis die Leichen der dort getöteten politischen Häftlinge gefunden worden sind, hat man sie alle in einen Hof gelegt und als Opfer der jüdischen Kommunisten dargestellt. Und fast sofort danach kam der Aufruf in der propagandistischen deutschen Zeitung in lettischer Sprache Tēvija, “das Vaterland”, sich zu formieren, um das Vaterland zu beschützen. Selbstschutzeinheiten, Polizeieinheiten zu bilden. Es wurden zuerst tatsächlich nur Freiwillige aufgenommen. Auf diesen Aufruf hin hat sich auch das Arājs-Kommando gebildet.

\textbf{Kann man abschätzen, wie viele Letten kooperiert haben mit den Nazis, was die Judenvernichtung betrifft?}

\textbf{Svetlana Bogojavlenska:} Man kann schauen, wie viele danach verurteilt worden sind durch die Sowjets. Das waren einige Tausende. Aivars Stranga hat die Zahl geschätzt in seinem Beitrag\footnote{``The accurate number of Latvians who took direct or indirect part in the annihilation of the Jews is not easy to establish; an approximate figure could constitute a few thousands.'' (Stranga (2005) 167) } in dem Sammelband "`The Hidden and Forbidden History of Latvia"', das von der Historikerkommission Lettlands herausgegeben wurde. Man hat in der sowjetischen Zeit allerdings auch vermieden zu sagen, dass die Opfer Juden waren. Die Kollaborateure wurden wegen Verbrechen gegen sowjetische Bürger verurteilt. Wenn die Überlebenden des Holocausts nach dem Krieg in verschiedenen Städten, wie z.B. in Riga, einen Gedenkstein aufstellen wollten, durfte man nicht darauf hinweisen, dass das jüdische Opfer waren. Es stand dann "`Opfern des Krieges"', "`Opfern des Faschismus"' oder "`den sowjetischen Opfern"' auf den Gedenktafeln. Das einzige, was in Riga gelungen war, war der Bau einer Gedenkstätte im Wald von Rumbula in den 1960er Jahren. Die Inschrift, dass dort sowjetische Bürger ermordet worden sind, wurde in drei Sprachen geschrieben: Auf Russisch, Lettisch und Jiddisch. Allerdings, wenn man an den Gedenktagen der Rumbula-Aktionen im November und im Dezember dorthin fuhr, da wurde man vom KGB oder vom Sicherheitsdienst überwacht und einige wurden dann sogar verhaftet. In das sowjetische Bild passte es nicht, dass ein Volk mehr als die anderen Völker der Sowjetunion während des Krieges gelitten hat. Die Akten der Verurteilten wurden geheim gehalten, bis 1991. Erst dann konnte man sie einsehen. Wenn man die Verhörprotokolle liest, dann sieht man ganz deutlich, dass es da um die Juden ging. Dass das die "`Selbstschutzbataillone"' waren, die eigentlich in dieser Zeit nur dazu dienten, um die Juden in den Kleinstädten Lettlands zu ermorden. Bis Ende August 1941 waren alle Juden in Kleinstädten Lettlands tot, sie wurden alle ermordet.
 
\textbf{Das waren also kleine Einsatzgruppen?}

\textbf{Svetlana Bogojavlenska:} Genau, in der Provinz waren es in der Regel die Freiwilligen. Die Befehle wurden von der deutschen Seite erteilt, also waren es keine spontanen Aktionen von Seiten der Letten, sondern es kam nur durch die Befehle der Besatzungsmacht. Aber die Schützen wussten schon, wofür sie da sind, und sie hätten sich auch dagegen aussprechen können.
Bei den großen Massenaktionen in Riga in Liepāja, Libau, haben sie nicht geschossen. Sie haben "`nur"' eskortiert. Aber in den Kleinstädten hatten die Deutschen noch nicht so viele Militärangehörige, und deshalb hat man die Einheimischen genommen. Das ist ja das Tragische: Es waren die Nachbarn, die die Leute umgebracht haben. Die Kranken, Kinder, Frauen, Älteren, das war egal, alle. In jeder Stadt wurden diese Einheiten gebildet, es waren nur Freiwillige. In dieser Zeit wurde noch keiner zwangseinberufen, wie das später der Fall war. In den ersten Monaten der Besatzung hat man nur auf Freiwillige gesetzt.

\textbf{Wurden bewusst die entsprechenden Nachbarn als Freiwillige eingesetzt?}

\textbf{Svetlana Bogojavlenska:} Ja, es ist in der Geschichtswissenschaft in Lettland inzwischen auch ein Konsens, dass es von der deutschen Seite so gewollt war, um das als spontane Aktionen zu verkaufen. Dass die Einheimischen die eigenen Nachbarn umbringen, dass der Antisemitismus, der da herrschte, einfach selbst ausgebrochen sei, die "`Vergeltung"' für die vermeintlichen Verbrechen der Juden während des einen Jahres der sowjetischen Besatzung, so wurde das dargestellt. In den Aufrufen, sich zu den freiwilligen "`Selbstschutzmannschaften"' zu melden, stand dann schon, dass diese Einheiten nicht nur für den Schutz da seien, sondern auch dazu, das Vaterland von "`fremden Elementen"' zu bereinigen. Von Kommunisten, von Juden und so weiter. Und das war auch allgemein in der lettischen Bevölkerung bekannt. Noch Jahrzehnte danach war bekannt, wer beteiligt war, auch wenn die Städte sich verwandelt hatten, denn die jüdische Bevölkerung fehlt ja völlig. In die Häuser, die der jüdischen Bevölkerung gehörten, zogen Letten ein. In kleineren Städten war auch bekannt, wessen Familien an der Ermordung beteiligt waren. Man wusste dann auch, wer von den Sowjets verurteilt wurde. Wer dann die 10 Jahre in Sibirien oder in einem Gefängnis verbracht hat und warum er da war. Und diese Familien, so wie ich das in einer Stadt in Lettgallen in Gesprächen mit den Einheimischen Anfang der 2000er Jahre verstanden habe, waren nicht so beliebt. Also die galten nicht als Opfer der sowjetischen Verfolgung. Diejenigen, die bei der zweiten Deportationswelle 1947 deportiert worden sind, die schon.

\textbf{Gab es innerhalb der Freiwilligenverbände im Nachhinein irgendwelche, die sich reuig geäußert haben?}

\textbf{Svetlana Bogojavlenska:} Es gibt eigentlich nur eine Forschung darüber, das ist eine Forschung von Rudīte Vīksne\footnote{Vīksne (2005)}.  Sie hat die Verhörprotokolle von den Freiwilligen aus dem Arājs-Kommando analysiert. Da wurden die Beschuldigten auch nach den Gründen gefragt, warum sie dem Kommando beigetreten waren. Nur sehr wenige meinten explizit, dass sie Juden hassten. Und viele meinten, man konnte das nicht mitansehen, wie die Juden ermordet worden sind, sie hätten es sich völlig anders vorgestellt. Aber wirkliche Reue zeigte keiner. Zumindest geben es diese Verhörprotokolle nicht wieder. Man muss aber damit auch sehr vorsichtig umgehen, die Verhörmethoden vom KGB sind ja weltbekannt. Das, was ich selbst gelesen habe, sind Unterhaltungen von Richtern und Verhörenden aus Preiļi: "`Was haben Sie denn gefühlt, wenn Sie diese Menschen umgebracht haben? Das waren doch Ihre Nachbarn, die sie gekannt haben! Das waren doch Kinder und Frauen, das sind doch auch Menschen!"'. Und dann kam die Gegenfrage, "`Sind die Juden etwa Menschen?"'. Also die Gehirnwäsche, oder tatsächlich die persönliche Überzeugung hat da wahrscheinlich gewirkt.

\textbf{Das sowjetische Regime wollte sich ja nach dem Krieg möglichst stark von der Naziideologie abgrenzen. Ist es dann nicht verwunderlich, dass der Antisemitismus nicht verschwunden ist? Wie passt das zusammen?}

\textbf{Svetlana Bogojavlenska:} Es gibt eine sehr bekannte Rede von Stalin, in der sich Stalin bei allen Völkern der Sowjetunion für den Sieg im Großen Vaterländischen Krieg bedankt, aber besonders bei dem russischen Volk, dass das größte Volk unter allen Völkern der Sowjetunion sei. Und jeder, der das bezweifelte, galt als Feind, obwohl Stalin selbst kein Russe war, sondern Georgier. Dieser ganze Internationalismus beinhaltete immer die Unterstreichung der Rolle eines Volkes. Alle mussten zu sowjetischen Bürgern werden, nicht zu lettischen Bürgern der Sowjetunion. Es gab eine Tendenz zur Russifizierung und Sowjetisierung. Diejenigen, die versucht haben, ihre nationale Kultur in der Zeit des Stalinismus zu pflegen oder wiederzubeleben, wie es Juden versucht haben, waren zum Scheitern verurteilt. Da es für die jüdische Bevölkerung sehr wichtig war, nach dem Krieg die Reste zusammenzuhalten, haben sie versucht das jüdische Theater in Riga wiederzueröffnen. Das wurde ihnen verboten. 
Das war noch Anfang der 50er vor Stalins Tod, nach der Zerschlagung des Jüdischen antifaschistischen Komitees 1948.

\textbf{Gab es da schon Auswanderungswellen nach Israel?}

\textbf{Svetlana Bogojavlenska:} Ja, es gab zumindest Versuche. Auch aus Litauen emigrierten viele Überlebenden Juden. Die erste größere Auswanderungswelle nach Israel gab es in den 1970ern, als es vielen gelungen war über Ungarn die Sowjetunion zu verlassen. Und danach noch einmal in den 80ern. 

\textbf{Gab es auch vorher schon Auswanderungswellen?}

\textbf{Svetlana Bogojavlenska:}Ja, nach den jüdischen Pogromen im 19. Jahrhundert, auch aus Lettland. Und nach der Revolution 1905, die dann auch mit jüdischen Pogromen einherging. 1905 ging es aber überwiegend in die USA und nur ein sehr kleiner Teil nach Palästina. Der Zionismus war ziemlich weit verbreitet, es gab verschiedene zionistische Organisationen. Es gab auch Sommerfreizeiten für die Jugendlichen, die speziell geschult worden sind, wie sie in Palästina die Wüstengebiete in Ackerland verwandeln konnten, "`\textit{Hachschara}"'. Das war sehr populär. Zu Beginn waren es nicht viele, die ausgewandert sind, denn die Zwischenkriegszeit war gerade die Blütezeit der jüdischen Gemeinde in Lettland. Diejenigen, die hingefahren sind und dort sich niedergelassen haben, wurden als Helden angesehen. Aber man kann da nicht von einer großen Auswanderungswelle nach Palästina sprechen. Den Staat gab es ja noch nicht und die Idealisten meinten, "`Wir bereiten jetzt unsere Jugend hier bei uns vor."' Und dann wurde auch Hebräisch in jeder jüdischen Schule gelehrt.

\textbf{Wie sahen die politischen Parteien in der Zwischenkriegszeit aus?}

\textbf{Svetlana Bogojavlenska:}  Die waren sehr zersplittert. Aus dieser Zeit stammt der jüdische Witz: Wo drei Juden sind, sind fünf Parteien. Die jüdische Minderheit vertrat sehr verschiedene Auffassungen, auch was das Leben im lettischen Staat anging. Und man kann auch von keiner Homogenität der jüdischen Bevölkerung in Lettland sprechen. Es gab die sozialistischen Bundisten, die Erzkonservativen Zionisten, die sich einen Gottesstaat in Palästina gewünscht hatten, und auch demokratische Bewegungen.

\textbf{Gab es von diesen Bewegungen Anti-Antisemitische Arbeit?}

\textbf{Svetlana Bogojavlenska:}  Nein. Anti-Antisemitische nicht; die Juden, wie die Russen, die Deutschen und die Weißrussen und die Polen, wie alle Minderheiten, mussten auch in der ganzen demokratischen Periode für ihre Rechte kämpfen. Für ihre Rechte als ethnische Minderheit, dafür, ihre eigenen Schulen zu haben und sie selbst zu verwalten. Weil es immer wieder von Seiten der Regierung Bestrebungen gab, das Schulsystem zu unifizieren und die Schulen zu lettisieren um eine homogene Bevölkerung zu bilden. Es gab Regierungen, die sehr aufgeschlossen den Minderheiten gegenüber waren, unter denen man weniger darum kämpfen musste. Und es gab auch Regierungen, die alles daran gesetzt haben, diese Gesetze, wenn nicht ganz abzuschaffen, so doch ihre Wirkungsbereiche zu begrenzen. Indem sie zum Beispiel die Inspekteure für ethnische Schulen abgeschafft haben und gesagt haben: Jetzt kontrolliert das Bildungsministerium die Schulen der Minderheiten. Das galt zwei Jahre lang, dann hat die neue Regierung die Stellen wieder eingeführt auf Nachdruck der Minderheiten, die dann dafür zusammengearbeitet haben.

\textbf{Wie wurde die Auswanderung nach dem Zusammenbruch der Sowjetunion in Lettland wahrgenommen?}

\textbf{Svetlana Bogojavlenska:}  Dazu habe ich schon Eingangs einiges erzählt. Das Bewusstsein war da, dass Juden auswandern. Aber auch Verwunderung. Was die Migration der jüdischen Bevölkerung in der Sowjetunion betrifft, so muss dazu gesagt werden, dass die Kontakte zwischen den Juden und der russischsprachigen Bevölkerung in Sowjetrussland besser ausgeprägt waren, weil fast alle Juden, die in Sowjetlettland gelebt haben, nach dem Krieg Russisch gesprochen haben. Viele von ihnen sind nach dem Krieg nach Lettland eingewandert, aus der Ukraine oder aus Russland. Und die hatten auch eigentlich keinen Bezug zum vorherigen Lettland. Den Holocaust haben knapp 1500 lettische Juden überlebt, und nicht alle von ihnen sind nach Lettland zurückgekehrt. Diese 1500 sind mit denjenigen zusammengerechnet, die es geschafft haben, vor der deutschen Besatzung Lettland zu verlassen. Von denjenigen, die dort geblieben waren, sind vielleicht knapp 1000 am Leben geblieben.

\textbf{Wie haben sie den Holocaust überlebt? Wurden sie versteckt?}

\textbf{Svetlana Bogojavlenska:} Ja, einige haben sich versteckt. Es sind sogar bis zu 600 Fälle der Judenrettung in Lettland heute bekannt. Das heißt aber nicht, dass sie alle überlebt haben. Anfang der 1990er waren um die 200 Fälle bekannt, in denen tatsächlich Juden überlebt haben. Einige Fälle sind in den Akten der Polizei aufgezeichnet worden, d.h. dass jemand gefunden wurde, der einen Juden versteckt hat. Das bedeute für beide den Tod. Wie überlebt man den Holocaust? Schwierig zu sagen. Viele sind 1944 nach Auschwitz gekommen und dort dann die Befreiung erlebt, einigen ist die Flucht 1944 gelungen, als die sowjetische Armee sich näherte.

\textbf{Haben Sie ein Beispiel von einer Person, die überlebt hat?}

\textbf{Svetlana Bogojavlenska:}  Herr Marģers Vestermanis ist die prominenteste überlebende Persönlichkeit in der jüdischen Gemeinde. Als ich dort noch gearbeitet habe, waren häufig seine Freunde da, also Leidensgenossen, mit denen er zusammen im Ghetto und im KZ war, Konzentrationslager Kaiserwald. Die haben sich auch gegenseitig Bestätigungen darüber ausgestellt, dass sie bezeugen können sich im Ghetto oder im KZ begegnet zu haben. Weil es keine Papiere gibt, die das belegen können. Als die Wehrmacht sich aus Lettland zurückgezogen hat, wurden die Dokumente verbrannt. Deshalb war es schwierig nachzuvollziehen, wie viele da inhaftiert wurden. Die Familie von Herrn Vestermanis ist bei der Rumbula-Aktion in Riga 1941 umgekommen. Er hatte zwei Geschwister, einen älteren Bruder und eine Schwester. Sein Vater war Textilfabrikant in Riga. Er war ein sogenannter kurländischer Jude aus dem Westen Lettlands. Seine Muttersprache war Deutsch, und seine Mutter war russische Jüdin. Die erste Muttersprache von Herrn Verstermanis war Lettisch, weil er eine lettische Nanny hatte. Dann hat er Deutsch gelernt, er war im deutschen Kindergarten. Er meinte, wäre die Familie bei der sowjetischen Deportation von Juni 1941 deportiert gewesen, hätten sie eine Chance gehabt zu überleben. Er selbst hat sich als Elektriker im Ghetto ausgegeben. Er meinte auch, dass er wie durch ein Wunder überlebt hat, weil ihm das abgekauft wurde, obwohl er einmal das ganze Haus, an dem er arbeiten musste, lahmgelegt hat. Und er hat sich danach im Gespräch mit mir auch gewundert, dass er damals deswegen nicht umgebracht worden war.

\textbf{Kommen wir zu der Veranstaltung am 16. März in Lettland, die jedes Jahr einen internationalen Aufschrei erregt. Vielleicht können Sie darauf eingehen, wie der Charakter dieser Veranstaltung ist und welche Interessen hinter dieser Veranstaltung stehen.}

\textbf{Svetlana Bogojavlenska:} Das ist eine Veranstaltung der Veteranen der lettischen Waffen-SS Legion, die 1943 gegründet wurde. Die lettische Waffen-SS hieß auch wie überall in Europa Freiwilligen Waffen-SS Division, was aber nicht stimmte. Freiwillig waren nur die Schutzmannschaftsbataillone, zu denen sich Letten 1941 tatsächlich freiwillig gemeldet haben. Diese wurden dann 1943 in die 15. Waffen-SS Division eingegliedert. Das waren auch diejenigen, die an der Judenvernichtung 1941 teilgenommen haben, auch an den "Sonderexpeditionen" und an der Vernichtung der Zivilbevölkerung in Weißrussland beteiligt waren. All die anderen wurden einberufen, sind aber auch zur Musterung erschienen. Das sage ich, weil uns aus dem Baltikum ein anderer Fall bekannt ist. Die Waffen-SS Legion wurde auch in Estland gegründet und in Litauen. In Estland ist es gelungen, in Lettland ist es gelungen, in Litauen nicht. Nur jeder Fünfte erschien dort zur Musterung. Und diejenigen die erschienen, nur dürftig ausgestattet, haben sich dann geweigert den Eid auf Hitler zu leisten. Und 1944 musste man einsehen, dass sie nutzlos sind. Und deswegen gibt es keine litauische Waffen-SS Legion. 

\textbf{Das heißt aber, Widerstand dagegen war offenbar möglich und wurde auch nicht so sehr geahndet?}

\textbf{Svetlana Bogojavlenska:} Zumindest nicht in Litauen, in Lettland schon mehr. Die ganzen Familien konnten inhaftiert werden, man hat nach ihnen gesucht. Aber wenn Sie den berüchtigten Film "`Lettische Legion"' anschauen, von Uldis Neiburgs,\footnote{Latvian Legion, Filmstudio "`Deviņi"', 2000.} da haben einige Einberufenen erzählt: "`Es kam dieses Zettelchen, ich muss dort erscheinen an dem und dem Datum. Und dann fragt der Vater: "`Und was machst du jetzt? Gehst du hin oder gehst du in den Wald?"'. Was konnte man machen, man wusste ja, die Familie wird verfolgt, dann ging ich hin."' Herr Vestermanis erzählte, dass die Partisaneneinheit, die er im Wald getroffen hatte, als er während des Todesmarsches geflohen war, zum großen Teil aus den Deserteuren der lettischen Waffen-SS bestand. Es gab Leute, die sich gewehrt haben. In Litauen sah es natürlich so aus, dass man sich massenhaft dagegen gewehrt hatte. In Lettland nicht. Sie haben sich dann tatsächlich zusammen mit der deutschen Wehrmacht an den Kämpfen an der Front beteiligt. Sie wurden aber nicht mehr zur Ermordung der Zivilbevölkerung eingesetzt, deshalb kann man tatsächlich sagen, dass die 1943 Einberufenen keine Verbrecher im Sinne des Nürnberger Tribunals sind. Aber es sind auch keine Helden. Sie haben den Eid auf Hitler geleistet, auch wenn viele danach erzählt hatten, "`Ich hatte so ein ungutes Gefühl dabei"'. Es ist auch so, dass dieser 16. März tatsächlich als ein offizieller Feiertag in den lettischen Kalender aufgenommen wurde. Ein paar Jahre später, als es einen großen internationalen Aufschrei gab, wurde es auf Vorschlag der damaligen Präsidentin Vaira Vīķe-Freiberga wieder aus dem Kalender gestrichen. Und jetzt ist es in Riga so, dass jede öffentliche Veranstaltung bei der Stadtverwaltung angemeldet sein muss. Jedes Jahr reichen die Veranstalter die Anmeldung ein, bekommen eine Absage und gehen gerichtlich dagegen vor. Und das Gericht erlaubt es jedes Mal wieder. Das Schlimme ist, dass diese Veranstaltung auch die Neonazis mobilisiert, die antisemitisch eingestellt sind, aus dem In- und Ausland. Auch Rechtsradikale aus der Ukraine, aus Ungarn, sind jetzt jedes Jahr wieder dabei.
Es gibt einen Legionären Friedhof in Vidzeme, in Lestene. Da wurde auch ein Denkmal errichtet, aber es ist anders als in Deutschland, wo es mit Scham gesehen wird. In Lettland gelten die Waffen SS Veteranen tatsächlich als Helden, nicht als Opfer. Zum Beispiel, sieht man in besagtem Film am Ende einen Mann, der in der sowjetischen Armee gekämpft hat, gegen seinen eigenen Bruder, im kurländischen Kessel. Er war jünger, er wurde also später einberufen, als sowjetische Armee nach Lettland einrückte und die Wehrmacht und die Legion sich nach Westen, Kurland zurückzogen. Sein Bruder ist im kurländischen Kessel gefallen. Und er weiß auch, dass er vielleicht derjenige ist, der ihn erschossen hat. Er weiß es ja nicht. Er sagt da: „Zu Sowjetzeit sind die die Verbrecher gewesen, jetzt sind wir die Verbrecher." Das ist eine lettische Familie, die auf beiden Seiten kämpfen musste.
Das Interessante ist, als die ersten Märsche zum 16. März organisiert worden sind, wurde Herr Vestermanis, der auch Historiker ist, gefragt, was er dazu meint. Er war damals der Meinung, die sind keine Verbrecher, die wurden einberufen. Die Verbrecher wurden danach tatsächlich auch von der Sowjetmacht verurteilt und zur Rechnung gezogen. Das sind einfach Veteranen. Es ist unklug, dass sie jetzt da rausgehen. Sie durften das die ganze Sowjetzeit nicht, sie galten ja als Verbrecher, wurden aber nicht bestraft dafür, dass sie in der Legion waren. Die Mehrheit nicht. Herr Vestermanis hat sie damals verteidigt.

\textbf{Aber er wird nicht die Sichtweise der Waffen-SS Veteranen verteidigen, oder?}

\textbf{Svetlana Bogojavlenska:} Nein, natürlich nicht. Er hat sie auch als Opfer des Krieges dargestellt. Man kann sie aber nicht mit den anderen Opfern vergleichen. Einmal war sogar der Verteidigungsminister bei deren Gedenkveranstaltung in Riga dabei. Seine Handlung wurde natürlich danach verurteilt von der Regierung. Die evangelische lutherische Kirche macht allerdings jedes Mal mit. Es gibt einen großen Gottesdienst im Dom. Die marschieren dann von der Domkirche durch die ganze Altstadt zum Freiheitsdenkmal. Früher war das nur auf dem Friedhof in Lestene. Aber jetzt hat es offizielle Züge durch die Beteiligung der lutherischen Kirche, die im Dom auch zum Beispiel am Unabhängigkeitstag Lettlands einen Gottesdienst feiert. In derselben Kirche, mit demselben Pfarrer. Jedes Jahr sind am 16. März auch mehrere Parlamentsabgeordnete dabei.

\textbf{Gibt es lettische Politiker oder andere bedeutende lettische Figuren, die sich antisemitisch äußern?}

\textbf{Svetlana Bogojavlenska:}  Ja, es gibt die Partei Tēvzemei un Brīvībai, “für das Vaterland und Freiheit”. In dieser Partei gibt es ultranationale Kräfte, die den extremen Nationalismus salonfähig machen wollen. Sie kommt immer ins Parlament hinein, seit Anbeginn. Von denen kommt ab und zu schon was, aber nicht von den Abgeordneten in der Saeima. Die sind natürlich gegen alle, nicht nur gegen Juden.

\textbf{Gibt es zivilgesellschaftliche Initiativen gegen Antisemitismus und Rassismus in Lettland?}

\textbf{Svetlana Bogojavlenska:}  Staatliche Maßnahmen gibt es jede Menge. Es gibt immer wieder europäische Projekte, es gibt immer wieder soziologische Untersuchungen, es gibt immer wieder Empfehlungen der Wissenschaft an die Politik, was man noch machen könnte, um die Gesellschaft zu konsolidieren und demokratischer zu gestalten. Aber ob das alles tatsächlich da ankommt, wo es ankommen sollte, das ist schwer zu beurteilen. Es sieht nicht danach aus. Offen antisemitische Politiker werden auch nicht so stark angeprangert. Die Kritik kommt dann immer nur von der Opposition und nicht von den etablierten lettischen Parteien.

\textbf{In welcher Weise werden jüdische Letten heute im Alltag mit Antisemitismus konfrontiert?}

\textbf{Svetlana Bogojavlenska:} Offen überhaupt nicht, aber sie zeigen sich sehr selten als Juden in der Öffentlichkeit. Ich habe meine jüdischen Freunde in Lettland gerade gefragt und die meinen, latenter Antisemitismus sei da. Man spürt, dass man anders behandelt wird. Eine gewisse Aggressivität im Verhalten, vor allem wenn es um den Beruf geht. Man wird da nicht so gerne gesehen und wird ausgegrenzt, wenn man als Jude erkannt wird. 

\textbf{Und wie groß ist das Problembewusstsein? Wie sehr ist Antisemitismus im Bewusstsein der Leute in Lettland?}

\textbf{Svetlana Bogojavlenska:} Ich glaube, es ist nicht im Bewusstsein. Es gibt lettische Wissenschaftler, die sehr wohl wissen und verstehen, dass es dieses Problem gibt. Aber ich denke nicht, dass in der Mehrheit der Bevölkerung ein Bewusstsein dafür da ist. Wenn es darauf ankommt, dann sagt man sowas wie hier in Deutschland: "`Man darf das wohl noch sagen..."', "`Ich bin kein Antisemit, aber..."' 

\textbf{Wissen Sie, ob jüdische Institutionen dem Antisemitismus ausgesetzt sind?}

\textbf{Svetlana Bogojavlenska:} Ich weiß, dass die Synagoge in Riga von der Polizei rund um die Uhr bewacht wird. Einmal wurde da eine Flasche mit Brennstoff reingeworfen. Und die Wände der Synagoge wurden beschmiert. Aber das liegt schon mehr als 10 Jahre zurück. Man hat dann auch auf den Bildern der Überwachungskamera gesehen, dass das Jugendliche waren. Und der damalige Rabbiner, Natans Barkans, meinte, die sollten einfach zuhause von ihren Eltern dafür bestraft werden. Er glaubte auch nicht, dass das irgendwie fundiert antisemitisch war. Aber dass die Synagoge als Zielscheibe diente, das ist schon markant. Man könnte dann sagen, wenn das Jugendliche sind, woher haben sie das? Und dann bietet sich die Antwort, die haben das aus dem Umfeld, aus der Familie. Höchstwahrscheinlich haben sie etwas gehört und dann ist es in die Taten übergegangen. Diese Jugendlichen sind jetzt inzwischen erwachsen. Was sie jetzt darüber denken, weiß man nicht. Die Synagoge wird seitdem rund um die Uhr von der Polizei bewacht. Da steht immer ein Bus voller Polizisten. Die jüdische Gemeinde hatte mal ein sehr offenes Durchgangssystem. Man konnte da einfach so rein. Und viele Ausländer haben das bewundert und uns im Museum gefragt: "`Haben Sie hier keine Angst, einem Angriff der Antisemiten zum Opfer zu fallen?"' Man hatte tatsächlich keine Angst. Inzwischen kommt man nur durch die eine Tür rein. Die zweite Tür ist geschlossen. Auch um ins Museum zu kommen muss man sagen wohin und zu wem man möchte.
Ab und zu sind es Letten, die dort arbeiten. Nur leider sind das Ausnahmen. Die Letten werden dann immer gefragt "`Was machst du da? Das ist doch ein jüdisches Museum."' Ich wurde auch immer gefragt "`Bist du Jüdin?"', und musste sagen "`Nein"', "`Und was machst du dann da?"', "`Arbeiten?"'. Als die renovierte Synagoge wiedereröffnet wurde, wurde darüber in der lettischen Presse berichtet. Ich habe die Kommentare der Leser gelesen und Angst bekommen. Einer der harmlosesten Kommentare war "`Müssen wir Letten tatsächlich davon in Kenntnis gesetzt werden?"'. Das ist diese Nicht-Bereitschaft, das als Teil der Kultur und Geschichte Lettlands anzuerkennen.
Was allerdings die Aufarbeitung des Holocaust betrifft, da hat Lettland, zumindest die lettische Wissenschaft, große Fortschritte gemacht. Es ist fast alles aufgearbeitet worden, was aufgearbeitet werden konnte. Fast alle Archivbestände sind gesichtet und systematisiert worden. Und das Okkupations-Museum hat sogar zusammen mit dem jüdischen Museum vor einigen Jahren, 2011, eine Ausstellung zusammen gemacht namens "`Rumbula. 1941. Anatomie des Verbrechens"',\footnote{Die Ausstellung ist auf der Internetseite des Museums digital zugänglich: http://okupacijasmuzejs.lv/rumbula/en} explizit zur Judenvernichtung.

\textbf{Ich hätte noch eine Frage zur Schulbildung in Lettland. Wie sehr sind in den Lehrplänen die Themen Judentum und Diskriminierung gegen Juden verankert?}

\textbf{Svetlana Bogojavlenska:} Das Thema ist auf jeden Fall präsent. Es war sogar eine Lettin, Ieva Gundare, sie arbeitete bis vor kurzem im Okkupationsmuseum, eine der ersten, die sich überhaupt überlegt hat, wie man das den Letten erklären kann, ohne den Nationalstolz zu verletzen. Zu sagen, das lettische Volk hat bei der Judenvernichtung mitgemacht, nicht das ganze Volk, sondern einige Letten. Dem jüdischen Museum war es sehr wichtig zu sagen, dass nicht das ganze Volk am Judenmord beteiligt war, sondern bestimmte Individuen, die namentlich bekannt sind. Man kann nicht das ganze Volk für den Judenmord verurteilen. Die anderen waren ja auch Opfer. Und das war sehr wichtig auch für Ieva Gundare und das Okkupationsmuseum. Es gab bestimmte Menschen die am Judenmord beteiligt waren, es gab aber auch Menschen, die den Juden geholfen haben. Und es gab auch diejenigen, die sich das gleichgültig angeschaut haben. Oder mit Schaudern. Oder, mit Staunen, "`was passiert jetzt?"', und sie wussten nicht, was sie tun könnten. Frau Gundare hat Pädagogik studiert und war diejenige, die 2001-2002 die Arbeitsmaterialien, die ersten Arbeitsmaterialien für lettische Schulen in engen Beratungen mit dem Museum "`Juden in Lettland"' ausgearbeitet hat, um das Thema den Schulkindern nahe zu bringen. Sie hat dann danach im Okkupationsmuseum das gleiche für die sowjetischen Deportationen gemacht. In dem sie auch aufgezeichnet hat, dass nicht nur Letten darunter gelitten haben, sondern auch andere ethnische Gruppen, die in Lettland zu dieser Zeit gewohnt haben.
Es gab schon die Tendenz zu sagen, dass es ein Genozid gegen das lettische Volk war. Was aber nicht stimmt, es war kein Genozid gegen das lettische Volk. Das waren Repressionen gegen alle Völker Lettlands, gegen die lettische Bevölkerung oder vielmehr gegen Bevölkerung Lettlands. Weil wenn man lettisch sagt, denkt man in Lettland nur an die Letten, nicht an die Bevölkerung Lettlands, die ja multiethnisch aufgestellt ist. Und das war ihr dann auch gut gelungen.
Es gibt in der Schule in jedem Fall ein Programm. Nur muss man bedenken, dass der Schulplan natürlich anders ist als hier. Während hier dem Nationalsozialismus viel Zeit gewidmet wird, ist es dort anders, man hat dort andere wichtige Themen. Und es hängt sehr vom Lehrer ab. Ich habe auch schon Berichte gehört, dass die Lehrer das Thema einfach weglassen, sodass der Holocaust gar nicht erwähnt wird. An der Universität gehört es dazu. In der Schule muss es wie gesagt mindestens eine Stunde sein, aber die wird nicht von jedem Lehrer durchgeführt. Aber ich weiß, dass man an den Schulen Projektwochen hat, und erstaunlicherweise entscheiden sich immer noch ziemlich viele Schulen, auch aus der Provinz, für einen Gang ins jüdische Museum.
\section{Vitaly Liberov}

\textit{Mr Liberov will introduce himself in the course of the interview, which took place in Nuremberg on March 8th, 2017. The interview waa conducted in Russian and translated into English for this publication.}\par
\vspace*{2em}
\textbf{Could you please tell us something about yourself first?} 

\textbf{Vitaly Liberov:} My name is Vitaly Liberov. I am forty years old. I am an entrepreneur. I was born in Bryansk, Russia. I was raised as usual Soviet kid, I was a Young Pioneer, then I joined Komsomol, so, I was not familiar with religion. But as I was growing up, I remember men in our family going together somewhere several times a year, whispering about something, coming home with big slices of Matzah wrapped up in paper, and things like that. But it all was so strange to me that I did not pay attention. The first moment I realised I was a Jew was when I came to government office to receive my passport at age of sixteen. The official asked me: “Kid, do you really want to be registered as a Jew?”. I naively answered: “How could that be otherwise?”. It was the time of Perestroika, 1989 or 1990. The first branches of the Jewish Agency Sokhnut\footnote{The Jewish Agency for Israel (Hebrew: \textit{HaSochnut HaYehudit L'Eretz Yisra'el}) is a Jewish non-profit organisation founded in the early 20th century that is seeking to connect Jews worldwide with their people and heritage, primarily by fostering immigration to Israel (\textit{aliyah}). With the collapse of the Soviet Union, many thousands of Jews emigrated from its territory into Israel. The Sochnut supported this process.} started to appear in the Soviet Union, and that was the reason why I decided to discover my origin. I discovered that my grandparents spoke Yiddish in their youth, my grandmother’s original name was not Galina as I thought, but Golda. The fear of the Soviet system was so great, that everybody in our family was afraid of talking about their Jewish origin. That is why I knew only a few words in Yiddish, as elderly people in our family used them occasionally. And there I was standing with my passport, in which the word “Jew” was written down, and I knew nothing about that. That is how I became an extremely active member of the Sokhnut Bryansk branch. I started working as a \textit{Madrich}\footnote{Hebrew word for caretaker}, and I liked it.  It was really amusing, because we had to organise camps for children without any knowledge of Judaism and traditions. So, there were evenings and nights before the events when we had to learn ourselves. That was quite an interesting time. Then my parents decided to emigrate to Germany. It was quite a difficult decision for me personally, I was not fond of my parents’ decision because I wanted to go to Israel. But it was not for me to decide, that is why I in 1996 I came to Germany, and now I live in Germany for a longer time than I lived in Bryansk. When I came to Germany, I directly applied to the Center Council of Jewish Communities of Germany. That is how I became a member of Jewish community.  

\textbf{Is it correct that you realised that you are a Jew at the age of sixteen?} 

\textbf{Vitaly Liberov:} I always knew I was a Jew. Even as a kid I had to fight with other kids who were mocking me because I was a Jew. But until the age of sixteen, I never considered myself as a Jew religiously. It is hard to explain, that there is a difference between being a Jew as ethnicity and being a Hebrew as religion. But at age of sixteen, I decided not to distinguish these two meanings – I accepted both ethnicity and religion.  

\textbf{Were your relationships with other children good, before the age of sixteen?} 

\textbf{Vitaly Liberov:} I do not want to say that I was an unhappy child. In Soviet school it was officially not acceptable to distinguish people by ethnicity, because all were Little Octobrists, then Young Pioneers and so forth. And if someone was trying to say something bad against Jews in school, then the teacher had to punish this kid. State propaganda showed the Soviet Union as a multi-national state of fraternal friendship among the peoples, but in fact there was a huge control over the number of Jews accepted to universities, so-called quotas. In the streets, of course, there was anti-Semitism among the children.  

\textbf{How did your family preserve Jewish traditions?} 

\textbf{Vitaly Liberov:} In a very rudimental form. For example, my grandparents used no to eat pork, not to eat bread during Passover or to restrain themselves from food during Yom-Kippur. It was not the question of preserving traditions, but rather some little nuances. And, of course, no Yiddish could be spoken. But, by the way, when I came to Germany, I surprisingly discovered that German people, especially elderly in Bavaria, use some Yiddish words without realising it. For example, ``Masel'', which means ``happiness'', or ``Tachlis'', which means being honest or speaking the truth.  

\textbf{How did you become a member of the Jewish Community of Nuremberg?} 

\textbf{Vitaly Liberov:} I just came to the synagogue for a daily prayer, and I got acquainted with other members. I wanted to keep doing what I was doing in Bryansk. Luckily, the Jewish emigration from post-Soviet countries started, and I got a lot of friends for whom Russian was their native language. And of course, there is a peculiarity of Judaism, that whenever you pray, whenever you celebrate Shabbat or any other holiday, there are people in every country of the world that are doing the same. That is why you cannot be alone. I was never alone when I became a member of Nuremberg community.  

\textbf{Is the Jewish community of Nuremberg different today from what it was like when you came?} 

\textbf{Vitaly Liberov:} Tremendously. I came before the large wave of emigration started. Before that the people at the synagogue could not gather for the Minyan to read the prayer. But in the end of the 1990s, the community hugely enlarged. There are a lot of members of community that regularly attend ceremonies. We have a nursing home, a big hall for events, classes for youth, we even think about opening a kindergarten. Sometimes hundreds of people attend celebrations.  But, of course, we have also other kind of events, for example lectures, classes, conferences. We invite people from other organisations, everybody is welcome. I think that Nuremberg is getting a new ``face'' thanks to the growing Jewish community, because German people are interested in coming to us.    

\textbf{How many Jews live in Nuremberg at the moment?} 

\textbf{Vitaly Liberov:} I know that there are two thousand members of our community, and I think it is forty percent of all Jews in Nuremberg. Actually, our community is not the only one. It just happened historically, that there is also Chabad community and several alternative groups. These groups even have their own rabbis, who are less conservative. Our community is a classic orthodox community. 
Our liturgy was written in the middle of 19th century, specifically for the needs of community.   

\textbf{Does your community have contacts with Israel?} 

\textbf{Vitaly Liberov:} We have Hebrew classes. Nuremberg’s sister city is Hadera, that is why we have exchange programmes with Israeli schools. Many Israeli students come to Nuremberg and especially to our community. 

\textbf{Do you think that there is anti-Semitism in German society?} 

\textbf{Vitaly Liberov:} My answer would be very subjective. I just want to remember summer of 2014, when there was a military operation in Gaza. At this time, I was really afraid, because everyone – far-right, far-left, Muslim organisations – united in hatred for the Jews. People were out on the streets shouting anti-Semitic things like ``\textit{Hamas, Hamas, Juden ins Gas}'', and the Police did not do a thing to stop them. A huge crowd of young men broke into the main station building because they thought the owners of Burger King and McDonald’s are Jews, whereas in Nuremberg, these stores belong to Muslims. They vandalised the building because of hatred for the Jews. That is why I can definitely say that there is anti-Semitism in Germany. I think there is a huge gap in the educational system. After the War, the topic of the Holocaust was broadly discussed at schools in Germany. Moreover, there were people who Survived during the War, and they could tell a lot. Nowadays, there are a lot of children at schools to whom the topic of War and the Holocaust is not as close as to people of my generation. Unfortunately, to children from some Muslim families, the topic of the Holocaust is irrelevant, or they consider it to be fake. In my opinion, this is a problem or the educational system. There is no unified system of teaching history among Federal Lands in Germany. There is no unified pedagogic plan. Of course, there are standards set by the Ministry of Education, but in the end, everything depends on the teacher, who are afraid to have a discussion. That is why many teachers just tell pupils to write an essay on Spielberg’s Schindler’s list, which does not show the Holocaust. Speaking about reasons of anti-Semitism, I would like to refer to director Sokhnut, Nathan Sharansky. He developed the so-called ``Three D'' test of anti-Semitism: deligitimisation, demonisation and double standards. Each of these ``three Ds'' is a basic of anti-Semitism. Some people say, ``Yes, Israelis are entitled to protect themselves, but they at least have automatic rifles, Palestinians do not'', or ``Yes, Jews were killed during the War, but other peoples were killed, too''. Nowadays, anti-Semitism is not as primitive as before, because some people tend to hide anti-Semitism under a leftist fight against capitalism and globalism. Some tend to use euphemisms like ``antizionism'' or ``critique of Israel'', which is a pure form of anti-Semitism. Sometimes, these are the same people that attack refugees, both in reality and in social media. People do not realise that anti-Semitism is not only a problem of Jews – this is a problem of society as a whole. Even when I speak with my German friends, they admit that they do not consider anti-Semitism as their problem. I remember that when I was a \textit{Madrich} in a Jewish camp in Frankfurt in 1998, some German kids sneaked to the venue to see, as they said, how the Jews drink blood. This example shows how deeply anti-Semitism is rooted in society, and how different the forms are that it can take.  

\textbf{What can be done to fight anti-Semitism?} 

\textbf{Vitaly Liberov:} First of all, we have to work with children. It is necessary to show at school the consequences of hatred and intolerance. Secondly, we must make our politicians listen. Of course, in the end summer 2015 there was a huge manifestation in Berlin near the Brandenburg gate organised by the Center Council of Jewish Communities of Germany. Even Angela Merkel attended. Some politicians came and said right things, but most of them do not care. There are still fascist parties in Germany, which are not prohibited. It just changes its name from ``German National Union'' to ``Gathering of German Nationalists'' and then to ``National Party of Germany''. They pretend to be democratic, but in reality, everyone knows its nature, despite that the Construction Protection Bureau inspect them. But there were even underground groups who murder people, vandalise shop and businesses. Unfortunately, the police are not efficient in catching them. 

\textbf{We would like to ask you about the Forum of Jewish Culture in Germany. Could you please tell us about this organisation?}  

\textbf{Vitaly Liberov:} This organisation is based in Nuremberg. I am a member of the managing board. It is a society which consists of Jewish and non-Jewish members. Our purpose is to tell about the history and traditions of Jews in Germany and in Nuremberg, particularly. We organise some events, lectures, we invite representatives of different confessions to discuss some topics, for example marriage, raising children and so on. We also organise excursions and seminars. People like it and the number of participants increases.  

\textbf{Does the Forum speak about Reformation and Martin Luther’s attitude toward the Jews, about his anti-Semitism?} 

\textbf{Vitaly Liberov:} Yes, and this topic was especially interesting for Catholics. But Nuremberg is mainly a Protestant city, that is why it is not so easy to speak about this. Unfortunately, in some churches in Germany there is still such an anti-Semitic thing as the \textit{``Judensau''}, ``Jewish pig''. Just a few years ago, stones from a Jewish cemetery were found in the floor of the southern tower of the Lorenzkirche in Nuremberg. The former chairman of the Community, Arno Hamburger, was working on taking these stones out of church, there were many obstacles imposed by the Lutheran church. But we managed it, and now these stones are back on the cemetery. The history of Jews in Nuremberg is very rich. One of the oldest Jewish cemeteries is in Nuremberg. Actually, Nuremberg is tragically connected to Riga, since many Nuremberg Jews were deported to Riga and shot in Biķernieki forest during the War. We have also so-called Riga Committee, which is working on commemorating those people.  
\section{Dr Axel Töllner}
\begin{otherlanguage}{ngerman}
\textit{Dr. Axel Töllner (*1968) is a Protestant pastor and, since 2014, Federal Church Commissioner for Christian-Jewish Dialogue in Bavaria, Germany. After graduating from high school in Munich in 1987, he completed his civil service and studied Protestant theology in Erlangen, Kiel and Jerusalem. During his studies he completed additional training in Christian journalism and journalism. From 2003 to 2011 he worked as a pastor in various communities. He received his doctorate in 2007 at the University of Koblenz-Landau and then worked until 2010 as a research associate at the project Synagogues-Memorial Band Bavaria at the Chair of Modern Church History at the Friedrich-Alexander-University Erlangen-Nuremberg. From 2010 to 2011 he worked as a research associate at the Working Group Catholicism and Protestantism Research for the NS Documentation Centre Munich, then until 2014 again at the Synagogue Memorial Band Bavaria. \\
The interview with him took place on April 21st, 2017, in Nuremberg, it was conducted in German.}\par  
\vspace*{2em}
\textbf{Dr. Axel Töllner:} Bei unseren Recherchen zu einem Projekt mit etwa 200 jüdischen Gemeinden in Unterfranken sind wir auf das Problem gestoßen, dass die Quellenlage zu diesen vielen kleinen Orten total disparat ist. In einem Ort gibt es unendlich viel und im anderen Ort gar nichts. Ein anderes Phänomen, auf das wir gestoßen sind, ist dass die Leute im einen Ort Leute schon relativ früh angefangen haben, sich mit ihrer jüdischen Geschichte zu beschäftigen, und in anderen Orten scheint bis heute irgendwie ein Deckel drauf zu sein. Aber das folgt auch keiner Gesetzmäßigkeit, das lässt sich keiner bestimmten Region zuordnen, im Nachbarort kann das schon wieder ganz anders sein. Das ist auch sehr abhängig von einzelnen Personen – oft sind das Leute, die da nicht groß geworden, sondern zugezogen sind.\\
Das ist auch ganz interessant, wenn man sich Ortschroniken anschaut, oder Vereinschroniken oder Ähnliches, da sind meistens nur kurze Notizen: 1933 wurde "`Der-und-der"' Vorsitzender und der nächste Eintrag ist dann erst 1945 – die ganzen Heimkehrer oder Flüchtlinge, die dann dazugekommen sind und dem Vereinsleben eine ganz neue Richtung gegeben haben… Die ganzen Ausschlüsse durch irgendwelche Arier-Paragraphen, solche Geschichten, das wurde nicht dokumentiert. 
Es ist auch oft schwierig an das Archivgut der Gemeinden ran zu kommen, weil die Gemeindearchive zum Teil nicht geordnet sind. Ich habe von einem Ort gehört, wo der frühere Bürgermeister sich mit der jüdischen Geschichte des Ortes beschäftigt hat und eine Publikation dazu geschrieben hat. Von seinen Quellen war dann aber überhaupt nichts mehr auffindbar, und jetzt haben sie in den letzten Jahren erst hinten in der Ecke des Gemeindearchivs den ganzen Stapeln gefunden, den hat der einfach beiseite geschafft, er hat das Material also nicht entsorgt, sondern im Archiv versteckt, und das ist eigentlich ein perfektes Versteck. 

\textbf{Man stößt ja bei solchen Recherchen immer noch auf sehr starke antisemitische Ressentiments.  
Es gab mal die Initiative eines Juden, der in den USA lebt, der einen ziemlich hohen Geldbetrag spenden wollte für eine Gedenktafel, die an die Synagoge und an die Juden erinnert, und da war der ganze Gemeinderat dagegen. Vielleicht kannst du mal deine Erfahrungen an ein paar Beispielen anführen?} 

\textbf{Axel Töllner:} Also, was mir immer wieder auffällt ist, dass selbst die Leute, die sich mit der jüdischen Ortgeschichte beschäftigen und sagen, dass das wichtig ist, immer von den Deutschen reden und von den Juden. Das ist also eine Sondergruppe, die anders ist als die Anderen. Dass die seit Jahrhunderten in dem Dorf gelebt haben, das Dorf mitgeprägt haben, das ist im Bewusstsein nicht so eingedrungen. Das finde ich sehr interessant, dass selbst bei den Leuten, die sich bemühen, die jüdische Geschichte ins Gedächtnis zu rufen, noch die Vorstellung da ist, dass die Juden irgendwie die Anderen sind.  
Und ich kann mich an ein Forschungsprojekt erinnern, auch aus einem unterfränkischen Dorf, mit den Schulen und den Kirchengemeinden vor Ort: "`jüdisches Leben in unserem Dorf"'. Das mündete dann in einer Broschüre und es wurden Stolpersteine verlegt. Und ein Künstler im Ort hat versucht, von der Synagoge, die 1938 zerstört wurde, eine schöne Zeichnung zu machen, basierend auf historischen Fotos. Und interessant ist, wie er die Leute dargestellt hat, die in die Synagoge reingehen. Das sind nämlich Leute, die gekleidet sind wie Ultraorthodoxe mit Schläfenlocken, mit schwarzen Hüten und Kaftanen und so etwas. Das ist überhaupt nicht durch historische Fotos gedeckt, sondern die Fotos zeigen Leute, die gekleidet sind wie alle anderen auch. Ich finde es interessant, dass sich dieses Bild so festgesetzt hat. 
Gelegentlich wurde ich an verschiedenen Orten mit der Haltung konfrontiert "`das alte Judenzeugs, das ist doch lange vorbei, das interessiert doch keinen mehr"' usw. - da gibt es also Leute, die scheinbar nicht wollen, dass man da was aufrührt. Ich habe auch mit verschiedenen Heimatforschern gesprochen, die sich zum Teil seit vielen Jahren auf diesem Gebiet engagiert beschäftigen. Die erzählen dann auch – besonders, wenn es darum geht, wer ein Haus gekauft hat, in dem früher eine jüdische Familie gelebt hat – unter welchen Umständen man an dieses Haus gelangt ist, oder an irgendwelche Möbelstücke. Das ist immer noch schwierig.  
Ich habe den Eindruck, die heutigen Leute identifizieren sich so sehr mit ihrer Großelterngeneration – die wissen schon, dass da irgendwas nicht mit rechten Dingen zuging, aber "`das sind ja unsere Verwandten, die werden es schon richtiggemacht haben."' 

\textbf{Wie gehen eigentlich die Gemeinden mit Synagogen um, die noch erhalten sind?} 

\textbf{Axel Töllner:} Das ist ganz unterschiedlich. Es gibt Synagogen, die restauriert oder wieder zugänglich gemacht wurden. Ein ganz schönes Beispiel ist die ehemalige Synagoge in Arnstein, ungefähr 30 Kilometer nördlich von Würzburg – eine wunderbare Synagoge, frühes 19. Jahrhundert, klassizistisch – die hat im frühen 20. Jahrhundert noch eine ganz tolle Jugendstilbemalung bekommen, neben dem früheren Toraschrein, der zerstört worden ist. Oben rechts und links waren Greife aufgemalt, und jeder dieser Greife hatte ein Wappen in der Hand - das eine ist weiß-blau und das andere ist schwarz-weiß-rot. Ein wunderbares Beispiel für das Selbstverständnis der Juden. 
Die Stadt hat diese Synagoge vor einiger Zeit gekauft, weil sie vorher in Privatbesitz war, sie war zu Wohnraum umgebaut worden und durch die nachträglich eingebauten Zwischengeschosse sind ein paar Sachen kaputtgegangen, aber es ist im Grunde noch relativ viel von der Bemalung erhalten – 2010 wurde die Synagoge saniert und ist seitdem als Museum und Veranstaltungsort zugänglich.  
In anderen Orten findet sich niemand, der bereit ist, etwas an der Synagoge zu machen oder Geld zu investieren. Viele der Kommunen, vor allem in Unterfranken, leiden darunter, dass die Leute wegziehen – da gibt es viele leerstehende Häuser im Ort, die Infrastruktur bricht langsam zusammen, und dann gibt es vielleicht noch eine Synagoge - das wertzuschätzen und da Energie rein zu stecken und das Ortsleben neu beleben zu können, das findet man eigentlich selten. Die Synagoge ist da, aber andere Sachen sind uns jetzt einfach wichtiger. Ob da irgendwie antisemitische Ressentiments dahinterstecken, das weiß ich nicht – das ist schwierig. 
Interessant ist, was verschiedene Heimatforscher erzählen, wie sie in ihren Orten angefeindet werden – für uns ist es ja in gewisser Weise komfortabel, wir gehen dahin, wir unterhalten uns da mit den Leuten und sind dann wieder weg. Und die, die da leben, sind damit konfrontiert, dass andere sie schief anschauen und immer sagen "`die mit ihrem Judenzeug da"' und die zum Teil auch Kontakte zu Nachkommen von Überlebenden oder Migranten pflegen und die dorthin einladen und durch den Ort führen, wo dann die Deutschen gucken, "`was ist denn jetzt mit denen, und wollen die womöglich irgendwelche Rückerstattungsansprüche geltend machen?"' 
Und man hört auch so Sachen wie "`meine Großeltern, die haben damals das Haus 1936 oder so einem Juden abgenommen, weil der halt wegmusste, die haben natürlich einen fairen Preis bezahlt! Und nach dem Krieg wollten die nochmal Geld haben, da fühlt man sich dann irgendwie schon beschissen!"', Dieses Stereotyp, dass die Juden irgendwie nach Geld gieren, dass die immer auf ihren eigenen Vorteil schauen und sowas, das kommt da schon durch, das hört man immer wieder mal.  
Aber man hört auch das Andere, "`die Juden, die sind ja besonders geschäftstüchtig gewesen, haben dann eben viele erfolgreiche Geschäfte gehabt hier im Ort"', also auch hier diese Sonderstellung, sie hätten besondere Qualitäten und deswegen waren sie so erfolgreich und deswegen haben sie sich den Neid der anderen zugezogen. Ich habe den Eindruck, das ist für viele Leute oft sehr schwierig, das so einzuordnen, dass das ganz normale Kaufleute waren, von denen einige gute Ideen hatten, wie sie bestimmte Vertriebsmodelle einfach ausbauen und dadurch auch sehr innovativ auf dem Land wirken konnten. 
Man hört auch mal "`bis 1933 war das Zusammenleben in unserem Ort eigentlich wunderbar; die Christen und die Juden haben sich super verstanden."' Die Frage, warum das dann so plötzlich anders geworden ist, können die Leute dann nicht beantworten. "`Ja, das war halt Propaganda und die Leute waren abgeschnitten von Informationen! Das einzige, was es im Ort gegeben hat, war der Stürmerkasten und wenn man da jahrelang geprägt ist, dann glaubt man dem irgendwann."' 
Da ist ja sicherlich was dran, aber die Vorstellung "`du bist mein jüdischer Nachbar, dich kenne ich, du bist natürlich anders, das weiß ich schon, aber die Juden sonst, die sind problematisch und gegen die muss man was machen"', das ist schon seltsam. 

 
\textbf{Zumindest in Mittelfranken war ja die evangelische Kirche schon ziemlich frühzeitig nationalistisch und antisemitisch geprägt, hast du dafür irgendwelche Erklärungen?} 

\textbf{Axel Töllner:} In Akten aus dem 19. Jahrhundert finden sich überhaupt keine konfessionellen Unterschiede, was judenfeindliche Ressentiments angeht. Man könnte vielleicht sagen, dass im katholischen Bereich Legenden – Geschichte, dass die Juden Ritualmorde verüben würden und sowas – verbreiteter waren. 
Der Katholizismus hat durch die Priester, die tendenziell in der bayrischen Volkspartei organisiert gewesen sind, aus parteipolitischen Gründen zunächst gegen die Nazis opponiert. Und auch die Erfahrungen aus dem Kaiserreich haben bedingt, dass da eine gewisse Distanz da war, aber es zeigt sich jetzt in Unterfranken, wobei das konfessionell total zersplittert ist, dass der Klerus, als der kirchliche Druck weg war, die Meinung vertreten hat, die Nationalsozialisten seien nicht wählbar für Katholiken, weil sie bestimmte christliche oder katholische Dogmen in Zweifel ziehen oder bestreiten. Sobald das wegbricht, setzt doch relativ schnell ein Nazifizierungsprozess ein. Was zum Beispiel antisemitische Gewalt in den einzelnen Orten angeht, kann man heute nicht sagen, dass das in den evangelisch geprägten Orten exzessiver wäre als in den katholisch geprägten Orten. Ich denke, Mittelfranken hat natürlich mit Julius Streicher auch noch eine Gallionsfigur gehabt, der hat insgesamt das Ganze unglaublich hochgeschaukelt und die Situation für die Juden wirklich schon extrem früh zur Hölle gemacht. Otto Hellmuth als Gauleiter in Unterfranken hat längst nicht diese Möglichkeiten gehabt wie Streicher, obwohl auch der natürlich auch schon sehr rigide gewesen ist.  
Also - Erklärungsmuster zu finden, das ist wirklich sehr, sehr kompliziert und viele Protestanten haben ihre Hoffnung in die NSDAP gesetzt, als "`evangelische Partei"' und so... 

\textbf{Hat das vielleicht auch was mit Luther zu tun?} 

\textbf{Axel Töllner:} Die Zwanzigerjahre sind dafür ganz interessant, da ist die Auffassung von Luther als Antisemit gar nicht mal so präsent, zumindest im klassisch-kirchlichen Protestantismus.  
Die völkische Bewegung – also Theodor Fritsch in seinem Antisemiten-Katechismus – hat verschiedene Lutherschriften zusammengestellt, um zu zeigen, dass Luther schon sehr früh erkannt hat, wie die Juden "`wirklich sind"' und so… Das hat sicherlich bei manchen dazu geführt, dass der Gedanke aufkam, das sei anschlussfähig. \\
Es gibt aber auch Stimmen, die sagen, "`die Nazis instrumentalisieren Luther und versuchen ihn dann noch irgendwie zu retten – Luther hat schon sehr heftige Sachen gegen die Juden gesagt, aber im Grunde ging es ihm ja um eine religiöse Auseinandersetzung und nicht darum, die Juden da in jeder Hinsicht zu diffamieren"', da würde ich heute sagen, das sehe ich skeptischer als manche damals, aber das muss man, denke ich, auch als Apologetik einordnen. Wir wollen die Deutungshoheit über Luther behalten. Das hat sicherlich auch etwas damit zu tun, aber es gibt im Katholizismus auch genau solche vehementen Judenhasser – das unterscheidet sich nicht sehr. 

\textbf{Das ist sehr interessant – Nürnberg war ja gewissermaßen die protestantische Hochburg des Antisemitismus, Wien dagegen zu gleicher Zeit die katholische Hochburg des Antisemitismus. Das scheint regional sehr unterschiedlich zu sein und von verschiedensten Faktoren abzuhängen. 
Jetzt ist ja gerade das Lutherjahr und es finde viele Veranstaltungen zu Luther statt. Luther wird zwar nicht gefeiert wegen seinem Rassismus, aber es scheint relativ wenig zu geben, was über seine antisemitischen Veröffentlichungen und Haltungen aufklärt, oder?} 

\textbf{Axel Töllner:} In der Forschung ist das inzwischen unübersehbar. Ich glaube, bei diesem Jubiläum wird das erstmals ein bisschen fundamentaler angegangen. Allerdings gibt es natürlich unterschiedliche Strömungen – manchen geht es viel zu weit, die sagen "`Wenn wir uns so kritisch mit dem Luther auseinandersetzen, dann betreiben wir Geschichtsklitterung"' und andere sagen "`es geht nicht weit genug und man hört nicht genug darüber."' Ich habe verschiedene Vorträge zu dem Thema gehalten und finde es total interessant, dass die immer gut besucht sind und ich wirklich auf ein großes Interesse stoße. Viele der Zuhörer sagen, das hätten sie gar nicht gewusst, das wäre sehr erschütternd. Dabei gibt es viele Gemeinden, sowohl Kirchengemeinden als auch Kommunen, die Veranstaltungen dazu anbieten, es gibt auch verschiedene Wanderausstellungen zu dem Thema, die zum Beispiel in ehemaligen Synagogen ausgestellt werden. Manche Gemeinden machen auch kleine Veranstaltungsreihen zu dem ganzen Thema und beschäftigen sich damit – ich würde sagen, das hat in den letzten Jahren zugenommen und die Fragen sind auch ernsthafter und selbstkritischer als früher.  
Die EKD-Synode hat im Herbst 2015 eine – wie ich finde – recht beachtenswerte Erklärung verabschiedet, wo sie sich nicht nur – und das ist meine Kritik – auf ein paar wenige Stellen innerhalb von Luthers Werk beschränkt, also diese klassischen Forderungen, die Synagogen anzuzünden, die Juden zu vertreiben und zu vernichten, Zwangsarbeit und solche Dinge. Dieser Forderungskatalog, der ist furchtbar genug, aber das ist eben nur die Spitze des Eisbergs, und man kann sich nicht damit begnügen, zu sagen, "`das ist furchtbar gewesen an Luther und war eine Entgleisung des späten Luthers, aber ansonsten sind wir froh, dass er auch noch anderes gesagt hat"'. Das geht sehr, sehr viel weiter und da versuche ich auch immer den Fokus darauf zu legen, dass die Art und Weise, wie Luther die Bibel interpretiert hat, unser eigentliches Problem ist. 
Gegen so exzessiven Judenhass werden sich schnell viele Leute stellen. Da sind wir uns schnell einig, da braucht man keine Vorträge zu halten, da braucht man auch keine Ausstellungen zu machen, sondern jeder wird das ekelhaft finden. Aber wenn es um die Frage geht, welches religiöse Recht die Juden in Luthers Werk haben und wie er sie insgesamt betrachtet – das müssten mehr in den Fokus treten. Also bei Luther ist es nicht ganz einheitlich, es gibt auch ein paar Stellen, die einen optimistischer stimmen, aber insgesamt finde ich es ziemlich ernüchternd. 
 
\textbf{Man liest ja manchmal heute noch die Positionierung, dass Christen Juden missionieren sollten. Wie verbreitet ist diese Haltung?} 

\textbf{Axel Töllner:} Die gibt es immer noch, aber auch die EKD-Synode hat im letzten November eine Erklärung veröffentlicht, wo sie das, was in den letzten Jahren als kirchlicher Konsens unter den Gremien und Kirchenleitungen gewachsen ist, zusammengefasst hat: "`Wir sind nicht dazu berufen, als Christen den Juden den Weg zum Heil zu weisen."' Das hat heftigen Widerspruch hervorgerufen. Die Frage, "`darf ich dann als Christ einem Juden nicht mehr sagen, dass ich an Jesus glaube, und dass Jesus für alle Menschen gut ist?"' – das ist damit nicht gemeint. Wenn ich mich mit Juden unterhalte, und die etwas von mir wissen wollen, dann erzähle ich denen natürlich auch, was ich glaube – weil ich von ihnen auch erwarte, dass sie mir das erzählen, so wie sie das erleben. Aber ich verfolge nicht das Ziel, dass sie das genauso glauben müssen, wie ich, und das ist der Unterschied. Jeder soll für seinen Glauben einstehen seinen Standpunkt zeigen, aber mein Gegenüber muss auf derselben Augenhöhe stehen und das selbe Recht haben. Der muss auch seinen Standpunkt sagen können und mich damit konfrontieren, so verstehe ich das.  
Aber es kommt natürlich drauf an, wie ich das mache. Ich würde ich mich jetzt nicht in Tel Aviv oder Jerusalem mit einem Stand auf die Straße stellen – aber, wenn ich in Vorträgen gefragt werde "`Sag mal was als Christ dazu"' oder, wenn ich in Diskussionen, in denen wir als Juden und Christen zusammen über die Bibel reden, gefragt werde "`Wie siehst du das als Christ?"', dann sage ich natürlich, wie ich das sehe. Und dann sagen die, wie sie das sehen und wir stellen fest, da gibt es vielleicht Gemeinsamkeiten und da gibt es auch Unterschiede und damit habe ich kein Problem. Und ich denke mir, Gott wird sich schon was dabei gedacht haben, dass er das so gemacht hat, er hätte es ja auch einfacher haben können.  

\textbf{Nochmal zu deinen Erfahrungen bei der Recherche zu den Synagogen – hast du Unterschiede zwischen den Generationen festgestellt bezüglich der Haltung zum Judentum oder dem Interesse bzw. Desinteresse?} 
 
\textbf{Axel Töllner:} Ich habe in allen Generationen engagierte und desinteressierte Leute getroffen – wobei mein Eindruck ist, dass in den Schulen das Thema jüdische Geschichte mittlerweile zum Heimat- und Sachkundeunterricht dazugehört. Ich verspreche mir davon, dass es selbstverständlicher ist für die jungen Leute. Ob daraus ein Fazit erwächst, da würde ich jetzt mal nicht zu optimistisch sein – aber zumindest, dass es dabei hilft, das mal als Realität anzuerkennen! Unseren Ort haben über Jahrhunderte hinweg Juden mitgeprägt. Dass das Realität ist, dass zur Heimatgeschichte auch die jüdische Geschichte dazu gehört, das ist nochmal ein Unterschied im Denken zu früheren Generationen. Ich habe den Eindruck, es gibt unter den älteren Leuten einige, die jetzt mittlerweile auch schon gestorben sind, die sich mit der jüdischen Geschichte beschäftigt haben, auch einige, die erzielen irgendeine Art Wiedergutmachung. Die haben als Kinder oder Jugendliche vielleicht miterlebt, wie ihre jüdischen Klassenkameraden gehänselt und verfolgt wurden, haben vielleicht selber mitgemacht. Die hatten dann hinterher das Gefühl "`Ich muss da irgendwas machen"' – mit den Veranstaltungen 1988, 50 Jahre Novemberpogrome, haben einige Leute angefangen, sich mit der jüdischen Geschichte im Ort zu beschäftigen und haben dann sehr engagierte Sachen geschrieben. Die können zwar nicht unbedingt wissenschaftlichen Kriterien standhalten, aber es wird deutlich, dass die Leute von irgendeiner Schuld getrieben sind und versuchen, das auf irgendeine Weise wieder gut zu machen. 

\textbf{Vielleicht ist da Gunzenhausen ein ganz gutes Beispiel? Da gibt es eine Schule, die sind da ziemlich engagiert - die haben Kontakt aufgenommen mit Nachfahren der Juden, die jetzt in den USA wohnen, haben einen Austausch organisiert und dadurch einen persönlichen Zugang geschaffen.}
 
\textbf{Axel Töllner:} Ja, das ist natürlich ein Beispiel für wirklich nachhaltige Arbeit. Es gibt ja auch Orte, da ist eine engagierte Lehrkraft, und wenn die wieder weg ist, dann schläft das wieder ein. Aber dass das dort schon über viele Jahre hinweg gelingt, das finde ich eine tolle Sache. Ich glaube, da haben die Kleinstädte auch gewisse Vorteile.  
Es gibt eine ganze Reihe von Beispielen, wo Schulklassen recherchiert haben und dann über die Heimatforscher an Überlebende oder deren Nachkommen gekommen sind und dadurch Informationen über die Menschen bekommen haben und zum Teil auch Leute Interesse bekommen haben, sich mal anzuschauen, wo ihre Familie eigentlich herkommt, wo dann auch Kontakte entstanden sind. Aber es hängt in der Tat von den einzelnen Leuten vor Ort ab, die da engagiert sind. 

\textbf{Inwiefern gibt es eigentlich noch die ganz archaischen, antisemitischen Vorurteile, zum Beispiel in dem Buch zum Thema 100 Jahre Wasserthüringen aus den 80er Jahren? Da steht drin, die Juden legen deshalb die Steine auf die Gräber, damit, wenn der Messias wiederkommt, sie ihn nochmal steinigen können. Gibt es sowas immer noch?} 

\textbf{Axel Töllner:} Sowas habe ich schon immer wieder mal gehört. Volksglaube ist unglaublich nachhaltig, das kommt, denke ich, auch von bestimmten Vorstellungen, die Juden als Sondergruppe sehen - "`die wollen unter sich bleiben"' und "`die haben so komische Bräuche, und am Shabbat mussten dann immer die Christen zu ihnen kommen und den Ofen anschüren"' und sowas. Natürlich, das hat es ja gegeben – aber die Vorstellung, dass Christen unter Juden dienen müssen, das gibt das Gefühl, die Juden hätten sich für etwas Besseres gehalten, und das merkt man schon auch noch immer wieder. Das ist schon interessant, wie zäh solche Vorstellungen sind. 

\textbf{Jetzt ist ja zum Glück ein direkter, offener Antisemitismus in Deutschland – im Gegensatz zu Polen, Ungarn, Österreich und vielen anderen Ländern – nicht gesellschaftsfähig, sondern eher tabuisiert. Antisemitismus ist trotzdem weiterhin vorhanden, was wissenschaftlich untersucht ist – die Frage ist, wie dieser versteckte Antisemitismus in Erscheinung tritt.} 

\textbf{Axel Töllner:} Es gibt zum Beispiel immer wieder Friedhofsschändungen, deswegen sind die Friedhöfe verschlossen, und es ist manchmal schwierig, herauszufinden, woher man den Schlüssel dafür kriegen kann.  

\textbf{Wir haben auch ein Interview mit dem Rabbiner aus Fürth geführt, und der sagt, typisch für ihn seien, auch während der Woche der Brüderlichkeit, antisemitische Tendenzen unter Freunden. die Christen leiten das dann immer ein, er hat da stapelweise Briefe, mit "`Kritik unter Freunden muss doch wohl erlaubt sein"', und dann legen sie los.} 

\textbf{Axel Töllner:} Was ich auch immer wieder finde, ist die Aussage "`Das war ja so schlimm damals, was die Nazis mit den Juden gemacht haben, aber jetzt, was die Israelis mit den Palästinensern machen, das ist doch eigentlich nicht sehr viel anders. Das ist doch eigentlich schlimm, die sollten es doch besser wissen."' und so. Das ist sehr weit verbreitet und das höre ich auch oft nach Vorträgen, wenn Leute sich melden und fragen "`Und, was sagen Sie denn jetzt dazu?"' – und natürlich, sowas wie "`Kritik unter Freunden"', da würde ich tendenziell sagen, das ist auch eine Generationenfrage – das scheint mir doch eher die ältere Generation zu sein, die das sagt, weil viele von denen haben angefangen in den Gesellschaften, weil sie natürlich was gegen den Antisemitismus machen wollten und mit Juden freundlich reden wollten, aber letztlich das Gefühl hatten "`ganz das richtige ist das Judentum nicht"'. 

\textbf{Da ist doch auch bei den deutschen Juden so eine Unsicherheit. Eine Art Selbsthass, weil ein "`anständiger Jude"' – sagen die Israelis zum Beispiel – ein Zionist ist und nicht auf den Gräbern seiner Vorfahren wohnt. Ich habe die Erfahrung gemacht: Israelis, die hier zu Besuch sind, wollen eigentlich mit den Nürnberger Juden nichts zu tun haben. Ein anständiger Zionist wohnt nun mal nicht in Nürnberg, sondern in Tel Aviv. Und ich denke, da ist auch bei den Nürnberger Juden, die ich so kenne, eine Unsicherheit bzw. Selbsthass – "`eigentlich müsste ich ja in Israel wohnen"'. Die meisten machen dann einen Kompromiss und haben eine kleine Eigentumswohnung in Tel Aviv, damit sie sagen können "`ich wohne auch in Israel"'.}  

\textbf{Axel Töllner:} Das kommt noch dazu. Das jüdische Leben in Deutschland nach 1945 ist schon immer ein angefochtenes Leben gewesen. Die wurden von den Nicht-Juden genauso wie von Juden in Israel oder europäischen Ländern kritisiert, die für sich in Anspruch genommen haben "`Wir haben wenigstens nicht die Shoah erfunden"'. Auf der anderen Seite nehme ich doch wahr, dass es auch bei den Leuten, die aus der ehemaligen Sowjetunion zugewandert sind, die älteren, die es ohnehin schwierig haben, die mit Deutschland erstmal nichts zu tun haben – und hier auch keine Wurzeln haben oder sich hier nicht in den Nachkriegswirren niedergelassen zu haben – die sich erstmal zu diesem Land und zu diesem Staat verhalten müssen. Wobei es interessant ist, dass diejenigen, die in den 90er Jahren als jüngere Leute gekommen sind, dass viele von denen sehr dankbar sind dafür, dass sie hier einen Ort gefunden haben, wo sie leben können – auch sehr sensibel sind, was so antisemitische Tendenzen angeht – wenn man weiß, dass sie das aus ihrer Kindheit und Jugend aus der Sowjetunion noch kennen und eigentlich sagen "`Deutschland ist für mich das Land, wo es diesen staatlichen Antisemitismus nicht gibt, und weswegen ich hier auch gut sein kann – und klar, ich hab viele Verwandte in Israel, aber da ist es mir zu heiß und da ist alles komisch, das ist so eine fremde Welt – hier ist es mir doch noch lieber."' \\ Und dann gibt es mittlerweile die Erfahrung, das hab ich in Israel gehört, letztes Jahr, es gibt ja die Klassenfahrten in der 9./10. Klasse nach Polen, nach Auschwitz und in andere Vernichtungslager, und das Interessante ist, dass viele junge Israelis ein ganz positives Deutschland-Bild haben, und die Shoah jetzt von Deutschland abspalten und auf Polen projizieren. So nach dem Motto "`Das hat alles da stattgefunden – Polen ist das böse Land und die Polen sind die Bösen! Und die Deutschen sind aber unsere Freunde"' und so – und das gibt dann zum Teil eine ganz komische Schlagseite... 

\textbf{Darum reagieren die polnischen Politiker dann gereizt, wenn von "`polnischen KZs"' die Rede ist.} 

\textbf{Axel Töllner:} Genau, das ist aber auch ein wirkliches Problem! Ich glaube, es gibt natürlich auch Deutsche, denen das gar nicht unlieb ist, die sagen "`Das können wir jetzt outsourcen!"', um das mal ganz flapsig zu sagen. Aber umso wichtiger ist es dann, hier die Begegnungen zu haben.\\
Was ich auch interessant finde ist, dass in den letzten ca. 10 Jahren auch die Vereine in den Dörfern langsam ihre jüdischen Mitglieder wiederentdecken, oder auch die Leute, die im späten 19. Jahrhundert diese Vereine mitgegründet haben - Turnverein, Freiwillige Feuerwehr, etc. Da gab es ja wirklich an vielen Orten Juden, die da ganz vorne mitgearbeitet haben. Auch wenn man sich neuere Festschriften anschaut, hat man Fotos, in welchen dann Aussagen stehen wie: "`Unser jüdisches Mitglied so-und-so"'. Da verändert sich schon einiges. Aber es ist nach wie vor sehr unterschiedlich, je nachdem, wo man sich befindet. 
Dabei fällt mir noch die Meiser-Debatte ein, bei der doch deutlich geworden ist, dass hierbei noch einiges schlummert, wobei es Leute versuchen, auf einer definitorischen Ebene zu lösen und zu sagen: "`Das ist Antisemitismus, das ist kein Antisemitismus"'. Das, finde ich, kann es eigentlich nicht sein. Und das Niveau der Diskussionen fand ich zum Teil schon echt erschütternd, muss ich sagen. 

\textbf{Ja, am besten haben das die Münchner gelöst. Die haben die Meiserstraße in Katharina-von-Bora-Straße umbenannt, haben also eine Antisemitin für einen Antisemiten...} 

 \textbf{Axel Töllner:} Wir wissen ja von ihr eigentlich zu wenig. Ich meine, es gibt eigentlich nur diese Geschichte, dass Luther sagt: "`Wenn du jetzt hier wärst, dann würdest du sagen, die Juden haben's gemacht"', aber er schreibt auch in diesem Brief an seine Frau 1546, er sei an einem Dorf vorbeigefahren, wo Juden wohnen und der giftige Judenwind wäre ihm kalt ins Kreuz gefahren oder so ähnlich und er schrieb dann: "`Wenn du jetzt da gewesen wärst, würdest du sagen, es wäre das Werk der Juden gewesen."' So sinngemäß. Dass er sterbenskrank sei und die Juden ihn vergiftet hätten.\\
Die Namensumbenennungen haben auch ihre Probleme. Also, ich denke, man kann auch auf die Art und Weise Geschichte entsorgen. Wenn da diese Meiserstraße Meiserstraße heißt, dann ist sie für mich jedenfalls ein permanenter Pfahl im Fleisch, dann muss ich mich damit auseinandersetzen. Und jetzt habe ich die Katharina von Bora, eine Frau, super. Ich habe jemanden, der nichts mit München zu tun hat, jemand von vor 500 Jahren. Das ist auch eine Form der Geschichtsbewältigung, die ich problematisch finde. 

\textbf{Wäre es nicht besser, die Straße so zu belassen und dann wenigstens irgendwie eine Erläuterung hinzuzufügen?} 

\textbf{Axel Töllner:} Das hatte ja der Eckart Dietzfelbinger\footnote{Wissenschaftlicher Mitarbeiter im Dokumentationszentrum Reichsparteitagsgelände} hier in Nürnberg vorgeschlagen. Das hätte ich persönlich auch besser gefunden, aber der hat von einem antisemitischen Nationalprotestanten und einem nationalprotestantischen Antisemiten gesprochen. Da fühlte sich wiederum die Familie Meiser diffamiert. Da geht es eben wirklich um: "`Was ist jetzt antisemitisch?"' Oder: "`Darf man jemanden Antisemiten nennen?"' Das hat in der Form nicht geklappt. Aber bis heute gibt es dann die Partei, die sagt, das sei alles ein Fehler gewesen. In München gab es eine Gemeinde, die einen Saal in ihrem Gemeindehaus danach dann in "`Hans-Meiser-Saal"' umbenannt haben, aus Trotz. Und ich finde, dass die inhaltliche Auseinandersetzung mit dem Meiser da ein bisschen auf der Strecke geblieben ist. Die Suche nach einem Etikett, um das auf eine bündige Formel zu bringen, die hat manches damals schwierig gemacht. Für mich persönlich war der Meiser ein Antisemit. 
Das Problem ist nur: Wenn ich diesen Satz sage, habe ich damit noch nicht sehr viel gesagt. Denn die einen Leute verstehen da eine Sache darunter und die anderen verstehen etwas ganz Anderes. Das ist, finde ich, nach wie vor schwierig auf den Punkt zu bringen, zu erklären, warum der Meiser Antisemit gewesen ist. Und dass Antisemit nicht gleich Antisemit bedeutet. Das ist etwas, dass man auch ganz schwer vermitteln kann, dass der Antisemitismus sehr viele Spielarten hat und auch die Frage, dass jemand antisemitische Denkstrukturen haben kann, ohne dass er jetzt in jedem zweiten Satz sagt: "`Die Juden sind unser Unglück!"', ich sage das jetzt mal bisschen plakativ. Sondern dass es da subtile Formen gibt, die bestimmte, jahrhundertealte Denkmuster weiter tradieren und in diese Schiene reingehören. Das finde ich schwer zu vermitteln. Zumindest habe ich damit gerade in so gut kirchlichen Kreisen meine Probleme, mich da verständlich zu machen. Und auch, dass auch jemand, der ein Antisemit ist, etwas getan haben kann, das aus unserer heutigen Sicht als gut bezeichnet werden kann. Fällst du mit dem Urteil, jemand war ein Antisemit, ein Totalurteil über das ganze Leben von ihm, oder kann man sagen: "`Der war Antisemit, aber wir haben ihm auch in mancher Hinsicht was zu verdanken"'? 
 
\textbf{Das war in Polen sehr extrem. Da gab es ja die Pelota, das war eine antisemitische polnische Organisation, eine erklärt antisemitische polnische Organisation, die Juden gerettet haben und dadurch ihr Leben aufs Spiel gesetzt haben. Weil sie gesagt haben: "`Ja, Juden nach Madagaskar, Juden raus aus Polen – aber umbringen? Nein."' Also auch extrem, die Position.} 

\textbf{Axel Töllner:} Ja, und das Problem ist, je intensiver man sich damit beschäftigt, desto vielschichtiger werden die Graustufen, die man dann wahrnimmt. Und das irgendwie so ein bisschen bündiger zu vermitteln, ich denke jetzt noch mal zurück an diese Erklärungstafel an so einem Straßenschild. Das ist schwierig. Aber ich finde, das wäre eigentlich das Ziel. 

\textbf{Ein positives Beispiel: Ich weiß nicht, ob ihr das Kriegerdenkmal kennt in der vorderen Ledergasse. Das ist so ein Denkmal für alle möglichen gefallenen Nürnberger im Boxeraufstand in China, Herero-Aufstand - und zu den Herero gibt es dann unten drunter noch eine Tafel, und auch zum Boxeraufstand. Das ist gut gelöst.} 

\textbf{Axel Töllner:} Das finde ich auch. Weil es im Grunde ja auch zur Geschichte Nürnbergs dazugehört, dass der Stadtrat damals beschlossen hat, die Spitalgasse nach dem Nürnberger Hans Meiser zu benennen und ich finde auch, wir können uns nicht nur die Sahnestückchen raussuchen und sagen: "`Das ist prima und den Rest, den schieben wir ganz nach hinten, wo ihn niemand finden kann. Im Archiv beispielsweise."' Ich denke, heutzutage könnte man mit elektronischen Medien, mit Apps sicherlich auch einiges machen. Zum Beispiel irgendeinen QR-Code, damit die Leute ein paar weitergehende Informationen bekommen, wenn sie sich dafür interessieren und mehr darüber erfahren möchten oder es noch nicht verstanden haben. Das wäre vielleicht noch eine Möglichkeit. \\
Es gibt auch eine ganze Reihe von Fotos und Zeichnungen, wo man zumindest noch die Häuser erkennen kann, die zum jüdischen Viertel in Nürnberg gehört haben. Ich finde ganz interessant, dass es in den 1980ern oder 1990ern, einmal Grabungen in der Frauenkirche gab und da wurde auch so ein Sockel von einer Säule der Synagoge gefunden. Und dann haben die Archäologen von dem Säulenabdruck sagen können, wie hoch die Säule wahrscheinlich war und wie groß die Synagoge dann wohl gewesen ist. Ein bisschen vergleichbar mit Regensburg, die von diesem Altdorfer Stich, die man so ein bisschen kennt. So was noch sichtbarer zu machen, das wäre eine tolle Sache. Aber es gibt so viele schöne Sachen, die man machen könnte. Bis dahin, dass ich mir denke, alles was da so unter dem Hauptmarkt drunter ist mal richtig archäologisch zu untersuchen, das wäre eigentlich eine Pflicht, die der Stadt des Friedens und der Menschenrechte gut anstehen würde. 
\end{otherlanguage}

\section{Dr. Katja Wezel}
\begin{otherlanguage}{ngerman}
\textit{Dr. Katja Wetzel has been a research assistant in the project ``The cosmopolitan city. Riga as a Global Port and International Trade Metropolis (1861-1939)'' at the Seminar for Medieval and Modern History of the Georg-August University Göttingen.\\
Her research focuses on the history of the Baltic countries, especially Latvia, on history politics and culture of remembrance, nationalism and ethnic conflicts, as well as spatial history and digital history.\\
From 1999, she studied history and English in Heidelberg, Aberystwyth and St. Petersburg; her Ph.D. thesis  at the Ruprecht-Karls-University Heidelberg is titled ``History as Politics. Latvia and the Historical Reappraisal after Dictatorship''. From 2013 to 2018, she worked in the field of history at the University of Pittsburgh. We interviewed her via Skype on June 30th, 2017.}\par 
\vspace*{2em}
\textbf{Inwiefern haben Sie sich mit Antisemitismus in Lettland auseinandergesetzt?}

\textbf{Katja Wezel:} Ich habe meine Dissertation zur Aufarbeitung des Kommunismus in Lettland geschrieben.\footnote{Wezel (2016)} Dabei habe ich mich ursprünglich gar nicht mit der Frage auseinandergesetzt, wie Lettland mit seiner nationalsozialistischen Vergangenheit umgegangen ist. Aber mir ist relativ schnell klar geworden, dass die beiden Themen immer wieder zusammenfallen, gerade bei der teilweisen Verwendung des Genozid-Begriffs. Es gibt bestimmte Gruppen oder auch Wissenschaftler, die diesen auf die sowjetischen Deportationen in Lettland anwenden, insbesondere die beiden großen Wellen 1941 und 1949. Da habe ich mich natürlich gefragt, inwiefern es eine Auseinandersetzung mit der jüdischen Geschichte und mit dem Holocaust in Lettland gibt. Dabei habe ich herausgefunden, dass der Holocaust als eine Art Vergleichsfolie verwendet wird. Natürlich liegt der Hauptfokus für die Letten auf der Aufarbeitung des Kommunismus und auf der Aufarbeitung der Verbrechen Stalins. Dann erfolgt ein Rückgriff nach dem Motto: Der Holocaust wurde aufgearbeitet und das müssen wir jetzt auch mit den Verbrechen Stalins tun. Dieses Vergleichsmoment birgt natürlich bestimmte Problematiken in sich und das habe ich primär untersucht. Meine These ist, dass in Lettland ein ganz starkes Augenmerk auf den Molotow-Ribbentrop oder Hitler-Stalin-Pakt gelegt wird, der als der Beginn allen Übels betrachtet wird. Dem wohnt dieser Vergleich der Verbrechen schon inne, denn wenn Sie sagen, damit hat alles begonnen, Hitler und Stalin haben sich verbündet gegen uns Kleine – so wird das aus lettischer Sicht wahrgenommen – dann ist dem implizit, dass die Verbrechen von Hitler und von Stalin auf gleiche Weise begangen wurden und deswegen auf gleiche Weise aufgearbeitet werden sollten. Weshalb der Hitler-Stalin-Pakt so zentral ist, geht dabei zurück auf den Beginn der lettischen Unabhängigkeitsbewegung. 

\textbf{Für manche steht Kārlis Ulmanis noch heute für die Unabhängigkeit Lettlands. Was war charakteristisch für seine Diktatur und wie ging es den Minderheiten unter seiner Herrschaft?} 

\textbf{Katja Wezel:} Ulmanis ist eine wichtige Gestalt, um die Zwischenkriegszeit zu verstehen, wobei man da ein bisschen früher ansetzen muss. Ulmanis war mehrfach gewählter Ministerpräsident, bis er 1934 einen Putsch gemacht hat. Nach dem Ersten Weltkrieg ging es sehr stark darum, dass die Letten die Verantwortung für ihren eigenen Staat übernehmen und die Minderheiten aus verantwortungsvollen Positionen entfernt werden. Für den Fall der Universität Lettlands wurde dies vom schwedischen Forscher Per Bolin in einer Detailstudie untersucht.\footnote{Bolin (2012)} Er weist darin nach, dass es immer darum ging, Letten in verantwortungsvolle Positionen zu bekommen und dass Minderheiten zwar akzeptiert wurden, aber nur für den Übergang. 
In der kompletten Zwischenkriegszeit spielten kulturelle Grenzen und die Identifikation mit der Muttersprache eine bedeutende Rolle. Aus lettischer Sicht war man Deutscher, wenn man Deutsch als Muttersprache sprach, obwohl sich die meisten Deutschen als Baltendeutsche und nicht dem Reich zugehörig gefühlt haben. 
Der Coup 1934 ist aus Ulmanis' Sicht notwendig, denn Lettland war eine instabile parlamentarische Demokratie, die mit genau solchen Problemen zu kämpfen hatte wie Deutschland und andere Staaten dieser Zeit auch. Lettische Regierungen haben immer nur drei, vier Monate gehalten und brachen dann wieder zusammen. Es gab wahnsinnig viele Parteien und Regierungen im Parlament, weil es keine 5\%-Klausel gab wie heute; das machte politisches Arbeiten sehr schwierig. Nach der Wirtschaftskrise wurde es politisch noch schwieriger, was einer der Gründe war, weshalb sich Ulmanis an die Spitze gestellt hat. Eine der wichtigsten Konsequenzen dieses Coups war sein Slogan ``Lettland den Letten!''. Es ging darum, vor allen Dingen die ganzen Deutschen loszuwerden. Das hängt damit zusammen, dass Riga sehr stark von der deutschen Kultur geprägt war. Es gab aber relativ viele Juden, die auch deutschsprachige Schulen besuchten und die im deutschsprachigen Milieu zu Hause waren. Und es gab gerade unter Händlern, Kaufleuten und Unternehmern relativ viele Juden, gegen die sich das natürlich auch richtete, auch in Form von Enteignung. Ich würde aber sagen, dass dieses Enteignungsvorgehen gegen die jüdische Bevölkerung relativ wenig mit Antisemitismus zu tun hatte und viel mehr mit einem sehr, sehr starken Nationalismus und diesem Verständnis, ``Lettland den Letten!''. Dieser starke ethnozentrische Nationalismus ist letztendlich ein Problem, das Lettland heute in einer anderen Form immer noch hat, jetzt richtet er sich gegen die Russen. 

\textbf{Wenn Minderheiten über die Sprache definiert wurden, galten deutschsprachige Juden dann überhaupt als eigene Minderheit?} 

\textbf{Katja Wezel:} Das Problem der jüdischen Bevölkerung in Riga, aber auch in Lettland allgemein war, dass sie sehr stark auseinanderfiel. Jiddisch sprachen eigentlich kaum noch Leute in Lettland. Es gab zwei Gruppen von Juden in Lettland. Die einen hatten sich der deutschen Gemeinschaft mehr oder weniger angeschlossen und schickten ihre Kinder auf deutsche Schulen, die anderen waren aus dem russischen Zarenreich gekommen und sprachen Russisch. Zwischen der deutschen und der russischsprachigen jüdischen Gemeinde gab es aufgrund der Sprachbarrieren relativ wenig Miteinander. 
Die Religion war dennoch von Bedeutung. Es ist wichtig festzustellen, dass es im späten 19., auch noch im frühen 20. Jahrhundert undenkbar war oder faktisch nicht vorgekommen ist, dass ein Nichtjude einen Juden oder eine Jüdin heiratete, weil die konfessionellen Grenzen sehr bestimmend waren. Die Deutschbalten waren alle Lutheraner. Es gab durchaus \textit{intermarriage}; es gab Verheiratung zwischen Letten und Deutschen, die den gleichen Glauben hatten, aber mit Juden kam es faktisch nicht vor. In dieser Hinsicht gab es einen großen Unterschied zum deutschen Reich. Dort gab es jüdische Bevölkerung, die sich komplett assimilierte und sogar zum protestantischen Glauben übertrat. Das gab es in Riga oder in Lettland gar nicht. Das führte dazu, dass die Gruppe als solche weiterhin sehr separat lebte, als ``die Juden'', selbst wenn sie Deutsch sprachen oder wie einige wenige anfingen, auch Lettisch zu sprechen. Sie wurden trotzdem als die Anderen gesehen. 

\textbf{Gab es wie in Lettland auch im Mittelalter schon Antisemitismus? Ab wann gab es überhaupt Juden in Lettland?}

\textbf{Katja Wezel:} Die ersten Juden, die auf dem Gebiet des heutigen Lettlands lebten, waren in Kurland, das zu Polen-Litauen gehörte, bis das komplette Gebiet 1795 von Katharina der Großen mit der polnischen Teilung einverleibt und Teil des russischen Zarenreiches wurde. Das heißt, in Riga gab es im Mittelalter zwei, drei jüdische Händler, die von Kurland kamen, nach Riga gereist sind und ihre Ware vertrieben haben. Aber das waren Handelsreisende und keine wirklich sesshaften Juden. Erst im 18. Jahrhundert fing es an, dass Juden auch sesshaft wurden. Es war ihnen zuerst nicht erlaubt, innerhalb der Stadtmauern zu leben. Die meisten Juden lebten in der Moskauer Vorstadt, wo es ein ursprünglich von der Stadt eingerichtetes und später selbst betriebenes Gebiet ähnlich einem Ghetto gab. Die meisten Juden lebten dort auch weiterhin, selbst als das Verbot, in der Stadt zu leben, aufgehoben war. In der Rigaer Altstadt wurde erst relativ spät eine große Synagoge gebaut, die 1905 fertiggestellt wurde. Vorher gab es noch keine Synagoge in der Altstadt, sondern nur in den Gebieten, in denen sie lebten. 
Juden waren in Lettland nie eine sehr große Gruppe. Das unterscheidet zum Beispiel eine Stadt wie Riga stark von Vilnius. Juden machten vielleicht sechs, sieben Prozent der Bevölkerung aus und auch in der Zwischenkriegszeit waren Juden mit bis zu 5\% in der Bevölkerung eine relativ kleine Gruppe. Diesen historischen Antisemitismus findet man in der Region des heutigen Lettlands in der Form daher nicht. 
Wie haben die Letten die Juden gesehen? Mein Eindruck ist, dass man sie distanziert als die Anderen betrachtet hat. ``Sie interessieren uns nicht, solange sie uns nichts tun und solange wir nicht das Gefühl haben, sie nehmen uns etwas weg.'' Wenn es Ausschreitungen gab, die man als antisemitisch aufgeladen sehen kann, was gerade an der Universität Lettlands in der Zwischenkriegszeit vorgekommen ist, ging das von einzelnen radikal-konservativen, nationalistischen Studentengruppierungen aus.\footnote{Cf. Bolin (2012)} Die Burschenschaften in Lettland sind in der Zwischenkriegszeit relativ stark gewesen, von denen ging das aus. Und aus den Burschenschaften speisen sich auch später diejenigen, die das sogenannte Arājs-Kommando bilden, das Mörderkommando, das dann für die Deutschen während der nationalsozialistischen Besatzung Lettlands letztlich die Arbeit macht. Das war eine Gruppe von sehr radikalen nationalistisch gesinnten Menschen. Es ist immer wieder in Forschungsarbeiten nachgewiesen worden, etwa von Katrin Reichelt, dass das eher eine Art des Opportunismus war.\footnote{Reichelt (2011)} Sie wollten davon profitieren, dass die Juden weg sind und waren weniger von antisemitischen Grundüberzeugungen geprägt. 

\textbf{Wie hat die Zusammenarbeit zwischen lettischer Bevölkerung und Nationalsozialisten ausgesehen?} 

\textbf{Katja Wezel:} Sehr wichtig ist das sogenannte Arājs-Kommando. Das war eine Gruppe von insgesamt ca. 3000 Männern, wobei nicht alle die ganze Zeit dabei waren. Es kam ja erst zur sogenannten sowjetischen Besatzung von 1940 bis 41 und dann zur nationalsozialistischen. Arājs ist ein absoluter Karrierist und Opportunist, der erst versuchte, bei den Kommunisten anzuheuern. Als diese dann weg waren, ist er umgeschwenkt und hat sich bei den Nationalsozialisten in Stellung gebracht und wollte für sie arbeiten. Er hat versucht persönlich zu profitieren, was bei vielen der Fall war, die mitgemacht haben. Das waren häufig Männer, die eine sehr starke nationalistische Überzeugung hatten, die sich auch aus der Ulmanis-Zeit speiste, ``Lettland den Letten und alle anderen sollen weg''. Das war allerdings eine Gruppe von sehr Radikalen und betraf nicht das Gros der Bevölkerung. Dieses Arājs-Kommando funktionierte dadurch, dass sie persönlich profitiert haben: Wenn sie Juden verhaftet haben, konnten sie hinterher in deren Wohnungen einziehen. ``Meine Familie braucht eine Wohnung, also mache ich da jetzt eine Weile mit, dann bekomme ich eine Wohnung'', so wurde häufig sehr einfach gedacht. Dabei handelte es sich um ein wirkliches Mörderkommando. Sie hatten ihre Zentrale in der Valdemāra-Straße und wenn die Deutschen angerufen haben, sind sie aktiv geworden. Sie hatten einen Fahrer, der hat sie dann dort hingefahren. Letztendlich hat das Arājs-Kommando die Ermordung der jüdischen Bevölkerung Lettlands in den Kleinstädten innerhalb von wenigen Monaten übernommen. Die Großstädte waren zunächst ausgenommen, in Riga, Liepāja und in Daugavpils hat man Ghettos eingerichtet. Aber in den Kleinstädten, in denen häufig nur vielleicht zwei jüdische Familien wohnten, haben die Deutschen Arājs beauftragt und dieser ist mit seinen Leuten losgezogen und hat die Leute einfach ermordet. Da kann man sich natürlich fragen: Warum haben die Nachbarn nicht geholfen? Um das zu verstehen, muss man wiederum die lettische Geschichte verstehen und die nationalsozialistische Besatzungszeit einordnen.\\
Im Jahr davor war das erste Jahr der sowjetischen Besatzung, die mit den großen Deportationen vom 14. Juli 1941 endete. Zwei Wochen später kamen die Deutschen. Das heißt, viele Letten waren sozusagen in Schockstarre, weil Familienmitglieder oder Freunde verhaftet und deportiert worden waren, das heißt sie wussten gar nicht, wo diese sind – und jetzt kommen die Nächsten. Viele Letten dachten erst einmal, dass es ihnen unter den Deutschen besser gehen würde, im Sinne: ``Mit den Deutschen haben wir schon eine Weile Erfahrung gemacht, früher und im Ersten Weltkrieg war es auch nicht so schlimm.'' Das haben übrigens auch viele Juden gedacht. Viele Juden sind nicht geflohen, obwohl sie die Möglichkeit gehabt hätten, mit der abrückenden sowjetischen Armee zu fliehen, weil sie nicht gedacht haben, dass es so schlimm kommen würde. Denn das, was in Lettland berichtet wurde, war gerade zur Zeit von Ulmanis sehr begrenzt. Das heißt, in Lettland wusste man nur sehr begrenzt Bescheid, was in Deutschland abging. Wie stark die Nürnberger Gesetze die Juden entrechteten, wusste man in Lettland nur bedingt. Die These, die manche lettische Wissenschaftler aufgestellt haben, dass den Letten erstens die sechs Jahre Ulmanis und dann das Jahr Sowjetzeit zivilgesellschaftliche Verhaltensweisen abgewöhnt haben, dass man sich lieber auf sich selbst und seine Familie und nicht auf das Ganze konzentrierte und man, wenn die Nachbarn abgeholt wurden, lieber nicht hinschaute, erscheint mir doch sehr glaubwürdig. Man muss auch noch dieses Verständnis mitbedenken, dass die Juden nicht wirklich zum Staat dazugehören, weil in der gesamten Ulmanis-Zeit ``Lettland den Letten'' gepredigt wurde. Auf der anderen Seite ist es wichtig, dass es auch Fälle von einigen österreichischen, auch deutschen Juden gab, die in den späten 30er Jahren nach dem Anschluss von Österreich in Lettland Zuflucht fanden. Vestermanis hat dazu geforscht. Sie wurden eigentlich ganz positiv aufgenommen oder hatten zumindest das Gefühl, dass sie hier existieren und überleben können. Solche Einzelfälle gab es auch. Aber der Einschnitt der sowjetischen Besatzung 1940/41 ist wirklich zentral, um zu verstehen, wie sich die Letten verhielten, als die Nazis einzogen. 

\textbf{Gab es die Bewertung, dass es unter der deutschen Besetzung besser gewesen sei, auch rückblickend von Letten, und gibt es das noch heute?} 

\textbf{Katja Wezel:} Kaum, ich habe immer den Eindruck, die Letten interessieren sich relativ wenig für die nationalsozialistische Besatzungszeit. Während der 50 Jahre Sowjetherrschaft wurden die Deutschen natürlich als die Faschisten und die Schlimmsten dargestellt und heutige Letten sind natürlich durch diese Schulbildung gegangen. Dass es heute noch ganz alte Leute gibt, die sagen, so schlimm war das gar nicht unter den Deutschen, halte ich zahlenmäßig nicht für sehr relevant. Die sowjetische Periode, in der das Thema auf eine ganz andere Weise unterrichtet wurde, war da prägender. 

\textbf{Immer am 16. März wird in Riga die SS geehrt. Das spricht doch dafür, dass es auch heute noch Letten gibt, die das irgendwie verherrlichen.} 

\textbf{Katja Wezel:} Das kann man so nicht sehen. Ich weiß, dass das von Deutschen gerne so wahrgenommen wird. Denn aus deutscher Sicht ist die SS natürlich gleichbedeutend mit den schlimmsten Naziverbrechern, weil sie die Konzentrationslager überwacht haben und so weiter. Man muss aber auch verstehen, dass die Letten die SS gar nicht so sehen. Es liegt schon am Namen, dass es aus lettischer Sicht nicht die SS ist, die da marschiert, sondern die sogenannte Lettische Legion. Diese wird quasi völlig losgelöst betrachtet von der SS. Viele der Letten, die das positiv sehen und sagen, denen müsse man doch gedenken, haben überhaupt gar keine Ahnung, was die SS war. Aus lettischer Sicht war die Lettische Legion eine verkappte Nationalarmee, die versucht hat, Lettland zu befreien oder zumindest dafür zu sorgen, dass Lettland nicht erneut von den Sowjets besetzt wird, denn das war das Schlimmste. Aus lettischer Sicht ist es tatsächlich so, dass es den Letten unter der nationalsozialistischen Besatzung besser ging, denn es wurden keine deportiert oder verhaftet, es sei denn man war jüdisch. Wenn man ethnisch lettisch war, dann ging es einem unter der nationalsozialistischen Besatzung auf jeden Fall besser und man war sicherer vor Deportationen, Verhaftungen, etc. als unter sowjetischer Besatzung. Als diese Lettische Legion 1943 gebildet wurde, haben die Letten das so interpretiert: ``Wir versuchen, unser Land vor den Bolschewisten zu schützen.'' Davon gibt es Poster. Es ging also gegen die Bolschewisten. Ob auf der Uniform das SS-Zeichen war, war für die Letten völlig irrelevant. Wichtig war für sie, dass auf dem Gewehr trotzdem das Zeichen Lettlands klebte. Es handelte sich um eine Fremdenlegion, die die Nazis ausgenutzt haben, indem sie sagten: ``Wir brauchen Kanonenfutter, wir brauchen Leute, die für uns kämpfen.'' Sie haben einen Deal mit den lettischen Militärs gemacht. Diese haben aber von vornherein gesagt, wir kämpfen nicht an der Westfront. Das war eine Kampftruppe, die auch nicht in Konzentrationslagern eingesetzt wurde, der kämpfende Teil der SS war letztendlich nichts anderes als eine Fremdenlegion, ein Teil der Wehrmacht. Nur da man keine Ausländer in die Wehrmacht aufgenommen hat, hat man diese ausländischen SS-Legionen gebildet. Sie haben also von vornherein gesagt, wir kämpfen nicht gegen die Amerikaner und nicht gegen Großbritannien, wir kämpfen nur an der Ostfront gegen die Sowjets und das haben sie dann auch gemacht. Die Lettische Legion hat mit dafür gesorgt, dass Kurland in sechs Schlachten nicht besiegt und besetzt wurde. Kurland war bis zum Schluss unter nationalsozialistischer Herrschaft. 
Nach der Kapitulation am 9. Mai hatten viele der Letten, die da gekämpft hatten, im Prinzip nur zwei Optionen. Sie konnten versuchen, noch übers Meer nach Schweden zu fliehen, das haben auch ein paar gemacht. Dann konnten sie versuchen, in die Wälder zu gehen und sich zu verstecken, was auch manche gemacht haben. Das war sozusagen der Ursprung von den sogenannten Waldbrüdern, die noch bis in die 50er Jahre als Partisanen gegen die Sowjets gekämpft haben. Oder sie sind in Gefangenschaft geraten und meistens im Gulag geendet, nur wenige davon haben überlebt. 
Das war die eine Lettische Legion, es gab zwei. Der andere Teil der Lettischen Legion hatte es ein bisschen besser. Die andere Lettische Legion kämpfte in Vorpommern, geriet letztendlich in britische Gefangenschaft und wurde auf den Nürnberger Prozessen freigesprochen, weil man gesagt hat, die kann man eigentlich nicht als SS-Truppen betrachten, sondern muss sie wirklich separat als Fremdenlegion behandeln. Sie waren nur partiell freiwillig beigetreten; junge Männer hatten Einweisungsbefehle erhalten, für die Legion zu kämpfen. Infolge dieses Urteils in den Nürnberger Prozessen konnten sie auswandern und haben eine Aufnahme beispielsweise in den USA und in anderen Staaten gefunden. 
Diejenigen, die am 16. März marschieren, sind natürlich nur noch ganz wenige Veteranen, da sind ja kaum noch welche übrig. Aber es gibt eben nationalistische Letten, die sagen, wir müssen an sie erinnern, weil sie für unser Land gestorben sind. Der Heldenstatus erklärt sich auch daraus, dass einige von ihnen im Gulag geendet sind oder als Waldbrüder und als Partisanen gekämpft haben. Das erklärt, weshalb man verschiedene Namen in einem Topf hat. 
Nun zur Beteiligung an den Verbrechen der Nationalsozialisten. Man kann die Anführer der Lettischen Legion nicht völlig davon freisprechen, weil ungefähr 15\% tatsächlich freiwillig beigetreten sind und unter diesen 15\% gab es auch welche, die tatsächlich vorher während des Holocaust dem Sicherheitsdienst geholfen haben. Aber das sind einzelne, es ist definitiv nicht die Mehrheit. Das ist wichtig zu unterscheiden, denn der Holocaust war in Lettland 1943, als die Legion gebildet wurde, abgeschlossen. Es gab noch ein paar Ghettos, aber der Großteil der Verbrechen war vorbei. Insofern muss man das trennen und muss die lettische Seite verstehen, für die die Lettische Legion nichts mit der SS zu tun hat, auch wenn es aus deutscher Sicht paradox klingt. 

\textbf{Die lettischen Kollaborateure im Holocaust waren also eine von der Lettischen Legion größtenteils verschiedene Gruppe?} 

\textbf{Katja Wezel:} Ja. Denn die wenigen, die wirklich kollaboriert hatten, wurden von den Nazis dann gerne als Hilfspolizeitruppen weiter eingesetzt. Diese Mördertrupps sind dann weiter nach Weißrussland gezogen und in der Ukraine eingesetzt worden. Unter den lettischen Legionären gab es bestimmt den einen oder anderen, der auch darunter war, aber man kann nicht sagen, dass es die Mehrheit war.

\textbf{Die Judenverfolgung in Lettland war unter den Nazis am größten. Aber trotzdem gab es Antisemitismus und Judenverfolgung auch unter Stalin. Sind der Nationalsozialismus und der Kommunismus in der UdSSR zwei gegensätzliche Ideologien, die dennoch denselben Feind in den Juden haben?}

\textbf{Katja Wezel:} Nein, weil der Ausgangspunkt ein anderer ist. Aber was stimmt ist, dass gerade von den ersten sowjetischen Deportationen und dem 14. Juli 1941 Juden prozentual stärker betroffen waren, als sie in der Gesamtbevölkerung repräsentiert waren. Das erklärt sich wiederum daraus, dass relativ viele Juden als Händler und Kaufleute in wichtigen Positionen ein bisschen bessergestellt waren. Sie galten dann aus dieser Perspektive als bürgerlich und die Kommunisten sind gegen alle Bürgerlichen vorgegangen. Sie wurden also nicht verhaftet und deportiert, weil sie Juden waren, sondern weil sie angeblich zu viel Geld hatten, bürgerlich waren, zu erfolgreich waren und so eingestuft wurden, dass sie sich gegen den Kommunismus wenden würden. 

\textbf{Wurde das dann auf alle Juden pauschalisiert, dass sie bürgerlich und wohlhabend seien oder wurden vorrangig die Juden verfolgt, die tatsächlich wohlhabender waren?} 

\textbf{Katja Wezel:} Gerade unter den lettischen Sozialdemokraten gab es auch sehr viele Juden und die waren natürlich ``die Guten''. Die haben die sowjetische Besatzung dann auch mitgetragen und unterstützt. Der Gegenpol zur Lettischen Legion während des Zweiten Weltkriegs waren die sogenannten lettischen Einheiten in der Roten Armee. Darunter waren zeitweise bis zu 17\% jüdische Letten\footnote{Laut Merritt (2019) waren es im Jahr 1941 17\%, am Kriegsende lag der Anteil bei 8.3\%. Dr. Katja Wezel nannte im Interview die Zahl von 30\%, dabei handelt es sich laut Merritt jedoch um die nachträgliche Übertreibung eines Generals. In jedem Fall war der Anteil im Vergleich zum jüdischen Bevölkerungsanteil sehr hoch.}, die entweder noch geflüchtet sind, kurz bevor die Nazis kamen und sich dann der Roten Armee angeschlossen haben, oder die eh schon in Russland waren und sich dann angeschlossen haben. Daher kann man das nicht pauschalisieren. 

\textbf{Wie aufgeklärt ist heute das Geschichtsbild, etwa in den Schulen oder auch in der lettischen Forschung, zum einen allgemein gegenüber den Besatzungszeiten und zum anderen speziell auf die Judenverfolgung bezogen?} 

\textbf{Katja Wezel:} Zum einen ist es wichtig zu sagen, dass Lettland diverse Abkommen unterzeichnet hat, z.B. in der Stockholmkonferenz 2000\footnote{Gemeint ist eine internationale Holocaust-Konferenz, die vom 26.-28. Januar 2000 (im Interview sagte Frau Dr. Wezel irrtümlicherweise 2001) auf Anregung des schwedischen Ministerpräsidenten in Stockholm stattfand. Aus der Konferenz ging eine Erklärung hervor, die die Beispiellosigkeit und anhaltende Bedeutsamkeit der Verbrechen der Shoah hervorhob und in der sich die beteiligten Staaten dazu verpflichteten, das kollektive Gedenken an die Shoah und die Bekämpfung von Rassismus und Antisemitismus hochzuhalten.}. Lettland hat bereits 1991 einen Gedenktag für die Ermordung der Juden eingerichtet. Er ist immer am 4. Juli, weil dieser Tag der Startschuss für die Entrechtung der Juden in Lettland war. Am 1. Juli sind die Nationalsozialisten in Riga einmarschiert und am 4. Juli haben die Synagogen gebrannt. Der Gedenktag ist deswegen ein vor allem lokal bedeutsamer Termin. Von Seiten der Politik gibt es eine Gedenkveranstaltung, aber es ist immer die Frage, inwiefern das den Großteil der Bevölkerung betrifft. Als ich die ersten Male in Lettland war und dann gefragt habe, wieso die lettischen Flaggen am 4. Juli mit Trauerflor sind, konnten vielleicht 80\% der Leute das nicht sagen. Natürlich gibt es gut Informierte, aber das Gros der Bevölkerung weiß es nicht. Da kann man natürlich schon sagen, dass die Schulbildung offensichtlich ein bisschen versagt hat.\\ 
Der Holocaust wird definitiv unterrichtet. Es gibt auch sehr gute Projekte wie das neue Haus von Jānis Lipke, der über 50 Juden gerettet hat. Das ist jetzt ein Museum, das gut angenommen wird. Es wirkt vielleicht erst einmal ein bisschen merkwürdig, dass man das Thema aus dieser Sicht betrachtet, indem man sich denjenigen anschaut, der Juden gerettet hat, denn es gab nur eine Handvoll Menschen, die das gemacht haben. Aber trotzdem ist die Bildungsarbeit, die gerade dieses Haus leistet, sehr wichtig, weil sehr viele Schulgruppen hingehen und dann auch gefragt wird: ``Was hättest du in der Situation gemacht?'' Man kann sich in diese Person Jānis Lipke sehr schön hineinfinden und verstehen: Die Situation war sehr schwierig und er hat trotzdem Juden gerettet. Wenn man dieses Programm durchlaufen hat, ist relativ klar, dass die meisten nicht so gehandelt haben, weil sie nicht diese Courage besaßen. Es gibt also aus meiner Sicht sehr gute Bildungsprogramme. Letztendlich hängt es aber sehr viel von der Eigeninitiative des einzelnen Lehrers ab, was er mit seiner Schulklasse macht. 
Was ich in Bezug auf die Wissenschaftler sagen kann: Es gab in den letzten Jahren zahlreiche Konferenzen, die das Thema Holocaust in Lettland bearbeitet haben, aber es ist letztendlich eine Handvoll von Wissenschaftlern, die sich wirklich damit beschäftigen. Das ist ein bisschen meine Kritik, wobei ich auch wenige Lösungsansätze habe, denn es ist sehr schwer, das in die Bevölkerung zu tragen, weil viele Letten der Meinung sind, ``erst einmal müssen wir darüber reden, was uns alles widerfahren ist, der Holocaust ist doch schon gut erforscht und die Juden sind doch überall präsent, aber von uns weiß keiner''. Es gibt auch die Ansicht, die ich teilweise verstehen kann, dass in Westeuropa die Verbrechen Stalins nicht sehr bekannt sind. 

\textbf{Wie ist die Situation der Juden, die heute in Lettland leben? Wie viele von ihnen haben die lettische Staatsbürgerschaft und sind gut integriert?} 

\textbf{Katja Wezel:} Das Problem ist, dass es nur eine Handvoll Holocaust-Überlebende gab, die tatsächlich nach Tallinn, Riga oder in andere Städte zurückgekommen sind. Die jetzige jüdische Gemeinde in Riga und auch in anderen Städten speist sich hauptsächlich aus Zugezogenen, die aus anderen Teilen der ehemaligen Sowjetunion gekommen sind, größtenteils aus Russland. Sie werden primär nicht als Juden wahrgenommen, sondern als Russen, weil sie Russisch sprechen, russisch assimiliert sind, auf russische Schulen gehen und so weiter. Sie haben dann die gleichen Probleme, die teilweise auch andere Russen haben, dadurch dass sie eben keine lettische Staatsbürgerschaft haben. Es hängt auch damit zusammen, dass die jüdische Bevölkerung häufig erst in den 80er Jahren zugewandert ist. Das heißt, sie leben häufig noch gar nicht so lang in Lettland. Diejenigen, die dort seit längerem leben, also vor allem die Nachkommen der historischen jüdischen Bevölkerung, haben natürlich auch alle Bürgerschafts- und Staatsbürgerschaftsrechte, weil das Staatsbürgerschaftsrecht Lettlands darauf fußt, dass man Bürger Lettlands in der Zwischenkriegszeit war. Alle, die vor 1940 Staatsbürger waren, haben 1990 automatisch die Staatsbürgerschaft gekriegt. Daher gilt die Staatsbürgerschaftsfrage nur für diejenigen, die als sowjetische Immigranten später dazugekommen sind. 

\textbf{Inwiefern gibt es heute in der Bevölkerung antisemitische Klischees?} 

\textbf{Katja Wezel:} Ich würde sagen, dass die meisten Letten sich mit dieser Frage nicht beschäftigen und auch überhaupt keine Meinung über Juden haben. Das große Problem, das es meiner Ansicht nach in Lettland gibt, ist die Schwierigkeit, Leute überhaupt für das Thema zu interessieren. Die jüdische Minderheit ist aus lettischer Sicht immer eine kleine Gruppe gewesen. Letztendlich ging es Jahrhunderte lang darum, sich mit den Deutschen und dann mit den Russen auseinanderzusetzen; die jüdische Minderheit ist irgendwie immer untergegangen. Die Feindbilder bauen sich also nicht primär gegen Juden auf. Das heißt nicht, dass es keinen Antisemitismus gibt; es gibt auf jeden Fall Nationalisten, die auch antisemitische Feindbilder integrieren, aber das ist nicht der Hauptfokus der Xenophobie in Lettland. 

\textbf{Inwiefern ist Antisemitismus in Lettland überhaupt vorhanden? Was müsste man ändern an Geschichtsaufarbeitung und allgemeinem Umgang mit dem Judentum in Lettland?} 

\textbf{Katja Wezel:} Ich glaube, dass man mit dem Antisemitismus-Begriff nicht sehr weit kommt, weil der nicht wirklich geeignet ist, um diese sehr spezifische Konstellation, die in Lettland vorherrscht, zu erklären und um auch zu erklären, warum es den Holocaust in Lettland in dieser spezifischen Form gegeben hat. Zweitens denke ich, dass man in Ansätzen schon genau das Richtige macht, indem man versucht, die Bevölkerung für Sachen zu interessieren, die tatsächlich in der eigenen Stadt passiert sind. Zum Beispiel gibt es jetzt an der ehemaligen großen Synagoge in Riga ein Denkmal. Schülergruppen gehen da wirklich hin, wie auch zu solchen Museen wie dem Jānis-Lipke-Haus. Das ist aus meiner Sicht das Wichtigste, dass man das lokalgeschichtlich einbettet, weil ich glaube, dass da auch in Deutschland Sachen wie die Stolpersteine am ergiebigsten sind. Das gibt es in Riga jetzt auch verstärkt, aber noch nicht sehr weit verbreitet. Das wäre auf jeden Fall schön, wenn es noch mehr davon gäbe und wenn die Leute dann auch wüssten, was das eigentlich ist und was da dahintersteckt. Es geht darum, diese Leerstelle aufzuzeigen, dass die Juden einst einen wichtigen Teil der Bevölkerung ausgemacht haben und jetzt in dieser Form nicht mehr da sind, weil die Juden, die jetzt in Lettland leben, größtenteils keine Nachfahren von den im Holocaust Verfolgten sind. 
\end{otherlanguage}

\section{Valters Nollendorfs}

\textit{Valters Nollendorfs was born in Riga in 1931. At the age of 13, he fled to Westphalia in Germany with his family. He emigrated to the United States of America in 1950, where he became professor of German language and literature at the University of Wisconsin-Madison. In 1988, he first returned to his home country and became a member of President of Latvia's Historians Commission and the Latvian Academy of Science. Until today, he is Chairman of the Board of the Museum of the Occupation of Latvia in Riga and offers tours through the museum. We met him in the museum \colorbox{yellow}{in September 2017}. Before we started the recording of the interview, we talked about the historical development of Latvia with him, namely what role occupations through foreign powers played in this history.}\par
\vspace*{2em}
\textbf{Valters Nollendorfs:} With a few exceptions, those countries did not accept the Soviet occupation by military Force, by threat and of course they didn't accept the Nazi occupation or the continued Soviet occupation, which means that there were military units in Latvia from 1939 to 1994, which could have overpowered any resistance. The government was dissolved, but in international recognition by law Latvia continued existing although it was occupied and de facto was not sovereign. And when Latvia redeclared independence in 1991, most of the countries wrote, we are willing to take up diplomatic relations again, so the first foreign minister who wrote to Latvian government after the coup in Moscow and after Latvia declared full independence in 1991 was the Foreign Minister of Iceland, and in his letter he wrote, we have never recognised the military occupation and the illegal annexation of Latvia by the Soviet Union, more or less, we are ready to take up relations again if you are ready. That has repercussions, that means that immigration at the time from other countries was illegal. That is one reason why people did not get immediate citizenship here, unless they were citizens or children of citizens, and second and even more important, that resistance was in international law legitimate with one exception: crimes committed during resistance were not. The Nuremberg Trials said very clearly, the SS is an illegal, a criminal organisation, but that doesn't mean each member is guilty until this member has committed crime against humanity or a war crime, that's individual. That means to some extent that the drafting of Latvian citizens in the Soviet army and in the Nazi army was illegal. When you saw that there were two Latvians serving in the two armies, Hitler saying, this is a volunteer operation, but the note says, this is a draft notice, you have to appear. So, it's the coercion of the local population into acts which otherwise could be criminal, or which are forbidden by international conventions. And that means that people who after the war went into the woods and carried out resistance until 1956, unless they committed crimes against humanity or war crimes, were legitimate fighters for Latvian independence, which they declared. Now, the Russian Foreign Ministry right now does not recognise this doctrine of continuity. They say, you wanted to join us in 1940, you saw how that was done, Crimea is a good example, and in 1991 you left. That's a different doctrine. They say, people who came here were legitimately here, but in international law there is no question, international law has recognised Latvia as a continuation state with some exceptions. 

\textbf{If it's ok for you, we would like to ask some general questions about you. How exactly did you get here, how did it happen that you just changed your profession and everything and just came here?}

\textbf{Valters Nollendorfs:} Well, I was born here. I was 13 years old when I left because my father had served in the police force. When the Soviets came, and it was very clear that we either had to take the ship to Germany, hoping that and well knowing that Germany has lost the war, rather than taking a free train ride to Siberia. so that was clear. I was in a refugee camp, displaced persons camp Greven, Westphalia until 1950, when my family went to the United States. And that's where I got my academic education. And then I became professor, and in 1996 I was bold enough to retire, they gave me emeritus status, which means that they recognised that I could use the lavatory, for example, and so, and since I saw what can be done, I was involved in academic reform. Then the museum had been founded, 1993, and I started volunteering, I didn't want to work here. They don't pay me. 
They say I’m too expensive to be paid. Anyway, I do this, this is not a state museum, keep in mind, this is a museum that is run by an organisation, and I'm the chairman of the board of this organisation. We own and we administer, basically, the museum, but the museum is day-to-day operated by the museum director. I work in the museum and then I'm under the director and when I'm not working in the museum, the director is under me.

\textbf{And why did you choose to educate people about the history instead of searching for more information and just do research?}

\textbf{Valters Nollendorfs:} Why I'm not in academic life anymore? I think freer. In academic life, there are a lot of things that you have to do, and I thought I have done enough for German literature as professor, as publisher of a professional journal, as a reformer, in effect, in America for German studies. So, I had to prove myself to get my salary increase and everything else. Now, I was 65, they said bye-bye, well, it was not a sad parting. I just felt, now I can be free and work for my country and in my country and be home. So, my children and my grandchildren are in America, but they understand that daddy and granddaddy have things to do in Latvia. And I can say, the Latvian anthem, it’s very simple, God save Latvia, God bless Latvia, God bless Latvia, God bless our fatherland, oh God, bless her. Where the Latvian maids are blooming, where the Latvian boys, where the Latvian sons are singing, let us be happy here in Latvia. And I'm happy there in Latvia. So, that much is for me. Yes, but I feel that in the situation that the world finds itself in these days, actually the museum is performing a very crucial task by telling the best we can and the most correctly we can what actually happened, and try not to gloss over, no fake news . Of course, Russia is accusing us of rewriting history, and indeed, we are rewriting history because the Soviets first started rewriting history and the Nazis rewrote history, so we are trying to put the history back where it should have been and trying to tell the truth, so that is very important, and this mission becomes certainly much more important because of what is going on in the internet, something that we didn’t know 50 years ago . The internet is full of everything. We’ll be trying, we’ll try to be honest, we’ll try to tell our story. For the most part, we’ll also want to convince the world that we have a story to tell and that is not simply something that can be, ``ah, yes, the same was in Estonia''. Yes, and no, the same in Lithuania, yes and no. So, the Lithuanians, when the Germans said, hey, the Estonians have a Legion, which of course was no Legion, it was not a fighting unit, Latvians have Legion, how about you, Lithuanians, Lithuanians said no. And the Lithuanians had more Forest Brethren after the War than the Latvians and Estonians because they had not decimated their young people in the War. 
You know the Polish story, horrible. If anybody that tells me about the Poles, I say Poles were the ones that got it from both sides first and who showed what Germans, Nazis, and the Soviets thought about Polish interests in being independent, being Poles in Poland after so many years of being spread out. I say, Poles are very proud people. And I think that Germans in the nineteenth century under the Kaiser finally became a unified country. Germans deserved to be a unified country, but they are not entitled to be a country that imposes its will on others. Latvia can't impose its will on others, so there's a difference between big countries and small countries, but we like this Germany much better. And I was in Wrocław a couple of years ago when they celebrated the famous bishops’ letter of coming rapprochement, coming to terms with the past, I think it is very significant. I think it is very significant that Germany accepted what certainly was not by any measure a just thing, making so many millions emigrate, which was caused by the Soviets pushing the borders of Belarus and Ukraine west, so you pushed the whole thing west, the question is whether that had to be achieved by force. In other words, so many injustices have been done there, but that if we dwell on those injustices these days in the world, we will never have the end of it, so the best thing is that we sometimes just simply say, forgive us. Let's not forget it, but let the past not stand in the way of the present and, especially, of the future. . And time and again, we still have this going back, the history to prehistoric times. So, again I say, democracy, the free exchange of opinions and ideas, but especially facts, trying to come near to truth, we will never get the truth, people who are religious would say, truth is up there, only God knows the truth, we can only approach it. I am a literary person. Goethe says in one poem that the poet’s words are only trying to knock at the gates of Heaven asking to be let in. They are trying to be true, but as in poetry, so in life, we can only strive for the absolute truth, and the striving effort is what is important. 
But now ask me, you certainly want to know more about Holocaust because otherwise, Dr Zinke will be angry at you. 
When the Jews were persecuted in Germany, Latvia allowed Jews to come into Latvia and gave them passports, so they could emigrate. This is a little-known story. There is also my story about how the Latvian exile dealt with the Holocaust or rather did not deal with the Holocaust. We are not dealing with the exile as such, but I've been involved a little bit in questions concerning Holocaust and questions of the guilt and the inability to come to terms with the Holocaust in Latvian society, here and in exile. In exile, there were 100,000, 150,000 Latvians, organisations and so on, activities. We kept our Latvian identity alive in the States, in Germany, in Great Britain, in Canada and all over the world. But there were certain things that were not discussed and one of those things that was not discussed was the Holocaust. In Latvia, it was well known that there had been a murder of the Jews, but again it was not a topic of discussion. In the Soviet Union, you probably know, Jews were a hot topic, so they were included in the total losses of civilian population, friedliche Einwohner, as it says in Soviet parlance. ``Oh yes, they were Jews'', but the Holocaust was not taken out of the framework. It was not dismissed. In the west, discussions, real discussions about the Holocaust came about late 1960s, 70s. There is an American book by historian Peter Novick, ``Holocaust in American Life'', where he points out that Israel was very ambivalent about the Jews that were killed, that allowed themselves to be killed, because they were not the fighters, not the warriors they needed. And only when Israel was threatened, when they felt that they are surrounded by enemy forces, did the Holocaust play a major role: Films, remember, Holocaust on television, Schindler's List and so on. The Holocaust as public discussion became great in the 70s and 80s while of course, there was no discussion and so the consciousness in the west is such, the Holocaust was the only crime against humanity, forgetting that another crime was committed here. And German historians were so politically correct that they were sometimes afraid to raise the question of Stalin's crimes because that would be interpreted as trying to relativise, we don’t have to relativise, we could call all things with the right name, but this is the context of the Holocaust discussion and that's why people here, having suffered under the Soviet regime, don't want to be bothered by it to some extent. To some extent it was not raised officially. I regret that. But it is possible to raise such questions only when the context becomes clearer, and here you will find in the public oftentimes a ``Yes, but''. Oh yes, there was Holocaust, but we also suffered. And if only we could get rid of the ``yes, but'' and say Holocaust, yes, Soviet atrocities yes, but not put the ``but'' in between them. That any suffering is equal to any other suffering. That is not only I, my suffering, that is important, it is important that other people suffered. And when we are talking to people who were in the Gulag, who suffered in war, who were tortured by the KGB and so on, oftentimes they say, I suffered so much, I cannot forgive. But then you meet people who say, yes, I suffered, but I understand that those who were in the camp with me suffered as well, and those were not only Latvians. Once, I took a poet whose picture is there in the exhibition, a good friend of mine, whom I met first in Münster at the Latvian Gymnasium when I was playing director there for a year in 1989, to my office, and we started talking, and one of the first thing he said was, ``I was kept for 8 years, and the one thing I learnt, I was not the only one who suffered''. And Western Europeans should learn a little bit more about Polish suffering, Latvian suffering, Estonian suffering, German suffering, all those who were suffering under these regimes, who died in Siberia, who were executed, who were killed. And yes, there is one thing that the Holocaust stands out for, that is, it is directed against one specific group with total annihilation. That was Hitler's aim, total annihilation. Latvians, if they had not liberated themselves with Western and other assistances, one more generation, we would still be Latvians here, we would still speak Latvian language, but we would be part of the Soviet Union, we would be as post-Soviet as Belarus, for example. It's the annihilation of national identity, which is the ultimate, not the physical annihilation, and that has been averted, but we have difficult time coming to terms with the difficult past we have, let alone coming to terms with the whole world. We are 50 years behind. And I have sometimes difficulties talking to my compatriots here, look, don't concentrate just on yourself, concentrate on the Syrians right now, concentrate on the Jews, concentrate on the Africans, and if you don’t concentrate, at least keep in mind that you're not the only ones who have the privilege of suffering. It could be a little bit more of this general understanding, but this coming out of yourself is very difficult after you have been into yourself. 

\textbf{I would like to ask just one more question about the museum. Of course, the mission of museum is really ambitious, but have you got a place here for the Jewish community?}

\textbf{Valters Nollendorfs:} The Jewish community has a building and the museum ``Jews in Latvia'' is located in that building. We have very good cooperation with the founding director of the museum, Marģers Vestermanis, who for example is spending his life concentrating on those people who rescued Jews. He himself escaped from the concentration camp in the last year of War, went to the woods, fought against the Germans with the partisans, and he has spent his lifetime on that, and we have a very good relationship. We have every year an Austrian coming here, a young Austrian in the alternate service, you have that in Germany as well, if you don't want to go into the obligatory, then you can go into the alternate service, and every year, a young Austrian comes here and works half-time in the Jewish museum and half-time in the Occupation Museum. So, the answer is yes, we have good relationships. The Jews do not meet, but the people who were deported, for example, people who publish a book, for example, when they hold a book presentation, it is open to those people who are involved and whose speeches you sometimes see in our exhibition or whose stories are told. We gather life stories, we gather testimonials, Zeitzeugen. We have close to 2,500 on video. You maybe saw those who were young people at the time when they were deported, they are in our collection, they are recollecting how it happened, how they were taken prisoner. And then you put one, two, three, four, five together, and you get a total picture, you get what these young people at that time experienced. And that ends up to the true story of the experience, not only to historical facts, but also of personal experience. And since these stories complement each other, we can come closer to the true psychology of how that, what that meant for these people. So, yes, the former deportees time and again come here, we talk to them. These are the sad stories, there are people whose minds have been affected. For example, once a month we had a lecture, and for seven years, one woman showed up always. And then there was discussion. Her hand always went up. She had only one question: ``When will I get my apartment back?'' When will I get my apartment back – of course, we couldn’t answer that question. See what the whole experience had done to this woman. She’s not the only one. And then you can admire those who come back and say, ``I understand I was not the only one who suffered.'' So, I sometimes call those people saints because they use this experience to become better people. And there are some people who couldn’t, who were broken. There are too many of those broken people who don’t want to talk about it. They keep it in themselves, they never tell their children. And then there are those who talk, and I think those who talk and who are willing to share are the ones… The Holocaust question, as I say, has been so long swept under the rug that it is difficult to talk about it. If there was somebody who was involved in the actual killings of the Holocaust -we don’t know, there may be some still, probably old people. It’s difficult to bring up such old items which are not alive, but we have to keep reminding this society. And we certainly avoid doing it, but we will continue with it. We are also the first ones who did the unimaginable thing, put these events side by side in 1941. Imagine, mass deportation, mass killings, and then comes the Holocaust, within one-month time. Psychologically, you can imagine shock after shock, and I would be telling you the untruth id I said that Latvians did not have certain anti-Semitic bias. But they were never violent, they have been no pogroms here. My father had two images of the Jew: Next to the police station at the street was a Jewish tailor, people who made suits, dresses and so on, there were little tailor shops there. He was a policeman; he was walking the beat. And he says, ``Oh, where is this Jewish tailor?'' And he says, ``Officer come on in, I can put cloth, I’ll make you a nice suit'', and he says, ``Such wonderful people''. And then there is ``the Jew'' [\textit{with emphasis}], big, now where does this image of the Jew come from, the Jewish image comes from the Protocols of the Elders of Zion. That’s an insidious document which has left huge impression because we are so apt in thinking in terms of conspiration - unfortunately, but that is a world-wide problem. Look at what is happening in America, whose citizen I am, too. I cannot imagine, Neo-Nazis walking in the city of Thomas Jefferson, shouting anti-Semitic slogans. Where are we coming to? So, it is a world-wide problem and therefore, I think it’s important in combating anti-Semitism of Latvians here, to keep that question alive. I keep reminding the people. But also keep reminding that the other side is not far away, that it’s right across the border. And so, people are worried and must be worried. There could be a resurrection of similar attitudes. So, let’s keep democracy alive, let’s keep asking questions, and let’s insist on getting honest answers.

\textbf{Was there any member of your family who was forced to serve at the Nazi or Soviet army?}

\textbf{Valters Nollendorfs:} No. Not my immediate family, and not the closes relatives that I have. My father during the German occupation served the so-called German self-administration. I am unhappy that at the time when my father was nearing his last days and I interviewed him I didn’t ask him such crucial questions, so I am not sure of how he was involved in the take-over. I know that during the Soviet occupation, he resigned, left the police force, and we lived in our summer home all through winter. I was in Riga with my aunt, I went to school, but he, his wife and my little brother were in this summer home, during the winter, and we were not arrested, we were not taken away in the deportation. So, when the Germans came, I know he worked for the so-called self-administration, and I know he inspected stores to make sure that the supplies are not stolen and so on, and that the book are in order. But obviously, in those terms he participated – as most people had to under military occupation. But I don’t know whether he was involved in any other capacity, so, I also have not really looked for documents, but right now, I feel it’s more important to do my job here and to simply say that I don’t know. But not in a military force – my father was too old, and I was too young. I was 13, if I would have been 16, I could have been called up as the Flagggehilfe, in the German air force as an anti-aircraft operator, supposedly. But those were 16, 17-year-olds. Towards the end of the War, even German young kids went into the army, which is also a crime, of course. Now, there are too many crimes which we should do the best to keep the world honest about. 

\textbf{I have one more question for you. Just imagine, if Holocaust never existed, how would the world look nowadays? Would it be better or not?}

\textbf{Valters Nollendorfs:} If the Holocaust had not occurred? I think it would be much better. For the Holocaust, in many ways, revealed that in the world – You have to imagine, when did we first find out that there was a world? In the 15th century, 16th century! And when did the world become more or less accessible, when was Australia discovered? In the late 1700s. Now, let’s say, 19th century, beginning of 20th century, we were already starting to understand the world, communicate around the world with telegraph and so on, but not the way we understand the world today. The Holocaust that the Nazis instigated broke down the basic moral structure by saying that there are people whom we can eliminate because they are not fit to be part of humanity. There were theories about it, as I heard, the Soviets did it in Holodomor and so on, but concentrated mass murder of one specific group of people to such an extent, it was something, in my opinion, unprecedented, and put in question the moral order. Whether your Christian or belonging to another religion anybody would say, probably, that killing is a crime, killing is a moral transgression – but here, a state, as state policy, was applying these methods. Gas chambers were not used in Latvia. It was only murder by bullet. As for the Polish Jews, Treblinka was opened in 1943, and the Germans didn’t want to kill the Jews on their land, so they sent them to Poland, they sent them here. And now of course, the Poles have to answer for the Nazi murders on their soil. That’s also a crime, in my opinion. But yes, I think that if the Holocaust had never happened, I hope we wouldn’t have come up with the idea that we could simply eliminate a group of people just because they are. But the communists did that, too, because they were trying to eliminate in the so-called Gulags. That’s the Holodomor in Ukraine. But they did it mostly by deportation, sending them away. So, the Holodomor is certainly a crime, but it is questionable to what extent it’s as central as the Holocaust. 
The first book about the Holocaust in Latvia came out in 1995. That’s by Andrew Ezergailis, who is a good friend, historian in America. And he had a great deal of difficulty; he had to take a lot of flak from certain Latvian circles for raising the question and raising the question of the involvement of our people in the Holocaust. I am a member of the historian’s commission - actually, the historian commission has done a lot of research on the Holocaust, unfortunately, most of it in Latvian. We know exactly what happened in small towns, for example, how it was carried out. There was a method to it. It was not a spontaneous action, it was certainly always an organised action, and behind the organisation was Stahlecker with his Einsatzgruppe which did not issue orders, there are no orders. Orders were issued by word of mouth or by telephone. That’s how we understood it. No written orders, because they said, ``Do it so they cannot trace it back to you, and make it look as if it was the local populations doing it''. But then, in November 1941 came the man whose picture is also there, Friedrich Jeckeln. Friedrich Jeckeln came from Ukraine. He was involved in Babi Yar Massacre, and he did the final killings in November/December with this SS men. But he, of course, involved again support groups, police and so on. If you take 25,000 people out of a ghetto, that’s a huge number of people. They killed them on two separate days. So, for taking them out of the ghetto, making them march kilometres, you have to have an organisation, but the killing was not by Jeckeln’s men. So, to me, the Holocaust is very significant, despite everything else that our people might say. But we, too – I say yes, we, too, but let’s keep in mind that there is a state involved in killing its own people, people who belong to other states, just because they are Jews. But you have to first eliminate the state structures, you have to make the field open and then, anything can happen. Shoot at will. But they didn’t shoot at will, even here, it was very much organised. And so it was, actually, in Poland. Our historians don’t quite believe the Jedwabne story. Cause there has been this sort of picture that the Jedwabne people were murderers and so on. Our people think that was also German-organised, German people certainly were present. And if you have military presence, well – don’t trust that what happened. The same in Lithuania, for example, infamous massacre in the Lietūkis Garage. You saw, in pictures, you see German militaries standing around. And according to the Hague Convention of 1907, according to the Hague Convention, the occupying power has to keep order. In other words, the Nazis at that time shouldn’t have stood around, but should have prevented. And so, I’m sorry for those people – Latvians, Lithuanians, Poles, Ukrainians, Belarusians, who participated then. Because they were in such conditions that they may be felt forced, but certainly, the Germans would have been, under international convention, obliged to prevent it, and they didn’t, of course, because that was their policy. And that’s why I think that this breakdown of this very basic order by state is very important and allowed for other things because, well, if you once have started killing, you might as well kill some other people. Kill the Tutsi for instance. That was a genocide, wasn’t it? Well, kill all these people with dark hair. Let’s live, let’s help people live. Angela Merkel is now suffering before allowing the refugees to come in, but it’s in German constitution. And maybe, the refugees came in and that was like breaking down the doors, of course. But lo and behold, if the ultimate right here in Germany comes to power. I like the European Union; we have certain peaceful ideals. It’s difficult to live up to the ideals, but we can strive to approach the ideal, and not leave it out of sight. But I don’t see any other way for the world to survive and for humanity to survive except this way. And the ultimate right answer is, ``Roll it all up'', and maybe we will someday. But I hope not. We still have children, we have grandchildren, we have great-grandchildren. Generations are coming, and I hope they will be better, and they will learn a little bit from the past. My regards to Mr Zinke – it’s a great thing that you’re doing, that you’re going after these stories, learning about it. Keep in mind that it’s for the present and future that you are doing it. And that it is not only the task that you’re doing that all of us should be doing.

\textbf{I have one last question. These future generations you have mentioned a moment ago, do you think that they will be full of grief or hatefulness to those crimes, or to the orders?}

\textbf{Valters Nollendorfs:} They should know that we should first of all intellectually understand that it happened and be convinced that it did happen. Now, there are people who deny it. That’s number one. You have to understand it, and then comes empathy. And then comes enlightenment. You should also be able to feel it, just to know it intellectually is not enough. You read that6 million Jews died in the Holocaust. This numbers game is no good. Peter Novick says, first of all, the number was 11 million victims of Nazis, of all kinds, that was the first. And when the Jewish question, the Holocaust came up, they decided what the figure is. So, they said, dividing sort of fifty-fifty is also is not possible. Let’s make it six million. And then, the rest were forgotten. Six million may be very close to truth. Counting that in Latvia, we had about 90,000, 70,000 of them were killed, and you start adding up the Polish Jews, the German Jews, the Hungarian Jews, the Jews in Russia, the Jews in Belarus, the Ukrainian Jews, and so on. And you’ll probably come up with millions, I don’t know whether anyone has done it the other way, from this general number that is being thrown around. In Latvia, we have been discussing the numbers, but 70,000 seems to be where it is all centred. And that’s a huge number, a huge number. You have to understand it. Then, you start to understand it in human terms, and then you can become a saint, as I say it. A poet, friend of mine said, you suffer, but you understand you’re not the only one who are suffering. And only when we develop that sense, then we can move forward. 

\section{Jānis Urbanovičs}

\textit{Jānis Urbanovičs\footnote{Check whether Urbanovics holds a doctoral degree} (*1959) is a Latvian politician and member of the Saeima (Latvian parliament) since 1994. In 1998, he became one of the founders of the Baltic Forum, one of whose aims it is to maintain good relations with Russia. The Forum has gradually become a dialogue platform for post-Soviet states. As part of his work for the Baltic Forum, Jānis Urbanovičs has published several books on culture, history and economics. From 2005 to 2010, he was chairman of the National Harmony Party. He was chairman of Harmony from 2010 to 2014.\\
Jānis Urbanovičs’ father fought Nazi Germany in the Red Army, while his uncle was in the Latvian Waffen-SS Legion.\\
We met him in Riga on September 25th, 2017, in the company of an English-Latvian translator.}\par
\vspace*{2em}
\textbf{How was it for your uncle who fought for the Legion, was he conscripted or did he go there as a volunteer?}\\
\\
\textbf{Translator:} Vai Jūsu onkulis gribēja brīvprātīgi iet leģionā vai viņu piespieda pierakstīties?\\
\textbf{Jānis Urbanovičs:} Es patiesībā zinu droši, ka tas bija piespiedu kārtā.\\
\textbf{Translator:} He says that he was pressured to go into this Legion.\\
\textbf{Jānis Urbanovičs:} Vispār par Latviju, par Latvijas ļaudīt, Otrā pasaules kara kontekstā, par brīvprātīgajiem runāt var ar ļoti, ļoti lielu izņēmumu.\\
\textbf{Translator}: And that people who went from their own will were very rare. They didn’t want to.\\
\\
\textbf{Jānis Urbanovičs:} Latvijā bija pavisam maz tikai Hitlera Vācijas vai Staļina Krievijas patrioti, kas gāja brīvprātīgi kauties. \\
\textbf{Translator:} There were very small part of people who went willingly on the side of the Nazis or the SSSR.\\ 
\textbf{Jānis Urbanovičs:} Jā, ja iet runa par SS Waffen leģionāriem, tad arī tur, vai Arāja slaveno brigādi, tad nu tas drīzāk ir izņēmums.\\
\textbf{Translator:} Well, there were exceptions, for example famous Arajs brigade.\\
\\
\textbf{Every 16th of March there is a celebration of the Waffen SS in Riga. What do you think about it?} \\
\\
\textbf{Jānis Urbanovičs:} Tā ir izrādīšanās, nevis leģionāru piemiņas diena.\\
\textbf{Translator:} He says that it’s just showing off, and not a day for remembering people.

\textbf{Is it also anti-Semitic?}  

\textbf{Jānis Urbanovičs:} Visi antisemīti neieredz arī krievus.\\ 
\textbf{Translator:} Everyone who is anti-Semitic is against to Russians, too.\\
\textbf{Jānis Urbanovičs:} Otrādi nav.\\ 
\textbf{Translator:} It doesn't work the other way around.  

\textbf{So, it’s a similar thing, anti-Semitism and anti-Russian racism?}  

\textbf{Translator:} What he wanted to say was those who are russophobes and anti-Semites are the same people. There are no people who thinks, I’m an anti-Semite, I’m not russophob, or vice versa. There is one small group that wants to provoke Russian people, those are the people who are going to this thing on March, you know, on the demonstration.\\ 
\textbf{Jānis Urbanovičs:} Latvijas neatkarības atjaunošanai bija par atdalīšanos no krieviem un visa krieviskā, un mēs turpinām šo atdalīšanos.\\
\textbf{Translator:} And that the declaration of independence here in Latvia was about the separation from Russia and Russian people and we are still continuing to do so.\\
\textbf{Jānis Urbanovičs:} Un no visām krieviskajām izpausmēm, mēs tā kā atdalāmies, līdz ar to arī pret saviem līdzpilsoņiem.\\  
\textbf{Translator:} And we are still separating from traditions and everything connected with Russia. We are still acting like that year in Latvia with our own people.\\

\textbf{Are there any other a signs or demonstrations of anti-Semitism today in Latvia?}  

\textbf{Translator:} Vai tagad ir kaut kādas antisemītisma izpausmes?\\ 
\textbf{Jānis Urbanovičs:} Šī ir raksturīgākā.\\  
\textbf{Translator:} This is the most typical one.\\ 
\textbf{Jānis Urbanovičs:} Ebreju tautības pārstāvji bija aktīvākie padomju varas ieviesēji šeit. Viņi runāja krieviski.\\  
\textbf{Translator:} Jewish people were the most active Russian supporters at that time. 

\textbf{That is the truth, or that is what many people believe?}  

\textbf{Translator:} Tā bija vai tika domāts, ka tā bija? \\
\textbf{Translator:} Both.\footnote{Check whether there is something missing here}\\
\textbf{Jānis Urbanovičs:} Bija, bija tādi personāži, bet bija arī tiražēts.\\
\textbf{Translator:} There were people in this communist party, but it was also talked about in a way not necessarily true. 

\textbf{In Germany and in Poland, we have a very strong tradition of the anti-Semitism. And in Latvia, according to my knowledge, there is not a strong traditional anti-Semitism, but there were some collaboration during the Nazi time, is it right?} 

\textbf{Jānis Urbanovičs:} Latvija ir mazāka par Poliju un Vāciju, mums viss ir mazāks\\  
\textbf{Translator:} Latvia is smaller than Germany and Poland, so everything is smaller. [\textit{laughs}]\\
\textbf{Jānis Urbanovičs:} Mēs esam arī nedaudz perifērijā. Krievijā ir aizgājusi arī ziņa par tirazēšanā par Latviju. Viņi mazāk zina, tāpēc ka par mums mazāk runā kā par Poliju.\\  
\textbf{Translator:} We are also in the periphery of European Union, that’s maybe the reason why you know a little less than about other countries, maybe people are not talking too much about Latvia.\\ 
\textbf{Jānis Urbanovičs:} Bet Latvijas iedzīvotāji ir bijuši gan vācu pusē, gan Hitlera, gan Staļina pusē.\\
\textbf{Translator:} Latvian citizens had collaborations with the Hitler German and the Stalin –Russian side. 

\textbf{Do the Jewish community or Israeli diplomats complain a lot about anti-Semitism in Latvia? Are there any complains?}  

\textbf{Jānis Urbanovičs:} No, there are none. 

\textbf{What do they say?}  

\textbf{Jānis Urbanovičs:} Viņi satraucās par šo pusi.\\
\textbf{Translator:} They are concerned about it.\\  
\textbf{Jānis Urbanovičs:} Jā, concerned is labākais vārds, lai to aprakstītu. Viņi ir bažīgi, bet viņi nav agresīvi. \\
\textbf{Translator:} They are not aggressive, they are just concerned.  

\textbf{Is there a lot of concern about the, say, anti-Semitism on the internet, or more about anti-Semitism in real life?} 

\textbf{Translator:} Vai antisemītisms ir vairāk internetā vai reālajā dzīvē?\\ 
\textbf{Jānis Urbanovičs:} Patiesībā es varu vairāk runāt par saviem vai tuvāko novērojumiem. Nav viņa daudz ne internetā, ne sadzīvē. Viņš ir vairāk politikā. \\
\textbf{Translator:} He can talk only about his experience and he says that it's not there in the internet, not there in the real life it’s just there in politics.\\ 
\textbf{Jānis Urbanovičs:} Piemēram, holokausta upuru īpašumu nodošana ebreju draudzēm radīja atkārtotu antisemītisma vilni.\\
\textbf{Translator:} The government just gave back to the Jewish society a few buildings, which belonged to the of Jewish community, and of course it raised some wave of anti-Semitism.  

\textbf{Are there any programs by the government to combat anti-Semitism or racism in general?}  

\textbf{Jānis Urbanovičs:} Nē, nav tādu programmu.\\
\textbf{Translator:} No, there is not programme like this.

\textbf{Are there any Jews in the parliament or in the Harmony party?}  

\textbf{Jānis Urbanovičs:} In parliament and in Harmony party. The Jewish is… 
\textbf{Translator:} …everywhere. 

\textbf{Also in other parties?}  

\textbf{Translator:} Yes. 

\textbf{Does it happen that politicians of the right wing in the parliament say anti-Semitic things?}  

\textbf{Jānis Urbanovičs:} No.  

\textbf{It doesn't happen?} 

\textbf{Jānis Urbanovičs:} Ir tikai daži izņēmumi.\\  
\textbf{Translator:} There are only a few exceptions.  

\textbf{It's a taboo?} 

\textbf{Jānis Urbanovičs:} It is a bad tradition. Not Latvian parliament tradition. 

\textbf{And outside the parliament?}  

\textbf{Jānis Urbanovičs:} Mēs esam ļoti kulturāls parlaments un uz āru mēs esam visi ļoti kulturāli - gan politiķi, gan sabiedrība. Uz āra.\\  
\textbf{Translator:} On the outside the parliament is very polite. [\textit{laughs}] 

\textbf{If you look at a museum, for example, or at the historiography: Is it more scientific or more folkloristic -only heroes, only victims?}  

\textbf{Jānis Urbanovičs:} Mums ir, mums ir visāda veida vēstures sacerējumi, tajā laikā arī antisemītiski ļoti. Ir.\\ 
\textbf{Translator:} We have different kinds of these books and they have different thoughts of them. \\
\textbf{Jānis Urbanovičs:} Ir, piemēram, pētījums par Cukuru, ļoti komplimentārs viņam. Lidotājs Cukurs, kurš bija Arāja..., nu, kas nodarbojās ar ebreju...em...antisemītismu kara laikā.\\  
\textbf{Translator:} There is a book about Herberts Cukurs, who was fighting against the Jewish people during the Nazi time. And there is a book very complimentary about him. 

\textbf{We heard that it's possible to Jewishness as a nationality put in the passport or ID document, like in the Soviet times. Is that still possible in Latvia?}  

\textbf{Jānis Urbanovičs:} Nav obligāti, bet var, ja grib.\\
\textbf{Translator:} If you want, you can put it.\\ 
\textbf{Jānis Urbanovičs:} Tas nav ar likumu norādīts, var neko nenorādīt pasē.\\ 
\textbf{Translator:} But the law doesn't say that you have to do it.  

\textbf{But do you have any advantages if you do it, for coming to Israel or something alike? Does it have any consequences if you do it?}  

\textbf{Translator:} It’s not the same as citizenship. You still have the Latvian citizenship, all rights are the same, you can just put your nationality, if you don't want to put it, just leave it out.  

\textbf{How is the relation between Latvia and Israel? And how do most of the Latvian population think about the state Israel?} 

\textbf{Jānis Urbanovičs:} Konfliktā ar arābu valstīm mums lielākā daļa sabiedrības ir Izraēlas pusē.\\
\textbf{Translator:} In the conflict in the Arabic states we are in the Israel side, supporting them.  

\textbf{And so is the population?} 

Jānis Urbanovičs: Yes. 

\textbf{How is the government supporting the Jewish community? Are there any measures taken?} 

\textbf{Translator:} Vai valdība kaut kā atbalsta ebreju kopienu?\\
\textbf{Jānis Urbanovičs:} Caur vēstniecību. Vēstniecībā tur viņi sadarbojas. Ir vēl savas programmas bijušajiem Latvijas ļaudīm, īpaši tiem, kuri ir saglabājuši pilsonību. Ar Izraēlu mums var būt dubultpilsonība.\\
\textbf{Translator:} There are some programmes in Latvian embassy in Israel and in the country. There are few programs and one of these says that you can have double citizenships with Israel. It’s not allowed with other countries, but you can be a citizen of Latvia and Israel.  

\textbf{We read that you have participated in the publication of a compendium of life stories of Latvians since 1934? Are there also life stories of Jewish people from Latvia in this book?} 

\textbf{Jānis Urbanovičs:} Es esmu rakstījis par pēckara laiku. Jā, protams, ka ir, skaidrs, ka tur ir, īpaši pirmajā grāmatā, nozīmīgi fakti par Izraēlas kopienu Latvijā.\\  
\textbf{Translator:} Of course, there are some, especially in the first book there are some very important facts about the Jewish/Israeli community.  

\textbf{Is it documents from archives that are published?}  

\textbf{Jānis Urbanovičs:} Mēs rakstījām 3 autori, lasot arhīva dokumentus, tā laika presi, izdevumus, kā arī hronikas, un apmainījāmies ar viedokļiem. Tā grāmata ir viedokļu apmaiņa.\\  
\textbf{Translator:} This book is an exchange of opinions and it was based on the facts from the archives, from publications, from chronicles. Everybody read them and then there was an exchange of opinions that is printed the book. 

\textbf{Are there any educational programmes about the Jewish religion, about Jews in Latvia, about the Holocaust, about anti-Semitism? Is it part of the school schedule, is it in the history lessons, in the religious education, maybe?}  

\textbf{Jānis Urbanovičs:} Tas viss ir skolā, bet, cik dziļi - nezinu, sen neesmu bijis skolā.\\  
\textbf{Translator:} He says that these topics are taught in school, but how deep, he doesn't know, he ended the school a long time ago. 

\textbf{Do you have anything to add what could be done to improve the situation of Jews in Latvia, especially in politics?}  

\textbf{Jānis Urbanovičs:} Politikā tam nav nozīme, tas tiek izmantots tikai, lai kādreiz iekostu otram.\\  
\textbf{Translator:} It doesn't matter in the politics, it’s just a thing with which you can bite someone.\\
\textbf{Jānis Urbanovičs:} Kad trūkst argumentu, tad sākās kaut kādas piesaukšanas. Piemēram, valodas prasmes vai izcelšanās, tas jau ir tad, kad politiķiem trūkst argumentu.\\ 
\textbf{Translator:} When the politicians don't have enough arguments, then they just pick on things like language skills or nationality. 

\textbf{How big is the anxiety because of Russia? Today, we had an interview with a scientist, and she was saying there is a big anxiety that Russia can invade in Latvia like in Ukraine? Is there a big fear?}  

\textbf{Translator:} Vai ir bailes no Krievijas?\\ 
\textbf{Jānis Urbanovičs:} [\textit{Nopūta}]. Blakus Latvijai ir Krievija.\\
\textbf{Translator:} Russia is next to Latvia.\\  
\textbf{Jānis Urbanovičs:} Latvijā ir daudz krievu.\\ 
\textbf{Translator:} There are many Russians in Latvia.  

\textbf{That's why I asked this question.} 

\textbf{Jānis Urbanovičs:} Ja Latvijā krievi un latvieši dzīvo draudzīgi, atrod savu kopdzīves formulu, kur abiem labi.\\  
\textbf{Translator:} And if the Latvians and Russians find a way to live together peacefully,…\\ 
\textbf{Jānis Urbanovičs:} Tad mums ir vienalga, kas notiek Kremlī.\\ 
\textbf{Translator:} …then it doesn't matter what happens in the Kremlin.\\  
\textbf{Jānis Urbanovičs:} Tad mums tik un tā būs labas attiecības ar Krieviju.\\
\textbf{Translator:} We will still have a good relationship with Russia. \\ 
\textbf{Jānis Urbanovičs:} Ja Latvijā latvieši un krievi dzīvos naidā un aizdomīgumā, tad arī mums vienalga, kas būs Kremlī, mums būs slikti, sliktas attiecības ar krieviem.\\ 
\textbf{Translator:} And if the Latvians and the Russians have bad relationships here, then it doesn't matter who holds power there in Russia, we will still have bad relationships.\\  
\textbf{Jānis Urbanovičs:} Iespējams, ja Krievija būtu tālāk, tad latvieši un krievi sadzīvotu ātrāk labāk.\\ 
\textbf{Translator:} And maybe if Russia would be somewhere further way, then the relationship would be easier.\\
\textbf{Jānis Urbanovičs:} Bet es domāju, ka ekonomiski krievi ir ļoti izdevīgi tuvu, ļoti izdevīgi tuvu.\\
\textbf{Translator:} But economically it's good that Russia is nearby.\\ 
\textbf{Jānis Urbanovičs:} Atliek vien tikai mums mājās sakārtot attiecības starp 2 etniskajām grupām - lielo latviešu un mazo krievu.\\  
\textbf{Translator:} The relationships between these two ethnic groups should be softened, and it would be better for both sides. 

\textbf{In history, the Baltic Germans were a ruling class in the Baltics. Is there still a anti-German resentment?} 

\textbf{Jānis Urbanovičs:} Līdz Pirmajam pasaules karam mēs neieredzējām vāciešus, pēc Otrā pasaules kara mēs neieredzējām krievus.\\ 
\textbf{Translator:} Until the Second World War, we hated the Germans, now we hate the Russians. [\textit{laughter}]  

\textbf{What about the Polish people?} [\textit{laughter}]  

\textbf{Jānis Urbanovičs:} Viņi arī ir Baltijas valsts. \\
\textbf{Translator:} They also are Baltics. [\textit{laughter}] 
\section{Inese Runce}

\textit{Inese Runce, Institute of Philosophy and Sociology of the University of Latvia in Riga.\\ 
The interview took place in Riga on September 25th, 2017.}\par
\vspace*{2em}
\textbf{Inese Runce:} You can imagine what kind of reaction there was in general towards this particular person\footnote{\colorbox{yellow}{Explain whom she is talking about}} and when the War started and the Soviets were preparing for leaving, when the Soviet army and the Soviet institutions left, they were about to organise an evacuation, and the communists were the first to leave.  He wanted to take this family, he wanted to leave, but then the wife's family came and said, ``you're responsible for everything they've done, for all of those atrocities, you're the one, if you want, you can leave, we haven't done anything wrong, we are staying''. When the Nazi army came and this cleansing started, this partly Latvian family and the Jewish family ended up in the nearby forest, for the same reason, the Latvian family because they were communists, and the Jewish family because they were Jewish.\\
The interregnum period actually was not about Jewish people, but there was revenge taking place, and in just a few moments, mainly in the Rūjiena area, if I'm correct, it was about the ones who were associated with the Soviet regime during the first year of the Soviet occupation. Why did those anti-Semitic Nazi slogans sometimes appear at the beginning of the Nazi occupation? Of course, we can relate this to Nazi ideology in general, according to which the Nazi were liberating Europe from the bolsheviks. But there also was another person, Semion Shustin, a Soviet Jew who was sent here to organise the  deportations, and for the Nazis - not only from Germany, but also local Nazis, the sympathizers with the Nazi regime -, Semion Shustin was a perfect example to be placed as the icon of the Soviet regime: He's a Jew, he's a communist, he's the one who's responsible for the Soviet deportations, and for the murder of the local population - a mystical Jew,in a way. You will be able to find something like this in Estonian and Lithuanian propaganda of the Nazi time as well, one particular person, a mystical figure, the Jew, the bolshevik. Here in Latvia, it had a particular face. There were very few Jewish people in Latvia in this interregnum period, and also only a few ones who were associated with this first year of the Soviet regime, it was not the Jewish community in general.
	
\textbf{Does anti-Semitism play a role in the public discussion at the moment?}

\textbf{Inese Runce:} I think it was a very important instrument when the discussion started in the 90s and in the beginning of the 2000, when the first Holocaust memorials started to be erected all across the Latvia, and it took years to finish the research, to deal with all of this not very pleasant topics in the history of modern Latvia, the issue of the collaboration with both the Nazis and the Soviets. I think we have worked on the issue of who was collaborating with the the Nazi regime, but we haven’t finished our discussions about the collaboration with the Soviet regime yet.
The recovery of the cultural memory is a very painful process for all the different groups. For example, it was a very typical and very strange leftover from the Nazi ideology - this was very popular and even the historians were arguing because there were no data to operate with - that there were so many Jewish people who were in the Communist Party, and that they were the ones who were organising the deportations. The slogans which were used by the Nazi regime,  during th occupation somehow appeared in the public in the 90s and the beginning of the 2000s, and then my colleague, Leo Dribins, who is right now the leading researcher here at the Institute of Philosophy and Sociology, and head of the Holocaust Survivors' community here in Latvia, did a lot of research, in particular about the Jewish people and the Communist Party in Latvia. And he found out that Jews were a minor group in the Latvian Communist Party in the late 30s and after the Second World War. The biggest group among the communists, both in relative and absolute numbers, were Latvians, the second group were Russians, and then the third the Jewish population.\\
Besides, professor Aivars Stranga researched the Archives of the Latvian Communist Party in the late 30s to find out about the percentage and the ethnic division in this context. The Latvians were the overwhelming majority, not the Jewish people, but this thinking is somehow left from the Nazi regime. The propaganda and ideology somehow stayed alive because after the Second World War, there was no possibility in general to deal with history in the Soviet era. There was no possibility to discuss what happened during the Second World War, and then there were those propaganda elements, appearing in the understanding of history and in the explaining of certain historical events, and it's interesting that in the 90s, in the beginning of the 2000, when those topics were researched, the results showed something totally different.

\textbf{The Museum of Occupation, is it rather scientific or rather folkloristic?}

\textbf{Inese Runce:} I must say that I like what the Museum of Occupation does. For example, I think that they did a very good exhibition on Rumbula a few years ago, together with the Jewish museum, where I was working for 5 years, and I must say that the exhibition that was made by the Museum of Occupation on Rumbula was very good, both from the historical perspective and from the artistic perspective. The groups that run the Occupation Museum are, I think, strong and good enough, they do as much as they can, but sometimes, groups are volunteering in the Museum of Occupation, and they tried to push their own  history forward, to emphasise certain topics. For example, they have a very good seminars and meetings between different groups on the occasion of May 8th, bringing people together, for example the Latvian soldiers who were in the Legion. Of course, nowadays they are not very many of them alive, its' just a few, but they bring people together to discuss the different experiences...

\textbf{But don't they equate the Nazi occupation with the Russian occupation? I think it's a problem to equate the Nazi occupation and the Russian occupation, and if you make a Museum of Occupation in general, I see the danger to equate it.} 
 
\textbf{Inese Runce:} The point is that the different groups have a different memories. For example, normal Latvians have nothing to do with the Nazi institutions or the Soviet institutions. They have a simple memory of what they experienced in daily life.  For example, the Soviet chaos, poor people, rudeness of Soviet soldiers - different things that they might remember. And then they remember the Nazi time. But they had nothing to do with the Nazi ideology, and they remember mainly the German soldiers, whom they had gotten in contact with if there was such a chance, and who were nicely  dressed up and organised. This is what they remember: they're totally different, as though they had nothing to do with this Nazi terror, they had nothing to do with official Nazi institutions, they remember, let's say, good, nice people whom they, because they knew German, were able to communicate with, and it was more or less the same German culture and milieu until the times of the First World War, so this was something familiar. And they didn't think of them as Nazis, but they thought of them as Germans. For example,  it's very typical to remember that the soldiers were giving chocolate  candies to the children around. And then, on the opposite side, there is the Russian army who, let's say, is robbing, who are not well-behaving, whom you try to escape, and first of all, you don't know the Russian language. \\
So, this might be the average Latvian's memory about the Nazi time. This is of course not valid for the whole nation, because the, the national history is composed of different memories, it’s normal to have different points of view, and therefore sometimes, you can also see such a presentation about the times of the Nazi occupation: As a better era than the Russian era.
 
\textbf{What about the other groups, for example the Russian people or the Polish people living in Latvia at that time, how did they perceive the Nazi occupation?}

\textbf{Inese Runce:} It depends, there is not one unified Russian community, neither one unified Polish community. The Polish community is composed of two groups, for example the Polish people who are local and the Polish people who came after the Second World War from different parts of the Soviet Union. The same applies for Russians, the Russians are a even more diverse group. There is no homogeneous Russian community, and there are Russian Old Believers, for example, who were drafted into the Latvian Legion, or the Putin's refugees, as we call them, the ones who are moving here from Russia in modern days. And then you have a Russian-speaking population that is composed of different groups, Ukrainians, Belarusians, Russians, the ``Russificated'', people from the different parts of the Soviet Union, and their cultural memory might be totally different. Then, for example, if you speak about Russian Old Believers, who are indigenous group for a long period of time, their memory is very close to the Latvian. They experienced the same events, except for the race theory which made the Slavic groups very attentive because the Latvians and Estonians, for example, were placed higher in this race pyramid. The Slavic groups were on a lower scale, yeah, so even for the Polish people, so this was a big issue, the Slavic background in general, that made them very attentive. The Latvians were not as much attentive to this.

\textbf{Do you maybe know what the people of Latvia expected from the Nazis before the occupation?}

\textbf{Inese Runce:} They didn't expect anything. At the end of the 30s, there major issue was how to survive. On one side, you have the Soviet regime, on the opposite side, you have the Nazi regime. On one side you have this crazy man called Stalin, on the opposite side you have this crazy man called Hitler - so, how to survive, how to keep independence?  This was the reality for the political elite, for the intellectual elite. 

\textbf{And I think there were many Latvians who were fighting either against Stalin and against Hitler.} 

\textbf{Inese Runce} And they weren't fighting voluntarily, like my family, for example: The grandfather from the paternal side, he was taken to serve in the Nazi army, and the grandfather from the maternal side went to serve in the Soviet army. I know many families in Latvia whose oldest son was taken to serve in the Nazi army and the youngest son was taken to serve in the Soviet army, and they were fighting around Pulkowa to each other, shooting at each other around Pulkowa. The Soviet army hierarchy sent the Latvian troops to Volkova, and the Nazi army sent Latvian Legion to Pulkowa.

\textbf{And did the Jewish community in Latvia know about the crimes that were committed from '39 to '41 by the German army,about the beginning of the Holocaust?}

\textbf{Inese Runce:} The authoritarian regime of Kārlis Ulmanis, which was established in May of 1934, very strongly controlled the newspapers and radio, all of the media radio, in order not to publish bad news whenever possible.

\textbf{It was like in Poland: People could have known it, but they didn’t believe it.}

\textbf{Inese Runce:} That’s also true. Latvia was actually the last country in Europe which accepted Jewish refugees from Austria and Germany. And when the people came and they started to share, the local Jewish people said: ``I don’t believe you'', ``you are simply exaggerating''.
There were, however, quite a few articles published about what’s going on, but censorship measures usually controlled that nothing wrong was said about Hitler, in order not to provoke Nazi Germany, not to give a reason for occupation, and on the other hand, that nobody spoke of what happened under the Soviet regime, because it could also give Stalin possibility to occupy us.\\
Another thing is that for the authoritarian regime of Kārlis Ulmanis, it was a very important to create an image of paradise - everything is good, everything is quiet, everything is perfect, we will make it, everything will be okay, the political crisis in the Europe will be over and there will be a bright future. This image started to disappear in 1939, which was also the beginning of the significant economical crisis, when the German population, the Baltic Germans started to be resettled. There is this local saying here: If the Germans are leaving, it means Russians are coming.
And therefore you also don't pay so much attention to other things, You're so concentrated on how to survive, on what to do, on what's going to be, and what's happening in Poland, it just loses its sense, and there were not very many news published in this regard. %check whether this is the correct place
\section{Kalevs Krelins}

\textit{Kalevs Krelins is a rabbi in the Peitav-Shul synagogue in Riga and at the same time chief rabbi in Lithuania and Vilna.\\
He was born in Moscow and studied in Jerusalem. He served as a rabbi in a school in Copenhagen, Denmark, in Heidelberg, Germany, and for a Young Israel community in the United States. In 2012, he was asked to serve as the rabbi of the community in Riga. Together with Shimshon Daniel Isaacson, he was appointed as the chief rabbi of Lithuania in 2016. Besides, he also works as the} mashgiach \textit{(Hebrew word for supervisor) of the European Council of Kashrut (EEK) in the Baltic states.\\
\sloppy
The interview took place in the Peitav-Shul synagogue on September 28th, 2017. Throughout the conversation, Mr Krelins exclusively expressed his own views, not those of the Jewish community in Riga.} \par 
\vspace*{2em}
\textbf{How did anti-Semitism develop in Latvia after World War II?}  

\textbf{Kalevs Krelins:} It’s hard for me to describe because I didn’t spend the years after the war here. As far as I know, the Soviet Union had a pretty powerful control over anti-Semitism. They were able to bring it higher or lower according to the international situation. As a child, I was raised in Moscow. For us, Riga was always the top of Jewish development compared to Moscow in terms of civility, and also in terms of religious freedom. There was no religious freedom in the Soviet Union, but in terms of what they could do here compared to what they could do in Moscow and other cities, it was an example of a very well-functioning community. They had kosher meals, they baked matzah and sent it to the whole Soviet Union. They always performed circumcisions. I do circumcisions myself. I travel to other countries and people say ``We had a circumcision thirty or forty years ago, and the guy came from Riga.'' Riga was always the place. For sure, there was the KGB, people were tortured. But if we compare it with Moscow, it was not so bad, and many very knowledgeable people were raised here after the war in Soviet times. That’s what I know from stories. \\
My own experience here is limited to the last five years. I don’t see major issues with anti-Semitism here.  We all know that we cannot judge any place according to strange local people. I believe that the government here is very nationalistic, but in my opinion, their nationalism is related to the conflict between Russians and Latvians, so the Jews are not in the picture. We used to say that we can’t relax, because they’ll finish the conflict with the Russians and then they come after us, but we thank G-d that we don’t have any problem. There is a joke: ``They finish with them, they’ll start with us.''  

\textbf{Do you think there is some reality behind this joke?}  

\textbf{Kalevs Krelins:} Historically, yes. History shows that Jews were always seen as an enemy. When a new enemy came, the Jews went to the second line. As soon as there is peace with that enemy, they go back to see the Jews as their enemies. I hope I exaggerate. But we have to be aware. 

\textbf{In Germany, synagogues are protected by the police. Is it the same in Riga?} 

\textbf{Kalevs Krelins:} Here we don’t talk about protection. I know how the synagogues in Germany are protected. When I was rabbi in Heidelberg, I once parked at my parking place next to the synagogue. It was September 11th, the year after the attack. The police didn’t recognize me, so they jumped on me and checked me. They take it very seriously. Here people go around the whole country, they feel that nothing happens. It’s not only about synagogues; in the whole country, they're relaxed. They’re only afraid that Russia will invade. In Germany and France, synagogues are much more protected. I personally never had any bad feeling in Germany, but I was in the most intelligent place, Heidelberg, which is a university city. Here in Riga, I don’t feel anti-Semitism.  

\textbf{So you don’t have any problems walking around with a kippah?} 

\textbf{Kalevs Krelins:} First of all, I don't walk around with a kippah. I wear a casquette, for a simple reason. I came here from New York, where it’s easy to walk around with my kippah. I know that there are many people who hate Jews, but if someone who hates Jews and wants to express himself to a Jew meets me with my kippah in New York, he can go further down the street and meet someone else with a kippah. Here in Riga, the guy who wants to say or do something has to wait a couple of years to meet the next Jew. That’s why they hurry up to express themselves. Sometimes on Saturday, I go with a hat, sometimes I go with my kippa, my kids too, but we've never had any issues. Maybe once or twice somebody said something. 

\textbf{What are the differences regarding anti-Semitism be-tween Germany and Latvia?}  

\textbf{Kalevs Krelins:} My personal experience is that Germany recognised and experienced the pain of what it has done, but Latvia never did. Latvia and the Latvian people were extremely cruel with the Jews. They say that the German government gave them free rein and the Latvians did the whole job. The Latvians, Lithuanians and Ukrainians were the most cruel, more cruel than the SS themselves. The people who remember say that the German government was certainly terrible, but as people, Latvians, Lithuanians and Ukrainians were much worse than the Germans.  

\textbf{Is the Holocaust remembered and spoken about in Latvia?} 

\textbf{Kalevs Krelins:} No, they say that there was the occupation. They always say that they were under occupation: ``We’re under German occupation, we’re under Russian occupation, we’re under Swedish occupation 400 years ago.'' They are always under occupation, it's not their fault. But they did their best to destroy all the Jews. The terrible thing is that the number one or number two in the destructions was Herberts Cukurs, who is a national hero. He's not officially recognized, but officially praised for his achievements: He built planes, he flew to the Arctic. There was a musical about his greatness here. We see that he has achievements, but these achievements can’t cover up what he did. He was killed by the Mossad during the sixties in Uruguay. He fled from America, from Russia, he fled from everyone. Many war criminals fled to Argentina. The Mossad killed them. They now say that it was a mistake that they didn't bring him to court. 

\textbf{Do you mean that Latvians are not remembering and recognizing their own guilt?} 

\textbf{Kalevs Krelins:} No, they are not recognizing their own guilt. They do say that unfortunately there were some Latvian people who collaborated. They have Waffen-SS marching here every 16th of March. They say it's not exactly against Jews, but against Russians. When they say it's against Russians, everyone is forgiven. 

\textbf{Is the Jewish community offended by these marches?}   

\textbf{Kalevs Krelins:} Definitely. The Jewish community is very offended by these celebrations. We know that some of these people from the Waffen SS participated in mass murder.   

\textbf{Do you think that the government is distancing itself enough from this march?}  

\textbf{Kalevs Krelins:} It’s very hard to follow what the government is doing here. I don’t want to speculate, but certain members of the parliament and of the government are participating in or praising the celebrations. And there are very few demonstrations against it, just expressions. Usually when they catch somebody who expresses himself against it, they say it is an agent of the Kremlin.  For example, when somebody screams ``You are Nazis'', ``You are killers'', they take him aside for disturbing the public order and say it’s inspired by Putin.  

\textbf{Does the Jewish community organise any demonstrations against the celebrations?} 

\textbf{Kalevs Krelins:} No, I don’t think so.  

\textbf{Would they be seen as agents of the Kremlin if they organised demonstrations?} 

\textbf{Kalevs Krelins:} Yes. Latvian Jews were almost all killed during the war, ninety percent. After the war, Latvia became a very developed part of the Soviet Union. It was the Silicon Valley of the Soviet Union. I remember that all sorts of computer parts and cars were made here. Latvia was among the top technologically. So they sent many professionals from Russia here, among them many Jews. A big part of the Jewish population of Latvia today are descendants of Russian Jews that came as professionals to develop the economy of Latvia. So, most Latvian Jews are actually not Latvian. For us, if a person is a Jew, it doesn't matter where they come from. But for them, Latvian Jews and Russian Jews are different. Latvian Jews were exterminated and now, these Russian Jews are associated with Russia.  

\textbf{Do you think that those who are anti-Russian are also anti-Semites and vice versa?} 

\textbf{Kalevs Krelins:} A bit. I have to admit that many Jews who were oppressed in Russia went to the communist movement after the Revolution. Jews were one of the most oppresed nations, so they went against the system and joined the reds, the communist party. 

\textbf{Did they consider themselves as Jews?} 

\textbf{Kalevs Krelins:} They considered themselves as Jews. They were secular, they considered themselves as people who stand against different types of oppression. And oppression of the Jews was one of the oppressions in Russia before the Revolution.   

\textbf{How big is the Jewish community here?} 

\textbf{Kalevs Krelins:} We don’t have official memberships like in Germany, for example. In Germany it works with \textit{Kirchensteuer}, the register. Here we don’t have such numbers. Very roughly, we talk about 8,000 Jews in Latvia. I know that the Jewish community in Riga has 100 members, but that doesn’t mean that 100 people are coming to the services. On a daily basis, about 15 people come. On Saturday, we talk about 40 people, more or less, and on high events it’s packed. It depends on the weather, the time of the year, but it’s a lot, 150-200, sometimes more.  

\textbf{How have the numbers developed since the independence of Latvia?}  

\textbf{Kalevs Krelins:} Many people moved to Israel after the Soviet Union broke apart. The numbers have certainly gone down. But the whole population of Latvia is going down.  

\textbf{Was it illegal to emigrate before independence?} 

\textbf{Kalevs Krelins:} No, it was legal, but the government did not appreciate it. People applied and they got a permission, but only a few. In the begining of the 70s, many people got the permission. After the war in Afghanistan in the early 80s, until 1987, it was very hard to get a permission. After 1987, it got easier again. 

\textbf{Is there a discussion in the Jewish community about moving to Israel?} 

\textbf{Kalevs Krelins:} I think it's already settled. Some people come to me and ask why I don't move to Israel. I think if people are here, they have a reason to be here. Young people go to Israel to learn, to go to school or college. There are many people who lived in Israel, experienced Israel, and then came back. They decide to settle here. More or less, everyone came to his or her own conclusion with this question until now. The same is happening in Germany.  

\textbf{Do you see any future for the Jewish community here in Riga?} 

\textbf{Kalevs Krelins:} When someone from the Jewish community asks such a question, I have to say it’s up to you. It’s like a free market. If there is a demand, we will supply the demand. Sometimes there is a rabbi who wants to have a synagogue, so he creates himself the crowd. I think this concept is wrong. If the people need it, if they show that they want to know, they want to learn, I want to teach, but the first step has to be done by the people. Otherwise it’s counterproductive.  
\section{Małgorzata Zajda}

\textit{Małgorzata Zajda (*1948) was born in Kraków, as her parents and grandparents. She is of Polish-Jewish descent. She studied Russian.\\
From the beginning of Jewish Community Centre in Kraków (2008), she cooperates with it as a coordinator of the Senior Club. She organises plenty of activities and events for seniors such as Shabbat dinners, choir meetings, yoga, mind training sessions and many other Jewish celebrations and everyday activities for the elderly.\\
The interview took place in the Jewish Community Centre in Kraków on January 26th, 2018. We talked about a variety of subjects with Mrs. Zajda; for this publication, the transcript of the interview was shortened to the sections directly related to anti-Semitism and Jewish life. }\par 
\vspace*{2em}
\textbf{Małgorzata Zajda:} My parents were Jewish descent. They introduced me into Jewish life as in the 50s, it was almost impossible to talk about Jewish culture. But since I was a child, I was aware that I am Jewish and I was really proud of it, because being a Jew after all those incidents during the War was something amazing. I felt like being better than other people.  

\textbf{Have you ever been confronted with anti-Semitism?} 

\textbf{Małgorzata Zajda:} I came from a really rich family; my father was a lawyer. All people think that Jewish people in Poland after the War were poor- but it is not true when it comes to me. My father told me a lot about Jews, he taught me a lot. He told me about my grandparents, about life before the War, but he never was willing to talk about it. My mother was trying to raise me in catholic life, she didn't know about her origin - probably her mum hid it from her. My father was organising meetings with influential people, every week they were meeting at our house and were resting, talking about art etc. - all of them were Jews. When I looked at them, I thought that every Jew is elegant and clean. When my father died, I got to know that Jews can be all kinds of people, not only intelligent ones, I saw that Jews are normal people and that they take up various jobs. 

\textbf{Can you tell us about the experiences of your father during the War?} 

\textbf{Małgorzata Zajda:} My father was in a camp in Janów. They were working by the train tracks. My father was athletic, he never had problems with sport and really liked it, so he could cope with this hard work. Somehow, he managed to escape from the camp and was wandering. He was in Czech Republic; he was a witness of Lidice's pacification in Czech. He was working in a bakery, selling bread. Later, he came to Kraków and hid. Unfortunately, he was caught, and was kept in Montelupich, and there he met a man who helped him to fake documents to prove he was of Aryan origin- that man was one of the Righteous Among the Nations of the World. With those Aryan documents he could lead a normal life in Poland. \\
My father told me a story when he was coming back from the prison about how Jews were treated by other people, it was about 45'/46'. When the Jews were coming back home, people were throwing rotten tomatoes at them, they were very disrespectful to them, and it stayed in his psyche. He was scared to be among the people, he was scared to leave his home, he was happy that he changed his surname and could lead a normal life thanks to it. He said that being a Jew isn't as good as some people think. In my childhood, I never encountered any anti-Semitic acts, but I remember when we had a polish housekeeper - her name was Marysia -, once I came back from holidays and Marysia came to me and told me that people don't accept the fact that she is working for Jews. This was my first encounter with an anti-Semitic act- it hurt me so much, because our housekeeper was leading a really good life with our family. The second encounter was when a friend of mine moved to the US and worked as a nanny, she told me that everything would be okay for her if those people who she was working for weren't a Jews. They told her to eat kosher, and she didn't like that. During my young life, I experienced it many times that people painted a Star of David at our house door- but I never took it seriously. They were also painting it on my car, I always treated it as a stupidity of people. Nowadays we encounter less anti-Semitic acts, in Kraków, for example, anti-Semitism almost doesn't exist. In my opinion, nobody has got to love Jews, and everybody has the right to hate Jews and I’m not going to change it, it is their choice. I think we should respect each other even if we don't like somebody. I am not a person who has bad memories of Catholic people if it comes to anti-Semitism. Some of my friends, when they learned that I am Jewish, broke off contact with me. I think we should respect each other even if we don't like it, and I always repeat if when I talk about anti-Semitism. Nowadays, we experience situations in which Jews who were faced with anti-Semitism during the War think that people are acting anti-Semitic to them because they know that they are Jews, for example they say, “when somebody does not hold the doors for me, they don't want to because I am a Jew”. My job is to work it through, to tell that it is just a random. They think like that because it stayed in their psyche after the war.  

\textbf{Are you able to say how many Jewish people live here in Kraków?} 

\textbf{Małgorzata Zajda:} There are about 100 people. There are much more, but they don't admit to be Jewish.\\ 
Before, people were more against Jews, now they respect us and don't show it, even if they don't like us. If somebody will try to find anti-Semitism, they will find it everywhere, in every single situation. 

\textbf{Do you think that there is a way to stop anti-Semitism in the future?} 

\textbf{Małgorzata Zajda:} I think that we will never stop it; in my opinion, it will always be there. I am happy that I am from the generation of people who weren’t faced with War so far. There might always be something that can change everything. Before, we never had security guards in the Jewish Community Centre, but now we have some special safety stuff. I can't say that anti-Semitism will not exist in the future; I think "anti-Semitism" is a wide word. If somebody wants to have anti-Semitism, they will get it.\\ 
A few years ago, I had a situation where a Polish priest came to my house and showed me Jewish images and asked me if I am a Jew and left my house, he didn't respect my religion.  

\textbf{Do you think there is a real danger for Jewish people in Poland?}  

\textbf{Małgorzata Zajda:} Before, I would say, not, but now I think that there is a little danger. There is no day without talking about Jewish issues.  

\textbf{People only see how bad are you and don't see how much good you do for people. Maybe instead of talking about anti-Semitism, we should talk about Jewish culture and learn people about it, so that it might change?} 

\textbf{Małgorzata Zajda:} Yes, exactly, it would help. We only talk about anti-Semitism and it might be a reason of this issue.  
There are lots of stereotypes, not only about Jews -"you are mean like a Jew", "you drink as much alcohol as Russian" etc.  

\textbf{People say that Poland is a country full of anti-Semitism, what is in your opinion?} 

\textbf{Małgorzata Zajda:} It was an anti-Semitic country, but after the War, it was caused by the fact that Poles thought Jews will never come back and took their houses, their shops and their jobs after them. But when the War was over, they came back and wanted their properties back, and in my opinion, it is mainly caused by that. Anti-Semitism always was and always will be, now it does exist, but it is hidden, and we can see it just like that.  
I think that the Jewish Community Centres is a place which shows the Jewish culture and tradition and thanks to that, we reduce anti-Semitism a little.  

\textbf{Do you think that the problem of the Jews as an affected group is exaggerated?}

\textbf{Małgorzata Zajda:} No, I think it is not. Jews always were not respected, mainly because of money and success. 
Jewish people are scared of anti-Semitic acts. An example of it is when a Jewish group comes to Poland to visit the camps, they have special securities, and in my opinion, it’s because of that they are more scared by seeing it. It makes them think that they are in danger and it might make them feel scared. People in Israel think that Poland is an anti-Semitic country, and they are scared of it because of history.  

 

 
\section{Dr Steffen Huber}

\textit{Dr Steffen Huber is a research assistant at the Department of Polish Philosophy at the Institute for Philosophy of Jagiellonian University in Krakow since 2005. He is also a member of the Policy Council of the Józef Tischner Institute, an institute that, according to its website, was founded by pupils and friends of the philosopher and priest Józef Tischner (1931-2000) for the purpose of preserving and spreading knowledge about his works and of continuing research and development about the most important aspects of his philosophy.\\ 
Dr Huber’s research interests are Polish philosophy of the Renaissance, social philosophy and translations of philosophical texts. We talked to Dr Huber and two of his students, Pawel Karpinski and Krysztof Turek, on the 26th of January in the Institute of Philosophy. Before Mr Karpinski entered, the interview was conducted in German; the respective part of the transcript has been translated into English for this publication.}\par
\vspace*{2em}
\textbf{In our project we examine the continuities and fractures of anti-Semitism since 1945. From the interviews we lead with scientists, we hope to get an overview over the country's history in its connection with the anti-Semitism found in this country. In your case, we would be particularly interested in a comparison of Germany and Poland, as you have lived in both countries. So, on the one hand, what you personally have experienced or heard second-hand of anti-Semitic incidents. And then, how anti-Semitism is rooted in the history of ideas in the two countries.} 

\textbf{Dr Steffen Huber:} That's a big task. I have not dealt scientifically with the subject of anti-Semitism, for me it is a marginal phenomenon which one has to deal with every once in a while. It is also a very contested area, especially the question of what anti-Semitism is, we’ve also had this here at the university. Even today, it is often the case that one encounters anti-Semitism and then has to hear people say that tit would be none. Even if the classical motives occur, of conspiracy and 'cultural destruction' and a low level of development of the Jews. \\
If I compare Germany and Poland, I would say the major difference is that anti-Semitism in Germany basically has no basis of experience for most people in the twentieth century. People had virtually no experience with Jews who were culturally recognizable as such, and certainly not as groups. There was practically no Jewish cultural group in Germany at the beginning of the 20th century. These were basically phantasies of the extreme right.\\
These two parameters are different in Poland at the moment. In other words, Jews in Poland have a much stronger recognizable Jewish identity, even as a group identity, which goes well beyond the religious and becomes tangible as a cultural linguistic, ethnic and economic identity. At first, there were considerably more areas of conflict. One should not only rate conflicts negatively, there are also productive differences in a society that stimulate development. There is a much larger base for this kind of conflict in Poland, also because Jews were not assimilated over the centuries for a much longer period than in Germany. Until the Holocaust, the Jews were clearly defined culturally, linguistically, ethnically and economically. This, in turn, is related to the fact that over the centuries the Jews have been very intensively involved in the political and social history of the country. A very important area that does not exist in Germany, is the history of self-government in Poland, this term still plays a major role in the state or in the universities today. What universities call university autonomy in Germany is called self-administration by universities here; according to this model, local social fragments that have formed, for example, out of religious or ethnic reasons, organise themselves from within. This is also the basis for this Polish history of republicanism. Poland is arguably the only country in Eastern Europe where there have been a large number of edicts comparable to the \textit{de non tolerandis judaeis} written in Western Europe. It was a privilege for cities not to have to tolerate Jews within their walls if they did not want to. In Poland, in turn, there was the \textit{de non tolerandis christianis} for Jewish-managed municipal units. That is, Jewish communities in Poland have a much longer history than parts of society that have developed in parallel, a phenomenon which is also often described in the philosophical and political literature. Western European observers in particular have been interested in how this works. In the course of this observation, critical voices multiply, which then say that they are parallel societies and that it also has dangerous proportions if the social groups develop so differently. For example, this development of Polish society had a major impact on Polish romanticism, which was related to national rebirth in the 19th century and had a great deal of Jewish input. With Adam Bernard Mickiewicze one hears for example a very positive relationship with Judaism when it comes to seeking metaphysical inspiration for how to maintain a Polish identity in this difficult situation without a state of its own. There was a very fruitful and friendly dialogue between, let's call it Christian Romanticism and Jewish traditions in Poland. The second point is that, of course, where there are Jews, there is also anti-Semitism. Anti-Semitism has always existed in Poland, but I would not say that Poland is particularly marked by anti-Semitism. Of course, there have always been tensions. There were pogroms, but these took place more in the east, especially in the Russian area of Poland. But to think of something similar to the Holocaust was something that never happened in Poland. Those who defined themselves as anti-Semites in the first half of the twentieth century have, for the most part, never gone as far as they did in Germany. On the contrary, there were very important people, including in this area, who started from racial nationalism and thought the Jews were a threat to Poland. These people then started to save Jews under the impression of the Holocaust. In that sense, I believe that the two countries cannot be compared directly at all.\\
There is a sore spot in Poland: one has the feeling that the Germans have made the Holocaust and now they are moving around, accusing people of anti-Semitism. This is a situation many people do not know how to handle. 

\textbf{Are there regions of which one could say that they were particularly anti-Semitic? Was there a difference between urban and rural areas?} 

\textbf{Dr Steffen Huber:} There are certain regional hotspots, we can see this even today in the election results. In Łomża there were nationalist groups in the 30s that very much defined themselves through anti-Semitism, they were very active. [\textit{Pawel Karpiński enters\footnote{Can we tell at what point Krzystof Turek enters? Listen to the audio recording once again}}] \\
In this region was more activity by right-wing parties in the, 1920s and 30s than in other regions. In this region the Jedwabne murder took place. But I would not say that this is only a problem of this specific region.\\
What I suppose is that in those regions where you find more anti-Semitism, you will also find more cultural contacts between the Christian majority and the Jewish minority. 

\textbf{You were talking about examples of anti-Semitism that you experience here. What would that be? Can you tell us about some of these experiences?} 

\textbf{Dr Steffen Huber:} First I have to clarify how I understand anti-Semitism. I think that there is a quite useful definitions by Hannah Arendt, according to which anti-Semitism is based on images of the dangerous Jew, of the Jew who rules, of the Jew who is not rooted in the society and the culture or even in the ethical history of a nation, and of course, you can find this in Poland as you can find this in any other country. I am travelling around in Eastern Europe and I would not say that Poland is a hotspot of anti-Semitism. We have to remember that there was a very intense development of Jewish culture in Poland which later on moved to Israel. As we know that in the first year of the Knesset, the informal second language of the parliament was Polish because everyone from this part of Europe more or less used the “Lingua franca” of Polish, for example also those from the Ukrainian and Belarusian parts. I encountered anti-Semitism in Poland in two ways: First, there is a classical form of anti-Semitism that we find in the writings for example of Feliks Koneczny. He was a historian working here in the University of Krakow and he wrote some books on history from a very specific perspective. He presented a theory of civilizations saying that the highest form of civilization is the Latin, then we have the eastern one, which means Russia, and then we have the Jewish civilization. Of course, the Jewish civilization is the worst and the least and the most dangerous. He argues that Jews who are present in any way in a culture, in the economy or in the society, more or less destroy this higher civilization of the Latin type. He even came to the conclusion that the Holocaust, which he called a crime that was not acceptable in any civilization, even when directed against the 'Jewish civilization', is the outcome of the, he uses Nazi terminology here, '\textit{Verjudung}' of the German civilization. In his opinion, what happened in the Holocaust was that the Jews made the Germans organize the Holocaust. This is an extreme example of the intrinsic stupidity of anti-Semitism. These texts are used by the right-wing political movement in Poland, they're quite popular. They are even used in some part of the academic discourse and of course, we have very hard conflict over that.  Another example of anti-Semitism from Krakow: I talked to a man whose family owns some flats in the former Jewish part of Krakow, Kazimierz, and when I asked him about the Jews who lived in that house, he stopped talking to me. This does not mean that this man is an anti-Semite. Perhaps this means that his family had some bad experiences with the Germans or with anyone else in this context. This also means that what seems to be anti-Semitism sometimes is just the inability to speak about something what has happened to your own family. Even if this was 70 years ago. And I think that we should also learn to understand what real anti-Semitism is and what a situation is that does not allow you to speak about some experiences that you’ve made. 

\textbf{So, the idea would be that you cannot talk about this experience and then, you switch back to some usual stereotypes that you can easily state and then use it?} 

\textbf{Dr Steffen Huber:} Yes, we should learn to differ between this and the classical anti-Semitism I saw in Poland. This intrinsic stupidity of anti-Semitism can be exported to other topics and to other ethnic or religious communities. This is what happens in Hungary now. We observed this in Poland as well. I think the technology of exporting the logic of anti-Semitism to other topics, such as refugees from Syria, was invented by the Hungarian government, and now it's being used by the Polish government. It’s a way of copying anti-Semitic patterns and using them in a new political discourse. 

\textbf{But these new discourses, they are directed against different groups, like refugees, for example, but not against Jews, are they?}  

\textbf{Dr Steffen Huber:} Just observe what the French right-wing Front National is doing. In a way, they argue that they want to protect Jews from being attacked by Muslims, but there is also the argument that the Jews are organizing Muslim immigration to destroy our culture, which is structurally quite close to the argument of Feliks Koneczny. I think we can clearly identify anti-Semitic patterns in the public discourse and they appeared to be directed against Jews sometimes, sometimes they are used to protect you from Muslims and sometimes they put together Jews and Muslims as this dangerous type of civilization, too strange and unacceptable for us. This is what happens in the official political discourse in Poland as well. 

\textbf{Is this limited to the present government? Do politicians from other parties also speak in this manner?} 

\textbf{Dr Steffen Huber:} I know some people who are really attached to the government of Prawo i Sprawiedliwość, de facto of Kaczyński. And I'm absolutely convinced they are no anti-Semites. They are deeply rooted in that kind of even pro Jewish romantic tradition in Poland and they would never accept any kind of anti-Semitism. And in the last weeks we also observed a process which I personally welcome, that government and the official Media tried to fight anti-Semitism and fight right-wing ideology - this has to be stated as well. On the other hand, I met people who are rather left-wing, or liberal, who are pro-European and so on, and they are anti-Semitic. This shows that there is no clear correlation. If there is a clear correlation of right-wing thought of the Conservative types with the nationalist or even racist type this is the extreme right. But this is not true for the main part of the conservatively thinking people in Poland. \\
\textbf{Pawel Karpiński:} The government and the media are trying to fight anti-Semitism and xenophobic ideology, but I doubt that it is a clear intention. For example, our new Minister of Interior once said that he clearly does not tolerate any kind of racism or xenophobia, but when legal procedures run against nationalist parties, nothing happens. These cases are dismissed, and this is a clear sign that it’s being tolerated.\\ 
\textbf{Dr Steffen Huber:} Take as an example Mr. Winnicki, he is the leader of Ruch Narodowyis. He is talking about racial separation and he came to Dresden and shouted the Nazi slogan “\textit{Deutschland erwache}”. If you shout “\textit{Deutschland erwache}” at a meeting of Pegida in Dresden, this is Nazi ideology. After that, he declared he did not know what it means.  

\textbf{This racial separation, would it also include Jews as a separate race?}

\textbf{Dr Steffen Huber:} Try to ask them.  

\textbf{Is there this kind of anti-Semitic stereotype that Jews appear to be keeping the Polish nation on its knees after the War and the Soviet rule?} 

\textbf{Dr Steffen Huber:} No.\\ 
\textbf{Krzystof Turek:} This anti-Semitic motive doesn’t really exist because we never had bad experiences with Jews in that context in our history for many centuries. Jews were an essential part of our Society. They've been building our economy and our science. They were also Polish citizens. There is only one person which is Jewish and which is also seen as an enemy of Poland by the government, George Soros. He is seen as an enemy because he wants, with his “dirty billions”, to force people around the world to be become atheists and liberals, but this isn’t connected to his Jewishness.\\ 
\textbf{Pawel Karpiński:} I suppose it's not something completely original and George Soros is just an ideal feature of this. I would say one of the pillars of anti-Semitism is the fear from being dominated from abroad - it might be by the Jews, it might be Berlin or Brussels, it might be Russia, and we are willing to do anything to be safe from that because of our history. We had grandfathers who fought wars for Polish independence, and even though they won, we still were not independent. It's so ironic.  

\textbf{Did you ever encounter the stereotype of Judeo-Marxism, is it there?} 

\textbf{Dr Steffen Huber:} Yes, it is used sometimes, but it's based on the romantic heritage of the 19th century in the fight against the Russians, the Prussians and the Austrians. This was very closely culturally connected to the Jewish experience, so this is a very difficult situation.\\ 
\textbf{Krzystof Turek:} There is another thing connected to Jews. It has a little bit of a different character, because the communist state has openly dismissed Jews in the 60s. Even the communists in Poland, 20 years after World War II, excluded thousands of Jews and people with Jewish routes. They lost their posts in public institutions, they were all removed from the universities, they were removed from the party, even those who were loyal party members. 

\textbf{You have mentioned several times that these people see liberalism and atheism as a big problem. What is the position of the Catholic Church in Poland regarding anti-Semitism? In Germany, the Nazi party was especially strong in Protestant regions, and rather less in Catholic regions. For example, in the region where we are from, it's in Bavaria, the City is Nuremberg, which is a largely protestant region were many prominent protestant reformers of the 19th century lived, and it was very anti-Semitic and very strongly supported the Nazis. Maybe you know Julius Streicher. He held many rallies in Nuremberg and was very successful. Apart from that, you can also look at the election results. Of course, in Poland the Catholic Church has a huge majority – so, what is their position?} 

\textbf{Dr Steffen Huber:}I think there is not just one Catholic church in Poland and I'm wondering why they don't split - which I think would have happened if Poland had a different history. The Church has been the strongest and most endurable institution in Poland for 1000 years, and it is very well trained not to split in a situation of conflict. Nevertheless, the Church was much stronger than the state for centuries. So, you have a part of the Catholic Church which is clearly pro-European; some texts written Pope John Paul II some 30 years ago would be unbearably liberal for a big part of the Polish Society. If you don't tell them that it was written by the Pope, they will say this is liberal ideology from the West. On the other hand, you have a very long tradition of Catholic nationalism of Catholic anti-Semitism in Poland. 

\textbf{When we are talking about the last three years of economic development, does it influence anti-Semitism anyhow?  For example, could you say that when nations are rising socioeconomically, they develop stronger feelings of anti-Semitism, or that richer people have stronger anti-Semitic attitudes? Is there anything like that?} 

\textbf{Dr Steffen Huber:} No. 

\textbf{Is there any connection between the socioeconomic status of a society and the level of anti-Semitism?} 

\textbf{Dr Steffen Huber:} No.\\ 
\textbf{Pawel Karpiński:} I suppose there might only be a connection between low status and identifying a threat: Some sentiment arising from seeing those who are well situated, who got money, and feeling it's somehow unfair. \\  
\textbf{Dr Steffen Huber:} That is true, but it's also a stereotype of anti-Semites and racists. And the people with conservative, nationalist or racist convictions in parts of the society were absolutely not bad situated or in a bad economic situation. The rural part of Poland has developed very strongly over the last 10 or 15 years. It is true it has not developed as fast in the 1990s and before the membership of Poland in the European Union, but over the last 15 years you could see a very, very strong development. I rather feel attached to those philosophical theories that say that anti-Semitism is not a political conviction or political instrument. Of course, it happens to be one, but its most substantial element is the need to feel better than someone else. \\  
\textbf{Pawel Karpiński:} When you are working hard and there is someone else who is working less hard in your opinion, then we might think it is unfair that they are better situated than we are, and this envy is the thing that might create such anti-Semitic views. I’m not saying that all the poor people are anti-Semitic.\\  
\textbf{Dr Steffen Huber:} Are you referring to cities or to rural communities? \\
\textbf{Pawel Karpiński:} I'm not sure.\\ 
\textbf{Dr Steffen Huber:} In my opinion, in rural Poland, of course they don't like liberals, they don't like the European Union. They use the European Union, but they don't like it. They don't like the liberal elites in Poland. But there are no anti-Semites. The anti-Semites I met where rather well-situated in terms of economy. But that depends on your personal experience...\\
\textbf{Krzystof Turek:} I want to give a very specific example of anti-Semitism in Krakow, but it is in no way related to the previous topic. There are two main soccer teams in Krakow, Wisła Kraków and KS Cracovia, and the fans don't like each other. Cracovia was founded by Jewish citizens of Krakow 113 years ago. Now for many fans of Wisła, anti-Semitism is a part of their identity as fans of the club. So, in Krakow, when you see anti-Semitic symbols drawn on a wall or a building, 95\% were done by football fans because of that. 

\textbf{This vandalism you're talking about, is it directed at synagogues or cemeteries which are not related to soccer?} 

\textbf{Krzystof Turek:} I don't know.\\ 
\textbf{Dr Steffen Huber:}They have to good security around the synagogues. If this happens it would be persecuted for sure. 

\textbf{This means that they need to have security?}  

\textbf{Dr Steffen Huber:} Yes, but not for that reason. There is a lot of tourists in Krakow, big parts of them from Israel and the United States, and they're travelling around. For example, Jewish groups that are travelling to Auschwitz. They need some level of security. Yes, there are a lot of anti-Semitic symbols around the city but I would say not close to the synagogues because this is where the police are looking the closest. 

\textbf{And you would say that these symbols are related to the fans of this soccer club?} 

\textbf{Krzystof Turek:} In Krakow yes. 

\textbf{Is the soccer club doing something against it?} 

\textbf{Krzystof Turek:} I have no idea. But as far as I know, soccer clubs in Poland accept that there are also radical groups of fans because these are the people who fill the stadiums every week and also the people that make 20 meters flags and bring them to the matches. \\
\textbf{Dr Steffen Huber:} If you ask them why they do not do anything about it they could possibly say that this is not Germany where you have a very long tradition of murders. This is what you sometimes here in Poland: There is no tradition of bloody murder in Poland, so you can allow much more aggression in a verbal way which will not become physical aggression. Of course, this is a very risky way to go, and it might at some point not work. But this is how many people look at this problem. 

\textbf{What about the Jewish Communities? Do you personally know any of their members? How do they view the situation of anti-Semitism in Poland? Do you have heard anything about that?}  

\textbf{Krzystof Turek:} I don’t know any person who is Jewish. After WWII and after the next cleansing made by the Communist Party, they were barely any Jews in Poland. \\ 
\textbf{Dr Steffen Huber:}I met some people. They are living in a quite normal manner and of course they will tell you about anti-Semitism. It’s not an everyday experience. It's not systematic physical aggression, but they will tell you that of course it happens from time to time. 

\textbf{You were also talking about the role of romanticism. Do you think it only is conducive to anti-Semitism because of its negative stance towards rationality?} 

\textbf{Dr Steffen Huber:} No, this has to be treated very carefully. You’ll find some roots of aggressive racist nationalism in romanticism. But some authors are clearly pro-Jewish and they take a lot of basics from the Jewish tradition. Those who fought romanticism in the 19th century in Poland belonged to the positivist movement which in the beginning was very liberal but after 30 or 40 years, at the beginning of the 20th century, it turned into the strongest and most serious strongest anti-Semitic force in Poland. The national democratic party is rooted the positivist movement. This is quite strange and you cannot say that anti-Semitism is romantic, anti-rational and so on, and positivism is pro-Western and liberal and rational. It is just not true. I think a substantial difference between Poland and Germany is that the Jewish culture in Poland was much more conservative and much more religious and much more community and family based. This was the experience of the Poles and this is how they tried in 19th century to make some kind of Polish Jewish dialogue which really worked out in a very great manner.  You have great pieces of literature and theater which deal with these common metaphysical feelings. This is really a great part of the Polish literature and culture, and this is also in the writings of Pope John Paul II, which of course are not read by the average Polish reader right now. ``Lingua Tertii Imperii'' by Klemperer shows it in the clearest way. He said that the Nazi ideology is based on a romantic pattern. But that this is a thing you cannot say about Polish culture; it wouldn't work out here.\\ 
\textbf{Krzystof Turek:} I want to give another example: ``Pan Twardowski'' is a work by Adam Mickiewicz, the Polish ``Faust''. The Jew of the story is one of the most politic characters in the whole complex work.\\
The main idea behind this is the idea of fight between good and evil. A Messiah is a person who stands up against evil but for the very cause of standing up against the evil even if the cause is lost, as it was during the many uprisings that also are very important for Polish identity. Even if you know you fail. In 1943, there was the Warsaw uprising in Warsaw ghetto, where thousand Jews rebelled against the good working war machinery of the Third Reich, and even though they knew they would lose, they still made the uprising. This is an obvious thing to do for someone who thinks this way. For both Poles and Jews. This is also the thinking now, about what the polish right-wing government is doing. They are standing against the evil of multiculturalism because it is evil and destroys European culture.\\ 
\textbf{Dr Steffen Huber:} Which is not a romantic message. If you read Mickiewicz, he is never talking like that. He was trying to restitute the ethnically and culturally heterogeneous Poland which existed until late 18th century. This was the romantic model of restitution in Poland, so, yes, Kaczyński is trying to put together two traditions, the romantic and positivist tradition. Yes, of course, there is some anti-Semitism. Of course, there is xenophobic material in romantic traditions, but there is more of it in the positivist tradition if you look at the late works. There was a strong conflict in the Second Republic before WWII between the romantic and the positivist parts of the society, and Kaczyński is trying to put these two traditions together. In a way, he is a pluralist talking about unity but practically, he is putting together very heterogeneous elements. If it helps him to get the effect he wants, he also uses anti-Semitism, but this is just an instrument for him. There is no deeper conviction. His conviction is that the west is bad. 
\section{Serhii Czupryna}

\textit{Serhii Czupryna (*1996) is a member of the Jewish community in Kraków. As such, he is involved in educational activities fostering dialogue between Jewish and non-Jewish people. He was born in Ukraine but is of Polish-Jewish descent, he is living in Kraków since 2013. He has lived in Israel for several years while studying as a kid and later as a university student. He studies at the Institute of the Middle and Far East of the Jagiellonian University. \\
We met him in the Jewish Community Centre (JCC) in Kraków on January 27th, 2018.}\par
\vspace*{2em}
\textbf{Have you ever experienced anti-Semitism in your life?}\par
\textbf{Serhii Czupryna:} Yes. Living in Israel, I’ve been to the West Bank a lot of times and I wanted very much to go to Gaza, but that was difficult for an Israeli. In Poland, I’ve never experienced anti-Semitism in its harshness as it may be seen by people from outside of Poland. The first question a lot of Jews, especially in Israel, ask me when I say I live in Poland and that’s where I chose to live, is: ``Why? Six million. Why?'' But I see Poland not only as a place of historical depreciation of Jewish culture, but as a great place for its revival. I feel super comfortable of being Jewish in here, I feel super safe to walk around the city with a kipah on my head. I’ve never seen anybody in Kraków being mean to me just because of the fact that I’m Jewish. \par
\textbf{So, do you think anti-Semitism is a serious problem in Poland?}\par 
\textbf{Serhii Czupryna:} I don’t think of it in terms of the Poles against the Jews and the Jews against the Poles. I see it as a lack of education about who Jews are, what they do. I’m very happy to be with such a group like you and to answer questions, because dialogue is the first step to get to understand another culture. \par
\textbf{Is the form of anti-Semitism in Kraków more passive than in other cities?}\par
\textbf{Serhii Czupryna:} It may be passive, as I see Kraków as a big globalised city, in Polish borders it’s one of the biggest cities, there are a lot of universities here, there are a lot of young people, young adults who come here for the outsourcing companies, to work here. From year to year, I see that this city is becoming more and more international and multicultural. In this situation, I don’t see growing hate for any other culture. That’s nice, it's kind of a change that I’ve seen since I came to Poland. Now, it’s not only tourists who come here and see the Old City, go to Auschwitz and go back home, but there are people who don’t even speak Polish yet they live here, work here, do their everyday stuff here. By this, they’re bringing their own culture that works for everybody, also for the Jewish culture which is here, and works positively for Jews as well.\par 
\textbf{Do you think people in Kraków want to know more about Jewish culture?}\par
\textbf{Serhii Czupryna:} Yes. There is a Jewish Culture Festival annually since twenty-something years. Visiting this event year by year, I see that more and more people are coming and simply because this event exists, I would say that people are definitely interested in getting to know Jewish culture more.\par
\textbf{Do you as a student work with secondary school students, or do you do anything to educate people about the Jewish culture and heritage?} \par
\textbf{Serhii Czupryna:} I would love to, but as I do my masters right now and I work full-time, I don’t have much time to do so. But as soon as I’m done with my masters, I’m going to educate people about the Jewish culture, the influence of it on the Polish culture.\par
\textbf{Is the Jewish culture matter of the teachings, generally?}\par
\textbf{Serhii Czupryna:} Yes, there is a whole department of Jewish studies in the Jagiellonian University, so it’s both historical and cultural. Academic studies on Jewish culture exist in the city, they thrive and revive.\\ 
I also know a few NGOs who go to schools and educate teachers. But it’s difficult to teach about the Holocaust, because according to the programme, it’s taught in an inappropriate way, maybe some teachers just go through it or don’t even mention this subject. And maybe some will say, ``six million people were murdered here during the Holocaust'', and then full stop. But a lot of people don’t mention that then there was communism, and that in the 60s, a lot of Jews had survived the Holocaust and either stayed or left the country. It’s very important to educate in a correct way about Jews and about the historical context that the Jews are in here. I’m very supportive of those NGOs. I’ve participated in a few of their events.\\
Also, here in the Centre there are groups coming from the US, Israel annually, so this kind of impact on teachers is not only on the Polish side. When Israeli groups for example come to Kraków, they see only the history, they see the Holocaust, and a lot of groups just go through this building and say ``ok, this is the Jewish Centre, it’s cool that the Jews were here'', but they don’t see the community in here. So, for two years there has been an annual course for the guides of those groups from the US and Israel. They teach them how to raise more awarness for the existence of this community here.\par  
\textbf{Let’s continue talking about education. Would you agree that some media say that Polish anti-Semitism operates on a verbal level, mostly? Do you think it’s a chance to decrease anti-Semitism just by educating young people, especially in small cities, small villages and towns? In Kraków, people are better educated, they have lots of possibilities to meet Jews and Jewish culture. What if we started educating people, students in smaller villages and towns? Probably the hate speech and anti-Semitism on the verbal level would decrease. I think it’s the biggest problem that people sometimes say, ``I hate Jews even if I haven’t ever met any''.} \par
\textbf{Serhii Czupryna:} Exactly. I would say that it’s a very important matter because the bigger the city is, the more chances you get to interact with people different from you. But for example, if you’re from a small village, you have one school in this village, you have the homogenous society of Poles, you may have certain stereotypes about other people and other nationalities. It’s like this, ``Everybody hates Jews, I also hate Jews, I don’t know why''. I definitely see that there’s a very high importance of coming to smaller cities or villages to educate people on those matters because very often, nobody even knows that before the war the whole village was eighty percent Jewish, while now it’s one hundred percent Polish. A historian would come and say that there was the house of that and that person, there was a cemetery there and there, but locals don’t even know about this. Education would help to change the verbal level of anti-Semitism. A lot of people, for example Polish, would use the word ‘Jew’ in this a little bit pejorative form of ‘Żydek’. There are two different words: the person who believes in Judaism and the person who is born Jewish by nationality, and those are both different from ‘Jew’, but the word itself, ‘Żyd’ comes from the Hebrew three letters, the base of a word Jehud which stands for the person who believes in Judaism. For me, knowing that fact when people use the form of the word ‘Jew’ in Polish to any pejorative meaning, I explain them ‘guys, this is how it works, and stop because it really has nothing to do with this matter’. But the level of the verbal anti-Semitism itself is, from what I’ve noticed, not even that harsh in Poland. It definitely exists, but even comparing to Ukraine, there is way less of verbal anti-Semitism. I’ve came across this a few times in my life, but I also don’t see it as anti-Semitism, because people maybe just didn’t know what they were saying. Education is the key to change all the negative meanings in our lives and our societies. As soon as we educate ourselves and other people on certain matters, the negative aspect of them definitely go away.\par 
\textbf{I’m not a history teacher, but my colleagues told me that students are usually not interested in the lessons of the Holocaust and anti-Semitism, it’s a little bit tiring for them. Don’t you think that it would be a better idea to introduce some practical classes about it, I mean instead of two or three lessons about anti-Semitism and the Holocaust, let’s go to Auschwitz, let’s go to your Community to show students what you do.}\par
\textbf{Serhii Czupryna:} I would totally agree with this. I mean, I would not necessarily agree with the fact that high school students are the best audience to go to Auschwitz. I went to Auschwitz for the first time when I was twenty, that was far away from my high school ages. And being twenty, I realised well that I was not ready for it yet. So, I would not say that visiting Auschwitz is the best way of practical studying about the Holocaust and anti-Semitism, but such things may on the other hand, increase the interest of the students.\par  
\textbf{Do you as a Jewish Community Center organise workshops for students?} \par  
\textbf{Serhii Czupryna:} There is an organisation based in Kraków, they have a project which is called ``Alev Bet of Jewish culture'', ``the alphabet of Jewish culture''. It takes place annually in the autumn or in winter. They educate people on the whole basis of the Jewish culture, explain the basics of certain traditions and celebrations.\par 
\textbf{Are you a religious person, and do you take part in Jewish celebrations?} \par
\textbf{Serhii Czupryna:} I’m not that religious, but yes, I would definitely say that what I believe in is Judaism. Not the orthodox version of it, but I also live in a kosher home. I don’t put lights on on Shabbat, and there are mezuzahs at the entrances to my rooms in the flat. I keep the kosher kitchen. I live with two more religious friends. From time to time, I go to synagogue. \\
There is the possibility of being a religious Jew in the city. There are three more or less full-time functioning synagogues. One is the Isaac synagogue, probably it’s the biggest in a matter of size, and it’s run right now by the movement Chabad-Lubawicz. So, their aim is to teach the Jews how to be a Jew and also to teach other people what the Jews are, and educate about Judaism. So, their main matter here is education, both to Jews and to non-Jews. They also run religious Jewish Sunday school. Here, there is a Jewish preschool, where children also get to know more about Judaism, about the tradition, religion, and certain religious celebrations. Three times a day there is a minjan, just like in the more orthodox Remus synagogue. The Chabad synagogue is more open for everybody. There is a full-time rabbi from an organisation which is called Szewa Israel, who works for a chief rabbi of Poland, Michael Schudrich, and he is a full-time employee of the JCC. He also provides the Thora studies on Shabbat and helps people to convert. There are many different options of being a religious Jew in this city, as well as in other cities in Poland.\par  
\textbf{Do you speak Yiddish and, in general, how many people do?} \par
\textbf{Serhii Czupryna:} I don’t speak Yiddish because of my family’s historical background. There is a certain amount of people who live here, who speak Yiddish. First of all, the more religious, orthodox Americans who chose to live here, they probably would speak Yiddish or at least some standard. Until this summer, we had one of the oldest members of the JCC, Mr Mundek, for whom Yiddish was the first language, so he learnt Polish as a second language, not native. And he was born somewhere around Kraków or maybe even in Kraków. He spoke perfect Yiddish; unfortunately, he passed away this summer. There are certainly other members of the community who speak Yiddish. In this building, there is an opportunity of having a private course of Yiddish. Besides, they teach Yiddish in the Jewish studies department of the Jagiellonian Univeristy.\par
\textbf{And what about Hebrew?} \par
\textbf{Serhii Czupryna:} I do speak Hebrew. Right now, I work as a kind of Hebrew native speaker in one company in the city. There is also the possibility of joining a Hebrew course here, it is provided for ten different levels of Hebrew language skills, so there are people who started learning it since the JCC opened in 2008, and they are still in their groups that continued from 2008, almost for ten years. Hebrew is nowadays very popular in Kraków, I’d say.  \par 
\textbf{Are there any orthodox Jews in Kraków and in Poland in general?} \par 
\textbf{Serhii Czupryna:} Yes, definitely there are a lot of orthodox members of the community in general, and are few orthodox members of the JCC community. First of all, not every orthodox will have a beard, not every orthodox will have sztrajmł, this huge hat. A lot of young people find themselves feeling way more comfortable in orthodox movement, they look just like us, you can spot only certain things which show that a person may be orthodox. When it comes to interacting with them in everyday life, you would notice that if you say ``let’s eat a sandwich'', this person says ``oh, but I need to wash my hands''. And there is the whole ritual, it’s complicated, Judaism is complicated. \par
\textbf{Last November there was the demonstration in Warsaw with sixty thousand people, and hundreds of them were shouting ``\textit{Sieg Heil!}'' and ``Jews out''.}\par 
\textbf{Serhii Czupryna:} It’s the question that I often get from the people who ask me, ``why do you chose to live in Poland? You’re Jewish, go to Israel''. You’re talking about the Independence Day demonstration. The thing is that there were the far-right neo-Nazi movements, and it may be not very safe to interfere with them in Warsaw on the Independence Day. I was there once, I just had to help my friends. I had two friends from Israel visiting me in Kraków, and we came to Warsaw together on  November 11th. And one of them is of half-Indian, half-Iraqi descent. He was born in Israel. And the second is from Ukraine. The problem was that none of them speaks Polish, and one of them looks definitely not Polish. So, I was just afraid of them going to Warsaw by themselves, especially seeing this in the media. But as I came there with them, I realized that as soon as we just don’t go any close to the demonstration, which is controlled by police to a certain extent, we’re fine, there’s no problem with speaking Hebrew in the centre of Warsaw in the evening of the 11th. I would say that it’s very important that media show this, but I also see it as just a different opinion as long as it doesn’t interfere. They may destroy certain parts of Warsaw, but they don’t hurt people. As long as they don’t hurt people, I’m fine with it. That’s not the majority of Poland. According to me, their ideology is definitely a bad thing, but I see the lack of knowledge about the Jews. Even trying to interact with far-right people or with people who perceive neo-Nazi movements, both in Ukraine and in here, one of the main claims was: Jews go home. Why? Lots of Jews were born here, Poland was pretty much Jewish before the War, so where should we go home to? I can go home to Ukraine, but my ancestor is from Poland. I see both of these countries as my home. So, to which home should I go? And then the conversation goes on, ``oh, okay, you were born here, so you’re fine, you speak Polish, it’s a different thing''. No. It’s totally the same thing, start educating those people about what is Jewish, who is a Jewish person. Sometimes, I can open their eyes and they say, ``okay, I’ve been doing wrong'' or ``that’s probably the ideology, maybe it’s not that prefect, so maybe I should switch for something less far-right''. I definitely see there’s lack of education, and what I see as a problem is that the Polish government is not intervening in that situation. Until the moment the government bans neo-Nazi organizations, I think that it’s going to be the same, but it’s definitely not these sixty thousand people.\par
\textbf{The whole demonstration consisted of 60,000 people. And some hundreds were shouting.}\par 
\textbf{Serhii Czupryna:} Yes, it’s a small percentage. \par
\textbf{But nobody from the sixty thousand cared about it. Normally, I would throw these people out of my demonstration.} \par 
\textbf{Serhii Czupryna:} Yes, that’s also an organisational problem. What I dislike more about those marches on 11th of November is that each year, there was a symbolical burning of the statue of a rainbow on one of the squares, which stood for diversity and tolerance. If it was removed, so that it’s not going to be burnt annually, that would also be a good step, first of all not to symbolise the intolerance while celebrating the pride of being Polish. I cannot call it patriotism; I call it nationalism. But certain people go there to celebrate patriotism, certain people go there to celebrate nationalism. That’s the problem that maybe the organisers of those marches face.\par
\textbf{Don’t you think that the media, especially the media abroad stress the nationalistic marches and the fact that there are some acts of nationalism, not patriotism? What do they want to stress?}\par 
\textbf{Serhii Czupryna:} Media never show something like ‘the panda was born in this and this zoo, let’s celebrate it’, it’s definitely not interesting for people abroad. But if people hear that something bad or something incredibly good is happening, only then they would be interested in what is shown to them. That’s why I would say that media stress this fact of nationalism, not patriotism. This year in Israel, it was the Independence Day of Poland, and I saw a lot of videos on the internet, on social networks, where were people saying ``I go to that demonstration not because of the fact that I’m nationalist, but because I’m patriotic.'' And a lot of them were translated to Hebrew or at least had Hebrew subtitles, so the situation about this matter of showing those marches as nationalistic starts to change, I would say. But it takes a certain amount of time to people to understand that there is a difference between people who go there to show some nationalist beliefs and between the people who are patriots of their country, who don’t try to put any other nationality below them, who just celebrate the fact that they are proud of who they are.\par
\textbf{Do you think that teenagers are more anti-Semitic than older people because they frequently use the internet and see a lot of hate about Jews? }\par
\textbf{Serhii Czupryna:} I think that the middle-aged group is the most anti-Semitic of all age groups. What I like about the globalisation is that if you’ve seen a lot of hate on the internet, you can just google certain things that may disprove this hate, and the availability of the information for teenagers I see as a very good thing. I would definitely not say that teenagers may be more anti-Semitic just because they see some propaganda on the internet or in the media. Now, in my generation, I’m twenty-one, it’s always like ``okay, I would look for this, but also I want to find the arguments from the other side''. I would say that a lot of people would choose the peaceful version of perceiving some other minority, not the hateful one, and they would probably want to learn more about the conflict.\par
\textbf{Do you think that the anti-Semitism problem will grow in the future?}\par
\textbf{Serhii Czupryna:} It’s an interesting thing. It would definitely depend on the region, and what is going to happen there in the future. I would say that in Poland no, because from what I see right now, non-Jewish Poles are getting really interested in the culture, and they start understanding the influence of Jewish culture on Polish culture. It decreases radically, which I like a lot. Maybe in countries like France and Great Britain, which are more globalised, it may grow because of a negative change in the society, connected to other negative factors that are happening with the society. And as the country struggles with certain things, there is a possibility of other negative factors to increase. But the world learnt through the Second World War that anti-Semitism was the depreciation of a certain national minority, and they learnt from the Holocaust that it happened once, so let’s not make it possible to happen anymore, with any group, not only the Jews.\par
\textbf{There’ is a new law in Poland called the law for protection of the good name of Poland. What do you think about this law?} \par
\textbf{Serhii Czupryna:} I would not agree with a lot of laws and the implementations into the law that the current government brings on, not only in term of this one. From the patriotic side, this matter actually stands in line of the tension between patriotism and nationalism of the current government. It’s really hard topic: Auschwitz, the relations between Israel and Poland, but the whole relations between Poland and Israel and also the exchange of tourists and students and people just coming and going back and forth, these relations also increased in the recent few years. I think that as soon as the government changes, at the point where this law may be partially changed, it’s going to be fine. I don’t think that it may drastically change the relations between Israel and Poland, because diplomatic relations have been built up for a long time, with a lot of effort, so I don’t think that one law like this may change a lot. I certainly see it as something which is really a problem, which depreciates the importance of other nations, national heroes of other countries. The same topic was raised recently in Ukrainian media, also with regard to this law because the national heroes that the Ukrainian government perceives to be national heroes were fighting against Polish forces in a certain period of time, and throughout the history, this whole interfering between the Poles and the Ukrainians took place. It was perceived negatively in Ukrainian media, and also in Israeli media, and it was depicted a anti-Semitic. But I would not say it will drastically change the situation between two countries.\par
\textbf{According to Jan Gross and his book ``Fear'', in Poland solidarity with the Jews during the Second World War was not a mainstream thing. That means in France, people were celebrating when they helped Jews, and in Poland, many people are ashamed and they don’t want to be mentioned to publicity. Is that right?}\par
\textbf{Serhii Czupryna:} The whole matter of helping Jews during the Second World War is a very touchy subject, because a lot of people say ``oh, Poland is this death camp where everything happened''. No, it’s not. It’s just because of the fact that, sorry, Jews were living here, and it was the easiest way from the Nazi point of view to make it all happen there. Most of the Jews in this world were living in Poland. Helping the Jews during the War, you could have got a death penalty, you and your whole family. And probably this fear still remains in certain people. But on the other hand, when Israelis ask me, ``Why do you live in Poland? Most of it happened in Poland.'', the other side is that also the biggest amount of Righteous Among the Nations live in Poland, they are Polish and they are proud of it, and they celebrate helping the Jews in Poland during the War.\par
\textbf{Have you experienced any anti-Semitic acts from Muslims when you were in Israel?}\par
\textbf{Serhii Czupryna:} Actually, no. I feel very safe living in Israel. It’s just a fact that the country exists in the region where it has to be militarized to a certain extent, just to exist. And that’s what I know: That in Poland, you would feel weird if you’d go to a bus and there’s a soldier with a gun. In Israel, it’s totally a different thing. I feel safer when I see a soldier with a gun, just a normal servant who is of my age and who just has to do the service for his country. I have a lot of Palestinian friends, certainly some of them are Muslims, and I’ve never experienced any negative acts from their side. Also, in Poland I have a lot of Muslim friends , and I get along with them very well just because of some common things between their cultures or our religious beliefs. \par 
\textbf{How do Polish people behave towards you? Are they curious about your culture and religion? Or are they intolerant and reserved?}\par
\textbf{Serhii Czupryna:} They’re definitely interested. I participate in few non-governmental organisations, except for the JCC, and it’s interesting to see how people who are not even friends are interested in the fact that I’m a Jew and my friend is a Muslim. How come that we sit at the same table? We explain and say it’s just a religion. Why should we sit at separate tables or not even go to the same café? There is a certain amount of curiosity. People may see me coming from the synagogue with a kipah, and my Muslim friend in hijab, and we’re just coming across the same street, stopping to say hello to each other. Seeing the face of these people with certain stereotypes that they may have about our cultures, and seeing us destroying those stereotypes – it’s really pleasant and funny. But I would say that to curiosity, there also comes a certain amount of stereotyping certain cultures, and we try to together destroy the stereotypes.\par 
\textbf{You said that people in Israel are always surprised when you are saying that you chose to live in Poland? Why do you think the do? Is it because of history and media or stereotypes or even something different?}\par
\textbf{Serhii Czupryna:} I would say because of history and because of stereotypes. The biggest after-War influx of Polish Jews to Israel was in the 60s and in ‘68. That generation already has kids and grand-children, and probably a lot of them have never been to Poland except for Auschwitz. They think, ``My family survived the Holocaust, then they were kicked out of Poland''. This is the vision that they have about Poland since their childhood, from their parent and grandparents. I love bringing my Israeli friends to Poland to show them that this country in not black and white, as you see it on the pictures of your granddad. It’s the country which has a lot of beautiful sites, except for Auschwitz, they definitely should visit. And I like to see those people being shocked when we talk Hebrew in public and everybody is fine with it. It’s also a matter of education, destroying the stereotypes, it definitely comes from the stereotypes of Poland, and those stereotypes are created by a certain amount of negative historical background of Poland with regards to the Jewish society.\par
\textbf{Mrs Zajda told us yesterday that these tourists from Israel are always isolated by bodyguards. They only go to Auschwitz,  theydon’t integrate with other people, so they can’t even get to know our culture, our behaviour.}\par
\textbf{Serhii Czupryna:} It’s true. Because of the fact that most tours from Israel are organised with the help of certain ministries, the ministry of education or the ministry of foreign affairs. They should have bodyguards, it’s fine. The problem comes when these groups go to synagogues. Can you imagine? We have the school trip to Italy and even if you’re not religious, you go with bodyguards to the Sunday mass, and you’re supposed to spend the whole time in the mass. That’s what happens when those groups come to the synagogues and, according to me, it’s like one of the small windows for them to interact with the local community because technically we also go to the same synagogues. Often when I try to enter the synagogue, I’m stopped and asked by Israeli bodyguards ``Wait! Stay here. Where are you going?'' Obviously, I have a kipah on my head. ``Come on, why should a Jewish person go to synagogue on Friday?'' I’ve noticed that the same bodyguards come with different groups. And after few situations like these, they start to remember your face and then say ``\textit{Shabat shalom, shalom}''. And certain Israeli teachers, if you get to interfere with them, anyhow talk to them, they start saying ``Okay, oh, so you live here? You’re Jewish? Fine. Maybe we can make some meeting of our group with you?'' It just needs some time to develop and also to explain the Israeli teens that Poland is not black and white, as on a picture from the War, and people here are smiling, and it’s not bad to talk to them and you don’t have to have bodyguards when coming here.\par
\textbf{I think there is a problem because in most of the classes, there are sixteen or seventeen- year-olds, and when they come to Poland, they visit two extermination camps a day. Of course, it’s too much. And the government of Israel makes a good silence with it. It’s a little bit propaganda.}\par  
\textbf{Serhii Czupryna:} It’s tricky because in the present situation in Israel, both political and military, in different contexts, I see that this propaganda is needed, the propaganda of ``let’s not let other people to destroy Jewish nation''. But it’s done inappropriately in the case of Poland, because instead of having this feeling that ``yes, we need to stay together as a nation, we should not fight between each other and keep everything as peaceful as possible'', those children probably learn about the fact that all the killings took place in Poland, and that’s the only thing they know about Poland. And that is a problem, it’s inappropriate propaganda. But Still, I see it as a need. I hope that Israeli will come up with change one day. \par 
\textbf{Don’t you think that the Jewish trips of students who come to visit Auschwitz or Bełżec and have gunned bodyguards can raise a bad attitude among Polish people? We may feel being perceived hostile. You said that the ministry tells them to do that.}\par
\textbf{Serhii Czupryna:} It’s actually a requirement by the ministry of education: As soon as you go to any tour, even in Israel, if you’re a group of more than ten people, you should have a bodyguard with a weapon with you. For them, it’s not exclusively for Poland. But I definitely understand, as a person born here, how people see these groups. It’s weird for Poles, it would be weird for Israelis not to do this, it’s just the conflict of two contemporary cultures. Obviously, they know that is safe in Poland, because the same kids in their free time are just rushing to the shopping mall and those guards don’t go there with them, obviously. That’s just a requirement. \par
\textbf{Don’t you think that Polish people are most sensitive about it? I mean when Jewish trip goes to e.g. Germany, Germans don’t care about a gunned man.} \par
\textbf{Serhii Czupryna:}: I’d say that Poles see it this way because of the history. In this historical context of Poland, after all that happened in this territory, it’s just unpleasant to see an armed person in any situation. But I don’t think that Germans are less sensitive in this matter.\par
\textbf{What are your thoughts of the people who say that Jews are very influential all over the world? Is it the reason that the people are anti-Semitic? I’m about the Rothschild family.}\par
\textbf{Serhii Czupryna:} Yes, definitely, in a certain historical context. Let’s start with the different variations of anti-Semitism. Let’s figure out that there are non-scientific terms, a ‘positive’ anti-Semitism and ‘negative’ anti-Semitism (but please, don’t stick to these terms in any other situation). A ‘positive’ anti-Semitism is, according to this understanding, based on the good stereotype about the Jews, but they are still stereotypes. For exampl,e those small figures of Jews holding a coin that you can buy as a souvenir on the main square. It is anti-Semitic, let’s face the truth. But on the other hand, it’s a ‘good’ stereotype, that they are good with money. O the thing that we’re all lawyers, doctors, musicians and maybe scientists, sometimes - three things a Jew can do in their life. It’s a very good stereotype when people say ``oh, you’re a good lawyer, it’s because you’re Jewish'' - ``No, it’s just because I’m smart''. And all those things, they have a negative context but they are not that negative as an actual anti-Semitism.\par
\textbf{How did the Polish society deal with some anti-Semitic, some Catholic anti-Semitic facts like the Madagascar option in the thirties, and the pogroms in Kielce or Jedwabne, was it critically discussed?}\par  
\textbf{Serhii Czupryna:} There is a critical discussion and there is a commemoration of those events by both Jews and Catholic priests. It is like the Holocaust: it happened, somehow the world did allow this to happen, but let’s not make this happen any time again. I would say that currently, the Catholics in Poland and the Catholic Church, especially in Poland stands for the same thing in terms of pogroms, \textit{pogrom kielecki}, it’s the field that is still studied and is not finished to be studied. It’s a very complicated matter, but I would say that the Catholic Church right now is very active in the field of commemoration, to let the people know this with the message of ``let’s not do this again, because it’s bad, let’s not kill people''. That’s one of the bases of the Catholic religion. %check out this interview to see the formatting style that Meggy chose, and stick to the following agreements: (1) direct speech/quotatiions and technical or pejorative terms in double quotation marks, as well as book/film names etc. No quotation marks for organisation names (2) phrases in another language in italics, applies e.g. for Nazi terms and Hebrew terms referring to Jewish life [aliyah, kashrut,...]. Put all newspaper names in italics!
\section{Dr Piotr Setkiewicz}

\textit{Piotr Setkiewicz (born in 1963) is the director of Centre for Research at the Auschwitz-Birkenau State Museum and a graduate of the Faculty of History at the Jagiellonian University in Kraków. Mr. Setkiewicz received his Ph.D. degree in 1999 at the University of Silesia in Katowice for the work entitled “IG Farben - Werk Auschwitz 1941-1945". He is the editor-in-chief of scientific publication The Auschwitz Journals (Zeszyty Oświęcimskie) as the head historian at the Auschwitz Museum. We met him in the Auschwitz Museum on January 27th, 2018.}\par 
\vspace*{2em}
\textbf{Thank you very much for your time for us today. We are a group of students from Latvia, Poland and Germany doing an Erasmus project about anti-Semitism and how it developed after the War. First, we'd like to ask you if you have had any first-hand or second-hand experience with anti-Semitism over the course of your life?}

\textbf{Piotr Setkiewicz:} Over the course of my life? Always, almost every day. Just yesterday, I was called by a man who is anti-Semitic, who is denying the truth, who said that I can be blamed for all mistakes, all the false information that appeared in the media about the Holocaust. That I'm supporting the fake version of history which is based on Jewish, Polish and communist lies about the Holocaust, that we are promoting the Survivors who are lying, and that we falsified the documents in my archive. And that we are taking money from Jews in Israeli shekels. Many such things. Many people in different countries, not only in Germany but also in United States, practically all over Europe, including Poland, surprisingly, tried to persuade everyone that the Holocaust never happened, that Auschwitz was built, or at least the most important objects here, like the crematoria and the gas chambers, only after the war, and that the gas chambers were actually used for delousing purposes, for disinfection. My reaction usually is that I don't want to be involved in any sort of discussions with Nazis because as far as I know from my own experience, it's just a waste of my time and the arguments always stay the same. For instance, that conditions in Auschwitz were relatively good, that it was necessary to isolate the enemies of the Reich because it was a war and that of course, sometimes it might happen that somebody died here. That, nevertheless, the number of victims is highly exaggerated, the number of one million people who were murdered in Auschwitz. That obviously, it's not true because the capacity of the crematoria was many times lower than we read in books, read in documents, etc. Even some strange and surprising arguments are in the discussions: For instance, there is a picture taken by somebody from the staff of \textit{IG Farben} works. \textit{IG Farben} was a huge factory of synthetic rubber; it was situated on the other side of the City. Nevertheless, during the War many thousand prisoners worked at the construction site there. Apart from prisoners, of course there was German staff, men and women, about 7,000 people, and, they resided in the Siedlung, the settlement, they lived somehow apart from the concentration camp, and they got, for instance, a sports club. And there is a picture of two people who are fencing, and there is a group of people in the background of the picture, and over them there is an inscription, the name of the sport club in German, the sports club \textit{IG Auschwitz}. Now, the Nazis believe that this is a typical representation of the living conditions in Auschwitz, that the people in the picture are prisoners - which is stupid, actually, because that was a German club and they got even the training courses for pilots of incomprehensible, 05:21 and many such things, the soccer section, the athletic section, etc. Other than that, one of the most important arguments is the presence of the swimming pool in Auschwitz, which is visible just behind us. We know that it was the one of many pools or rather tanks for water for the fire brigade. In 1944, the Allies, American or British planes began to appear over Auschwitz and the SS suspected the threat of mass bombings and a fire, particularly in Auschwitz-Birkenau So, they built a number of such pools, reserves of water for the camp fire brigade. And surprisingly enough, in Auschwitz, the SS, or rather prisoners themselves, the functionaries, the capos or block overseers, mostly the German habitual criminal prisoners, they organised something like a competitional swimming in this pool. There was something like that, as we know from the testimonies of Survivors and from somebody who was taking pictures, who was a fan. Nevertheless, there was a difference between the situation of the functionary prisoners and the real fate of regular prisoners at the concentration camp. 
So, as I said, there a many such arguments, many stupid discussions. The most difficult and rather new problem is the huge amount of such articles, comments, entries on the internet. On the internet you can write anything, and there’s plenty of web pages explaining that Auschwitz never happened. The problem is that recently, we are faced with more and more people denying the Holocaust in Poland which, for me , was something I could not believe because many Poles lost their lives in Auschwitz, as well as the Polish Jews, and the knowledge about the German crimes in Auschwitz was obvious for everyone in Poland, for many years after the War. Now, I am afraid that there is a third generation of people who have no personal experience, they probably heard ``My grand-grandfather was in Auschwitz'', but the distance to history is too long. Just a few days ago, the Polish television showed a ceremony of the anniversary Adolf Hitler's birthday on the 20th of April, a group of Neo-Nazis meeting in the forest somewhere in Silesia. They were celebrating the anniversary, they had the Wehrmacht uniforms, they baked a cake for the Führer with the swastika. Perhaps that is still marginal in Poland. Nevertheless, these views are shared by the people who recently protested against immigrants, against attempts from the European Union to persuade the Polish government to accept a certain amount of people from Syria and other countries. It's still not the same as with the Nazis, nevertheless, within the relatively large group of Poles who don't want to even hear about immigrants from Arab countries, there is a narrow margin of people, particularly young people – I’d say 20-25 years old, football fans, something like that - who try to make a link between their anti-immigrant views and the Nazi ideology. In the Museum there are very rare cases where we can meet such people here. It's not a problem for us, but of course, we believe that one day, we will face the situation that a group of people with the Nazi insignia will wish to visit the Museum - so, the problem is, what to do then? Our security guards should do something and prevent it, but something like this might happen. That's a good reason for us to do our work, to publish books. I don’t want to be involved any sort of direct discussions with the Nazis, nevertheless, we are providing arguments for a lot of teachers, arming them with the documents. German documents, I'd say, a more useful in such discussions between students and teachers because if we gather the testimony of a Survivor, it’s easier for a denier to say ``Well, it’s just nonsense, what this man is talking about''. For example, a few weeks ago I a critical review of a book written by a certain Survivor on the internet, and the Survivor wrote about 300 pages book about his experience in Auschwitz. And this is basically a good witness - I mean, there are many Survivors that are simply trying to tell us their stories. Nevertheless, there is unfortunately, and it's quite normal and natural, some people who believe that they should add something to their stories. For instance, if they were in Auschwitz, they testify about certain happenings that took place in Birkenau, and it’s not true because they could have heard about the Sonderkommando rebellion, but they had no chance to bear witness of this uprising. Nevertheless, I understand that something like this might happen, that particularly the Survivors who used to take part in certain ceremonies used to meet with schoolchildren, being asked about the War experience, wish to say ``I was in Auschwitz, I saw everything''. Going back to this book: Although on one side he was generally credible, at one point in his book he said that he remembers that there was a small room in one of the barracks where the Germans, the capos, collected the bodies of those prisoners who died during the nights. That it was only after three of four days when they collected the bodies and they were put in the truck and the truck was taking the bodies to the crematoria to be burned. And, the author of this article on the internet, he made use of the fact that arguments like that are obviously not true, because I found a German document in which the commander of the Birkenau ordered to remove the bodies every day. For him, this is an obvious argument that this man was lying. And if he was lying in this particular point, in this particular page of the 300-pages book, probably the remaining 300 pages, are also not credible. 
Another category of sources, of course, are the testimonies by the guards themselves They testified before the military tribunals of the Allied powers in West Germany, immediately after the War before the Polish courts, and then in Germany in the Frankfurt trials, and in many other locations. And at least some of them they accepted their guilt and admitted ``Yes, I saw the people being selected on the ramp. I saw the people being driven to the gas chambers, I saw the thousands of bodies and the flames over the chimneys of crematoria.'' So, what is an argument of deniers against it? That they were forced to testify in this way, beaten by the guards, by British, Polish, whatever, communist, Jewish guards. That was why they had to give these false descriptions. 
And the last category - the German documents: The problem is that most of the German files from the chancelleries, from the office administration of Auschwitz, they have been burned immediately before the final evacuation of the camp. If you take any testimony of a Survivor who was here in January ‘45, typically you can find therein information like ``I saw the piles of papers, of documents on the crossroads of the camp being burned by the guards''. And that's true. We've got a very small amount of original documents from German administration, a relatively large amount of documents from the German construction office, the Bauleitung, probably because that was a separate unit of the SS-administration, which was situated in several barracks at a distance about 300 metres from here, not inside the camp. So, by chance, for some reason, when the SS escaped from Auschwitz, the just forgot to burn these files. If you take into account all these surviving documents, they are good material for historical research, alone these plans and construction files of the central construction office of the SS. The amount is about 150,000 pages of documents, a large number of documents, and they are a very valuable source, very useful for our discussions – indirect discussions with these deniers. There is, for instance, a well-known author who wrote a number of books, and in one of them, he tried to summarise the most important arguments of the deniers. And one of them was, that if we talk about the first temporary gas chambers in Birkenau, called ``the little red house'' and ``the little white house'', the ruins of little white house are still in Birkenau, nevertheless he said that it’s impossible that in so many surviving documents there is no mentioning, not any hint about these temporary gas chambers. So, the prisoners were lying, Survivors were lying, because they were Communists, Jews, Poles, whatever – that’s obvious, and the main proof that these provisional gas chambers - they were abused mostly in ‘42/’43 – did not exist at all is that there is nothing written in the documents. His argument is as follows: That, if something like that had existed in Birkenau, then it must have been in the records of the trips of the trucks, especially for delivering bricks, and other construction materials to the site of these gas chambers – there must be anything. Information about the bills, about the amount of money spent for the construction of the gas chambers – but there is nothing. Thus, he tried to persuade his readers. But unfortunately, we have found seven such documents that clearly indicate that these gas chambers existed. That one of them costed 14,000 Reichsmark, for instance. That the SS ordered three wooden barracks for each of these gas chambers, which were used as a temporary store of clothes of the victims. And each of these barracks costed about 17,000 Reichsmark, and they were removed from the area in the moment when the Germans completed more modern and huge crematories in Birkenau, so the old gas chambers were dismantled and destroyed. We have found requests from the Bauleitung asking for the remaining wooden barracks, because they might be used for some other purposes in the camp. And we’ve got the information that there is a need to send 500 prisoners on a day to dig trenches around the bunkers. That was because one day in July 1942, Heinrich Himmler, during his visit in Auschwitz, ordered to empty the old mass graves and burn the bodies on piles of woods. 
Then, what else have we got to prove? Apart from testimonies of Survivors, apart from reports of Polish resistance members, some of them Survived. There is a document in which the SS ordered a cable, a long electric cable, to provide electricity to nineteen burning places. It was necessary to install lamps there because the burning took place not only in the day, but also during the nights. There are many direct or indirect proofs that these burning pits existed. I don’t think that you are familiar with the arguments by deniers, but we’ve got a relatively large number of counterarguments. The problem is that it’s difficult to discuss with such people because when they say that all – let’s say, 10,000 testimonies, the witnesses were lying, that these thousands of documents were fabricated by Poles, Jews, communists after the War - in such circumstances, it’s rather difficult to raise any arguments.

\textbf{What do you think, what makes people want to deny the fact of Auschwitz or the Shoah?}

\textbf{Piotr Setkiewicz:} Why? For many reasons. Perhaps, some of them believe that this form of government in Germany in the 30ies, that it was ideal, that everyone knew his place in the society - but now, what do we see around us? The politicians, thinking only about how to gain more money, who are no true patriots. And the Führer, that was a man who solved the problem of the economic crisis in Germany. He built the highways. He restored the pride of the German nation. These might be the arguments of the German deniers. In other countries, I think, the reason is mainly anti-Semitism. Because the Holocaust, as a part of the Jewish history, by most Jewish historians is identified as a marking point in the history of the modern nation. In some books written by Jewish historians, you may see that they divided Jewish history in two periods: Before and after the Holocaust. It’s a very important point on the map of the Jewish history. So, for those people who are anti-Semitic, who don’t like the modern state of Israel, the Holocaust is something which is a guilt and a giant point which should be attacked. And, in the history of the Holocaust, Auschwitz is as central point. There is the famous quotation from the interview with David Irving, the leading British denier, who said ``If we sunk the battleship of Auschwitz, it would solve all the problems of the 20th century’s history''. Auschwitz is a focus, the most important part of the story. If we would be able to proof that Auschwitz never happened it would be easier to destroy the myth of 6 million of Jewish victims.

\textbf{Would you say that these Holocaust deniers are becoming more or rather less over the last twenty-five years? Is there any development, so that it becomes more usual?}

\textbf{Piotr Setkiewicz:} Immediately after the War, it was the group of mostly the German veterans, particularly from the unions of the SS, who were deniers. It was a number of even the guards from Auschwitz who took part in certain ceremonies, where they solemnly declared that they were in Auschwitz, but they never heard about any gas chambers or crematoria. That was the first wave of denial. Then, it was in the 60s, there was a rise of interest in the Holocaust. On the one hand, we got the trial of Eichmann in Jerusalem, and on the other hand, a few years later, a series of so-called Auschwitz trials in Frankfurt. And it was the time of student’s rebellion in Europe, the leftists began to be interested the history of their own country, in the leftist press appeared the pictures of leading Nazis who were still living in Germany, the titles of the articles were as follows, ``The assassins, the killers are still among us''. So, there was this popular belief of the leftist that something must be done in order to cope with our past. And to punish the people who, in many cases, continued their pre-war career after the War. All the judges prosecuted those who had a Nazi past – the technicians, engineers from the companies like \textit{IG Farben}, they were renowned chemists and board directors in different companies in Germany. So, these people, the young people at this time, thought that we should do something with this problem. The problem of the past, of the history of our grandparents. On the one hand, if you observe the rise of the interest in the Holocaust, it was a, so to say, natural reaction from the side of the right-wing radicals to deny the Holocaust. Then, in the 70ies, they got some financial problems, some newspapers and journals published by the Nazi organisations disappeared. And now, I believe, we can observe again the wave is going up – perhaps because of the internet. If, in the 70ies, there was a problem with publishing the journals, then now, in the era of internet, its costs almost nothing. 
And there is perhaps also reaction on an idea of a united Europe. As for Brussels, we should rather think of ourselves as Europeans, to a certain extent, not only as Poles, Czechs, whatever, but we are also Europeans, the members of a European family, which share the same values, the same budget. So, there is a reaction of the people who don’t like this concept of a united Europe, who think ``Primarily, we are Poles. And we don’t want to be governed from Brussels, we got our own government, our own Polish Złoty, our own history'' And , of course, the problem of immigrants, the terrorist attacks in other countries, among Poles there is a tendency to talk about it in this way. ``Because our friends - England, Germany, accepted many thousand Arab refugees, and among them, there are of course terrorists, they got a problem. We don’t want to have any refugees on the Polish soil, so, in the consequence, we haven’t got the terrorist attacks and bombs in Polish cities. Therefore, we are right.''

\textbf{And how is this connected to Anti-Semitism?} 

\textbf{Piotr Setkiewicz:} Because, if you believe that we are the better ones, that our values are better because we are Christians, or Europeans, it’s just one step to not liking the Jews because the Jews, they got too much money, they are directors of different banks, they control the world bank system, and the Jews are in primary leftists, they promote the leftist values, so, because we are nationalist to a certain extent, we don’t like to accept these outer values. There is Mr Soros, who is a Jewish billionaire, and who is supporting the leftist organisations in Poland – on the one hand we don’t like him because he is leftist, and because he is Jewish. It is not far from this concept of national unity, of ``Because we are united, we represent the united nation, all Poles are brothers, and when we see these poor waves of Ukrainians, coming to Poland to work in different jobs on the construction sites, in the shops...'' Unfortunately, there is – quite human, I’d say – a tendency to look at them as people who are not us, of course, and who represent some lower cultural standards. Surprisingly, when Poles are going to the Western countries to work, they hear these anti-Polish jokes. There are very bad anti-Polish jokes, aren’t there? It is not nice to listen such things, but now, if we are looking at these Ukrainians, who come to Poland, at least some Poles unfortunately believe ``Now, we are in the positions of the British, we can look at these Ukrainians from a certain higher level'' It’s difficult – there is the question of policy, the question of a lack of education, and such tendencies to look at other countries as an enemy, as the potential threat, are a result of the modern nationalist eruptive at the end of the 19th century, and it continued. There is a model of patriotic education in the schools, in many countries, of course, to cultivate remembrance of national heroes. In many countries, there is a tendency to talk about the history of these countries in such a way: That we were always good guys, that we are always attacked by the neighbours. We’ve got of course our heroes, nevertheless, we are still fighting for our independence, and it is necessary to remember that independence is the most important value. We, as nation, have to remember that ourselves in such a way. And there is also this looking at the strangers in a specific way, that we can accept a limited number of foreigners, but if they ever exceed a certain amount, that is a potential threat for our identity. We are Christians, we are not Muslims.
Nevertheless, Auschwitz is still present somewhere in the background of such discussions as a good illustration of what might happen if we would forget about universal values. When we see the strangers not only as the other people, but also as the representatives of the black forces of international conspiracies. And then, it’s very easy to cross the line, to be on the black side of the force. There is the question of reactions to the Nazism of German society in the 30ies, here, in the camp, the reaction on certain people who were not Germans, but who accepted the posts of Blockältester, the Capos – many Poles and even Jews here in Auschwitz who received a stick and tried to exercise force over prisoners, to beat them, because the prisoners could not work with efficiently, the stick is the best argument. These were symbols and pieces of man history that might have been used in the constant discussions about the condition of the modern world and the societies here. What should we do when we hear about certain atrocities, like a genocide in other countries? Is it possible to do such a thing? We heard about the massacres in Syria, or in Iraq, or in Sudan and such countries. What should we do? There’s not a single good answer for it. Nevertheless, if we talk about it in terms of constant discussions about human rights, for instance about the need to combat racism in many countries, Auschwitz is always somewhere, is being used as a symbol, as a turning point, as a milestone in such discussions. I was surprised when I saw once a textbook for history in Singapore schools. And there is a quite comprehensive chapter about Auschwitz, on the other side of the globe. If you talk about the history of Europe, Second World War, with, say, Koreans, or people from Argentine, most people say that they know very little about it. But in 99\% of all cases, they have heard the name of Auschwitz.
 
\textbf{Do you think that Holocaust denial is the most dangerous form of anti-Semitism nowadays?}

\textbf{Piotr Setkiewicz:} This is a part of the ideology - important, but not the only one. Of course, anti-Semitism is something much wider, it’s a popular belief, that the Jews are controlling the world economy, that, in some cases, leading politicians in certain countries have Jewish roots. That the governments are under the constant pressure from the national Jewish organisations. Holocaust denial is only a part of the ideology. But nevertheless, it is a very important one because anti-Semites believe that Jews are using the Holocaust as an argument to defend themselves, to blame other nations, that they wish to control the market in countries like in Poland, to go back to their property that was confiscated during the War by the Nazis, and after the War, it was taken over by the communist government in these countries. That’s a good illustration of the situation in Poland. Taken altogether, the Jews are members of the world conspiracy, and they are trying to use the argument of the Holocaust in their propaganda. ``We are poor guys who were persecuted in the course of centuries, and it’s virtually impossible that we can represent a threat for the others. We were the victims, not the oppressors.'' So, if we, the deniers, would destroy this argument that the Holocaust happened, it would be much easier to preserve our arguments against Jews. Therefore, the history of Holocaust occupied an important position in the ideology of anti-Semites. Including the ones in Arab countries – you can buy Mein Kampf translated into the Arab very easily in the larger cities in the Arab world. And these people believe in these claims by Hitler, that Jews are insects, that something must be done in order to remove this constant threat, the Jewish threat, the danger for the German nation, and so on. So, is it possible to find an example of an official institution in Arab countries that combats such crimes by Nazis? No! I never heard about documental initiatives in Egypt or other Arab countries against deniers, against the movies or the internet, YouTube and Facebook activists. There is the question of, of course, politics, and the question of the legacy of the Arab-Israeli wars, etc. In Europe, the situation is different, nevertheless, the result is sometimes the same: The anti-Jewish propaganda, which is in most cases combined with fighting with the history of the Holocaust. 
But, if you ever have the opportunity to go to my archive, to the museum for a couple of days, I’m able, probably, to persuade the most stupid denier that Auschwitz happened. It would require some work, some time, nevertheless, that’s possible. I believe that we have enough arguments, enough documents. The problem is that of course, there is nothing like the one single document, the order written by Hitler saying ``Today, I decided to solve the Jewish problem in Europe, and to kill all Jews, Adolf Hitler''. Such a document does not exist. But, after carefully analysing, say, 200 documents, even the most stupid denier would be persuaded by the force of the arguments. Because, even assuming that the documents were falsified, they would have to recognise that it could not be done in such a way. There is, for instance, a document in which the commander of a certain unit in Auschwitz tried to give his opinion about his sergeant, promoting him before the high military ranks – some guards received military medals of their services in Auschwitz. He explained that the sergeant, was very active and very enthusiastic in collecting clothes during the action of evacuation of Jews. This means that in the result of this action of evacuation of Jews, the Jews were losing their clothes. Or, if we got the orders for the drivers of the truck. The trucks were bringing Zyklon-B from the factory in Erfurt. The problem was that Zyklon-B was used in Germany on a wide scale for delousing purposes, for killing insects. In one order we’ve got a sentence saying that the driver is going to Erfurt and the reason of the trip is delivery of Zyklon-B. Okay, why not. Somebody might say that Zyklon-B was used here, and that is the truth, for delousing of clothes. But there is a second order in which the reason of the trip is that the driver should bring the ``material for special Jew treatment'' or ``\textit{Materialien für die Judenumsiedlung}'', materials for resettlement of Jews. Assuming that everything was okay with the Zyklon and that the Zyklon was always used for delousing purposes here – so why, in other documents, did the German administration use such complicated euphemisms like ``materials for special treatment''? Why they did not simply name this as Zyklon-B, or as material for disinfection? My answer is that because this particular clerk who wrote the document, he knew that actually, a large quantity of Zyklon-B was used not for delousing purposes in Auschwitz, but for killing people. In this way, he wished to hide the truth. \\
So, if you give me, say, ten hours, I’d be able to persuade the strongest denier, I believe. Excuse me, I’m afraid I have to go to for the central ceremony in the city of the International Holocaust Remembrance Day.

\section{Dr Michał Bilewicz}
\textit{Michał Bilewicz (*1980) is an Assistant Professor at the Faculty of Psychology of the University of Warsaw since 2008. \\
He studied sociology and philosophy in Warsaw between 1999 and 2003. His Ph.D. thesis, completed in 2007, carries the title `'Between self-verification and social identity processes: Social psychology of threatened ingroup status'. 
He has worked in New York and Delaware in the USA and in Jena, Germany. \\
His research interests, among others, are conspiracy theories, prejudices, intergroup conflict, the threat of positive social identity and dehumanization, mainly on the example of xenophobia, anti-Semitism and ethnic conflicts, reconciliation mechanisms after genocides, the influence of cognitive mechanisms and public language on the exclusion of minorities.
He is the vice president of the Forum Dialogue Foundation, that is dedicated to fostering the relation between contemporary Poland and the Jewish people. \\
The interview took place in the Faculty of Psychology of the University of Warsaw on January 30th, 2018.}
\vspace*{2em}

Parts of the transcript as well as the edition are still missing!
\section{Bartosz Gajdzik}

\textit{Bartosz Gajdzik is an educator at the ``Grodzka Gate ‐ NN Theatre'' Centre in Lublin since 2012. Through collecting archives of Jews from the Lublin area, composing exhibitions to commemorate particular Jews of Lublin, individuals and families killed in the Holocaust, educating, conducting workshops, and publishing, he is participating in ``repairing the world'', as in the Jewish concept ``Tikkun olam''. Mr. Gajdzik is responsible for organising workshops for secondary school students and also for teachers, often mixed groups of Poles and Israelis. He is focused on propagating Jewish culture among Polish students and also fighting stereotypes in Polish society. He studied philosophy.\\
The interview took place at ``Grodzka Gate ‐ NN Theatre'' Centre on January 31st, 2018.}\par
\vspace*{2em}
\textbf{Bartosz Gajdzik:} First of all, we still have a problem with anti-Semitism in Poland. I’d like to say a few words about anti-Semitic attacks against this institution. It was attacked a few times by anti-Semites. They printed some anti-Semitic posters about it, even though all 55 people who work here in full-time jobs are not Jewish. We are in a municipal institution where Polish people, inhabitants of Lublin work to commemorate the Jewish community that lived here before the Second World War. It’s very meaningful that those attacks were prepared by anti-Semites, but against Poles, not against Jews.\\ 
Our institution exists thanks to local people. I’m trying to show you the two sides of this situation. On one side local people and the local government decided that we cannot understand the history of Lublin without the Jews who lived here and that this is important. If we don’t want to be ignorant, we must remember the Jews who lived here. Before the Second World War, one third of the population was Jewish, and for example at the beginning of the 19th century, Jews made up more than half of the population. For this reason, we cannot understand the history of our town without the history of Jews.\\ 
It really doesn’t matter that we aren’t Jewish. But for some groups, it’s the reason for attacking us. There were not only posters. The director of this institution, Tomasz Pietrasiewicz, was also attacked. It was a personal attack, someone broke the window of his kitchen in his private house. Those people used bricks with swastikas, it was a very strong symbol. It happened a few years ago. The police started an investigation and it took a few months for them to find those people. One of the men who was in this group was a full-time worker of the printing house of the State Museum at Majdanek. I want to show you this paradox: a man who works in a memorial site, in the State Museum at Majdanek, can also be an anti-Semite. Reality is very complex. On the one hand, we have inhabitants of Lublin who are trying to remember Jews in this town. On the other hand, they can be attacked.\\  
We co-operate with almost every school in Lublin, we conduct workshops. When I come to a school as an educator, for example in the lesson of Polish language or history, students usually don’t know what will happen, they are not prepared. At the beginning, I introduce myself and then I’m trying to introduce the topic, ``we’re talking about Jews''. When I say this word in Polish, \textit{żydzi}, sometimes students start to laugh, as if this word was a bit shameful, or some kind of bad word. It shows the many stereotypes which students have about Jews: I just said that we were talking about Jews and they started laughing as if it were some funny topic. We can just imagine what kind of stereotypes they have in their minds. 

\textbf{Have you encountered any anti-Semitic behaviour among students of secondary schools, except for laughing at the word ``Jew''?} 

\textbf{Bartosz Gajdzik:} It’s not so often about their behaviour. Sometimes I ask: ``Why are you laughing? What kind of associations do you have with this word?'' Then they explain that they start to laugh because sometimes people use the word ``Jew'' to imply that someone is very rich and doesn’t want to share their money with others. Usually, we talk about the reasons of this kind of behaviour and we talk about stereotypes. Then, I try to show them that of course we have stereotypes, which can be negative and positive. For example, there is a stereotype about Polish readiness to help or Polish hospitality, or about Germans, that they are very good workers. Of course, it’s again a stereotype and reality is always much more diverse and complex than stereotypes. 

\textbf{Can you name any positive stereotype about Jews?} 

\textbf{Bartosz Gajdzik:} It’s usually something between positive and negative, there’s this stereotype that they can make a good business. It’s something in between, of course, it’s good to be a good businessman, but on the other hand, there’s a negative side of this stereotype.

\textbf{One of the Jewish students in Kraków told us about so-called ``positive'' anti-Semitism\footnote{See interview with Serhii Czupryna.}: Some Poles reckon that Jews are good lawyers or doctors, but there is a slight negative side in it. They admire them as good workers, good craftsmen, intelligent ones, but at the same time, there is some kind of envy. There is a paradox of ``positive anti-Semitism''.} 

\textbf{Bartosz Gajdzik:} Yes, that’s the paradox. Now, we are in the gallery which is dedicated to the Righteous Among the Nations. I think this is a good occasion to talk about anti-Semitism in the countryside. We co-operate with the Yad Vashem Institute. The workers of this institute try to collect documents about non-Jews who helped Jews during the Second World War. All testimonies we recorded here are given to the Yad Vashem Institute. Sometimes, we have some information that someone somewhere in the Lublin Region helped Jews during the Second World War and we are trying to find those people. We knock on their door, but probably even if they helped Jews during the war, they don’t want to talk about it. Because of their neighbours, unfortunately. For those people it’s kind of shameful: they helped the Jews during the Second World War, which means that they maybe got money because of that, and they are afraid that their neighbours will say so.\\ 
Besides, we must remember that it was Nazi law that anyone who helped Jews during the Second World War was treated like a traitor. We don’t know exactly how many people, probably a thousand Polish people were murdered in Poland because they helped Jews. Sometimes the Nazis murdered not only the family who helped, but the whole village. It was a risk for the whole village. Maybe those people don’t want to talk about it also because it was a risk not only for them, but also for their neighbours. We don’t know exactly, but at the end, what was wrong about it? 75 years after the Second World War it is still a problem to talk about this. We still have problems with history, with memories.\\  
Now, let me tell you about the reasons of anti-Semitism. I think the first reason is that in communities and societies there is always a kind of identity, a process of building the community identity for which we need to find an enemy. It was not necessary that those enemies were Jews. But some people can build their identities only if they know who the enemy is. I know my enemy, so I know who I am. Unfortunately, in my opinion, in these days they can be also refugees.\\ 
The second perspective, in my opinion, is a conflict between Christianity and Judaism. In this case, religion is the reason. I think that this conflict was the most important reason in this part of Europe until the end of the 19th century. After the war, the reason of contemporary anti-Semitism in Poland, in my opinion, is Jewish properties. Many Poles don’t want to remember that Polish people after the war started to use Jewish properties and that before the Second World War, 3.5 million Jews lived in Poland. All of them had homes and private properties. Usually during the Second World War and in the aftermath, some building were destroyed, some not; usually Polish people started to use these properties. It’s very difficult for Jews to take them back right now. It’s a good solution not to remember that Poles used Jewish properties and that’s the moment when we start to use all these anti-Semitic stereotypes: the Jews were bad people, they were like aliens, and still they want something from us, they want our money, they want our properties, they want our house. We don’t talk about it in public. 

\textbf{Do you think that the Jews who emigrated from Poland to Palestine or Israel claim to get back their properties?} 

\textbf{Bartosz Gajdzik:} It depends. Usually not. It’s very difficult, it’s even impossible. After the war the communist state in Poland didn’t allow to take back private properties. It was possible only for religious communities, for example, it was possible for the Jewish religious community or the Catholic religious community – take, for example, the building of the Chachmei Lublin Yeshiva. After the war, there was a medical university in this building. In 2002, the Jewish religious community took it back. This is possible just for religious communities. It also works for Polish farmers. After the war, huge farms were divided into small pieces and it was also impossible to take it back for Polish people. So, this law was not only against Jews, it was sometimes also directed against Polish people who were rich.\\ 
Recently, we are in strained situation. I mean this discussion about the relations between Poland and Israel and the law which is devoted to keeping the good name of Poland. It has different aspects. The official statement is that somewhere in the world, people use the phrase ``Polish death camps''. Of course, it’s not true, it’s a huge mistake to use it, but in this new law there’s nothing about ``Polish death camps''. This new law says, more or less, that everyone who will say that some complicity lies with Polish people regarding the Nazi crimes can go to prison for three years. But sometimes the complicity of Poles in some particular crimes, Jedwabne for example, is clear - it happened. Israeli people and other people who survived feel afraid if they've had some bad experience during the war because of Polish people. This law is not only for the Polish, but also for people from abroad. It’s a huge mistake. I think that we cannot change history and we cannot change history through a law. We must use educational tools, first of all. 

\textbf{Do you think there’s a real danger for Polish Jews today because of that law? A few days ago, we made an appointment with a Jewish woman who lives in Kraków and she was really scared. The day before, she went to Auschwitz, she heard the speech of the Israeli ambassador, and she was afraid of what would happen to them right now.}  

\textbf{Bartosz Gajdzik:} If they feel fear, that’s enough. It was not necessary to start this process. For example, yesterday I spent half of the day here at my work explaining to people from Israel what was going on here because they had many questions: ``Can we visit Poland right now?'' They feel really afraid. This law is not so clear, we don’t know exactly what it means. For example, during discussions in parliament, there was a question about the latest book of Jan Gross about Jedwabne. There was a question: is this book against the law or not? Some vice minister answered that it depends on whether the author talked about facts or not, if he was a researcher and if he used the correct tools or not. Who will decide about this? No one knows.\\  
I also want to say that in my opinion, we cannot talk about anti-Semitism without education about different kinds of exclusion and stereotypes. Anti-Semitism is only one kind of exclusion. We must learn not only about the exclusion of Jews, but also about exclusions of different nations, of women, and about homophobia. In my opinion, there are the same reasons that form the basis of all these exclusions. People cannot accept that reality is very complex and diverse. We cannot change these diversities, we must accept the diversity of reality.  

\textbf{Usually, visiting groups from Israel have many guards with them. What do you think about it? Is it necessary?} 

\textbf{Bartosz Gajdzik:} That’s a good question. Quite often, Polish students ask me why. It looks like they need security because they feel afraid of us. Is it true? Israel is a very untypical country because security is one of the most important matters in Israel, even inside the state. The level of security is much higher than here, in Poland, in Europe. The groups of Israeli students have security guards if they go somewhere as a full class, even inside Israel, when they travel to Galilee or to the Dead Sea from Jerusalem. They have these security measures because they live on the territories which are occupied. They have them not only in Poland, but also in Germany, in France, in Israel, wherever. So, it’s not because they feel afraid of Poles, but because they feel afraid of terrorism, wherever they are. Of course, it can be misunderstood.  

\textbf{Getting back to the past, I would like to ask about villagers who obviously had some anti-Semitic attitude because of Catholic priests’ teaching. How would you call a person who is anti-Semitic because of the Catholic Church and at the same time saves Jews’ lives? There were such cases. Would you call the person anti-Semitic or not?} 

\textbf{Bartosz Gajdzik:} Yes, I would because of their thinking. Many people who were anti-Semites helped Jews during the Second World War because they felt that it was something obligatory for Christians. Even if I have stereotypes about Jews, even if I hate them, I must do it. It means that anti-Semitic thinking and behaviour are not the same thing. It can cross, as in the case of Zofia Kossak-Szczucka. She was a writer, she wrote many anti-Semitic texts, but she helped Jews, many of them survived thanks to her.  

\textbf{Can we compare it to today’s immigrant crisis? If a person is against keeping refugees in their country, but is Catholic?  I can donate some money to support refugees in Syria for example, so I help them, but I don’t want them here.}  

\textbf{Bartosz Gajdzik:} Yes. In my opinion, it’s also our Polish contemporary paradox. Again, stereotypes of Poles, they consider themselves very helpful and hospitable. They also like to remember that Polish Righteous Among the Nations are the biggest group. Usually, when we talk about the Holocaust in Poland, the mainstream only claims that Poles are the biggest group among them. It’s the most important thing. Except for that and the fact that the death camps were not Polish, it’s not necessary to know anything. A few years ago, there was a poll in high schools in Warsaw and there was the question: ``What do you think about relationships between Poles and Jews during the Second World War?'' 80\% answered that the Polish people helped Jews often or very often. We must remember that the group of people who helped Jews was really small, probably it was less than one percent. We know about seven thousand Righteous among the Nations from Poland. On the one hand, it’s 25\% of all the Righteous, but on the other hand, this percentage is highest in proportion because here in Poland lived the biggest Jewish community. For example in Latvia, there are not so many Righteous, firstly because there were not so many Jews there. Around fifteen million Poles lived in Poland before the Second World War. When we compare fifteen million Poles to seven thousand Righteous, what is the percentage? But right now, students in schools, not only in Warsaw, think that during the Second World War Poles, as a whole nation, were very helpful to Jews. I don’t know if the same group of students wants to help today. They don’t want to have refugees here. On the one hand, they regard Poles as very helpful in the past, but on the other hand, nowadays it doesn’t work anymore. So that’s also the paradox. It’s a good example of using experiences from the history in contemporary challenges. It’s not only history: anti-Semitism, the Holocaust, Shoah.  

\textbf{Do you see any chance to reduce anti-Semitism in the world and do you have any ideas?} 

\textbf{Bartosz Gajdzik:} Of course, education and meetings can help, because many people have anti-Semitic stereotypes even if they haven’t met any Jew. I imagine this Catholic priest Weksler-Waszkinel. A few years ago, he told me that he was in hospital, he was standing in line, waiting to see a doctor. As usually, people were talking to each other and someone said that we have a problem with doctors in this country, with the health service because there are always some Jews in the government and they do all the troubles to us. He asked: ``Jews? What do you have against the Jews? I’m a Jew. Did I do anything bad against you?'' This man was confused, because probably he hadn’t met any Jew before. I think that such meetings are the best way to recognise stereotypes and to break them. That’s also the reason why we organise meetings between Polish students and students from Israel. They can understand that in the modern world everyone uses Facebook, everyone listens to the same music, and people more or less are very similar. It’s also good context to understand that many stereotypes about Jews are just not true.  

\textbf{What are the main interests among students when you organise workshops? Is it rather anti-Semitism or the Jewish culture?}   

\textbf{Bartosz Gajdzik:} It’s the Jewish culture. There are of course some students who are anti-Semites and want to talk about anti-Semitism, but usually they just know nothing about Judaism, about the Jewish culture, so usually they have questions like, ``why those kippot on Jews’ heads, why the mezuzas on a front door?''. They need explanations, they have some knowledge, but also some false opinions, and they didn’t have a chance to ask these questions before. Usually, we prepare students before a meeting with Israelis. Having the image of a Jew in black clothes in mind, in a black hat, with those funny elements on the both sides of heads, they ask me if the Jews look like other people, just like me and you.  

\textbf{How did your personal interest in this theme develop?} 

\textbf{Bartosz Gajdzik:} I don’t know exactly, probably there are several reasons. The deepest is in my personality. The first reason is this priest Weksler-Waszkinel. He was my professor of philosophy. He talked a lot about Jewish philosophy, about the history of Jews, and I started being interested in Jewish philosophy from the beginning. I graduated from philosophy at the university and I wrote my thesis about the Jewish philosopher Emmanuel Levinas. He was a Jewish Frenchman who lived in the 20th century. When I was a student ten years ago, I rented an apartment with my friend in a building more than one hundred years old. One day, I met some people who were just looking around there, as if they were searching for someone. It was the family of this Jewish boy, whom you saw on a photo in the exhibition, Henio Żytomirski, and he had lived in this very building. Of course, I didn’t know this. I invited this Israeli family to my place. In this building, there were four apartments and they didn’t know exactly in which one they had lived, they just knew the address Szewska 6. Neta Żytomirska, the lady who was a cousin of Henio, said that she was in this building ten years before, but she was afraid to come inside because of her experience, probably some memories from the past. But when she met me, she felt she could enter. We became friends. Always when I’m in Israel, I visit this family. But then, ten years ago, I just asked myself: What’s going on? I live in the building where Jewish people lived before, I don’t know anything about them, anything about Jews. I know there was a concentration camp at Majdanek, of course, I know that before the Second World War, Jewish people lived in Poland, but actually I was ignorant. I just felt that I should change it. In 2008, I came to this institution for the first time. I still was a student, I started co-operating as an educator. They trained me how to conduct workshops in mixed groups of Poles and Israelis. Then I started in a part-time job and since 2012 I work here full-time. For me, it’s the mix of some reflections, some personal experience, and history. I just felt that this is something very important. I don’t have any Jewish roots. 

\textbf{What about professor Weksler-Waszkinel? He is not a priest anymore, is he?} 

\textbf{Bartosz Gajdzik:} Officially he is, but he’s out of the Church in Poland, he doesn’t work as a priest. His bishop decided they didn’t need him; it’s actually his problem. The Church in Poland and also in Israel allowed him to move, but they also didn’t need him, so he could do whatever he wanted. He doesn’t have a church to be a priest in Israel. It’s very personal for him, if he’s a priest or not, I know he had some internal dialogue between those two identities.  

\textbf{Do you know perhaps how he is perceived at the Catholic University of Lublin? Especially with his view of John Paul II’s teaching, as he is the patron of the university.}  

\textbf{Bartosz Gajdzik:} I know that he felt bad at the university. It’s hard to discuss feelings, he started feeling like an alien at the university among his colleagues. He’s a priest, but he’s a Jew. They didn’t accept his identity and he decided to retire and they accepted it. In his opinion, they didn’t want him to stay longer at the university, even though he wasn’t forced to leave. On the other hand, in Israel he is like an alien as well. There are not so many Catholic priests or people who used to be priests in Israel. I met him in Israel in Decembe, and he told me he decided to stay in Israel because many people have some trouble with their identity there. There are people from all over the world and in this context, in this heterogeneous society, he feels similar to the majority of Jewish survivors.  
\section{Teresa Klimowicz}

\textit{Teresa Klimowicz graduated from Paideia - The European Institute for Jewish Studies in Stockholm and from the Hochschule für Jüdische Studien in Heidelberg. She obtained a PhD in philosophy in Lublin. For more than ten years, she has been professionally involved in Holocaust education and multicultural education with her NGO ``The Well of Memory'', which is based in Lublin. She also works at the ``Grodzka Gate - NN Theatre'' Centre, a cultural institution in Lublin, where she researches the Jewish history of Lublin. \\ 
The interview took place in Lublin on January 31st, 2018.}\par
\vspace*{2em}
\textbf{Teresa Klimowicz:} I think it's really intriguing how anti-Semitism serves as a function in the society throughout the ages in general and I guess it could be interesting to see whether it has the same function in all those different contexts that you are researching. I need to give you a little bit of background about who I am. I'm not an expert on anti-Semitism, I haven't been doing academic research on it, but I've been involved in Holocaust education and multicultural education in general for over 10 years in my small NGO, which is called the Well of Memory, from here in Lublin. For a couple of years now, I've been involved here at the Grodzka Gate centre where similar activities are taking place. I'm researching here the Jewish history of Lublin. I should also say that I'm not Jewish and that I am leftist. This is my perspective.\\
Poland has experienced the whole spectrum of anti-Semitism, starting with religious anti-Semitism. Since Poland is a Catholic country, that would be not necessarily Polish anti-Semitism, but rather Catholic anti-Semitism. This is based on blaming the Jews for the death of Christ.\\
Religious anti-Judaism later developed to become a more national concept, maybe in the 19th century when the concept of the nation was invented, so to speak. This birth of national identity also included here in Poland the aspect of anti-Semitism. I think the interwar period of Poland of the 20s and 30s would be the proper example for what was going on. Another aspect would be what was going on during the Second World War and the Holocaust and another aspect is something that was part of the post-war history of Poland, so anti-Semitism based on the association of Jews with communism, what we call in Poland żydokomuna. And then another aspect is what is going on today. Until two weeks ago, I would probably have said that the function that anti-Semitism used to play for the Poles is now exchanged for general xenophobia and in particular, in my opinion, Islamophobia, so I think that is becoming a bigger problem than anti-Semitism in Poland. Through the very recent events, we see however that anti-Semitism is being awakened again in Poland.  

\textbf{What about the period between the two world wars? What about the plan to resettle Polish Jews to Madagascar?}

\textbf{Teresa Klimowicz:} These were the ideas of the right-wing parties in Poland. They were in the parliament, but not in the government. There was discrimination on different levels. A famous example is the so-called ghetto benches, so Jews were supposed to sit in a segregated place in class at university. There was numerus clausus at university, so only a certain number of Jews would be accepted. In the 30s, there were pogroms, and the independence of Poland was also followed by pogroms. In the 30s, there is a general wave of anti-Semitism all over Europe. I think it's important to remember that in many of those cases, it's not something particular for Poland. Both this Catholic anti-Semitism and this wave of anti-Semitism in the 30s are part of a larger problem with local variations. \\
I think the figure of ``the Jew'' in Poland in general serves as the figure of ``the other'' that helps to define a national identity. That's why it can be easily exchanged for something else, some other ``other''. Right now, it's refugees or Muslims or Ukrainians, especially here in Lublin because this is an important minority in this region. And whenever there is an issue that evokes this need for referring to ``the other'', the figure of ``the Jew'' also. Dealing with Jewish heritage and Jewish history myself, I cannot say that I've experienced any violent anti-Semitism. It’s rather a sort of myth based on some nuances of the language, like when a bird shits on the window of your car, you can say that you have a Jew on your car. Problems of this kind in the language were also researched by Professor Tokarska-Bakir. What has been going on in Poland for the last couple of days is the confrontation between the Polish version of the Second World War and the Israeli or Jewish version of the events. When you look at today's newspapers, all those myths about the Jews appear except for the one in which the Jews made matzah out of Christian blood: ``Why do the Jews blame the Poles for the Holocaust? They are ungrateful, the Poles were rescuing the Jews, we have the biggest number of Righteous among the Nations. The government gives a lot of money for the restoration of the Jewish cemetery in Warsaw. Why don't the rich Jews pay for their cemeteries themselves? Also, let’s not forget that the Jews were part of the oppressive system of communism, especially in the 40s.'' So we have the rich Jews, we have żydokomuna, and we have Jews as ``the other''.  

\textbf{You're talking about the new law to protect the good name of Poland.} 

\textbf{Teresa Klimowicz:} We had a law protecting the good name of Poland before. The new thing is to penalize accusations against Poles. If somebody claims that they took part in the crimes of the Third Reich, for example. 

\textbf{Like Jan Gross?} 

\textbf{Teresa Klimowicz:} For example, but there is a debate whether it would fall under those articles. Anyway, I think this kind of ideas is derived from Jan Gross’ book ``Neighbours''. That was the book about Jedwabne that initially sparked this whole debate, together with several other books. Interestingly enough, the crimes at Jedwabne were also researched by the Institute of National Remembrance. They admitted that Poles participated in those crimes. The new law is connected to the same institution that actually admitted this. They try to make it about facts, but it's not about facts, it's about myths and the ways we want to remember things. There is a conviction of Jewish conspiracy in Poland where you don't need to be Jewish to be called or seen as a Jew. When you’re successful and powerful, that could be enough.\\ 
Also, there have been competing versions of the past between the Jews and the Poles because the Poles see themselves as victims above anything else. A couple of years ago, I did an international project with people from Germany, Ukraine, Israel, and Poland. When we had a workshop about the position of each nation during the Second World War and the Holocaust, everybody wrote that they were victims. That was a very interesting experience. The Germans hesitated for a moment, but still, and everybody else had absolutely no doubts about the fact that they were victims of the war.  

%\textbf{Was anti-Semitism used in Poland historically as part of the creation of a Polish identity that says that Poles are Catholics and Jews are something separate?} 

%\textbf{Teresa Klimowicz:} The process of the creation of nations started late and then need you some reference point. I would say in Poland, it was mainly the religious denomination creating the difference. The Ukrainians were Orthodox, the Poles were Catholics and the Jews were Jews.  

\textbf{How did anti-Semitism develop after the Holocaust?}  

\textbf{Teresa Klimowicz:} Maybe Lublin is a good starting point because Lublin was the first city to be liberated by the Red Army already in July 1944. Lublin became a centre of the re-establishment of Jewish life in Poland. People that had survived in the Soviet Union and in the surrounding shtetls or camps would come to Lublin even before Warsaw was liberated. During the Warsaw Uprising, Jewish institutions were already being created here in Lublin in August 1944. 

\textbf{For example, Abba Kovner with his Nakam group was also in Lublin.} 

\textbf{Teresa Klimowicz:} Right. There were Jewish committees that moved to Łodz and Warsaw after the whole of Poland was taken over by the victorious army. They became like a base for self-government within communist Poland. In the beginning, the communists had an idea of creating a more open society. Their politics towards minorities in Poland had evolved. Their statements in the beginning were to provide full rights to the Jews in this newly established political system. But this very soon changed into a persecution, which was represented by the centralisation of different institutions. It was not applied only to the Jewish community, though. Some historians even claim that the Jewish community was privileged in communist Poland over the Ukrainian community, Łemkowie or other minorities. There were certain Jewish institutions functioning, for example the central Jewish committee, which later became a secular institution. That was cancelled in 1950 and another Jewish institution was created, the Towarzystwo Społeczno-Kulturalne Żydów w Polsce, the so-called TSKŻ. Parallelly, there was a religious organisation called Kongregacje Wyznaniowe. These two institutions were functioning in Poland throughout the whole communist time.\\
On the other hand, in the 40s in Lublin region, there were still pogroms, especially in Parczew, but there were also attempts at pogroms here in Lublin, so many people would flee from this area, first of all because a bigger part of the Jewish community were refugees from the Soviet Union. They were just in transit to other parts of Poland or other parts of the world. Still, it's important to stress that Lublin region was, apart from Kielce region, the area where most of the pogroms of the 40s occurred. Another important point is the anti-Semitic campaign of 1968, which is also a case of competing memories because for the Poles, March 1968 is a glorious event of student strikes and the formation of the opposition, whereas in the Jewish experience, it's mainly a factor contributing to another wave of forced emigration. On the one hand, this anti-Semitic campaign was a political play between different factions of the leading party and on the other hand, it was a propaganda wave in newspapers all over Poland, also in local newspapers like here in Lublin.  

\textbf{Was the background of the discourse of emigration the Six-Day-War?} 

\textbf{Teresa Klimowicz:} It started with the student strikes. In Warsaw, they banned the performance of a play by Adam Mickiewicz titled Dziady, which is important for the Polish identity. It was considered anti-Russian, so it was banned. Then, the students of the University of Warsaw went on strike and protested the censorship. That's how the riots began. Within the party, there was a propaganda that was probably intentional, it's quite a complex issue. They were saying that those that started the strikes are Jewish or have Jewish parents, that they are banana youth, and that they should get back to their studies instead of protesting. Banana youth was a term in opposition to the workers because bananas were an exclusive product. Only those people from the intelligentsia, who were well-off supposedly, could afford them. This was used as an argument against the students, that they are pretty well-off, that they protest against a system that is created for the promotion of the working class, and that they are Jewish. Actually, they were not called Jewish, it was disguised as anti-Zionism. There were protests with people carrying flags saying ``Zionists to Zion''. Supposedly, these demonstrations were organised by the party.\\ 
This is the main narrative of historical writing today: On the one hand, there were the students that represent true Polishness, on the other hand, there are these people inspired by the governing party holding anti-Zionist flags. All the newspapers would also refer to the military conflict in the Middle East because it all started in the international background of the Six-Day-War. Several layers of issues were in picture in March 1968 because of the political situation in the world and the involvement of the United States and the Soviet Union in the conflict in the Middle East. Poland had to take the same position in that situation as the Soviet Union and it became connected with those student protests against censorship.\\ 
Now from the Jewish perspective, hundreds of people were forced to leave the country. There was a famous speech by Gomułka in which he was saying that we are happy to give emigration passports to those people who consider Israel their fatherland. This was a credible statement that they should leave. They were given one-way passports, so once they had left, their Polish passport would be taken away and their citizenship was cancelled. Again, the Poles saw it in a different way because many people wanted to leave Poland and they couldn't, so in the testimonies that we gather, I also hear that people thought that the Jews were lucky to leave, privileged even. Even I, when I’m writing about this March 1968, I have a problem to put it together and to create one narrative because there seem to be two different experiences, the glorious one for the Poles and this fear of expulsion, of being rejected for the Jews.\\ 
This was truly a nail in the coffin for the functioning of Jewish communities. Usually, we think it was the Holocaust that ended the Jewish presence in Poland, but I think it happened in March 1968. Today we don't even know how many Jews live in Poland because the statistics differ a lot. According to official statistics where you declare your nationality, there are about 1000 people declaring that they are Jewish. This is not reliable, though, because many people have Jewish roots and don't consider themselves Jewish or they just don't want to admit it. The Jewish community of Warsaw, for example, counts 700 people. In Lublin, we don’t know either because we don't have a separate Jewish community, it's a branch of Warsaw Jewish community.  

\textbf{Is there such a thing in Poland as not considering Jews as a part of Polish society?} 

\textbf{Teresa Klimowicz:} Yes and no at the same time, I think. There is a huge debate about that on the example of the POLIN Museum of the History of Polish Jews in Warsaw, which is supposedly showing the history of Polish Jews. Those who criticize it say that it's rather presenting the history of Poland with some Jewish aspect. The problem is again that there are competing memories between the Jewish experience and the Polish experience concerning events both hundreds of years ago and more contemporary ones. On the other hand, there has been, I believe, deliberate politics from 1989 onwards to include Jewish heritage into the narrative of many places in Poland, little towns, shtetls, to confront them with their Jewish past. 

\textbf{That’s very interesting because, for example, in Latvia there is an opinion that Jews are separate. It’s very difficult to explain to people that the Jews that died during the Holocaust were actually neighbours, they were citizens of Latvia. People just don't accept this fact. How would you deal with it, how would you explain to people that those people were not strangers, but neighbours?} 

\textbf{Teresa Klimowicz:} In fact, this is what I'm doing with the NGO ``The Well of Memory''. We have a project supported by the Ministry of Culture and National Heritage, which is currently right-wing. We have a sizeable amount of money now to work with Jewish cemeteries. I have a feeling that I have found the key to talking to people in different places in the province and to convince them that the Jews were part of the history of their own town and had connections to their families. Going to the Jewish cemetery, engaging the school, the local government, the local media, and so on, bringing attention to those places, it seems to work. Pointing out to people the abandoned Jewish cemetery close to them and telling them that it's their responsibility to take care of it, I think this is something that can work and unblock things. This is my ideal.\\ 
I think it's also important from the Israeli perspective. When they come here, I can tell them about our common heritage. In Israel, they also have a different narrative, they are focused on the new state in Palestine, so they are not being educated that much about the heritage they had in the diaspora. Sometimes, I must explain to them important Talmudists of Lublin. They don't have a clue about that because they are not necessarily religious, either. 

\textbf{Is there such a thing in Poland as justification of the Holocaust?}  

\textbf{Teresa Klimowicz:} I don't think so.  

\textbf{Is there such an idea that Jews got what they deserved?}

\textbf{Teresa Klimowicz:} I know that's what some people in Israel think, I've been talking about that with my Israeli friends recently, but I don't think this is the general feeling of the Poles. I think with the Holocaust it’s rather about this competition of victims, so they were the victims, we were also the victims. People want to hear an acknowledgement of general suffering of the Poles, I believe. I don't think missing knowledge about the Holocaust is a problem in Poland, either. Remembering my own education, living in Lublin and in the shadow of Majdanek, so to speak, I think the awareness of the fact that the Holocaust took place and that it was important is clearly there. The problem is rather the connection to the fact that the Jews that died during the Holocaust were the neighbours that lived in the same street of Lublin. The narrative of the Holocaust was created in a way centralised around Auschwitz mainly. Even the camps of Aktion Reinhardt, which are in Lublin region, are still forgotten in many ways. That's why it became an abstract idea, not something connected to the place you live in. Another Lublin example: There was a murder in Lublin of a guy that survived Bełżec, Chaim Hirschmann. He was killed on the streets of Lublin in the 40s and now the debate is whether he was killed by the Polish Home Army. Lublin was already liberated then, so the question is whether he was killed because he was Jewish or because he was thought to be a member of the secret police.  

%\textbf{What is the main narrative?}  

%\textbf{Teresa Klimowicz:} I cannot say that there is a main narrative because there are two competing narratives. When I guide and talk about that, I always say both versions because I don't know what was in the head of those that were shooting. 
\section{Dr Sonia Ruszkowska}

\textit{Sonia Ruszkowska is an educator at the POLIN Museum of the History of Polish Jews since 2013. She has participated in the educational program of the main exhibition from the beginning. Mrs. Ruszkowska is responsible for the educational programme of the museum directed at schools. She is specialised in drama theatre workshops and anti-discriminatory education with a special focus on civil rights and the fight against anti-Semitism. Her educational programmes are mainly aimed at children, youth, and teachers. She studied philosophy and wrote her doctoral thesis about the Holocaust, the death in the gas chambers, and the possibility of bringing back subjectivity for victims of the mass deaths. Before she started her work at the museum, she worked as a schoolteacher for philosophy and ethics and participated in the organization ``Forum for Dialogue'', which conducts projects with school children in different towns and cities in Poland, raising awareness about their Jewish history. We met her at POLIN Museum on January 29th, 2018.}\par  
\vspace*{2em}
\textbf{You told us about ``Forum for Dialogue''. Is it only devoted to small towns and villages in Poland or also to bigger cities? As an educator, do you see a greater need of educating in smaller towns?} 

\textbf{Sonia Ruszkowska:} There are programmes of ``Forum for Dialogue'' in bigger cities like Białystok or Warsaw, but generally the project is run in smaller towns. I think the need is the same in smaller and bigger cities, but in bigger cities there are more educational programmes available for students.\\
These kinds of programme are very necessary because kids do not know the Jewish history of their towns at all, they have no idea that Jews were living there. If they know, they don't know what it means because they know nothing about Jewish culture. But they are very interested. If we go there and start talking about the topic, they are really.\\
In their homes, many times they encounter very anti-Semitic perspectives, so it’s harder for them to really go beyond that, if they never meet somebody for whom Jews are positively connoted. 

\textbf{Where do these anti-Semitic attitudes come? Are the parents anti-Semitic because their parents were anti-Semitic?} 

\textbf{Sonia Ruszkowska:} That’s a big topic. We know from history that anti-Semitism comes for example from the Christian-Jewish relationship, but also from economic rivalry. We show that in our exhibition. From an individual perspective, I meet mostly pupils and they say, for example: ``My grandma heard that there’s a Jewish workshop in the museum and she told me not to bring my wallet.'' This kid was laughing, like, ``oh, what is grandma saying'', so this is already okay, because he trusted us, assuming that we are Jewish if we make the Jewish workshops. This is not the case for me, for example, but there is this assumption. They hear it at home, from their grandparents and parents. Very small children, who do not really think in a conscious way, have very bad assumptions about the word ``Jew''. At the beginning of the workshop in the Museum, I normally ask them: ``What does it mean that somebody is a Jew, what does this word mean?''. Many times they said: ``Oh, a Jew is a person who does not want to share with others''.\\ 
In Polish, it’s ``skąpiec'', ``greedy''. I always ask myself: ``How does this six-year-old child know these stereotypes already?''. This is in the language because the Polish verb ``żydzić'', which comes from ``Żyd'', ``Jew'', and means ``being stingy''. So it’s basically the language that gives them these stereotypes, but I also have the impression that generally, in Polish society, there is a very bad association with Jewish culture, with Jews. Of course, there are always people who are fascinated by Jewish culture. 

\textbf{How we can fight these stereotypes in your opinion?}

\textbf{Sonia Ruszkowska:} In my workshops, I use this method called Nonviolent Communication (NVC). I try to give empathy to a person and not to judge this person, like: ``You’re stupid, you’re anti-Semitic, how can you say these things?'' I try to understand why this person is saying this. Mostly, people want to feel safe in the end. If they don’t know something, they have this association of danger, for example, that their values are in a way endangered. So I try to understand this person, what this person is saying, and then convert it into a language of needs ― what does this person need? Then, when this person really feels that I’m listening to him or her and is calm, I can say how it is for me. I can say that I know many Jews, many of them are my friends and I would like people to respect them. Of course, I can also say that some Jews can do negative things as everybody else, but generally, for me the respect is important. I can say that I wrote a book about the Holocaust and know many testimonies, so it’s really important for me not to make jokes about the Holocaust. So, I try to make it personal.\\ 
Another thing is that people need knowledge because they don’t know many things and have only some images. For example, this topic that is always coming back that Poles were so great for Jews during the war and how Jews could say that Poles were murdering Jews. If people do not know that before the war there was a lot of discrimination against Jews in Poland and a lot of tension, they will never understand why Poles reacted the way they did during the war, why these tensions were even greater because of the occupation. If people learn things and understand things, they are more open.\\  
Another thing: I believe in meetings. If people have only images and they have never seen any Jewish person, they have really only an image. So, Israeli-Polish meetings for example, even if they last only for one hour, can really change something. What we do here is that we want to give a positive association with Jews and the Jewish culture. Not like ``Jews means only Holocaust and Antisemitism – everything which is negative or difficult'', but also to show life, to show joy in Jewish culture, tradition and all these positive things. We want to show both. 

%\textbf{Many interview partners have said that with the internet, there is much more anti-Semitism and hate speech and it’s spreading. What do you think normal people could do against that? Writing back comments is usually not that helpful.} 

%\textbf{Sonia Ruszkowska:} I invented an anti-discriminatory workshop for youth, which is about hate speech, for example on Facebook. It is a drama workshop, so there is a hero that is the target of hate speech and others who use hate speech. People have to play both parts and then we discuss and try to invent a solution: what they could do in this situation. I believe in transferring the online situation to a real situation between people because this way I feel that we can do more. In both cases the conclusion was that it’s always good to search the people who can support you, who can help you. The person who is using hate speech doesn't have the whole power, it’s mostly like one or two people and there’s a lot of people in the class or in the group that are bystanders. If you can bring them to your side or ask them for help, in many cases they will join you.  

\textbf{How common are anti-Semitic stereotypes in Poland?} 

\textbf{Sonia Ruszkowska:} I think everybody knows them. Every little child knows this anti-Semitic perspective, somehow. I’m not saying that children consciously think like this. Not everybody says that they are true of course, but the knowledge is common. I think in smaller towns, many people really think this way, also because of the trauma after the war. In many little towns, people live in ``post-Jewish houses''. It is too hard for them to really confront themselves with this story, to really work with it. It’s easier to forget about this, but this conflict stays. It always makes you frustrated, so anti-Semitism is maybe the way to express this frustration against Jews.\\
I think anti-Semitic stereotypes exist, but it’s not necessarily the case that when Jews are in the public space in Poland, there would be some act of violence against them. It is more about saying to other people that Jews are for example greedy. In big cities, there are some small communities that are fascinated by Jewish culture. There are people coming to our museum for every festival, every concert, or event. Many of them are well-educated or interested in culture in a general way. There are also people who just don’t care about Jews or Jewish culture. 

\textbf{You said that pupils do not know much about Jews. Is that part of the compulsory education?} 

\textbf{Sonia Ruszkowska:} You can teach something in school, but children do not know everything that’s in their books. It’s only a small part of the curriculum and it depends on the teacher. Of course, there are teachers who are interested in the Jewish history. It is possible to talk a lot about this history during lessons, but if you don’t want to, you can do only one lesson about this. Then there is the Holocaust. The Holocaust is really present, but you can also teach it in different ways: is it about this Polish heroism or is it more about the Holocaust? Mostly, I think, Jewish history in the school programme is connected to war. It creates such an image: there are no Jewish people, then we see, out of the blue, there are many of them, and then they are murdered. That is basically the image that children have after school education. Now there is also a lot of Islamophobia in Poland. These two things are combined If I ask children about Jews, they often say that the book of the Jews is the Quran and that Jews are Muslims. Really, it’s very common now. The Jew is the alien, now the Muslim is an alien, so Jews are Muslims.  

\textbf{During the Holocaust, there were Polish people who were helping Jews to survive. You can read a lot about that and there are whole exhibitions in the museums about that. But there were also collaborators, Polish collaborators. Is this also well-known? Is it easy to do research on that topic?} 

\textbf{Sonia Ruszkowska:} It’s a very interesting story because there are many scientific books about it. If you want to know about it, it’s no problem. But no, people don’t know about this. Students don’t know about it at all. We have the workshop about Jewish-Polish relationships during the war, I made a film with material from the USC Shoah Foundation Archives, and there are people talking about their deportation to ghettos. They told about the reaction of Poles during these deportations, when they were entering the ghetto. They said that people were laughing, people were doing bad things to them, they were cruel.\\ 
We show this short film to students and they are always shocked. They had no idea. Of course, it’s not that every Pole during the war was doing such things. There were many Poles that were sad and full of compassion during the deportations of Jews to the ghettos. But there were many people reacting with cruelty.\\ 
In fact, I think I realized that only during my studies. Nobody ever told me about this in my whole school education. Personally, I remember this moment of realization very well and I felt that they lied to me all my life. I finished school in Warsaw, humanistic classes, and nobody ever told me about this. Especially in primary school, it was always ``Poles ― the heroes, great''. Then I discovered it’s not true, it’s a lot more complicated. This positive image of Polish people is everywhere and the truth is not so beautiful. It’s kind of hidden in the consciousness of society, but the knowledge is really there, if you want to see it. 

\textbf{Do you think the government wants everybody to think that the Poles have helped the Jews, but no Poles have helped the Nazis kill the Jews? Is it difficult to do research in that direction? Does the government control what is shown in the museums?} 

\textbf{Sonia Ruszkowska:} Yes, I think so. I think there is such a danger because people who do research need grants and the government has influence on who will receive these grants. There is always this question how much the researchers will be dependent, which university will get more money, which less, and so on. It’s never really objective, there’s always some interest in there. 

\textbf{Why do you think normal people don’t want to admit that Poles didn’t help enough?} 

\textbf{Sonia Ruszkowska:} It’s not nice to see yourself like this. 

\textbf{But why do you see yourself immediately as one of them? It’s just a fact, nothing personal.} 

\textbf{Sonia Ruszkowska:} It is personal because they identify with a nation. You want to have the good image of your nation. So many people say: ``It's not true that Poles were not helping enough, only Jews say this.''\\
There is a good book about this by Grzegorz Niziołek, ``Polski Teatr Zagłady''\footnote{``The Polish Theatre of the Holocaust''}. It’s about the theatre spectacles about the war that were made just after the war. He writes from a psychoanalytical point of view. He claims that the bystander position is very difficult because it’s somebody who sees what is happening, who feels guilty that he or she didn’t do anything, and who is afraid because he thinks, ``I can be the next one''. He's also happy that it’s not him who was killed and then he feels guilty because of this joy. There are many emotional layers. It’s a very difficult position. There is a big, big guilt. The situation is not black and white. With Jews and Germans, it’s somehow black and white. Germans started the war, they were murderers and Jews were victims. Poles are kind of in between. They are victims, but they are also murderers and they are bystanders.\\ 
The second thing is that many Jews have also no true idea about the Polish history of the war. Many of them never heard about the Warsaw uprising, for example, and about the fact that Poles were actually fighting against Germans, that so many Poles – non-Jewish Poles – were killed. These two narrations about the war (Polish and Jewish) are so different and neither of them is entirely true. I think the truth is very complex, it’s not at all black and white. Poles want to see themselves as total heroes, totally purified people who only helped Jews. Jews often think that Poles didn’t help them, that they were perpetrators. I think the truth is somewhere in between, but somehow people prefer to have a clear image. Why do people simplify the truth? That’s a philosophical question, it’s very interesting. Psychologists say that ambivalence is the most difficult feeling or state of mind. We prefer clarity. 

\textbf{I can say from the perspective of a teacher that in Polish education, there is still not enough time to teach about everything in detail. If you've only got several hours to devote to the Holocaust, you would prefer to talk about ``good Poles'', not about ``bad ones''. Probably, that’s why students are not well educated about the bad sides of Poles. Do you agree that it might get uncomfortable for teachers because of the lack of time and the comfort?} 

\textbf{Sonia Ruszkowska:} I don’t really agree, it’s an excuse. I think it’s just a difficult topic. Teachers maybe do not feel really prepared to teach about it because they would have to really think about it personally, make peace with it in themselves. It took me many years to really dig in and to really feel acceptance with this topic. It’s adifficult topic and teachers also need education about it. In the museum we make many workshops for teachers about the Holocaust. If you have only one hour to teach about the Holocaust, maybe you could devote half an hour to tell about Polish heroism and half an hour to tell about Poles murdering Jews. 

\textbf{As you said, it’s a good excuse. I wouldn’t like to do it. I would avoid being accused by parents of my students that I said something about bad Polish people during the war.} 

\textbf{Sonia Ruszkowska:} Yes, of course. We in the museum are in a privileged position in this respect because we don’t have contact with children’s parents. Sometimes teachers also say that we show too much of Poles’ ``bad side''. So, we always try to say that it’s only one perspective and that there is also different one. But I think that children will hear about this positive side of Poles anyway, about the Righteous among the Nations, it’s a hundred percent certain. I would prefer to say other things which they might not know about. 

\textbf{What would you say about the development of anti-Semitism? Does is get less because people become more educated or does it get stronger because the Holocaust happened seventy years ago?} 

\textbf{Sonia Ruszkowska:} I think there is a possibility that anti-Semitism will grow because if there is more Islamophobia and homophobia and all this kind of attitudes, it is always connected with anti-Semitism. I don’t know why, but it happens. I think it will grow. Besides, anti-Semitism is now mixed with anti-Israeli reactions. This is also very complex topic. I think the situation in Israel will not be very calm in the near future, I fear, so it will also not be the element that can help. I don’t see anything in the big scale that would really help. 
 
\textbf{I had a discussion with a roommate and he told me that he doesn’t like it that little kids learn about the Holocaust. He thinks that it will do some harm to them. At what age, do you think, can we tell a child about the Holocaust and how can you start telling these things?} 

\textbf{Sonia Ruszkowska:} In the museum, we start these workshops in fourth class, so the kids are around ten years old, and we do it with literature, not with images. I think it’s more about the question of how to teach it, not if we teach it. I also had an interesting discussion about this with friends from Israel and they said, ``we teach about the Holocaust very early, very little children.'' In Israel, there are a many institutionally organised days about the Holocaust during the school year, so these children will hear about it anyway. It’s only the question if they do it in an appropriate way and prepare them for it.\\ 
I think in Poland it’s a bit the same. If you walk around in Warsaw, for example, every few metres there’s a plaque saying something like, ``here, fifteen people were shot by Germans''. So you cannot really avoid the memory of war, it’s really everywhere. Children know it very early, probably earlier than in other countries, so I think it’s okay to teach them, but in a wise way, not with images of naked corpses or other horror things. Even if they say they want to see it. In fact, whenever we go to the interwar period gallery with small children, they say: ``Oh, we want to go to the war, we want to go to the war!'' We say, ``no, you will do it when you’re older'', and they react: ``No, we want now, we want to see it!'' It’s like a computer war game for them. Of course, they don’t understand what the Second World War really was.\\ 
I think it should be a process. First you talk about values and tell children about individual stories. Then there are more and more details. I think, for example, that you should only take people to Auschwitz at the end of high school. For me, you shouldn’t talk too early about gas chambers and the mass death. It’s so horrible that people should be prepared for it to really understand it.  
\section{Dr Katrin Stoll} 

\textit{Katrin Stoll is a Holocaust scholar from Germany who has lived and worked in Warsaw for many years. From 2015 to 2018 she worked at the German Historical Institute in Warsaw (in the research group "`Functionality of History in Late Modernity"'). She is a member of the German-French research group "`Early Modes of Writing the Shoah: Practices of Knowledge and Textual Practices of Jewish Survivors in Europe"' (1942–1965). As a member of this group she retrieved and safeguarded the Nachman Blumental Collection at the University of British Columbia in Vancouver in 2018 and ensured that 32 boxes containing Holocaust-related material were shipped to YIVO in New York. \\
Her research interests include anti-Semitism; Holocaust historiography and testimonies; Täterforschung; criminal prosecution of Nazi crimes in the Federal Republic of Germany; representations of the Holocaust in Germany and Poland. \\
We met Katrin at the German Historical Institute in Warsaw on the 30th of January 2018.}\par 
\vspace*{2em}
\textbf{Katrin Stoll:} Can I maybe start with the following: On Friday, a new law was passed by the Polish Sejm, by the parliament. It states: “Whoever accuses, publicly and against the facts, the Polish nation, or the Polish state, of being responsible or complicit in the Nazi crimes committed by the Third German Reich… shall be subject to a fine or a penalty of imprisonment of up to three years.” What we are witnessing at the moment is an attempt to regulate history from above, through legal measures.  What is at stake for the Polish government is the so-called good name of Poland and the Poles. This law was passed on Friday, one day before International Holocaust Memorial Day, the day of the liberation of Auschwitz on 27 January. It’s an interesting development: the fact that the crimes were committed in the first place are not the cause of outrage. I mean, I think that everybody knows that it was the SS who established the concentration and extermination camps, and that the Holocaust was a German Nazi state crime, that’s pretty obvious, but the very fact that these crimes were committed does not produce outrage. What evokes outrage and fear among high-ranking bureaucrats of the Polish state is that they might be falsely ascribed any responsibility for the crime. This is their fear: That they, before the world, appear as the ones who committed this crime. It’s an attempt to regulate history from above. The experience of the Jews is of no importance whatsoever in this kind of history politics. It is very disturbing. I mean, the first priority should be to reflect of what happened at Auschwitz instead of thinking of one’s own narcissistic feelings. 

\textbf{Do you feel that this fear is substantiated in any way, for example when articles regularly refer to Polish concentration camps, will it lead people to believe that these were connected to the Polish nation?}

\textbf{Katrin Stoll:} As I said, only people who are completely ignorant could come to the conclusion that Poles established the Nazi concentration camps. The Nazis established Auschwitz for Polish political prisoners. Auschwitz is the symbol of the murder of the European Jews. The real issue here is to regulate public discourse and to suppress a free discussion about the whole dimension of the Holocaust. The Holocaust was not only a German state crime, not only a confrontation between Germans and Jews. It was also a confrontation between the non-Jewish majority communities and the Jews, in the case of Poland under German occupation: between the Christian Poles and the Jewish population living in this country. One third of the population in Warsaw was Jewish, the largest ghetto was in Warsaw, 350,000 Jews were deported to Treblinka by the Germans in the summer of 1942. However, the majority of Polish Jews lived in small cities and shtetls. The Germans carried out the deportations of the ghettos in a very cruel manner and shot many Jews on the spot in front of the eyes of non-Jewish Poles who largely accepted the genocidal project of the German occupiers. It was a visible, public crime, visible to the Polish population, to Polish society, so the question is, how did they respond to that?  And the real issue is the involvement in the crimes and the role of the “participating observers” (Elżbieta Janicka). This is what the advocates of Polish nationalist history politics are trying to regulate: the discourse. They don’t want to speak about certain things. We have certain avoidance discourses like the Righteous discourse, it’s like a fig leaf. Jan Grabowski speaks of ‘Righteous defence’. The nationalists always talk about the Polish Righteous in order not to talk about difficult subjects, like anti-Semitism, for example, and the participation of sections of Polish society in the Holocaust. no sane person would ascribe responsibility to the Polish nation as a whole because it’s always individuals who act. We’re talking about individuals, individual participation or individuals acting as part of an organization. Sometimes several individuals got together to act, like for example in Jedwabne where Polish neighbours murdered the entire Jewish population – men, women, children of the town – in 1941, burning them alive in a barn. Since 2015, when PiS came to power, the highest leaders of the Polish state have openly negated the fact that Polish citizens murdered the Jews of Jedwabne. I think this is the direction of this Holocaust-speech law, it is directed against Holocaust recognition, i.e. the active and passive participation of parts of Polish society in the Holocaust. The ‘Polish camp’ issue is not the real issue, because, as I said everybody knows that the SS established the Nazi death camps. 

\textbf{Don’t you think that there are a lot of journalists and influential people all over the world – as you said, ignorant people – the ones who spread the idea of Polish, so-called Polish concentration camps, for example in Italy. So probably, it’s maybe not the main, but an [important] idea of this law, that we should stop it, because this mistake is spread over the world. And there are many people [who] have no clue about Poland, about what happened during the War, and if they listen to or read journalists who say ‘Polish concentration camps’, they create a kind of idea of Polish nation at the time. So, probably, it’s still necessary to say ‘not Polish, but Nazi concentration camps’, I don’t think it’s just a minor issue.} 

\textbf{Katrin Stoll:} I didn’t say that it was a minor issue, I said that the real issue is evaded. The real issue is to ask ourselves: Why were these crimes committed? How was Auschwitz possible? Why was there no solidarity with the Jews? Apparently, it’s no big deal that we have ‘a little Auschwitz’, ‘a little Holocaust’?! Why are we not talking about what happened at Auschwitz, why are we not talking about what happened in the Warsaw Ghetto, in Treblinka? For people, the ‘Polish camp’ subject is an issue because they want to avoid a confrontation with the reality of what happened during the Holocaust, this is what I’m saying. And I think it’s an obsession: [For example], the Polish foreign ministry has established the term ‘false code of memory’. “Polish concentration camps”, according to them, fall into this category. Of course, it’s false, but what matters to them is that their good name is kept clean. I think the real issue is avoided.

\textbf{But do you think that in this law, the Polish government wants to negate what happened in Auschwitz?} 

\textbf{Katrin Stoll:} No, of course, nobody negates it, nobody would negate Auschwitz. But it’s rather: ‘I don’t want to be connected with these crimes in any way’, and there is the issue of Jedwabne, and of the attitude of Polish majority society to the persecution and murder of Jews. The Poles were forced to become eyewitnesses and co-presents of the Holocaust carried out by the German occupiers and they benefitted from the crimes.  “The lie of Jedwabne” has become official state doctrine, the head of the Institute of National Remembrance (IPN) has publicly propagated the idea that the Germans murdered the Jews of Jedwabne, not the Poles. Again, we are talking here about avoidance and defence mechanisms. 

\textbf{Would you say that it’s completely avoided? In history politics, are there some political measures that deal with the victims and take responsibility for them?} 

\textbf{Katrin Stoll:} Yes, I mentioned the Centre for Holocaust Research in Poland, they have done a lot, published a lot, also testimonies that actually focus on the experience of the victims, Jewish victims, and the difficulties of surviving on the so-called Aryan side. We have to get the facts straight. Now, what are the facts? The Germans murdered 90\% of the Polish Jews – 3.5 million. Out of the 3.5 million Jews, only 30,000 to 50,000 Jews managed to Survive in the German-occupied territory. It is a very small number. And after the Germans had liquidated all the ghettos and murdered most of the Jews in the Nazi extermination camps – in Bełżec, Treblinka, and Sobibór – there were approximately 200,000 Jews still alive – out of them, only 30,000 to 50,000 Survived. So, the question is – what happened to them? And, the Polish Holocaust researchers – Barbara Engelking, Jan Grabowski and others – have studied what happened during this so-called third phase of the Holocaust. Of course, nobody negates that the Nazis were the main perpetrators. But in order to make the Holocaust so terribly effective, the Germans depended on parts of the Polish population. Take, for example, the Blue Police under German command: the police participated in the deportations of Jews, Jan Grabowski has written about the so-called Blue Police. This is what the Polish Holocaust researchers have studied. In my view, the Polish government attempts to block this self-critical and analytical Holocaust discourse and to marginalize the Polish school of Holocaust research.  

\textbf{Is there a connection between such a, well, let’s call it one-sided view on the history of the Holocaust, and the topic of anti-Semitism? How is it connected?} 

\textbf{Katrin Stoll:}This is an interesting question. Yesterday, the Polish president, Andrzej Duda said that Jews have the right to fight anti-Semitism, and that Poles have the right to fight the slandering of the Polish nation and the good name of Poland. You can see that a false symmetry is being created here, as if anti-Semitism and anti-Polishness were the same thing. Anti-Polishness is a trope that appeared in Polish discourse after the anti-Semitic violence against Jews after World War I. People who condemned the pogroms was reproached with being “anti-Polish”. Maria Janion and others have pointed out that in the history of anti-Semitism, the ‘Jew’ has been portrayed as the intruder, as a state within the state, as the one who is not part of the German nation, the Polish nation any nation in Europe. This is an integral notion of European anti-Semitism: the Jew as the intruder, and, in the Polish case, very often, the Jew as the non-Pole: There is a concept of the Pole that says that a Polish person is only a Catholic person (polak-katolik). But anti-Semitism, as I understand it – there are different definitions – is based on false projections, it’s something that happens in your mind, it’s a phantasmatic concept. Perpetrators of anti-Semitic violence respond to a certain image that has been created and has circulated in their cultural tradition. [And there are different variations:] For example, Nazi-anti-Semitism was a redemptive anti-Semitism, the Nazis wanted to redeem the world of the Jews. In Polish anti-Semitism, we have a strong tradition of Christian anti-Semitism. German anti-Semitism led to Auschwitz, Polish anti-Semitism to pogroms. But we have a certain mental structure that we always find in anti-Semitism. And this is, I think, the conspiracy aspect. In the anti-Semitic worldview an endless power is ascribed to Jews, the power to rule the world, to undermine things, in Christian anti-Semitism, the power to kill God, in modern anti-Semitism the power to rule the world by means of capitalism, socialism, communism, liberalism. We have to understand that it is a mental concept that functions relatively independent of what Jews do or don’t do. Anti-Semitic images can be activated in certain situations. You have probably heard about the nationalist march, last year, on the 11th of November in Warsaw. This manifestation was organised by the so-called Radical Nationalist Camp, ONR, an organization that refers to pre-war fascism. 60,000 people marched behind nationalist, racist banners like “White Europe” and so on, and one person – he was on TV – was asked “Why are you participating in this manifestation?”, and he replied, “In order to remove Jews from power”. This is a classic example of an anti-Semitic worldview, the idea that it is the Jews who rule the world – it doesn’t matter if there are real Jews in the government or not, the anti-Semite reacts to an imagined Jew.\\
It’s interesting to see then how this was discussed afterwards: A lot of people argued that we cannot say that all these people who participated in this manifestation were fascists. But the very fact is that by marching behind these banners they agreed to that, and they knew that it was the ONR who organised this manifestation. I have noticed – I’ve been living in Poland for nine years now – that there has always been an attempt to differentiate between patriotism and nationalism [a case in point is Adam Michnik in his book “Kościół, lewica, dialog” , 1977]. It’s the idea that aggressive nationalism is bad, and that patriotism is good. But sometimes you can see that these things intermingle. And there is an article – I think, 256 of the Polish penal code – which makes hate speech a criminal offence, but the problem is that this article is hardly ever applied, even though the Polish president officially condemns these racist statements. We have an official distancing, but if we look a bit closer, we can see that, for example when it comes to criminal persecution, cases have been closed. For example, there was one case against Justyna Helcyk, she’s an ONR leading figure in Silesia, and this case was closed, she intervened and the Minister of Justice, who is the chief public persecutor at the same time, made sure that the case was closed. And we have several of those cases, maybe you have heard of the case of Piotr Rybak, who publicly burnt an effigy symbolizing a religious Jew. He was sentenced, but later on, the sentence was reduced. So, we have a double-bind situation: Officially, the Polish government distances itself from right-wing extremism, but when it comes to the legislation and the prosecution of these crimes, they are lenient. They don’t see a connection between their policies and the rise of right-wing extremism. A survey demonstrated recently that 30\% of Polish men support this fascist organisation ONR. For me the puzzling question is – all of this is happening after the Holocaust – that the ONR, an openly fascist organization, is not outlawed in Poland, that they have freedom of speech.  

\textbf{But on the other hand, there are criminal cases against Jan Gross, the Polish Holocaust historian. He was interrogated five hours, and there is still a criminal case.} 

\textbf{Katrin Stoll:} Yes, the criminal code contains an article whereby those who slander the good name of Poland can be prosecuted. What was his ‘crime’? Gross wrote an article in which he said that Polish people had killed more Jews than they had killed Germans during the German occupation of Poland. He argued that the present-day stance of Polish society towards the refugees has its roots in the fact that Polish society has not dealt with its own role during the Holocaust and the murder of Jews. Jan Gross seeks to take Poland, the self-declared Christ of nations, from the cross. This is the real reason why he is being prosecuted and portrayed as a vampire.  

\textbf{You mentioned this article by Maria Janion, and she writes about these intellectuals who seem to spend their time fantasising about Jews, demonising them, reading anti-Semitic pamphlets and writing them anew. I was wondering if that kind of activity was only done by intellectuals, or if writings also had an impact on general society? Or was there like a two-layered transmission of anti-Semitism – you have a kind of intellectual anti-Semitism and you have this general population anti-Semitism?}\par
\textbf{And how are these writings perceived today?}\\ 

\textbf{Katrin Stoll:} I think we have to look at the Polish Catholic Church here because anti-Semitism was an integral part of Catholic Church teaching. So, when we ask, “what ideas did the Polish peasants have?”, we can say that their idea of the Jews was not shaped by what the intellectual elite wrote. I don’t know how many of them were able to read. Their idea of “the Jews” was shaped by the Polish Catholic Church. For example, that the Jews were the Christ-killers, this anti-Semitic idea. I think that you can say that the population in the 18th and 19th century was more influenced by religious anti-Semitic teachings, not so much by what intellectuals wrote, like Krasiński and others. And, regarding your second question, I think there has not been enough distancing from these people and their writings. Take Staszic, for example, the Polish Academy of Science is named after him. I think, when it comes to that, a lot has still to be done. The deconstruction of, let’s say, the negative side of Polish culture. This is what Maria Janion1 analyses, she wants to understand what the characteristic features of Polish culture and anti-Semitism are. According to her, anti-Semitism is an integral part of Polish culture She wants to understand why people need the anti-Semitic figure of the ‘Jew’ for the construction of their own identity. 

\textbf{And were the clergy of the Polish Catholic church in turn influenced by these intellectuals or did they just always transmit their idea of Jews as the Christ-killers and that’s it? Or did they also fantasise?}

\textbf{Katrin Stoll:} I haven’t studied this period, it’s not my field of expertise, the end of the 19th century, so I cannot say anything about that, but I can give you an example of a more recent case, the anti-Semitic campaign in Poland in 1968 for example. It was an official state campaign; the highest officials of the party unleashed this campaign. And the consequence was that approximately 14,000 Jews or those considered as Jews by the authorities left Poland. And the Catholic Church in Poland did not condemn this campaign thereby approving of it. And there are cases where anti-Semitism functions in political discourse, for example in the presidential elections in 1990, when there were the candidates Wałęsa and Mazowiecki, and Wałęsa suggested that Mazowiecki had Jewish roots. And what did the Mazowiecki campaign do? Instead of naming this as anti-Semitism they propagated that way back to previous centuries Mazowiecki’s family was of ethnically Polish origin, that there was no Jew in their family. It is very disturbing that they did that. I think that the subject of the Catholic Church as well as anti-Semitism within the Catholic Church is a great taboo in Polish discourse. I am convinced that if the Catholic clergy who in the countryside were the main authority during World War II and the German occupation, in 1941, when the pogroms happened, I think that if the Catholic clergy had intervened and said “We are not going to kill the Jews” that people would have listened to them. I think that this subject – the stance of the Catholic Church towards anti-Semitism and the Holocaust – has not been dealt with enough. 

\textbf{But let us not forget about hundreds of Poles who helped Jews, and children, babies, who stayed alive, and the whole villages saving Jews. You cannot forget that, can you?}

\textbf{Katrin Stoll:} Nobody is forgetting about that. To the contrary.  

\textbf{I mean, the role of the Church is not only negative, there is also some positive aspect of their acting and they also helped thousands of people – Jewish people.} 

\textbf{Katrin Stoll:} Nobody denies that. I’m just saying that anti-Semitism has been an integral part of Christianity in general and the Catholic Church in particular.  

\textbf{One Polish historian told me that Polish people were very extremist: They were very extremist in collaborating with the Nazis, and they were very extremist in helping the Jews.}\par 
\textbf{But what is discussed about widely is the bad extremism. What I think is that the bad extremes are mostly shown. And those who helped, we don’t talk about them...}

\textbf{Katrin Stoll:} Can I maybe say something about this issue of help. The official Polish state discourse – for example during the anti-Semitic campaign of 1968 – portrays the entire Polish nation as rescuers of Jews. But we have to understand that those few people who rescued Jews were an absolute minority, they were exceptions, complete exceptions, and afterwards people blow this out of proportion and say these people were representative of the Polish nation. They were not! They were not representative, they represented only themselves, they acted against the norm, the societal norm. Under the German occupation, rescuing and helping Jews was not considered an act of resistance against the German occupiers. The helpers also had to hide Jews from their own neighbours, not only from the German occupiers. We have to differentiate between real people who helped and between the discursive figure of the Polish Righteous in public discourse. This discursive figure, which is detached from real people, always appears in public discourse in Poland when there is anti-Semitic violence. The Polish Righteous were brought up in 1946, after the Kielce pogrom, they were brought up in 1968, they were brought after the Jedwabne debate in order to cover up the anti-Semitic acts. Irena Sendler for example said that after the Jedwabne debate a hero was needed. We have to ask ourselves: What is the function of the discursive figure of the so-called Polish Righteous which portrays Poland as a nation of rescuers of Jews? 

\textbf{Another problem is that there are Polish Catholics who saved and hid some Jews, but they were not able to state it or to publish it. They were shamed by the mainstream after 1945.} 

\textbf{Katrin Stoll:} Yes, they were considered outlaws and they were persecuted in the immediate aftermath of the events. After liberation from German occupation they were in mortal danger. These people did not want to reveal [what they did], and this testifies to the fact that the majority disapproved of their behaviour. I have to say that I was a bit shocked when Morawiecki, the Polish Prime Minister, mentioned the Righteous in the context of Auschwitz. What do the Righteous have to do with of Auschwitz? Nothing, nothing at all, we are talking about a German Nazi concentration and extermination camp where so many Jews from all countries in Europe were murdered – the Germans deported Jews from Corfu in Greece to Auschwitz, and the only thing that this man can think of is the Polish Righteous? This is so absurd, so grotesque because they were no Righteous at Auschwitz. It was the site of mass murder of the European Jews. The Jewish victims don’t really matter in Polish discourse. Hanna Krall once said, when the Poles speak about the Holocaust, they always speak about themselves. 

\textbf{Why do you think that people always take this so personal? I mean, in Germany and Poland, when we talk about crimes of Poles, Germans, Latvians people take it personally even if it was so many years ago.}

\textbf{Katrin Stoll:} I think it’s an inability to face up to reality. I think that you can only bear reality if you can face it, and you cannot uphold a Polish identity of Christ among the nations, of Poles as the eternal victims, if you acknowledge that there were people in Polish society who murdered Jews, right? So, it’s not possible to create an unspoiled, clean identity. I think, this is why there is this strong reaction.  

\textbf{But this idea of the Christ among the nations doesn’t exist, does it?} 

\textbf{Katrin Stoll:} I would say that is what differentiates, let’s say, Polish nationalism from Hungarian nationalism, or Latvian nationalism, or Lithuanian nationalism etc. The characteristic feature of Polish nationalism is that Poland, in the Polish literary canon, has been portrayed as the Christ among the nations. It’s undeniable, and it has shaped people’s perception of reality and their behaviour, it has shaped the way people look at the world. I mean why is it so hard to get rid of nationalism? According to Pierre Bourdieu, the sociologist, it’s so hard to get rid of nationalism because we are talking here about dispositions, which are internalised, like bodily dispositions. You grow up with your education at school, you read specific texts, you build a certain understanding of the world based on ethnically homogeneous groups2. You can see how this works in Poland: During the so-called refugee crisis for example the head of the party PiS, Kaczyński, made openly racist statements. He said that refugees were carriers of diseases and so on, thereby winning the election. And if we look at this, we can see again that we are talking about phantasmatic concepts. There is no single refugee in Poland. But what he tried to do was to stimulate this fear of imagined refugees who undermine the religiously homogeneous nation. This was the enemy portrayal by Kaczyński. Again, if we want to understand how concepts like nationalism and anti-Semitism function, we have to look at the phantasmatic conceptions. 

\textbf{I didn’t want to negate that there is the anti-Semitic picture in Polish literature. But they asked one million Polish students whether they liked the poems and the drama, and they hate it. They do not identify with this idea of being a Christ of nations, not at all, not today.} 

\textbf{Katrin Stoll:}  I would say that the concept of the nation is still the main reference: that people don’t necessarily define themselves as human beings or as Europeans, but as Poles. And of course, you are right, there are different variations, but I think it’s still the main frame of reference in talking about so many issues in Poland, all kinds of problems are discussed under the heading of the nation. 

\textbf{Maybe, it’s again this thing that intellectuals read this kind of literature and they maybe identify with this idea of Christ among the nations, but regular people do not do this, they never ever read 19th century literature unless they are forced to at school, but still they are nationalist. But it’s transmitted in another way.} 

\textbf{Katrin Stoll:} Okay, I’ll tell you one thing. Poland has been part of the European Union since 2004. There has been a free Poland since 1989, general elections and so on and afterwards a liberal democracy in Poland. So, explain this to me: Why has there never been another narrative? A narrative based on an understanding of a nation based on citizenship, not ethnicity. It has not happened! And this is something that people in the so-called West do not understand, that when we talk about, let’s say, PO and PiS, the main political parties: they are both right-wing conservative political parties. For both parties the nation is the main frame of reference. There has been no other narrative of what it means to be Polish, or I haven’t come across that. The question is: Why? Why is there no narrative of Poland in the European context, or of what it means to be a European Pole, a Polish European? 

\textbf{So, do you think that Poland is a dangerous place for Jews? Cause you’re saying it like that.}  
 
\textbf{Katrin Stoll:} I’m not saying that. I think I would say it like this: It’s a dangerous place for anybody who does not fit a certain image of what it means to be Polish. It could also [be dangerous] if you’re a homosexual and do not fit into the idea of what it means to belong to a good Polish family. It’s a dangerous place for people who do not correspond to a certain idea of Polishness. I was extremely frightened, I have to say, [on] the 11th of November 2017 when, in Warsaw, which was completely destroyed by the Germans, where the Germans killed so many Poles and Jews, 60,000 people were marching under the heading of “white power”, marching through the city. During this fascist manifestation I had to hide in my flat because I felt that I could not go outside. You could say, “Okay, these are extreme people” but what is dangerous is the fact that the right-wing discourse has become hegemonic. The right-wing nationalist discourse has become the dominant discourse. I would always say that in Germany it is worse, you have Nazis in the Bundestag, we have attacks on asylum homes and so on, we have the National Socialist underground, we have the legacy of National Socialism, but the important, the dangerous thing is that the right-wing discourse is the dominant discourse, not only in Poland, in other countries as well.  

\textbf{In Poland, there was a very strong anti-Semitic movement, for example, in the 1930s, they wanted to expel all the Jews to Madagascar, in 1946, there was the Kielce pogrom. Is there a political historical discussion about this anti-Semitic past?} 

\textbf{Katrin Stoll:} Yes, among outsiders. It’s not mainstream Polish historiography which focuses on these issues. Cultural anthropologists have studies these questions, Joanna Tokarska-Bakir3 for example has written on the Kielce pogrom. I think the problem is not that studies do not exist or that the subject is not being studied. The problem is that these subjects have not entered official mainstream historical discourse and historiography. It’s not that knowledge doesn’t exist. The problem is that it does not circulate.  

\textbf{So – how could it trickle down?}

\textbf{Katrin Stoll:} It can only be achieved by means of education, as I said, you have to understand how anti-Semitism functions: it has nothing to do with reality, it is something that is made up and used for specific purposes You have to make anti-Semitism morally, politically, socially unacceptable. This is the disturbing thing: That history is repeating itself. Are we again at the end of the 1920s, when democracies collapsed in Europe and there was the rise of fascism and National Socialism, are we witnessing this again?  Why is it okay to be an anti-Semite after Auschwitz; what it means to be an anti-Semite after Auschwitz is to approve of Auschwitz, it’s to approve of the murder, the mass murder of Jews, an anti-Semite wants Jews to be dead.  

\textbf{How can we fight such an expression of anti-Semitism, justifying anti-Semitism and especially the Holocaust? Because for example in Latvia, there is [this myth, [it’s in the] media, that Jews were those who betrayed the state of Latvia, who supported the Soviet occupation How can I deal with this problem? Because when I try to tell people “It is a disgusting lie”, people think that I don’t know history, that I’m a Jew sympathiser...}

\textbf{Katrin Stoll:} Very important question it is also a problem in Poland, this anti-Semitic stereotype and a myth of żydokomuna, the merging of Jews and communists.  in Nazi anti-Semitism this was also very important element, this notion of a Jewish-Bolshevik world conspiracy. So, how to dismantle that? Again, you have to ask: what function does this myth fulfil in your country’s public discourse, who disseminates it, propagates it, which purposes does it serve? So, what would you? 

\textbf{Well, mainly, this myth is not officially supported, of course, you can not officially support it, but it exists between people, just ordinary people. The majority of Latvians think that Jews were traitors.} 

\textbf{Katrin Stoll:} So, we have another element here which is very important in the anti-Semitic world view, namely the notion of Jewish aggression: The idea that Jews do something against your nation, the Latvian nation in this case. The only thing you can do is to say that this is a myth, and to say that it is a phantasmatic construct. For the construction of this phantasma it doesn’t matter if there were real Jews who were communists and part of the Soviet authorities. The problem is that people imagine Jews in a certain way and imagine Jews doing things in a certain way. So, then we have to ask: Why? So, why is this particular notion so important for the Latvian identity? 

\textbf{ Maybe people just don’t want to see themselves as murderers. Because when it comes to the Holocaust, we can just say “Okay, Jews got what they deserved. We had the right to revenge”.} 

\textbf{Katrin Stoll:} This is a very important point: the anti-Semitic construct is used to justify one’s own crimes against the Jews. In the case of Lithuania and Latvia, we have certain groups and organisations actively helping the German Nazis murder the Jews shooting, murdering the Jews in the forest and elsewhere. In order not to admit to the crime itself, people simply say “We took revenge. Because the Jews were the first who did something bad to us, and we just took revenge and penalised them because they were supporters of communism.” We see here the false notion of double genocide theory, the notion of a ‘red Holocaust’ and a ‘brown Holocaust’, amounting to an equalization of Soviet policies or with the unprecedented Nazi persecution and extermination policy.  

\textbf{In the case of Latvia and Lithuania, Germans helped Latvians and Lithuanians to kill Jews, because mainly, neighbours killed Jews in the forest.}

\textbf{Katrin Stoll:} Yes. In September, I was travelling through Belarus and Latvia. I visited a lot of Holocaust burial sites and sites of mass execution in the forests, and it occurred to me that without local people, the Germans would have never found these spots and places. They needed collaborators.  But this is an important point: The notion of “Jewish Communism” served as a justification for anti-Semitic crimes. The same happened in Jedwabne, there it was also used as a justification for the murder of Jews. The perpetrators make use of an anti-Semitic idea in order to retrospectively justify their own acts of murder. We can name several crimes and study the mechanisms. And afterwards we have to deconstruct these mechanisms. 

\textbf{Would that work? I mean, if you deconstruct it, you show to people the false things they believe, you present them the historical reality. Will they listen or will the just continue to ignore it? Would it be an alternative to speak about the positive role that Jews did always have for the country, how they worked.} 

\textbf{Katrin Stoll:} No, that’s completely wrong, because again:  We are not talking about real people, we are talking about a certain notion of what a Jew is, what a Jew does, what a Jew thinks, what a Jew looks like, of a certain idea of being.  You have to destroy the concept, the narrative. And you can only do that if you are aware of the fact that it is a mental construct with the element of a conspiracy at its core. I mean, there are a lot of anti-Semites who have never seen a real Jewish person. I’m completely against this idea that you can fight any form of racism, anti-Semitism, homophobia, xenophobia, by having a real meeting with the people who are stigmatized because this is not how it works. 

\textbf{So, we were talking to the leader of the education programme at the Polin museum yesterday. And she said – especially when she goes to rural villages, she often meets children who have no idea what Jews actually are, but they just get an impression because of how the adults talk about Jews – for example, a greedy person who doesn’t want to share. And she said that it is really helpful for them to meet with Jews so that they can learn that these are all just stereotypes.} 

\textbf{Katrin Stoll:} This just shows that we have a completely different understanding of how anti-Semitism functions. Because, I believe, in agreement with Slavoj Žižek, that anti-Semites react to a certain image of ‘the Jew’ that has been circulated in their tradition4. The idea of the greedy Jew for example has circulated for centuries, and people have internalised this idea. And you have to destroy the idea. I mean maybe you are lucky, and, in this case, the person meets somebody and thinks “Oh, it’s a nice person” – but what happens if she meets another person and thinks – “This is a representation of the greedy type.” This is not how you are able to get rid of anti-Semitic concepts. 

\textbf{So, in your opinion, is there anything we can do about these anti-Semitic projections, in order to fight them?} 

\textbf{Katrin Stoll:} We can fight them by demonstrating that they are false projections. The Holocaust perpetrators projected all kinds of thing onto the Jews calling them traitors, aggressors, dishonest, greedy people. The perpetrators projected these notions onto a whole group, and this is also something that we have to understand about anti-Semitism and other forms of racism: The Nazis persecuted Jews as a group, they were persecuted independent of their social status. It did not matter if they had green eyes, blue eyes, if they were religious, non-religious Jews, if they were involved in a political party or not. For the Nazis ‘the Jews’ constituted a homogeneous group. Every individual was put in a certain group constructed by a Nazi mindset. If we want to fight these things, we have to start with language. Everybody has only one life and is an individual. And we have to teach critical thinking.   

\textbf{According to my personal experience, education can sometimes help against anti-Semitism. For example, I was socialised anti-Semitic in the kindergarten, and until I was 14 or 15, I was an anti-Semite. And afterwards, I read some books about the Holocaust, and I changed my mind.}

\textbf{Katrin Stoll:} Yes, in Germany, people discussion the question of whether it makes a difference if people visit memorial sites and learn about the Holocaust, if a visit to a memorial site is a way of protection against all kinds of bad things like racism, anti-Semitism and so on. I have my doubts because either you know that you don’t murder a human being or you don’t, and if you don’t know that, you won’t learn it by visiting a memorial site. It’s a decision that you make: in my view you need to become aware of the indoctrinations you have been exposed to so that you have the chance to distance yourself from them. I think, it’s a matter of awareness, of critical reflection, critical thinking… 

\textbf{For me, it was a matter of education, of information.} 

\textbf{Katrin Stoll:} Okay. But generally speaking, I don’t think that information helps. Let’s take the example of the Holocaust deniers.  These people are obsessed with details. Real Holocaust deniers know an awful lot about the facts themselves. For example, they argue that it’s not possible to kill so many people in the gas chambers and so on. You cannot fight Holocaust denial on the level of facts, this is impossible. You have to ask: Why do they make these statements in the first place? This is how you confront Holocaust deniers. What is their agenda, what are they up to? 

\textbf{In Latvia, every time when there is some news about a Holocaust memorial site or some commemoration ceremonies, all those people write in the commentary sections things like “Jews were not the only people killed, Roma were killed, or other groups...“ Do you think it’s an expression of anti-Semitism?} 

\textbf{Katrin Stoll:} No, I think it’s more an expression of this competition for victimhood, maybe, and the inability to comprehend the nature of the German Nazi Holocaust. Why was the murder of the European Jews different from the Nazi persecution of other groups? It was different because the Nazis attempted to murder every Jew on this planet, this was unprecedented, because they tried to erase what they called “the Jewish spirit” – the Jewish spirit they imagined to be in all kinds of things – in language, in literature – in everything, so everything had to be erased completely – complete extermination. It’s an inability, I think, to face the real nature of the events, an inability to confront reality. 

\textbf{Now, this also comes from a lack of education.} 

\textbf{Katrin Stoll:} No, I think it’s not a lack of education because in Latvia for example the Jews were not murdered in extermination camps but on the spot: with non-Jewish people watching, observing, benefiting from the murder. So, I think, it’s maybe also the fear of being in any way connected with this event and held responsible for one’s actions. 

\textbf{I think, at least in the Silesian history, the comparisons and relativizations of the victims are an expression of anti-Semitism. For example, my parents said “Okay, we killed six million Jews. But we also were victims. We were expelled.”}  

\textbf{Katrin Stoll:}This is a very typical German way of talking about the past, this Aufrechnungsdiskurs and Schuldabwehrdiskurs, which is aimed at negating the perpetration of the crime itself, minimizing its magnitude and negating that the Germans were the main perpetrators.\\
My hero is Beate Klarsfeld, an anti-fascist who gave Nazi \textit{Bundeskanzler}\footnote{chancellor} Kiesinger a slap in the face saying, “\textit{Nazi Kiesinger, abtreten}!” – “Step down!”, in November 1968 at a Bundesparteitag of the CDU. Why? Because Kiesinger was chancellor of the new Federal Republic which defined itself as the anti-Nazi state. How was it possible that the Nazi Kiesinger, member of the NSDAP from 1933 onwards, a high figure in the Nazi state, a propagandist, was elected by the Germans as a chancellor?  I think, she achieved a lot by this symbolic gesture demonstrating that certain things are unacceptable. She was charged with one year in prison for this supposed act of violence and she said, “It’s an act of violence to have a Nazi chancellor as \textit{Bundeskanzler}”. So, we have to analyse reality and don’t be fooled by propagandists.  
\section{Maria Fornal}
\begin{otherlanguage}{ngerman}
\textit{Maria Fornal (*1960) works as an expert on historic buildings and monuments in the Province Office in Zamość. \\
She studied ethnography at the Jagiellonian University in Kraków. She is passionate about Jewish culture, especially cares about protecting cultural heritage, also as a tour guide around Zamość and the Roztocze regions. In 2003-2011 she organised the Zamość Meetings of Cultures, presenting the history and heritage of the minorities living in Zamość in 16-20th centuries. She wrote many articles about Jews and Jewish culture, monuments and cultural heritage of the Zamość, Lublin and Roztocze regions. She is also an author of numerous radio and TV programmes on preserving and spreading cultural heritage and on national minorities in the region of Zamość. In 2010, she was cooperating in collecting documents to a Polish and Jewish documentary ``Tam był kiedyś nasz dom'' (``There used to be our home there''), showing the lives of Jews in Zamość before the Second World War. For her activity, she was awarded by the Polish President (in 2000) and the Minister of Culture and National Heritage (in 1998 and 2006). \\
The interview took place on February 1st, 2018 in Zamość.}\par
\vspace*{2em}
\textbf{Wiemy, że interesuje się Pani i zajmuje historią i kulturą Żydów.}
	
\textbf{Maria Fornal:} Zajmuję się już trochę mniej. Studiowałam etnografię na UJ w Krakowie. Ponieważ każdy student w tamtych czasach szukał miejsc, gdzie można było dorobić, więc ja też trafiłam do takiego miejsca i jak się okazało, było to u osoby, która miała kontakty z Żydami. Nie wiedziałam wtedy, że te osoby, które do niej przychodzą, to Żydzi. Poznawaliśmy się, rozmawialiśmy, stopniowo poznawałam ich coraz bliżej. Krakowski Kazimierz wyglądał wtedy diametralnie różnie od tego, jak wygląda teraz. Codziennie przechodziłam obok tego żydowskiego Kazimierza i tak naprawdę poza tym, czego dowiadywałam się w trakcie zajęć – a miałam zajęcia z historii, z historii sztuki, z historii różnych nacji – słabo znałam historię Żydów krakowskich. Kazimierz wtedy to w zasadzie był jeden piękny budynek, czyli dawna synagoga na placu Wolnica, gdzie jest muzeum etnograficzne i tam miałam czasem zajęcia. Pozostałe budynki były w złym stanie i pamiętam do dzisiaj, było mnóstwo komórek, takich małych pomieszczeń, myśmy nazywali to "`kibelkami"', dlatego że było tam dosyć dużo osób, które załatwiały swoje potrzeby i strasznie był zaniedbany ten Kazimierz. Były to czasy, kiedy społeczność żydowska chyba nie bardzo odnajdywała się. Pod koniec lat osiemdziesiątych, w latach dziewięćdziesiątych, jak jeździłam już po studiach do Krakowa, ta dzielnica zaczęła zupełnie inaczej wyglądać. Teraz na Kazimierzu właśnie w tych dawnych "`kibelkach"', komórkach są małe kawiarenki i piękne sklepiki z pamiątkami. Kiedy jadę do Krakowa, staram się odwiedzać moich dawnych znajomych, ale też zawsze biegnę i patrzę, co nowego jest w tych miejscach, które zupełnie inaczej wtedy wyglądały. Wśród moich znajomych w Krakowie znalazł się pan Tadeusz Jakubowicz, on jest obecnie przewodniczącym gminy żydowskiej w Krakowie (wtedy jeszcze gmina żydowska nie funkcjonowała jako jednostka). Często spotykaliśmy się, zabierał mnie do synagogi Remuh, na cmentarz Remuh, na którym te najcenniejsze nagrobki jeszcze były pod warstwą ziemi. Od niego dowiedziałam się pierwszych rzeczy o Żydach, o kulturze żydowskiej. Po skończonych studiach musiałam wyjechać z Krakowa, przyjechałam do Zamościa i zaczęłam pracować w Biurze Badań Dokumentacji Zabytków, była to wówczas instytucja przy Wojewódzkim Konserwatorze Zabytków. Przyszłam do pracy jako etnograf, żeby wykonywać przede wszystkim dokumentację dla budownictwa drewnianego. Zaczęłam trochę czytać o historii Zamościa. Od razu zrobiłam kurs przewodnicki. I okazało się, że historia Żydów Zamościa i Zamojszczyzny jest tak bogata i nie do końca odkryta i tyle jest w niej rozbieżności, że pomyślałam, że skoro mam już jakieś podstawy, że mam możliwość dostępu do pewnych materiałów, pytania, wyjaśniania rzeczy, które dla mnie jeszcze były niezrozumiałe, to może tak po trochę zajmę się tym. No i tak się zaczęło. Z czasem troszkę więcej interesowały mnie te sprawy, dostałam propozycję pisania artykułów do Zamojskiego Kwartalnika Kulturalnego, miałam swoją rubrykę \textit{Judaika}. Pisałam i zachęcałam do pisania swoich znajomych, którzy też mieli wiedzę o Żydach. Kilka lat to funkcjonowało. Później musiałam zostawić tę rubrykę. Ale kiedy w Hrubieszowie odkopano macewy na cmentarzu żydowskim, powstał pomnik upamiętniający Żydów hrubieszowskich, była uroczystość, na którą przyjechali Żydzi i dawni mieszańcy Hrubieszowa i ich potomkowie, wydany Kwartalnik cały poświęcono właśnie Hrubieszowowi. I napisałam do niego artykuł o żydowskim Hrubieszowie. Wtedy zaczęłam głębiej wchodzić w historię \textit{sztetli}, czyli miasteczek żydowskich, gdzie czasami kilka różnych narodowości mieszkało koło siebie. Dla mnie jako dla etnografa, badacza kultury była to najciekawsza historia. Goście, którzy przyjechali z Izraela, zabrali ze sobą kilka egzemplarzy Kwartalnika. I po jakimś czasie zadzwoniła do mnie pani, powiedziała, że przeczytała mój artykuł i prosi o zgodę o przetłumaczenie i zamieszczenie w periodyku w Izraelu. Oczywiście, zgodziłam się. Tak się zaczęła moja znajomość z ziomkostwem zamojskim. Później okazało się, że ta pani, Ewa Bar-Zeev była szefową Związku Żydów Zamojskich, pochodziła z Zamościa, z rodziny Szperów. Wtedy zaczęła się między nami  wymiana informacji: ja ją pytałam o różne rzeczy w związku z tym, że ona miała kontakt z dawnymi Żydami zamojskimi, z tymi, którzy przeżyli wojnę, wyjechali do Izraela, z ich dziećmi, a ja jej pomagałam tutaj odnajdywać miejsca, ludzi, historie. Zżyłyśmy się dosyć mocno. Korespondowałyśmy przez kilkanaście lat, do 2008 roku. Była w Zamościu chyba trzy razy. Raz przyjechała z ekipą filmową, która kręciła film o Żydach zamojskich. Dzięki niej poznawałam kolejne osoby z ziomkostwa żydowskiego i w 2000 roku pojechałam do Izraela na zaproszenie tych osób. Oczywiście zwiedzałam wszystkie te miejsca, które dla nas jako katolików są istotne, ale dla mnie najważniejsze i najcenniejsze było to, że jeździłam od domu do domu, czasami wożona przez nich, czasami autobusem, stopem i poznawałam wewnętrzne życie rodzin żydowskich. Każda rodzina starała się pokazać mi jak najwięcej ciekawych rzeczy, ale też spędzaliśmy dużo czasu przy stole, siedząc, opowiadając sobie różne historie. Kiedyś w małej miejscowości kawałek za Galileą byłam gościem u pewnego młodego człowieka, którego zresztą przez przypadek poznałam w Zamościu i pomogłam mu znaleźć historię jego rodziny. Jego rodzina zaprosiła mnie do siebie. Jego babcia mieszkała w Zamościu na Nowym Mieście, na ulicy Ogrodowej. Przeżyła Holokaust jako jedyna z całej rodziny, tylko dlatego, że uciekła w trzydziestym dziewiątym roku na teren Związku Radzieckiego. Bardzo smutna historia, a dla mnie było ważne to, że mogłam pojechać tam i z nią porozmawiać, posłuchać jej wspomnień o tym wszystkim, co się działo w Zamościu, o tym, co ona zapamiętała. Przyjmowano mnie tam bardzo ciepło, nie czułam się gojką, nie czułam się osobą obcą, po prostu wchodziłam do domu i byłam członkiem każdej z tych rodzin, u której gościłam. Ta podróż jeszcze bardziej scementowała nasze kontakty, a nawet przyjaźń z kilkoma rodzinami. Myślę, że mam wyjątkowe szczęście, że poznałam tych ludzi.  Później oni przyjeżdżali do Zamościa, zapraszałam ich do siebie do domu i też starałam się robić wszystko, żeby im się przypomniały albo czasy dzieciństwa, albo też, jeśli byli to młodsi ludzie, to żeby poczuli, jakie smaki towarzyszyły ich rodzicom czy dziadkom. Tę atmosferę i moje kolacje potem wspominali w listach. Zawsze też każdy, z kim się spotykałam, na drogę dostawał de mnie woreczek grzybków suszonych, konfiturę z jeżyn leśnych. Te historie są w zasadzie może nic nieznaczące, ale jak patrzę z perspektywy na to wszystko, to dopiero teraz wiem, jak to było ważne dla tych osób, ważniejsze niż zwiedzanie miasta. Chociaż duże znaczenie miało również chodzenie do miejsc, które były związane z ich młodością, tam, gdzie była szkoła, kino, opowiadanie o historiach, o randkach między Żydami a katolikami, a Polakami. Ostatni człowiek, z którym utrzymywałam bliskie kontakty, często bywał w Zamościu. Nazywał się Yoram Goran, był szefem kinematografii Izraela, a w Zamościu mieszkał w domu Pereca. Był ostatnim strażnikiem upamiętniania Żydów i tego, co się działo w Zamościu. Zaczęło do mnie docierać, jak ci ludzie stali się ważni również dla mnie. Wciągnęłam się bardzo mocno w tę tematykę, zaczęłam badać kolejne miejscowości wokół Zamościa, poznawać więcej z historii Żydów zamojskich. Tutaj kiedyś była biblioteka w synagodze, a w 2004 roku wyprowadziła się do nowej siedziby. Synagoga pozostała pusta, nikt nie miał pomysłu, co z nią zrobić. Wtedy wspólnie z kolegą poprosiliśmy Fundację Ochrony Dziedzictwa Żydowskiego, która była właścicielem synagogi, o udostępnienie nam jej nieodpłatnie. Przez dwa lata organizowaliśmy tam spotkania, koncerty, spektakle i ta synagoga żyła niesamowicie. Mimo że warunki były straszne, udostępnialiśmy ją do zwiedzania. Urządziliśmy ją, część mebli zdobyliśmy i ustawiliśmy tak, żeby jak najbardziej przypominała wnętrze przedwojenne synagogi. Oczywiście była też muzyka. I wtedy okazało się, że mnóstwo Żydów tu przyjeżdżało, o których myśmy w ogóle nie wiedzieli. Poznawaliśmy ich historie. Zdarzało się, że ktoś w ławce siedział przez kilka godzin i pewnie coś sobie wspominał, w tle grała muzyka. To były dobre chwile. Trwało to do 2006 roku. Potem ludzie wspominali koncerty przy świecach, atmosferę, często mówili, że przywróciliśmy ducha w tym budynku i mimo że jest pusto, nie ma wyposażenia, które tam wcześniej było, że nie odprawiają się nabożeństwa, że nie ma Żydów, to kiedy tam wchodzą, jakby cofali się do czasów przedwojennych. Później znalazły się pieniądze na remont, ale ja już nie utrzymywałam kontaktu z synagogą. Teraz wygląda pięknie, tylko że nie ma tam ducha...\\
W tej chwili w zasadzie poza kontaktami, które mam, od czasu do czasu dzielę się swoimi materiałami. Dostałam od Żydów około tysiąca zdjęć ich, ich rodzin, większość z nich niestety zginęła w Holokauście. Trochę zdjęć udało mi się kupić. I zostały mi te zdjęcia. Ale cieszę się, że od czasu do czasu jeszcze w jakiś sposób mogę się podzielić z kimś czy swoimi doświadczeniami, czy wiedzą. Kiedyś wygrzebywaliśmy materiały z każdego zakątka, skąd się tylko dało. Dostęp do archiwów nie był taki łatwy, jak jest teraz. No, ale mam satysfakcję. Udało mi się też być swego rodzaju konsultantem przy powstawaniu niektórych książek, na przykład \textit{Szebreszin} Philipa Bibla, wspomnienia szczebrzeszyńskiego Żyda, dzięki niej można poznać ten prawdziwy świat żydowski, przedwojenny. Robiłam też dokumentację do filmu dokumentalnego \textit{Tam był kiedyś nasz dom}. Film powstał właśnie dzięki Yoramowi, ze strony polskiej w zdobywaniu pieniędzy w zasadzie na cały ten projekt uczestniczył Mirosław Chojecki, a Ewa Szprynger pisała scenariusz, reżyserowała. Zresztą też zaprzyjaźniłyśmy się przy tym filmie. Ja robiłam dokumentację, robiłam rozpoznanie oraz materiały ikonograficzne. Minęło trochę czasu, trochę inaczej to wszystko wygląda. Coraz więcej ludzi zajmuje się problematyką, tematami żydowskimi. Natomiast chcę wam powiedzieć, że jednej rzeczy się trzymam: nie uczestniczę w żadnych projektach, które dotyczą Holokaustu. Obiecałam sobie, że będę mówiła o tym, jak powstawała historia i kultura żydowska, o stosunkach międzyludzkich, bo to jest mi bliskie jako etnografowi. Wolę poznawać, spisywać, wysłuchiwać historii obyczajowych. I wolę też uczyć się o historii, która się tworzyła, a nie o tej, która była niszczona. 

\textbf{A czy są wśród tych historii takie, które dotyczą antysemityzmu? Czy ktoś z tych osób go doświadczył?}
 
\textbf{Maria Fornal:} A jak wy interpretujecie antysemityzm?

\textbf{Właśnie od tygodnia próbujemy znaleźć definicję.}

\textbf{Maria Fornal:} Opowiem wam historię. Kiedyś w czasie spaceru z moimi gośćmi natknęliśmy się na młodych ludzi, którzy rysowali szubienicę i gwiazdy Dawida. Podeszłam, zapytałam jednego z tych chłopaków: "`Co ty, synu, rysujesz?"' On odpowiedział: "`Śmierć Żydom"'. Ja mówię: "`Dlaczego? O co ci chodzi?"' "`Bo ich nie powinno być."' Ja mówię: "`A podnieś koszulkę. Masz na sobie Levi-Straussy. Przecież to są spodnie Żyda, on je wymyślił"', mówię. "`A jakiej muzyki słuchasz? Powiedz mi, jakich zespołów słuchasz, to ja ci powiem, ile w tych zespołach jest muzyków żydowskich."' I on zrobił się bordowy, powiedział brzydkie słowo i zwiał. Więc dlatego pytam was, jak wy interpretujecie antysemityzm. W filmie, o którym mówiłam, Yoram Golan, który jest w nim przewodnikiem, opowiada historie, które się działy przed wojną: o gettach ławkowych, o odsuwaniu młodzieży żydowskiej od wspólnych projektów typu spektakle teatralne, czy wspólnej zabawy. Ale też mówi tam bardzo ważne zdanie, że tak naprawdę wszyscy byli tutaj jedną rodziną, że różne rzeczy się działy, że Polacy podglądali Żydów, Żydzi podglądali Polaków i jedni od drugich czerpali pewne rzeczy i przenosili do swoich domów. Mimo że dochodziło do bójek w szkołach, bo czasami jacyś chłopcy uważali, że trzeba dać łupnia Żydom, to kilka dni później razem szli już do kina albo zastanawiali się razem i patrzyli, komu by tu można dosunąć. Już wtedy w grupie, bez podziału, że tu Żydzi, a tu Polacy. Ja nie chcę nazywać tego antysemityzmem, nie chcę tego w ogóle nazywać, bo by można to nazywać różnie. W filmie są też wspomnienia pani, która wróciła do Zamościa po wojnie, jej się ten Zamość cały czas śnił, wiedziała, że musi tutaj wrócić, bo w ogóle nie mogła spać, to było coś potwornego. W końcu wybrała się i kiedy zobaczyła, że mieszkanie było zajęte, a ludzie w nim mieszkający byli przestraszeni, bo obawiali się tego, że przyjechała zabrać im to mieszkanie, natychmiast wiedziała, że musi wrócić do Izraela, że to już nie jest jej miejsce. To na pewno wywoływało przykre sytuacje. Ja sama spotkałam się kiedyś z dwiema kobietami, których matka była ukrywana przez polską rodzinę niedaleko Tarnawatki. Pojechałyśmy to tego miejsca. Ci ludzie, którzy ukrywali tę matkę, już nie żyją, syn, czy córka tej osoby żyje do tej pory. W każdym bądź razie ta dziewczyna przeżyła. Potem była ukrywana w Zamościu, później po wojnie przez pewien czas tu mieszkała. Przeżyła wojnę, wróciła tam, do tej rodziny i powiedziała, że wszystko to, co było ich majątkiem, czyli dom jakiś stary i sporo gruntu, ona tej rodzinie zostawia za to, że uratowali ją w czasie wojny. (A kiedy córki tej pani przyjechały i szukały historii swojego wuja, przez przypadek trafiliśmy do pana, który był w oddziale Batalionów Chłopskich razem z tym wujem. Od niego dowiedziały się, że pod Adamowem w czasie jakiejś potyczki on niestety wpadł w ręce Niemców, przez tydzień był torturowany na Rotundzie, po czym rozstrzelany też na Rotundzie. Ponieważ w ogóle nie wiedziały, co się z nim stało, z jednej strony wiadomość o śmierci kolejnej osoby z ich rodziny była bolesna, ale z drugiej strony były szczęśliwe i tak wdzięczne, że poznały tę historię. I dla nich było ważne, że ten człowiek z taką dumą mówił, że był w oddziale właśnie z tym ich wujem, Żydem.) I udałam się z tymi kobietami do tej rodziny jako tłumacz, weszłyśmy, powiedziałam, kim one są i po co przyjechały. Trzeba było zobaczyć twarze tych ludzi, ich przerażenie, bo byli przekonani, że właśnie ktoś przyjechał zabrać im wszystko, co mieli, bo przecież oni sobie pobudowali już nowy dom na tych gruntach. Te kobiety powiedziały mi: "`Maria, powiedz im od razu, że my niczego nie chcemy. My tylko chciałyśmy przyjechać, zobaczyć to miejsce, gdzie nasza matka była chroniona."' Specjalnie dla niej była wykopana jama, na której postawiono komórkę ze składem drewna. Potem, gdy miejscowi dowiadywali się, podejrzewali tę rodzinę, że ukrywali Żydówkę, ta rodzina zabrała ją z tej jamy, a miejscowi spalili komórkę, przekonani, że była tam Żydówka. A ona była już ukryta na strychu. I tam kiedyś wpadli, by wyciągnąć tę ukrywaną Żydówę, a ona siedziała w wielkim pojemniku pełnym zboża. Na pewno miała problem z oddychaniem, nie wiem, ale zrobili tam jakąś rurkę, żeby odrobina powietrza docierała. I nie znaleźli jej, poszli gdzieś dalej, przeszukali całe gospodarstwo. I ta dziewczyna przeżyła, przeżyła wojnę, urodziła później dzieci. Widać więc, jakie było nastawienie.\\
Miałam kiedyś gości tutaj, przyjechali do rodziny niedaleko za Zamość. Ci ludzie znali adres, doskonale wiedzieli, że tam jest ktoś z ich rodziny, dziadkowie byli braćmi. Jeden przeżył, wyjechał, trafił do Palestyny, a drugi tutaj przeszedł na wiarę katolicką. Rodzina nie wiedziała o tym, że mają korzenie żydowskie. Aż tu nagle przyjeżdżają bliscy krewni, bo ze strony brata ich dziadka, więc oni szczęśliwi, zadowoleni, stół zastawiony, herbatka, przyjęcie, rozmowy o wszystkim. I nagle pada pytanie: "`A skąd wy przyjechaliście?"' Oni mówią: "`Z Izraela."' I cisza. Popatrzyli po sobie przestraszeni. "`Jak to z Izraela? No bo nasz dziadek tak się nazywał. A wasz dziadek jak się nazywał?"' "`Tak."' "`No to bracia byli przecież."' I przerażenie, żeby nie wyszło na jaw, że mają korzenie żydowskie. Nie wiem, co to jest, nie chcę definiować antysemityzmu, bo nie wiem, czy my możemy mówić o takim ogólnym, globalnym określeniu tego słowa. Dla jednego antysemityzmem będzie to, że ktoś napisze coś paskudnego na ścianie, tak jak mieliśmy niedawno przykład na synagodze, kiedy dzieciaki w ogóle niemające nawet pojęcia o tym, co to jest, napisały, już nie pamiętam, jakieś słowa o Żydach i Fundacja Ochrony Dziedzictwa Żydowskiego rozpętała wielką aferę, że antysemickie działania. Ja mówię, to napisało dziecko może ośmio- dziesięcioletnie, które gdzieś usłyszało, że ktoś coś powiedział i poszło sobie i tam napisało. Dla mnie to są działania, które wynikają z niewiedzy, z głupoty ludzkiej, ze złośliwości.\\
Ale nie słyszałam nigdy, żeby ci, którzy przeżyli wojnę i przyjeżdżali tutaj, nawet wiedząc, że nie mogą wrócić do swojego domu, widząc miejsca, gdzie stał kiedyś ich dom, oskarżali tych, którym się udało, że przyszli, zabrali. Nie wiem, może trafiałam na takich ludzi.

\textbf{Może dlatego, jak Pani mówiła, że było bardzo dobre współżycie tych dwóch społeczności, polskiej i żydowskiej, prawda? Że było podglądanie, o którym Pani wspomniała, czerpanie od siebie, a nie rywalizacja, zawiść. Może to coś zostawiło?}

\textbf{Maria Fornal:} No właśnie. W miejscu, gdzie jest parking koło hotelu "`Renesans"', kiedyś było kino "`Jutrzenka"', drewniany budynek. Przyjechali kiedyś tutaj dwaj bracia, którzy nie wiedzieli o tym, że o sobie nawzajem, że przeżyli, bo w trzydziestym dziewiątym roku obydwaj pojechali z transportem z Armią Czerwoną na teren Związku Radzieckiego. Potem ich drogi rozeszły się, nie wiedzieli o sobie. I po wojnie spotkali się zupełnie przypadkowo w Lublinie. Szukali tutaj śladów swojej rodziny. Opowiadali przeróżne historie, między innymi, że to kino "`Jutrzenka"' tak bardzo zapamiętali, dlatego że to było jedyne miejsce, gdzie mogli z polskimi dziewczynami chodzić na randki. Mówi: "`Przecież wie pani, po co się chodzi do kina..."' Babcia mojego przyjaciela opowiadała mi historię z lat trzydziestych, gdy był już nacisk na odsuwanie Żydów, niekupowanie w żydowskich sklepach, niekorzystanie z usług żydowskich rzemieślników. Estera była z dosyć ubogiej rodziny, jej ojciec był dekarzem, wszystkie dachy robił. Trudno mu było o pracę, bo po sąsiedzku były polskie warsztaty dekarzy i przede wszystkim oni byli brani do roboty. Opowiadała, że Janek, jej najbliższy przyjaciel (podejrzewam, że wielka miłość, chociaż nigdy tego słowa od niej nie usłyszałam, ale tak ciepło o nim mówiła i tak jej oczy błyszczały, jak wspominała tę historię), jak się dowiedział, że nie mają pieniędzy na jedzenie, że jest im ciężko, któregoś dnia wziął kamień, biegał i tłukł dachy po to, żeby zrobić dziurę i żeby jej ojciec miał robotę. I wiecie, można słuchać różnych historii, które zostały gdzieś opowiedziane, są zapisane. Ale ja zawsze obserwowałam tych ludzi, gdy o tym mówili. Widziałam, jak zamieniali się w szczeniaków, w młode dziewczyny z tamtego czasu, te oczy jak pięknie błyszczały i buzia uśmiechnięta… Miałam wrażanie, że gdy opowiadali, wracali na tę ulicę, na swoje podwórko, do tej gromadki, gdzie takie dobre rzeczy się zdarzały… 
\end{otherlanguage}
\nocite{*}
\printbibliography
\end{document}
